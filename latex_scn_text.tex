\section*{Правила оформления sc.n-текста}
\markboth{Правила оформления sc.n-текста}{Правила оформления sc.n-текста}

Унификация спецификации глав и параграфов.

\begin{SCn}
	\scnidtf{синонимичный заголовок при необходимости}
	
	\bigskip
	
	\scntext{эпиграф}{текст эпиграфа}
	
	\bigskip
	
	\begin{scnrelfromlist}{автор}
		\scnitem{Автор1}
		\scnitem{Автор2}
	\end{scnrelfromlist}
	
	\bigskip
	
	\scntext{аннотация}{текст аннотации}
	
	\bigskip
	
	\begin{scnrelfromlist}{подраздел}
		\scnitem{\ref{chap_intro}~\nameref{chap_intro}}
		\scnitem{\ref{chapter_new_generation_systems}~\nameref{chapter_new_generation_systems}}
	\end{scnrelfromlist}

	\bigskip

	\begin{scnrelfromlist}{ключевой знак}
		\scnitem{ключевой знак 1}
		\scnitem{ключевой знак 2}
	\end{scnrelfromlist}
	
	\bigskip
	
	\begin{scnrelfromlist}{ключевой понятие}
		\scnitem{ключевое понятие 1}
		\scnitem{ключевое понятие 2}
	\end{scnrelfromlist}
	
	\bigskip
	
	\begin{scnrelfromlist}{ключевое отношение}
		\scnitem{ключевое отношение 1}
		\scnitem{ключевое отношение 2}
	\end{scnrelfromlist}
	
	\bigskip
	
	\begin{scnrelfromlist}{ключевой параметр}
		\scnitem{ключевой параметр 1}
		\scnitem{ключевой параметр 2}
	\end{scnrelfromlist}
	
	\bigskip
	
	\begin{scnrelfromlist}{ключевое знание}
		\scnitem{ключевое знание 1}
		\scnitem{ключевое знание 2}
	\end{scnrelfromlist}

	\bigskip
	
	\begin{scnrelfromlist}{библиографическая ссылка}
		\scnitem{\scncite{Standart2021}}
	\end{scnrelfromlist}
	
\end{SCn}

Для оформления списков внутри естественно-языкового файла, находящегося в рамках формального sc.n-текста, необходимо использовать окружение scnitemize:
\begin{lstlisting}              
\begin{scnitemize}
	\item text
\end{scnitemize}
\end{lstlisting}  

Для оформления списков в рамках естественного-языкового текста (не SCn), следует использовать окружение textitemize:
\begin{lstlisting}              
\begin{textitemize}
	\item text
\end{textitemize}
\end{lstlisting}  

Использование \textbf{шрифтов}

\begin{lstlisting}
\underline{text} or \uline{text}
\textit{text}
\emph{text}
\textbf{text}
\textup{text}
\end{lstlisting}

Результат компиляции:

\underline{подчеркивание} или \uline{подчер\-кивание}

\textit{курсив}

\emph{выделенный шрифт}

\textbf{полужирный}

\textup{прямой шрифт}

Использование \textbf{символов}

\begin{lstlisting}
text\scnrolesign
text\textasciicircum
\end{lstlisting}

Результат компиляции:

	текст\scnrolesign
	
	текст\scnsupergroupsign

Рассмотрим примерную структуру описания sc.n-предложения. Пример исходного кода:

\begin{lstlisting}              
\begin{SCn}
	\scnheader{header}
	\scnidtftext{frequently used sc-identifier}{text}
	\scnidtf{text}
	\scniselement{text}
	\scnhaselement{text}
	\scnsuperset{text}
	\scnsubset{text}
	\scnidtfdef{text}
	\scndefinition{text}
	\scnidtfexp{text}
	\scnexplanation{text}
	\scntext{text}
	{Text
		\begin{textitemize}
			\item text;
			\item text
		\end{textitemize}
	}
	\scnnote{text}
	\scncomment{text}
	%dividing
	\begin{scnsubdividing}
		\scnitem{text}
		\scnitem{text}
		%nesting change
		\begin{scnindent}
			\scnidtf{text}
			\scneq{\textup{(} text \textup{)}}
			\begin{scneqtoset}
				\scnitem{text}
				\scnitem{text}
			\end{scneqtoset}
		\end{scnindent} 
	\end{scnsubdividing}
	\scnrelfrom{dividing}{attribute}
	\begin{scnindent}
		\begin{scneqtoset}
			\scnitem{text}
			\scnitem{text}
		\end{scneqtoset}
	\end{scnindent} 
	\begin{scnrelfromset}{relation}
		\scnitem{text}
		\scnitem{text}
	\end{scnrelfromset}
	\begin{scnrelfromset}{relation}
		\scnfileitem{text}
		\scnfileitem{text}
	\end{scnrelfromset}
	\begin{scnrelfromlist}{note}
		\scnfileitem{text}
		\scnfileitem{text}
	\end{scnrelfromlist}
	\scnrelboth{analog}{dividing*}
	\begin{scnrelfromlistcustom}{principles underlying}
		\scnitemcustom{text}
		\scnitemcustom{text}
	\end{scnrelfromlistcustom}
	\scnrelfrom{second domain}{text}
\end{SCn}
\end{lstlisting} 

Результат компиляции:
	\begin{SCn}
	\scnheader{ключевой sc-элемент}
	\scnidtftext{часто используемый sc-идентификатор}{text}
	\scnidtf{синонимичный идентификатор}
	\scniselement{принадлежность}
	\scnhaselement{принадлежность}
	\scnsuperset{строгое надмножество --- <<включает в себя>>}
	\scnsubset{cтрогое подмножество --- <<включено в>>}
	\scnidtfdef{определение}
	\scndefinition{определение}
	\scnidtfexp{пояснение}
	\scnexplanation{пояснение}
	\scntext{пояснение}
	{Текст
		\begin{textitemize}
			\item текст;
			\item текст.
		\end{textitemize}
	}
	\scnnote{примечание}
	\scncomment{комментарий}
	%разбиение
	\begin{scnsubdividing}
		\scnitem{текст}
		\scnitem{текст}
		%изменение вложенности
		\begin{scnindent}
			\scnidtf{текст}
			\scneq{\textup{(} текст \textup{)}}
			\begin{scneqtoset}
				\scnitem{текст}
				\scnitem{текст}
			\end{scneqtoset}
		\end{scnindent} 
	\end{scnsubdividing}
	%отношение разбиение с признаком
	\scnrelfrom{разбиение}{Признак}
	\begin{scnindent}
		\begin{scneqtoset}
			\scnitem{текст}
			\scnitem{текст}
		\end{scneqtoset}
	\end{scnindent} 
	\begin{scnrelfromset}{отношение}
		\scnitem{текст}
		\scnitem{текст}
	\end{scnrelfromset}
	\begin{scnrelfromset}{отношение}
		\scnfileitem{текст}
		\scnfileitem{текст}
	\end{scnrelfromset}
	\begin{scnrelfromlist}{примечание}
		\scnfileitem{текст}
		\scnfileitem{текст}
	\end{scnrelfromlist}
	\scnrelboth{аналог}{разбиение*}
	\begin{scnrelfromlistcustom}{принципы, лежащие в основе}
		\scnitemcustom{текст}
		\scnitemcustom{текст}
	\end{scnrelfromlistcustom}
	\scnrelfrom{второй домен}{текст}
\end{SCn}

Пример исходного кода:

\begin{lstlisting}
\begin{SCn}
	\scnheader{relation*}
	\scnrelto{second domain}{text}
\end{SCn}
\end{lstlisting}

Результат компиляции:

\begin{SCn}
	\scnheader{отношение*}
	\scnrelto{второй домен}{текст}
\end{SCn}

Пример исходного кода (следует отличать):

\begin{lstlisting}
\begin{SCn}
	\scnheader{should be distinguished*}
	\begin{scnhaselementset}
		\scnitem{text}
		\scnitem{text}
	\end{scnhaselementset}
\end{SCn}
\end{lstlisting}

Результат компиляции:

\begin{SCn}
	\scnheader{следует отличать*}
	\begin{scnhaselementset}
		\scnitem{текст}
		\scnitem{текст}
	\end{scnhaselementset}
\end{SCn}
	
\newpage
