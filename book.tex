%%%%%%%%%%%%%%%%%%%% book.tex %%%%%%%%%%%%%%%%%%%%%%%%%%%%%
%
% sample root file for the chapters of your "monograph"
%
% Use this file as a template for your own input.
%
%%%%%%%%%%%%%%%% Springer-Verlag %%%%%%%%%%%%%%%%%%%%%%%%%%


% RECOMMENDED %%%%%%%%%%%%%%%%%%%%%%%%%%%%%%%%%%%%%%%%%%%%%%%%%%%
\documentclass[graybox,envcountchap,sectrefs]{svmono}

% choose options for [] as required from the list
% in the Reference Guide

%\usepackage{mathptmx}
%\usepackage{helvet}
%\usepackage{courier}
%
%\usepackage{type1cm}         

\usepackage[utf8]{inputenc}
\usepackage[T2A,T1]{fontenc}
\usepackage[english,main=russian]{babel}
\renewcommand{\familydefault}{Tempora-TLF}

\usepackage{makeidx}         % allows index generation
\usepackage{graphicx}        % standard LaTeX graphics tool
                             % when including figure files
\usepackage{multicol}        % used for the two-column index
\usepackage[bottom]{footmisc}% places footnotes at page bottom

\usepackage{nameref}
\usepackage{ulem}
\usepackage{float}
\usepackage{caption}
\usepackage{parskip}

\usepackage{geometry}
\geometry{
  a4paper,
  left=20mm,
  right=20mm,
  top=20mm,
  bottom=15mm,
  heightrounded
}

\usepackage{tocloft}

\usepackage{scn/scn}
\input{scn_alias}

\usepackage[]{algorithm2e}

\newlistentry[section]{authors}{toc}{1}
\cftpagenumbersoff{authors}
\cftsetindents{authors}{\cftchapindent}{0pt}

\newlistentry[section]{authorssec}{toc}{1}
\cftpagenumbersoff{authorssec}
\cftsetindents{authorssec}{\cftsecindent}{0pt}

\newcommand{\chapauthortoc}[1]{\addcontentsline{toc}{authors}{\textit{#1}\vspace{0.5\baselineskip}}}

\newcommand{\secauthortoc}[1]{\addcontentsline{toc}{authorssec}{\textit{#1}\vspace{0.5\baselineskip}}}

% see the list of further useful packages
% in the Reference Guide

\makeindex             % used for the subject index
                       % please use the style svind.ist with
                       % your makeindex program

%%%%%%%%%%%%%%%%%%%%%%%%%%%%%%%%%%%%%%%%%%%%%%%%%%%%%%%%%%%%%%%%%%%%%

\graphicspath{{./}{images/}}

\renewcommand\thepart{\arabic{part}.}
\renewcommand{\thechapter}{\arabic{part}.\arabic{chapter}.}

\renewcommand\cftchapnumwidth{4.7em}
\renewcommand\cftchappresnum{Глава~}

%\renewcommand\cftsecnumwidth{6.8em}
%\renewcommand{\thesection}{Параграф \thechapter.\arabic{section}}

\renewcommand\cftsecnumwidth{3.4em}
\renewcommand{\thesection}{\S~\arabic{part}.\arabic{chapter}.\arabic{section}.}

\cftsetindents{subsection}{4.9em}{6.2em}
%\renewcommand\cftsubsecnumwidth{6em}
\renewcommand{\thesubsection}{Пункт \arabic{part}.\arabic{chapter}.\arabic{section}.\arabic{subsection}.}

\DeclareCaptionFormat{monography}
{
\textbf{=~\textit{#3}}
}
\captionsetup{format=monography,singlelinecheck=false}

\setlength{\parindent}{0pt}


\usepackage{etoolbox}
\usepackage[
style=ieee,
citestyle=authoryear,
maxnames=3
]{biblatex}

\newcommand{\scncite}[1]{
	\hspace{-0.2em}\textit{\cite{#1}}\hspace{-0.2em}
}

\newcommand{\scnciteheader}[1]{
	\scnheader{\cite{#1}}
}

\newcommand{\scnfullcite}[1]{
	\scnidtftext{библиографическое описание}{\fullcite{#1}}
}

\newcommand{\scnciteannotation}[1]{
	\scntext{аннотация}{\citefield{#1}{annotation}}
}

\bibliography{biblio}

\begin{document}

\author{}
\title{Технология комплексной поддержки жизненного цикла семантически совместимых интеллектуальных компьютерных систем нового поколения}
\subtitle{-- Монография --}
\maketitle

\frontmatter%%%%%%%%%%%%%%%%%%%%%%%%%%%%%%%%%%%%%%%%%%%%%%%%%%%%%%

\begin{partbacktext}
\part*{Предисловие}
\markboth{ПРЕДИСЛОВИЕ}{ПРЕДИСЛОВИЕ}
\label{part_preface}
\addcontentsline{toc}{part}{Предисловие}

В настоящее время практически для любых государств является актуальным рассмотрение проблем обеспечения технологического суверенитета.

Всеобщая цифровизация потребовала массовой разработки софта и в программисты массово хлынула молодежь, начиная уже со школьной скамьи.

Действительно, зачем толковым школьникам идти на инженерные, естественнонаучные, конструкторские, сложносистемные специальности, если можно быстро вписаться в "технологический конвейер"{} разработки софта и гарантированно получать зарплату, которая намного выше зарплаты инженеров, конструкторов, проектировщиков сложных систем? И этот технологический конвейер быстро и массово превращает, среднестатистического программиста в ремесленника, даже, наверное, более точно в IT-пролетариат. Причем это не локальный IT-пролетариат, а глобальный IT-пролетариат (можно сказать даже глобально управляемый), формируемый по клише "граждан мира"{}. Хотя, с резким торможением глобализации перспективы гражданства мира стали не такими уж и очевидными и ясными.

Известный английский историк А.Дж.Тойнби в свое время ввел понятие "опасные классы"{}. Он использовал его при описании процессов конца 18-начала 19 веков, когда вырванное из сельской местности население превратилось в "опасные классы"{}. За десятилетия удалось их превратить в более-менее устойчивый пролетариат. Возможно сейчас процессы с опасными классами повторяются, в частности, с айтишниками, которые оказываются мало приспособленными к реалиям современной жизни. Но глобализация с цифровизацией превратила их в "граждан мира"{} (как им очень хотелось), но в виде недоформированного IT-пролетариата (что их жестко и неприятно приземлило).

Сейчас затруднительно адекватно использовать термины айтишник, программист. Всеобщая цифровизация практически быстро размывает эти понятия. На постсоветском пространстве понятие айтишник, а иногда даже и программист часто рассматривается как нечто единое, объединяющее и тестировщиков, и кодировщиков, и специалистов по искусственному интеллекиу, анализу больших данных, и системных архитекторов и так далее. В современной мировой практике этой цельности нет -- все это совершенно разные категории специалистов. Поэтому, когда иногда говорят о новом классе -- айтишников, то такого класса строго говоря и нет. Есть, наверное, достаточно массовый айтишный пролетариат (IT-пролетариат), которому очень хотелось бы идентифицировать себя с хайтеком. Но, если специалисты в области искусственного интеллекта, анализа больших данных, системные архитекторы, бизнес-аналитики, получившие серьезное образование, будут востребованы, то профессии тестировщика и кодировщика и близкие к ним, но с модными названиями могут быстро исчезнуть. В частности, искусственный интеллект способен этому сильно помочь. А своего рода миф о том, что нужно так много программистов отчасти поддерживается тем, что просто создается очень много некачественного кода (софта).

Мы всегда должны помнить, что в эпоху всеобщей цифровизации роль технологического суверенитета резко возрастает. Но без молодежи невозможно развивать технологический суверенитет. А молодежь мы упускаем. То есть молодежь вроде бы и стремится к новым технологиям, мы вроде бы и поощряем такое стремление, но этот процесс какой-то однобокий. Возьмем опять IT-сферу на постсоветском пространстве.

В университетах, где готовят IT-специалистов, много учебных центров и классов по продуктам Microsoft, Oracle, Cisco, SAP и тому подобные. Но это не научно-исследовательские центры, а банальные центры навыков, компетенций по простому кодированию, тестированию, внедрению и эксплуатации. Конечно, это неплохо, если бы в итоге не подменялись основные университетские курсы. Студенты видели, что в таких центрах разработки не нужны серьезная математика, физика. Терялась мотивация. Наиболее способные молодые специалисты плавно перетекали из центров разработки в центры за пределами страны, где этапы жизненных циклов разработки более интересны. Как правило, они не возвращались. Например, для нашей небольшой страны это существенные потери. Как выясняется, нередко в подобной эмиграции главное даже не заработки, а возможность творчества вместо ремесла и промышленного программирования. По сути, такая модель ускорила разрушение инженерного, конструкторского, общесистемного образования. Утрачивалась мотивация к изучению математических, физических, сложносистемных дисциплин. Действительно, студенту незачем тратить время на их глубокое освоение, если достаточно получить навыки по кодированию и тестированию софта на основе несложных технологических платформ и легко найти работу в одной из конвейерных компаний по разработке софта. По сути образование подстраивалось под такую модель и планомерно разрушалось при активном участии тех же западных конвейеров.

Вспомним, что традиционные западные обыватели мало интересуются глобальными вещами. Их интерес в основном локален. Например, решения муниципальных властей для них интереснее решений властей штата или земли, не говоря уже о решениях федеральных властей. Конечно могут быть исключения, но они лишь, как обычно, подтверждают общность. Точно таким же формируется у нас и молодое поколение, особенно айтишное. Действительно система образования и воспитания не воспроизводит сейчас системность во взглядах и оценках. Хорошая системность предполагает умение анализировать и синтезировать информацию, в том числе и по-глобальному. Советская система образования и воспитания, пусть и несколько односторонне, все-таки этому способствовала. Теперь же, в связи со старением и уходом поколений-носителей такой концепции, такая система образования и воспитания полностью исчезает. А новая не создана, вернее лихорадочно, бессистемно создается как ухудшенная калька с быстроразрушающейся западной системы, причем, зачастую, даже не с исконно западной, а как калька с восточно-европейской кальки. И у молодежи формируется основная парадигма, зачем думать о глобальном, системно, главное, чтобы в локальном мирке было все хорошо и как следствие нынешние молодежные установки: "жизнь в моменте"{}, "хорошо в моменте"{}, "хорошо здесь и сейчас"{} и тому подобное.

И именно айтишный пролетариат являются основным носителем таких установок. И понятно почему. Они работают практически полностью в западном (американском) конвейере, причем выполняя в основном самые примитивные операции, то есть работают как современный IT-пролетариат, даже не как инженера, технологи, не говоря уже о конструкторах или архитекторах. То есть совершенно не элита.

По такой модели мы получили лишь ускорение потери суверенитета технологического, и как следствие разрушение странового суверенитета.

Хотя последние несколько лет наконец-то многие в мире спохватились, что нужен суверенитет для стран. Глобалистский мир оказался не столь уж безопасен и привлекателен, как был распиарен за последние лет тридцать, граждане мира, как мировые кочевники, не столь уж и конструктивны и креативны -- собственно, все как всегда: хороша только виртуальная картинка, в существенно иных условиях идеи глобализма могли бы быть и продуктивны, но народ не тот, элиты не те и тому подобное. Модно стало говорить о цифровом суверенитете, \myuline{суверенитете в области высоких технологий}, суверенитете в области инфокоммуникационных технологий и так далее. Соответственно, откликаясь на моду, хайп (а сейчас многое пытаются делать на волнах хайпа) стали писать концепции и стратегии для разного вида суверенитета. Хотя после стольких лет построения глобалисткого мира (как видим теперь не очень-то удачного), строить суверенитеты многим странам крайне затруднительно.

Для небольших стран хотелось бы отметить следующее.

В условиях достаточно жесткого противостояния и противоборства нет смысла стремится к топовым с точки зрения глобальных экспертов, существующих, как обычно, на средства транснациональных компаний, решениям в области инфокоммуникационных технологий. Те же топовые чипы 2-3 нано зачем нужны на данном этапе для оборонки, государства, индустрии, социальной сферы и тому подобное. Можно и 90 нано обойтись. Не нужно путать требования потребительского рынка и требования для решения государственных, страновых задач. Тот же Apple, или Microsoft, или Samsung, создавая гаджеты и сервисы для конечного потребителя решают совершенно иные задачи. Им нужно удержать лидерство на рынке, чтобы сохранить миллиардные прибыли, Отсюда и 2-3 нано в чипах, и все новые и новые операционные системы, и все новые и новые приложения. Бег по кругу, новый системный и прикладной софт требует нового железа, новое железо требует нового софта -- и за все расплачивается конечный потребитель. А решение задачи сохранения суверенитета страны в текущей ситуации такой безудержной гонки совершенно и не требует. И чипы нужны попроще, но понадежнее, и операционные системы нужны не навороченные, и приложения нужны не такие уж многофункциональные. Классический пример с MS Word. Обычный пользователь использует не более 5\% возможностей редактора. А, например, госслужащим и им подобным и 5\% много. Точно также и с большинством софта. Нужны гораздо более простые, но стабильные и надежные системы. И тогда, чтобы создать такие системы нужны не сотни тысяч IT-пролетариата, а сотни, ну может быть тысячи высококвалифицированных разработчиков и относительно небольшое количество IT-пролетариев. Еще раз подчеркну, \myuline{страна для своего суверенитета} \myuline{и транснациональный IT-бизнес} \myuline{решают совершенно разные задачи и преследуют совершенно разные цели}. Можно еще напомнить, что массы IT-пролетариата бизнесу нужны и для того, чтобы раздувать значимость фирмы, создавать иллюзию гигантизма, особенно, если твоя фирма на IPO в Нью-Йорке или Лондоне, или хотя бы в Гонконге. Массовый инвестор больше поверит в фирму с десятками тысяч работников, чем в небольшую команду.

Но коль уж мы от цифровизации никуда не денемся, то \myuline{нужна цифровизация хорошо продуманная,} \myuline{просчитанная,} \myuline{с глубоким прогнозированием последствий}, а не безудержная, с агрессивной рекламой, что создает у людей совершенно ложные ориентиры. В подавляющем случае это делается опять же непрофессионально айтишным пролетариатом, втянутым в разработку на волне хайпа о неограниченной цифровизации всего и вся. А для того чтобы это понимать, молодежи нужно соответствующее образование. И выбора нет -- \myuline{либо мы такое} \myuline{образование очень} \myuline{оперативно сформируем,} \myuline{либо окончательно} \myuline{потеряем} \myuline{думающую} \myuline{молодежь,} \myuline{и, соответственно,} \myuline{окончательно} \myuline{потеряем} \myuline{и суверенитет} в условиях глобального перехода информационных технологий на принципиально новый уровень, требующий создания комплексов эффективно взаимодействующих компьютерных систем.
 
%\vspace{\baselineskip}
\begin{flushright}\noindent
\hfill {\it А.~Н. Курбацкий}\\
\hfill доктор технических наук, профессор,\\
\hfill заведующий кафедрой технологий программирования\\
\hfill факультета прикладной математики и информатики\\
\hfill Белорусского государственного университета
\end{flushright}

\end{partbacktext}

\tableofcontents


\chapter*{\LARGE Список сокращений}
\label{chap_preface}
\addcontentsline{toc}{part}{Список сокращений}

\begin{description}
\item[ACL] англ. Agent Communication Language. Язык взаимодействия агентов, предложенный FIPA в качестве стандарта
\item[FIPA] англ. Foundation for Intelligent Physical Agents. Организация, осуществляющая разработку и продвижение стандартов в области многоагентных систем
\item[GPS] англ. General Problem Solver. Компьютерная программа, созданная в 1959 г. и предназначенная для работы в качестве универсальной машины для решения задач, сформулированных на языке хорновских дизъюнктов
\item[IACPaaS] англ. Intelligent Applications, Control and Platform as a Service. Исследовательская облачная платформа, объединяющая различные модели парадигмы облачных вычислений
\item[IMS] англ. Intelligent MetaSystem. Интеллектуальная метасистема поддержки проектирования интеллектуальных систем
\item[KIF] англ. Knowledge Interchange Language. Компьютерноориентированный язык для обмена знаниями между различными компьютерными программами
\item[KQML] англ. Knowledge Query and Manipulation Language. Язык взаимодействия между программными агентами и системами, основанными на знаниях
\item[OSTIS] англ. Open Semantic Technology for Intelligent Systems. Открытая семантическая технология проектирования интеллектуальных систем
\item[OWL] англ. Web Ontology Language. Язык описания онтологий для семантической паутины
\item[QA3] англ. Question Answer system ver. 3. Вопросно-ответная дедуктивная система, созданная в 1969 г. на языке LISP
\item[RDF] англ. Resource Description Framework. Разработанная Консорциумом Всемирной паутины модель для представления данных
\item[SCg-код] англ. Semantic Code graphic. Графический нелинейный вариант визуализации текстов SC-кода
\item[SCn-код] англ. Semantic Code natural. Гипертекстовый вариант визуализации текстов SC-кода
\item[SCP] англ. Semantic Code Programming. Графовый процедурный язык программирования, построенный на базе SC-кода
\item[SC-код] англ. Semantic Code. Универсальный базовый способ смыслового представления знаний в виде семантических сетей с базовой теоретикомножественной интерпретацией
\item[SPARQL] англ. SPARQL Protocol and RDF Query Language. Язык запросов к данным (является рекомендацией консорциума W3C и одной из технологий семантической паутины), представленным по модели RDF, а также протокол для передачи этих запросов и ответов на них
\item[SQL] англ. Structured Query Language --- ''язык структурированных запросов``. Универсальный язык запросов, применяемый для создания, модификации и управления данными в реляционных базах данных
\item[STRIPS] англ. Stanford Research Institute Problem Solver. Планирующая система, использующая декларативно-процедуральное представление знаний в сочетании с эвристическим поиском, создана в 1971 г.
\item[W3C] англ. World Wide Web Consortium, W3C. Консорциум Всемирной паутины, организация, разрабатывающая и внедряющая технологические стандарты для Всемирной паутины
\item[ГРЗ] гибридный решатель задач
\item[ИСС] интеллектуальная справочная система
\item[НИЦ ЭВТ] московский Научно-исследовательский центр электронной вычислительной техники
\item[ППР] Программа принятия решений, планирующая система для интеллектуального робота, созданная в 1977 г. под руководством В. П. Гладуна
\item[ПРИЗ] Пакет прикладных инженерных задач, система программирования, созданная под руководством Э. Х. Тыугу в 1970–1976 гг
\item[РЗ] решатель задач
\item[УСК] универсальный семантический код, разработанный В. В. Мартыновым
\end{description}


\mainmatter%%%%%%%%%%%%%%%%%%%%%%%%%%%%%%%%%%%%%%%%%%%%%%%%%%%%%%%


\section*{Правила оформления библиографических источников}

Для добавления нового библиографического источника необходимо выполнить следующие шаги:
\begin{itemize}
	\item Убедиться, что нужный источник еще не присутствует в файле biblio.bib, который находится в репозитории с исходными текстами Стандарта OSTIS. В настоящее время все библиографические источники изначально описываются в этом файле.
	\item Добавить в файл biblio.bib описание библиографического источника в соответствии с форматом описания BibTex. Более подробно про формат можно почитать на сайте https://www.bibtex.com/g/bibtex-format/. Для помощи в оформлении можно использовать различные бесплатные средства, например, сервис https://www.doi2bib.org/ позволяет сгенерировать bib-описание на основе идентификатора DOI, кроме того, многие онлайн-библиотеки позволяют выгрузить описание нужного источника в формат BibTex.
	\item Каждому источнику в соответствии с форматом BibTex присваивается некоторое условное имя (цитатный ключ или просто ключ), по которому затем можно процитировать соответствующий источник. В рамках Стандарта OSTIS рекомендуется цитатные ключи источников в формате BibTex формировать путем транслитерации в латинский алфавит фамилии первого автора и добавления года издания источника, например:
	
	\begin{itemize}
		\item \textit{Trudeau1993}
		\item \textit{Golenkov2011}
	\end{itemize}
	
	Если при этом возникает неоднозначность, связанная с тем, что существует несколько работ того же автора в один год, то в конце ключа рекомендуется добавлять строчные латинские буквы a, b, c и так далее, например:
	
	\begin{itemize}
		\item \textit{Gribova2015a}
		\item \textit{Gribova2015b}
	\end{itemize}
	
	При формировании ключа для электронного источника или коллективной публикации, где невозможно выделить ключевого автора, рекомендуется формировать ключ из 1-2 английских слов или аббревиатур, позволяющих однозначно идентифицировать соответствующий источник. При использовании нескольких слов их можно соединять знаком нижнее подчеркивание, пробелы в ключах запрещены. При необходимости в конце ключа можно добавлять год издания. Например:
	
	\begin{itemize}
		\item \textit{IMS} (библиографическая ссылка на сайт Метасистемы OSTIS)
		\item \textit{CYPHER} (библиографическая ссылка на сайт с описанием языка Cypher)
		\item \textit{AIDictionary1992} (библиографическая ссылка на Словарь по искусственному интеллекту 1992 года издания)
	\end{itemize}

Для добавленного источника необходимо описать его идентификатор, который далее будет использоваться в рамках текста Стандарта. Это делается при помощи BibTex поля shorthand, например (см. \textit{Правила идентификации библиографических источников}):

\begin{verbatim}
shorthand = {Trudeau.R.J.IntroGT-1993кн}
shorthand = {Duchi.J..AdaptiveSubgradMethods-2011ст}
shorthand = {Грибова.В.В..БазоваяТРИСОП-2015ст}
\end{verbatim}

Далее этот идентификатор может использоваться как в формальном тексте, также как и идентификатор любой другой сущности, так и в рамках естественно-языкового текста. Для автоматической вставки идентификатора библиографического источника в формальный либо естественно-языковой текст используется команда \begin{verbatim}\scncite{<цитатный ключ>}\end{verbatim}

Пример исходного кода:

\begin{verbatim}
\scnheader{конвергенция\scnsupergroupsign}
\scnidtf{уровень конвергенции (близости)\scnsupergroupsign}
\scnsuperset{конвергенция кибернетических систем\scnsupergroupsign}
\begin{scnreltolist}{ключевой знак}
	\scnitem{\scncite{Yankovskaya2017}}
	\scnitem{\scncite{Palagin2013}}
	\scnitem{\scncite{Yankovskaya2010}}
	\scnitem{\scncite{Kovalchuk2011}}
\end{scnreltolist}		
\end{verbatim}

Результат компиляции:

\begin{SCn}
\scnheader{конвергенция\scnsupergroupsign}
\scnidtf{уровень конвергенции (близости)\scnsupergroupsign}
\scnsuperset{конвергенция кибернетических систем\scnsupergroupsign}
\begin{scnreltolist}{ключевой знак}
	\scnitem{\scncite{Yankovskaya2017}}
	\scnitem{\scncite{Palagin2013}}
	\scnitem{\scncite{Yankovskaya2010}}
	\scnitem{\scncite{Kovalchuk2011}}
\end{scnreltolist}
\end{SCn}

\item Для каждого источника крайне желательно добавить его краткую аннотацию. Это делается при помощи BibTex поля annotation, например:

\begin{verbatim}
annotation = {В этой книге представлены исследования по внедрению концептуальных основ, стратегий, методов, методологий, информационных платформ и моделей для разработки современных систем, основанных на знаниях}
\end{verbatim}

В рамках аннотации допускается использование средств форматирования естественно-языковых текстов, принятых в рамках Стандарта OSTIS, например, выделение курсивом или полужирным курсивом.

Для вставки аннотации в формальный scn-текст используется команда 

\begin{verbatim}
\scnciteannotation{<цитатный ключ>}
\end{verbatim}

Пример исходного кода:

\begin{verbatim}
\scnheader{\scncite{McBride2021}}
\scnciteannotation{McBride2021}
\end{verbatim}

Результат компиляции:

\scnheader{\scncite{McBride2021}}
\scnciteannotation{McBride2021}

\end{itemize}

\section*{Правила идентификации библиографических источников}

Идентификаторы статей, книг и других печатных работа строятся следующим образом:

\begin{itemize}
	\item Пишется фамилия первого автора на том языке, на котором опубликована данная работа. Затем через точку ставятся инициал(-ы) первого автора
	\item Если работа опубликована с участием только одного автора, то ставится одна точка, если нескольких - две точки
	\item Пишется первое слово из названия работы на том языке, на котором опубликована данная работа. Допускается сокращение, если слово очень длинное.
	\item Перечисляются Заглавные первые буквы всех остальных слов названия работы за исключением служебных слов, таких как предлоги, частицы, артикли и т.п.
	\item Ставится дефис
	\item Указывается год издания работы
	\item Указывается 2-3 буквенный код, обозначающий тип работы, например:
	\begin{itemize}
		\item \textit{кн} или \textit{bk} -- книга
		\item \textit{ст} или \textit{art} -- статья
	\end{itemize}
	
	Например:
	
	\textit{Мартынов.В.В.СемиологОИ-1974кн}

	Полное библиографическое описание:
	Мартынов В. В., Семиологические основы информатики, Минск, 1974
	
	\textit{Golenkov.V.V..MethodsTECCS-2019art}
	
	Полное библиографическое описание:
	Methods and tools for ensuring compatibility of computer systems / V. Golenkov [et al.] // Открытые семантические технологии проектирования интеллектуальных систем = Open Semantic Technologies for Intelligent Systems (OSTIS-2019) : материалы международной научно-технической конференции, Минск, 21 - 23 февраля 2019 г. / Белорусский государственный университет информатики и радиоэлектроники; редкол.: В. В. Голенков (гл. ред.) [и др.]. - Минск, 2019. - С. 25 - 52.
	
	\item Идентификаторы электронных и прочих ресурсов формируются аналогичным образом, с учетом того, что опускается год издания и фамилии авторов, а также ставится буквенный код \textit{эл} для обозначения электронного ресурса:
	
	\textit{МетасистIMS-эл} 
	\textit{Cypher-2017эл} 
	
\end{itemize}

\section*{Примеры библиографии в тексте}

Работа \scncite{Golenkov2018} посвящена вопросам обучения и обучаемости в интеллектуальных системах.

Работа \scncite{Wooldridge2009} описывает основные принципы многоагентных систем.

\section*{Примеры рисунков}

Рисунок \textit{\nameref{fig:example}} показывает, как надо оформлять рисунки. Размер рисунка можно задавать при помощи параметра scale.

\begin{figure}[H]
	\includegraphics[scale=0.5]{images/fig_example.png}
	\caption{Пример обычного рисунка}
	\label{fig:example}
\end{figure}

Рисунок \textit{\nameref{fig:example_scg}} показывает, как надо оформлять рисунки в SCg-коде. Параметр scale должен быть выставлен равным 0.8, для того чтобы все рисунки в SCg-коде имели одинаковый масштаб и размер идентификаторов примерно соответствовал размеру шрифта основного текста.  

\begin{figure}[H]
	\includegraphics[scale=0.8]{images/fig_example_scg.png}
	\caption{Пример рисунка в SCg-коде}
	\label{fig:example_scg}
\end{figure}


\begin{partbacktext}
\part{Введение в интеллектуальные компьютерные системы нового поколения и технологию комплексной поддержки их жизненного цикла}
\label{part1}

\begin{SCn}
	\scntext{аннотация}{Уточнение понятия интеллектуальной компьютерной системы. Требования, предъявляемые к интеллектуальным компьютерным системам следующего (нового) поколения и принципы, лежащие в их основе. Структура и особенности жизненного цикла интеллектуальных компьютерных систем нового поколения. Технология комплексной поддержки жизненного цикла интеллектуальных компьютерных систем нового поколения (\textit{Технология OSTIS} --- Open Semantic Technology for Intelligent Systems)}
	
	\bigskip
	
	\begin{scnrelfromlist}{подраздел}
		\scnitem{Глава~\ref{chap_intro}~\nameref{chap_intro}}
		\scnitem{Глава~\ref{chapter_new_generation_systems}~\nameref{chapter_new_generation_systems}}
		\scnitem{Глава~\ref{chapter_ostis_tech}~\nameref{chapter_ostis_tech}}
	\end{scnrelfromlist}
\end{SCn}

\end{partbacktext}

\chapauthor{Загорский А.С.\\Голенков В.В.\\Шункевич Д.В.}
\chapter{Факторы, определяющие уровень интеллекта кибернетических систем}
\chapauthortoc{Загорский А.С.\\Голенков В.В.\\Шункевич Д.В.}
\label{chapter_intro}

\abstract{Аннотация к главе.}

\section{Уровень интеллекта кибернетических систем}
\section{Уровень интеллекта многоагентных систем}
\label{section_mas}

\section{Уровень интеллекта компьютерных систем}

%\input{author/references}
\chapauthor{Голенков В.В.\\Шункевич Д.В.\\Ковалёв М.В.\\Садовский М.Е.}
\chapter{Интеллектуальные компьютерные системы нового поколения}
\chapauthortoc{Голенков В.В.\\Шункевич Д.В.\\Ковалёв М.В.\\Садовский М.Е.}
\label{chapter_new_generation_systems} 

\abstract{Аннотация к главе.}

\section{Требования, предъявляемые к интеллектуальным компьютерным системам нового поколения}
\section{Принципы, лежащие в основе смыслового представления информации}
\section{Принципы, лежащие в основе многоагентных моделей решателей задач, основанных на смысловом представлении информации}
\section{Принципы, лежащие в основе онтологических моделей мультимодальных интерфейсов интеллектуальных компьютерных систем нового поколения}
\section{Достоинства предлагаемых принципов, лежащих в основе интеллектуальных компьютерных систем нового поколения}

%\input{author/references}
\chapter{Принципы, лежащие в основе технологии комплексной поддержки жизненного цикла интеллектуальных компьютерных систем нового поколения}
\chapauthortoc{Голенков В.В.\\Шункевич Д.В.\\Ковалёв М.В.\\Садовский М.Е.}
\label{chapter_ostis_tech} 

\vspace{-7\baselineskip}

\begin{SCn}
\begin{scnrelfromlist}{автор}
	\scnitem{Голенков В.В.}
	\scnitem{Шункевич Д.В.}
	\scnitem{Ковалёв М.В.}
	\scnitem{Садовский М.Е.}
\end{scnrelfromlist}

\bigskip

\scntext{аннотация}{В главе рассмотрены принципы построения комплексной технологии разработки и поддержки жизненного цикла интеллектуальных компьютерных систем нового поколения -- Технологии OSTIS.}

\bigskip

\begin{scnrelfromlist}{подраздел}
	\scnitem{\ref{sec_ostis_tech}~\nameref{sec_ostis_tech}}
	\scnitem{\ref{sec_sem_compatible_os}~\nameref{sec_sem_compatible_os}}
\end{scnrelfromlist}

\end{SCn}

\section{Технология OSTIS (Open Semantic Technology for Intelligent Systems)}
\label{sec_ostis_tech}

\begin{SCn}
\begin{scnrelfromlist}{ключевой знак}
	\scnitem{Технология OSTIS}
	\scnitem{ostis-система}
	\scnitem{Стандарт ostis-систем}
	\scnitem{Метасистема OSTIS}
	\scnitem{Стандарт OSTIS}
	\scnitem{платформа ostis-систем}
	\scnitem{Экосистема OSTIS}
\end{scnrelfromlist}

\begin{scnrelfromlist}{ключевое знание}
	\scnitem{Обобщенный жизненный цикл ostis-систем}
	\scnitem{Принципы, лежащие в основе Технологии OSTIS}
\end{scnrelfromlist}	 

\scnheader{жизненный цикл интеллектуальной компьютерной системы нового поколения}
\begin{scnrelfromlistcustom}{включает в себя}
	\scnitemcustom{проектирование интеллектуальной компьютерной системы нового поколения}
	\begin{scnindent}
		\begin{scnrelfromlistcustom}{включает в себя}
			\scnitemcustom{проектирование базы знаний интеллектуальной компьютерной системы нового поколения}
			\scnitemcustom{проектирование решателя задач интеллектуальной компьютерной системы нового поколения}
			\scnitemcustom{проектирование интерфейса интеллектуальной компьютерной системы нового поколения}
		\end{scnrelfromlistcustom}
	\end{scnindent}
	\scnitemcustom{реализацию интеллектуальной компьютерной системы нового поколения}
	\scnitemcustom{начальное обучение интеллектуальной компьютерной системы нового поколения}
	\scnitemcustom{мониторинг качества интеллектуальной компьютерной системы нового поколения}
	\scnitemcustom{восстановление требуемого уровня интеллектуальной компьютерной системы нового поколения}
	\scnitemcustom{реинжиниринг интеллектуальной компьютерной системы нового поколения}
	\scnitemcustom{обеспечение безопасности интеллектуальной компьютерной системы нового поколения}
	\scnitemcustom{эксплуатация интеллектуальной компьютерной системы нового поколения конечными пользователями}
\end{scnrelfromlistcustom}
\end{SCn}

Построение \textit{Технологии} \textit{комплексной поддержки жизненного цикла интеллектуальных компьютерных систем нового поколения} предполагает:

\begin{textitemize}
	\item
	Четкое описание текущей версии \textit{стандарта интеллектуальных компьютерных систем нового поколения}, обеспечивающего семантическую совместимость разрабатываемых систем;
	\item
	Создание мощных библиотек семантически совместимых и многократно (повторно) используемых компонентов разрабатываемых \textit{интеллектуальных компьютерных систем};
	\item
	Уточнение требований, предъявляемых к создаваемой комплексной технологии и обусловленных особенностями \textit{интеллектуальных компьютерных систем нового поколения}, разрабатываемых и эксплуатируемых с помощью указанной технологии.
\end{textitemize}

Создание инфраструктуры, обеспечивающей интенсивное перманентное развитие \textit{Технологии} \textit{комплексной поддержки жизненного цикла интеллектуальных компьютерных систем нового поколения} предполагает:

\begin{textitemize}
	\item
	Обеспечение низкого порога вхождения в \textit{технологию проектирования интеллектуальных компьютерных систем} как для пользователей технологии (т.е. разработчиков прикладных или специализированных интеллектуальных компьютерных систем), так и для разработчиков самой технологии;
	\item
	Обеспечение высоких темпов развития \textit{технологии} за счет учета опыта разработки различных приложений путем активного привлечения авторов приложений к участию в развитии (совершенствовании) \textit{технологии}.
\end{textitemize}

В основе создания предлагаемой нами \textbf{\textit{технологии комплексной поддержки жизненного цикла интеллектуальных компьютерных систем нового поколения}}, лежат следующие положения:
\begin{textitemize}
	\item реализация предлагаемой \textit{технологии} разработки и сопровождения \textit{интеллектуальных компьютерных систем нового поколения} в виде \textbf{\textit{интеллектуальной компьютерной метасистемы}}, которая полностью соответствует \textit{стандартам} предлагаемых \textit{интеллектуальных компьютерных систем нового поколения}, разрабатываемым по предлагаемой \textit{технологии}. В состав такой \textit{интеллектуальной компьютерной метасистемы}, реализующей предлагаемую технологию входит:
	
	\begin{textitemize}
		\item  формальное онтологическое описание текущей версии \textit{стандарта интеллектуальных компьютерных систем нового поколения};
		\item  формальное онтологическое описание текущей версии \textit{методов и средств проектирования, реализации, сопровождения, реинжиниринга и эксплуатации интеллектуальных компьютерных систем нового поколения}.
	\end{textitemize}
	
	Благодаря этому технология проектирования и реинжиниринга интеллектуальных компьютерных систем нового поколения и технология проектирования и реинжиниринга самой указанной технологии (т.е. интеллектуальной компьютерной метасистемы) суть одно и тоже;
	
	\item \textbf{\textit{унификация}} и \textbf{\textit{стандартизация} интеллектуальных компьютерных систем нового поколения}, а также \textit{методов} их \textit{проектирования, реализации, сопровождения, реинжиниринга и эксплуатации};
	\item перманентная эволюция \textbf{\textit{стандарта интеллектуальных компьютерных систем нового поколения}}, а также \textit{методов} их \textit{проектирования, реализации, сопровождения, реинжиниринга и эксплуатации;}
	\item \textbf{\textit{онтологическое проектирование} интеллектуальных компьютерных систем нового поколения}, предполагающее:
	
	\begin{textitemize}
		\item  четкое согласование и оперативную формализованную фиксацию (в виде \textit{формальных онтологий}) утвержденного \textit{текущего состояния} иерархической системы всех \textit{понятий}, лежащих в основе перманентно эволюционируемого \textit{стандарта интеллектуальных компьютерных систем нового поколения}, а также в основе каждой разрабатываемой \textit{интеллектуальной компьютерной системы};
		
		\item  достаточно полное и оперативное документирование текущего состояния каждого проекта;
		
		\item  использование \textit{методики проектирования} \textit{"сверху-вниз"{}}.
	\end{textitemize}
	
	\item \textbf{\textit{компонентное проектирование} интеллектуальных компьютерных систем нового поколения}, т.е. проектирование, ориентированное на сборку \textit{интеллектуальных компьютерных систем} из готовых компонентов на основе постоянно расширяемых библиотек \textit{многократно используемых компонентов};
	
	\item  \textbf{\textit{комплексный характер}} предлагаемой \textit{технологии}, осуществляющей:
	
	\begin{textitemize}
		\item  поддержку \textit{проектирования} не только \textit{компонентов} \textit{интеллектуальных компьютерных систем нового поколения} (различных \textit{фрагментов баз знаний, баз знаний} в целом, различных \textit{методов решения задач}, различных \textit{внутренних информационных агентов, решателей задач} в целом, формальных онтологических описаний различных \textit{внешних языков}, \textit{интерфейсов} в целом), но также и \textit{интеллектуальных компьютерных систем} в целом как самостоятельных \textit{объектов проектирования} с учетом специфики тех классов, которым принадлежат проектируемые \textit{интеллектуальных компьютерных системы};
		
		\item  поддержку не только \textit{комплексного} \textit{проектирования} \textit{интеллектуальных компьютерных систем} \textit{нового поколения}, но также и поддержку их реализации (сборки, воспроизводства), сопровождения, реинжиниринга в ходе эксплуатации и непосредственно самой эксплуатации.
		
	\end{textitemize}
\end{textitemize}

Для создания \textit{технологии} комплексного проектирования и комплексной поддержки последующих этапов жизненного цикла \textit{интеллектуальных компьютерных систем нового поколения} необходимо:

\begin{textitemize}
	\item  Унифицировать формализацию различных моделей представления различного вида используемой информации, хранимой в памяти \textit{интеллектуальных компьютерных систем} и различных моделей решения интеллектуальных задач для обеспечения \textit{семантической совместимости} и простой автоматизируемой интегрируемости различных видов \textit{знаний} и \textit{моделей решения задач} в \textit{интеллектуальных компьютерных системах}. Для этого необходимо разработать базовую \textit{универсальную} абстрактную модель представления и обработки знаний, обеспечивающую реализацию всевозможных моделей решения задач.
	
	\item  Унифицировать структуризацию \textit{баз знаний} интеллектуальных компьютерных систем в виде иерархической системы онтологий разного уровня.
	
	\item  Унифицировать систему используемых \textit{понятий}, специфицируемых соответствующими \textit{онтологиями} для обеспечения \textit{семантической совместимости} и \textit{интероперабельности} различных \textit{интеллектуальных компьютерных систем}.
	
	\item  Унифицировать архитектуру \textit{интеллектуальных компьютерных систем}, обеспечивающую \textit{семантическую совместимость}:
	
	\begin{textitemize}
		\item между \textit{интеллектуальными компьютерными системами} и их пользователями
		\item между \textit{индивидуальными интеллектуальными компьютерными системами};
		\item между \textit{коллективными интеллектуальными компьютерными системами},
	\end{textitemize}
	
	а также обеспечивающую \textit{интероперабельность} сообществ, состоящих из:
	
	\begin{textitemize}
		\item \textit{индивидуальных интеллектуальных компьютерных систем};
		\item \textit{коллективных интеллектуальных компьютерных систем};
		\item пользователей интеллектуальных компьютерных систем
	\end{textitemize}
	
	\item Разработать \textit{базовую модель интерпретации} всевозможных формальных моделей решения задач в интеллектуальных компьютерных системах с ориентацией на максимально возможное упрощение такой интерпретации в \textit{компьютерах нового поколения}, которые специально предназначены для реализации индивидуальных \textit{интеллектуальных компьютерных систем}.
	\item Разработать \textit{компьютеры нового поколения}, принципы функционирования которых максимально близки к базовой абстрактной модели, обеспечивающей интеграцию всевозможных моделей представления знаний и моделей решения задач. При этом базовая машина обработки информации, лежащая в основе указанных компьютеров, должна существенно отличаться от фон-Неймановской машины и должна быть близка к базовой модели решения задач в интеллектуальных компьютерных системах для того, чтобы существенно снизить сложность интерпретации всего многообразия моделей решения задач в интеллектуальных компьютерных системах.
	
\end{textitemize}

Реализация всех перечисленных этапов развития \textit{технологий Искусственного интеллекта} представляет собой переход на принципиально новый технологический уклад, обеспечивающий существенное повышение эффективности практического использования результатов работ в области \textit{Искусственного интеллекта} и существенное повышение уровня автоматизации \textit{человеческой деятельности.}

Предложенную нами \textit{технологию комплексной поддержки жизненного цикла интеллектуальных компьютерных систем нового поколения} мы назвали \textbf{\textit{Технологией OSTIS}} (Open Semantic Technology for Intelligent Systems). Соответственно этому \textit{интеллектуальные компьютерные системы нового поколения}, разрабатываемые по этой технологии называются \textbf{\textit{ostis-системами}}. Сама \textit{Технология OSTIS} реализуется нами в форме специальной \textit{ostis-системы}, которая названа нами \textbf{\textit{Метасистемой OSTIS}} и \textit{база знаний} которой содержит:

\begin{textitemize}
	\item Формальную теорию \textit{ostis-систем};
	\item Стандарт \textit{ostis-систем}
	\begin{textitemize}  
		\item Стандарт баз знаний \textit{ostis-систем}
		\begin{textitemize}  
			\item Стандарт внутреннего универсального языка смыслового представления знаний в памяти \textit{ostis-систем}
			\item Стандарт внутреннего представления онтологий верхнего уровня в памяти \textit{ostis-систем}
			\item Стандарт представления исходных текстов баз знаний \textit{ostis-систем}
		\end{textitemize}  
		
		\item Стандарт решателей задач \textit{ostis-систем}
		\begin{textitemize}  
			\item Стандарт базового языка программирования \textit{ostis-систем}
			\item Стандарт языков программирования высокого уровня для \textit{ostis-систем}
			\item Стандарт представления искусственных нейронных сетей в памяти \textit{ostis-систем}
			\item Стандарт внутренних информационных агентов в \textit{ostis-систем}
		\end{textitemize}  
		
		\item Стандарт интерфейсов \textit{ostis-систем}
		\begin{textitemize}  
			\item Стандарт внешних языков \textit{ostis-систем}, близких к внутреннему универсальному языку смыслового представления знаний
		\end{textitemize}  
	\end{textitemize}
	\item Стандарт ostis-систем и Технологии OSTIS (\textbf{\textit{Стандарт OSTIS}}) \cite{Standart2021};
	\item Ядро Библиотеки многократно используемых компонентов \textit{ostis-систем} (\textbf{\textit{Библиотеки OSTIS}});
	\item Методики и \textit{инструментальные средства поддержки жизненного цикла} \textit{ostis-систем} и их компонентов.
\end{textitemize}

\begin{SCn}
	
\scnheader{Технология OSTIS}
\scnidtf{Предлагаемая нами комплексная технология поддержки всех этапов жизненного цикла всех компонентов для всех классов (видов) интеллектуальных компьютерных систем нового поколения при перманентной поддержке их семантической совместимости}
\begin{scnrelfromlistcustom}{принципы, лежащие в основе}
	\scnitemcustom{Комплексный характер технологии, заключающийся в том, что осуществляется поддержка: 
	\begin{itemize}[labelsep=\tabsize-\bulletsize,leftmargin=\tabsize,label=$\bullet$]
	\item всех этапов жизненного цикла создаваемых продуктов
	\item для всех компонентов интеллектуальных компьютерных систем нового поколения
	\item для всех классов интеллектуальных компьютерных систем нового поколения
	\end{itemize}}

	\scnitemcustom{Обеспечивается перманентная поддержка семантической совместимости между всеми создаваемыми интеллектуальными компьютерными системами нового поколения}		
	\scnitemcustom{Ориентация на комплексную автоматизацию всего многообразия человеческой деятельности}		
	\scnitemcustom{Реализация технологии и, соответственно, комплексная автоматизация поддержки жизненного цикла интеллектуальных компьютерных систем нового поколения (со всеми их компонентами и классами) осуществляется в виде семейства интеллектуальных компьютерных систем нового поколения, построенных по той же технологии}
\end{scnrelfromlistcustom}
	
\scnheader{ostis-система}
\begin{scnrelfromset}{разбиение}
	\scnitem{ostis-субъект}
	\begin{scnindent}
		\scnidtf{самостоятельная \textit{ostis-система}}
		\begin{scnrelfromset}{разбиение}
			\scnitem{индивидуальная ostis-система}
			\scnitem{коллективная ostis-система}
		\end{scnrelfromset}
	\end{scnindent}
	\scnitem{встроенная ostis-система}
	\begin{scnindent}
		\scnidtf{\textit{ostis-система}, являющаяся частью некоторой \textit{индивидуальной ostis-системы}}
	\end{scnindent}
\end{scnrelfromset}

\scnheader{индивидуальная ostis-система}
\scnidtf{минимальная самостоятельная \textit{ostis-система}}
\begin{scnrelfromset}{разбиение}
	\scnitem{персональный ostis-ассистент}
	\begin{scnindent}
		\scnidtf{\textit{ostis-система}, осуществляющая комплексное адаптивное обслуживание конкретного пользователя по \textit{всем} вопросам, касающимся его взаимодействия с любыми другими \textit{ostis-системами}, а также представляющая интересы этого пользователя во всей глобальной сети \textit{ostis-систем}}
	\end{scnindent}
	\scnitem{корпоративная ostis-система}
	\begin{scnindent}
		\scnidtf{\textit{ostis-система}, осуществляющая координацию совместной деятельности \textit{ostis-систем} в рамках соответствующего коллектива \textit{ostis-систем}, осуществляющая мониторинг и реинжиниринг соответствующего множества \textit{ostis-систем} и представляющая интересы этого коллектива в рамках других коллективов \textit{ostis-систем}}
	\end{scnindent}
	\scnitem{индивидуальная ostis-система, не являющаяся ни персональным ostis-ассистентом, ни корпоративной ostis-системой}
\end{scnrelfromset}

\scnheader{коллективная ostis-система}
\scnidtf{многоагентная система, представляющая собой коллектив индивидуальных и коллективных \textit{ostis-систем}, деятельность которого координируется соответствующей корпоративной \textit{ostis-системой}}
\scntext{примечание}{В состав коллектива \textit{ostis-систем} могут входить индивидуальные \textit{ostis-системы} могут входить индивидуальные \textit{ostis-системы} любого вида -- в том числе, корпоративные \textit{ostis-системы}, представляющие интересы других коллективов \textit{ostis-систем}}

\end{SCn}

\section{Семантически совместимые ostis-системы}
\label{sec_sem_compatible_os}
%\input{author/references}
%\begin{partbacktext}
\part{Смысловое представление и онтологическая систематизация знаний в интеллектуальных компьютерных системах нового поколения}
\noindent Описание к главе
\end{partbacktext}

\begin{SCn}
	\scntext{аннотация}{
		Описание видов языков и информационных конструкций в рамках соответствующей онтологии. Уточнение модели смыслового представления знаний, использующей понятие смыслового пространства. Рассмотрение способов внутреннего и внешнего представления информации ostis-систем. Онтологическая систематизация фактографической и логической информации. Описание структуры баз знаний. Средства описания (спецификации) денотационной семантики и синтаксиса естественных языков в ostis-системах.
	}
	\bigskip
	
	\begin{scnrelfromlist}{подраздел}
		\scnitem{Глава~\ref{chapter_inf_constr}~\nameref{chapter_inf_constr}}
		\scnitem{Глава~\ref{chapter_sc_code}~\nameref{chapter_sc_code}}
		\scnitem{Глава~\ref{chapter_ext_lang}~\nameref{chapter_ext_lang}}
		\scnitem{Глава~\ref{chapter_top_ontologies}~\nameref{chapter_top_ontologies}}
		\scnitem{Глава~\ref{chapter_kb}~\nameref{chapter_kb}}
		\scnitem{Глава~\ref{chapter_logic}~\nameref{chapter_logic}}
		\scnitem{Глава~\ref{chapter_lang}~\nameref{chapter_lang}}
	\end{scnrelfromlist}
	
\end{SCn}

\chapauthor{Никифоров С.А.\\Гойло А.А.\\Бобёр Е.С.}
\chapter{Информационные конструкции и языки}
\chapauthortoc{Никифоров С.А.\\Гойло А.А.\\Бобёр Е.С.}
\label{chapter_inf_constr}

\abstract{Аннотация к главе.}

\section{Формализация понятия информационной конструкции}
\section{Файлы ostis-систем}

%\input{author/references}
\chapauthor{Ивашенко В.П.\\Голенков В.В.}
\chapter{Смысловое представление знаний в памяти ostis-систем}
\chapauthortoc{Ивашенко В.П.\\Голенков В.В.}
\label{chapter_sc_code}

\abstract{Аннотация к главе.}

%разбиение
%пересечение множеств
\begin{SCn}
	\begin{scnrelfromlist}{ключевое понятие}
		\scnitem{sc-элемент}
%		\scnitem{Структурный признак классификации sc-элементов}
		\scnitem{обозначение sc-множества}
		\scnitem{обозначение sc-связки}
		\scnitem{обозначение sc-синглетона}
		\scnitem{обозначение sc-пары}		
		\scnitem{обозначение неориентированной sc-пары}		
		\scnitem{обозначение ориентированной sc-пары}		
		\scnitem{обозначение sc-пары принадлежности}		
		\scnitem{обозначение sc-пары нечёткой принадлежности}		
		\scnitem{обозначение sc-пары позитивной принадлежности}
		\scnitem{обозначение sc-пары негативной принадлежности}		
		\scnitem{обозначение ориентированной sc-пары, не являющейся sc-парой принадлежности}		
		\scnitem{обозначение sc-связки, не являющейся синглетоном и парой}		
		\scnitem{обозначение sc-класса}		
		\scnitem{обозначение sc-структуры}		
		\scnitem{обозначение внешней сущности}		
		\scnitem{обозначение внутреннего файла}		
		\scnitem{обозначение внешней сущности, не являющейся внутренним файлом}		
		\scnitem{обозначение постоянной сущности}
		\scnitem{обозначение временной сущности}				
		\scnitem{обозначение статической сущности}
		\scnitem{обозначение динамической сущности}				
		\scnitem{sc-константа}
		\scnitem{sc-переменная}				
		\scnitem{sc-множество}
		\scnitem{sc-связка}
		\scnitem{sc-синглетон}
		\scnitem{sc-пара}
		\scnitem{неориентированная sc-пара}
		\scnitem{ориентированная sc-пара}
		\scnitem{sc-пара принадлежности}
		\scnitem{sc-пара нечёткой принадлежности}
		\scnitem{sc-пара позитивной принадлежности}
		\scnitem{sc-пара негативной принадлежности}
		\scnitem{ориентированная sc-пара принадлежности, не являющаяся sc-парой принадлежности}
		\scnitem{sc-связка,не являющаяся синглетоном и парой}
		\scnitem{sc-класс}						
		\scnitem{sc-структура}						
		\scnitem{внешняя сущность}				
		\scnitem{внутренний файл}				
		\scnitem{внешняя сущность, не являющаяся внутренним файлом}				
	\end{scnrelfromlist}
\end{SCn}
%ключевые понятия:

%sc-элемент

\begin{SCn}
	\begin{scnrelfromlist}{подраздел}
		\scnitem{\ref{sec_sr_sccode}~\nameref{sec_sr_sccode}}
		\scnitem{\ref{sec_sr_scdsemantics}~\nameref{sec_sr_scdsemantics}}
		\scnitem{\ref{sec_sr_scsyntax}~\nameref{sec_sr_scsyntax}}
		\scnitem{\ref{sec_sr_ostisfiles}~\nameref{sec_sr_ostisfiles}}
		\scnitem{\ref{sec_sr_semspace}~\nameref{sec_sr_semspace}}
	\end{scnrelfromlist}
\end{SCn}


\section{SC-код (Semantic Computer Code)}
\label{sec_sr_sccode}
Общие положения

Ниже (в \textit{\ref{sec_sr_scdsemantics}~\nameref{sec_sr_scdsemantics}} и в \textit{\ref{sec_sr_scsyntax}~\nameref{sec_sr_scsyntax}}) приведено формальное описание денотационной семантики и синтаксиса SC-кода. Но сделано это будет не совсем обычно. Поскольку все элементы конструкции являются обозначениями описываемых сущностей и, в том числе, обозначениями различных выделяемых классов \textit{sc-элементов}, (!)нам ничего не стоит(!) явно ввести различные семантически значимые и синтаксически выделяемые классы \textit{sc-элементов} и на основе этого явно описать средствами SC-кода базовую денотационную семантику и синтаксис SC-кода. Синтаксис SC-кода задаётся семейством классов синтаксический задаваемых (выделяемых) \textit{sc-элементов}. 

Элементы, принадлежащие каждому синтаксически выделяемому классу \textit{sc-элементов} должны иметь одинаковые синтаксические признаки (синтаксические метки). При этом очевидно, что синтаксис SC-кода существенно упростится, если синтаксически выделяемые классы sс-элементов будут одновременно иметь и чёткую семантическую интерпретацию. Таким образом, формализацию синтаксиса SC-кода целесообразно осуществлять после формализации базовой денотационной семантики SC-кода. Путём синтаксического выделения тех семантически выделенных классов \textit{sc-элементов}, которые, во-первых, необходимы для кодирования sc-конструкций памяти ostis-систем (в sc-памяти) и во-вторых, обеспечивают максимально возможное упрощение обработки sc-конструкций (например, путём упрощения анализа часто проверяемых семантических характеристик обрабатываемых \textit{sc-элементов}).

Особенности SC-кода как одного из возможных вариантов смыслового представления знаний (см.\textit{~\ref{sec_ngics_sense_principles}~\nameref{sec_ngics_sense_principles}}) 

Понятие \textit{sc-элемента} 

Понятие sc-конструкции 

Понятие внутреннего файла ostis-систем 

Понятие sc-идентификатора

\newpage
\section{Базовая денотационная семантика SC-кода}
\label{sec_sr_scdsemantics}

\begin{SCn}
	\begin{scnrelfromlist}{подраздел}
		\scnitem{Пункт 2.2.2.1. Семантическая классификация sc-элементов по базовым признакам}
		\scnitem{Пункт 2.2.2.2. Уточнение смысла выделенных классов sc-элементов и связей между этими классами}
		\scnitem{Пункт 2.2.2.3. Структура базовой семантической спецификации sc-элемента}
		\scnitem{Пункт 2.2.2.4. Онтологическая формализация Базовой денотационной семантики SC-кода}
	\end{scnrelfromlist}
\end{SCn}
	
%\begin{SCn}
%	\scntext{содержание}{
%		Структурная классификация sc-элементов с пояснением смысла понятий, используемых в структурной классификации элементов 
%		
%		Логико-семантическая классификация sc-элементов с пояснением смысла вводимых понятий 
%		
%		Классификация sc-элементов по темпоральным характеристикам обозначаемых ими сущностей с пояснением смысла вводимых понятий 
%		
%		Понятие базовой спецификации sc-элементов 
%		
%		Онтологическая формализация базовой денотационной семантики SC-кода 
%	}

%	В основе базовой денотационной семантики SC-кода лежит:
%	\begin{itemize} 
%		\item семантическая классификация sc-элементов по основным семантически значимым признакам 
%		\item уточнение структуры базовой семантической спецификации элементов
%		\item …
%	\end{itemize}
%\end{SCn}

\newpage
\subsection{Семантическая классификация sc-элементов по базовым признакам}
К числу базовых признаков классификации \textit{sc-элементов} относятся:

\begin{textitemize}
	\item структурный признак;
	\item логико-семантический признак;
	\item темпоральная характеристика сущностей, обозначаемых \textit{sc-элементами}, которая в свою очередь, включает в себя:
	\begin{textitemize}
		\item постоянство или временность существования обозначаемой сущности;
		\item статичность (стационарность) или динамичность (изменчивость) обозначаемой сущности.
	\end{textitemize}
\end{textitemize}

\begin{SCn}
	\scnheader{Структурная классификация sc-элементов}
	\scnstartstruct
	
	\scnheader{sc-элемент}
	\scnrelfrom{разбиение}{Структурный признак классификации sc-элементов}
	\begin{scnindent}
		\begin{scneqtoset}
			\scnitem{обозначение sc-множества}
			\begin{scnindent}
			\begin{scnsubdividing}
				\scnitem{обозначение sc-связки}
				\begin{scnindent}
				\begin{scnsubdividing}
					\scnitem{обозначение sc-синглетона}
					\scnitem{обозначение sc-пары}
					\begin{scnindent}
					\begin{scnsubdividing}
						\scnitem{обозначение неориентированной sc-пары}
						\scnitem{обозначение ориентированной sc-пары}
						\begin{scnindent}
							\begin{scnsubdividing}
								\scnitem{обозначение sc-пары принадлежности}
								\begin{scnindent}
								\begin{scnsubdividing}
									\scnitem{обозначение sc-пары нечеткой принадлежности}
									\scnitem{обозначение sc-пары позитивной принадлежности}
									\scnitem{обозначение sc-пары негативной принадлежности}
								\end{scnsubdividing}
								\end{scnindent}
								\scnitem{обозначение ориентированной sc-пары, не являющейся парой принадлежности}
							\end{scnsubdividing}
						\end{scnindent}
					\end{scnsubdividing}
					\end{scnindent}
					\scnitem{обозначение sc-связки, не являющейся синглетоном и парой}
				\end{scnsubdividing}
				\end{scnindent}
				\scnitem{обозначение sc-класса}
				\scnitem{обозначение sc-структуры}
			\end{scnsubdividing}
			\end{scnindent}
			\scnitem{обозначение внешней сущности}
			\begin{scnindent}
			\begin{scnsubdividing}
					\scnitem{внутренний файл}
					\scnitem{внешняя сущность, не являющаяся внутренним файлом}
					\scnitem{обозначение файла}
					\scnitem{обозначение информационной конструкции, не являющейся ни sc-множеством, ни файлом}
					\scnitem{обозначение внешней сущности, не являющейся информационной конструкцией}
			\end{scnsubdividing}
			\end{scnindent}
		\end{scneqtoset}
	\end{scnindent} 

	\scnendstruct \scnsourcecommentinline{Завершили представление \textit{Структурной классификации sc-элементов}}
\end{SCn}

\begin{comment}
Поясним смысл понятий, структурной
классификации sc-элементов 


\begin{SCn}
	\scnheader{sc-элемент}
	\scnidtf{sc-элемент, обозначающий множество}
	\scnidtf{sc-обозначение множества}
	\scnidtf{множество, представимое в SC-коде}
	\scnidtf{множество} 
	
	\bigskip
	
	\scnsourcecommentinline{так как любое множество можно
		представить в виде sc-множества}
	
	\begin{scnrelfromlist}{примечание}
		\scnfileitem{Каждый sc-элемент является обозначением
			соответствующего множества}
		\scnfileitem{Строго говоря, не каждое множество может быть
			обозначено соответствующим sc-элементом. 
			К таким множествам относятся либо множества,
			элементами которых являются sc-элементы
			(sc-множества), либо синглетоны элементами
			которых являются сущности, не являющиеся
			элементами (синглетоны внешних сущностей). Но
			каждое множество, не являющийся sc-множеством
			или синглетоном указанного вида, может быть
			однозначно преобразовано в sc-множество и
			описано средствами SC-кода. При этом
			теоретико-множественные свойства
			"нестандартных" для SC-кода множеств совпадают
			со свойствами соответствующих им
			"стандартных" для SC-кода множеств.}
	\end{scnrelfromlist}
\end{SCn}

\begin{SCn}
	\scnheader{sc-элемент}
	\scnidtf{обозначение множества}
	\scnnote{Тот факт, что каждый sc-элемент является
		обозначением соответствующего множества
		(частым случаем которых являются синглетоны
		внешних описываемых сущностей), означает то,
		что базовым видом объектов, которыми
		оперирует SC-код на синтаксическом,
		семантическом и логическом уровне являются
		множества знаков, обозначающих различные
		множества. 
		В этом смысле SC-код имеет базовую
		теоретико-множественную основу.}
\end{SCn}

\begin{SCn}
	\scnheader{sc-элемент}
	\begin{scnrelfromset}{разбиение}
		\scnitem{обозначение sc-множества}
		\begin{scnindent}
			\scnidtf{обозначение множества sc-элементов}
			\scnidtf{обозначение множества, все элементы
				которого являются sc-элементами}
		\end{scnindent}
		\scnitem{обозначение внешней сущности}
		\begin{scnindent}
			\scnidtf{обозначение синглетона внешней сущности}
			\scnidtf{терминальный sc-элемент}
			\scnidtf{синглетон внешней сущности}
		\end{scnindent}
	\end{scnrelfromset}	
\end{SCn}

\begin{SCn}
	\scnheader{sc-множество}
	\scnidtf{sc-элемент, обозначающий множество
		sc-элементов}
	\scnidtf{sc-обозначение множества sc-элементов}
	\scnidtf{множество sc-элементов}
	\scnidtf{множество, каждый элемент которого
		является sc-элементом}
	\scnsubset{sc-элемент}
	\begin{scnindent}
		\scnidtf{множество, представимое в SC-коде}
		
		\scneq{\normalfont{(}sc-множество $\cup$ синглетон внешней сущности\normalfont{)}}
		
		\scnidtf{информационная конструкция SC-кода}
		\scnidtftext{часто используемый sc-идентификатор}
		{sc-конструкция}
		
	\end{scnindent}
\end{SCn}

\begin{SCn}
	\scnheader{sc-связка}
	\scnidtf{sc-элемент, обозначающий связку sc-элементов}
	\scnidtf{sc-обозначение связки sc-элементов}
	\scnidtf{обозначение sc-связки}
\end{SCn}

\begin{SCn}
	\scnheader{sc-связка}
	\scnidtf{обозначение связи между sc-элементами}
	\scnsuperset{отображение связи между сущностями, которые
		обозначаются sc-элементами, связанными sc-связкой}
\end{SCn}

\begin{SCn}
	\scnheader{sc-синглетон}
	\scnidtf{sc-множество, являющееся синглетоном}
	\scnidtf{одномощное sс-множество}
	\scnidtf{sc-множество, имеющее мощность, равную
		единице}
	\scnidtf{sc-элемент, обозначающий унарную sc-связку}
	\scnidtf{sc-обозначение унарной sc-связки}
	\scnidtf{унарная sc-связка}
	\scnidtf{обозначение sc-синглетона}
	\scnidtf{обозначение одномощного множества,
		единственный элемент которого является}
	sc-элементом]
\end{SCn}

\begin{SCn}
	\scnheader{sc-пара}
\end{SCn}

\begin{SCn}
	\scnheader{неориентированная sc-пара}
\end{SCn}

\begin{SCn}
	\scnheader{ориентированная sc-пара}
\end{SCn}

\begin{SCn}
	\scnheader{sc-пара принадлежности}
\end{SCn}

\begin{SCn}
	\scnheader{sc-пара нечёткой принадлежности}
\end{SCn}

\begin{SCn}
	\scnheader{sc-пара позитивной принадлежности}
\end{SCn}

\begin{SCn}
	\scnheader{sc-пара константной постоянной позитивной
		принадлежности}
	\scnidtf{константная позитивная постоянная sc-пара
		принадлежности}
\end{SCn}

\begin{SCn}
	\scnheader{sc-пара константной временной позитивной
		принадлежности}
\end{SCn}

\begin{SCn}
	\scnheader{sc-пара негативной принадлежности}
\end{SCn}

\begin{SCn}
	\scnheader{sc-пара, не являющаяся парой принадлежности}
\end{SCn}

\begin{SCn}
	\scnheader{sc-связка, не являющаяся синглетоном и парой}
\end{SCn}

\begin{SCn}
	\scnheader{sc-класс}
	\scnnote{Требованиями, предъявляемыми к каждому
		sc-классу являются:
		\begin{itemize}
			\item бесконечность этого sc-множества 
			\item наличие общего свойства, присущего всем
			элементам этого sc-множества, в частности,
			наличие его определения
	\end{itemize} }
\end{SCn}

\begin{SCn}
	\scnheader{sc-класс}
	\scnsuperset{sc-отношение}
	
	\begin{scnindent}
		\scnidtf{sc-класс sc-связок}
		\scnsuperset{бинарное sc-отношение}
		
		\begin{scnindent}
			\begin{scnrelfromset}{разбиение}
				\scnitem{бинарное неориентированное sc-отношение}
				\scnitem{бинарное ориентированное sc-отношение}
				\begin{scnindent}
					\scnsuperset{ролевое sc-отношение}
				\end{scnindent}		
			\end{scnrelfromset}	
		\end{scnindent}
	\end{scnindent}
	
	\scnsuperset{sc-класс sc-классов}
	
	\scnsuperset{sc-параметр}
	
	
	\scnsuperset{sc-класс sc-структур}
	
	
	\scnsuperset{sc-класс эквивалентности}
	\begin{scnindent}
		\scnidtf{фактор-множество соответствующего
			отношения эквивалентности}
	\end{scnindent}
	
	\scnsuperset{sc-класс внешних сущностей}
	
	\scnsuperset{sc-класс внутренних файлов}
	
	
	\begin{scnrelfromset}{следует отличать}
		\scnitem{sc-класс}
		\scnitem{sc-связка}
	\end{scnrelfromset}	
	
	
	\scntext{сравнение}
	{В отличие от sc-связки принципом формирования
		является наличие общего свойства, присущего
		всем элементам этого sc-класса и только им (или
		присущего всем сущностям, которые
		обозначаются указанными sc-элементами). Таким
		общим свойством может быть определение
		sc-класса либо принадлежность одному из
		значений некоторого параметра, то есть одному
		из элементов фактор-множества,
		соответствующего некоторому отношению
		эквивалентности или толерантности}
	
	\scntext{сравнение}{Примерами sc-классов являются:
		\begin{itemize}
			\item конкретная окружность (множество всех
			точек, равноудалённых от некоторой заданной
			точки)
			\item конкретный отрезок (множество всех точек,
			лежащих между двумя заданными точками, с
			включением этих точек)
			\item конкретный линейный треугольник (множество
			всех точек, лежащих между каждыми двумя из
			трёх заданных точек, с включением этих точек)
		\end{itemize} 
		и, соответственно этому, примерами sc-связок
		являются: 
		\begin{itemize}
			\item пары граничных точек различных отрезков
			\item тройки вершин различных треугольников
		\end{itemize} 
	}
	\begin{scnrelfromset}{следует отличать}
		\scnitem{sc-связка попарно эквивалентных сущностей}
		\scnitem{sc-класс эквивалентности}
		
		\begin{scnindent}
			\scntext{пояснение}{В sc-класс входит не просто некоторые количество попарно эквивалентных между собой сущностей, а абсолютно все такие сущности}
		\end{scnindent}		
	\end{scnrelfromset}	
	
	Приведём пример: 
	\begin{scnrelfromset}{следует отличать}
		\scnitem{множество всех треугольников, подобных
			одному из них}
		\begin{scnindent}
			\scnsubset{sc-класс}
		\end{scnindent}
		\scnitem{конечное множество подобных треугольников}
		\begin{scnindent}
			\scnsubset{sc-связка попарно эквивалентных треугольников}		
		\end{scnindent}
	\end{scnrelfromset}	
\end{SCn}

\begin{SCn}
	\scnheader{sc-структура}
	
	\begin{scnrelfromset}{следует отличать}
		\scnitem{sc-структура}
		\scnitem{sc-связка}
	\end{scnrelfromset}	
	
	\scntext{сравнение}{В отличие от sc-связок в каждую sc-структуру
		должна входить по крайней мере одна sc-связка
		вместе с компонентами этой sc-связки}
\end{SCn}

\begin{SCn}
	\scnheader{внешняя сущность}
	\scnidtf{синглетон внешней сущности}
	\scnidtf{обозначение синглетона внешней сущности}
	\scnidtf{sc-элемент, обозначающий синглетон,
		элементом которого является некоторая
		внешняя описываемое сущность}
	\scnidtf{множество, обозначаемое sc-элементом,
		являющиеся одномощным множеством,
		единственным элементом которого является
		сущность, внешняя по отношению к
		sc-конструкции, то есть сущность, не являющиеся
		sc-элементом}
	\scnnote{обозначение синглетона внешний сущности, то
		есть sc-элемент, обозначающий этот синглетон,
		можно также трактовать как sc-элемент,
		обозначающий соответствующую внешнюю
		описываемую сущность, которую, в свою очередь,
		можно считать денотатом указанного
		sc-элемента}
	
	\scnnote{Очевидно, что пара принадлежности,
		связывающая sc-элемент, обозначающий
		синглетон внешней сущности, не может быть
		непосредственно представлена в виде
		соответствующей sc-дуги принадлежности, так
		как второй компонент этой sc-дуги не находится
		в sc-памяти}
	
	\begin{scnrelfromset}{следует отличать}
		\scnitem{синглетон внешней сущности}
		\scnitem{sc-синглетон}
		\begin{scnindent}	
			\scnidtf{sc-синглетон, единственным элементом
				которого является некоторый sc-элемент}
			
			\scnsubset{sc-множество}
			\begin{scnindent}
				\scnidtf{sc-элемент, обозначающий множество,
					элементами которого являются только
					sc-элементы}
				\scnidtf{множество sc-элементов}
			\end{scnindent}
		\end{scnindent}
	\end{scnrelfromset}	
\end{SCn}

\begin{SCn}
	\scnheader{внутренний файл}
	\scnidtf{внутренний файл ostis-системы}
	\scnidtf{внутренний образ (копия), внешней
		информационной конструкции, хранимый в
		файловой памяти ostis-системы}
	\scnnote{Файловая память ostis-системы, хранящая
		различного рода информационные конструкции
		(образы, модели), не являющиеся
		sc-конструкциями, должна быть тесно связана с
		sc-памятью этой же ostis-системы. Как минимум
		каждый файл ostis-системы должен быть связан с
		тем sc-узлом, который является знаком этого
		файла (точнее, знаком синглетона, элементом
		которого является указанный файл)}
\end{SCn}

\begin{SCn}
	\scnheader{внешняя сущность, не являющаяся внутренним
		файлом}
\end{SCn}
\end{comment}

\vskip 5cm

\begin{SCn}
	\scnheader{Логико-семантическая классификация sc-элементов}
	\scnstartstruct
	
	\scnheader{sc-элемент}
	\begin{scnsubdividing}
		\scnitem{sc-константа}
		\begin{scnindent}
			\scnidtf{sc-элемент, логико-семантическим значением которого является он сам}
		\end{scnindent}
		\scnitem{sc-переменная}
		\begin{scnindent}
			\begin{scnsubdividing}
				\scnitem{sc-переменная 1-го уровня}
				\begin{scnindent}
					\scnidtf{sc-элемент, областью возможных значений которого является множество sc-констант}
				\end{scnindent}
				\scnitem{sc-переменная 2-го уровня}
				\begin{scnindent}
					\scnidtf{sc-элемент, возможными значениями которого являются переменные 1-го уровня}
				\end{scnindent}
				\scnnote{такие переменные (метапеременные) необходимы для описания логических языков}
				\scnitem{sc-переменная универсального типа}
				\begin{scnindent}
					\scnidtf{sc-переменная, на значения которой не накладывается никаких ограничений}
				\end{scnindent}
			\end{scnsubdividing}
		\end{scnindent}
	\end{scnsubdividing}	
	
	\scnendstruct \scnsourcecommentinline{Завершили представление \textit{Логико-семантической классификации sc-элементов}}
\end{SCn}

\begin{SCn}
	\scnheader{Классификация sc-элементов по темпоральным характеристикам обозначаемых ими сущностей}
	\scnstartstruct
	
	\scnheader{sc-элемент}
	\scnrelfrom{разбиение}{Признак постоянства существования сущностей, обозначаемых sc-элементами}
	\begin{scnindent}
		\begin{scneqtoset}
			\scnitem{обозначение постоянной сущности}
			\begin{scnindent}
				\scnidtf{обозначение постоянно существующей сущности}
			\end{scnindent}
			\scnitem{обозначение временной сущности}
			\begin{scnindent}
				\begin{scnsubdividing}
					\scnitem{обозначение внешней временной сущности}
					\begin{scnindent}
						\scnsuperset{обозначение внешней ситуации}
						\scnsuperset{обозначение внешнего события}
						\scnsuperset{обозначение внешнего процесса}
					\end{scnindent}
					\scnitem{обозначение внутренней временной сущности в sc-памяти}
					\begin{scnindent}
						\begin{scnsubdividing}
							\scnitem{обозначение ситуации в sc-памяти}
							\begin{scnindent}
								\scnidtf{обозначение ситуации, которая возникла или возникает в процессе обработки информации в sc-памяти}
							\end{scnindent}
							\scnitem{обозначение события в sc-памяти}
							\begin{scnindent}
								\scnidtf{обозначение события, которое произошло или произойдет в процессе обработки информации в sc-памяти}
							\end{scnindent}
							\scnitem{обозначение информационного процесса в sc-памяти}
							\begin{scnindent}
								\scnidtf{обозначение внутреннего процесса в sc-памяти, который происходит, произошёл или будет происходить}
							\end{scnindent}
						\end{scnsubdividing}
					\end{scnindent}
				\end{scnsubdividing}
			\end{scnindent}
		\end{scneqtoset}
	\end{scnindent} 
	
	\scnrelfrom{разбиение}{Признак статичности сущностей, обозначаемых sc-элементами}
	\begin{scnindent}
		\begin{scneqtoset}
			\scnitem{обозначение статической сущности}
			\begin{scnindent}
				\scnidtf{обозначение статичной сущности}
				\scnidtf{обозначение стационарной сущности}
				\scnidtf{обозначение сущности, изменения которой в рамках соответствующего отрезка времени считаются несущественными}
				\scnsuperset{обозначение статического sc-множества}
			\end{scnindent}
			\scnitem{обозначение динамической сущности}
			\begin{scnindent}
				\scnidtf{обозначение сущности изменяющейся во времени}
				\scnsuperset{обозначение динамического sc-множетсва}
			\end{scnindent}
		\end{scneqtoset}
	\end{scnindent} 
	
	\scnendstruct \scnsourcecommentinline{Завершили представление \textit{Классификации sc-элементов по темпоральным характеристикам обозначаемых ими сущностей}}
\end{SCn}

Когда речь идёт о темпоральных свойствах sc-элементов, следует чётко отличать:
\begin{textitemize}
	\item временный характер присутствия любого sc-элемента в составе той базы знаний (в той sc-памяти) ostis-системы, в которой он находится (когда-то он появляется, когда-то может быть удалён из sc-памяти;
	\item временный характер присутствия в sc-памяти всей заданной sc-конструкции (заданного множества sc-элементов) –-- такую sc-конструкцию будем называть ситуацией в sc-памяти;
	\item временный характер существования внешней сущности, которую sc-элемент обозначает;
	\item статичный или динамичный (изменчивый) характер внешней сущности, обозначаемой sc-элементом динамический характер внешней сущности, предполагает наличие в sc-памяти описания процесса изменения состояния или конфигурации указанной внешней сущности;
	\item динамическое sc-множество (динамическую sc-конструкцию), являющееся отражением (динамической модели) соответствующего внешнего процесса (процесса, происходящего во внешней среде);
	\item динамическое sc-множество (динамическую sc-конструкцию), являющуюся отражением (динамической моделью) соответствующего внутреннего процесса (информационного процесса, происходящего в той же sc-памяти, в которой находится sc-элемент, обозначающий указанное динамическое sc-множество)
\end{textitemize}

\begin{SCn}
	\scnheader{Структурная классификация sc-констант}
	\scnnote{Данная классификация полностью аналогична Структурной классификации sc-элементов, в отличие от которой она описывает структурную классификацию только константных sc-элементов (sc-констант)}
	\scniselement{sc-структура}
	\scnrelboth{аналог}{Структурная классификация sc-элементов}
\end{SCn}
	
\begin{SCn}
	\scnheader{Структурная классификация sc-констант}
	\scnstartstruct
	
	\scnheader{sc-константа}
	\begin{scnsubdividing}
		\scnitem{sc-множество}
		\begin{scnindent}
			\begin{scnsubdividing}
				\scnitem{sc-связка}
				\begin{scnindent}
					\begin{scnsubdividing}
						\scnitem{sc-синглетон}
						\scnitem{sc-пара}
						\begin{scnindent}
							\begin{scnsubdividing}
								\scnitem{неориентированная sc-пара}
								\scnitem{ориентированная sc-пара}
								\begin{scnindent}
									\begin{scnsubdividing}
										\scnitem{sc-пара принадлежности}
										\begin{scnindent}
											\begin{scnsubdividing}
												\scnitem{sc-пара нечёткой принадлежности}
												\scnitem{sc-пара позитивной принадлежности}
												\begin{scnindent}
												\scnsuperset{sc-пара постоянной позитивной принадлежности}
												\begin{scnindent}
													\begin{scnreltoset}{пересечение множеств}
														\scnitem{sc-константа}
														\scnitem{постоянная сущность}
														\scnitem{статическая сущность}
														\scnitem{sc-пара позитивной принадлежности}	
														\begin{scnindent}
															\scnsuperset{sc-пара постоянной позитивной принадлежности}
															\begin{scnindent}
																\begin{scnreltoset}{пересечение множеств}
																	\scnitem{sc-константа}
																	\scnitem{постоянная сущность}
																	\scnitem{статическая сущность}
																	\scnitem{sc-пара позитивной принадлежности}	
																\end{scnreltoset}
															\end{scnindent}
															\scnsuperset{sc-пара временной позитивной принадлежности}
															\begin{scnindent}
																\begin{scnreltoset}{пересечение множеств}
																	\scnitem{sc-константа}
																	\scnitem{временная сущность}
																	\scnitem{динамическая сущность}
																	\scnitem{sc-пара позитивной принадлежности}	
																\end{scnreltoset}
															\end{scnindent}
														\end{scnindent}
													\end{scnreltoset}
												\end{scnindent}
												\scnsuperset{sc-пара временной позитивной принадлежности}
												\begin{scnindent}
													\begin{scnreltoset}{пересечение множеств}
														\scnitem{sc-константа}
														\scnitem{временная сущность}
														\scnitem{динамическая сущность}
														\scnitem{sc-пара позитивной принадлежности}	
													\end{scnreltoset}
												\end{scnindent}
												\end{scnindent}
												\scnitem{sc-пара негативной принадлежности}
											\end{scnsubdividing}
										\end{scnindent}
										\scnitem{ориентированнная sc-пара, не являющаяся sc-парой принадлежности}	
									\end{scnsubdividing}
								\end{scnindent}
							\end{scnsubdividing}
						\end{scnindent}
						\scnitem{sc-связка, не являющаяся синглетоном и парой}
					\end{scnsubdividing}
				\end{scnindent}
				\scnitem{sc-класс}
				\scnitem{sc-структура}
			\end{scnsubdividing}
		\end{scnindent}
		\scnitem{внешняя сущность}
		\begin{scnindent}
			\scnidtf{sc-элемент, являющийся знаком внешней сущности}
			\scnidtf{знак внешней сущности}
			\scnidtf{знак сущности, не являющейся sc-множеством (sc-конструкцией)}
		\end{scnindent} 
		\begin{scnindent}
			\begin{scnsubdividing}
				\scnitem{файл}
				\scnitem{внутренний файл}
				\scnitem{внешняя сущность, не являющаяся внутренним файлом}
				\scnitem{внешняя сущность, не являющаяся информационной конструкцией}
				\scnitem{информационная конструкция, не являющаяся ни sc-множеством, ни файлом}
			\end{scnsubdividing}
		\end{scnindent}
	\end{scnsubdividing}
	
	\scnendstruct\scnsourcecommentinline{Завершили представление \textit{Структурной классификация sc-констант}}
\end{SCn}

\newpage
\subsection{Уточнение смысла выделенных классов sc-элементов и связей между этими классами}
Перейдём к детальному рассмотрению смысла классов \textit{sc-элементов} (sc-классов), введенных в представленных выше классификациях.

Указанные sc-классы рассматриваются в порядке их введения в представленных выше классификациях \textit{sc-элементов}.

Сначала поясним смысл понятий (sc-классов), введенных в Структурной классификации \textit{sc-элементов} и в Структурной классификации sc-констант.

\begin{SCn}
	\scnheader{sc-элемент}
	\scnidtf{обозначение множества}
	\scnidtf{sc-обозначение множества, представимого в SC-коде}
	\begin{scnsubdividing}
		\scnitem{sc-множество}
		\begin{scnindent}
			\scnidtf{обозначение множества \textit{sc-элементов}}
			\scnidtf{обозначение множества, все элементы которого являются \textit{sc-элементами}}
			\scnidtf{обозначение внутренней для sc-памяти сущности, то есть сущности, хранимой в sc-памяти}
		\end{scnindent}	
		\scnitem{обозначение внешней сущности}
		\begin{scnindent}
			\scnidtf{обозначение синглетона внешней сущности}
			\scnidtf{терминальный \textit{sc-элемент}}
		\end{scnindent}	
	\end{scnsubdividing}
	\begin{scnrelfromlist}{примечание}
			\scnfileitem{Каждый \textit{sc-элемент} является обозначением соответствующего множества.}
			\scnfileitem{Ко множествам, представимым в SC-коде, относятся либо множества, элементами которых являются \textit{sc-элементы} (sc-множества), либо синглетоны, элементами которых являются сущности, не являющиеся \textit{sc-элементами} (синглетоны внешних сущностей). Таким образом, строго говоря, не каждое множество может быть обозначено соответствующим \textit{sc-элементом}. Но каждое множество, не являющееся sc-множеством или синглетоном указанного выше вида может быть однозначно преобразовано в sc-множество и описано средствами \textit{SC-кода}. При этом теоретико-множественные свойства "нестандартных" для \textit{SC-кода} множеств совпадают со свойствами соответствующих им "стандартных" для SC-кода множеств.}
			\scnfileitem{Тот факт, что \underline{каждый} \textit{sc-элемент} является обозначением соответствующего множества (частным случае которых являются синглетоны \underline{внешних} описываемых сущностей), означает то, что базовым видом объектов, которыми оперирует \textit{SC-код} на синтаксической, семантическом и логическом уровне являются множества знаков, обозначающих различные множества. В этом смысле \textit{SC-код} имеет базовую теоретико-множественную основу.}
	\end{scnrelfromlist}

	\scnrelfrom{правила построения внешних идентификаторов sc-элементов заданного класса}{Общие правила построения внешних идентификаторов sc-элементов}
	\begin{scnindent}
		\scnidtf{Общие правила идентификации sc-элементов}
		\begin{scneqtoset}
			\scnfileitem{Принадлежность идентифицируемого \textit{sc-элемента} некоторым классам \textit{sc-элементов} (sc-классам) явно указывается во внешнем идентификаторе этого \textit{sc-элемента} (в SC-идентификаторе) с помощью соответствующих условных признаков:
			\begin{scnitemize}
				\item если первым символом sc-идентификатора является знак подчеркивания, то идентифицируемый \textit{sc-элемент} принадлежит Классу sc-переменных. По умолчанию считается, что идентифицируемый \textit{sc-элемент} принадлежит Классу sc-констант;
				\item если последним символом sc-идентификатора является символ ''звёздочка'', то идентифицируемый \textit{sc-элемент} принадлежит Классу обозначений неролевых отношений;
				\item если последним символом sc-идентификатора является апостроф, то идентифицируемый \textit{sc-элемент} принадлежит Классу обозначений ролевых отношений, каждое из которых является подмножеством Отношения принадлежности, то есть Класса всех константных позитивных пар принадлежности;
				\item если последним символом sc-идентификатора является символ ''\textasciicircum'', то идентифицируемый \textit{sc-элемент} принадлежит Классу обозначений параметров.
			\end{scnitemize}} 
			\scnfileitem{Слово ''обозначение'' в sc-идентификаторе означает то, что обозначаемая сущность может быть как константной, так и переменной.}
			\scnfileitem{В sc-идентификаторах можно делать следующие сокращения:
			\begin{scnitemize}
				\item ''sc-элемент, обозначающий \ldots '' --- ''обозначение''
				\item ''обозначение константного'' --- ''знак константного''
				\item ''знак константного'' --- ''константный''
				\item слово "константный" в sc-идентификаторах можно опускать, так как константность подразумевается по умолчанию
			\end{scnitemize}}
			\scnfileitem{Для каждого sc-элемента можно построить sc-идентификатор, являющийся именем собственным, которое всегда начинается с большой буквы (заглавной) буквы.}
			\scnfileitem{Если sc-элемент является обозначением некоторого класса sc-элементов, то этому sc-элементу можно поставить в соответствие не только имя собственное, но и имя нарицательное, которое начинается маленькой (строчной) буквы. Приведём пример семейства эквивалентных (синонимичных) sc-идентификатор, среди которых есть как имена собственные, так и имена нарицательные:
			\begin{scnitemize}
				\item Множество всевозможных обозначений sc-множеств				\item Класс обозначений sc-множеств
				\item обозначение sc-множества
			\end{scnitemize}}
		\end{scneqtoset}
	\end{scnindent} 
\end{SCn}

\begin{SCn}
	\scnheader{обозначение sc-множества}
	\scnidtf{sc-элемент, являющийся знаком множества всевозможных обозначений sc-множеств}
	\begin{scnindent}
		\scniselement{имя собственное}
	\end{scnindent} 
	\scnidtf{Знак множества всевозможных обозначений sc-множеств}
	\begin{scnindent}
		\scniselement{имя собственное}
	\end{scnindent} 
	\scnidtf{Множество всевозможных обозначений sc-множеств}
	\begin{scnindent}
		\scniselement{имя собственное}
	\end{scnindent} 
	\scnidtf{Класс обозначений sc-множеств}
	\begin{scnindent}
		\scniselement{имя собственное}
	\end{scnindent} 
	\scnidtf{sc-элемент, являющийся обозначением множества sc-элементов}
	\begin{scnindent}
		\scniselement{имя нарицательное}
	\end{scnindent} 
	\scnidtf{sc-обозначение множества sc-элементов}
	\begin{scnindent}
		\scniselement{имя нарицательное}
	\end{scnindent}
	\scnidtf{обозначение множества, каждый элемент которого является sc-элементом}
	\scnidtf{обозначение информационной конструкции, принадлежащей SC-коду}
	\scnidtftext{часто используемый sc-идентификатор}{обозначение sc-конструкции}
	\begin{scnsubdividing}
		\scnitem{sc-множество}
		\begin{scnindent}
			\scnidtf{знак константного sc-множества}
			\scneq{\textup{(}обозначение sc-множества $ \bigcap $ sc-константа\textup{)}}
		\end{scnindent} 
		\scnitem{переменное sc-множество}
		\begin{scnindent}
			\scneq{\textup{(}обозначение sc-множества $ \bigcap $ sc-переменная\textup{)}}
		\end{scnindent}
	\end{scnsubdividing}
\end{SCn}

\begin{SCn}
	\scnheader{следует отличать*}
	\begin{scnhaselementset}
		\scnitem{обозначение sc-множества}
		\begin{scnindent}
			\scnidtf{обозначение sc-множества, которое может быть как константным sc-множеством, так и переменным sc-множеством}
			\scnidtf{обозначение внутренней для sc-памяти сущности}
			\scnidtf{обозначение внутренней sc-памяти информационной конструкции (sc-конструкции)}
			\begin{scnsubdividing}
			\scnitem{sc-множество}
			\begin{scnindent}
				\scnidtf{обозначение конкретного множества}
				\scnidtf{знак множества}
				\scneq{\textup{(}sc-константа $ \bigcap $ обозначение sc-множества\textup{)}}
				\scnidtf{конкретное sc-множество}
				\scnidtf{знак константного sc-множества}
				\scnidtf{константное sc-множество}
			\end{scnindent}
			\scnitem{переменное sc-множество}
			\begin{scnindent}
				\scnidtf{произвольное sc-множества}
				\scnidtf{обозначение произвольного sc-множества}
				\scneq{\textup{(}sc-переменная $ \bigcap $ обозначение sc-множества\textup{)}}
			\end{scnindent} 
			\end{scnsubdividing}
		\end{scnindent}
		\scnitem{sc-множество}
		\scnitem{переменное sc-множество}
		\scnitem{обозначение внешней сущности}
		\begin{scnindent}
			\scnidtf{обозначение сущности, не являющейся множеством sc-элементов (sc-множеством)}
			\scnsuperset{обозначение файла}
			\begin{scnindent}
				\scnidtf{обозначение файла, хранимого либо в файловой памяти той же ostis-системы, в sc-памяти которой хранится знак этого файла, либо в файловой памяти другой дополнительно указываемой компьютерной системы}
			\end{scnindent} 
			\scnsuperset{обозначение информационной конструкции, не являющейся ни sc-множеством, ни файлом}
			\begin{scnindent}
				\scnnote{Примером такой информационной конструкции является напечатанный текст, речевое сообщение, которой следует отличать от его записи в виде аудио-файла \ldots}
			\end{scnindent}
			\scnsuperset{обозначение внешней сущности, не являющейся информационной конструкцией}
			\begin{scnindent}
				\scnnote{Примером такой внешней сущности \ldots}
			\end{scnindent}
		\end{scnindent} 
	\end{scnhaselementset}
\end{SCn}

\begin{SCn}
	\scnheader{sc-множество}
	\scnidtf{sc-конструкция} 
	\scnidtf{информационная конструкция, принадлежащая SC-коду} 
	\scnidtftext{часто используемый sc-идентификатор}{SC-код} 
	\begin{scnindent}
		\scniselement{имя собственное}
	\end{scnindent} 
	\scnidtf{Множество всевозможных sc-конструкций}
	
	\scnheader{обозначение sc-связки}
	\begin{scnsubdividing}
		\scnitem{sc-связка}
		\scnitem{переменная sc-связка}
	\end{scnsubdividing}
\end{SCn}

\begin{SCn}
	\scnheader{sc-связка}
	\scnidtf{знак связи (связки) между sc-элементами} 
	\scnnote{Если элементами sc-связки являются знаки внешних сущностей, то sc-связка является отображением (моделью) некоторой связи, которая связывает указанные внешние сущности}
	\scntext{пояснение}{Понятие sc-связки --- это попытка формализации понятия \underline{целостности}, понятия перехода некоторой совокупности сущности в некоторое новое качество, которое не сводится к свойсвам каждой сущности, входящей в эту совокупность.
		Таким образом, связками следует считать:
		\begin{textitemize}
			\item множество всех чисел, являющихся слагаемыми для заданного числа;
			\item множество всех сотрудников заданной организации, в заданный момент времени;
			\item множество всех сотрудников заданной организации, которые работают в ней;
			\item множество всех точек некоторого отрезка;
			\item множество всех точек некоторого треугольника.
		\end{textitemize}}
	\scntext{примеры}{Примерами sc-связок являются:
		\begin{textitemize}
			\item конкретная окружность, (множество \underline{всех} точек, равноудаленных от некоторой заданной точки);
			\item конкретный отрезок (множество \underline{всех} точек, лежащих между двумя заданными точками с включением этих точек);
			\item конкретный линейный треугольник (множество \underline{всех} точек, лежащих между каждыми двумя из трёх заданных точек с включением этих точек);
			\item пары граничных точек различных отрезков;
			\item тройки вершин различных треугольников.
		\end{textitemize}}
\end{SCn}

\begin{SCn}
	\scnheader{обозначение sc-синглетона}
	\begin{scnsubdividing}
		\scnitem{sc-синглетон}
		\scnitem{переменный sc-синглетон}
	\end{scnsubdividing}
\end{SCn}

\begin{SCn}
	\scnheader{sc-синглетон}
	\scnidtf{sc-множество, являющиеся синглетоном}
	\scnidtf{одномощное sc-множество}
	\scnidtf{sc-множество, имеющее мощность, равную единице}
	\scnidtf{sc-элемент, обозначающий унарную sc-связку}
	\scnidtf{sc-обозначение унарной sc-связки}
	\scnidtf{унарная sc-связка}
	\scnidtf{обозначение sc-синглетона}
	\scnidtf{обозначение одномощного множества, единственный элемент которого является sc-элементом}
\end{SCn}

\begin{SCn}	
	\scnheader{обозначение sc-пары}
	\scniselement{sc-константа}
	\scniselement{sc-класс}
	\begin{scnindent} 
		\begin{scnsubdividing}
			\scnitem{sc-пара}
			\begin{scnindent}
				\scnidtf{константная sc-пара}
				\scnsubset{sc-константа}
				\scniselement{sc-константа}
				\scniselement{sc-класс}
			\end{scnindent} 
			\scnitem{переменная sc-пара}
			\begin{scnindent}
				\scniselement{sc-переменная}
				\scniselement{sc-константа}
				\scniselement{sc-класс}
			\end{scnindent} 
		\end{scnsubdividing}
	\end{scnindent}
\end{SCn}

\begin{SCn}
	\scnheader{sc-пара}
\end{SCn}

\begin{SCn}
	\scnheader{обозначение неориентированной sc-пары}
	\scnidtf{неориентированная sc-пара}
\end{SCn}

\begin{SCn}
	\scnheader{обозначение ориентированной sc-пары}
	\scnidtf{ориентированная sc-пара}
\end{SCn}

\begin{SCn}
	\scnheader{обозначение sc-пары принадлежности}
	\scnidtf{sc-пара принадлежности}
\end{SCn}

\begin{SCn}
	\scnheader{обозначение sc-пары принадлежности}
	\scnidtf{sc-пара принадлежности}
\end{SCn}

\begin{SCn}
	\scnheader{обозначение sc-пары нечёткой принадлежности}
	\scnidtf{sc-пара нечёткой принадлежности}
\end{SCn}

\begin{SCn}
	\scnheader{обозначение sc-пары  позитивной принадлежности}
	\scnidtf{sc-пара позитивной принадлежности}
\end{SCn}

\begin{SCn}
	\scnheader{sc-пара константной постоянной позитивной принадлежности}
	\scnidtf{константная позитивная постоянная sc-пара принадлежности}
\end{SCn}

\begin{SCn}
	\scnheader{sc-пара константной временной позитивной принадлежности}
\end{SCn}

\begin{SCn}
	\scnheader{обозначение sc-пары негативной принадлежности}
	\scnidtf{sc-пара негативной принадлежности}
\end{SCn}

\begin{SCn}
	\scnheader{обозначение sc-пары, не являющейся парой принадлежности}
	\scnidtf{sc-пара, не являющаяся парой принадлежности}
\end{SCn}

\begin{SCn}
	\scnheader{обозначение sc-связки, не являющейся синглетоном и парой}
	\scnidtf{sc-пара, не являющаяся синглетоном и парой}
\end{SCn}

\begin{SCn}
	\scnheader{обозначение sc-класса}
	\scnidtf{sc-класс}
	\begin{scnsubdividing}
		\scnitem{sc-класс}
		\scnitem{переменный sc-класс}
	\end{scnsubdividing}
	\begin{scnsubdividing}
		\scnitem{обозначение sc-класса обозначений sc-связок}
		\scnitem{обозначение sc-класса обозначений sc-классов}
		\scnitem{обозначение sc-класса обозначений sc-структор}
		\scnitem{обозначение sc-классов обозначений внешних сущностей}
	\end{scnsubdividing}
\end{SCn}

\begin{SCn}
	\scnheader{sc-класс}
	\begin{scnsubdividing}
		\scnitem{sc-класс sc-связок}
		\begin{scnindent}
			\scnsuperset{sc-отношение}
		\end{scnindent}
		\scnitem{sc-класс sc-классов}
		\begin{scnindent}
			\scnsuperset{sc-параметр}
		\end{scnindent}
		\scnitem{sc-класс sc-структур}
		\scnitem{sc-класс внешних сущностей}
		\begin{scnindent}
			\scnidtf{sc-класс sc-элементов, являющихся знаками внешних сущностей}
		\end{scnindent} 
		\scnitem{sc-класс sc-элементов разного структурного типа}
		\begin{scnindent}
			\scnhaselementrole{пример}{\scnfilelong{sc-константа}}
		\end{scnindent} 
	\end{scnsubdividing}
\end{SCn}

Перечислим основные основные виды sc-классов

\begin{SCn}
	\scnheader{sc-класс}
	\scnsuperset{sc-отношение}
	\begin{scnindent}
		\scnidtf{sc-класс sc-связок}
		\scnsuperset{бинарное sc-отношение}
		\begin{scnindent} 
			\begin{scnsubdividing}
				\scnitem{бинарное неориентированное sc-отношение}
				\scnitem{бинарное ориентированное sc-отношение}
				\begin{scnindent}
					\scnsuperset{ролевое sc-отношение}
				\end{scnindent} 
			\end{scnsubdividing}
		\end{scnindent}
	\end{scnindent} 
	\scnsuperset{sc-класс sc-классов}
	\begin{scnindent}
		\scnsuperset{sc-параметр}
	\end{scnindent}
	\scnsuperset{sc-класс sc-структур}
	\scnsuperset{sc-класс внешних сущностей}
	\begin{scnindent}
		\scnsuperset{sc-класс внутренних файлов}
	\end{scnindent}
	\scnsuperset{sc-класс эквивалентности}
	\begin{scnindent}
		\scnidtf{фактор-множество соответствующего отношения эквивалентности}
	\end{scnindent} 
	\scntext{пояснение}{Требованиями, предъявляемыми к каждому sc-классу являются:
		\begin{textitemize}
			\item бесконечность этого sc-множества;
			\item наличие общего свойства, присущего \underline{всем} элементам этого sc-множества, в частности, наличие его определения.
		\end{textitemize}
	}
\end{SCn}

\vskip 5cm

\begin{SCn}
	\scnheader{следует отличать*}
	\begin{scnhaselementset}
		\scnitem{sc-связка}
		\scnitem{sc-класс}
	\end{scnhaselementset}
	\begin{scnindent}
		\scntext{сравнение}{В отличие от sc-связки принципом формирования sc-класса является наличие общего свойства, присущего \underline{всем} элементам этого sc-класса \underline{и только им}, (или присущего всем сущностям, которые обозначаются указанными sc-элементами). Таким общим свойством может быть \underline{определение sc-класса} либо принадлежность одному из значений некоторого параметра, то есть одному из элементов фактор-множества, соответствующего некоторому отношению эквивалентности или толерантности.}
		\scntext{пояснение}{Примерами связок являются:
		\begin{textitemize}
			\item множество людей живущих сейчас (динамическое множество);
			\item множество сотрудников некоторой организации (динамическое множество);
			\item отрезок, треугольник.
		\end{textitemize}
		Здесь речь не идёт об эквивалентности свойств самих людей и геометрических точек безотносительно к тому, в состав чего они входят. Поэтому это не является sc-классом.
		}
	\end{scnindent}
	\begin{scnhaselementset}
		\scnitem{sc-класс эквивалентности}
		\begin{scnindent}
			\scnexplanation{В sc-класс входит не просто некоторое количество попарно эквивалентных между собой сущностей, а абсолютно \underline{все} такие сущности}
		\end{scnindent}
		\scnitem{sc-связка попарно эквивалентных сущностей}
	\end{scnhaselementset}
\end{SCn}

Приведём пример:

\begin{SCn}
	\scnheader{следует отличать*}
	\begin{scnhaselementset}
		\scnitem{множество \underline{всех} треугольников, подобных одному из них}
		\begin{scnindent}
			\scnsubset{sc-класс}
		\end{scnindent}
		\scnitem{конечное множество подобных треугольников}
		\begin{scnindent}
			\scnsubset{sc-связка попарно эквивалентных треугольников}
		\end{scnindent}
	\end{scnhaselementset}
	\begin{scnhaselementset}
		\scnitem{параметр}
		\begin{scnindent}
			\scnidtftext{часто используемый sc-идентификатор}{sc-параметр}
			\scnsubset{класс классов}
		\end{scnindent}
		\scnitem{признак различия}
		\begin{scnindent}
			\scnidtf{признак классификации}
		\end{scnindent}
	\end{scnhaselementset}
	\scnrelfrom{пояснение}{
	\scnstartset
	\scnheaderlocal{параметр}
	\scnsubset{бесконечное множество}
	\bigskip
	\scnheaderlocal{признак различия}
	\scnsubset{конечное множество}
	\begin{scnhaselementrolelist}{пример}
		\scnitem{Признак конечности множеств}
		\begin{scnindent}
			\begin{scneqtoset}
				\scnitem{конечное множество}
				\scnitem{бесконечное множество}				
			\end{scneqtoset}
		\end{scnindent}
		\scnitem{Признак наличия кратных элементов}
		\begin{scnindent}
			\begin{scneqtoset}
				\scnitem{мультимножество}
				\scnitem{множество без кратных вхождений элементов}
			\end{scneqtoset}
		\end{scnindent}
	\end{scnhaselementrolelist}
	
	\scnendstruct
}
\end{SCn}

\vskip 5cm

\begin{SCn}
	\scnheader{sc-класс}
	\scnrelfrom{правила построения внешних идентификаторов sc-элементов заданного класса}{Правила построения внешних идентификаторов sc-элементов, являющихся знаками sc-классов}
	\begin{scnindent}
		\begin{scneqtoset}
			\scnfileitem{Слово ''обозначение'' в начале идентификатора используется тогда, когда в идентифицируемый класс sc-элементов включаются знаки как константных, так и переменных сущностей соответствующего вида}
			\scnfileitem{Слово ''переменный'' в начале идентификатора и ...}
		\end{scneqtoset}
	\end{scnindent}
\end{SCn}

\begin{SCn}
	\scnheader{обозначение sc-структуры}
\end{SCn}

\begin{SCn}
	\scnheader{sc-структура}
\end{SCn}

\begin{SCn}
	\scnheader{следует отличать*}
	\begin{scnhaselementset}
		\scnitem{sc-структура}
		\scnitem{sc-связка}
	\end{scnhaselementset}
	\begin{scnindent} 
		\begin{scnrelfromset}{сравнение}
			\scnfileitem{В отличие от sc-связок в каждую sc-структуру должна входить по крайней мере одна sc-связка вместе с компонентами этой sc-связки}
		\end{scnrelfromset}
	\end{scnindent}
\end{SCn}

\begin{SCn}
	\scnheader{обозначение внешняя сущность}
\end{SCn}

\begin{SCn}
	\scnheader{внешняя сущность}
	\scnidtf{синглетон внешней сущности}
	\scnidtf{обозначение синглетона внешней сущности}
	\scnidtf{sc-элемент, обозначающий синглетон, элементом которого является некоторая внешняя описываемая сущность}
	\scnidtf{множество обозначаемой sc-элементом, являющееся 1-мощным множеством, единственным элементом которого является сущность, внешняя по отношению к sc-конструкции, то есть сущность, не являющаяся sc-элементом}
	\begin{scnrelfromlist}{примечание}
		\scnfileitem{обозначение внешней сущности, то есть sc-элемент, обозначающий этот синглетон, можно также трактовать как sc-элемент, обозначающий соответсвтующую внешнюю описываемую сущность, которую в свою очередь можно считать денодом указанного sc-элемента}
		\scnfileitem{очевидно, что пара принадлежности, связывающая sc-элемент, обозначающий синглетон внешней сущности, не может быть непосредственно представлена в виде соответствующей sc-дуги принадлежности, так как второй компонент этой sc-дуги не находится в sc-памяти}
	\end{scnrelfromlist}
\end{SCn}

\begin{SCn}
	\scnheader{следует отличать*}
	\begin{scnhaselementset}
		\scnitem{синглетон внешней сущности}
		\scnitem{sc-синглетон}
		\begin{scnindent}
			\scnidtf{синглетон, единственным элементом которого является некоторый sc-элемент}
			\scnsubset{sc-множество}
			\begin{scnindent}
				\scnidtf{sc-элемент, обозначающий множество, элементами которого являются \underline{только} sc-элементы}
				\scnidtf{множество sc-элементов}
			\end{scnindent} 
		\end{scnindent}
	\end{scnhaselementset}
\end{SCn}

\begin{SCn}
	\scnheader{обозначение файла}
\end{SCn}

\begin{SCn}
	\scnheader{файл}
\end{SCn}

\vskip 5cm

\begin{SCn}
	\scnheader{внутренний файл}
	\scnidtf{внутренний образ (копия), внутренней информационной конструкции, хранимый в файловой памяти ostis-системы}
	\scnidtf{файл ostis-системы}
	\begin{scnrelfromlist}{примечание}
		\scnfileitem{файловая память ostis-системы, хранящая различного рода информационные конструкции (образы, модели), не являющиеся sc-конструкциями, должна быть тесно связана с sc-памятью этой же ostis-системы. Как минимум каждый файл ostis-системы должен быть связан с тем sc-узлом, которых является знаком этого файла (точнее, знаком синглетона, элементом которого является указанный файл)}
	\end{scnrelfromlist}
\end{SCn}

Перейдем к пояснению смысла понятий используемых в \textit{Логической классификации sc-элементов}.

\begin{SCn}
	\scnheader{sc-константа}
	\scnidtf{sc-элемент, обозначающий константную сущность}
	\begin{scnindent}
		\scntext{сокращение}{обозначение константной сущности}
	\end{scnindent}
	\scnidtf{обозначение константной сущности}
	\scnidtf{знак константной сущности}
	\begin{scnindent}
		\scntext{сокращение}{константная сущность}
		\begin{scnindent}
			\scntext{сокращение}{сущность}
		\end{scnindent} 
	\end{scnindent}
	\scnidtf{константная сущность}
	\scnidtf{конкретная сущность}
	\scnidtf{сущность}
	\scnidtf{константный sc-элемент}
	\scnidtf{sc-элемент, имеющий одно логико-семантическое значение, каковым является он сам}
	\scnidtf{sc-элемент, являющийся знаком константной (конкретной, фиксированной) сущности}
	\scntext{сокращение}{знак константной (конкретной, фиксированной) сущности}
		\begin{scnindent} 
			\scntext{сокращение}{константная (конкретная, фиксированная) сущность}
			\begin{scnindent} 
				\scntext{сокращение}{константная сущность}
			\end{scnindent}
		\end{scnindent}
\end{SCn}

\begin{SCn}
	\scnheader{sc-переменная}
	\scnidtf{переменный sc-элемент}
	\scnidtf{sc-элемент, являющийся обозначением некоторой произвольной (нефиксируемой, переменной) сущности}
	\begin{scnindent}
		\scntext{сокращение}{обозначение произвольной (переменной) сущности}
		\begin{scnindent}
			\scntext{сокращение}{переменная сущность}
		\end{scnindent}
	\end{scnindent}
	\scniselement{sc-константа}
	\scniselement{sc-класс}
	\begin{scnrelfromlist}{примечание}
		\scnfileitem{Сам sc-элемент, имеющий внешний идентификатор ''sc-переменная'' является sc-константой (константным sc-элементом), которая является знаком одного из классов sc-элементов}
	\end{scnrelfromlist}
\end{SCn}

\begin{SCn}
	\scnheader{sc-элемент}
	\scnidtf{обозначение константной или переменной сущности}
	\scnidtf{константная или переменная сущность}
	\scnidtf{sc-константа или переменная сущность}
	\scnidtf{обозначение описываемой сущности, которая может быть как константой, так и переменной сущностью, как внутренней, так и внешней sc-конструкцией для заданной ostis-системы, как информационной конструкцией, так и сущностью которая информационной конструкцией не является, как временной сущностью, так и постоянной, как динамической, так и статической сущностью}
\end{SCn}

\vskip 5cm

\begin{SCn}
	\scnheader{обозначение sc-множества}
	\begin{scnsubdividing}
		\scnitem{sc-множество}
		\begin{scnindent}
			\scnidtf{часто используемый sc-идентификатор}
			\begin{scnindent}
				\scntext{сокращение}{множество sc-элементов}
			\end{scnindent} 
			\scnidtf{константное (конкретное) sc-множество}
			\scnidtf{обозначение (знак) конкретного множества}
			\scnsubset{sc-константа}
			\scniselement{sc-константа}
			\begin{scnsubdividing}
				\scnitem{множество констант}
				\begin{scnindent}
					\scnidtf{множество, каждый элемент которого является константой}
					\scnidtf{множество, являющееся подмножеством Множества всевозможных констант}
				\end{scnindent}
				\scnitem{множество переменных}
				\scnitem{множество констант и переменных}
				\begin{scnindent}
					\scnidtf{множество, элементами которого являются как константы, так и переменные}
				\end{scnindent}
				\scnitem{sc-множество sc-констант}
				\begin{scnindent}
					\scnidtf{sc-множество, элементами которого являются только sc-константы}
				\end{scnindent} 
				\scnitem{sc-множество sc-переменных}
				\begin{scnindent}
					\scnidtf{sc-множество, элементами которого являются только sc-переменные}
				\end{scnindent} 
				\scnitem{sc-множество sc-констант и sc-переменных}
				\begin{scnindent}
					\scnidtf{произвольное множество}
					\scnidtf{обозначение переменного (произвольного) sc-множества)}
					\scnsubset{sc-переменная}
					\scniselement{sc-константа}
					\scnrelboth{следует отличать}{sc-множество sc-переменных}
				\end{scnindent} 
			\end{scnsubdividing}
		\end{scnindent}
		\scnitem{переменное sc-множество}
		\begin{scnindent}
			\scnidtf{обозначение переменного (произвольного) sc-множества}
		\end{scnindent}
	\end{scnsubdividing}
\end{SCn}

Перейдем к пояснению смысла понятий, используемых в \textit{Классификации sc-элементов по темпоральным характеристикам обозначаемых ими сущностей}.

\begin{SCn}
	\scnheader{обозначение временной сущности}
	\begin{scnsubdividing}
		\scnitem{обозначение временной сущности существующей сейчас}
		\begin{scnindent}
			\scnidtf{обозначение временной сущности, существующей в текущий (настоящий) момент}
		\end{scnindent} 
		\scnitem{обозначение прошлой временной сущности}
		\begin{scnindent}
			\scnidtf{обозначение бывшей временной сущности}
			\scnidtf{обозначение временной сущности, которая уже перестала существовать, прекратила своё существование}
		\end{scnindent} 
		\scnitem{обозначение будущей временной сущности}
		\begin{scnindent}
			\scnidtf{обозначение временной сущности, появление которой прогнозируется (планируется, обеспечивается)}
			\scnnote{проектирование и производство новых, ранее не существующих полезных сущностей -- это основное направление человеческой деятельности}
		\end{scnindent} 
	\end{scnsubdividing}
	\begin{scnindent}
		\scnnote{ostis-системы должны постоянно мониторить состояние временных сущностей, так как в процессе их функционирования будущие сущности становятся настоящими, а настоящие -- прошлыми}
\end{scnindent} 
\end{SCn}

\vskip 5cm

\begin{SCn}
	\scnheader{динамическое sc-множество}
	\scnidtf{sc-процесс}
	\scnidtf{процесс}
	\scntext{определение}{sc-множество, у которого некоторые позитивные пары принадлежности, связывающие знак этого множества с его элементами, носят временный характер}
	\scnnote{Сами элементы динамического sc-множества, связанные с ним временными позитивными парами принадлежности, могут обозначать как временные, так и постоянные сущности. Но чаще всего такими временными элементами динамического sc-множества являются знаки временных связок.}
	\begin{scnsubdividing}
		\scnitem{внешний процесс}
		\scnitem{процесс в sc-памяти}
	\end{scnsubdividing}
\end{SCn}

\begin{SCn}
	\scnheader{темпоральная декомпозиция динамического sc-множества}
	\scnidtf{покадровая развертка динамического sc-множества}
	\scnidtf{представление sc-множества в виде кортежа (последовательности) ситуаций}
\end{SCn}

\begin{SCn}
	\scnheader{следует отличать*}
	\begin{scnhaselementset}
		\scnitem{временная сущность}
		\scnitem{обозначение временной сущности}
		\scnitem{переменная временная сущность}
	\end{scnhaselementset}
\end{SCn}

\begin{SCn}
	\scnheader{обозначение временной сущности}
	\begin{scnsubdividing}
		\scnitem{временная сущность}
		\begin{scnindent}
			\scnidtf{знак конкретной (константной) временной сущности}
		\end{scnindent} 
		\scnitem{переменная временная сущность}
		\begin{scnindent}
			\scnidtf{обозначение произвольной временной сущности}
		\end{scnindent} 
	\end{scnsubdividing}
\end{SCn}

\begin{SCn}
	\scnheader{сформированное sc-множество}
	\scnidtf{sc-множество, у которого в текущем состоянии sc-памяти перечислены все его элементы}
	\scniselement{динамическое sc-множество}
	\scnnote{Очевидно, что сформированным sc-множеством может стать только конечное sc-множество}
\end{SCn}

\begin{SCn}
	\scnheader{формируемое sc-множество}
\end{SCn}

\begin{SCn}
	\scnheader{sc-множество, элементы которого не известны}
\end{SCn}

\begin{SCn}
	\scnheader{сформированный файл}
\end{SCn}

\begin{SCn}
	\scnheader{формируемый файл}
\end{SCn}

\begin{SCn}
	\scnheader{файл, структура которого не известна}
\end{SCn}

Перейдем к рассмотрению семантически выделяемых классов sc-элементов, которые необходимо ввести \underline{дополнительно} к выше рассмотренным классам sc-элементов.

\begin{SCn}
	\scnheader{sc-элемент, не являющийся sc-синглетоном и sc-парой}
\end{SCn}

\begin{SCn}
	\scnheader{sc-элемент, копируемый в других компьютерных системах}
	\scnidtf{sc-элемент, имеющий в других компьютерных системах свои копии и/или копии обозначаемой им информационной конструкции}
\end{SCn}

\begin{SCn}
	\scnheader{отношение, заданное на множестве sc-элементов, копируемых в других компьютерных системах*}
	\scnhaselement{}
	\scnhaselement{}
	\scnhaselement{}
\end{SCn}

\begin{SCn}
	\scnheader{отношение, заданное на множестве sc-элементов, имеющих копии в других компьютерных системах*}
	\scnhaselement{ostis-система, в sc-памяти которой хранится копия заданного sc-элемента*}
	\scnhaselement{компьютерная система, в файловой памяти которой хранится заданный файл*}
	\begin{scnindent}
		\scnnote{указанная компьютерная система назначается хранителем файла}
	\end{scnindent}
	\scnhaselement{ostis-система, в sc-памяти которой хранится копия знака заданного sc-множества и все известные в текущий момент его элементы*}
	\begin{scnindent}
		\scnnote{указанная ostis-система назначается основным хранителем указанного sc-множества}
	\end{scnindent}
\end{SCn}

\begin{SCn}
	\scnheader{информационная конструкция}
	\begin{scnsubdividing}
		\scnitem{sc-множество}
		\begin{scnindent}
			\scnidtf{sc-конструкция}
			\scnidtf{информационная конструкция SC-кода}
			\scnidtf{внутренняя информационная конструкция ostis-системы, хранимая в её sc-памяти}
		\end{scnindent}
		\scnitem{файл}
		\begin{scnindent}
			\scnidtf{файл ostis-системы}
			\scnidtf{внутреняя информационная конструкция ostis-системы, хранимая в её файловой памяти}
			\scnnote{файл, может храниться в памяти другой компьютерной системы и, в частности, в файловой памяти другой ostis-системы}
		\end{scnindent}
		\scnitem{внешняя информационная конструкция, не являющаяся ни файлом, ни sc-конструкцией}
	\end{scnsubdividing}
\end{SCn}

\begin{SCn}
	\scnheader{sc-идентификатор}
	\scnidtf{внешний идентификатор sc-элемента}
	\scnsuperset{файл}
	\begin{scnsubdividing}
		\scnitem{основной идентификатор}
		\scnitem{часто используемый sc-идентификатор}
		\scnitem{дополнительный sc-идентификатор}
	\end{scnsubdividing}
\end{SCn}

\begin{SCn}
	\scnheader{sc-идентификатор*}
	\scnidtf{Бинарное ориентированное отношение, связывающее sc-элементы с их внешними идентификаторами}
\end{SCn}


\newpage
\subsection{Структура базовой семантической спецификации sc-элемента}
\begin{SCn}
	\scnheader{базовая семантическая спецификация sc-элемента}
	\scnidtfexp{Класс sc-структур, каждая из которых описывает базовые семантические свойства (характеристики) соответствующего (описываемого, специфицируемого) sc-элемента}
	\scnsubset{sc-структура}
	\begin{scnindent}
		\scnsubset{sc-спецификация}
		\begin{scnindent}
			\scnidtf{представленная в SC-коде семантическая окрестность (спецификация) некоторого (специфицируемого) sc-элемента}
		\end{scnindent}
	\end{scnindent}
	\scnsubset{sc-спецификация}
	\scnrelto{второй домен}{базовая семантическая спецификация sc-элемента*}
	\begin{scnindent}
		\scnidtfexp{Бинарное ориентированное отношение, каждая пара которого связывает sc-элемент с его базовой семантической спецификацией*}
	\end{scnindent}
	\scnidtfexp{хранимая в sc-памяти ostis-система спецификация каждого sc-элемента, необходимая для эффективной обработки этого sc-элемента}
	\scnnote{базовая спецификация sc-элементов осуществляется как явно с помощью соответствующих sc-конструкций, так и неявно с помощью соответствующих семантических меток, приписываемых sc-элементам}
	\scntext{пояснение}{Базовая семантическая спецификация каждого sc-элемента включает в себя:
		\begin{scnitemize}
			\item перечисление всех тех \textit{базовых классов sc-элементов}, которым принадлежит специфицируемый sc-элемент;
			\item уточнение ''привязки'' рассматриваемых временных сущностей к текущему моменту времени и другим моментам времени;
			\item уточнение того, какие важные характеристики специфицируемого sc-элемента в текущем состоянии sc-памяти и файловой памяти ostis-системы не известны.
		\end{scnitemize}
	}
\end{SCn}

\textbf{Базовая семантическая спецификация sc-элемента, обозначающего временную сущность} включает в себя указание дополнительных темпоральных характеристик, позволяющих уточнить темпоральные ''координаты'' этих временных сущностей (то есть их ''координаты'' во времени), а также их основные темпоральные связи с другими временными сущностями. К числу понятий, обеспечивающих описание указанных темпоральных характеристик временных сущностей, относятся:
\begin{textitemize}
	\item момент времени\textasciicircum
	\item Текущий момент времени
	\item прошлая сущность
	\item будущая сущность 
	\item момент начала*
	\item момент завершения*
	\item внешняя ситуация
	\item ситуация в sc-памяти
	\item внешнее событие
	\item событие в sc-памяти
	\item внешний процесс
	\item процесс в sc-памяти
\end{textitemize}

\vskip 5cm

\begin{SCn}
	\scnheader{момент времени\textasciicircum}
	\scniselement{параметр}
	\scniselement{параметр, заданный на множестве временных сущностей}
	\scnidtf{глобальная приблизительно точная ситуация\textasciicircum}
	\scnidtf{глобальная ситуация пренебрежительно малого отрезка времени\textasciicircum}
	\scnidtf{множество (класс) \underline{всех} временных сущностей, существующих одновременно в соответствующий момент времени\textasciicircum}
	\scnnote{момент времени, соответствующий глобальной точечной ситуации может быть задан с различной и \underline{дополнительно указываемой} степенью точности -- с точностью до секунды, до минуты, до часа, до даты, до календарного месяца, до календарного года и так далее. В том смысле корректнее говорить не о моменте времени, а об интервале времени, длительность которого считается пренебрежимо малой для рассмотрения описываемых процессов}
\end{SCn}

\begin{SCn}
	\scnheader{Текущий момент времени}
	\scnidtf{Глобальная ситуация текущего (настоящего) момента времени}
	\scnidtf{Глобальная ситуация, имеющая место сейчас}
	\scnidtf{Класс всех сущностей, существующих в настоящий момент времени}
	\scniselement{sc-синглетон}
	\scniselement{динамическое sc-множество}
	\scnrelto{включение множества}{момент времени}
	\scnexplanation{Из знака \textit{текущего момента времени} (который является также знаком \textit{sc-синглетона}) ''выходит'' sc-дуга (sc-пара) \underline{временной} принадлежности, представляющая собой, образно говоря, ''стрелку'' внутренних часов ostis-системы, которая всегда указывает только на один элемент множества моментов времени, но в разные моменты времени указывает на разные элементы этого множества}
\end{SCn}

\begin{SCn}
	\scnheader{прошлая сущность}
	\scnidtf{временная сущность, уже завершившая своё существование}
\end{SCn}

\begin{SCn}
	\scnheader{будущая сущность}
	\scnidtf{прогнозируемая, планируемая или создаваемая временная сущность}
\end{SCn}

\begin{SCn}
	\scnheader{момент начала*}
	\scnidtf{момент времени, соответствующий началу существования заданной временной сущности}
	\scnidtf{Бинарное ориентированное отношение, каждая пара которого, связывает (1) знак некоторой временной сущности и (2) глобальную точечную ситуацию (значение параметра ''\textit{момент времени}\textasciicircum''), элементом которой является условно точечная временная сущность, представляющая собой начальный этап существования временной сущности, указанной в первом компоненте рассматриваемой ориентированной пары}
	\scnnote{Начальный этап существования временной сущности (переходный процесс от небытия к реальному существованию) может рассматриваться с любой степенью детализации}
	\scnrelfrom{первый домен}{временная сущность}
	\scnrelfrom{второй домен}{момент времени\textasciicircum}
\end{SCn}

\begin{SCn}
	\scnheader{момент завершения*}
	\scnidtf{момент времени, соответствующий завершению существования заданной временной сущности}
\end{SCn}

\begin{SCn}
	\scnheader{ситуация}
	\begin{scnsubdividing}
		\scnitem{внешняя ситуация}
		\scnitem{ситуация в sc-памяти}
	\end{scnsubdividing}
\end{SCn}

\vskip 5cm

\begin{SCn}
	\scnheader{событие}
	\begin{scnsubdividing}
		\scnitem{внешнее событие}
		\scnitem{событие в sc-памяти}
	\end{scnsubdividing}
\end{SCn}

\begin{SCn}
	\scnheader{динамическое sc-множество}
	\begin{scnsubdividing}
		\scnitem{внешний процесс}
		\begin{scnindent}
			\scnidtf{процесс, происходящий в окружающей среде ostis-системы}
		\end{scnindent}
		\scnitem{процесс в sc-памяти}
	\end{scnsubdividing}
\end{SCn}

\begin{SCn}
	\scnheader{внешняя ситуация}
	\scnidtf{ситуация во внешней среде}
	\scnidtf{ситуация \underline{одновременного} существования (в соответствующий период времени) указанных временных внешних сущностей}
	\scnsubset{временная сущность}
	\scnsubset{sc-структура}
	\scnsubset{sc-константа}
	\scnsubset{обозначение внешней ситуации}
	\scniselement{sc-класс}
	\end{SCn}

\begin{SCn}
	\scnheader{класс внешних ситуаций}
\end{SCn}

\begin{SCn}
	\scnheader{обобщенное описание}
\end{SCn}

\begin{SCn}
	\scnheader{класс внешних ситуаций}
	\scnnote{В простейшем случае внешние ситуации, входящие в класс внешних ситуаций являются изоморфными}
\end{SCn}

\begin{SCn}
	\scnheader{внешний процесс}
	\scnidtf{темпоральная детализация внешней динамической сущности}
\end{SCn}

\begin{SCn}
	\scnheader{внешнее событие}
	\scnidtf{факт появления (возникновения) некоторой внешней сущности (в том числе некоторой внешней ситуации) или факт завершения существования некоторой внешней сущности (в том числе некоторой внешней ситуации)}
\end{SCn}

\begin{SCn}
	\scnheader{ситуация в sc-памяти}
	\scnidtf{внутрення ситуация}
	\begin{scnindent}
		\scnidtf{sc-ситуация}
		\scnidtf{хранимый в sc-памяти фрагмент базы знаний, рассматриваемый в контексте его появления в sc-памяти или его исчезновения (из-за удаления некоторые sc-элементов)}
	\end{scnindent}
\end{SCn}

\begin{SCn}
	\scnheader{класс ситуаций в sc-памяти}
	\scnidtf{класс внутренних ситуаций}
\end{SCn}

\begin{SCn}
	\scnheader{обобщённое описание класса ситуаций в sc-памяти}
\end{SCn}

\begin{SCn}
	\scnheader{процесс в sc-памяти}
	\scnidtf{внутренний процесс}
	\scnidtf{информационный процесс, происходящий в sc-памяти}
	\scnidtf{sc-процесс}	
\end{SCn}

\begin{SCn}
	\scnheader{событие в sc-памяти}
\end{SCn}

Важной частью \textit{базовой семантической спецификации sc-элемента} является фиксация того, что ostis-система знает и чего она не знает о специфицируемом sc-элементе или об обозначенной им сущности:

\begin{textitemize}
	\item Если в спецификации sc-элемента указывается его принадлежность к некоторому классу sc-элементов, но не указывается его принадлежность \underline{одному} из подклассов, на которые \textit{разбивается} указанный выше класс, то это означает, что в текущий момент времени ostis-система этого \underline{не знает};
	\item Если специфицируемый sc-элемент является обозначением \underline{конечного} множества sc-элементов (в частности, пары sc-элементов), и если в текущий момент времени ostis-системе не известны \underline{все} этого множества (то есть специфицируемый sc-элемент не соединён соответствующими парами принадлежности со \underline{всеми} элементами обозначаемого им множества sc-элементов), то этот специфицируемый sc-элемент следует отнести к sc-классу ''\textbf{\textit{обозначение несформированного sc-множества}}'';
	\item Если специфицируемый sc-элемент является обозначением ориентированной sc-пары и если в текущий момент времени ostis-системе не известна \underline{направленность} этой ориентированной пары sc-элементов (то есть не известно, какой элемент этой пары является первым её компонентом, а какой её элемент является её вторым компонентом), то этот специфицируемый sc-элемент следует отнести к sc-классу ''\textbf{\textit{обозначение ориентированной sc-пары неизвестной направленности}}''.
\end{textitemize}

К числу понятий, используемых для описания полноты базовой спецификации sc-элементов, относятся:

\begin{textitemize}
	\item \textit{обозначение бесконечного sc-множества};
	\item \textit{обозначение конечного sc-множества};
	\item \textit{мощность обозначаемого sc-множества*};
	\item \textit{обозначение sc-множества неизвестной мощности};
	\item \textit{обозначение sc-множества, о котором не известно, является ли оно sc-парой};
	\item \textit{обозначение sc-пары, о которой не известно, является ли она ориентированной или нет};
	\item \textit{обозначение ориентированной sc-пары неизвестной направленности};
	\item \textit{обозначение сформированного sc-множества};
	\item \textit{обозначение \underline{частично} сформированного sc-множества};
	\item \textit{обозначение \underline{полностью} несформированного sc-множества};
	\item \textit{обозначение сформированного файла};
	\item \textit{обозначение частично сформированного файла};
	\item \textit{обозначение полностью несформированного файла};
\end{textitemize}

Подчеркнём то, что базовую семантическую спецификацию должны иметь абсолютно все \textit{sc-элементы}, хранимые в \textit{sc-памяти} в текущий момент времени, в том числе и все \textit{sc-элементы}, являющиеся ключевыми знаками в рамках \textit{Предметной области Базовой денотационной семантики SC-кода}. Приведём пример базовой семантической sc-спецификации одного из таких sc-элементов:
 
\begin{SCn}
	\scnheader{обозначение sc-множества}
	\scnidtf{Множество всевозможных sc-элементов, обозначающих sc-множества}
	\begin{scnindent}
		\scniselement{имя собственное}
	\end{scnindent}
	\scniselement{обозначение sc-множества}
	\scnnote{Одним из элементов данного множества является знак, обозначающий это множество. Это означает, это данное множество является рефлексивным множеством}
	\scniselement{обозначение множества sc-элементов разного структурного типа}
	\scntext{примечание}{Элементами данного множества являются:
		\begin{textitemize}
			\item обозначения sc-синглетонов;
			\item обозначения sc-пар;
			\item обозначения sc-связок, не являющихся sc-синглетонами или sc-парами;
			\item обозначения sc-классов;
			\item обозначения sc-структур.
		\end{textitemize}
	}
	\scniselement{обозначение множества sc-элементов, содержащего как константные, так и переменные sc-элементы}
	\scniselement{sc-константа}
	\scnnote{Само данное множество является константным, несмотря на то, что его элементами являются как sc-константы, так и sc-переменные}
	\scniselement{обозначение множества sc-элементов, содержащего sc-элементы, обозначающие как постоянные, так и временные сущности}
	\scniselement{постоянная сущность}
	\scnnote{Слудует отличать постоянство~/~временность сущности, обозначаемой sc-элементом и постоянство~/~временность sc-множества, одним из элементов которого указанный sc-элемент является}
	\scniselement{обозначение множества sc-элементов, содержащего sc-элементы, обозначающие как статические, так и динамические sc-множества}
	\scniselement{статическое sc-множество}
	\begin{scnrelfromlist}{примечание}
		\scnfileitem{Следует отличать статичность~/~динамичность sc-множества, обозначаемого соответствующим sc-элементом и статичность~/~динамичность sc-множества, одним из элементов которого указанный выше sc-элемент является}
		\scnfileitem{Напомним, что статический характер sc-множества означает отсутствие временных sc-пар принадлежности (временных sc-дуг принадлежности), выходящих из нака этого sc-множества}
	\end{scnrelfromlist}
	\scniselement{sc-класс}
	\scnidtf{Класс всевозможных sc-элементов, обозначающих sc-множества}
	\scnidtf{Класс обозначающий sc-множеств}
	\scnnote{Следует отличать разные sc-элементы, являющиеся обозначениями соответствующих sc-множеств, и класс, элементами которого являются \underline{всевозможные} такие sc-элементы}
\end{SCn}

\newpage
\subsection{Онтологическая формализация Базовой денотационной семантики SC-кода}

Суть онтологической формализации различных областей знаний, различных фрагментов \textit{баз знаний} интеллектуальных компьютерных систем заключается в следующем:
\begin{textitemize}
	\item Выделяется достаточно большой \underline{семантически целостный} фрагмент \textit{баз знаний}, включающий в себя:
	\begin{textitemize}
		\item все элементы некоторого одного ключевого класса рассматриваемых объектов (объектов исследования) или \underline{конечного} числа таких ключевых классов объектов исследования;
		\item \underline{все связи} между выделенными объектами исследования, соответствующие заданному \underline{семейству} отношений, параметров и классов структур, которое условно будем называть предметом исследования. 
	\end{textitemize}
	\item Указанный семантически целостный фрагмент \textit{базы знаний}, являющийся чаще всего \underline{бесконечной} структурой, будем называть \textbf{\textit{предметной областью}}.
	\item Сама формальная \textbf{\textit{онтология}} представляет собой формальную спецификацию выделенной \textit{предметной области} и включает в себя следующие \textbf{\textit{частные онтологии}}:
	 \begin{textitemize}
	 	\item \textbf{\textit{структурную спецификацию предметной области}}, в которой указываются роли всех ключевых элементов (ключевых знаков), входящих в состав \textit{предметной области}. К числу таких ролей относятся:
	 	\begin{textitemize}
	 		\item \textit{максимальный класс объектов исследования\scnrolesign}
	 		\item \textit{немаксимальный класс объектов исследования\scnrolesign}
	 		\item \textit{ключевой объект исследования\scnrolesign}
	 		\item \textit{исследуемый класс связок\scnrolesign}
	 		\item \textit{исследуемый класс классов\scnrolesign}
	 		\item \textit{исследуемый класс структур\scnrolesign}
	 		\item \textit{неисследуемый класс\scnrolesign}
	 		\begin{SCn}
	 			\scnidtf{sc-класс, исследуемый в другой (смежной) предметной области}
	 		\end{SCn}
	 	\end{textitemize}
 		\item \textbf{\textit{теоретико-множественную онтологию}}, в которой описываются теоретико-множественные связи между всеми классами (\textit{sc-классами}), исследуемыми в рамках заданной (специфицируемой) \textit{предметной области}
 		\item \textbf{\textit{логическую онтологию}}, которая включает в себя
 		\begin{textitemize}
 			\item определения исследуемых классов (исследуемых понятий);
 			\item логическую иерархию исследуемых понятий, которая связывает каждое понятие со множеством тех понятий, которые явно используются в определении этого понятия;
 			\item аксиомы и теоремы, описывающие свойства специфицируемой предметной области;
 			\item тексты доказательств теорем;
 			\item логическую иерархию теорем, которая связывает каждую теорему со множеством теорем, на основе которых она доказывается.
 		\end{textitemize}
 		\item \textbf{\textit{терминологическую спецификацию предметной области}}, в которой указывается \textit{sc-идентификаторы} всех ключевых \textit{sc-элементов} специфицируемой \textit{предметной области}, а также приводятся правила построения \textit{основных sc-идентификаторов} для всех \textit{sc-классов} (понятий), исследуемых в рамках специфицируемой \textit{предметной области};
 		\item \textbf{\textit{дидактическую спецификацию предметной области}}, в которой приводится дополнительная информация, предназначенная для того, чтобы пользователи и разработчики (инженеры знаний), которые используют или совершенствуют специфицируемую предметную область и её онтологию, могли быстрее усвоить их особенности (см.~\textit{\ref{section_knowledge_control}~\nameref{section_knowledge_control}})
 		\item \textbf{\textit{проектную спецификацию предметной области и соответствующей ей онтологии}}, в которой приводится информация об истории эволюции этой \textit{предметной области и онтологии}, а также о направлениях и планах организации дальнейшего их развития.
	 \end{textitemize}
\end{textitemize}	 

Более подробно о предметных областях см. в~\textit{\ref{sec_sd}~\nameref{sec_sd}}, а более детальное рассмотрение формальных онтологий, представленных в SC-коде см. в~\textit{\ref{sec_ontology}~\nameref{sec_ontology}}.

Онтологическая формализация базовой денотационной семантики \textit{SC-кода} трактуется нами как \textit{формальная онтология}, представленная в \textit{SC-коде} и описывающая детонационную семантику \textit{семантически корректных sc-конструкций}. Указанную \textit{формальную онтологию} будем называть \textbf{\textit{Базовой денотационной семантикой SC-кода}}. Для того, чтобы уточнить \textit{предметную область}, специфицируемую этой \textit{онтологией}, введём следующие понятия:

\vskip 5cm

\begin{SCn}
	\scnheader{синонимия sc-элементов}
	\scnidtf{Бинарное ориентированное \textit{отношение эквивалентности}, каждая пара которого связывает два \textit{sc-элемента}, обозначающие одну и ту же сущность*}
	\scnnote{Синонимия двух \textit{sc-элементов} возможна только в том случае, если эти \textit{sc-элементы} хранятся в \textit{sc-памяти} (входят в состав \textit{базы знаний}) \underline{разных} \textit{ostis-систем}. В рамках каждой \textit{ostis-системы} синонимичные \textit{sc-элементы} совпадают (отождействляются, склеиваются, считаются одним и тем же \textit{sc-элементом}).}
\end{SCn}

\begin{SCn}
	\scnheader{отношение эквивалентности}
	\scnrelto{ключевое понятие}{\textsection~2.4.2. Формальная онтология связок и отношений}
\end{SCn}

\begin{SCn}
	\scnheader{sc-память}
\end{SCn}

\begin{SCn}
	\scnheader{база знаний ostis-системы}
\end{SCn}

\begin{SCn}
	\scnheader{ostis-система}
	\begin{scnsubdividing}
		\scnitem{индивидуальная ostis-система}
		\scnitem{коллективная ostis-система}
	\end{scnsubdividing}
\end{SCn}

\begin{SCn}
	\scnheader{sc-конструкция}
	\scnrelto{часто используемый sc-идентификатор}{sc-множество}
	\scnidtf{информационная конструкция, представляющая собой множество sc-элементов}
	\scnsuperset{sc-текст}
	\begin{scnindent}
		\scnidtf{текст SC-кода}
		\scnidtf{sc-конструкция, являющаяся семантически корректной по отношению к Базовой денотационной семантике SC-кода}
		\scnidtf{sc-конструкция, удовлетворяющая (соответствующая) правилам Базовой денотационной семантики SC-кода}
		\scnidtftext{часто используемый sc-идентификатор}{SC-код}
		\begin{scnindent}
			\scniselement{имя собственное}
			\scnidtf{Класс (Множество всевозможных) sc-текстов}
		\end{scnindent}
		\scnsuperset{sc-знание}
	\end{scnindent} 
\end{SCn}

\begin{SCn}
	\scnheader{sc-знание}
	\scnidtf{sc-текст, являющийся либо фрагментом (подструктурой) соответствующей \textit{предметной области}, либо \textit{высказыванием}, описывающим некоторое свойство (в частности, некоторую закономерность) этой \textit{предметной области}}
	\scnidtf{знание, представленное в SC-коде}
	\scnidtf{\textit{sc-текст}, обладающий истинным значением по отношению к соответствующей \textit{предметной области}}
	\scnsubset{связная sc-конструкция}
	\scnnote{Разные sc-знания могут противоречить друг другу, то есть отражать разные точки зрения на некоторую предметную область, но любое sc-знание должно быть sc-текстом, то есть не должно противоречить правилам Базовой денотационной семантики SC-кода}
\end{SCn}

\begin{SCn}
	\scnheader{интеграция sc-конструкций*}
	\scnidtf{объединение sc-конструкций*}
	\scnidtf{объединение sc-множеств*}
	\scnnote{При интеграции sc-конструкций sc-элементы, обозначающие одну и ту же сущность, то есть синонимичные sc-элементы, считаются одинаковыми (совпадающими, тождественными) и, следовательно, должны склеиваться (отождествляться)}
\end{SCn}

\begin{SCn}
	\scnheader{SC-пространство}
	\scnidtf{Результат интеграции \underline{всевозможных} sc-конструкций, \textit{семантически корректных} по отношению к \textit{Базовой денотационной семантики SC-кода}}
	\scnidtf{Предметная область, специфицируемая (описываемая) \textit{Базовой денотационной семантикой SC-кода}, которая является формальной онтологией, представленной средствами SC-кода}
	\scnidtf{Результат интеграции всевозможных sc-текстов (текстов SC-кода)}
	\scnidtf{Максимальный sc-текст}
	\scnidtf{Текст SC-кода, включающий в себя всевозможные sc-тексты}
	\scnidtf{Пространство sc-конструкций, семантически корректных по отношению к \textit{Базовой денотационной семантике SC-кода}}
	\begin{scnrelfromlist}{примечание}
		\scnfileitem{Особенностью \textit{SC-пространство} является то, что оно включает в себя и формальную онтологию, описывающую его свойства}
		\scnfileitem{очевидно, что \textit{SC-пространство} является \underline{бесконечным} \textit{sc-текстом}, то есть текстом, содержащим бесконечное количество \textit{sc-элементов}. В частности, в состав \textit{SC-пространства} входят \underline{все} \textit{sc-элементы}, являющиеся элементами \underline{всех} \textit{sc-множеств}, знаки которых входят в состав \textit{SC-пространства}}
		\scnfileitem{\textit{SC-пространство} является ''вместилищем'' семантически корректных (по отношению к \textit{Базовой денотационной семантике SC-кода}) частей баз знаний всевозможных ostis-систем и, в том числе, глобальной (объединенной) \textit{Базы знаний Экосистемы OSTIS}. Подчеркнём при этом, что \textit{Экосистема OSTIS} является примером распределённых иерархических \textit{ostis-систем}}
		\scnfileitem{Тот факт, что корректная (с точки зрения \textit{Базовой денотационной семантики SC-кода}) часть базы знаний \underline{каждой} \textit{ostis-системы} входит в состав \textit{SC-пространства}, позволяет трактовать описание соотношения между текущим состоянием \textit{базы знаний ostis-системы} и \textit{Sc-пространством} как описание того, что указанная \textit{ostis-система} в текущий момент времени не знает. Например, \textit{ostis-система} в некоторый момент времени может не знать (1) всех элементов некоторого конкретного \underline{конечного} \textit{sc-множества} (конечно sc-конструкции), (2) количества элементов указанного конечного \textit{sc-множества}, (3) какому подклассу заданного \textit{sc-класса} принадлежит указанный элемент этого \textit{sc-класса} и так далее}
		\scnfileitem{В \textit{памяти ostis-системы} каждый \textit{sc-элемент} считается в рамках этой памяти \underline{временной} сущностью (имеется в виду сам \textit{sc-элемент}, а не обозначаемая им сущность), поскольку он появляется в \textit{памяти ostis-системы} и удаляется из неё независимо от того, что он обозначает. В отличие от этого в \textit{SC-пространстве} все sc-элементы считаются постоянными (\underline{постоянно} присутствующими) в рамках этого пространства}
	\end{scnrelfromlist}
\end{SCn}

\begin{SCn}
	\scnheader{Базовая денотационная семантика SC-кода}
	\scnidtf{Онтология Базовой денотационной семантики SC-кода}
	\scnidtf{Формальная \textit{онтология}, представленная в \textit{SC-коде} и являющаяся материнской \textit{онтологией} (онтологией самого высокого уровня) для всех остальных \textit{формальных онтологий}, представленных в \textit{SC-коде}}
	\scnidtf{Онтология SC-пространства}
	\scnidtf{Описание (представление) системы \textit{правил построения семантически корректируемых sc-конструкций}, удовлетворяющих требованиям Базовой денотационной семантики SC-кода}
	\scniselement{sc-онтология}
	\begin{scnindent}
		\scnidtf{формальная онтология, представленная в SC-коде}
	\end{scnindent} 
\end{SCn}

В состав \textit{Базовой денотационной семантики SC-кода} включается:
\begin{textitemize}
	\item Приведенный выше текст \textit{Пункта 2.2.2.1. Семантическая классификация sc-элементов по базовым признакам}
	\item Приведенный выше текст \textit{Пункта 2.2.2.2. Уточнение смысла выделенных классов sc-элементов и связей между этими классами}
	\item Средства базовой семантической спецификации sc-элементов, рассмотренные в  \textit{Пункте 2.2.2.3. Структура базовой семантической спецификации sc-элемента}
\end{textitemize}

\begin{SCn}
	\scnheader{Логическая онтология SC-пространства}
	\scnrelto{логическая онтология}{Базовая денотационная семантика SC-пространства}
	\scntext{примечание}{Приведём пример правил, входящих в состав данной логической онтологии:
		\begin{textitemize}
			\item Вторыми компонентами sc-пар константной парой принадлежности могут быть sc-элементы \underline{любого}типа(в том числе, и sc-переменные), но первыми компонентами таких sc-пар могут быть только \underline{константные} sc-множества
			\item Знак \textit{sc-ситуации} связан с элементами этой ситуации sc-парами константной \underline{постоянной} позитивной принадлежности. То есть позитивная принадлежность счиатется постоянной в рамках интервала времени существования соответствующей ситуации. В этом смысле ситуацию можно считать квазистатической
			\item Знак атомарной логической формулы связан со всеми элементами этой формулы sc-парами \underline{константной} постоянной позитивной принадлежности, в том числе, и с теми элементами атомарной формулы, которые являются sc-переменными
			\item Из переменного sc-множества могут выходить только переменные sc-пары принадлежности
			\item Не существует sc-пар принадлежности выходящих из обозначений внешних сущностей и sc-пар
			\item и другие
		\end{textitemize}
	}
\end{SCn}

\newpage
\section{Синтаксис SC-кода}
\label{sec_sr_scsyntax}

\begin{SCn}
	\scnheader{sc-элемент}
	\begin{scnsubdividing}
		\scnitem{sc-коннектор}
		\begin{scnindent}
			\begin{scnsubdividing}
				\scnitem{sc-дуга}
				\begin{scnindent}
					\begin{scnsubdividing}
						\scnitem{базовая sc-дуга}
						\scnitem{sc-дуга общего вида}
						\begin{scnindent}
							%							\scneq{\textup{(}sc-синглетон $\cup$ sc-связка не являющаяся синглетоном и парой\textup{)}}
						\end{scnindent}	
					\end{scnsubdividing}
				\end{scnindent}	
				\scnitem{sc-ребро}
				\begin{scnindent}
					\scneq{неориентированная sc-пара}
				\end{scnindent}	
			\end{scnsubdividing}
		\end{scnindent}	
		\scnitem{sc-узел общего вида}
		\begin{scnindent}
			\scneq{\textup{(}sc-синглетон $\cup$ sc-связка, не являющаяся синглетоном и парой\textup{)}}
		\end{scnindent}	
	\end{scnsubdividing}
\end{SCn}

\newpage
\section{Файлы ostis-системы}
\label{sec_sr_ostisfiles}
\section{Смысловое пространство ostis-систем}
\label{sec_sr_semspace}
\begin{SCn}
	\begin{scnrelfromlist}{ключевое понятие}
		\scnitem{sc-отношение}
		\scnitem{бинарное sc-отношение}
		\scnitem{слотовое sc-отношение}
		\scnitem{sc-структура*}
		\scnitem{элементарно представленный элемент'}
		\scnitem{полносвязно представленный элемент'}
		\scnitem{полностью представленный элемент'}
		\scnitem{sc-связка'}
		\scnitem{sc-отношение'}
		\scnitem{sc-класс'}
		\scnitem{сущностное замыкание*}
		\scnitem{содержательное замыкание*}
		\scnitem{sc-отношение сходства по слотовым отношениям*}
		\scnitem{sc-отношение семантического сходства по слотовым отношениям*}
		\scnitem{связная sc-структура*}
		\scnitem{семантическое сходство sc-структур*}
		\scnitem{семантическое непрерывное сходство sc-структур*}
		\scnitem{ключевой запрос’}
		\scnitem{минимальный ключевой запрос’}
		\scnitem{полная семантическая окрестность элемента*}
		\scnitem{интроспективный ключевой элемент’}
		\scnitem{топологическое пространство}
		\scnitem{топологическое пространство замыкания инцидентности коннекторов}
		\scnitem{топологическое пространство синтаксического замыкания}
		\scnitem{топологическое пространство сущностного замыкания}
		\scnitem{топологическое пространство содержательного замыкания}
		\scnitem{метрика}
		\scnitem{семантическая метрика}
		\scnitem{метрическое пространство}
		\scnitem{метрическое конечное синтаксическое пространство}
		\scnitem{метрическое конечное семантическое пространство}
		\scnitem{псевдометрика}
		\scnitem{псевдометрическое пространство}
		\scnitem{псевдометрическое конечное семантическое пространство}
	\end{scnrelfromlist}
\end{SCn}

Понятие SC-пространства наряду с понятием SC-кода является необходимым для уточнения и формализации понятия смысла информационных конструкций и в унификации смыслового представления информации. В SC-пространстве можно выделять структуры, связанные как с синтаксическими свойствами текстов SC-кода, так и с его семантикой. В последнем случае речь можно вести о «смысловом пространстве». Смысл информационной конструкции, в конечном счёте, определяется (1) внутренними связями всех элементарных фрагментов этой конструкции и (2) её внешними связями с элементами смыслового пространства (её положением в контексте). Смысловое пространство является результатом естественного становления знаний в процессе их интеграции.
Важнейшим достоинством SC-пространства является возможность уточнения таких понятий, как понятие аналогичности (сходства и отличия) различных описываемых «внешних» сущностей, аналогичности унифицированных семантических сетей (текстов SC-кода), понятие семантической близости описываемых сущностей (в том числе, и текстов SC-кода).

Следует отметить, что в силу абстрактности языков модели унифицированного семантического представления знаний и условности выбора меток элементов их текстов, нельзя исключить, что объединение двух произвольных текстов таких языков не будет текстом языка модели унифицированного семантического представления знаний. Чтобы избежать результатов подобных эклектических объединений с точки зрения синтаксиса или семантики, для абстрактных языков следует рассматривать множество «смысловых пространств». Однако, для конкретных (реальных) языков может оказаться достаточным рассмотрение одного «смыслового пространства».
Далее рассмотрим:

\begin{textitemize}
	\item возможность перехода от sc-текстов к графовым структурам и от них к топологическому пространству;
	\item возможность перехода от sc-текстов к графовым структурам и от них к многообразию (топологическому пространству);
	\item возможность перехода от sc-текстов к графовым структурам и от них к метрическому пространству.
\end{textitemize}

Чтобы исследовать структурные свойства SC-пространства, можно использовать уже разработанные модели пространств и связь их известными топологическими моделями например такими как графы. При этом изначально не будем принимать в расчёт динамические особенности, связанные с обработкой знаний, однако позже будет показано, что учёт динамики в процессах обработки и при становлении знаний является необходимым для вычисления семантической метрики, являющейся одним из определяющих признаков знаний.
Обратимся к исследованию структурно-топологических свойств пространства.

Структурно-топологические свойства могут свидетельствовать о синтаксических или семантических зависимостях обозначений текстов языка, позволяющих упростить работу со структурами за счёт перехода к более простым структурам на уровнях управления данными или знаниями в характерных задачах управления для библиотек компонентов.
На множестве элементов, образующих SC-пространство, можно изучать топологические свойства и рассматривать SC-пространство как топологическое пространство. Следует заметить, что, несмотря на то, что SC-код ориентирован на смысловое представление знаний, в силу наличия НЕ-факторов, не все смыслы могут быть представлены в некоторый момент времени и не будет известна структура соответствующего представления. Поэтому структурно-топологические свойства текстов языка представления знаний скорее определяют синтаксическое пространство, нежели семантическое (смысловое). Хотя оба могут приближаться друг к другу по мере устранения неопределённостей, вызванных НЕ-факторами.

Рассмотрим следующие виды топологических пространств:
\begin{textitemize}
	\item топологическое пространство замыкания инцидентности коннекторов;
	\item топологическое пространство синтаксического замыкания;
	\item топологическое пространство сущностного замыкания;
	\item топологическое пространство содержательного замыкания.
\end{textitemize}

\begin{SCn}
	\scnheader{топологическое пространство}
	\scnexplanation{Топологическое пространство -- множество с определённым над ним множеством (семейством) (открытых) подмножеств, включая само множество и пустое множество. Для любого счётного подмножества семейства результат объединения принадлежит семейству, а для любого подмножества семейства результат пересечения также принадлежит семейству. Дополнения множеств семейства до наибольшего из множеств называются замкнутыми множествами.}
\end{SCn}

Чтобы рассмотреть более детально некоторые виды топологических пространств введём следующие понятия.

\begin{SCn}
	\scnheader{sc-отношение}
	\scnexplanation{sc-отношение -- sc-множество sc-связок.}
\end{SCn}

\begin{SCn}
	\scnheader{бинарное sc-отношение}
	\scnexplanation{Бинарное sc-отношение -- sc-множество sc-пар.}
\end{SCn}

\begin{SCn}
	\scnheader{слотовое sc-отношение}
	\scnexplanation{Слотовое sc-отношение -- sc-множество (ориентированных) sc-пар, которые не являются узловыми sc‑парами.}
\end{SCn}

\begin{SCn}
	\scnheader{sc-структура*}
	\scnexplanation{sc-структура* -- sc-множество, в котором есть непустое sc-подмножество-носитель (множество первичных элементов sc-структуры*).}
\end{SCn}

\begin{SCn}
	\scnheader{элементарно представленный элемент’}
	\scnexplanation{Элементарно представленный элемент’ -- элемент sc-структуры*, sc‑множество, все элементы которого являются элементами sc-структуры*.}
\end{SCn}

\begin{SCn}
	\scnheader{полносвязно представленный элемент’}
	\scnexplanation{Полносвязно представленный элемент’ -- элемент sc-структуры*, sc‑множество, все элементы и все принадлежности которому являются элементами sc-структуры*, или sc‑дуга, являющаяся элементарно представленным элементом’ этой sc-структуры*.}
\end{SCn}

\begin{SCn}
	\scnheader{полностью представленный элемент’}
	\scnexplanation{Полностью представленный элемент’ -- полносвязно представленный элемент’ sc‑структуры*, с любым элементом, не являющимся sc-дугой, выходящей из него, связанный принадлежащей этой sc-структуре* sc-дугой принадлежности или sc-дугой непринадлежности.}
\end{SCn}

\begin{SCn}
	\scnheader{sc-связка’}
	\scnexplanation{sc-связка’ -- полносвязно представленный элемент’ sc-структуры*, принадлежащий sc‑отношению’ этой sc-структуры*, являющийся sc-связкой.}
\end{SCn}

\begin{SCn}
	\scnheader{sc-отношение’}
	\scnexplanation{sc-отношение’ -- полносвязно представленный элемент’ sc-структуры*, sc-отношение, все элементы которого являются sc-связками’ этой sc-структуры*.}
\end{SCn}

\begin{SCn}
	\scnheader{sc-класс’}
	\scnexplanation{sc-класс’ -- полносвязно представленный элемент’ sc-структуры*, все элементы которого являются элементами sc‑структуры*, не являющийся ни sc-отношением’, ни sc-связкой’ этой sc‑структуры*.}
\end{SCn}

\begin{SCn}
	\scnheader{сущностное замыкание*}
	\scnexplanation{Сущностное замыкание* -- наименьшая надмножество* (структура*), в котором каждый элемент является элементарно представленным’.}
\end{SCn}
◦ ;

\begin{SCn}
	\scnheader{содержательное замыкание*}
	\scnexplanation{Содержательное замыкание* -- наименьшее надмножество* (структура*), в котором каждый элемент является полносвязно представленным’}
\end{SCn}

\begin{SCn}
	\scnheader{sc-отношение сходства по слотовым отношениям*}
	\scnexplanation{sc-отношение сходства по слотовым sc-отношениям* -- sc-отношение, являющееся рефлексивным по этим слотовым отношениям, т.е. для любого элемента, входящего в связку этого sc-отношения под одним из слотовых sc-отношений, найдётся связка этого sc‑отношения, в которую он входит под каждым из этих слотовых sc-отношений.}
\end{SCn}

\begin{SCn}
	\scnheader{sc-отношение семантического сходства по слотовым отношениям*}
	\scnexplanation{sc-отношение семантического сходства по слотовым отношениям* -- sc-отношение сходства по слотовым sc-отношениям* si и sj, в котором каждый элемент под слотовым sc-отношением si, может быть преобразован к элементу синтаксического типа элемента под слотовым sc-отношением sj; два инцидентных sc-элемента под слотовым sc-отношением si, в рамках этого sc-отношения семантического сходства соответствуют инцидентным элементам соответственно под слотовым sc-отношением sj.}
\end{SCn}

\begin{SCn}
	\scnheader{связная sc-структура*}
	\scnexplanation{Связная sc-структура* -- sc-структура*, являющаяся связной.}
\end{SCn}

\begin{SCn}
	\scnheader{семантическое сходство sc-структур*}
	\scnidtf{семантическое подобие sc-структур*}
		\scnexplanation{Семантическое сходство sc-структур* -- связывает sc-множество sc-структур* с sc‑структурой* sc‑отношением семантического сходства по слотовым sc-отношениям si, sj так, что для каждой sc-структуры* из sc‑множества найдётся её элемент и связка этого sc‑отношения сходства, в которую он входит под слотовым sc‑отношением si, а под слотовым sc‑отношением sj входит элемент sc-структуры*, также для каждого элемента sc‑структуры найдётся связка этого sc-отношения сходства, в которую он входит под слотовым sc‑отношением sj, а под слотовым sc‑отношением si входит элемент sc-структуры* из sc‑множества.}
\end{SCn}

\begin{SCn}
	\scnheader{семантическое непрерывное сходство sc-структур*}
	\scnidtf{семантическое непрерывное подобие sc-структур*}
	\scnexplanation{Семантическое непрерывное сходство sc-структур* -- связывает sc-множество sc‑структур* со связной sc‑структурой* sc‑отношением семантического сходства по слотовым sc-отношениям si, sj так, что для каждой sc-структуры* из sc‑множества найдётся её элемент и связка этого sc‑отношения сходства, в которую он входит под слотовым sc‑отношением si, а под слотовым sc‑отношением sj входит элемент связной sc-структуры*, также для каждого элемента связной sc-структуры найдётся связка этого sc-отношения сходства, в которую он входит под слотовым sc‑отношением sj, а под слотовым sc‑отношением si входит элемент sc-структуры* из sc‑множества.}
\end{SCn}

\begin{SCn}
	\scnheader{ключевой запрос’}
	\scnexplanation{Ключевой запрос’ -- поисковый-проверочный запрос (от одного известного элемента), который выполняется хотя бы от одного элемента и не выполняется хотя бы от одного элемента.}
\end{SCn}

\begin{SCn}
	\scnheader{минимальный ключевой запрос’}
	\scnexplanation{Минимальный ключевой запрос’ -- ключевой запрос, который находит sc‑подмножества множеств элементов, находимых всеми другими ключевыми запросами, которые имеют те же области известных элементов выполнимости и невыполнимости.}
\end{SCn}

\begin{SCn}
	\scnheader{полная семантическая окрестность элемента*}
	\scnexplanation{Полная семантическая окрестность элемента* -- все элементы, находимые выполнимыми минимальными ключевыми запросами от этого элемента (c учётом дизъюнктивного поиска и отрицания поиска).}
\end{SCn}

\begin{SCn}
	\scnheader{интроспективный ключевой элемент’}
	\scnexplanation{Интроспективный (базовый) ключевой элемент’ -- элемент множества (из класса наименьших таких множеств) элементов такого, что любая полная семантическая окрестность любого элемента является sc-подмножеством объединения их полных семантических окрестностей}
\end{SCn}


\begin{SCn}
	\scnheader{топологическое пространство замыкания инцидентности коннекторов}
	\scnexplanation{Топологическое пространство замыкания инцидентности коннекторов на множестве sc-элементов -- топологическое пространство, все замкнутые множества которого содержат все sc-элементы этого множества, до которых есть маршрут по ориентированным связкам отношения инцидентности коннекторов.}
\scncomment{В общем случае не удовлетворяет аксиоме отделимости по Тихонову. Прагматика рассмотрения таких пространств обуславливается операциями удаления sc-элементов и коннекторов, которым они инцидентны. Удаление sc-элемента требует удаления всех коннекторов, замыканию любой открытой окрестности которых он принадлежит.}
\end{SCn}

\begin{SCn}
	\scnheader{топологическое пространство синтаксического замыкания}
	\scnexplanation{Топологическое пространство синтаксического замыкания на множестве sc-элементов -- топологическое пространство, все замкнутые множества которого содержат все sc-элементы этого множества, до которых есть маршрут по ориентированным связкам отношения инцидентности.}
\scncomment{В общем случае не удовлетворяет аксиоме отделимости по Колмогорову. В качестве основы замкнутых множеств топологического пространства можно выделить синтаксическое замыкание, однако в силу возможности проведения дуг из любого sc-узла в любой в итоговом случае (в итоге процесса устранения НЕ-факторов) такое пространство является тривиальным (антидискретным) пространством. Отношение объединения топологических пространств синтаксического замыкания алгебраически не замкнуто на множестве топологических пространств синтаксического замыкания. По той же причине для любого неполного топологического пространства синтаксического замыкания можно рассмотреть топологическое пространство синтаксического замыкания, носитель которого является надмножеством носителя первого и которое не сохраняет замкнутые множества. В этом смысле топология на основе синтаксического замыкания не является устойчивой относительно процессов становления знаний и её рассмотрение прагматически не оправдывается. Топология полного же топологического пространства синтаксического замыкания антидискретна (тривиальна). Таким образом, у полного топологического пространства синтаксического замыкания все топологические подпространства синтаксического замыкания обладают антидискретной (тривиальной) топологией.}
\end{SCn}
\begin{SCn}
	\scnheader{топологическое пространство сущностного замыкания}
	\scnexplanation{Топологическое пространство сущностного замыкания на множестве sc-элементов -- топологическое пространство, все замкнутые множества которого являются сущностными замыканиями.}
	\scncomment{В общем случае не удовлетворяет аксиоме отделимости по Тихонову. В качестве носителя топологического (под)пространства можно выделить сущностное замыкание. Топологическое пространство сущностного замыкания сохраняет замкнутые множества любых топологических пространств сущностного замыкания, носитель которых является подмножеством его носителя и сущностным замыканием. Такие пространства образуют структуру топологических подпространств‑топологических надпространств сущностного замыкания. Топология пространств в этой структуре разнообразна.}
\end{SCn}

\begin{SCn}
	\scnheader{топологическое пространство содержательного замыкания}
	\scnexplanation{Топологическое пространство содержательного замыкания на множестве sc-элементов -- топологическое пространство, все замкнутые множества которого являются содержательными замыканиями.}
	\scncomment{В общем случае не удовлетворяет аксиоме отделимости по Тихонову. В качестве носителя топологического (под)пространства можно выделить содержательное замыкание. Топологическое пространство содержательного замыкания сохраняет замкнутые множества любых топологических пространств содержательного замыкания, носитель которых является подмножеством его носителя и содержательным замыканием. Такие пространства образуют структуру топологических подпространств‑топологических надпространств содержательного замыкания. Топология пространств в этой структуре разнообразна.
	}
\end{SCn}

Возможен переход от sc-структур к многообразиям и топологическим пространствам путём сведения sc-структур к графовым структурам, подробно вопросы сведения sc-структур к графовым структурам и далее -- к многообразиям и топологическим пространствам рассмотрены в работе [].

Для более сложных структур таких, как полная семантическая окрестность, топологические свойства подлежат дальнейшему изучению.

Далее можно рассмотреть метрические пространства, в частности -- конечные подпространства с полностью представленными sc-элементами. 

\begin{SCn}
	\scnheader{метрика}
	\scnexplanation{Метрика -- функция двух аргументов, принимающая значения на (линейно) упорядоченном носителе группы, неотрицательна, равна нейтральному элмененту (нулю) только при равенстве аргументов, симметрична, удовлетворяет неравенству треугольника.}
\end{SCn}

\begin{SCn}
	\scnheader{метрическое пространство}
	\scnexplanation{Метрическое пространство -- множество, с определённой на нём функцией двух аргументов, являющейся метрикой, принимающей значения на упорядоченном носителе группы.}
\end{SCn}

\begin{SCn}
	\scnheader{семантическая метрика}
	\scnidtf{семантическая близость}
	\scnexplanation{Семантическая метрика -- метрика, определённая на знаках и выражающая количественно близость их значений.}
\end{SCn}

\begin{SCn}
	\scnheader{метрическое конечное синтаксическое пространство}
	\scnexplanation{Метрическое конечное синтаксическое пространство SC-кода -- метрическое пространство с конечным носителем, элементами которого являются обозначения (sc-элементы), а значение метрики может быть определено через отношения инцидентности элементов без учёта их семантического типа.}
\end{SCn}

\begin{SCn}
	\scnheader{метрическое конечное семантическое пространство}
	\scnexplanation{Метрическое конечное семантическое пространство SC-кода -- метрическое пространство с конечным носителем, элементами которого являются обозначения (sc-элементы), а значение метрики не может быть определено через отношения инцидентности элементов без учёта их семантического типа.}
\end{SCn}

\begin{SCn}
	\scnheader{псевдометрика}
	\scnexplanation{Псевдометрика -- функция двух аргументов, принимающая значения на (линейно) упорядоченном носителе группы, неотрицательна, симметрична, удовлетворяет неравенству треугольника.}
\end{SCn}

\begin{SCn}
	\scnheader{псевдометрическое пространство}
	\scnexplanation{Псевдометрическое пространство -- множество, с определённой на нём функцией двух аргументов, являющейся псевдометрикой, принимающей значения на упорядоченном носителе группы.}
\end{SCn}

\begin{SCn}
	\scnheader{псевдометрическое конечное семантическое пространство}
	\scnexplanation{Псевдометрическое конечное семантическое пространство SC-кода -- псевдометрическое пространство с конечным носителем, элементами которого являются обозначения (sc-элементы), а значение псевдометрики не может быть определено через отношения инцидентности элементов без учёта их семантического типа.}
\end{SCn}



В силу неполноты выразительных средств для представления изменяющихся со временем знаний, отсутствия определённой пространственно-временной модели, наличия семантически неопределённых или слабоопределённых обозначений в текстах да и наличия недоопределённости самих текстов описанного в предыдущих разделах языка, на данном этапе в этом описании затруднительно предложить какую-либо модель метрического пространства для более сложных структур, учитывающих НЕ‑факторы, связанные с пространством‑временем. Это станет возможным при проявлении желания идти навстречу, готовности к конвергенции, интероперабельности и после достижения консенсуса, достаточного для соответствующего описания развития предлагаемого стандарта и языка.

Тем не менее, некоторые такие модели были успешно предложены в работе []. Предложенные модели полагались на представление, способное выразить семантику переменных обозначений и операционную семантику расширенными средствами алфавита. Для построения подобных моделей, кроме расширенных средств алфавита, предлагается полагаться на модели, описывающие процессы интеграции и становления знаний [], на средства спецификации знаний [], ориентированные на рассмотрение финитных структур, что позволяет перейти к рассмотрению сложных метрических соотношений в рамках метамодели смыслового пространства.

В современных работах в технических науках [bManin], возможно, наиболее близкими понятиями являются понятия, выражающие смысл термина «семантическое пространство» (интериорный подход (Табл. 1)).
Общим во многих подходах к работе с «семантическим пространством» является рассмотрение словоформ или лексем (множеств словоформ) и их признаков (Табл. 1). В литературе [bManin] встречаются следующие подходы (Табл. 2):
\begin{textitemize}
	\item подход на основе семантических осей и пространства признаков (бинарных $\left\lbrace 0,1\right\rbrace ^{n}$, монополярных $\left[0;1\right]^{n}$, биполярных $\left[-1;1\right]^{n}$);
	\item подход на основе семантических осей и нейронного кодирования места в поле смыслов (слова и словосочетания имеют области (подмножества) значений, связываясь другими частями речи как включением и пересечением, тексты соответствуют пути связанных областей, бинарное кодирование групп нейронов, распознающих смыслы);
	\item подход на основе модели «смысл-текст» [bMeaningText] (отражение неполноты семантических шкал и анализ синтагм и поверхностно-синтаксической структуры);
	\item нейролингвистические данные отражает процессы синтеза и восприятия речи в нейронных сетях (сеть лексического синтеза), близка к модели «смысл‑текст»;
	\item модели, построенные на основе статического анализа (корпусов) текстов (модель векторного пространства).
\end{textitemize}
Статистический подход к обработке естественного языка противопоставляется интуиции и коммуникативному опыту учёных [bManin].

В основе подхода лежит семантическая статическая гипотеза, что смысл слов (лексем) определяется контекстом использования (его статистическим образом) в языке (с коммуникативной структурой) [bManin].

Модель векторного пространства семантики [bManin]. Модель рассматривается для двух случаев: большого словаря ($N\leq{M}$) и задачи информационного поиска ($M\leq{N}$). $M$ -- размер словаря, $N$ -- количество контекстов.

На основе статистики строится матрица размерности $M\times{N}$ частот $p_{ij}$ появления лексемы (слова) $w_{i}$ в документе (контексте, подтексты, которые могут перекрываться) $c_{j}$.

В знаменателе -- оценки вероятности слова и контекста соответственно.

В случае невырожденной матрицы $r=N$ каждая такая матрица задаёт точку в грассманиане $N$‑мерных подпространств $M$‑мерного пространства ($N\leq{M}$).

В случае невырожденной матрицы $r=M$ каждая такая матрица задаёт точку в грассманиане $M$‑мерных подпространств $N$‑мерного пространства ($M\leq{N}$).

Каждый текст -- точка в грассманиане [bGrassmanian], соответствующем проективному пространству , относительно одного выделенного контекста. Для всех контекстов получая ориентированную $N$-ку, в соответствии с порядком контекстов в текстах, можно построить маршрут (путь), соединяя геодезическими соседние точки в $N$-ке. Для двух текстов  и  это будут две ломанные, между которыми можно вычислить метрику Фреше [bFrechet], используя метрику Фубини-Штуди [bFubiniStudy] в , для этого следует параметризовать пути  и  через  (,): 
.

Другой способ задать линейный порядок -- это рассмотреть фильтрацию (флаг (флаговое многообразие)) [bFlagManifold] в , заданную расширяющимися контекстами. В итоге для текста получаем точки (флаги) во флаговом многообразии. Для флаговых многообразий тоже можно вычислить метрику Фубини-Штуди [bFubiniStudy].

Этот порядок соответствует временному измерению (процессу коммуникации во времени), что может быть существенным. Другой порядок может быть не зависимым от этого, например алфавитный или порядок в соответствии с законом Ципфа [bZipf], [bZipf2]. 

Сравнение подходов к построению «семантических пространств»

\begin{tabular}{|>{\centering\arraybackslash}m{3cm}|>{\centering\arraybackslash}m{2cm}|>{\centering\arraybackslash}m{2cm}|>{\centering\arraybackslash}m{3cm}|>{\centering\arraybackslash}m{3cm}|>{\centering\arraybackslash}m{3cm}|}
	\hline
& семанти-ческие оси и простран-ства признаков
& семанти-ческие оси и нейронное кодирование признаков
& модель <<смысл‑текст>>
& нейролингвисти-ческое кодирование
& статистическая модель (модель векторного пространства семантики)
\\
	\hline
	определённые семантические оси
	& +
	& +
	& -
	& -
	& -
	\\
	\hline
	динамическая (вычислительная) декомпозиция
	& -
	& +
	& -
	& +
	& -
	\\
	\hlineсеманти-ческие оси и простран-ства признаков
	семанти-ческие оси и нейронное кодирование признаков
	модель смысл‑текст
	нейролингвисти-ческое кодирование
	статистическая модель (модель векторного пространства семантики)
	
	определённые семантические оси
	+
	+
	-
	-
	-
	
	динамическая (вычислительная) декомпозиция
	-
	+
	-
	+
	-
	
	анализ когнитивных процессов (интроспекция)
	-
	+
	-
	+
	-
	
	учёт НЕ‑факторов (неполнота)
	-
	-
	+
	+
	+
анализ когнитивных процессов (интроспекция)
 & -
 & +
 & -
 & +
 & -
 \\
	\hline
учёт НЕ‑факторов (неполнота)
 & -
 & -
 & +
 & +
 & +
 \\
	\hline
\end{tabular}






Вопросы соотнесения смыслов, их формализации, развития языков в пространстве и времени рассмотрены в работах В.В. Мартынова [bMartynov], [bMartynov2], [bGordey].

%\input{author/references}
\chapauthor{Садовский М.Е.\\Жмырко А.В.}
\chapter{Семейство внешних языков интеллектуальных компьютерных систем нового поколения, близких языку внутреннего смыслового представления знаний (SCg, SCs, SCn)}
\chapauthortoc{Садовский М.Е.\\Жмырко А.В.}
\label{chapter_ext_lang}

\abstract{Аннотация к главе.}

\section{Заголовок параграфа}
%\label{sec}
Текст параграфа

%\input{author/references}

\chapter{Формальные онтологии базовых классов сущностей -- множеств, связей, отношений, параметров, величин, чисел, структур, темпоральных сущностей}
\label{chapter_top_ontologies}

\abstract{Аннотация к главе.}

\section{Заголовок параграфа}
%\label{sec}
Текст параграфа

%\input{author/references}
\chapauthor{Голенков В.В.\\Банцевич К.А.}
\chapter{Структура баз знаний интеллектуальных компьютерных систем нового поколения: иерархическая система предметных областей и соответствующих им онтологий. Онтологии верхнего уровня}
\chapauthortoc{Голенков В.В.\\Банцевич К.А.}
\label{chapter_kb}

\abstract{Аннотация к главе.}

\section{Формализация понятия знания и формальные модели баз знаний ostis-систем}
\section{Формализация понятия структуры}
\section{Формализация понятия семантической окрестности}
\section{Формализация понятия предметной области}
\section{Формализация понятия онтологии}

%\input{author/references}
\chapauthor{Шункевич Д.В.\\Василевская А.П.\\Орлов М.К.}
\chapter{Смысловое представление логических формул и высказываний в различного вида логиках}
\chapauthortoc{Шункевич Д.В.\\Василевская А.П.\\Орлов М.К.}
\label{chapter_logic}

\abstract{Аннотация к главе.}

\section{Заголовок параграфа}
%\label{sec}
Текст параграфа

%\input{author/references}
\chapauthor{Никифоров С.А.\\Гойло А.А.}
\chapter{Языковые средства формального описания синтаксиса и денотационной семантики естественных языков в ostis-системах}
\chapauthortoc{Никифоров С.А.\\Гойло А.А.}
\label{chapter_lang}

\abstract{
    Аннотация к главе.

    Список доработок:
    \begin{textitemize}
        \item написать аннотацию;
        \item вставить отредактированный анализ из статьи (в частности, тут не нужно говорить про sem web и нужно убрать пару опасных высказываний, погорячился);
        \item формализовать все обобщенные правила,
        \item адекватное определения лексемы, пересмотреть термины для исклюения возможности появления омонимии / синонимии,
        \item примеры и пояснения,
        \item сказать что-то про отношения на лексемах,
        \item выделить пр. о. лексики,
        \item пройтись по подсказкам от плагина, поправит форматирование,
        \item формализовать пр. о..
    \end{textitemize}
    Список вопросов:
    \begin{textitemize}
        \item нужно ли фомрализвать все частные правила, их несколько десятков.
    \end{textitemize}
}

Проблема совместимости результатов исследований также остро стоит и в лингвистике -- науке, в которой существует множество различных теорий, часто несовместимых друг с другом.
В лингвистических исследованиях используются разные варианты разметки данных, нет одного подхода к структуризации корпусов текстов и различаются способы представления данных в них~\cite{Farrar2002ACO},~\cite{Chiarcos2012}.

В качестве решения проблемы несовместимости различных способов описания данных в лингвистике предлагались варианты стандартизации форматов такого описания.
Примером могут служить Text Encoding Initiative -- консорциум по стандартизации представления текстов в цифровом виде~\cite{Text_Encoding_Initiative} и гайдлайны экспертной группы по стандартизации представления языковых данных EAGLES (например, рекомендации по разметке корпусов текстов\cite{EAGLES_Recommendations}). Однако ни один из таких стандартов не получил распространения и не стал использоваться лингвистами повсеместно~\cite[p.~4]{Ide2010WhatDI}.

Вместо создания рекомендаций по разметке языкового материала в качестве более эффективного средства решения указанных выше проблем предлагается создание онтологий~\cite{schalley_2019},~\cite{mccrae_2015}. Помимо того, что онтология верхнего уровня для предметной области языкознания может служить связующим звеном между различными лингвистическими теориями, она также представляет собой формализованное описание лингвистических концептов, представленное в удобном для компьютеров формате, что обусловливает ее применимость в системах, способных понимать аннотированные языковые данные, совершать интеллектуальный поиск по корпусам текстов, а также потенциально выполнять анализ существующих лингвистических исследований~\cite{Farrar2002ACO}.

В качестве такой онтологии ПрО лингвистики выступает The General Ontology of Linguistic Description (GOLD)~\cite{gold}.
В этой онтологии формализованы наиболее базовые категории и отношения, используемые в лингвистике, а сама онтология интегрирована с онтологией верхнего уровня Suggested Upper Merged Ontology (SUMO)~\cite{pease_2002_sumo}. Авторы GOLD пишут, что создавали онтологию в первую очередь для того, чтобы решить проблему интероперабельности данных лингвистической типологии и для того, чтобы с ее помощью экспертные системы могли обрабатывать научные данные по естественным языкам -- т.е. целью создателей онтологии не являлось непосредственно решение задач из области обработки текстов на естественном языке~\cite{farrar_2003} (с.4).

Онтологией естественных языков, нацеленной непосредственно на использование при решении задач по обработке ЕЯ, является Ontologies of Linguistic Annotation (OLiA)~\cite{chiarcos-2012-ontologies}. Основной идеей онтологии является обеспечение совместимости разметки языковых данных, полученных в результате выполненного компьютерными системами анализа текстов на естественном языке с соответствующими им лингвистическими концептами из онтологии -- в отличие от других лингвистических онтологий, OLiA предоставляет не только инвентарь концептов и отношений, но и необходимую спецификацию интеграции этих элементов с разметкой языковых данных (например, в корпусах).~\cite[p.~4]{chiarcos-2012-ontologies}.

При создании онтологий естественного языка, встает вопрос о статусе спецификации лингвистической информации в таких онтологиях. Дж. Бейтман выделяет три типа онтологий в зависимости от интегрированности естественно-языковой информации в онтологию~\cite{Bateman_1997}:
\begin{enumerate}
    \item онтологии, представляющие собой абстрактную семантико-концептуальную репрезентацию знаний о мире, которая используется непосредственно в качестве денотационной семантики для синтаксиса и лексики естественного языка;
    \item онтологии, в которых есть отдельная спецификация денотационной семантики естественного языка, которая служит интерфейсом между синтаксисом естественных языков и собственно концептуальной онтологией;
    \item онтологии, представляющие собой абстрактную спецификацию знаний о реальном мире вне зависимости от ограничений естественного языка
\end{enumerate}

Популярность в сфере обработки естественного языка приобрел второй тип онтологии~\cite[p.~8]{Bateman_1997}, так как он, в отличие от третьего подхода, который совсем не формализует лингвистическую информацию, позволяет специфицировать больше информации о естественных языках. Так, одна из самых популярных онтологий, используемых в системах для обработки естественного языка, the Generalized Upper Model~\cite{Bateman2002TheGU}, является онтологией второго типа~\cite{Bateman_1997}.
П. Буителар и др. подчеркивают, что всем формальным онтологиям необходима связь с языковой информацией для решения таких задач как выделение информации из ЕЯ-текстов, автоматизированное заполнение онтологий и генерации текста на естественном языке~\cite{Buitelaar_2009}.

Так как использование онтологий в NLP позволяет задать семантику получаемым в результате обработки естественного языка данным и потенциально повысить качество NLP-анализа, начинается переход к созданию движимых онтологиями систем обработки естественных языков~\cite{Kostareva2016UsingOM},~\cite{nevzorova_2019}.
Онтологии естественного языка активно применяются для генерации ЕЯ-текстов на основе некоторой онтологии предметной области~\cite{cimiano-etal-2013-exploiting},~\cite{Bouayad_2014}.

Онтологический подход также используется в системах естественно-языковых запросов для баз данных, в которых запрос на естественном языке транслируется в язык запросов по онтологиям конкретных предметных областей, конструкции которого затем транслируются в SQL для обеспечения взаимодействия с реляционными базами данных~\cite{saha_2016}.

Кроме того, спецификация лингвистической информации в виде онтологий помогает решать задачу автоматизированного создания онтологий на основе естественно-языковых текстов~\cite{SHAMSFARD200417}.

Создаются онтологии частных областей лингвистики: например, онтология пространственных выражений в естественных языках~\cite{BATEMAN20101027}, онтология темпоральных сущностей на основе естественного языка~\cite{Moens_1987}, онтологии конкретных естественных языков~\cite{Dobrov_2018}.
При использовании онтологий для обработки естественного языка необходимо "связать"{} концепты из онтологии с лексикой конкретного ЕЯ.
Для этого создаются различные расширения существующих языковых БД, таких как WordNet~\cite{wordnet}, VerbNet~\cite{verbnet} и FrameNet~\cite{framenet}, направленные на их использование совместно с онтологиями верхнего уровня (например,~\cite{pease_fellbaum_2010}).
Активные разработки идут в сфере создания онтологий словарного состава естественных языков, в результате которых появилось множество формализованных описаний лексики~\cite{matsukawa-yokota-1991-development},~\cite{calzolari_1991},~\cite{buitelaar2006linginfo},~\cite{Cimiano2007LexOntoAM},~\cite{buitelaar_2006}.
Так как распространенные базы данных лексики ЕЯ не являются онтологиями и не имеют достаточной степени формализации (например, WordNet), создаются онтологии, являющиеся своего рода "надстройкой"{} над такими базами данных, самой известной из которых является lemon~\cite{McCrae_2012}.

Многие из приведенных выше онтологий созданы с использованием технологии Semantic Web~\cite{sem_web}, который является внешней технологией по отношению к существующим решениям для обработки естественных языков, поэтому последним приходится обращаться к ней с помощью API и стандартизированных языков запросов (в частности, SPARQL)~\cite{Bouayad_2014}.

Стоит отметить, что несмотря на активное развитие в направлении применения онтологий для обработки ЕЯ, многие популярные NLP-библиотеки (например, NLTK~\cite{nltk} и spaCy~\cite{spacy}) в принципе не поддерживают использование онтологий, а большинство инструментов для разметки естественно-языковых текстов используют обычно свой формат, что требует использования специфичных для таких инструментов парсеров и конвертеров, чтобы данные можно было применить при решении каких-либо задач~\cite[p.~3]{Erekhinskaya2020TenWO}.

Таким образом, в настоящее время в данной области можно выделить следующие проблемы:
\begin{enumerate}
    \item Отсутствие унификации (стандартизации) приведенных выше решений приводит к существенным накладным расходам на их интеграцию и значительно усложняет построение различных систем с их использованием в силу большой трудоемкости их интеграции~\cite{Standard2021},\cite{GolenkovProblems2021}.
    \item Несмотря на то, что онтологии потенциально способствуют решению широкого круга задач в сфере обработки естественного языка, большинство движимых онтологиями NLP-систем сконцентрированы на решении специализированных задач (например, только генерации текста, только заполнения онтологии или только обеспечения поиска с помощью естественного языка).
    \item Создано довольно большое количество частных лингвистических онтологий, формализующих, однако, лишь некий подраздел предметной области лингвистики (в особенности лексики), что отчасти вытекает из предыдущего пункта. В то же время, существующие лингвистические онтологии верхнего уровня (например, OLiA) все равно не до конца решают проблему унификации, т.к. им требуется вводить промежуточный уровень для интеграции полученных в результате NLP-анализа данных с фрагментами онтологии.
\end{enumerate}

Так как используемый в технологии OSTIS язык -- sc-код -- обладает достаточной экспрессивностью для описания знаний любого вида, а сама технология нацелена на создание интероперабельных интеллектуальных систем нового поколения, естественно-языковые интерфейсы ostis-систем смогут справляться с широким кругом задач по обработке текстов на естественных языках -- будь то синтез естественно-языковых текстов в целом, ведение диалога в диалоговых системах, поиск с использованием естественного языка, выделение информации из текстов и т.п. При этом в то время как в текущем состоянии сферы обработки естественных языков данные классы задач выполняются зачастую специализированными средствами и требуют дополнительных затрат на обеспечение потенциальной совместимости с конкретными компьютерными системами, в рамках технологии OSTIS для их решения будет использоваться один универсальный язык смыслового представления знаний, на котором будут написаны как компоненты решателя задач, так и онтология языков и конкретных предметных областей, что позволит решить проблему интероперабельности.

Более того, онтология естественных языков, разработанная в рамках такой технологии, могла бы быть использована не только для решения прикладных задач по обработке естественного языка, но и для обеспечения интероперабельности данных, полученных в ходе лингвистических исследований, что было бы ценным вкладом в область теоретической лингвистики.

Наконец, онтологию естественных языков можно рассматривать в качестве подмножества онтологии языков вообще (как естественных, так и искусственных и формальных), чего не делают рассмотренные выше существующие онтологии. Это позволит концептуализировать естественный язык в одной системе с языками программирования и более тесно связать используемые в соответствующих предметных областях понятия для более эффективного решения задач по обработке естественного языка в интеллектуальных компьютерных системах.

Цель данной работы -- предложить базовые средства формального описания синтаксиса и денотационной семантики различных языков в виде фрагмента онтологии языков и информационных конструкций, который можно будет использовать при проектировании интеллектуальных компьютерных систем нового поколения.
%Конец анализа

\section{Формализация естественных языков}

Как уже говорилось выше, для использования достижений лингвистики при проектировании интеллектуальных компьютерных систем требуется представить полученные результаты в формальном виде. В данном разделе мы предложим формализацию основных лингвистических концептов, выполненную на формальном языке представления знаний -- sc-коде.

\begin{SCn}

    \scnheader{язык}
    \begin{scnrelfromset}{разбиение}
        \scnitem{естественный язык}
        \begin{scnindent}
            \scntext{пояснение}{Естественный язык представляет собой язык, который не был создан целенаправленно}
        \end{scnindent}
        \scnitem{искусственный язык}
        \begin{scnindent}
            \scntext{пояснение}{Искусственный язык представляет собой язык, специально разработанный для достижения определённых целей}
            \scnhaselement{Эсперанто}
            \scnhaselement{Python}
            \scnsuperset{сконструированный язык}
            \begin{scnindent}
                \scntext{пояснение}{Сконструированный язык представляет собой искусственный язык, предназначенный для общения людей}
                \scnhaselement{Эсперанто}
            \end{scnindent}
        \end{scnindent}
    \end{scnrelfromset}

    \scnsuperset{международный язык}
    \begin{scnindent}
        \scntext{пояснение}{Международный язык представляет собой естественный или искусственный язык, использующийся для общения людей из разных стран}
        \scnhaselement{Английский язык}
        \scnhaselement{Русский язык}
    \end{scnindent}

    \scnheader{плановый язык}
    \begin{scnreltoset}{пересечение}
        \scnitem{сконструированный язык}
        \scnitem{международный язык}
    \end{scnreltoset}

    \scnheader{язык общения}
    \begin{scnreltoset}{объединение}
        \scnitem{естественный язык}
        \scnitem{сконструированный язык}
    \end{scnreltoset}
    \scnhaselement{Английский язык}
    \scnhaselement{Русский язык}
    \scnhaselement{Эсперанто}
    \begin{scnreltoset}{объединение}
        \scnitem{корневой язык}
        \begin{scnindent}
            \scntext{пояснение}{Корневой язык представляет собой язык, для которого характерно полное отсутствие словоизменения и наличие грамматической значимости порядка слов, состоящих только из корня.}
            \scnhaselement{Английский язык}
        \end{scnindent}
        \scnitem{агглютинативный язык}
        \begin{scnindent}
            \scntext{пояснение}{Агглютинативный язык характеризуется развитой системой употребления суффиксов, приставок, добавляемых к неизменяемой основе слова, которые используются для выражения категорий числа, падежа, рода и др.}
            \scnhaselement{Английский язык}
        \end{scnindent}
        \scnitem{флективный язык}
        \begin{scnindent}
            \scntext{пояснение}{Для флективного языка характерно развитое употребление окончаний для выражения категорий рода, числа, падежа, сложная система склонения глаголов, чередование гласных в корне, а также строгое различение частей речи.}
            \scnhaselement{Русский язык}
        \end{scnindent}
        \scnitem{профлективный язык}
        \begin{scnindent}
            \scntext{пояснение}{Для профлективного языка характерны агглютинация (в случае именного словоизменения), флексия и чередование гласных (аблаут)(в случае глагольного словоизменения).}
        \end{scnindent}
    \end{scnreltoset}

\end{SCn}

\section{Формализация синтаксиса естественных языков}

Лексема -- минимальная единица языка, имеющая семантическую интерпретацию и обозначающая концепт, отражающий взгляд на мир некоторого языкового сообщества \scncite{glossary}.

Грамматическая категория -- система противопоставленных друг другу рядов грамматических форм с однородными значениями. В рамках нашей формализации предлагается представить грамматические категории как классы ролевых отношений, каждый из которых соответствует определенному грамматическому значению. Следует отметить, что приводится онтология основных грамматических категорий, часто встречающихся в естественных языках, а не всех возможных.

\begin{SCn}

    \scnheader{грамматическая категория}
    \scnhaselement{лицо}
    \begin{scnindent}
        \scnrelto{семейство подмножеств}{ролевое отношение}
        \scnhaselement{первое лицо\scnrolesign}
        \scnhaselement{второе лицо\scnrolesign}
        \scnhaselement{третье лицо\scnrolesign}
    \end{scnindent}
    \scnhaselement{число}
    \begin{scnindent}
        \scnrelto{семейство подмножеств}{ролевое отношение}
        \scnhaselement{единственное число\scnrolesign}
        \scnhaselement{множественное число\scnrolesign}
        \scnhaselement{двойственное число\scnrolesign}
        \scnhaselement{тройственное число\scnrolesign}
        \scnhaselement{паукальное число\scnrolesign}
    \end{scnindent}
    \scnhaselement{род}
    \begin{scnindent}
        \scnrelto{семейство подмножеств}{ролевое отношение}
        \scnhaselement{мужской род\scnrolesign}
        \scnhaselement{средний род\scnrolesign}
        \scnhaselement{женский род\scnrolesign}
    \end{scnindent}
    \scnhaselement{падеж}
    \begin{scnindent}
        \scnrelto{семейство подмножеств}{ролевое отношение}
        \scnhaselement{именительный падеж\scnrolesign}
        \scnhaselement{родительный падеж\scnrolesign}
        \scnhaselement{дательный падеж\scnrolesign}
        \scnhaselement{винительный падеж\scnrolesign}
        \scnhaselement{творительный падеж\scnrolesign}
        \scnhaselement{предложный падеж\scnrolesign}
        \scnhaselement{звательный падеж\scnrolesign}
        \scnhaselement{абсолютивный падеж\scnrolesign}
        \scnhaselement{эргативный падеж\scnrolesign}
    \end{scnindent}
    \scnhaselement{время}
    \begin{scnindent}
        \scnrelto{семейство подмножеств}{ролевое отношение}
        \scnhaselement{настоящее время\scnrolesign}
        \scnhaselement{прошедшее время\scnrolesign}
        \scnhaselement{будущее время\scnrolesign}
    \end{scnindent}
    \scnhaselement{наклонение}
    \begin{scnindent}
        \scnrelto{семейство подмножеств}{ролевое отношение}
        \scnhaselement{изъявительное наклонение\scnrolesign}
        \scnhaselement{повелительное наклонение\scnrolesign}
        \scnhaselement{сослагательное наклонение\scnrolesign}
        \scnhaselement{условное наклонение\scnrolesign}
    \end{scnindent}
    \scnhaselement{залог}
    \begin{scnindent}
        \scnrelto{семейство подмножеств}{ролевое отношение}
        \scnhaselement{действительный залог\scnrolesign}
        \scnhaselement{страдательный залог\scnrolesign}
        \scnhaselement{средний залог\scnrolesign}
        \scnhaselement{возвратный залог\scnrolesign}
        \scnhaselement{взаимный залог\scnrolesign}
    \end{scnindent}
    \scnhaselement{вид}
    \begin{scnindent}
        \scnrelto{семейство подмножеств}{ролевое отношение}
        \scnhaselement{совершенный вид\scnrolesign}
        \scnhaselement{несовершенный вид\scnrolesign}
        \scnhaselement{общий вид\scnrolesign}
        \scnhaselement{прогрессивный вид\scnrolesign}
        \scnhaselement{перфектный вид\scnrolesign}
    \end{scnindent}
    \scnhaselement{степень сравнения}
    \begin{scnindent}
        \scnrelto{семейство подмножеств}{ролевое отношение}
        \scnhaselement{положительная степень сравнения\scnrolesign}
        \scnhaselement{сравнительная степень сравнения\scnrolesign}
        \scnhaselement{превосходная степень сравнения\scnrolesign}
    \end{scnindent}
\end{SCn}

Пример формализации части приведенных выше отношений на языке sc.g приведен на рисунке~\ref{fig:lexeme_example}.

\begin{figure}[h]
    \centering
    \includegraphics[width=0.4\textwidth]{images/part2/chapter_lang/lexeme_example.png}
    \caption{Пример спецификации лексемы в базе знаний.}
    \label{fig:lexeme_example}
\end{figure}

Часть речи -- категория, представляющая собой класс синтаксически эквивалентных знаков ЕЯ.

\begin{SCn}

    \scnheader{часть речи}
    \scnrelto{семейство подмножеств}{лексема}
    \scnhaselement{существительное}
    \scnhaselement{прилагательное}
    \scnhaselement{глагол}
    \scnhaselement{наречие}
    \scnhaselement{предлог}
    \scnhaselement{комплементатор}
    \scnhaselement{вспомогательный глагол}
    \scnhaselement{детерминант}

\end{SCn}

\textit{морфологическая парадигма*} - бинарное ориентированное отношение, связывающее лексему и множество ее словоформ.

Словоформа -- подмножество лексемы, которому принадлежат все вхождения лексемы с определенными грамматическими значениями. В рамках нашей онтологии словоформа понимается несколько иначе, чем принято в лингвистике, так как все вхождения лексемы в технологии OSTIS являются файлами.

При формализации синтаксиса в основном использовались стандартные положения генеративной грамматики \scncite{adger2003core}, \scncite{jackendoff1977x}, \scncite{haegeman1994introduction}, \scncite{carnie2012syntax}.

Дистрибуция знака –- это подмножество синтаксических правил, в которые входит данный знак.

Составляющая -- элемент множества C подмножеств кортежа вхождений лексем S, которое содержит в качестве элементов как сам S, так и все вхождения лексем в S, таким образом, что любые два подмножества, входящие в C, либо не пересекаются, либо одно из них включается в другое.

Непосредственно составляющая --  есть множество составляющих S, в которое входят составляющие A и B. В является непосредственно составляющей А если и только если В является подмножеством А и нет такой составляющей С, которая является подмножеством А и подмножеством которой является В.

Элементарная составляющая -- элемент кортежа вхождений лексем L, являющихся непосредственно составляющими множества составляющих C и не имеющих дочерних составляющих.

Синтаксическая группа -- класс составляющих, в который входят составляющие с вершинами, принадлежащими к одной части речи. Синтаксические группы представляют собой либо синглетон (минимально включают в себя вершину), либо упорядоченную пару, состояющую из вершины и другой синтаксической группы.

Вершина –- составляющая, дистрибуция которой совпадает с дистрибуцией всей синтаксической группы.

\begin{SCn}

    \scnheader{составляющая}
    \begin{scnrelfromset}{разбиение}
        \scnitem{синтаксическая группа}
        \scnitem{вершина}
    \end{scnrelfromset}

    \scnheader{синтаксическая группа}
% максимальная, промеж и вершины, вывести классы из примера как их пересечения
    \begin{scnrelfromset}{разбиение}
        \scnitem{именная группа}
        \begin{scnindent}
            \scntext{пояснение}{\textit{Именная группа} -- синтаксическая группа, вершиной которой является существительное.}
        \end{scnindent}
        \scnitem{глагольная группа}
        \begin{scnindent}
            \scntext{пояснение}{\textit{Глагольная группа} -- синтаксическая группа, вершиной которой является глагол.}
        \end{scnindent}
        \scnitem{группа прилагательного}
        \begin{scnindent}
            \scntext{пояснение}{\textit{Группа прилагательного} -- синтаксическая группа, вершиной которой является прилагательное.}
        \end{scnindent}
        \scnitem{наречная группа}
        \begin{scnindent}
            \scntext{пояснение}{\textit{Наречная группа} -- синтаксическая группа, вершиной которой является наречие.}
        \end{scnindent}
        \scnitem{предложная группа}
        \begin{scnindent}
            \scntext{пояснение}{\textit{Предложная группа} -- синтаксическая группа, вершиной которой является предлог.}
        \end{scnindent}
        \scnitem{группа комплементатора}
        \begin{scnindent}
            \scntext{пояснение}{\textit{Группа комплементатора} -- синтаксическая группа, вершиной которой является комплементатор.}
        \end{scnindent}
        \scnitem{временная группа}
        \begin{scnindent}
            \scntext{пояснение}{\textit{Временная группа} -- синтаксическая группа, вершиной которой является вспомогательный либо модальный глагол.}
        \end{scnindent}
        \scnitem{группа детерминанта}
        \begin{scnindent}
            \scntext{пояснение}{\textit{Группа детерминанта} -- синтаксическая группа, вершиной которой является детерминант.}
        \end{scnindent}
    \end{scnrelfromset}
    \begin{scnrelfromset}{разбиение}
        \scnitem{максимальная проекция вершины}
        \scnitem{промежуточная проекция вершины}
    \end{scnrelfromset}

\end{SCn}

При этом для упрощения могут быть введены более узкие классы, являющиеся пересечением приведенных выше, например \textit{максимальная проекция вершины группы детерминанта}.

\begin{SCn}

    \scnheader{максимальная проекция вершины группы детерминанта}
    \begin{scnreltoset}{пересечение}
        \scnitem{группа детерминанта}
        \scnitem{максимальная проекция вершины}
    \end{scnreltoset}

\end{SCn}

Пример синтаксической структуры предложения, записанный с применением введенных выше понятий представлен на рисунке~\ref{pic_tree}.

\begin{figure*}[h]
    \centering
    \includegraphics[width=\textwidth]{images/part2/chapter_lang/syntactic.png}
    \caption{Пример синтаксической структуры предложения.}
    \label{pic_tree}
\end{figure*}

Структуры синтаксических групп не являются произвольными -- элементы внутри группы могут граничить только с определенными множествами элементов.
Ниже приводятся возможные структуры синтаксических групп.
Знак "->"{} следует читать как "состоит из"{}.
В скобках указаны опциональные элементы.

Группа детерминанта:
\begin{textitemize}
    \item Максимальная проекция вершины группы детерминанта -> (Максимальная проекция вершины группы детерминанта) Промежуточная проекция вершины группы детерминанта
    \item Промежуточная проекция вершины группы детерминанта -> Вершина группы детерминанта (Максимальная проекция вершины именной группы)
\end{textitemize}

Именная группа:
\begin{textitemize}
    \item Максимальная проекция вершины именной группы -> (Максимальная проекция вершины группы детерминанта) Промежуточная проекция вершины именной группы
    \item Промежуточная проекция вершины именной группы -> (Максимальная проекция вершины группы прилагательного) Промежуточная проекция вершины именной группы ИЛИ Промежуточная проекция вершины именной группы (Максимальная проекция вершины предложной группы)
    \item Промежуточная проекция вершины именной группы -> Вершина именной группы (Максимальная проекция вершины предложной группы)
\end{textitemize}

Глагольная группа:
\begin{textitemize}
    \item Максимальная проекция вершины глагольной группы -> Промежуточная проекция вершины глагольной группы
    \item Промежуточная проекция вершины глагольной группы -> Промежуточная проекция вершины глагольной группы (Максимальная проекция вершины предложной группы) ИЛИ Промежуточная проекция вершины глагольной группы (Максимальная проекция вершины наречной группы)
    \item Промежуточная проекция вершины глагольной группы -> Вершина глагольной группы (Максимальная проекция вершины именной группы)
\end{textitemize}

Наречная группа:
\begin{textitemize}
    \item Максимальная проекция вершины наречной группы -> Промежуточная проекция вершины наречной группы
    \item Промежуточная проекция вершины наречной группы -> (Максимальная проекция вершины наречной группы) Промежуточная проекция вершины наречной группы
    \item Промежуточная проекция вершины наречной группы -> Вершина наречной группы (Максимальная проекция вершины предложной группы)
\end{textitemize}

Группа прилагательного:
\begin{textitemize}
    \item Максимальная проекция вершины группы прилагательного -> Промежуточная проекция вершины группы прилагательного
    \item Промежуточная проекция вершины группы прилагательного -> (Максимальная проекция вершины наречной группы) + Промежуточная проекция вершины группы прилагательного
    \item Промежуточная проекция вершины группы прилагательного -> Вершина группы прилагательного (Максимальная проекция вершины предложной группы)
\end{textitemize}

Предложная группа:
\begin{textitemize}
    \item Максимальная проекция вершины предложной группы -> Промежуточная проекция вершины предложной группы
    \item Промежуточная проекция вершины предложной группы -> Промежуточная проекция вершины предложной группы (Максимальная проекция вершины предложной группы) ИЛИ (Максимальная проекция вершины наречной группы) Промежуточная проекция вершины предложной группы
    \item Промежуточная проекция вершины предложной группы -> Вершина предложной группы (Максимальная проекция вершины именной группы)
\end{textitemize}

Временная группа:
\begin{textitemize}
    \item Максимальная проекция вершины временной группы -> (Максимальная проекция вершины группы детерминанта) Промежуточная проекция вершины временной группы
    \item Промежуточная проекция вершины временной группы -> Вершина временной группы (Максимальная проекция вершины глагольной группы)
\end{textitemize}

Группа комплементатора:
\begin{textitemize}
    \item Максимальная проекция вершины группы комплементатора -> (Максимальная проекция вершины некоторой синтаксической группы) Промежуточная проекция вершины группы комплементатора
    \item Промежуточная проекция вершины группы комплементатора -> Вершина группы комплементатора Максимальная проекция вершины временной группы
\end{textitemize}

В формальном виде данные правила можно представить следующим образом (см. рисунок~\ref{fig:pic_tree_structure_rule}).

\begin{figure*}[h]
    \centering
    \includegraphics[width=0.8\textwidth]{images/part2/chapter_lang/tree_structure_rule.png}
    \caption{Пример правила структуры синтаксической группы.}
    \label{fig:pic_tree_structure_rule}
\end{figure*}

Комплемент -- синтаксическая группа, являющаяся сестрой вершины. Сестрами считаются составляющие, являющиеся непосредственно составляющими одной и той же составляющей.

Адъюнкт -- синтаксическая группа, являющаяся дочерью (непосредственно составляющей) промежуточной проекции и сестрой промежуточной проекции вершины той же синтаксической группы.

Спецификатор -- синтаксическая группа, являющейся дочерью максимальной проекции и сестрой промежуточной проекции.

Приведенные выше правила структуры синтаксических групп можно обобщить и свести к трем более абстрактным.

Правило спецификатора: XP -> (YP) X'

Правило адъюнкта: X' -> X' (ZP) | X' -> (ZP) X'

Правило комплемента: X' -> X (WP)

Формальное представление данных правил аналогично приведенному на рисунке~\ref{fig:pic_tree_structure_rule}.

\section{Формализация денотационной семантики естественных языков}

Денотационная семантика языка специфицирует интерпретацию элементов синтаксиса данного языка и представляет собой множество формул, описывающих то, каким образом знаковым конструкциям языка ставятся в соответствие обозначаемые ими сущности и конфигурации отношений между этими сущностями.

Денотационная семантика естественных языков должна обладать свойством композициональности -- т.е. интерпретация всего высказывания должна выводиться из интерпретации отдельных его частей. Таким образом, необходимо предоставить формальное описание интерпретации элементов синтаксиса ЕЯ, представленных в предыдущем разделе, а также описание правил совмещения интерпретации отдельных элементов для получения смысла всего высказывания.

В данной главе мы предлагаем вариант формализации денотационной семантики естественных языков в рамках технологии OSTIS, для составления которой использовались стандартные положения формальной семантики \scncite{heim1998semantics}, \scncite{Winter+2016}, \scncite{portner2008formal}.

Рассмотрим примеры правил, реализующих денотационную семантику языка. Приведенные ниже правила должны применяться последовательно и позволяют получить смысл текста естественного языка по его синтаксической структуре, "поднимаясь"{} по дереву составляющих от вершин к максимальным проекциям.

На рисунке~\textit{\nameref{fig:d_sem_1}} приведено правило, по которому происходит интерпретация вершин именной группы и группы прилагательного.
Смыслом таких вершин является класс, например: прилагательному "черный"{} соответствует множество черных объектов, а существительному "кот"{} -- множество котов.

\begin{figure}[H]
    \centering
    \includegraphics[width=0.8\textwidth]{images/part2/chapter_lang/d_sem_1}
    \caption{Правило интерпретации вершины группы прилагательного и вершины именной группы}
    \label{fig:d_sem_1}
\end{figure}
%в этом правиле мы интерпретируем вершины групп прилагательного и существительного - т.е. отдельные "слова"{}-прилагательные и существительные, которые у нас соответствуют классам

На рисунке~\textit{\nameref{fig:d_sem_2}} приведено правило, по которому происходит интерпретация именной группы, максимальная проекция которой включается в себя также группу прилагательного.
Как говорилось выше, для применения данного правила необходимо предварительное применение правила, представленного на рисунке~\textit{\nameref{fig:d_sem_1}}.
Смыслом таких конструкций является класс, являющийся результатом пересечения классов, полученных в результате интерпретации вершин групп прилагательного и именной группы по отдельности. Например: "черный кот"{} -- множество черных котов, пересечение множества котов и черных объектов. %ну все, теперь включаю мою кошку в соавторы

\begin{figure*}[h]
    \centering
    \includegraphics[width=0.8\textwidth]{images/part2/chapter_lang/d_sem_2.png}
    \caption{Правило интерпретации максимальной проекции вершины именной группы}
    \label{fig:d_sem_2}
\end{figure*}
%тут мы комбинируем смыслы прилагательного и существительного, которые входят в одну именную группу

На рисунке~\textit{\nameref{fig:d_sem_3}} приведено правило, по которому происходит интерпретация глагольной группы. Необходимость включения в посылку правила всей ветки глагольной группы объясняется ее необходимостью для определения типа глагола -- данное правило предназначено для интерпретации непереходных глаголов. Смыслом такой конструкции является класс действий.

\begin{figure*}[h]
    \centering
    \includegraphics[width=0.8\textwidth]{images/part2/chapter_lang/d_sem_3.png}
    \caption{Правило интерпретации максимальной проекции вершины глагольной группы, содержащей непереходный глагол}
    \label{fig:d_sem_3}
\end{figure*}
%тут мы задаем интерпретацию всей глагольной группы (макс проекции) только для непереходных глаголов. написать, что смотрим по всей структуре группы целиком, потому что для того, чтобы отличить непереходный от переходного нам нужна вся ветка глагольной группы в дереве целиком

На рисунке~\textit{\nameref{d_sem_4}} приведено правило, по которому происходит интерпретация группы детерминанта с неопределенным артиклем. Смыслом такой конструкции является существование элемента класса, являющегося смыслом входящей в состав данной группы детерминанта именной группы.

\begin{figure*}[h]
    \centering
    \includegraphics[width=0.8\textwidth]{images/part2/chapter_lang/d_sem_4.png}
    \caption{Правило интерпретации максимальной проекции вершины группы детерминанта}
    \label{d_sem_4}
\end{figure*}
%тут задается интерпретация сочетания именной группы с артиклем (в данном случае неопределенным)

На рисунке~\textit{\nameref{fig:d_sem_5}} приведено правило, по которому происходит интерпретация промежуточной проекции вершины временной группы, состоящей из вспомогательного глагола и полнозначного глагола. Вспомогательный глагол в данном случае задет класс действий по времени (является ли оно запланированным, выполняемым, уже выполненным и т.д.).

\begin{figure*}[h]
    \centering
    \includegraphics[width=0.8\textwidth]{images/part2/chapter_lang/d_sem_5.png}
    \caption{Правило интерпретации промежуточной проекции вершины временной группы}
    \label{fig:d_sem_5}
\end{figure*}
%тут задаем интерпретацию сочетания вспомогательного глагола и основного глагола. вспомогательный у нас соответствует классу действий по времени

На рисунке~\textit{\nameref{fig:d_sem_6}} приведено правило, по которому происходит интерпретация максимальной проекции вершины временной группы на основе полученного на предыдущем шаге смысла промежуточной проекции вершины временной группы и смысла максимальной проекции группы детерминанта.

\begin{figure*}[h]
    \centering
    \includegraphics[width=0.8\textwidth]{images/part2/chapter_lang/d_sem_6.png}
    \caption{Правило интерпретации максимальной проекции вершины временной группы}
    \label{fig:d_sem_6}
\end{figure*}
%тут задаем интерпретацию для аргументной структуры непереходного глагола (сочетания подлежащего с непереходным глаголом)

На рисунке~\textit{\nameref{fig:d_sem_7}} приведено правило, по которому происходит интерпретация максимальной проекции вершины группы комплементатора на основе полученных на предыдущих шагах смыслов более частных конструкций. Данным правилом задается интерпретация предложения с переходным глаголом.

\begin{figure*}[h]
    \centering
    \includegraphics[width=0.8\textwidth]{images/part2/chapter_lang/d_sem_7.png}
    \caption{Правило интерпретации предложения с переходным глаголом}
    \label{fig:d_sem_7}
\end{figure*}
%тут задаем интерпретацию аргументной структуры переходного глагола (сочетания переходного глагола с его аргументами - подлажещим и дополнением)

%\input{author/references}
%\begin{partbacktext}
\part{Многоагентные модели решателей задач интеллектуальных компьютерных систем нового поколения}

\begin{SCn}
	\scnidtf{Часть 3. Многоагентные модели обработки различных видов знаний в интеллектуальных компьютерных систем нового поколения}
	\scnidtf{Часть 3. Принципы конвергенции и интеграции различных методов и моделей решения задач в многоагентных интеллектуальных компьютерных систем нового поколения}
	\scnidtf{Часть 3. Представление и обработка различных методов и моделей решения задач в многоагентных ostis-системах}
	
	\scntext{аннотация}{Уточнение понятий действия, задачи, метода, средства, навыка и технологии. Принципы программирования в ostis-системах. Принципы разработки различных видов решателей задач в многоагентных ostis-системах. Ситуационное управление в многоагентных ostis-системах. Базовый язык программирования для ostis-систем. Принципы конвергенции и интеграции различных моделей решения задач в ostis-системах.}
	
	\bigskip
	
	\begin{scnrelfromlist}{подраздел}
		\scnitem{Глава~\ref{chapter_library}~\nameref{chapter_library}}
		\scnitem{Глава~\ref{chapter_kb_design}~\nameref{chapter_kb_design}}
		\scnitem{Глава~\ref{chapter_ps_design}~\nameref{chapter_ps_design}}
		\scnitem{Глава~\ref{chapter_ui_design}~\nameref{chapter_ui_design}}
	\end{scnrelfromlist}
\end{SCn}

\end{partbacktext}

\chapter{Формализация понятий действия, задачи, метода, средства, навыка и технологии}
\chapauthortoc{Шункевич Д.В.\\Ковалёв М.В.\\Никифоров С.А.}
\label{chapter_actions}

\vspace{-7\baselineskip}

\begin{SCn}
	\begin{scnrelfromlist}{автор}
		\scnitem{Шункевич Д.В.}
		\scnitem{Ковалёв М.В.}
		\scnitem{Никифоров C.F.}
	\end{scnrelfromlist}

	\bigskip

	\scntext{аннотация}{В главе уточнена формальная трактовка таких понятий, как действие, задача, класс действий, класс задач, метод, навык, что в совокупности позволило определить на их основе понятие модели решения задач.}

	\bigskip

	\begin{scnrelfromlist}{подраздел}
		\scnitem{\ref{sec_action_concept}\nameref{sec_action_concept}}
		\scnitem{\ref{sec_problem_concept}\nameref{sec_problem_concept}}
		\scnitem{\ref{sec_3}~Субъектно-объектные спецификации воздействий и действий}
		\scnitem{\ref{sec_4}~Формализация понятий сложного действия, класса задач и метода}
		\scnitem{\ref{sec_5}~Формализация понятий навыка, класса методов и модели решения задач}
		\scnitem{\ref{sec_6}~Формализация понятий деятельности, вида деятельности и технологии}
	\end{scnrelfromlist}

	\bigskip

	\begin{scnrelfromlist}{ключевое понятие}
		\scnitem{воздействие}
		\scnitem{действие}
		\scnitem{задача}
		\scnitem{класс действий}
		\scnitem{класс задач}
		\scnitem{метод}
		\scnitem{язык представления методов}
		\scnitem{модель решения задач}
		\scnitem{навык}
		\scnitem{деятельность}
		\scnitem{вид деятельности}
		\scnitem{технология}
	\end{scnrelfromlist}

	\bigskip

	\begin{scnrelfromlist}{библиографическая ссылка}
		\scnitem{\scncite{Standard2021}}
		\scnitem{\scncite{Tuzov1986}}
		\scnitem{\scncite{Martynov1984}}
		\scnitem{\scncite{Ikeda1998}}
		\scnitem{\scncite{Studer1996}}
		\scnitem{\scncite{Benjamins1999}}
		\scnitem{\scncite{Chandrasekaran1999}}
		\scnitem{\scncite{Fensel1998Reuse}}
		\scnitem{\scncite{Kemke2001}}
		\scnitem{\scncite{Tu1995}}
		\scnitem{\scncite{Trypuz2007}}
		\scnitem{\scncite{Fang2019}}
		\scnitem{\scncite{McBride2021}}
		\scnitem{\scncite{Crowther2020}}
		\scnitem{\scncite{McCann1998}}
		\scnitem{\scncite{Yan2014}}
		\scnitem{\scncite{Crubezy2004}}
	\end{scnrelfromlist}

\end{SCn}

\section*{Введение в Главу \ref{chapter_actions}}
Возможности решателя задач интеллектуальной системы в значительной степени определяются качеством ее базы знаний. Другими словами, при разработке решателей задач необходимо описывать не только \textit{операционную семантику} решателя, то есть семейство интерпретаторов соответствующих моделей решения задач, но и \textit{декларативную семантику} модели решения задач, то есть собственно тексты программ (не программ низкоуровневых агентов, а программ более высокого уровня, интерпретируемых соответствующим набором агентов), логические утверждения, конкретные конфигурации искусственных нейронных сетей и т.д.

В рамках данной главы формально уточняются в рамках соответствующего набора онтологий такие понятия, как \textit{действие}, \textit{задача}, \textit{модель решения задач}, \textit{метод}, \textit{навык} и другие, на основе которых в \textit{Главе \ref{chapter_situation_management} \nameref{chapter_situation_management}} уточняется собственно модель гибридного \textit{решателя задач ostis-системы}.

%TODO Уточнить, будет ли где-то ранее говориться про sc-агенты и многоагентный подход
Разработка указанного семейства онтологий позволяет:
\begin{textitemize}
	\item явно связать \textit{класс задач} и способ (метод) решения задач данного класса;
	\item это, в свою очередь, позволит накапливать более сложные компоненты решателей задач и еще больше упростить их интеграцию, поскольку вместе с коллективом sc-агентов в соответствующий компонент также будут входить необходимые фрагменты базы знаний, априори согласованные с указанным коллективом sc-агентов;
	\item это, в свою очередь, позволит сделать средства автоматизации разработки решателей задач более интеллектуальными, в частности, позволит автоматизировать процесс подбора компонентов решателя на основе спецификации классов задач, которые должна уметь решать проектируемая интеллектуальная система;
	\item в дальнейшем это позволит интеллектуальной системе самостоятельно обращаться в библиотеку компонентов решателей задач (см. \ref{ps_components_section}~\nameref{ps_components_section}) и подбирать компоненты, исходя из новых классов задач, с которыми столкнулась система, то есть позволит интеллектуальной системе самостоятельно изучать новые \textit{навыки};
	\item с другой стороны, такой подход позволит интеллектуальной системе самостоятельно подобрать комбинацию моделей решения задач для решения задач определенного класса (точнее говоря, поскольку в основу решателя положен многоагентный подход, то коллектив sc-агентов, интерпретирующих различные модели решения задач, сможет лучше определить, какие именно из sc-агентов и в каком порядке должны работать при решении конкретной комплексной задачи).
\end{textitemize}


\section{Формализация понятия действия}
\label{sec_action_concept}

\begin{SCn}
	\begin{scnrelfromlist}{ключевое понятие}
		\scnitem{воздействие}
		\scnitem{действие}
		\scnitem{информационное действие}
		\scnitem{поведенческое действие}
		\scnitem{эффекторное действие}
		\scnitem{рецепторное действие}
		\scnitem{элементарное действие}
		\scnitem{сложное действие}
	\end{scnrelfromlist}

	\begin{scnrelfromlist}{ключевое отношение}
		\scnitem{субъект\scnrolesign}
		\scnitem{объект\scnrolesign}
		\scnitem{цель*}
	\end{scnrelfromlist}
\end{SCn}

Прежде чем говорить о моделях решения задач и решателя задач, необходимо формально уточнить понятие задачи и понятие действия, направленного на решение той или иной задачи или ее подзадач.

В рамках \textit{Технологии OSTIS} задачу будем трактовать как формальную спецификацию некоторого действия, поэтому целесообразно вначале уточнить понятие \textbf{\textit{действия}}, которое определяется через понятие \textbf{\textit{воздействия}}. Рассмотрим спецификацию понятия \textit{воздействие} в SCn-коде.

\begin{SCn}
	\scnheader{воздействие}
	\scnidtf{\textit{процесс} воздействия одной сущности (или некоторого множества \textit{сущностей}) на другую \textit{сущность} (или на некоторое множество других \textit{сущностей})}
	\scnidtf{\textit{процесс}, в котором могут быть явно выделены хотя бы одна воздействующая сущность (\textit{субъект\scnrolesign}) и хотя бы одна \textit{сущность}, на которую осуществляется воздействие (\textit{объект\scnrolesign})}
	\scnidtf{акция}
	\scnsubset{процесс}
	\begin{scnindent}
		\scnidtf{динамическая структура}
	\end{scnindent}

	\scnheader{субъект\scnrolesign}
	\scnidtf{воздействующая сущность\scnrolesign}
	\scnidtf{сущность, создающая \textit{причину} изменений другой сущности (объекта)\scnrolesign}

	\scnheader{объект\scnrolesign}
	\scnidtf{воздействуемая сущность\scnrolesign}
	\scnidtf{сущность, являющаяся в рамках заданного воздействия исходным условием (аргументом), необходимым для выполнения этого воздействия\scnrolesign}
\end{SCn}

Каждому \textit{воздействию} может быть поставлен в соответствие (1) некоторый \textit{субъект\scnrolesign}, т.е. сущность, осуществляющая \textit{воздействие} (в частности, это может быть некоторое физическое поле), и (2) некоторый \textit{объект\scnrolesign}, т.е. сущность, на которую воздействие направлено. Если \textit{воздействие} связано с \textit{материальной сущностью}, то его объектом является либо сама эта \textit{материальная сущность}, либо некоторая ее пространственная часть.

Поскольку \textit{воздействия} являются частным видом \textit{процессов}, воздействиями наследуются все свойства \textit{процессов}. В частности, используются все \textit{параметры}, заданные на множестве \textit{процессов}, например, \textit{длительность*}, \textit{момент начала процесса*}, \textit{момент завершения процесса\scnsupergroupsign}. Подробнее эти отношения описываются в \textit{Главе \ref{chapter_top_ontologies} \nameref{chapter_top_ontologies}}.

Так же, так как воздействие является процессом и, соответственно, представляет собой \textit{динамическую структуру}, то и знак \textit{субъекта воздействия\scnrolesign}, и знак \textit{объекта воздействия\scnrolesign} являются элементами данной структуры. В связи с этим можно рассматривать отношения \textit{субъект воздействия\scnrolesign} и \textit{объект воздействия\scnrolesign} как \textit{ролевые отношения}. Данный факт не запрещает вводить аналогичные \textit{неролевые отношения}, однако это нецелесообразно.

Рассмотрим спецификацию понятия \textit{действие} в SCn-коде.
\begin{SCn}
	\scnheader{действие}
	\scnidtf{\textit{воздействие}, в котором \textit{субъект\scnrolesign} осуществляет \textit{воздействие} целенаправленно, т.е. в соответствии с некоторой \textit{целью*}}
	\scnidtf{целенаправленное воздействие, выполняемое одним или несколькими субъектами (кибернетическими системами) с возможным применением некоторых инструментов}
	\scnidtf{акт}
	\scnidtf{операция}
	\scnidtf{осознанное воздействие}
	\scnidtf{активное воздействие}
	\scnsubset{воздействие}
	\begin{scnindent}
		\scnidtf{процесс, в котором могут быть явно выделены хотя бы одна воздействующая сущность (субъект воздействия') и хотя бы одна сущность, на которую осуществляется воздействие (объект воздействия')}
		\scnsubset{процесс}
	\end{scnindent}

	\scnidtf{целенаправленный ("осознанный"{}) процесс, выполняемый (управляемый, реализуемый) неким субъектом}
	\scnidtf{работа}
	\scnidtf{процесс решения некоторой задачи}
	\scnidtf{процесс достижения некоторой цели}
	\scnidtf{целостный фрагмент некоторой деятельности}
	\scnidtf{целенаправленный процесс, управляемый некоторым субъектом}
	\scnidtf{процесс выполнения некоторого действия некоторым субъектом (исполнителем) над некоторыми объектами}
%
%	\scnsuperset{элементарное действие}
%	\begin{scnindent}
%		\scnidtf{действие, выполнение которого не требует его декомпозиции на множество поддействий (частных действий, действий более низкого уровня)}
%		\scnexplanation{Элементарное действие выполняется одним индивидуальным субъектом и является либо элементарным действием, выполняемым в памяти этого субъекта (элементарным действием его "процессора"{}), либо элементарным действием одного из его эффекторов.}
%	\end{scnindent}
%	\scnsuperset{сложное действие}
%	\begin{scnsubdividing}
%		\scnitem{действие, выполняемое кибернетической системой в собственной памяти}
%		\scnitem{действие, выполняемое кибернетической системой в своей внешней среде}
%		\scnitem{действие, выполняемое кибернетической системой над своей физической оболочкой}
%	\end{scnsubdividing}

	\scnheader{цель*}
	\scnidtf{целевая ситуация*}
	\scnsubset{спецификация*}
	\scnidtf{описание того, что требуется получить (какая ситуация должна быть достигнута) в результате выполнения заданного (специфицируемого) действия*}
\end{SCn}

Каждое \textit{действие}, выполняемое тем или иным \textit{субъектом}, трактуется как процесс решения некоторой задачи, т.е. процесс достижения заданной \textit{цели*} в заданных условиях, и, следовательно, выполняется целеноправленно. При этом явное указание \textit{действия} и его связи с конкретной задачей может не всегда присутствовать в памяти. Некоторые \textit{задачи} могут решаться определенными субъектами перманентно, например, оптимизация \textit{базы знаний}, поиск некорректностей и т.д. и для подобных задач не всегда есть необходимость явно вводить \textit{структуру}, являющуюся формулировкой \textit{задачи}.

Каждое \textit{действие} может обозначать сколь угодно малое преобразование, осуществляемое во внешней среде либо в памяти некоторой \textit{кибернетической системы}, однако в памяти явно вводятся знаки только тех \textit{действий}, для которых есть небходимость явно хранить в памяти их спецификацию в течение некоторого времени.

%TODO подумать, нужно ли в этом параграфе
При выполнении \textit{действия} можно выделить этапы:
\begin{scnitemize}
	\item{построение \textit{плана действия}, декомпозиция (детализация) действия в виде системы его \textit{поддействий};}
	\item{выполнение построенного плана \textit{действия}}
\end{scnitemize}

Класс действий имеет разбиение по следующим признакам:
\begin{scnitemize}
	\item{место выполнения действия;}
	\item{функциональная сложность действия;}
	\item{многоагентность выполнения действия;}
	\item{текущее состояние действия;}
\end{scnitemize}

Далее рассмотрим разбиение по каждому признаку.

\begin{SCn}
	\scnheader{действие}
	\scnrelfrom{разбиение}{Типология классов действий по признаку места выполнения действия}
	\begin{scnindent}
		\begin{scneqtoset}
			\scnitem{действие, выполняемое в памяти субъекта действия}
			\begin{scnindent}
				\scnidtf{информационное действие}
				\scnidtf{действие, выполняемое в памяти}
				\scnidtf{действие кибернетической системы, направленное на обработку информации, хранимой в её памяти}
			\end{scnindent}
			\scnitem{действие, выполняемое во внешней среде субъекта действия}
			\begin{scnindent}
				\scnidtf{поведенчиское действие}
			\end{scnindent}
			\scnitem{рецепторное действие}
			\begin{scnindent}
				\scnidtf{сенсорное действие}
			\end{scnindent}
			\scnitem{эффекторное действие}
		\end{scneqtoset}
	\end{scnindent}
\end{SCn}

Результатом выполнения \textbf{\textit{действия, выполняемого в памяти субъекта действия}} в общем случае является некоторое новой состояние памяти информационной системы (не обязательно \textit{sc-памяти}), достигнутое исключительно путем преобразования информации, хранящейся в памяти системы, то есть либо посредством генерации новых знаний на основе уже имеющихся, либо посредством удаления знаний, по каким-либо причинам ставших ненужными. Следует отметить, что если речь идет об изменении состояния \textit{sc-памяти}, то любое преобразование информации можно свести к ряду элементарных действий по генерации, удалению или изменению инцидентности \textit{sc-элементов} относительно друг друга.

В случае \textbf{\textit{действия, выполняемого во внешней среде субъекта действия}} результатом его выполнения будет новое состояние внешней среды. Очень важно отметить, что под внешней средой в данном случае понимаются также и компоненты системы, внешние с точки зрения памяти, то есть не являющиеся хранимыми в ней информационными конструкциями. К таким компонентам можно отнести, например, различные манипуляторы и прочие средства воздействия системы на внешний мир, то есть к поведенческим задачам можно отнести изменение состояния механической конечности робота или непосредственно вывод некоторой информации на экран для восприятия пользователем.

Каждое \textbf{\textit{рецепторное действие}} обозначает класс \textit{действий}, которые осуществляют преобразования в памяти субъекта действия под воздействием внешней среды.

Соответственно, каждое \textbf{\textit{эффекторное действие}} обозначает класс \textit{действий}, которые осуществляют преобразования внешней среды под воздействием памяти субъекта воздействия.

Так же действия могут разбиваться по признакой их функциональной сложности.
\begin{SCn}
	\scnheader{действие}
	\scnrelfrom{разбиение}{Типология классов действий по признаку функциональной сложности действия}
	\begin{scnindent}
		\begin{scneqtoset}
			\scnitem{атомарное действие}
			\begin{scnindent}
				\scnidtf{элементарное действие}
			\end{scnindent}
			\scnitem{неатомарное действие}
			\begin{scnindent}
				\scnidtf{неэлементарное действие}
				\scnidtf{сложное действие}
				\begin{scnrelfromset}{покрытие}
					\scnitem{легко выполнимое неатомарное действие}
					\scnitem{трудно выполнимое сложное действие}
					\begin{scnindent}
						\scnidtf{интеллектуальное действие}
					\end{scnindent}
				\end{scnrelfromset}
			\end{scnindent}
		\end{scneqtoset}
	\end{scnindent}
\end{SCn}

Разберем подробнее каждый элемент приведенной иерархии.
\textbf{\textit{Атомарное действие}}  выполняется одним индивидуальным субъектом и не включает в себя выполнение каких-либо дочерних действий.

Соответственно, \textbf{\textit{неатомарное действие}} является действием, выполнение которого требует декомпозиции этого действия на множество его \uline{поддействий}, т.е. частных действий более низкого уровня. Более частные действия могут выполняться как последовательно, так и параллельно.

Декомпозиция неатомарного действия на поддействия может иметь весьма сложный иерархический вид с большим числом уровней иерархии, т.е. поддействиями \textit{неатомарного действия} могут также \textit{неатомарные действия}. Уровень сложности действия можно определять (1) общим числом его поддействий и (2) числом уровней иерархии этих поддействий. Примером может служить запись одной и той же процедурной программы на языке программирования более высокого уровня и на языке программирования более низкого уровня. В данном случае атомарность действий строго определяется на уровне языка.

Темпоральные соотношения между \textit{поддействиями} неатомарного \textit{действия} могут быть самые различные, но в пройстейшем случае \textit{неатомарное действие} представляет собой строгую последовательность \textit{действий} более низкого уровня иерархии.

В состав \textit{неатомарного действия} могут входить не только \textit{собственно поддействия} этого \textit{неатомарного действия}, но также и специальные \textit{поддействия}, осуществляющие \uline{управление} процессом выполнения \textit{неатомарного действия}, и, в частности, \textit{поддействия}, осуществляющие инициирование поддействий, передачу управления \textit{поддействиям}.

Действие можно назвать \textbf{\textit{легко выполнимым неатомарным действием}} в случае, когда для выполнения этого действия известен соответствующий \textit{метод} и соответствующие этому методу исходные данные, а также (для действий, выполняемых во внешней среде) имеются в наличии все необходимые исходные объекты (расходные материалы и комплектация), а также средства (инструменты)

В свою очередь, \textbf{\textit{трудно выполнимым неатомарным действием}} действие является тогда, когда для его выполнения в текущий момент либо неизвестен соответствующий \textit{метод}, либо возможные \textit{методы} известны, но отсутствуют условия их применения. Это действие декомпозируется на несколько самостоятельных поддействий, каждое из которых выявляет (локализует) противоречия (ошибки) конкретного формализуемого вида, для которого в базе знаний существует точное определение.

Признак многоагентности выполнения действия характеризует количество субъектов, выполнеяющих действие.
\begin{SCn}
	\scnheader{действие}
	\scnrelfrom{разбиение}{Типология классов действий по признаку многоагентности выполнения действия}
	\begin{scnindent}
		\begin{scneqtoset}
			\scnitem{индивидуальное действие}
			\begin{scnindent}
				\scnidtf{действие, выполняемое одним субъектом (агентом)}
				\scnidtf{действие, выполняемое индивидуальной кибернетической системой}
			\end{scnindent}
			\scnitem{коллективное действие}
			\begin{scnindent}
				\scnidtf{действие, выполняемое коллективом субъектов (многоагентной системой)}
				\scnidtf{действие, выполняемое коллективом кибернетических систем (коллективом субъектов)}
				\scnsuperset{действие, выполняемое коллективом людей}
				\scnsuperset{действие, выполняемое коллективом индивидуальных компьютерных систем}
				\scnsuperset{действие, выполняемое коллективом людей и индивидуальных компьютерных систем}
				\begin{scnindent}
					\scnsuperset{действие, выполняемое Экосистемой OSTIS}
					\scnsuperset{действие, выполняемое одним человеком во взаимодействии с одной индивидуальной компьютерной системой}
				\end{scnindent}
			\end{scnindent}
		\end{scneqtoset}
	\end{scnindent}
\end{SCn}


Последним признаком разбиения классов действия является признак текущего состояния действия.
\begin{SCn}
	\scnheader{действие}
	\scnrelfrom{разбиение}{Типология классов действий по признаку текущего состояния действия}
	\begin{scnindent}
		\begin{scneqtoset}
			\scnitem{планируемое действие}
			\begin{scnindent}
				\scnidtf{запланированное, но не инициированное действие}
				\scnidtf{будущее действие}
			\end{scnindent}
			\scnitem{инициированное действие}
			\begin{scnindent}
				\scnidtf{действие, ожидающее начала своего выполнения}
				\scnidtf{действие, подлежащее выполнению}
			\end{scnindent}
			\scnitem{выполняемое действие}
			\begin{scnindent}
				\scnidtf{активное действие}
				\scnidtf{действие, выполняемое в текущий момент}
				\scnidtf{настоящее действие}
			\end{scnindent}
			\scnitem{прерванное действие}
			\begin{scnindent}
				\scnidtf{действие, ожидающее продолжения своего выполнения}
				\scnidtf{отложенное действие}
			\end{scnindent}
			\scnitem{выполненное действие}
			\begin{scnindent}
				\scnidtf{завершенное действие}
				\scnidtf{прошлое действие}
				\begin{scnsubdividing}
					\scnitem{успешно выполненное действие}
					\scnitem{безуспешно выполненное действие}
					\begin{scnindent}
						\scnsuperset{действие, выполненное с ошибкой}
					\end{scnindent}
				\end{scnsubdividing}
			\end{scnindent}
			\scnitem{отмененное действие}
		\end{scneqtoset}
	\end{scnindent}
\end{SCn}

Во множество \textbf{\textit{планируемых действий}} входят \textit{действия}, начало выполнение которых запланировано на какой-либо момент в будущем(\textit{начало*}).

Во множество \textbf{\textit{инициированных действий}} входят \textit{действия}, выполнение которых инициировано в результате какого-либо события.
В общем случае, \textit{действия} могут быть инициированы по следующим причинам:
\begin{scnitemize}
	\item явно путем проведения соответствующей \textit{sc-дуги  принадлежности} каким-либо \textit{субъектом (заказчиком*)}. В случае действия в \textit{sc-памяти}, оно может быть инициировано как внутренним \textit{sc-агентом} системы, так и пользователем при помощи соответствующего пользовательского интерфейса. При этом, спецификация действия может быть сформирована одним \textit{sc-агентом}, а собственно добавление во множество \textit{инициированных действий} может быть осуществлено позже другим \textit{sc-агентом};
	\item в результате того, что одно или несколько \textit{действий}, предшествовавших данному в рамках некоторой декомпозиции, стали \textit{прошлыми сущностями} (процедурный подход);
	\item в результате того, что в \textit{памяти} системы появилась конструкция, соответствующая некоторому условию инициирования \textit{sc-агента}, который должен выполнить данное \textit{действие} (декларативный подход).
\end{scnitemize}

Следует отметить, что декларативный и процедурный подходы можно рассматривать как две крайности, использование только одной из которых не является удобным и целесообразным. При этом, например, принципы инициирования по процедурному подходу могут быть полностью сведены к набору декларативных условий инициирования, но как было сказано, это не всегда удобно и наиболее рациональным будет комбинировать оба подхода в зависимости от ситуации.

По сути, попадание некоторого \textit{действия} во множество \textit{инициированных действий} говорит о том, что, по мнению некоторого \textit{субъекта} (заказчика, инициатора), оно готово к выполнению и должно быть выполнено, то есть спецификация данного \textit{действия} по мнению данного \textit{субъекта} сформирована в степени, достаточной для решения поставленной \textit{задачи} и существует некоторый другой \textit{субъект} (исполнитель), который может приступать к выполнению \textit{действия}. Однако стоит отметить, что с точки зрения \textit{исполнителя} такая спецификация \textit{действия} в общем случае может оказаться недостаточной или некорректной.

Во множество \textbf{\textit{выполняемых действий}} входят \textit{действия}, к выполнению которых приступил какой-либо из соответствующих \textit{субъектов}.
Попадание \textit{действия} в данное множество говорит о следующем:
\begin{scnitemize}
	\item рассматриваемое \textit{действие} уже попало во множество \textit{инициированных действий};
	\item существует как минимум один \textit{субъект}, условие инициирования которого соответствует спецификации данного \textit{действия}.
\end{scnitemize}

После того, как собственно процесс выполнения завершился, \textit{действие} должно быть удалено из множества \textit{выполняемых действий} и добавлено во множество \textit{выполненных действий} или какое-либо из его подмножеств.

Понятие \textit{выполняемое действие} является неосновным, и вместо того, чтобы относить конкретные действия к данному классу, их относят к классу \textit{настоящих сущностей}.

Во множество \textbf{\textit{прерванных действий}} входят \textit{действия}, которые уже были инициированы, однако их выполнение невозможно по каким-либо причинам, например в случае, когда у исполнителя в данный момент есть более приоритетные задачи

Во множество \textbf{\textit{выполненных действий}} попадают \textit{действия}, выполнение которых завершено с \uline{точки зрения \textit{субъекта}}, осуществлявшего их выполнение. Таким образом, понятие \textit{выполненного действия} является относительным, поскольку с точки зрения разных субъектов одно и то же действие может считаться выполненным или все еще выполняющимся.

В зависимости от результатов конкретного процесса выполнения, рассматриваемое \textit{действие} может стать элементом одного из подмножеств множества \textit{выполненных действий}.

Понятие \textit{выполненное действие} является неосновным, и вместо того, чтобы относить конкретные \textit{действия} к данному классу, их относят к классу \textit{прошлых сущностей}.

Во множество \textbf{\textit{успешно выполненных действий}} попадают \textit{действия}, выполнение которых успешно завершено с точки зрения \textit{субъекта}, осуществлявшего их выполнение, т.е. достигнута поставленная цель, например, получены решение и ответ какой-либо задачи, успешно преобразована какая-либо конструкция и т.д. Очень важно отметить, что в общем случае выделить критерии успешности или безуспешности выполнения действий того или иного класса \uline{невозможно}, поскольку эти критерии, во-первых, зависят от контекста, во-вторых, могут быть разными с точки зрения разных субъектов. Однозначно критерии успешности выполнения действий могут быть сформулированы для некоторых частных классов действий, например, классов операторов некоторого процедурного языка программирования (например, \textit{scp-операторов}). Таким образом, понятие \textit{успешно выполненное действие} является относительным.

Если действие было выполнено успешно, то, в случае действия по генерации каких-либо знаний, к \textit{действию} при помощи связки отношения \textit{результат*} приписывается \textit{sc-конструкция}, описывающая результат выполнения указанного действия. В случае, когда действие направлено на какие-либо изменения базы знаний, \textit{sc-конструкция}, описывающая результат действия, формируется в соответствии с правилами описания истории изменений базы знаний.

В случае, когда успешное выполнение \textit{действия} приводит к изменению какой-либо конструкции в \textit{sc-памяти}, которое необходимо занести в историю изменений базы знаний или использовать для демонстрации протокола решения задачи, то генерируется соответствующая связка отношения \textit{результат*}, связывающая задачу и \textit{sc-конструкцию}, описывающую данное изменение.

Во множество \textbf{\textit{безуспешно выполненных действий}} попадают \textit{действия}, выполнение которых не было успешно завершено с точки зрения \textit{субъекта}, осуществлявшего их выполнение, по каким-либо причинам.

Можно выделить две основные причины, по которым может сложиться указанная ситуация:
\begin{scnitemize}
	\item соответствующая \textit{задача} сформулирована некорректно;
	\item формулировка соответствующей \textit{задачи} корректна и понятна системе, однако решение данной задачи в текущий момент не может быть получено за удовлетворительное с точки зрения заказчика или исполнителя сроки.
\end{scnitemize}

Для конкретизации факта некорректности формулировки задачи можно выделить ряд более частных классов \textit{безуспешно выполненных действий}, например:
\begin{scnitemize}
	\item \textit{действие, спецификация которого противоречит другим знаниям системы} (например, не выполняется неравенство треугольника);
	\item \textit{действие, при спецификации которого использованы понятия, неизвестные системе};
	\item \textit{действие, выполнение которого невозможно из-за недостаточности данных} (например, найти площадь треугольника по двум сторонам);
	\item и другие.
\end{scnitemize}

Для конкретизации факта безуспешности выполнения некоторого действия в системе могут также использоваться дополнительные подмножества данного множества, при необходимости снабженные естественно-языковыми или формальными комментариями.

Во множество \textbf{\textit{действий, выполненных с ошибкой}}, попадают \textit{действия}, выполнение которых не было успешно завершено с точки зрения \textit{субъекта}, осуществлявшего их выполнение, по причине возникновения какой-либо ошибки, например, некорректности спецификации данного \textit{действия} или нарушения её целостности каким-либо \textit{субъектом} (в случае \textit{действия в sc-памяти}).


\section{Формализация понятия задачи}
\label{sec_problem_concept}

\begin{SCn}
	\begin{scnrelfromlist}{ключевое понятие}
		\scnitem{задача}
	\end{scnrelfromlist}

	\begin{scnrelfromlist}{ключевое отношение}
		\scnitem{субъект\scnrolesign}
		\scnitem{объект\scnrolesign}
		\scnitem{цель*}
	\end{scnrelfromlist}
\end{SCn}

В свою очередь, \textit{задачу} будем трактовать как спецификацию некоторого действия, в рамках которой, в зависимости от ситуации, при помощи перечисленных выше отношений может быть заранее указан контекст выполнения действия, способ его выполнения, исполнитель, заказчик, планируемый результат и т.д.

Рассмотрим спецификацию понятия \textit{задача} в SCn-коде.

\begin{SCn}
	\scnheader{задача}
	\scnidtf{описание некоторого желаемого состояния или события либо в базе знаний, либо во внешней среде}
	\scnidtf{формулировка задачи}
	\scnidtf{задание на выполнение некоторого действия}
	\scnidtf{постановка задачи}
	\scnidtf{задачная ситуация}
	\scnidtf{спецификация некоторого действия, обладающая достаточной полнотой для выполнения этого действия}
\end{SCn}

Каждая \textbf{\textit{задача}} представляет собой спецификацию действия, которое либо уже выполнено, либо выполняется в текущий момент (в настоящее время), либо планируется (должно) быть выполненным, либо может быть выполнено (но не обязательно). В зависимости от конкретного класса задач, описываться может как внутреннее состояние самой интеллектуальной системы, так и требуемое состояние внешней среды.

Классификация задач может осуществляться по дидактическому признаку в рамках каждой предметной области, например, задачи на треугольники, задачи на системы уравнений и т.п.

Каждая \textit{задача} может включать:
\begin{textitemize}
	\item факт принадлежности \textit{действия} какому-либо частному классу \textit{действий} (например,\textit{ действие. сформировать полную семантическую окрестность указываемой сущности}), в том числе состояние \textit{действия} с точки зрения жизненного цикла (инициированное, выполняемое и т.д.);
	\item описание \textit{цели*} (\textit{результата*}) \textit{действия}, если она точно известна;
	\item указание \textit{заказчика*} действия;
	\item указание \textit{исполнителя* действия} (в том числе коллективного);
	\item указание \textit{аргумента(-ов) действия\scnrolesign};
	\item указание инструмента или посредника \textit{действия};
	\item описание \textit{декомпозиции действия*};
	\item указание \textit{последовательности действий*} в рамках \textit{декомпозиции действия*}, т.е. построение процедурного плана решения задачи. Другими словами, построение плана решения представляет собой декомпозицию соответствующего \textit{действия} на систему взаимосвязанных между собой поддействий;
	\item указание области \textit{действия};
	\item указание условия инициирования \textit{действия};
	\item момент начала и завершения \textit{действия}, в том числе планируемый и фактический, предполагаемая и/или фактическая длительность выполнения.
\end{textitemize}
Некоторые задачи могут быть дополнительно уточнены контекстом -- дополнительной информацией о сущностях, рассматриваемых в формулировке \textit{задачи}, т.е. описанием того, что дано, что известно об указанных сущностях.

Кроме этого, \textit{задача} может включать любую дополнительную информацию о действии, например:
\begin{textitemize}
	\item перечень ресурсов и средств, которые предполагается использовать при решении задачи, например, список доступных исполнителей, временные сроки, объем имеющихся финансов и т.д.;
	\item ограничение области, в которой выполняется \textit{действие}, например, необходимо заменить одну \textit{sc-конструкцию} на другую по некоторому правилу, но только в пределах некоторого \textit{раздела базы знаний};
	\item ограничение знаний, которые можно использовать для решения той или иной задачи, например, необходимо решить задачу по алгебре, используя только те утверждения, которые входят в курс школьной программы до седьмого класса включительно, и не используя утверждения, изучаемые в старших классах;
	\item и прочее.
\end{textitemize}

Как и в случае с действиями, решаемыми системой, можно классифицировать \textit{информационные задачи} и \textit{поведенческие задачи}.

С другой стороны, с точки зрения формулировки поставленной задачи, можно выделить \textit{декларативные формулировки задачи} и \textit{процедурные формулировки задачи}. Следует отметить, что данные классы задач не противопоставляются друг другу, и могут существовать формулировки задач, использующие оба подхода.

\begin{SCn}
	\scnheader{задача}
	\scnsuperset{процедурная формулировка задачи}
	\scnsuperset{декларативная формулировка задачи}
	\scnsuperset{вопрос}
	\scnsuperset{команда}
	\scnsubset{знание}
	\scnsuperset{инициированная задача}
	\begin{scnindent}
		\scnidtf{формулировка задачи, которая подлежит выполнению}
	\end{scnindent}
	\scnsuperset{декларативная формулировка задачи}
	\scnsuperset{процедурная формулировка задачи}
	\scnsuperset{декларативно-процедурная формулировка задачи}
	\begin{scnindent}
		\scnidtf{задача, в формулировке которой присутствуют как декларативные (целевые), так и процедурные аспекты}
	\end{scnindent}
	\scnsuperset{задача, решаемая в памяти кибернетической системы}
	\begin{scnindent}
		\scnsuperset{задача, решаемая в памяти индивидуальной кибернетической системы}
		\scnsuperset{задача, решаемая в общей памяти многоагентной системы}
		\scnidtf{информационная задача}
		\scnidtf{задача, направленная либо на \underline{генерацию} или поиск информации, удовлетворяющей заданным требованиям, либо на некоторое \underline{преобразование} заданной информации}
		\scnsuperset{математическая задача}
	\end{scnindent}
\end{SCn}

Формулировка \textit{задачи} может не содержать указания контекста (области решения) \textit{задачи} (в этом случае областью решения \textit{задачи} считается либо вся \textit{база знаний}, либо ее согласованная часть), а также может не содержать либо описания исходной ситуации, либо описания целевой ситуации. Так, например, описание целевой ситуации для явно специфицированного противоречия, обнаруженного в \textit{базе знаний}, не требуется.

\textbf{\textit{декларативная формулировка задачи}} представляет собой описание исходной (начальной) ситуации, являющейся условием выполнения соответствующего действия, и целевой (конечной) ситуации, являющейся результатом выполнения этого действия, то есть
описание ситуации (состояния), которое должно быть достигнуто в результате выполнения планируемого действия. Другими словами, такая формулировка задачи включает явное или неявное описание:
\begin{textitemize}
	\item того, что \underline{дано}, -- исходные данные, условия выполнения специфицируемого действия;
	\item того, что \underline{требуется}, -- формулировка цели, результата выполнения указанного действия.
\end{textitemize}

В случае \textbf{\textit{процедурной формулировки задачи}} явно указывается характеристика действия, специфицируемого этой задачей, а именно, например, указывается:
\begin{textitemize}
	\item субъект или субъекты, выполняющие это действие;
	\item объекты, над которыми действие выполняется, -- аргументы действия;
	\item инструменты, с помощью которых выполняется действие;
	\item момент и, возможно, дополнительные условия начала и завершения выполнения действия;
	\item явно указывается класс или классы, которым принадлежит каждое \textit{действие} (включая поддействия).
\end{textitemize}

При этом явно не уточняется, что должно быть результатом выполнения соответствующего действия.

Заметим, что, при необходимости, \textit{процедурная формулировка задачи} может быть сведена к \textit{декларативной формулировке задачи} путем трансляции на основе некоторого правила, например, определения класса действия через более общий класс.

Частными видами задач являются \textit{вопрос} и \textit{команда}.

\begin{SCn}
	\scnheader{вопрос}
	\scnidtf{запрос}
	\scnsubset{задача, решаемая в памяти кибернетической системы}
	\scnidtf{непроцедурная формулировка задачи на поиск (в текущем состоянии базы знаний) или на генерацию знания, удовлетворяющего заданным требованиям}
	\scnsuperset{вопрос -- что это такое}
	\scnsuperset{вопрос -- почему}
	\scnsuperset{вопрос -- зачем}
	\scnsuperset{вопрос -- как}
	\begin{scnindent}
		\scnidtf{каким способом}
		\scnidtf{запрос метода (способа) решения заданного (указываемого) вида задач или класса задач либо плана решения конкретной указываемой задачи}
	\end{scnindent}
	\scnidtf{задача, направленная на удовлетворение информационной потребности некоторого субъекта-заказчика}

	\scnheader{команда}
	\scnidtf{инициированная задача}
	\scnidtf{спецификация инициированного действия}
\end{SCn}


Следует отметить, что, наряду в приведенной предельно общей классификацией задач, по сути отражающей классы задач с точки зрения их формулировки, должна существовать классификация задач с точки зрения их семантики, то есть с точки зрения сути специфицируемого действия. За основу такой классификация можно взять классификацию, представленную в работе \cite{Fayans2020}.

В рамках же данной работы, как уже было сказано, наибольший интерес представляют задачи, решаемые в sc-памяти.

\subsection{Понятие класса действий и класса задач}
\label{subsec_action_and_problem_classes}

С точки зрения организации процесса решения задач более важными являются не столько понятия \textit{действия} и \textit{задачи}, сколько понятия \textit{класса действий} и \textit{класса задач}, поскольку именно для них разрабатываются соответствующие алгоритмы выполнения и способы решения.

\textbf{\textit{Класс действий}} определим как \underline{максимальное} множество аналогичных (похожих в определенном смысле) действий, для которого существует (но не обязательно известный в текущий момент) по крайней мере один \textbf{метод} (или средство), обеспечивающий выполнение \underline{любого} действия из указанного множества действий.

\begin{SCn}
	\scnheader{класс действий}
	\scnrelto{семейство подклассов}{действие}
	\scnidtf{множество однотипных действий}
	\scnsuperset{класс элементарных действий}
	\scnsuperset{класс легковыполнимых сложных действий}
\end{SCn}

Каждому выделяемому \textit{классу действий} соответствует по крайней мере один общий для них \textit{метод} выполнения этих \textit{действий}. Это означает то, что речь идет о \underline{семантической} "кластеризации"{} множества \textit{действий}, т.е. о выделении \textit{классов действий} по признаку \underline{семантической близости} (сходства) \textit{действий}, входящих в состав выделяемого \textit{класса действий}. При этом прежде всего учитывается аналогичность (сходство) \textit{исходных ситуаций} и \textit{целевых ситуаций} рассматриваемых \textit{действий}, т.е. аналогичность \textit{задач}, решаемых в результате выполнения соответствующих \textit{действий}. Поскольку одна и та же \textit{задача} может быть решена в результате выполнения нескольких \underline{разных} \textit{действий}, принадлежащих \underline{разным} \textit{классам действий}, следует говорить не только о \textit{классах действий} (множествах аналогичных действий), но и о \textbf{\textit{классах задач}} (о множествах аналогичных задач), решаемых этими \textit{действиями}. Так, например, на множестве \textit{классов действий} заданы следующие \textit{отношения}:
\begin{textitemize}
	\item \textit{отношение}, каждая связка которого связывает два разных (непересекающихся) \textit{класса действий}, осуществляющих решение одного и того же \textit{класса задач};
	\item \textit{отношение}, каждая связка которого связывает два разных \textit{класса действий}, осуществляющих решение разных \textit{классов задач}, один из которых является \textit{надмножеством} другого.
\end{textitemize}

Кроме класса действий также выделяется понятие \textbf{\textit{класса элементарных действий}}, то есть множества элементарных действий, указание принадлежности которому является \underline{необходимым} и достаточным условием для выполнения этого действия. Множество всевозможных элементарных действий, выполняемых каждым субъектом, должно быть \underline{разбито} на классы элементарных действий.

Принадлежность некоторого \textit{класса действий} множеству \textbf{\textit{класс элементарных действий}} фиксирует факт того, что при указании всех необходимых аргументов принадлежности \textit{действия} данному классу достаточно для того, чтобы некоторый субъект мог приступить к выполнению этого действия.

При этом, даже если \textit{класс действий} принадлежит множеству \textbf{\textit{класс элементарных действий}}, не запрещается вводить более частные \textit{классы действий}, для которых, например, заранее фиксируется один из аргументов.

Если конкретный \textbf{\textit{класс элементарных действий}} является более частным по отношению к \textit{действиям в sc-памяти}, то это говорит о наличии в текущей версии системы как минимум одного \textit{sc-агента}, ориентированного на выполнение действий данного класса.

Кроме того, целесообразно также ввести понятие \textit{класса легковыполнимых сложных действий}, то есть множества \textit{сложных действий}, для которых известен и доступен по крайней мере один \textbf{\textit{метод}}, интерпретация которого позволяет осуществить полную (окончательную, завершающуюся элементарными действиями) декомпозицию на поддействия \underline{каждого} сложного действия из указанного выше множества.

Принадлежность некоторого \textit{класса действий} множеству \textbf{\textit{класс легковыполнимых сложных действий}} фиксирует факт того, что даже при указании всех необходимых аргументов принадлежности \textit{действия} данному классу недостаточно для того, чтобы некоторый \textit{субъект} приступил к выполнению этого действия, и требуются дополнительные уточнения.

В свою очередь, под \textbf{\textit{классом задач}} будем понимать множество задач, для которого можно построить обобщенную формулировку задач, соответствующую всему этому множеству задач. Каждая \textit{обобщенная формулировка задач соответствующего класса}, по сути, есть не что иное, как строгое логическое определение указанного класса задач.

\begin{SCn}
	\scnheader{класс задач}
	\scnrelto{семейство подмножеств}{задача}
\end{SCn}

Конкретный класс действий может быть определен как минимум двумя способами.

\begin{SCn}
	\scnheader{класс действий}

	\begin{scnsubdividing}
		\scnitem{\textbf{класс действий, однозначно задаваемый решаемым классом задач}}
		\begin{scnindent}
			\scnidtf{\textit{класс действий}, обеспечивающих решение соответствующего \textit{класса задач} и использующих при этом любые, самые разные \textit{методы} решения задач этого класса}
		\end{scnindent}
		\scnitem{\textbf{класс действий, однозначно задаваемый используемым методом решения задач}}
	\end{scnsubdividing}
\end{SCn}

Далее более подробно рассмотрим формальную трактовку понятия \textit{метода}.

\subsection{Понятие метода}
\label{subsec_method_concept}

Под методом будем понимать описание того, \underline{как} может быть выполнено любое или почти любое (с явным указанием исключений) действие, принадлежащее соответствующему классу действий.

Формально, \textit{метод} -- это спецификация решения задачи какого-то класса \cite{Standard2021}, \cite{Tuzov1986}. В состав спецификации каждого класса задач входит описание способа \scnqq{привязки} метода к исходным данным конкретной задачи, решаемой с помощью этого метода.

\begin{SCn}
	\scnheader{метод}
	\scnidtf{метод решения соответствующего класса задач, обеспечивающий решение любой или большинства задач указанного класса}
	\scnidtf{обобщенная спецификация выполнения действий соответствующего класса}
	\scnidtf{обобщенная спецификация решения задач соответствующего класса}
	\scnidtf{программа решения задач соответствующего класса, которая может быть как процедурной, так и декларативной (непроцедурной)}
	\scnidtf{знание о том, как можно решать задачи соответствующего класса}
	\scnsubset{знание}
	\scniselement{вид знаний}
	\scnidtf{способ}
	\scnidtf{знание о том, как надо решать задачи соответствующего класса задач (множества эквивалентных (однотипных, похожих) задач)}
	\scnidtf{метод (способ) решения некоторого (соответствующего) класса задач}
	\scnidtf{информация (знание), достаточная для того, чтобы решить любую \textit{задачу}, принадлежащую соответствующему \textit{классу задач}, с помощью соответствующей \textit{модели решения задач}}
\end{SCn}

В состав спецификации каждого \textit{класса задач} входит описание способа "привязки"{} \textit{метода} к исходным данным конкретной \textit{задачи}, решаемой с помощью этого \textit{метода}. Описание такого способа "привязки"{} включает в себя:
\begin{textitemize}
	\item набор переменных, которые входят как в состав \textit{метода}, так и в состав \textit{обобщенной формулировки задач соответствующего класса}, и значениями которых являются соответствующие элементы исходных данных каждой конкретной решаемой задачи;
	\item часть \textit{обобщенной формулировки задач} того класса, которому соответствует рассматриваемый \textit{метод}, являющихся описанием \underline{условия применения} этого \textit{метода}.
\end{textitemize}

Сама рассматриваемая "привязка"{} \textit{метода} к конкретной \textit{задаче}, решаемой с помощью этого \textit{метода}, осуществляется путем \underline{поиска} в \textit{базе знаний} такого фрагмента, который удовлетворяет условиям применения указанного \textit{метода}. Одним из результатов такого поиска и является установление соответствия между указанными выше переменными используемого \textit{метода} и значениями этих переменных в рамках конкретной решаемой \textit{задачи}.

Другим вариантом установления рассматриваемого соответствия является явное обращение (вызов, call) соответствующего \textit{метода} (программы) с явной передачей соответствующих параметров. Но такое не всегда возможно, т.к. при выполнении процесса решения конкретной \textit{задачи} на основе декларативной спецификации выполнения этого действия нет возможности установить:
\begin{textitemize}
	\item когда необходимо инициировать вызов (использование) требуемого \textit{метода};
	\item какой конкретно \textit{метод} необходимо использовать;
	\item какие параметры, соответствующие конкретной инициируемой \textit{задаче}, необходимо передать для "привязки"{} используемого \textit{метода} к этой \textit{задаче}.
\end{textitemize}

Процесс "привязки"{} \textit{метода} решения \textit{задач} к конкретной \textit{задаче}, решаемой с помощью этого \textit{метода}, можно также представить как процесс, состоящий из следующих этапов:
\begin{textitemize}
	\item построение копии используемого \textit{метода};
	\item склеивание основных (ключевых) переменных используемого \textit{метода} с основными параметрами конкретной решаемой \textit{задачи}.
\end{textitemize}

В результате этого, на основе рассматриваемого \textit{метода}, используемого в качестве образца (шаблона), строится спецификация процесса решения конкретной задачи -- процедурная спецификация (\textit{план}) или декларативная.

Заметим, что \textit{методы} могут использоваться даже при построении \textit{планов} решения конкретных \textit{задач} в случае, когда возникает необходимость многократного повторения неких цепочек \textit{действий} при априори неизвестном количестве таких повторений. Речь идет о различного вида \textbf{циклах}, которые являются простейшим видом процедурных \textit{методов} решения задач, многократно используемых (повторяемых) при реализации \textit{планов} решения некоторых \textit{задач}.

Очевидно также, что одному \textit{классу действий} может соответствовать несколько \textit{методов}.

Термин ``метод'', таким образом, будем считать синонимичным термину ``программа'' в обобщенном понимании этого термина.

\begin{SCn}
	\scnheader{метод}
	\scnidtf{программа}
	\scnidtf{программа выполнения действий некоторого класса}
	\scnsuperset{процедурная программа}
	\begin{scnindent}
		\scnidtf{обобщенный план}
		\scnidtf{обобщенный план выполнения некоторого класса действий}
		\scnidtf{обобщенный план решения некоторого класса задач}
		\scnidtf{обобщенная спецификация декомпозиции любого действия, принадлежащего заданному классу действий}
		\scnsubset{алгоритм}
	\end{scnindent}
\end{SCn}

Рассмотрим более подробно понятие процедурной программы (процедурного метода). Каждая \textbf{\textit{процедурная программа}} представляет собой обобщенный план выполнения \textit{действий}, принадлежащих некоторому классу, то есть \textit{семантическую окрестность; ключевым sc-элементом\scnrolesign} является \textit{класс действий}, для элементов которого дополнительно детализируется процесс их выполнения.

Входным параметрам \textit{процедурной программы} в традиционном понимании соответствуют аргументы, соответствующие каждому \textit{действию} из \textit{класса действий}, описываемого данной {\textit{процедурной программой}. При генерации на основе данной программы \textit{плана} выполнения конкретного \textit{действия} из данного класса эти аргументы принимают конкретные значения.

Каждая \textit{процедурная программа} представляет собой систему описанных действий с дополнительным указанием для действия:
\begin{textitemize}
	\item либо \textit{последовательности выполнения действий*} (передачи инициирования), когда условием выполнения (инициирования) действий является завершение выполнения одного из указанных или всех указанных действий;
	\item либо события в базе знаний или внешней среде, являющегося условием его инициирования;
	\item либо ситуации в базе знаний или внешней среде, являющейся условием его инициирования.
\end{textitemize}
}

Понятие метода позволяет определить отношение \textit{эквивалентность задач*} на множестве задач. Задачи являются эквивалентными в том и только в том случае, если они могут быть решены путем интерпретации одного и того же \textit{метода} (способа), хранимого в памяти кибернетической системы.
Некоторые \textit{задачи} могут быть решены разными \textit{методами}, один из которых, например, является обобщением другого. Таким образом, на множестве методов можно также задать ряд отношений.

Отметим, что понятие \textit{метода} фактически позволяет локализовать область решения задач соответствующего класса, то есть ограничить множество знаний, которых достаточно для решения задач данного класса определенным способом. Это, в свою очередь, позволяет повысить эффективность работы системы в целом, исключая число лишних действий.

\begin{SCn}
	\scnheader{отношение, заданное на множестве методов}
	\scnhaselement{подметод*}
	\begin{scnindent}
		\scnidtf{подпрограмма*}
		\scnidtf{быть методом, использование которого (обращение к которому) предполагается при реализации заданного метода*}
		\scnrelboth{следует отличать}{частный метод*}
		\begin{scnindent}
			\scnidtf{быть методом, обеспечивающим решение класса задач, который является подклассом задач, решаемых с помощью заданного метода*}
		\end{scnindent}
	\end{scnindent}
\end{SCn}

В литературе, посвященной построению решателей задач, встречается понятие \textbf{\textit{стратегии решения задач}}. Определим его как метаметод решения задач, обеспечивающий либо поиск одного релевантного известного метода, либо синтез целенаправленной последовательности акций применения в общем случае различных известных методов.

\begin{SCn}
	\scnheader{стратегия решения задач}
	\scnsubset{метод}
\end{SCn}

Можно говорить об универсальном метаметоде (универсальной стратегии) решения задач, объясняющем всевозможные частные стратегии.
В частности, можно говорить о нескольких глобальных \textit{стратегиях решения информационных задач} в базах знаний. Пусть в базе знаний появился знак инициированного действия с формулировкой, соответствующей информационной цели, т.е. цели, направленной только на изменение состояния базы знаний. И пусть текущее состояние базы знаний не содержит контекст (исходные данные), достаточный для достижения указанной выше цели, т.е. такой контекст, для которого в доступном пакете (наборе) методов (программ) имеется метод (программа), использование которого позволяет достичь указанной выше цели. Для достижения такой цели, контекст (исходные данные) которой недостаточен, существует три подхода (три стратегии):
\begin{textitemize}
	\item декомпозиция (сведение изначальной цели к иерархической системе и/или подцелей (и/или подзадач) на основе анализа текущего состояния базы знаний и анализа того, чего не хватает в базе знаний для использования того или иного метода).

	При этом наибольшее внимание уделяется методам, для создания условий использования которых требуется меньше усилий. В конечном счете мы должны дойти (на самом нижнем уровне иерархии) до подцелей, контекст которых достаточен для применения одного из имеющихся методов (программ) решения задач;
	\item генерация новых знаний в семантической окрестности формулировки изначальной цели с помощью \underline{любых} доступных методов в надежде получить такое состояние базы знаний, которое будет содержать нужный контекст (достаточные исходные данные) для достижения изначальной цели с помощью какого-либо имеющегося метода решения задач;
	\item комбинация первого и второго подходов.
\end{textitemize}
Аналогичные стратегии существуют и для поиска пути решения задач, решаемых во внешней среде.

\subsection{Спецификация методов и понятие навыка}
\label{subsec_skill_concept}

Каждый конкретный \textit{метод} рассматривается нами не только как важный вид спецификации соответствующего класса задач, но также и как объект, который и сам нуждается в спецификации, обеспечивающей непосредственное применение этого метода. Другими словами, метод является не только спецификацией (спецификацией соответствующего класса задач), но и \underline{объектом} спецификации. Важнейшим видом такой спецификации является указание \textit{операционной семантики метода}.

\begin{SCn}
	\scnheader{операционная семантика метода*}
	\scnsubset{спецификация*}
	\scnidtf{семейство методов, обеспечивающих интерпретацию заданного метода*}
	\scnidtf{формальное описание интерпретатора заданного метода*}
	\scnrelfrom{второй домен}{\textbf{операционная семантика метода}}
	\begin{scnindent}
		\scnsuperset{\textbf{полное представление операционной семантики метода}}
		\begin{scnindent}
			\scnidtf{представление \textit{операционной семантики метода}, доведенное (детализированное) до уровня всех \textit{спецификаций элементарных действий}, выполняемых в процессе интерпретации соответствующего \textit{метода}}
		\end{scnindent}
	\end{scnindent}

	\scnheader{декларативная семантика метода*}
	\scnsubset{спецификация*}
	\scnidtf{описание системы понятий, которые используются в рамках данного метода*}
\end{SCn}

Отношение \textit{декларативная семантика метода*} связывает \textit{метод} и формальное описание системы понятий (фрагмент \textit{логической онтологии} соответствующей \textit{предметной области}), которые используются (упоминаются) в рамках данного метода. Это необходимо для того, чтобы гарантировать однозначность трактовки одного и того же понятия в рамках метода и остальной части базы знаний, что особенно актуально при заимствовании метода из библиотеки многократно используемых компонентов решателей задач. Важно отметить, что тот факт, что какие-либо понятия используются в рамках метода, не означает, что формальная запись их определений является частью данного метода. Например, в состав метода, позволяющего решать задачи на вычисление площади треугольника, будут входить различные формулы расчета площади треугольника, но не будут входить сами определения понятий ``площадь'', ``треугольник'' и т.д., поскольку при наличии априори верных формул эти определения не будут непосредственно использоваться в процессе решения задачи. В то же время формальные определения указанных понятий будут входить в состав декларативной семантики данного метода.

Объединение \textit{метода} и его операционной семантики, то есть информации о том, каким образом должен интерпретироваться данный \textit{метод}, будем называть \textbf{\textit{навыком}}.

\begin{SCn}
	\scnheader{навык}
	\scnidtf{умение}
	\scnidtf{объединение \textit{метода} с его исчерпывающей спецификацией -- \textit{полным представлением операционной семантики метода}}
	\scnidtf{метод, интерпретация (выполнение, использование) которого полностью может быть осуществлено данной кибернетической системой, в памяти которой хранится указанный метод}
	\scnidtf{метод, который данная кибернетическая система умеет (может) применять}
	\scnidtf{метод + метод его интерпретации}
	\scnidtf{умение решать соответствующий класс эквивалентных задач}
	\scnidtf{метод плюс его операционная семантика, описывающая то, как интерпретируется (выполняется, реализуется) этот метод, и являющаяся одновременно операционной семантикой соответствующей модели решения задач}

	\begin{scnsubdividing}
		\scnitem{активный навык}
		\begin{scnindent}
			\scnidtf{самоинициирующийся навык}
		\end{scnindent}
		\scnitem{пассивный навык}
	\end{scnsubdividing}
\end{SCn}

Таким образом, понятие \textit{навыка} является важнейшим понятием с точки зрения построения решателей задач, поскольку объединяет в себе не только декларативную часть описания способа решения класса задач, но и операционную.

\textit{Навыки} могут быть \textit{пассивными навыками}, то есть такими \textit{навыками}, применение которых должно явно инициироваться каким-либо агентом, либо \textit{активными навыками}, которые инициируются самостоятельно при возникновении соответствующей ситуации в базе знаний. Для этого в состав \textit{активного навыка}, помимо \textit{метода} и его операционной семантики, включается также \textit{sc-агент}, который реагирует на появление соответствующей ситуации в базе знаний и инициирует интерпретацию \textit{метода} данного \textit{навыка}.
Такое разделение позволяет реализовывать и комбинировать различные подходы к решению задач, в частности, \textit{пассивные навыки} можно рассматривать в качестве способа реализации концепции интеллектуального пакета программ.

\subsection{Понятие класса методов и языка представления методов}
\label{subsec_method_lang_concept}

Как действия и задачи, методы могут быть классифицированы по различным классам. Будем называть \textbf{\textit{классом методов}} множество методов, для которых можно \underline{унифицировать} представление (спецификацию) этих методов.

\begin{SCn}
	\scnheader{класс методов}
	\scnrelto{семейство подклассов}{метод}
	\scnidtf{множество методов, для которых можно унифицировать представление (спецификацию) этих методов}
	\scnidtf{множество всевозможных методов решения задач, имеющих общий язык представления этих методов}
	\scnidtf{множество методов, для которых задан язык представления этих методов}

	\scnhaselement{процедурный метод решения задач}
	\begin{scnindent}
		\scnsuperset{алгоритмический метод решения задач}
	\end{scnindent}
	\scnhaselement{непроцедурный метод решения задач}
	\begin{scnindent}
		\scnsuperset{логический метод решения задач}
		\begin{scnindent}
			\scnsuperset{продукционный метод решения задач}
			\scnsuperset{функциональный метод решения задач}
		\end{scnindent}
		\scnsuperset{продукционный метод решения задач}
		\scnsuperset{функциональный метод решения задач}
		\begin{scnindent}
			\scnsuperset{искусственная нейронная сеть}
			\begin{scnindent}
				\scnidtf{класс методов решения задач на основе искусственных нейронных сетей}
			\end{scnindent}
			\scnsuperset{генетический \scnqq{алгоритм}}
		\end{scnindent}
	\end{scnindent}

	\scnidtf{множество методов, основанных на общей онтологии}
	\scnidtf{множество методов, представленных на одинаковом языке}
	\scnidtf{множество методов решений задач, которому соответствует специальный язык (например, sc-язык), обеспечивающий представление методов из этого множества}
	\scnidtf{множество методов, которому ставится в соответствие отдельная модель решения задач}
\end{SCn}

Каждому конкретному \textit{классу методов} взаимно однозначно соответствует \textit{язык представления методов}, принадлежащих этому (специфицируемому) \textit{классу методов}. Таким образом, спецификация каждого \textit{класса методов} сводится к спецификации соответствующего \textit{языка представления методов}, т.е. к описанию его синтаксической, денотационной и операционной семантики. Примерами \textit{языков представления методов} являются все \textit{языки программирования}, которые, в основном, относятся к подклассу \textit{языков представления методов} -- к \textit{языкам представления методов обработки информации}. Но сейчас все большую актуальность приобретает необходимость создания эффективных формальных языков представления методов выполнения действий во внешней среде кибернетических систем. Без этого комплексная автоматизация \scncite{Pospelov2021}, в частности, в промышленной сфере, невозможна.

Таких специализированных языков может быть выделено целое множество, каждому из которых будет соответствовать своя модель решения задач (т.е. свой интерпретатор).

Под \textit{языком представления методов} будем подразумевать формальный язык, (1) знаковыми конструкциями которого являются соответствующие методы, для которых существуют общие правила построения и (2) общие правила соотнесения с теми сущностями и связями между ними, которые описываются этими методами.

\begin{SCn}
	\scnheader{язык представления методов}
	\scnidtf{язык методов}
	\scnidtf{язык представления методов, соответствующих определенному классу методов}
	\scnidtf{язык (например, sc-язык) представления методов соответствующего класса методов}
	\scnsubset{язык}
	\scnsubset{формальный язык}
	\scnidtf{формальный язык, (1) знаковыми конструкциями которого являются соответствующие методы, для которых существуют общие правила построения и (2) общие правила соотнесения с теми сущностями и связями между ними, которые описываются этими методами}
	\scnidtf{язык программирования}
	\scnsuperset{язык представления методов обработки информации}
	\begin{scnindent}
		\scnidtf{язык программирования внутренних действий кибернетической системы, выполняемых в их памяти}
		\scnidtf{язык представления методов решения задач в памяти кибернетических систем}
	\end{scnindent}
	\scnsuperset{язык представления методов решения задач во внешней среде кибернетических систем}
	\begin{scnindent}
		\scnidtf{язык программирования внешних действий кибернетических систем}
	\end{scnindent}
\end{SCn}

Метод принадлежит языку представления методов, если он является синтаксически корректным, синтаксически целостным, семантически корректным и семантически целостным методом заданного языка представления методов (!).

\begin{SCn}
	\scnheader{отношение, заданное на множестве языков представления методов\scnsupergroupsign}
	\scnidtf{отношение, область определения которого включает в себя множество всевозможных языков представления методов}
	\scnhaselement{метод заданного языка представления методов*}
	\scnhaselement{синтаксически корректный метод для заданного языка представления методов*}
	\begin{scnindent}
		\scnidtf{метод, не содержащий синтаксических ошибок для заданного языка представления методов*}
		\scnsubset{синтаксически корректная знаковая конструкция для заданного языка*}
	\end{scnindent}
	\scnhaselement{синтаксически целостный метод для заданного языка представления методов*}
	\begin{scnindent}
		\scnsubset{синтаксически целостная знаковая конструкция для заданного языка*}
	\end{scnindent}
	\scnhaselement{семантически корректный метод для заданного языка представления методов*}
	\begin{scnindent}
		\scnidtf{метод, не содержащий семантических ошибок для заданного языка представления методов*}
		\scnsubset{семантически корректная знаковая конструкция для заданного языка*}
	\end{scnindent}
	\scnhaselement{семантически целостный метод для заданного языка представления методов*}
	\begin{scnindent}
		\scnsubset{семантически целостная знаковая конструкция для заданного языка*}
		\scnidtf{метод заданного языка представления методов, содержащий достаточную информацию для установления его
		истинности*}
	\end{scnindent}
\end{SCn}

\begin{SCn}
	\scnheader{метод заданного языка представления методов*}
	\scnidtf{метод, принадлежащий заданному языку программирования*}
	\scnsubset{текст заданного языка*}
	\scnrelfrom{второй домен}{метод}
	\begin{scnreltoset}{объединение}
		\scnitem{\scnnonamednode}
		\begin{scnindent}
			\begin{scnreltoset}{объединение}
				\scnitem{синтаксически корректный метод для заданного языка представления методов*}
				\scnitem{синтаксически целостный метод для заданного языка представления методов*}
			\end{scnreltoset}
		\end{scnindent}
		\scnitem{\scnnonamednode}
		\begin{scnindent}
			\begin{scnreltoset}{объединение}
				\scnitem{семантически корректный метод для заданного языка представления методов*}
				\scnitem{синтаксически целостный метод для заданного языка представления методов*}
			\end{scnreltoset}
		\end{scnindent}
	\end{scnreltoset}
\end{SCn}

\subsection{Общая классификация языков представления методов}
\label{subsec_method_lang_classes}

Языки представления методов в современном информационном обществе различают по их парадигмам: \textit{процедурные}, \textit{функциональные}, \textit{логические}, \textit{объектно-ориентированные} и т. д. Таким, например, в методах процедурного я.п.м. решение задачи компьютером формируется в виде последовательности операторов, в методах функционального я.п.м. — указанием других методов. В логическом я.п.м. применяются высказывания, а в объектно-ориентированном — объекты.

\begin{SCn}
	\scnheader{язык представления методов}
	\scnsuperset{язык представления методов общего назначения}
	\begin{scnindent}
		\scnidtf{язык программирования общего назначения}
	\end{scnindent}
	\scnsuperset{предметно-ориентированный язык представления методов}
	\begin{scnindent}
		\scnidtf{предметно-ориентированный язык программирования}
	\end{scnindent}
	\scnrelfrom{разбиение}{парадигма языка представления методов\scnsupergroupsign}
	\begin{scnindent}
		\begin{scneqtoset}
			\scnitem{процедурный язык представления методов}
			\scnitem{непроцедурный язык представления методов}
		\end{scneqtoset}
	\end{scnindent}
\end{SCn}

\textit{Процедурные языки представления методов} задают вычисления как последовательность операторов (команд).
Они ориентированы на компьютеры с архитектурой фон Неймана. Основные понятия процедурных я.п.м. тесно связаны с компонентами компьютера:
\begin{textitemize}
	\item переменными различных типов, которые моделируют ячейки памяти компьютера;
	\item операторами присваивания, которые моделируют пересылки данных между участками памяти;
	\item повторений действий в форме итерации, которые моделируют хранение информации в смежных ячейках памяти;
	\item и другое.
\end{textitemize}

\begin{SCn}
	\scnheader{процедурный язык представления методов}
	\scnidtf{императивный язык представления методов}
	\scnsuperset{структурный язык представления методов}
	\begin{scnindent}
		\begin{scnhaselementrolelist}{пример}
			\scnitem{Fortran}
			\scnitem{Си}
			\scnitem{Pascal}
		\end{scnhaselementrolelist}
	\end{scnindent}
	\scnsuperset{объектно-ориентированный язык представления методов}
	\begin{scnindent}
		\begin{scnhaselementrolelist}{пример}
			\scnitem{Java}
			\scnitem{Smalltalk}
			\scnitem{HTML}
		\end{scnhaselementrolelist}
		\scnsuperset{аспектно-ориентированный язык представления методов}
	\end{scnindent}
	\scnsuperset{скриптовый язык представления методов}
	\begin{scnindent}
		\scnidtf{склеивающий язык представления методов}
	\end{scnindent}
\end{SCn}

\textit{Непроцедурные языки представления методов}, в отличие от процедурных, задают вычисления как последовательность связанных между собой объектов. Основные понятия непроцедурных я.п.м. обычно не связаны с компонентами компьютера.

\begin{SCn}
	\scnheader{непроцедурный язык представления методов}
	\scnidtf{декларативный язык представления методов}
	\scnsuperset{логический язык представления методов}
	\begin{scnindent}
		\begin{scnhaselementrolelist}{пример}
			\scnitem{Prolog}
		\end{scnhaselementrolelist}
	\end{scnindent}
	\scnsuperset{продукционный язык представления методов}
	\scnsuperset{функциональный язык представления методов}
	\begin{scnindent}
		\scnidtf{аппликативный язык представления методов}
		\begin{scnhaselementrolelist}{пример}
			\scnitem{LISP}
		\end{scnhaselementrolelist}
	\end{scnindent}
\end{SCn}

\subsection{Понятие модели решения задач}
\label{sec_problem_solving_model}

По аналогии с понятием стратегии решения задач введем понятие \textbf{\textit{модели решения задач}}, которое будем трактовать как метаметод интерпретации соответствующего класса методов.

\begin{SCn}
	\scnheader{модель решения задач}
	\scnsubset{метод}
	\scnidtf{метаметод}
	\scnidtf{абстрактная машина интерпретации соответствующего класса методов}
	\scnidtf{иерархическая система "микропрограмм"{}, обеспечивающих интерпретацию соответствующего класса методов}
	\scnsuperset{алгоритмическая модель решения задач}
	\scnsuperset{процедурная параллельная синхронная модель решения задач}
	\scnsuperset{процедурная параллельная асинхронная модель решения задач}
	\scnsuperset{продукционная модель решения задач}
	\scnsuperset{функциональная модель решения задач}
	\scnsuperset{логическая модель решения задач}
	\begin{scnindent}
		\scnsuperset{четкая логическая модель решения задач}
		\scnsuperset{нечеткая логическая модель решения задач}
	\end{scnindent}
	\scnsuperset{"нейросетевая"{} модель решения задач}
	\scnsuperset{"генетическая"{} модель решения задач}
\end{SCn}

Каждая \textit{модель решения задач} задается:
\begin{textitemize}
	\item соответствующим классом методов решения задач, т.е. языком представления методов этого класса;
	\item предметной областью этого класса методов;
	\item онтологией этого класса методов (т.е. денотационной семантикой языка представления этих методов);
	\item операционной семантикой указанного класса методов.
\end{textitemize}

Важно отметить, что для интерпретации \underline{всех} моделей решения задач может быть использован агентно-ориентированный подход, рассмотренный в работе \cite{Shunkevich2018}.

\begin{SCn}
	\scnheader{спецификация*}
	\scnsuperset{\textbf{модель решения задач}*}
	\begin{scnindent}
		\scneq{сужение отношения по первому домену (спецификация*; класс методов)*}
		\scnidtf{спецификация \textit{класса методов}*}
		\scnidtf{спецификация \textit{языка представления методов}*}
		\begin{scnsubdividing}
			\scnitem{\textbf{синтаксис языка представления методов соответствующего класса}*}
			\scnitem{\textbf{денотационная семантика языка представления методов соответствующего класса}*}
			\scnitem{\textbf{операционная семантика языка представления методов соответствующего класса}*}
		\end{scnsubdividing}
	\end{scnindent}
\end{SCn}

Модель решения задач ставит в соответствие некоторому классу методов синтаксис, денотационную и операционную семантику языка представления методов соответствующего класса.

\begin{SCn}
	\scnheader{денотационная семантика языка представления методов соответствующего класса}
	\scnidtf{онтология соответствующего класса методов}
	\scnidtf{денотационная семантика соответствующего класса методов}
	\scnidtf{денотационная семантика языка (sc-языка), обеспечивающего представление методов соответствующего класса}
	\scnidtf{денотационная семантика соответствующей модели решения задач}
	\scnnote{Если речь идет о языке, обеспечивающем внутреннее представление методов соответствующего класса в ostis-системе, то синтаксис этого языка совпадает с синтаксисом sc-кода}
	\scnsubset{онтология}

	\scnheader{операционная семантика языка представления методов соответствующего класса}
	\scnidtf{метаметод интерпретации соответствующего класса методов}
	\scnidtf{семейство агентов, обеспечивающих интерпретацию (использование) любого метода, принадлежащего соответствующему классу методов}
	\scnidtf{операционная семантика соответствующей модели решения задач}
\end{SCn}

Поскольку каждому \textit{методу} соответствует \textit{обобщенная формулировка задач}, решаемых с помощью этого \textit{метода}, то каждому \textit{классу методов} должен соответствовать не только определенный \textit{язык представления методов}, принадлежащих указанному \textit{классу методов}, но и определенный \textit{язык представления обобщенных формулировок задач для различных классов задач}, решаемых с помощью \textit{методов}, принадлежащих указанному \textit{классу методов}.


\section{Локальные предметные области и онтологии действий}
\label{sec_local_sd_actions}

\section*{Заключение к Главе \ref{chapter_actions}}

Дальнейшее развитие представленных в данной главе онтологий предполагает формализацию классификации задач, решаемых интеллектуальными системами, унификацию описания задач и классов задач, описания целей, хода и результата решения задачи, методов решения задач, связей между классами задач и методами решения задач данного класса.
Это позволит обеспечить возможность глубокой интеграции всевозможных моделей решения задач различных классов и возможность облегчить процесс интеграции новых моделей решения задач в интеллектуальную систему, а также положит основу для решения ряда проблем в области разработки гибридных решателей задач, рассмотренных в \textit{Главе \ref{chapter_situation_management} \nameref{chapter_situation_management}}.

%\input{author/references}
\chapauthor{Сердюков Р.Е.\\Зотов Н.В.\\Шункевич Д.В.}
\chapter{Семантическая теория программ в интеллектуальных компьютерных системах нового поколения}
\chapauthortoc{Сердюков Р.Е.\\Зотов Н.В.\\Шункевич Д.В.}
\label{chapter_programs}

\abstract{}

Современный персональный компьютер теперь имеет производительность большой электронной вычислительной машины 80-х годов
прошлого века. За долгий период развития компьютерных систем практически сняты аппаратные ограничения на решение ими
задач. Оставшиеся ограничения отводятся на долю программного обеспечения.
Прежде всего эти ограничения связаны с текущими проблемами развития программного обеспечения:
\begin{scnitemize}
    \item аппаратная сложность опережает умение человечества строить программные системы, использующее потенциальные
    возможности аппаратуры;
    \item навыки и технологии разработки программ отстают от требований, предъявлемых к разработке программ нового
    поколения;
    \item возможностям эксплуатировать существующие программы угрожает низкое качество их разработки.
\end{scnitemize}
Ключом к решению этих проблем является глубокое понимание и грамотное использование существующих языков программирования 
как основного инструмента для массового создания программных систем нового поколения, в том числе разумная организация 
процесса разработки и реинжиниринга таких систем.

Авторы данной главы стремятся достичь следующих результатов:
\begin{scnitemize}
    \item изложить классические основы, отражающие накопленный мировой опыт в области языков программирования;
    \item показать научные и практические достижения, характеризующие динамику развития языков программирования;
    \item систематизировать все основные результаты в этой области и представить их в виде единой универсальной
    семантической теории программ.
\end{scnitemize}

В первом параграфе данной главы подробно описывается текущее состояние в области программ и языков программирования,
которые могут и должны быть использования для разработки интеллектуальных компьютерных систем нового поколения. Он
посвящен базовым понятиям языков программирования, дается обзорная характеристика пяти типовых областей применения
языков программирования, достаточно востребованных современным человеческим обществом, подробно описываются формы и
содержание критериев для оценки эффективности языков и рассматриваются способы построения этих критериев.

% добавить параграф для компьютерных языков

\section{Программы и языки программирования для ostis-систем}

В современную эру развития информационных технологий существует огромное количество языков программирования, каждый из
который имеет своё важное назначение в области проектирования программных систем. Каждый язык демонстрирует не только
свою специфику, но имеет свои достоинства и недостатки. Многообразие языков программирования и решений, созданных
на них, настолько велико, что очень легко потеряться в море информации о всех аспектах применения и проектирования
языках программирования.

Зачем необходима общая теория программ и языков программирования?
\begin{scnnumerize}
    \item Ключом к легкому и глубокому освоению конкретного языка как основного профессионального инструмента
    программиста является понимание общих принципов построения и применения языков программирования, описываемых их
    общей теорией.
    \item Достижение максимума услуг и средств при минимуме затрат возможно только путём глубокого понимания принципов
    построения языков программирования.
\end{scnnumerize}

Каждая теория должна быть согласована понятийна. Не смотря на то, что в литературе сложилась разное трактование понятия
языка программирования, должно быть одно универсальное. Под \textit{языком программирования} будем подразумевать
формальный язык, (1) знаковыми конструкциями которого являются соответствующие программы, для которых существуют
общие правила построения и (2) общие правила соотнесения с теми сущностями и связями между ними, которые описываются
этими программами. Понятие синтакиса, денотационной и операционной семантики языков программирования сводятся к
понятию синтаксиса, денотационной и операционной семантики вообще любого языка.

С помощью языка программирования формируются сообщения (программы) для компьютера. Эти сообщения должны быть понятны
(семантически корректны и целостны) компьютеру.

\begin{SCn}
\scnheader{язык программирования}
\scnsubset{компьютерный язык общего назначения}
\begin{scnindent}
    \scnsubset{компьютерный язык}
    \begin{scnindent}
        \scnsubset{формальный язык}
    \end{scnindent}
\end{scnindent}
\scnidtf{формальный язык, (1) знаковыми конструкциями которого являются соответствующие программы, для которых
существуют общие правила построения и (2) общие правила соотнесения с теми сущностями и связями между ними, которые
описываются этими программами}
\scnidtf{средство общения между человеком (пользователем) и компьютером (исполнителем)}
\scnidtf{инструмент для производства программных услуг}
\end{SCn}

\begin{SCn}
\scnheader{отношение, заданное на множестве языков программирования\scnsupergroupsign}
\scnidtf{отношение, область определения которого включает в себя множество всевозможных языков программирования}
\scnhaselement{программа заданного языка программирования*}
\begin{scnindent}
    \scnidtf{программа, принадлежащая заданному языку программирования*}
    \scnsubset{текст заданного языка*}
    \scnrelfrom{второй домен}{программа}
    \begin{scnreltoset}{объединение}
        \scnitem{\scnnonamednode}
        \begin{scnindent}
            \begin{scnreltoset}{объединение}
                \scnitem{синтаксически корректная программа для заданного языка программирования*}
                \scnitem{синтаксически целостная программа для заданного языка программирования*}
            \end{scnreltoset}
        \end{scnindent}
        \scnitem{\scnnonamednode}
        \begin{scnindent}
            \begin{scnreltoset}{объединение}
                \scnitem{семантически корректная программа для заданного языка программирования*}
                \scnitem{синтаксически целостная программа для заданного языка программирования*}
            \end{scnreltoset}
        \end{scnindent}
    \end{scnreltoset}
\end{scnindent}
\scnhaselement{синтаксически корректная программа для заданного языка программирования*}
\begin{scnindent}
    \scnidtf{программа, не содержащая синтаксических ошибок для заданного языка программирования*}
    \scnsubset{синтаксически корректная знаковая конструкция для заданного языка*}
\end{scnindent}
\scnhaselement{синтаксически целостная программа для заданного языка программирования*}
\begin{scnindent}
    \scnsubset{синтаксически целостная знаковая конструкция для заданного языка*}
\end{scnindent}
\scnhaselement{семантически корректная программа для заданного языка программирования*}
\begin{scnindent}
    \scnidtf{программа, не содержащая семантических ошибок для заданного языка программирования*}
    \scnsubset{семантически корректная знаковая конструкция для заданного языка*}
\end{scnindent}
\scnhaselement{семантически целостная программа для заданного языка программирования*}
\begin{scnindent}
    \scnsubset{семантически целостная знаковая конструкция для заданного языка*}
    \scnidtf{программа заданного языка программирования, содержащий достаточную информацию для установления его
    истинности*}
\end{scnindent}
\end{SCn}

\begin{SCn}
\scnheader{язык программирования}
\scnrelfrom{разбиение}{парадигма языка программирования\scnsupergroupsign}
\begin{scnindent}
    \begin{scneqtoset}
        \scnitem{процедурный язык программирования}
        \begin{scnindent}
            \scnidtf{императивный язык программирования}
            \begin{scnhaselementrolelist}{пример}
                \scnitem{Fortran}
                \scnitem{Си}
            \end{scnhaselementrolelist}
        \end{scnindent}
        \scnitem{функциональный язык программирования}
        \begin{scnindent}
            \scnidtf{аппликативный язык программирования}
            \begin{scnhaselementrolelist}{пример}
                \scnitem{LISP}
            \end{scnhaselementrolelist}
        \end{scnindent}
        \scnitem{логический язык программирования}
        \begin{scnindent}
            \scnidtf{декларативный язык программирования}
            \begin{scnhaselementrolelist}{пример}
                \scnitem{Prolog}
            \end{scnhaselementrolelist}
        \end{scnindent}
        \scnitem{объектно-ориентированный язык программирования}
        \begin{scnindent}
            \begin{scnhaselementrolelist}{пример}
                \scnitem{Smalltalk}
                \scnitem{HTML}
            \end{scnhaselementrolelist}
        \end{scnindent}
    \end{scneqtoset}
\end{scnindent}
\scnrelfrom{разбиение}{типология языков программирования по цели использования\scnsupergroupsign}
\begin{scnindent}
    \begin{scneqtoset}
        \scnitem{традиционный язык программирования}
        \begin{scnindent}
            \begin{scnhaselementrolelist}{пример}
                \scnitem{C++}
                \scnitem{Fortran}
                \scnitem{Java}
            \end{scnhaselementrolelist}
        \end{scnindent}
        \scnitem{скриптовой язык программирования}
        \begin{scnindent}
            \scnidtf{склеивающий язык программирования}
            \begin{scnhaselementrolelist}{пример}
                \scnitem{Python}
                \scnitem{Perl}
                \scnitem{JavaScript}
                \scnitem{PHP}
                \scnitem{Ruby}
                \scnitem{Lua}
            \end{scnhaselementrolelist}
        \end{scnindent}
        \scnitem{гибридный язык программирования}
        \begin{scnindent}
            \begin{scnhaselementrolelist}{пример}
                \scnitem{XSLT}
                \scnitem{JSP}
            \end{scnhaselementrolelist}
        \end{scnindent}
    \end{scneqtoset}
\end{scnindent}
\end{SCn}

\begin{SCn}
\scnheader{программа}
\end{SCn}

\begin{SCn}
\scnheader{эффективность языка программирования\scnsupergroupsign}
\end{SCn}

\section{Принципы интерпретации современных языков программирования в ostis-системах}

%\input{author/references}

\chapauthor{Шункевич Д.В.}
\chapter{Ситуационное управление обработкой знаний в интеллектуальных компьютерных системах нового поколения}
\chapauthortoc{Шункевич Д.В.}
\label{chapter_situation_management}

\abstract{Аннотация к главе.}

\section{Решатели задач ostis-систем}

Одним из ключевых компонентов интеллектуальной системы, обеспечивающим возможность решать широкий круг задач, является решатель задач. Особенностью решателей задач интеллектуальных систем по сравнению с другими современными программными системами является необходимость решать задачи в условиях, когда сведения, необходимые для решения задачи, не локализованы явно в базе знаний интеллектуальной системы и должны быть найдены в процессе решения задачи на основании каких-либо критериев. Говоря другими словами, если в традиционных системах при решении задачи всегда подразумевается, что есть некоторые локализованные исходные данные (''дано'') и некоторое описание желаемого результата (''что требуется''), то в интеллектуальной системе в качестве исходных данных при решении большого числа задач выступает вся имеющаяся на текущий момент в системе информация, то есть вся база знаний. Кроме того, при невозможности решения задачи в текущем состоянии базы знаний интеллектуальная система должна иметь возможность понять, чего именно не хватает для решения задачи и попытаться добыть недостающие сведения во внешней среде (например, запросить у пользователя).

К настоящему времени в рамках различных направлений искусственного интеллекта разработано большое количество различных моделей решения задач, каждая из которых позволяет решать задачи определенного класса. Расширение областей применения интеллектуальных систем требует от таких систем возможности решать так называемые комплексные задачи, решение каждой из которых требует комбинирования нескольких моделей решения задач, при этом априори неизвестно, в каком порядке и сколько раз будет применяться так или иная модель. Решатели задач, в рамках которых комбинируются несколько моделей решения задач, получили название \textit{гибридных решателей задач}, а интеллектуальные системы, в рамках которых комбинируются различные виды знаний и различные модели решения задач -- название \textit{гибридных интеллектуальных систем}.

В рамках Технологии OSTIS решатель задач ostis-системы определяется как совокупность всех \textit{навыков}, которыми обладает ostis-система на текущий момент времени (более подробно о понятии навыка смотрите в \textit{Главе \ref{chapter_actions} \nameref{chapter_actions}}).

Предлагаемый в рамках \textit{Технологии OSTIS} подход к построению решателей задач позволяет обеспечить их модифицируемость, что, в свою очередь, позволяет \textit{ostis-системе} при необходимости легко приобретать новые \textit{навыки}, модифицировать (совершенствовать) уже имеющиеся, и даже избавляться от некоторых навыков с целью повышения производительности системы. Таким образом, имеет смысл говорить не о жестко фиксированном решателе задач, который разрабатывается один раз при создании первой версии системы и далее не меняется, а о совокупности навыков, фиксированной в каждый текущий момент времени, но постоянно эволюционирующей.

%TODO суть предлагаемого подхода и его обоснование

\begin{SCn}
\scnheader{решатель задач ostis-системы}
\scnrelto{семейство подмножеств}{навык}
\scnsuperset{гибридный решатель задач ostis-системы}
\begin{scnindent}
	\scnidtf{решатель задач ostis-системы, реализующий две и более модели решения задач}
\end{scnindent}
\scnsuperset{объединенный решатель задач ostis-системы}
\begin{scnindent}
\scnidtf{полный решатель задач ostis-системы}
\scnidtf{интегрированный решатель задач ostis-системы}
\scnidtf{решатель задач ostis-системы, реализующий все ее функциональные возможности, как основные, так и вспомогательные}
\end{scnindent}
\end{SCn}

В общем случае \textit{объединенный решатель задач ostis-системы}, решает задачи, связанные с:
	\begin{itemize}
		\item обеспечением основных функциональных возможностей системы (например, решение явно сформулированных задач по требованию пользователя);
		\item обеспечением корректности и оптимизацией работы самой ostis-системы (перманентно на протяжении всего жизненного цикла ostis-системы);
		\item обеспечением повышения квалификации конечных пользователей и разработчиков ostis-системы;
		\item обеспечением автоматизации развития и управления развитием ostis-системы.
\end{itemize}

Под \textit{машиной обработки знаний} будем понимать совокупность интерпретаторов всех \textit{навыков}, составляющих некоторый \textit{решатель задач}. С учетом многоагентного подхода к обработке информации, используемого в рамках Технологии OSTIS, \textit{машина обработки знаний} представляет собой \textit{sc-агент} (чаще всего -- \textit{неатомарный sc-агент}), в состав которого входят более простые sc-агенты, обеспечивающие интерпретацию соответствующего множества \textit{методов}. Таким образом, \textit{машина обработки знаний} в общем случае представляет собой иерархическую систему \textit{sc-агентов}.

\begin{SCn}
\scnheader{машина обработки знаний}
\scnsubset{sc-агент}
\end{SCn}

Рассмотрим классификацию решателей задач ostis-систем по различным признакам.

Классификация решателей задач ostis-систем по типу соответствующей ostis-системы:

\begin{SCn}
\scnheader{решатель задач ostis-системы}
\scnhaselement{Решатель задач Метасистемы IMS.ostis}
\scnsuperset{решатель задач вспомогательной ostis-системы}
\begin{scnindent}
\scnsuperset{решатель задач интерфейса компьютерной системы}
\begin{scnindent}
\begin{scnrelfromset}{разбиение}
	\scnitem{решатель задач пользовательского интерфейса компьютерной системы}
	\scnitem{решатель задач интерфейса компьютерной системы с другими компьютерными системами}
	\scnitem{решатель задач интерфейса компьютерной системы с окружающей средой}
\end{scnrelfromset}
\end{scnindent}
\scnsuperset{решатель задач ostis-подсистемы поддержки проектирования компонентов определенного класса}
\begin{scnindent}
\scnsuperset{решатель задач ostis-подсистемы поддержки проектирования баз знаний}
\begin{scnindent}
\scnsuperset{решатель задач повышения качества базы знаний}
\begin{scnindent}
\scnsuperset{решатель задач верификации базы знаний}
\begin{scnindent}
\scnsuperset{решатель задач поиска и устранения некорректностей в базе знаний}
\scnsuperset{решатель задач поиска и устранения неполноты}
\end{scnindent}
\scnsuperset{решатель задач оптимизации структуры базы знаний}
\scnsuperset{решатель задач выявления и устранения информационного мусора}
\end{scnindent}
\end{scnindent}
\scnsuperset{решатель задач ostis-подсистемы поддержки проектирования решателей задач ostis-систем}
\begin{scnindent}
\begin{scnrelfromset}{разбиение}
	\scnitem{решатель задач ostis-подсистемы поддержки проектирования программ обработки знаний}
	\scnitem{решатель задач ostis-подсистемы поддержки проектирования агентов обработки знаний}
\end{scnrelfromset}
\end{scnindent}
\end{scnindent}
\scnsuperset{решатель задач подсистемы управления проектирования компьютерных систем и их компонентов}
\end{scnindent}
\scnsuperset{решатель задач самостоятельной ostis-системы}
\end{SCn}

Классификация решателей задач ostis-систем по типу интерпретируемой модели решения задач:

\begin{SCn}
\scnheader{решатель задач ostis-системы}
\scnsuperset{решатель задач с использованием хранимых методов}
\begin{scnindent}
\scnidtf{решатель, способный решать задачи тех классов, для которых на данный момент времени известен соответствующий метод решения}
\scnsuperset{решатель задач на основе нейросетевых моделей}
\scnsuperset{решатель задач на основе генетических алгоритмов}
\scnsuperset{решатель задач на основе императивных программ}
\begin{scnindent}
\scnsuperset{решатель задач на основе процедурных программ}
\scnsuperset{решатель задач на основе объектно-ориентированных программ}
\end{scnindent}
\scnsuperset{решатель задач на основе декларативных программ}
\begin{scnindent}
\scnsuperset{решатель задач на основе логических программ}
\scnsuperset{решатель задач на основе функциональных программ}
\end{scnindent}
\end{scnindent}
\scnsuperset{решатель задач в условиях, когда метод решения задач данного класса в текущий момент времени не известен}
\begin{scnindent}
\scnidtf{решатель, реализующий стратегии решения задач, позволяющие породить метод решения задачи, который в текущий момент времени не известен ostis-системе}
\scnidtf{решатель, использующий для решения задач метаметоды, соответствующие более общим классам задач по отношению к заданной}
\scnidtf{решатель задач, позволяющий породить метод, который является частным по отношению какому-либо известному ostis-системе методу и интерпретируется соответствующей машиной обработки знаний}
\scnsuperset{решатель, реализующий стратегию поиска путей решения задачи в глубину}
\scnsuperset{решатель, реализующий стратегию поиска путей решения задачи в ширину}
\scnsuperset{решатель, реализующий стратегию проб и ошибок}
\scnsuperset{решатель, реализующий стратегию разбиения задачи на подзадачи}
\scnsuperset{решатель, реализующий стратегию решения задач по аналогии}
\scnsuperset{решатель, реализующий концепцию интеллектуального пакета программ}
\end{scnindent}
\end{SCn}

Отдельно выделим классификацию машин обработки знаний, которые в общем случае могут соответствовать одним и тем же фрагментам базы знаний, но при этом в совокупности с ними образовывать разные навыки и соответственно разные решатели задач:

\begin{SCn}
\scnheader{машина обработки знаний}
\scnsuperset{машина логического вывода}
\begin{scnindent}
\scnsuperset{машина дедуктивного вывода}
\begin{scnindent}
\scnsuperset{машина прямого дедуктивного вывода}
\scnsuperset{машина обратного дедуктивного вывода}
\end{scnindent}
\scnsuperset{машина индуктивного вывода}
\scnsuperset{машина абдуктивного вывода}
\scnsuperset{машина нечеткого вывода}
\scnsuperset{машина вывода на основе логики умолчаний}
\scnsuperset{машина логического вывода с учетом фактора времени}
\end{scnindent}
\end{SCn}

Классификация решателей задач ostis-систем по типу решаемой задачи (цели решения задачи):

\begin{SCn}
\scnheader{решатель задач ostis-системы}
\scnsuperset{решатель задач информационного поиска}
\begin{scnindent}
\begin{scnrelfromset}{разбиение}
	\scnitem{решатель задач поиска информации, удовлетворяющей заданным критериям}
	\scnitem{решатель задач поиска информации, не удовлетворяющей заданным критериям}
\end{scnrelfromset}
\end{scnindent}
\scnsuperset{решатель явно сформулированных задач}
\begin{scnindent}
\scnidtf{решатель задач, для которых явно сформулирована цель}
\scnsuperset{решатель задач поиска или вычисления значений заданного множества величин}
\scnsuperset{решатель задач установления истинности заданного логического высказывания в рамках заданной формальной теории}
\scnsuperset{решатель задач формирования доказательства заданного высказывания в рамках заданной формальной теории}
\scnsuperset{машина верификации ответа на указанную задачу}
\scnsuperset{машина верификации решения указанной задачи}
\begin{scnindent}
\scnsuperset{машина верификации доказательства заданного высказывания в рамках заданной формальной теории}
\end{scnindent}
\end{scnindent}
\scnsuperset{решатель задач классификации сущностей}
\begin{scnindent}
\scnsuperset{машина соотнесения сущности с одним из заданного множества классов}
\scnsuperset{машина разделения множества сущностей на классы по заданному множеству признаков}
\end{scnindent}
\scnsuperset{решатель задач синтеза информационных конструкций}
\begin{scnindent}
\scnsuperset{решатель задач синтеза естественно-языковых текстов}
\scnsuperset{решатель задач синтеза изображений}
\scnsuperset{решатель задач синтеза сигналов}
\begin{scnindent}
\scnsuperset{решатель задач синтеза речи}
\end{scnindent}
\end{scnindent}
\scnsuperset{решатель задач анализа информационных конструкций}
\begin{scnindent}
\scnsuperset{решатель задач анализа естественно-языковых текстов}
\begin{scnindent}
\scnsuperset{решатель задач понимания естественно-языковых текстов}
\scnsuperset{решатель задач верификации естественно-языковых текстов}
\end{scnindent}
\scnsuperset{решатель задач анализа изображений}
\begin{scnindent}
\scnsuperset{решатель задач сегментации изображений}
\scnsuperset{решатель задач понимания изображений}
\end{scnindent}
\scnsuperset{решатель задач анализа сигналов}
\begin{scnindent}
\scnsuperset{решатель задач анализа речи}
\begin{scnindent}
\scnsuperset{решатель задач понимания речи}
\end{scnindent}
\end{scnindent}
\end{scnindent}
\end{SCn}

\section{Действия, задачи, планы, протоколы и методы, реализуемые ostis-системой, а также внутренние агенты, выполняющие эти действия}

Рассмотрим основные принципы обработки информации, лежащие в основе предлагаемого подхода:

\begin{itemize}
\item В основе решателя задач каждой \textit{ostis-системы} лежит многоагентная система, агенты которой взаимодействуют между собой \uline{только}(!) через общую для них \textit{sc-память} посредством спецификации в этой памяти выполняемых ими \textit{действий в sc-памяти}. При этом пользователи \textit{ostis-системы} также считаются агентами этой системы. Кроме того, \textit{sc-агенты} делятся на внутренние, рецепторные и эффекторные. Взаимодействие между агентами через общую \textit{sc-память} сводится к следующим видам действий:
\begin{enumerate}
		\item К использованию общедоступной для соответствующей группы sc-агентов части хранимой базы знаний;
		\item К формированию (генерации) новых фрагментов базы знаний и/или к корректировке (редактированию) каких-либо фрагментов доступной части базы знаний;
		\item К интеграции (погружению) новых и/или обновленных фрагментов в состав доступной части базы знаний;
\end{enumerate}

Подчеркнем, что sc-агенты не общаются между собой напрямую путем отправки сообщений, как это делается в большинстве современных подходов к построению многоагентных систем. Кроме того, sc-агенты имеют доступ к общей для них базе знаний за счет чего гарантируется семантическая совместимость (взаимопонимание) между агентами, включая и пользователей ostis-систем.

\item Пользователь \textit{ostis-системы} не может сам непосредственно выполнить какое-либо действие в \mbox{sc-памяти}, но он может средствами пользовательского интерфейса инициировать построение (генерацию, формирование в \textit{sc-памяти}) \textit{sc-текста}, являющегося спецификацией \textit{действия в \mbox{sc-памяти}}, выполняемого либо одним \textit{атомарным sc-агентом} за один акт, либо одним \textit{атомарным sc-агентом} за несколько актов, либо коллективом \textit{sc-агентов} (\textit{неатомарным sc-агентом}). В спецификации каждого такого \textit{действия в sc-памяти}, инициированного пользователем, этот пользователь указывается как заказчик этого действия. Таким образом, пользователь \textit{ostis-системы} дает поручения (задания, команды) \textit{sc-агентам} этой системы на выполнение различных специфицируемых им действий в \textit{sc-памяти}.

\item Каждый \textit{sc-агент}, выполняя некоторое \textit{действие в sc-памяти}, должен "помнить"{}, что \textit{sc-память}, над которой он работает, является общим ресурсом не только для него, но и для всех остальных \textit{\mbox{sc-агентов}}, работающих над этой же \textit{sc-памятью}. Поэтому \textit{sc-агент} должен соблюдать определенную этику поведения в коллективе таких \textit{sc-агентов}, которая должна минимизировать помехи, которые он создает другим \textit{sc-агентам}.

\item Деятельность каждого агента \textit{ostis-системы} дискретна и представляет собой множество элементарных действий (актов). При этом при выполнении каждого акта агент может устанавливать блокировки нескольких типов на фрагменты базы знаний. Указанные блокировки позволяют запретить другим агентам изменять указанный фрагмент базы знаний или вообще сделать его "невидимым"{} для других агентов. Блокировки устанавливаются самим агентом при выполнении соответствующего акта и снимаются им же на последнем этапе выполнения этого акта или раньше, если это возможно.
\item Если некий \textit{sc-агент} выполняет некоторое \textit{действие в sc-памяти}, то он на время выполнения этого действия может:
\begin{enumerate}
	\item Запретить другим \textit{sc-агентам} изменять состояние некоторых sc-элементов, хранимых в \textit{sc-памяти} -- удалять их, изменять тип;
	\item Запретить другим \textit{sc-агентам} добавлять или удалять элементы некоторых множеств, обозначаемых соответствующими \textit{sc-узлами};
	\item Запретить другим \textit{sc-агентам} доступ на просмотр некоторых \textit{sc-элементов}, то есть эти \textit{\mbox{sc-элементы}} становятся полностью "невидимыми"{} (полностью заблокированными) для других \textit{sc-агентов} но только на время выполнения соответствующего действия.
\end{enumerate}

Указанные блокировки должны быть полностью сняты до завершения выполнения соответствующего действия. Подчеркнем, что число \textit{sc-элементов}, блокируемых на время выполнения некоторого действия, в основном входят атомарные и неатомарные связки, и не должны входить \textit{sc-узлы}, обозначающие бесконечные классы каких-либо сущностей, и, тем более, sc-узлы, обозначающие различные понятия (ключевые классы различных предметных областей).

Этичное (неэгоистичное) поведение \textit{sc-агента}, касающееся блокировки \textit{sc-элементов} (то есть ограничения к ним доступа другим \textit{sc-агентам}) предполагает соблюдение следующих правил:
\begin{enumerate}
	\item Не следует блокировать больше \textit{sc-элементов}, чем это необходимо для решения задачи;
	\item Как только для какого-либо \textit{sc-элемента} необходимость его блокировки отпадает до завершения выполнения соответствующего действия, этот \textit{sc-элемент} желательно сразу деблокировать (снять блокировку);
\end{enumerate}

Для того, чтобы \textit{sc-агент} имел возможность работы с каким-либо произвольным \textit{sc-элементом}, он должен либо убедиться в том, что этот \textit{sc-элемент} не входит во фрагмент базы знаний, входящий в \textit{полную блокировку}, либо убедиться в том, что эта блокировка не установлена самим этим агентом.

Особой группой полностью заблокированных \textit{sc-элементов} (на время выполнения действия \textit{\mbox{sc-агентом}}) являются вспомогательные \textit{sc-элементы} ("леса"{}), создаваемые только на время выполнения этого действия. Эти sc-элементы в конце выполнения действия должны не деблокироваться, а удаляться.
	
\item Если \textit{действие в sc-памяти}, выполняемое \textit{sc-агентом}, завершилось (т.е. стало прошлой сущностью), то \textit{sc-агент} оформляет результат этого \textit{действия}, указывая (1) удаленные \textit{sc-элементы} и (2) сгенерированные sc-элементы. Это необходимо, если по каким-либо причинам придется сделать откат этого \textit{действия}, т.е возвратиться к состоянию базы знаний до выполнения указанного \textit{действия}.
\end{itemize}

\subsection{Понятие действия в sc-памяти}

\begin{SCn}
\scnheader{действие в sc-памяти}
\scnidtf{внутреннее действие ostis-системы}
\scnidtf{действие, выполняемое в sc-памяти}
\scnidtf{действие, выполняемое в абстрактной унифицированной семантической памяти}
\scnidtf{действие, выполняемое машиной обработки знаний ostis-системы}
\scnidtf{действие, выполняемое агентом или коллективом агентов ostis-системы}
\scnidtf{информационный процесс над базой знаний, хранимой в sc-памяти}
\scnidtf{процесс решения информационной задачи в sc-памяти}
\scnsubset{процесс в sc-памяти}
\end{SCn}

Каждое \textbf{\textit{действие в sc-памяти}} обозначает некоторое преобразование, выполняемое некоторым \textit{sc-агентом} (или коллективом \textit{sc-агентов}) и ориентированное на преобразование \textit{sc-памяти}. Спецификация действия после его выполнения может быть включена в протокол решения некоторой задачи. 
	
Преобразование состояния базы знаний включает, в том числе и информационный поиск, предполагающий (1) локализацию в базе знаний ответа на запрос, явное выделение структуры ответа и (2) трансляцию ответа на некоторый внешний язык.

Во множество \textbf{\textit{действий в sc-памяти}} входят знаки действий самого различного рода, семантика каждого из которых зависит от конкретного контекста, т.е. ориентации действия на какие-либо конкретные объекты и принадлежности действия какому-либо конкретному классу действий.

Следует четко отличать:
\begin{itemize}
	\item Каждое конкретное \textbf{\textit{действие в sc-памяти}}, представляющее собой некоторый переходный процесс, переводящий sc-память из одного состояния в другое;
	\item Каждый тип \textbf{\textit{действий в sc-памяти}}, представляющий собой некоторый класс однотипных (в том или ином смысле) действий;
	\item sc-узел, обозначающий некоторое конкретное \textbf{\textit{действие в sc-памяти}};
	\item sc-узел, обозначающий структуру, которая является описанием, спецификацией, заданием, постановкой соответствующего действия.
\end{itemize}

Рассмотрим боле детально классификацию действий в sc-памяти:

\begin{SCn}
\scnheader{действие в sc-памяти}
\scnsuperset{действие в sc-памяти, инициируемое вопросом}
\scnsuperset{действие редактирования базы знаний ostis-системы}
\scnsuperset{действие установки режима ostis-системы}
\scnsuperset{действие редактирования файла, хранимого в sc-памяти}
\scnsuperset{действие интерпретации программы, хранимой в sc-памяти}
\begin{scnindent}
	\scnsuperset{действие интерпретации scp-программы}
\end{scnindent}

\scnheader{действие в sc-памяти, инициируемое вопросом}
\scnidtf{действие, направленное на формирование ответа на поставленный вопрос}
\scnsuperset{действие. cформировать заданный файл}
\scnsuperset{действие. cформировать заданную структуру}
\begin{scnindent}
\scnsuperset{действие. верифицировать заданную структуру}
\begin{scnindent}
\scnsuperset{действие. установить истинность или ложность указываемого логического высказывания}
\scnsuperset{действие. установить корректность или некорректность указываемой структуры}
\scnsuperset{действие. сформировать структуру, описывающую некорректности, имеющиеся в указываемой структуре}
\end{scnindent}
\scnsuperset{действие. уточнить тип заданного sc-элемента}
\begin{scnindent}
\scnsuperset{действие. установить позитивность/негативность указываемой sc-дуги принадлежности или непринадлежности}
\end{scnindent}
\scnsuperset{действие. сформировать семантическую окрестность}
\begin{scnindent}
\scnsuperset{действие. сформировать полную семантическую окрестность указываемой сущности}
\scnsuperset{действие. сформировать базовую семантическую окрестность указываемой сущности}
\scnsuperset{действие. сформировать частную семантическую окрестность указываемой сущности}
\end{scnindent}
\scnsuperset{действие. сформировать структуру, описывающую связи между указываемыми сущностями}
\begin{scnindent}
\scnsuperset{действие. сформировать структуру, описывающую сходства указываемых сущностей}
\scnsuperset{действие. сформировать структуру, описывающую различия указываемых сущностей}
\end{scnindent}
\scnsuperset{действие. сформировать структуру, описывающую способ решения указываемой задачи}
\scnsuperset{действие. сформировать план генерации ответа на указанный вопрос}
\scnsuperset{действие. сформировать протокол выполнения указываемого действия}
\scnsuperset{действие. сформировать обоснование корректности указываемого решения}
\scnsuperset{действие. верифицировать обоснование корректности указываемого решения}
\scnsuperset{действие, направленное на установление темпоральных характеристик указываемой сущности}
\scnsuperset{действие, направленное на установление пространственных характеристик указываемой сущности}
\end{scnindent}

\scnheader{действие редактирования базы знаний}
\scnsuperset{действие. изменить направление указанной sc-дуги}
\scnsuperset{действие. исправить ошибки в заданной структуре}
\scnsuperset{действие. преобразовать указанную структуру в соответствии с указанным правилом}
\scnsuperset{действие. отождествить два указанных sc-элемента}
\scnsuperset{действие. включить множество}
\begin{scnindent}
\scnidtf{сделать все элементы множества \textbf{\textit{Si}} явно принадлежащими множеству \textbf{\textit{Sj}}, то есть сгенерировать соответствующие sc-дуги принадлежности}
\end{scnindent}
\scnsuperset{действие генерации sc-элементов}
\begin{scnindent}
\scnsuperset{действие генерации, одним из аргументов которого является некоторая обобщенная структура}
\begin{scnindent}
\scnsuperset{действие. сгенерировать структуру, изоморфную указываемому образцу}
\end{scnindent}
\scnsuperset{действие. сгенерировать sc-элемент указанного типа}
\begin{scnindent}
\scnsuperset{действие. сгенерировать sc-коннектор указанного типа}
\scnsuperset{действие. сгенерировать sc-узел указанного типа}
\end{scnindent}
\scnsuperset{действие. сгенерировать файл с заданным содержимым}
\scnsuperset{действие. установить указанный файл в качестве основного идентификатора указанного sc-элемента для указанного внешнего языка}
\end{scnindent}
\scnsuperset{действие. обновить понятия}
\begin{scnindent}
\scnidtf{действие. заменить неосновные понятия на их определения через основные понятия}
\scnidtf{действие. заменить некоторое множество понятий на другое множество понятий}
\end{scnindent}
\scnsuperset{действие. интегрировать информационную конструкцию в текущее состояние базы знаний}
\begin{scnindent}
\scnsuperset{действие. интегрировать содержимое указанного файла в текущее состояние базы знаний}
\begin{scnindent}
\scnsuperset{действие. протранслировать содержимое указанного файла в sc-память}
\end{scnindent}
\scnsuperset{действие. интегрировать указанную структуру в текущее состояние базы знаний}
\end{scnindent}
\scnsuperset{действие. дополнить описание прошлого состояния ostis-системы}
\begin{scnindent}
\scnsuperset{действие. дополнить структуру, описывающую историю эволюции ostis-системы}
\scnsuperset{действие. дополнить структуру, описывающую историю эксплуатации ostis-системы}
\end{scnindent}

\scnsuperset{действие удаления sc-элементов}
\begin{scnindent}
\scnsuperset{действие. удалить указанные sc-элементы}
\begin{scnindent}
\scnsuperset{действие. удалить sc-элементы, входящие в состав указанной структуры и не являющиеся ключевыми узлами каких-либо sc-агентов}
\end{scnindent}
\end{scnindent}

\scnheader{действие. отождествить два указанных sc-элемента}
\scnidtf{действие. совместить два указанных sc-элемента}
\scnidtf{действие. склеить два указанных sc-элемента}
\begin{scnrelfromset}{разбиение}
	\scnitem{действие. физически отождествить два указанных sc-элемента}
	\scnitem{действие. логически отождествить два указанных sc-элемента}
\end{scnrelfromset}
\end{SCn}

Каждое \textbf{\textit{действие. отождествить два указанных sc-элемента}} может быть выполнено как \textit{действие. физически отождествить два указанных sc-элемента} или \textit{действие. логически отождествить два указанных sc-элемента}. В случае логического отождествления в протоколе деятельности агентов сохраняется само действие с его спецификацией, включающей обязательное указание того, какие элементы были сгенерированы, а какие удалены. В случае физического отождествления протокол действия не сохраняется.

Каждое \textbf{\textit{действие. обновить понятия}} обозначает переход от какой-то группы понятий, использовавшихся ранее, к другой группе понятий, которые будут использоваться вместо первых, и станут \textit{основными понятиями}.
В общем случае \textbf{\textit{действие. обновить понятия}} состоит из следующих этапов:

\begin{itemize}
	\item Определить заменяемые понятия на основе заменяющих;
	\item Внести соответствующие изменения в программы sc-агентов, ключевыми узлами которых являются обновляемые понятия;
	\item Заменить все конструкции в базе знаний, содержащие заменяемые понятия, в соответствии с определениями этих понятий через заменяющие их понятия;
	\item При необходимости,\textit{ sc-элементы}, обозначающие замененные таким образом понятия, могут быть полностью выведены из текущего состояния базы знаний.
\end{itemize}

Первым аргументом (входящим в знак \textit{действия} под атрибутом \textit{1\scnrolesign}) \textbf{\textit{действия. обновить понятия}} является знак множества \textit{sc-узлов}, обозначающих заменяемые понятия, вторым (входящим в знак \textit{действия} под атрибутом \textit{2\scnrolesign}) - знак множества \textit{sc-узлов}, обозначающих заменяющие понятия. В общем случае любое или оба этих множества могут быть \textit{синглетонами}.

\begin{SCn}
\scnheader{действие. удалить указанные sc-элементы}
\begin{scnrelfromset}{разбиение}
	\scnitem{действие. физически удалить указанные sc-элементы}
	\scnitem{действие. логически удалить указанные sc-элементы}
\end{scnrelfromset}
\end{SCn}

Каждое \textbf{\textit{действие. удалить указанные sc-элементы}} может быть выполнено как \textit{действие. физически удалить указанные sc-элементы} или \textit{действие. логически удалить указанные sc-элементы}. В случае логического удаления в протоколе деятельности агентов сохраняется само действие с его спецификацией, включающей обязательное указание того, какие элементы были удалены, т.е. по сути, элементы просто исключаются из текущего состояния базы знаний. В случае физического удаления протокол действия не сохраняется.
	
В случае удаления какого-либо \textit{sc-элемента}, инцидентные ему \textit{связки}, в том числе \textit{sc-коннекторы}, также удаляются.

Для того, чтобы выполнить \textbf{\textit{действие. интегрировать указанную структуру в текущее состояние базы знаний}}, необходимо склеить \textit{sc-элементы}, входящие в интегрируемую \textit{структуру} с синонимичными им \textit{sc-элементами}, входящими в текущее состояние базы знаний, заменить неиспользуемые (например, устаревшие) понятия, входящие в интегрируемую \textit{структуру}, на используемые (т.е. заменить неиспользуемые понятия на их определения через используемые), явно включить все элементы интегрируемой \textit{структуры} в число элементов утвержденной части базы знаний и явно включить все элементы интегрируемой \textit{структуры} в число элементов одного из атомарных разделов утвержденной части базы знаний.

\begin{SCn}
\scnheader{задача, решаемая в sc-памяти}
\scnsubset{задача}
\scnidtf{спецификация действия, выполняемого в sc-памяти}
\scnidtf{структура, являющая таким описанием (постановкой, заданием) соответствующего действия в sc-памяти, которое обладает достаточной полнотой для выполения указанного действия}
\scnidtf{семантическая окрестность некоторого действия в sc-памяти, обеспечивающая достаточно полное задание этого действия}
\end{SCn}

\begin{SCn}
\scnheader{класс действий}
\scnsuperset{класс действий в sc-памяти}
\begin{scnindent}
\scnrelto{семейство подмножеств}{действие в sc-памяти}
\end{scnindent}

\begin{scnrelfromset}{разбиение}

\scnitem{класс логически атомарных действий}
	\begin{scnindent}
	\scnidtf{класс автономных действий}
	\scnsuperset{класс логически атомарных действий в sc-памяти}
	\end{scnindent}
	\scnitem{класс логически неатомарных действий}
	\begin{scnindent}
	\scnidtf{класс неавтономных действий}
	\end{scnindent}
\end{scnrelfromset}
\end{SCn}

Каждое \textit{действие}, принадлежащее некоторому конкретному \textit{классу логически атомарных действий}, обладает двумя необходимыми свойствами:
\begin{itemize}
	\item выполнение действия не зависит от того, является ли указанное действие частью декомпозиции более общего действия. При выполнении данного действия также не должен учитываться тот факт, что данное действие предшествует каким-либо другим действиям или следует за ними (что явно указывается при помощи отношения \textit{последовательность действий*});
	\item указанное действие должно представлять собой логически целостный акт преобразования, например, в семантической памяти. Такое действие по сути является транзакцией, т. е. результатом такого преобразования становится новое состояние преобразуемой системы, а выполняемое действие должно быть либо выполнено полностью, либо не выполнено совсем, частичное выполнение не допускается. 
\end{itemize}
	
В то же время логическая атомарность не запрещает декомпозировать выполняемое действие на более частные, каждое из которых, в свою очередь, также будет являться логически атомарным.
	
На логически атомарные действия предлагается делить всю деятельность, направленную на решение каких-либо задач ostis-системой. Соответственно \textit{решатель задач ostis-системы} предлагается делить на компоненты, соответствующие таким \textit{классам логически атомарных действий в sc-памяти}, что является основой для обеспечения его \textit{модифицируемости}.

\subsection{Понятие sc-агента и абстрактного sc-агента}

\begin{SCn}
\scnheader{sc-агент}
\scnidtf{единственный вид \textit{субъектов}, выполняющих преобразования в \textit{\textit{sc-памяти}}}
\scnidtf{\textit{субъект}, способный выполнять \textit{действия в sc-памяти}, принадлежащие некоторому определенному \textit{классу логически атомарных действий}}
\end{SCn}

Логическая атомарность выполняемых sc-агентом действий предполагает, что каждый sc-агент реагирует на соответствующий ему класс ситуаций и/или событий, происходящих в sc-памяти, и осуществляет определенное преобразование sc-текста, находящегося в семантической окрестности обрабатываемой ситуации и/или события. При этом каждый sc-агент в общем случае не имеет информацию о том, какие еще sc-агенты в данный момент присутствуют в системе и осуществляет взаимодействие в другими sc-агентами исключительно посредством формирования некоторых конструкций (как правило – спецификаций действий) в общей sc-памяти. Таким сообщением может быть, например, вопрос, адресованный другим sc-агентам в системе (заранее не известно, каким конкретно), или ответ на поставленный другими sc-агентами вопрос (заранее не известно, каким конкретно). Таким образом, каждый sc-агент в каждый момент времени контролирует только фрагмент базы знаний в контексте решаемой данным агентом задачи, состояние всей остальной базы знаний в общем случае непредсказуемо для sc-агента.

Поскольку предполагается, что копии одного и того же \textit{sc-агента} или функционально эквивалентные \textit{sc-агенты} могут работать в разных ostis-системах, будучи при этом физически разными sc-агентами, то целесообразно рассматривать свойства и классификацию не sc-агентов, а классов функционально эквивалентных sc-агентов, которые будем называть \textbf{\textit{абстрактными sc-агентами}}.
Под \textbf{\textit{абстрактным sc-агентом}} понимается некоторый класс функционально эквивалентных \textit{sc-агентов}, разные экземпляры (т.е. представители) которого могут быть реализованы по-разному.
	
Каждый \textbf{\textit{абстрактный sc-агент}} имеет соответствующую ему спецификацию. В спецификацию каждого \textbf{\textit{абстрактного sc-агента}} входит:
\begin{itemize}
	\item указание ключевых \textit{sc-элементов} этого \textit{sc-агента}, т.е. тех \textit{sc-элементов}, хранимых в \textit{sc-памяти}, которые для данного \textit{sc-агента} являются «точками опоры»;
	\item формальное описание условий инициирования данного \textit{sc-агента}, т.е. тех \textit{ситуация} в \textit{sc-памяти}, которые инициируют деятельность данного \textit{sc-агента};
	\item формальное описание первичного условия инициирования данного \textit{sc-агента}, т.е. такой ситуации в \textit{sc-памяти}, которая побуждает \textit{sc-агента} перейти в активное состояние и начать проверку наличия своего полного условия инициирования (для \textit{внутренних абстрактных sc-агентов});
	\item строгое, полное, однозначно понимаемое описание деятельности данного \textit{sc-агента}, оформленное при помощи каких-либо понятных, общепринятых средств, не требующих специального изучения, например на естественном языке.
	\item описание результатов выполнения данного \textit{sc-агента}.
\end{itemize}

Рассмотрим классификацию абстрактных sc-агентов:

\begin{SCn}
\scnheader{абстрактный sc-агент}
\begin{scnrelfromset}{разбиение}
	\scnitem{неатомарный абстрактный sc-агент}
	\scnitem{атомарный абстрактный sc-агент}
\end{scnrelfromset}
\begin{scnrelfromset}{разбиение}
	\scnitem{внутренний абстрактный sc-агент}
	\scnitem{эффекторный абстрактный sc-агент}
	\scnitem{рецепторный абстрактный sc-агент}
\end{scnrelfromset}
\begin{scnrelfromset}{разбиение}
	\scnitem{абстрактный sc-агент, не реализуемый на Языке SCP}
	\scnitem{абстрактный sc-агент, реализуемый на Языке SCP}
\end{scnrelfromset}
\begin{scnrelfromset}{разбиение}
	\scnitem{абстрактный sc-агент интерпретации scp-программ}
	\scnitem{абстрактный программный sc-агент}
	\scnitem{абстрактный sc-метаагент}
\end{scnrelfromset}

\begin{scnrelfromset}{разбиение}	
	\scnitem{платформенно-зависимый абстрактный sc-агент}
	\begin{scnindent}
		\scnsuperset{абстрактный sc-агент, не реализуемый на Языке SCP}
	\end{scnindent}
	\scnitem{платформенно-независимый абстрактный sc-агент}
\end{scnrelfromset}

\scnheader{абстрактный sc-агент, не реализуемый на Языке SCP}
\scnidtf{абстрактный sc-агент, который не может быть реализован на платформенно-независимом уровне}
\begin{scnrelfromset}{разбиение}
	\scnitem{эффекторный абстрактный sc-агент}
	\scnitem{рецепторный абстрактный sc-агент}
	\scnitem{абстрактный sc-агент интерпретации scp-программ}
\end{scnrelfromset}

\scnheader{абстрактный sc-агент, реализуемый на Языке SCP}
\scnidtf{абстрактный sc-агент, который может быть реализован на платформенно-независимом уровне}
\begin{scnrelfromset}{разбиение}
	\scnitem{абстрактный sc-метаагент}
	\scnitem{абстрактный программный sc-агент, реализуемый на Языке SCP}
\end{scnrelfromset}

\scnheader{абстрактный программный sc-агент}
\begin{scnrelfromset}{разбиение}
	\scnitem{эффекторный абстрактный sc-агент}
	\scnitem{рецепторный абстрактный sc-агент}
	\scnitem{абстрактный программный sc-агент, реализуемый на Языке SCP}
\end{scnrelfromset}	

\scnheader{атомарный абстрактный sc-агент}
\begin{scnrelfromset}{разбиение}
	\scnitem{платформенно-независимый абстрактный sc-агент}
	\scnitem{платформенно-зависимый абстрактный sc-агент}
\end{scnrelfromset}	
\end{SCn}

Под \textbf{\textit{неатомарным абстрактным sc-агентом}} понимается \textit{абстрактный sc-агент}, который декомпозируется на коллектив более простых \textit{абстрактных sc-агентов}, каждый из которых в свою очередь может быть как \textit{атомарным абстрактным sc-агентом}, так и \textbf{\textit{неатомарным абстрактным sc-агентом}}. При этом в каком либо варианте \textit{декомпозиции абстрактного sc-агента*} дочерний \textbf{\textit{неатомарный абстрактный sc-агент}} может стать \textit{атомарным абстрактным sc-агентом}, и реализовываться соответствующим образом.

Под \textbf{\textit{атомарным абстрактным sc-агентом}} понимается \textit{абстрактный sc-агент}, для которого уточняется платформа его реализации, т.е. существует соответствующая связка отношения \textit{программа sc-агента*}.

К \textbf{\textit{платформенно-независимым абстрактным \mbox{sc-агентам}}} относят \textit{атомарные абстрактные sc-агенты}, реализованные на базовом языке программирования Технологии OSTIS, т.е. на \textit{Языке SCP}.
	
При описании \textbf{\textit{платформенно-независимых абстрактных sc-агентов}} под платформенной независимостью понимается платформенная независимость с точки зрения Технологии OSTIS, т.е реализация на специализированном языке программирования, ориентированном на обработку семантических сетей (\textit{Языке SCP}), поскольку \textit{атомарные sc-агенты}, реализованные на указанном языке могут свободно переноситься с одной платформы интерпретации \textit{sc-моделей} на другую. При этом языки программирования, традиционно считающиеся платформенно-независимыми в данном случае не могут считаться таковыми.
	
Существуют \textit{sc-агенты}, которые принципиально не могут быть реализованы на платформенно-независимом уровне, например, собственно \textit{sc-агенты} интерпретации \textit{sc-моделей} или рецепторные и эффекторные \textit{sc-агенты}, обеспечивающие взаимодействие с внешней средой.

К \textbf{\textit{платформенно-зависимым абстрактным sc-агентам}} относят \textit{атомарные абстрактные sc-агенты}, реализованные ниже уровня sc-моделей, т.е. не на \textit{Языке SCP}, а на каком-либо другом языке описания программ.
	
Существуют \textit{sc-агенты}, которые принципиально должны быть реализованы на платформенно-зависимом уровне, например, собственно \textit{sc-агенты} интерпретации \textit{sc-моделей} или рецепторные и эффекторные \textit{sc-агенты}, обеспечивающие взаимодействие с внешней средой.

Каждый \textbf{\textit{внутренний абстрактный sc-агент}} обозначает класс \textit{sc-агентов}, которые реагируют на события в \textit{sc-памяти} и осуществляют преобразования исключительно в рамках этой же \textit{sc-памяти}.

Каждый \textbf{\textit{эффекторный абстрактный sc-агент}} обозначает класс \textit{sc-агентов}, которые реагируют на события в \textit{sc-памяти} и осуществляют преобразования во внешней относительно данной \textit{ostis-системы} среде.

Каждый \textbf{\textit{рецепторный абстрактный sc-агент}} обозначает класс \textit{sc-агентов}, которые реагируют на события во внешней относительно данной \textit{ostis-системы} среде и осуществляют преобразования в памяти данной системы.

Каждый \textbf{\textit{абстрактный sc-агент, не реализуемый на Языке SCP}} должен быть реализован на уровне платформы интерпретации sc-моделей, в том числе, аппаратной. К таким \textit{абстрактным sc-агентам} относятся абстрактные sc-агенты интерпретации scp-программ, а также эффекторные и рецепторные абстрактные sc-агенты.

Каждый \textbf{\textit{абстрактный sc-агент, реализуемый на Языке SCP}} может быть реализован на Языке SCP, то есть платформенно-независимом уровне, но при необходимости, может реализовываться и на уровне платформы, например, с целью повышения производительности.

К \textbf{\textit{абстрактным sc-агентам интерпретации scp-программ}} относятся не реализуемые на платформенно-независимом уровне \textit{абстрактные sc-агенты}, обеспечивающие интерпретацию \textit{scp-программ} и \textit{\mbox{scp-метапрограмм}}, в том числе создание \textit{scp-процессов}, собственно интерпретацию \textit{scp-операторов}, а также другие вспомогательные действия. По сути, агенты данного класса обеспечивают работу sc-агентов более высоких уровней (программных sc-агентов и sc-метаагентов), реализованных на Языке SCP, в частности, обеспечивают соблюдение указанными агентами общих принципов синхронизации.

К \textbf{\textit{абстрактным программным sc-агентам}} относятся все \textit{абстрактные sc-агенты}, обеспечивающие основной функционал системы, то есть ее возможность решать те или иные задачи. Агенты данного класса должны работать в соответствии с общими принципами синхронизации деятельности субъектов в sc-памяти.

Задачей \textbf{\textit{абстрактных sc-метаагентов}} является координация деятельности \textit{абстрактных программных sc-агентов}, в частности, решение проблемы взаимоблокировок. Агенты данного класса могут быть реализованы на Языке SCP, однако для синхронизации их деятельности используются другие принципы, соответственно, для реализации таких агентов требуется Язык SCP другого уровня, типология операторов которого полностью аналогична типологии scp-операторов, однако эти операторы имеют другую операционную семантику, учитывающую отличия в принципах синхронизации (работы с \textit{блокировками*}). Программы такого языка будем называть \textit{scp-метапрограммами}, соответствующие им \mbox{\textit{процессы в sc-памяти} -- \textit{scp-метапроцессами}}, операторы -- \textit{scp-метаоператорами}.

%TODO учесть два разных разделения агентов по уровням

\begin{SCn}
\scnheader{декомпозиция абстрактного sc-агента*}
\scniselement{отношение декомпозиции}
\end{SCn}

Отношение \textbf{\textit{декомпозиции абстрактного sc-агента*}} трактует \textit{неатомарные абстрактные sc-агенты} как коллективы более простых \textit{абстрактных sc-агентов}, взаимодействующих через \textit{sc-память}.
	
Другими словами, \textbf{\textit{декомпозиция абстрактного sc-агента*}} на \textit{абстрактные sc-агенты} более низкого уровня уточняет один из возможных подходов к реализации этого \textit{абстрактного sc-агента} путем построения коллектива более простых \textit{абстрактных sc-агентов}.

\begin{SCn}
\scnheader{sc-агент}
\scnidtf{агент над sc-памятью}
\scnsubset{субъект}
\scnrelfrom{семейство подмножеств}{абстрактный sc-агент}
\end{SCn}

Под \textbf{\textit{sc-агентом}} понимается конкретный экземпляр (с теоретико-множественной точки зрения - элемент) некоторого \textit{атомарного абстрактного sc-агента}, работающий в какой-либо конкретной интеллектуальной системе.
	
Таким образом, каждый \textit{sc-агент} - это субъект, способный выполнять некоторый класс однотипных действий либо только над \textit{sc-памятью}, либо над sc-памятью и внешней средой (для эффекторных \textit{sc-агентов}). Каждое такое действие инициируется либо состоянием или ситуацией в sc-памяти, либо состоянием или ситуацией во внешней среде (для рецепторных sc-агентов-датчиков),  соответствующей условию инициирования \textit{атомарного абстрактного sc-агента}, экземпляром которого является заданный \textit{sc-агент}. В данном случае можно провести аналогию между принципами объектно-ориентированного программирования, рассматривая \textit{атомарный абстрактный sc-агент} как класс, а конкретный \textit{sc-агент} – как экземпляр, конкретную имплементацию этого класса.
	
Взаимодействие \textit{sc-агентов} осуществляется только через \textit{sc-память}. Как следствие, результатом работы любого \textit{sc-агента} является некоторое изменение состояния \textit{sc-памяти}, т.е. удаление либо генерация каких-либо \textit{sc-элементов}.
	
В общем случае один \textit{sc-агент} может явно передать управление другому \textit{sc-агенту}, если этот \textit{sc-агент} априори известен. Для этого каждый \textit{sc-агент} в \textit{sc-памяти} имеет обозначающий его \textit{sc-узел}, с которым можно связать конкретную ситуацию в текущем состоянии базы знаний, которую инициируемый \textit{sc-агент} должен обработать.
	
Однако далеко не всегда легко определить того \textit{sc-агента}, который должен принять управление от заданного \textit{sc-агента}, в связи с чем описанная выше ситуация возникает крайне редко. Более того, иногда условие инициирования \textit{sc-агента} является результатом деятельности непредсказуемой группы \textit{sc-агентов}, равно как и одна и та же конструкция может являться условием инициирования целой группы \textit{sc-агентов}.

При этом общаются через \textit{sc-память} не \textit{программы sc-агентов*}, а сами описываемые данными программами \textit{sc-агенты}.

В процессе работы \textit{sc-агент} может сам для себя порождать вспомогательные \textit{sc-элементы}, которые сам же удаляет после завершения акта своей деятельности (это вспомогательные \textit{структуры}, которые используются в качестве "информационных лесов"{} только в ходе выполнения соответствующего акта деятельности и после завершения этого акта удаляются).

\begin{SCn}
\scnheader{sc-агент}
\scnsuperset{активный sc-агент}
\begin{scnrelfromlist}{первый домен}
	\scnitem{ключевые sc-элементы sc-агента*}
	\scnitem{программа sc-агента*}	
	\scnitem{первичное условие инициирования*}
	\scnitem{условие инициирования и результат*}
\end{scnrelfromlist}
\end{SCn}

Под \textbf{\textit{активным sc-агентом}} понимается \textit{sc-агент} ostis-системы, который реагирует на события, соответствующие его условию инициирования, и, как следствие, его \textit{первичному условию инициирования*}. Не входящие во множество \textbf{\textit{активных sc-агентов}} \textit{sc-агенты} не реагируют ни на какие события в \textit{sc-памяти}.

Связки отношения \textbf{\textit{ключевые sc-элементы sc-агента*}} связывают между собой \textit{sc-узел}, обозначающий \textit{абстрактный sc-агент} и \textit{sc-узел}, обозначающий множество \textit{sc-элементов}, которые являются ключевыми для данного \textit{абстрактного sc-агента}, то данные \textit{sc-элементы} явно упоминаются в рамках программ, реализующих данный \textit{абстрактный sc-агент}.

Связки отношения \textbf{\textit{программа sc-агента*}} связывают между собой \textit{sc-узел}, обозначающий \textit{атомарный абстрактный sc-агент} и \textit{sc-узел}, обозначающий множество программ, реализующих указанный \textit{атомарный абстрактный sc-агент}. В случае \textit{платформенно-независимого абстрактного sc-агента} каждая связка отношения \textit{программа sc-агента*} связывает \textit{sc-узел}, обозначающий указанный \textit{абстрактный sc-агент} с множеством \textit{scp-программ}, описывающих деятельность данного \textit{абстрактного sc-агента}. Данное множество содержит одну \textit{агентную scp-программу}, и произвольное количество (может быть, и ни одной) \textit{scp-программ}, которые необходимы для выполнения указанной \textit{агентной scp-программы}.
	
В случае \textit{платформенно-зависимого абстрактного sc-агента} каждая связка отношения \textit{программа \mbox{sc-агента*}} связывает \textit{sc-узел}, обозначающий указанный \textit{абстрактный sc-агент} с множеством файлов, содержащих исходные тексты программы на некотором внешнем языке программирования, реализующей деятельность данного \textit{абстрактного sc-агента}.

Связки отношения \textbf{\textit{первичное условие инициирования*}} связывают между собой \textit{sc-узел}, обозначающий \textit{абстрактный sc-агент} и бинарную ориентированную пару, описывающую первичное условие инициирования данного \textit{абстрактного sc-агента}, т.е. такой спецификацию \textit{ситуации} в \textit{sc-памяти}, возникновение которой побуждает \textit{sc-агента} перейти в активное состояние и начать проверку наличия своего полного условия инициирования.
	
Первым компонентом данной ориентированной пары является знак некоторого класса \textit{элементарных событий в sc-памяти*}, например, \textit{событие добавления sc-дуги, выходящей из заданного sc-элемента*}.

Вторым компонентом данной ориентированной пары является произвольный в общем случае \textit{sc-элемент}, с которым непосредственно связан указанный тип события в \textit{sc-памяти}, т.е., например, \textit{sc-элемент}, из которого выходит либо в который входит генерируемая либо удаляемая \textit{sc-дуга}, либо \textit{файл}, содержимое которого было изменено.

После того, как в \textit{sc-памяти} происходит некоторое событие, активизируются все \textit{активные sc-агенты}, \textbf{\textit{первичное условие инициирования*}} которых соответствует произошедшему событию.

Связки отношения \textbf{\textit{условие инициирования и результат*}} связывают между собой \textit{sc-узел}, обозначающий \textit{абстрактный sc-агент} и бинарную ориентированную пару, связывающую условие инициирования данного \textit{абстрактного sc-агента} и результаты выполнения данного экземпляров данного \textit{sc-агента} в какой-либо конкретной системе.
	
Указанную ориентированную пару можно рассматривать как логическую связку импликации, при этом на \textit{sc-переменные}, присутствующие в обеих частях связки, неявно накладывается квантор всеобщности, на \textit{sc-переменные}, присутствующие либо только в посылке, либо только в заключении неявно накладывается квантор существования.

Первым компонентом указанной ориентированной пары является логическая формула, описывающая условие инициирования описываемого \textit{абстрактного sc-агента}, то есть конструкции, наличие которой в \textit{sc-памяти} побуждает \textit{sc-агент} начать работу по изменению состояния \textit{sc-памяти}. Данная логическая формула может быть как атомарной, так и неатомарной, в которой допускается использование любых связок логического языка.

Вторым компонентом указанной ориентированной пары является логическая формула, описывающая возможные результаты выполнения описываемого абстрактного \textit{sc-агента}, то есть описание произведенных им изменений состояния \textit{sc-памяти}. Данная логическая формула может быть как атомарной, так и неатомарной, в которой допускается использование любых связок логического языка.

\begin{SCn}
\scnheader{описание поведения sc-агента}
\scnsubset{семантическая окрестность}
\end{SCn}

\textbf{\textit{Описание поведения sc-агента}} представляет собой \textit{семантическую окрестность}, описывающую деятельность \textit{sc-агента} до какой-либо степени детализации, однако такое описание должно быть строгим, полным и однозначно понимаемым. Как любая другая \textit{семантическая окрестность}, \textbf{\textit{описание поведения sc-агента}} может быть протранслировано на какие-либо понятные, общепринятые средства, не требующие специального изучения, например на естественный язык.

Описываемый \textit{абстрактный sc-агент} входит в соответствующее \textbf{\textit{описание поведения sc-агента}} под атрибутом \textit{ключевой sc-элемент\scnrolesign}.

\subsection{Принципы синхронизации деятельности sc-агентов}

Понятия \textit{действие в sc-памяти}, и \textit{процесс в sc-памяти} (информационный процесс, выполняемый агентом в семантической памяти), являются синонимичными, поскольку все процессы, протекающие в sc-памяти, являюся осознанными и выполняются каким-либо sc-агентами. Тем не менее, когда идет речь о синхронизации выполнения каких-либо преобразований в памяти компьютерной системы, в литературе принято использовать именно термины ``процесс'', ``взаимодействие процессов'' \cite{Dijkstra1972,Hoare1989}, в связи с чем будем использовать этот термин при описании принципов синхронизации деятельности sc-агентов при выполнении ими параллельных процессов в sc-памяти.

\begin{SCn}
\scnheader{процесс в sc-памяти}
\begin{scnrelfromset}{разбиение}
	\scnitem{процесс в sc-памяти, соответствующий платформенно-зависимому sc-агенту}
	\scnitem{scp-процесс}	
	\begin{scnindent}
	\begin{scnrelfromset}{разбиение}
		\scnitem{scp-процесс, не являющийся scp-метапроцессом}
		\scnitem{scp-метапроцесс}	
	\end{scnrelfromset}
	\end{scnindent}
\end{scnrelfromset}

\scnheader{процесс в sc-памяти, соответствующий платформенно-зависимому sc-агенту}
	\begin{scnrelfromset}{разбиение}
	\scnitem{процесс в sc-памяти, соответствующий платформенно-зависимому sc-агенту и не являющийся действием абстрактной scp-машины}
	\scnitem{действие абстрактной scp-машины}	
	\begin{scnindent}
		\scnsuperset{действие интерпретации scp-программы}
	\end{scnindent}
\end{scnrelfromset}
\end{SCn}

Для синхронизации выполнения \textit{процессов в sc-памяти} используется механизм блокировок. Отношение \textbf{\textit{блокировка*}} связывает знаки \textit{действий в sc-памяти} со знаками \textit{структур} (ситуативных), которые содержат элементы, заблокированные на время выполнения данного действия или на какую-то часть этого периода. Каждая такая \textit{структура} принадлежит какому-либо из \textit{типов блокировки}.

Первым компонентом связок отношения \textbf{\textit{блокировка*}} является знак \textit{действия в sc-памяти}, вторым – знак заблокированной \textit{структуры}.

\begin{SCn}	
\scnheader{блокировка*}
\scniselement{бинарное отношение}

\scnheader{тип блокировки}
\scnhaselement{полная блокировка}
\scnhaselement{блокировка на любое изменение}
\scnhaselement{блокировка на удаление}
\end{SCn}	

\begin{figure}[h]
	\centering
	\includegraphics[scale=0.8]{images/part3/chapter_situation_management/lock.png}
	\caption{Пример использования блокировок}
	\label{fig:lock}
\end{figure}

Множество \textbf{\textit{тип блокировки}} содержит все возможные классы блокировок, т.е. \textit{структуры}, содержащие \textit{sc-элементы}, заблокированные каким-либо \textit{sc-агентом} на время выполнения им некоторого \textit{действия в sc-памяти}.

Каждая \textit{структура}, принадлежащая множеству \textbf{\textit{полная блокировка}} содержит \textit{sc-элементы}, просмотр и изменение (удаление, добавление инцидентных \textit{sc-коннекторов}, удаление самих \textit{sc-элементов}, изменение содержимого в случае файла) которых запрещены всем \textit{sc-агентам}, кроме собственно \textit{sc-агента}, выполняющего соответствующее данной структуре \textit{действие в sc-памяти}, связанное с ней отношением \textit{блокировка*}.
	
Для того, чтобы исключить возможность реализации \textit{sc-агентов}, которые могут внести изменения в конструкции, описывающие блокировки других \textit{sc-агентов}, все элементы этих конструкций, в том числе, сам знак \textit{структуры}, содержащей заблокированные \textit{sc-элементы} (принадлежащей как множеству \textbf{\textit{полная блокировка}}, так и любому другому \textit{типу блокировки}) и связки отношения \textit{блокировка*}, связывающие эту \textit{структуру} и конкретное \textit{действие в sc-памяти}, добавляются в \textbf{\textit{полную блокировку}}, соответствующую данному \textit{действию в sc-памяти}. Таким образом, каждой \textbf{\textit{полной блокировке}} соответствует петля принадлежности, связывающая ее знак с самим собой.

Каждая \textit{структура}, принадлежащая множеству \textbf{\textit{блокировка на любое изменение}} содержит \textit{sc-элементы}, изменение (физическое удаление, добавление инцидентных \textit{sc-коннекторов}, физическое удаление самих \textit{\mbox{sc-элементов}}, изменение содержимого в случае файла) которых запрещено всем \textit{sc-агентам}, кроме собственно \textit{sc-агента}, выполняющего соответствующее данной структуре \textit{действие в sc-памяти}, связанное с ней отношением \textit{блокировка*}. Однако не запрещен просмотр (чтение) этих \textit{sc-элементов} любым \textit{sc-агентом}.

Каждая \textit{структура}, принадлежащая множеству \textbf{\textit{блокировка на удаление}} содержит \textit{sc-элементы}, удаление которых запрещено всем \textit{sc-агентам}, кроме собственно \textit{sc-агента}, выполняющего соответствующее данной структуре \textit{действие в sc-памяти}, связанное с ней отношением \textit{блокировка*}. Однако не запрещен просмотр (чтение) этих \textit{sc-элементов} любым \textit{sc-агентом}, добавление инцидентных sc-коннекторов.

Рассмотрим принципы работы с блокировками:

\begin{itemize}
	\item в каждый момент времени одному процессу в sc-памяти может соответствовать только одна блокировка каждого типа;
	\item в каждый момент времени одному процессу в sc-памяти может соответствовать только одна блокировка, установленная на некоторый конкретный sc-элемент;
	\item при завершении выполнения любого процесса в sc-памяти все установленные им блокировки автоматически снимаются;
	\item для повышения эффективности работы системы в целом каждый процесс должен в каждый момент времени блокировать минимально необходимое множество sc-элементов, снимая блокировку с каждого sc-элемента сразу же, как это становится возможным (безопасным);
	\item В случае когда в рамках \textit{процесса в sc-памяти} явно выделяются более частные подпроцессы (при помощи отношений \textit{темпоральная часть*, поддействие*, декомпозиция действия*} и т. д.), то каждый такой подпроцесс с точки зрения синхронизации выполнения рассматривается как самостоятельный процесс, которому в соответствие могут быть поставлены все необходимые блокировки.
	\begin{itemize}
		\item все дочерние процессы в sc-памяти имеют доступ к блокировкам родительского процесса так же, как если бы это были блокировки соответствующие каждому из таких дочерних процессов;
		\item в свою очередь, родительский процесс не имеет какого-либо привилегированного доступа к sc-элементам, заблокированным дочерними процессами, и работает с ними так же, как любой другой процесс в sc-памяти. Исключение составляют sc-элементы, обозначающие сами дочерние процессы, поскольку родительский процесс должен иметь возможность управления дочерним, например, приостановки или прекращения их выполнения;
		\item все дочерние процессы по отношению друг к другу работают так же, как и по отношению к любым другим процессам;
		\item в случае, когда родительский процесс приостанавливает выполнение (становится \textit{отложенным действием}), \uline{все} его дочерние процессы также приостанавливают выполнение. В свою очередь, приостановка одного из дочерних процессов в общем случае не инициирует явно остановку всего родительского процесса и соответственно других дочерних.
		\end{itemize}
\end{itemize}

Рассмотрим принципы работы с \textit{полными блокировками}:
\begin{itemize}
	\item если sc-элемент, инцидентный некоторому sc-коннектору, попадает в какую-либо полную блокировку, то сам этот sc-коннектор по умолчанию также считается заблокированным этой же блокировкой. Обратное в общем случае неверно, т. к. часть sc-коннекторов, инцидентных некоторому sc-элементу, может быть полностью заблокирована, при этом сам этот элемент заблокирован не будет. Такая ситуация типична, например, для sc-узлов, обозначающих классы понятий;
	\item каждый процесс в sc-памяти может свободно изменять или удалять любые sc-элементы, попадающие в полную блокировку, соответствующую этому процессу.
\end{itemize}

Принципы работы с \textit{полными блокировками}, с одной стороны, наиболее просты, поскольку все процессы, кроме установившего такую блокировку, не имеют доступа к заблокированным \mbox{sc-элементам} и конфликты возникнуть не могут. С другой стороны, частое использование блокировок такого типа может привести к тому, что система не сможет использовать в полной мере имеющиеся у нее знания и давать неполные или даже некорректные ответы на поставленные вопросы.

Рассмотрим принципы работы с \textit{блокировками на любое изменение} и \textit{блокировками на удаление}:
\begin{itemize}
	\item на один и тот же sc-элемент в один момент времени может быть установлена только одна блокировка одного типа, но разные процессы могут одновременно установить на один и тот же элемент блокировки двух разных типов. Это касается случая, когда первый процесс установил на некоторый sc-элемент блокировку на удаление, а второй процесс затем устанавливает блокировку на любое изменение. В других случаях возникает конфликт блокировок;
	\item установка блокировки любого типа также считается изменением, таким образом, если на некоторый \mbox{sc-элемент} была установлена блокировка на любое изменение, то другой процесс не сможет установить на этот же sc-элемент блокировку любого типа, пока первый процесс не снимет свою;
	\item если блокировка на удаление устанавливается на некоторый sc-коннектор, то по умолчанию та же блокировка устанавливается на инцидентные этому sc-коннектору sc-элементы, поскольку удаление этих элементов приведет к удалению этого коннектора.
\end{itemize}

\begin{SCn}
\scnheader{процесс в sc-памяти}
\scnidtf{действие в sc-памяти}
\scnrelfrom{разбиение}{Классификация процессов в sc-памяти с точки зрения синхронизации их выполнения}
\begin{scnindent}
\begin{scneqtoset}
	\scnitem{действие поиска sc-элементов}
	\scnitem{действие генерации sc-элементов}
	\scnitem{действие удаления sc-элементов}
	\scnitem{действие установки блокировки некоторого типа на некоторый sc-элемент}
	\scnitem{действие снятия блокировки с некоторого sc-элемента}
\end{scneqtoset}
\end{scnindent}
\end{SCn}

В некоторых случаях для того, чтобы обеспечить синхронизацию, необходимо объединять несколько элементарных действий над sc-памятью в одно неделимое действие (\textbf{\textit{транзакцию в sc-памяти}}), для которого гарантируется, что ни один сторонний процесс не сможет прочитать или изменить участвующие в этом действии sc-элементы, пока действие не завершится. При этом, в отличие от ситуации с полной блокировкой, процесс, пытающийся получить доступ к таким элементам, не продолжает выполнение так, как если бы этих элементов просто не было в sc-памяти, а ожидает завершения транзакции, после чего может выполнять с данными элементами любые действия согласно общим принципам синхронизации процессов. Проблема обеспечения транзакций не может быть решена на уровне SC-кода и требует реализации таких неделимых действий на уровне \textit{платформы интерпретации sc-моделей}.

В случае выполнения \textit{действия поиска sc-элементов} все найденные и сохраненные в рамках какого-либо процесса sc-элементы попадают в соответствующую данному процессу \textit{блокировку на любое изменение}. Таким образом, гарантируется целостность фрагмента базы знаний, с которым работает некоторый процесс в sc-памяти. При этом поиск и автоматическая установка такой блокировки должны быть реализованы как \textit{транзакция в sc-памяти}.
	
Такой подход также позволяет избежать ситуации, когда один процесс заблокировал некоторый sc-элемент на любое изменение, а второй процесс пытается сгенерировать или удалить \textit{sc-коннектор}, инцидентный данному \textit{sc-элементу}. В таком случае второй процесс должен будет предварительно найти и заблокировать указанный \textit{sc-элемент} на любое изменение, что вызовет конфликт блокировок (\textit{взаимоблокировку*}).

В случае генерации любого sc-элемента в рамках некоторого процесса он автоматически попадает в полную блокировку, соответствующую данному процессу. При этом генерация и автоматическая установка такой блокировки должны быть реализованы как \textit{транзакция в sc-памяти}. При необходимости сгенерированные элементы могут быть удалены (т. е. их временное существование вообще никак не отразится на деятельности других процессов) или разблокированы в случае, когда сгенерирована информация, которая может иметь некоторую ценность в дальнейшем.

В случае если какой-либо процесс пытается установить блокировку любого типа на какой-либо sc-элемент, уже заблокированный каким-либо другим процессом, то, с одной стороны, блокировка не может быть установлена, пока другой процесс не разблокирует указанный sc-элемент; с другой стороны, для того чтобы обеспечить возможность поиска и устранения \textit{взаимоблокировок}, необходимо явно указывать тот факт, что какой-либо процесс хочет получить доступ к какому-либо заблокированному другим процессом sc-элементу. Для того чтобы иметь возможность указать, какие процессы пытаются заблокировать уже заблокированный \textit{sc-элемент}, предлагается наряду с отношением \textit{блокировка*} использовать отношение \textit{планируемая блокировка*}, полностью аналогичное отношению \textit{блокировка*}.
	
Описанный механизм регулирует также и процессы поиска, поскольку поиск и сохранение некоторого sc-элемента предполагает установку \textit{блокировки на любое изменение}. Кроме того, следует учитывать, что на один sc-элемент \textit{блокировка на любое изменение} может быть установлена после \textit{блокировки на удаление}, соответствующей другому процессу. В этом случае использовать отношение \textit{планируемые блокировки*} нет необходимости.
	
Действие проверки наличия на некотором sc-элементе блокировки и в зависимости от результата проверки, установки блокировки или планируемой блокировки (с указанием приоритета при необходимости) должно быть реализовано как транзакция.

\begin{SCn}
\scnheader{планируемая блокировка*}
\scnsubset{блокировка*}
\end{SCn}

Процесс, которому в соответствие поставлена \textit{планируемая блокировка*}, приостанавливает выполнение до тех пор, пока уже установленные блокировки не будут сняты, после чего \textit{планируемая блокировка*} становится реальной \textit{блокировкой*} и процесс продолжает выполнение в соответствии с общими правилами.

\begin{SCn}
\scnheader{приоритет блокировки*}
\scnrelfrom{область определения}{планируемая блокировка*}
\end{SCn}

В случае, когда на один и тот же sc-элемент планируют установить блокировку сразу несколько процессов, используется отношение \textit{приоритет блокировки*}, связывающее между собой пары отношения \textit{планируемая блокировка*}. Как правило, приоритет блокировки определяется тем, какой из процессов раньше попытался установить блокировку на рассматриваемый sc-элемент, хотя в общем случае приоритет может устанавливаться или меняться в зависимости от дополнительных критериев.

В случае попытки удаления некоторого sc-элемента некоторым процессом удаление может быть осуществлено только в случае, когда на данный sc-элемент не установлена (и не планируется) ни одна блокировка каким-либо другим процессом.
	
В других случаях необходимо обеспечить корректное завершение выполнения всех процессов, работающих с данным sc-элементом, и только потом удалить его физически.
	
Для реализации такой возможности каждому процессу в соответствие может быть поставлено множество удаляемых данным процессом sc-элементов.

Действие проверки наличия блокировок или планируемых блокировок на удаляемый sc-элемент и собственно его удаление или добавление во множество удаляемых sc-элементов для соответствующего процесса должно быть реализовано как транзакция.

\begin{SCn}
\scnheader{удаляемые sc-элементы*}
\scnrelfrom{первый домен}{процесс в sc-памяти}
\end{SCn}

Sc-элементы, попавшие во множество удаляемых sc-элементов некоторого процесса в sc-памяти, доступны процессам, уже установившим (или планирующим установить) на эти sc-элементы блокировки ранее (до попытки его удаления), а для всех остальных процессов эти sc-элементы уже считаются удаленными. Процесс, пытающийся удалить sc-элемент, приостанавливает свое выполнение до того момента, пока все заблокировавшие и планирующие заблокировать данный sc-элемент процессы не разблокируют его. В общем случае один sc-элемент может входить во множества удаляемых элементов одновременно для нескольких процессов, в этом случае все такие процессы одновременно продолжат выполнение после снятия с этого sc-элемента всех блокировок. Если удаление пытается осуществить один из процессов, уже установивший на указанный sc-элемент блокировку, то алгоритм действий остается прежним -- sc-элемент добавляется во множество удаляемых данным процессом sc-элементов, и будет физически удален, как только все остальные процессы, установившие на данный sc-элемент блокировки, снимут их.

Рассмотрим алгоритм снятия блокировки с некоторого sc-элемента:
\begin{enumerate}
	\item если на данный sc-элемент установлена одна или несколько \textit{планируемых блокировок*}, то первая из них по приоритету (или единственная) становится \textit{блокировкой*}, соответствующий ей процесс продолжает выполнение (становится настоящей сущностью); связка отношения приоритет выполнения, соответствовавшая удаленной связке отношения \textit{планируемая блокировка*} также удаляется, т. е. приоритет смещается на одну позицию;
	\item если \textit{планируемых блокировок*}, установленных на данный sc-элемент, нет, но он попадает во множество удаляемых sc-элементов для одного или нескольких процессов, то рассматриваемый sc-элемент физически удаляется, а приостановленные до его удаления процессы продолжают свое выполнение (становится настоящими сущностями);
	\item если на данный sc-элемент не установлены планируемые блокировки и он не входит во множество удаляемых для какого-либо процесса, то блокировка просто снимается без каких-либо дополнительных изменений.
\end{enumerate}

\begin{SCn}
\scnheader{транзакция в sc-памяти}
\begin{scnrelfromset}{разбиение}
	\scnitem{поиск некоторой конструкции в sc-памяти и автоматическая установка блокировки на любое изменение на найденные sc-элементы}
	\scnitem{генерация некоторого sc-элемента и автоматическая установка на него полной блокировки}
	\scnitem{проверка наличия на некотором sc-элементе блокировки и в зависимости от результата проверки установка блокировки или планируемой блокировки}	
	\scnitem{проверка наличия блокировок или планируемых блокировок на удаляемый sc-элемент и собственно его удаление или добавление во множество удаляемых sc-элементов для соответствующего процесса}	
	\scnitem{снятие блокировки с заданного sc-элемента и при необходимости установка первой по приоритету планируемой блокировки или удаление данного sc-элемента, если он входит во множество удаляемых sc-элементов для некоторого процесса}	
	\scnitem{поиск подпроцессов процесса и добавление их во множество отложенных действий в случае добавления самого процесса в данное множество}	
	\scnitem{поиск подпроцессов процесса и удаление их из множества отложенных действий в случае удаления самого процесса из данного множества}	
\end{scnrelfromset}
\end{SCn}

При реализации \textit{абстрактных программных sc-агентов} на \textit{языке SCP}, соблюдение всех принципов синхронизации соответствующих этим sc-агентам процессов обеспечивается на уровне \textit{sc-агентов интерпретации scp-программ}, т. е. средствами \textit{платформы интерпретации sc-моделей}. При реализации \textit{абстрактных программных sc-агентов} на уровне платформы, соблюдение всех принципов синхронизации возлагается, во-первых, непосредственно на разработчика агентов, во-вторых, -- на разработчика платформы. Так, например, платформа может предоставлять доступ к хранимым в sc-памяти элементам через некоторый программный интерфейс, уже учитывающий принципы работы с блокировками, что избавит разработчика агентов от необходимости учитывать все эти принципы вручную.

Кроме того, выделяется ряд специфичных принципов работы \textit{абстрактных программных sc-агентов}, реализованных на \textit{языке SCP}:
\begin{itemize}
\item в результате появления в sc-памяти некоторой конструкции, удовлетворяющей условию инициирования какого-либо \textit{абстрактного sc-агента}, реализованного при помощи \textit{Языка SCP}, в \textit{sc-памяти} генерируется и инициируется \textit{scp-процесс}. В качестве шаблона для генерации используется \textit{агентная scp-программа}, соответствующая данному \textit{абстрактному sc-агенту}.
\item каждый такой \textit{scp-процесс}, соответствующий некоторой \textit{агентной \mbox{scp-программе}}, может быть связан с набором структур, описывающих блокировки различных типов. Таким образом, синхронизация взаимодействия параллельно выполняемых \textit{scp-процесcов} осуществляется так же, как и в случае любых других \textit{действий в sc-памяти}.
\item несмотря на то что каждый \textit{scp-оператор} представляет собой атомарное действие в sc-памяти, являющееся поддействием в рамках всего \textit{\mbox{scp-процесса}}, блокировки, соответствующие одному оператору, не вводятся, чтобы избежать громоздкости и избытка дополнительных системных конструкций, создаваемых при выполнении некоторого \textit{scp-процесса}. Вместо этого используются блокировки, общие для всего \textit{scp-процесса}. Таким образом, \textit{агенты интерпретации scp-программ} работают только с учетом блокировок, общих для всего интерпретируемого \textit{scp-процесса}.
\item процессы, описывающие деятельность агентов интерпретации \textit{scp-программ}, как правило, не создаются, следовательно, и не вводятся соответствующие им блокировки. Поскольку такие агенты работают с уникальным scp-процессом и их число ограничено и известно, то использование блокировок для их синхронизации не требуется.
\item в случае приостановки \textit{scp-процесса} (добавления его во множество \textit{отложенных действий}) в соответствии с общими правилами синхронизации все его дочерние процессы также должны быть	приостановлены. В связи с этим все \textit{scp-операторы}, которые в	этот момент являются \textit{настоящими сущностями}, становятся	\textit{отложенными действиями}.
\item во избежание нежелательных изменений в самом теле \textit{scp-процесса}, вся конструкция, сгенерированная на основе некоторой \textit{scp-программы} (весь \textit{sc-текст}, описывающий декомпозицию \textit{scp-процесса} на \textit{scp-операторы}), должна быть добавлена в \textit{полную блокировку}, соответствующую данному \textit{scp-процессу}.
\item при необходимости разблокировать или заблокировать некоторую конструкцию каким-либо типом блокировки используются соответствующие \textit{scp-операторы} класса \textit{scp-оператор управления блокировками}.
\item после завершения выполнения некоторого scp-процесса его текст, как правило, удаляется из \textit{sc-памяти}, а все заблокированные конструкции освобождаются (разрушаются знаки структур, обозначавших блокировки).
\item как правило, частный \textit{класс действий}, соответствующий конкретной \textit{scp-программе}, явно не вводится, а используется более общий класс \textit{scp-процесс}, за исключением тех случаев, когда введение	специального \textit{класса действий} необходимо по каким-либо другим соображениям.
\end{itemize}

В общем случае весь механизм блокировок может описываться как на уровне SC-кода (для повышения уровня платформенной независимости), так и при необходимости может быть реализован на уровне \textit{платформы интерпретации sc-моделей}, например для повышения производительности. Для этого каждому выполняемому в sc-памяти процессу на нижнем уровне может быть поставлена в соответствие некая уникальная таблица, в каждый момент времени содержащая перечень заблокированных элементов с указанием типа блокировки.

Рассмотрим пример применения описанного механизма.

\begin{figure}[h]
	\centering
	\includegraphics[scale=0.8]{images/part3/chapter_situation_management/plan_lock_1.png}
	\caption{Пример использования планируемых блокировок}
	\label{fig:plan_lock_1}
\end{figure}

В данном примере \textit{Процесс1} непосредственно работает с sc-элементом \textit{\textbf{e1}},\textit{Процесс2} и \textit{Процесс3} планируют установить блокировку на любое изменение и блокировку на удаление соответственно, причем \textit{Процесс2} попытался установить свою блокировку раньше, чем \textit{Процесс3}, поэтому согласно направлению связки отношения \textit{приоритет блокировки*}, его блокировка будет установлена раньше. \textit{Процесс4} и \textit{Процесс5} ожидают снятия всех блокировок и планируемых блокировок, после чего \textit{\textbf{e1}} будет удален и \textit{Процесс1} и \textit{Процесс2} продолжат свое выполнение. Никакие другие планируемые блокировки установлены быть уже не могут, поскольку \textit{\textbf{e1}} попал во множество удаляемых sc-элементов как минимум одного процесса и, в соответствии с изложеннымивыше правилами, все остальные процессы кроме \textit{Процесс1}-\textit{Процесс5}, уже несмогут получить доступ к этому sc-элементу.		
Выполняемый процесс принадлежит множеству настоящая сущность, приостановленные -- множеству отложенное действие.

\begin{figure}[h]
	\centering
	\includegraphics[scale=0.8]{images/part3/chapter_situation_management/plan_lock_2.png}
	\caption{Пример использования планируемых блокировок (продолжение)}
	\label{fig:plan_lock_2}
\end{figure}

После того как \textit{Процесс1} разблокировал sc-элемент \textit{\textbf{e1}}, этот элемент будет заблокирован \textit{Процессом2}, и \textit{Процесс2} продолжит выполнение. \textit{Планируемая блокировка*}, установленная \textit{Процессом2}, становится обычной \textit{блокировкой*}.

\begin{figure}[h]
	\centering
	\includegraphics[scale=0.8]{images/part3/chapter_situation_management/plan_lock_3.png}
	\caption{Пример использования блокировки на удаление}
	\label{fig:plan_lock_3}
\end{figure}

После того как \textit{Процесс2} разблокировал sc-элемент \textit{\textbf{e1}}, этот элемент будет заблокирован \textit{Процессом3}, и \textit{Процесс3} продолжит выполнение.

\begin{figure}[h]
	\centering
	\includegraphics[scale=0.8]{images/part3/chapter_situation_management/plan_lock_4.png}
	\caption{Удаляемые sc-элементы}
	\label{fig:plan_lock_4}
\end{figure}

Когда все процессы снимут блокировки с sc-элемента \textit{\textbf{e1}}, он может быть физически удален и \textit{Процесс4} и \textit{Процесс5} продолжат выполнение.

В зависимости от конкретных \textit{типов блокировок} установленных паралельно выполняемыми процессами на некоторые sc-элементы и того, какие конкретно действия с этими \textit{sc-элементами} предполагается выполнить далее в рамках выполнения этих процессов, возможны ситуации взаимоблокировки, когда каждый из указанных процессов будет ожидать снятия блокировки вторым процессом с нужного \textit{sc-элемента}, не снимая при этом установленной им самим блокировки с \textit{sc-элемента}, доступ к которому необходим второму процессу.
	
В случае когда хотя бы одна из блокировок является \textit{полной блокировкой}, ситуация взаимоблокировки возникнуть не может, поскольку \textit{sc-элементы}, попавшие в \textit{полную блокировку} некоторого \textit{scp-процесса}, не доступны другим \textit{scp-процессам} даже для чтения и, таким образом, остальные \textit{scp-процессы} будут работать так, как будто заблокированные \textit{sc-элементы} просто отсутствуют в текущем состоянии \textit{sc-памяти}.

В случаях, когда ни одна из установленных блокировок не является \textit{полной блокировкой}, возможно появление взаимоблокировок.

Устранение \textit{взаимоблокировки} невозможно без вмешательства специализированного \textit{sc-метаагента}, который имеет право игнорировать блокировки, установленные другими процессами. 
	
В общем случае проблема конкретной взаимоблокировки может быть решена путем выполнения специализированным \textit{sc-метаагентом} следующих шагов:	
\begin{itemize}
	\item откат нескольких операций, выполненных одним из участвующих в взаимоблокировке процессов настолько шагов назад, насколько это необходимо для того, чтобы второй процесс получил доступ к необходимым \textit{sc-элементам} и смог продолжить выполнение;
	\item ожидание выполнения второго процесса вплоть до завершения или до снятия им всех блокировок с \textit{sc-элементов}, доступ к которым необходимо получить первому процессу;
	\item повторное выполнение в рамках первого процесса отмененных операций и продолжение его выполнения, но уже с учетом изменений в памяти, внесенных вторым процессом.		
\end{itemize}

Для \textit{sc-метаагентов} все sc-элементы, в том числе описывающие блокировки, планируемые блокировки и т. д. полностью эквивалентны между собой с точки зрения доступа к ним, т. е. любой \textit{sc-метаагент} имеет доступ к любым sc-элементам, даже попавшим в полную блокировку для какого-либо другого процесса. Это необходимо для того, чтобы \textit{sc-метаагенты} смогли выявлять и устранять различные проблемы, например, описанную выше проблему взаимоблокировки.

Таким образом, проблема синхронизации деятельности \textit{sc-метаагентов} требует введения дополнительных правил.

Указанную проблему разделим на две более частные:
\begin{itemize}
	\item обеспечение синхронизации деятельности \textit{sc-метаагентов} между собой;
	\item обеспечение синхронизации деятельности \textit{sc-метаагентов} и \textit{программных sc-агентов}.		
\end{itemize}

Первую проблему предлагается решить за счет запрета параллельного выполнения \textit{sc-метаагентов}. Таким образом, в каждый момент времени в рамках одной \textit{ostis-системы} может существовать только один процесс, соответствующий \textit{sc-метаагенту} и являющийся \textit{настоящей сущностью}. 

Вторую проблему предлагается решить за счет введения дополнительных привилегий для \textit{sc-метаагентов} при обращении к какому-либо sc-элементу. Для этого достаточно одного правила: 

Если некоторый sc-элемент стал использоваться в рамках процесса, соответствующего \textit{sc-метаагенту} (например, стал элементом хотя бы одного scp-оператора, входящего в данный процесс), то все процессы, в блокировки соответствующие которым попадает указанный sc-элемент, становятся отложенными действиями (приостанавливают выполнение). Как только указанный sc-элемент перестает использоваться в рамках процесса, соответствующего \textit{sc-метаагенту}, все приостановленные по этой причине процессы продолжают выполнение.

Рассмотренные ограничения не ухудшают производительность ostis-системы существенно, поскольку \textit{sc-метаагенты} предназначены для решения достаточно узкого класса задач, которые, как показал опыт практической разработки прототипов различных \textit{ostis-систем}, возникают достаточно редко.
	
Стоит отметить, что возможна ситуация, при которой выполнение некоторого процесса в sc-памяти прервано по причине возникновения какой-либо ошибки. В таком случае существует вероятность того, что блокировка, установленная данным процессом не будет снята до тех пор, пока этого не сделает sc-метаагент, обнаруживший подобную ситуацию. Однако указанная проблема на уровне sc-модели может быть решена лишь частично, для случаев, когда ошибка возникает при интерпретации scp-программы, отслеживается scp-интепретатором и в памяти формируется соответствующая конструкция, сообщающая о проблеме sc-метаагенту. Случаи, когда возникла ошибка на уровне scp-интерпретатора или sc-хранилища, должны рассматриваться на уровне платформы интерпретации sc-моделей.

\section{Базовый язык программирования ostis-систем}

В качестве базового языка для описания программ обработки текстов \textit{SC-кода} предлагается \textit{Язык SCP}.

\textit{Язык SCP} -- это графовый язык процедурного программирования, предназначенный для эффективной обработки \textit{sc-текстов}. \textit{Язык SCP} является языком параллельного асинхронного программирования.

\begin{SCn}
	\scnheader{Язык SCP}
	\scnidtftext{часто используемый sc-идентификатор}{scp-программа}
\end{SCn}

Языком представления данных для текстов \textit{Языка SCP}
(\textit{scp-программ}) является \textit{SC-код} и, соответственно, любые варианты его внешнего представления. \textit{Язык SCP} сам построен на основе \textit{SC-кода}, вследствие чего \textit{scp-программы} сами по себе
могут входить в состав обрабатываемых данных для \textit{scp-программ}, в т.ч. по отношению к самим себе. Таким образом, \textit{язык SCP} предоставляет возможность
построения реконфигурируемых программ. Однако для обеспечения
возможности реконфигурирования программы непосредственно в процессе ее интерпретации необходимо на уровне интерпретатора \textit{Языка SCP (Aбстрактной scp-машины)} обеспечить уникальность каждой исполняемой копии исходной программы. Такую исполняемую копию, сгенерированную на основе \textit{scp-программы}, будем называть \textit{scp-процессом}.
Включение знака некоторого \textit{действия в sc-памяти} во множество \textit{scp-процессов} гарантирует тот факт, что в декомпозиции данного действия будут присутствовать только знаки элементарных действий (\textit{scp-операторов}), которые может интерпретировать реализация \textit{Aбстрактной scp-машины} (интерпретатора scp-программ).

\textit{Язык SCP} рассматривается как ассемблер для семантического компьютера.

\begin{SCn}
\scnheader{Абстрактная scp-машина}
\scnrelfrom{модель}{Модель Абстрактной scp-машины}
\end{SCn}

\textit{Базовая модель обработки sc-текстов} включает в себя \textit{Предметную область Базового языка программирования ostis-систем}, то есть описание синтаксиса и денотационной семантики Языка SCP, а также \textit{Модель Абстрактной scp-машины}. \textit{Абстрактная scp-машина} представляет собой интерпретатор \textit{scp-программ}, который должен являться частью \textit{платформы интерпретации sc-моделей компьютерных систем} (хотя в общем случае могут существовать варианты платформы, не содержащие такого интерпретатора, что, однако, не позволит использовать достоинства предлагаемой базовой модели.

Рассмотрим ключевые особенности и достоинства \textit{Базовой модели обработки sc-текстов}:
\begin{itemize}
\item Тексты программ \textit{Языка SCP} записываются при помощи тех же унифицированных семантических сетей, что и обрабатываемая информация, таким образом, можно сказать, что \textit{Синтаксис языка SCP} на базовом уровне совпадает с \textit{Синтаксисом SC-кода}.\item Подход к интерпретации \textit{scp-программ} предполагает создание при	каждом вызове \textit{scp-программы} уникального \textit{scp-процесса}.
\item Одновременно в общей памяти могут выполняться несколько независимых \textit{sc-агентов}, при этом разные копии \textit{sc-агентов} могут выполняться на разных серверах, за счет распределенной реализации интерпретатора sc-моделей (\textit{платформы реализации sc-моделей компьютерных систем}). Более того, \textit{Язык SCP} позволяет осуществлять параллельные асинхронные вызовы подпрограмм с последующей синхронизацией, и даже параллельно	выполнять операторы в рамках одной \textit{scp-программы}.
\item Перенос \textit{sc-агента} из одной системы в другую заключается в простом переносе фрагмента базы знаний, без каких-либо дополнительных операций, зависящих от платформы интерпретации.
\item Тот факт, что спецификации \textit{sc-агентов} и их программы могут быть записаны на том же языке, что и обрабатываемые знания, существенно сокращает перечень специализированных средств, предназначенных для проектирования машин обработки знаний, и упрощает их разработку за
счет использования более универсальных компонентов.
\item Тот факт, что для интерпретации \textit{scp-программы} создается соответствующий ей уникальный \textit{\mbox{scp-процесс}}, позволяет по возможности оптимизировать план выполнения перед его реализацией и
даже непосредственно в процессе выполнения без потенциальной опасности испортить общий универсальный алгоритм всей программы. Более того, такой подход к проектированию и интерпретации программ позволяет говорить о возможности создания самореконфигурируемых программ.
\end{itemize}

\begin{SCn}
\scnheader{scp-программа}
\scnsubset{программа в sc-памяти}
\scnsuperset{агентная scp-программа}
\end{SCn}

Каждая \textbf{\textit{scp-программа}} представляет собой \textit{обобщенную структуру}, описывающую один из вариантов декомпозиции действий некоторого класса, выполняемых в sc-памяти. Знак \textit{sc-переменной}, соответствующей конкретному декомпозируемому действию является в рамках \textbf{\textit{scp-программы}} \textit{ключевым sc-элементом\scnrolesign}. Также явно указывается принадлежность данного знака множеству \textit{scp-процессов}.
	
Принадлежность некоторого действия множеству \textit{scp-процессов} гарантирует тот факт, что в декомпозиции данного действия будут присутствовать только знаки элементарных действий (\textit{scp-операторов}), которые может интерпретировать реализация абстрактной scp-машины.

Таким образом, каждая \textbf{\textit{scp-программа}} описывает в обобщенном виде декомпозицию некоторого \textit{\mbox{scp-процесса}} на взаимосвязанные \textit{scp-операторы}, с указанием, при их наличии, аргументов для данного \textit{scp-процесса}.

По сути каждая \textbf{\textit{scp-программа}} представляет собой описание последовательности элементарных операций, которые необходимо выполнить над семантической сетью, чтобы выполнить более сложное действие некоторого класса.

На рисунке \ref{fig:program_example} приведен пример простой scp-программы. В приведенном примере показана \textit{scp-программа}, состоящая из трех \textit{scp-операторов}. Данная программа проверяет, содержится ли в заданном множестве (первый параметр) заданный элемент (второй параметр), и, если нет, то добавляет его в это множество.
	
\begin{figure}[H]
	\centering
	\includegraphics[scale=0.8]{images/part3/chapter_situation_management/program_example.png}
	\caption{Пример scp-программы}
	\label{fig:program_example}
\end{figure}

\textbf{\textit{Агентные scp-программы}} представляют собой частный случай \textit{scp-программ} вообще, однако заслуживают отдельного рассмотрения, поскольку используются наиболее часто. \textit{Scp-программы} данного класса представляют собой реализации программ агентов обработки знаний, и имеют жестко фиксированный набор параметров. Каждая такая программа имеет ровно два \textit{in-параметра\scnrolesign}. Значение первого параметра является знаком бинарной ориентированной пары, являющейся вторым компонентом связки отношения \textit{первичное условие инициирования*} для абстрактного \textit{sc-агента}, в множество \textit{программ sc-агента*} которого входит рассматриваемая \textbf{\textit{агентная scp-программа}}, и, по сути, описывает класс событий, на которые реагирует указанный sc-агент.
	
Значением второго параметра является \textit{sc-элемент}, с которым непосредственно связано событие, в результате возникновения которого был инициирован соответствующий \textit{sc-агент}, т.е., например, сгенерированная либо удаляемая \textit{sc-дуга} или \textit{sc-ребро}.

Рассмотрим принципы реализации \textit{абстрактных sc-агентов, реализуемых на Языке SCP}:
\begin{itemize}
\item общие принципы организации взаимодействия \textit{sc-агентов} и пользователей \textit{ostis-системы} через общую \textit{sc-память};
\item в результате появления в sc-памяти некоторой конструкции,
удовлетворяющей условию инициирования какого-либо \textit{абстрактного sc-агента}, реализованного при помощи \textit{Языка SCP}, в \textit{sc-памяти} генерируется и инициируется \textit{scp-процесс}. В качестве шаблона для генерации используется \textit{агентная scp-программа}, указанная во множестве программ соответствующего \textit{абстрактного sc-агента};
\item каждый такой \textit{scp-процесс}, соответствующий некоторой \textit{агентной scp-программе}, может быть связан с набором структур, описывающих блокировки различных типов. Таким образом, синхронизация взаимодействия параллельно выполняемых \textit{scp-процесcов} осуществляется так же, как и в случае любых других \textit{действий в sc-памяти};
\item В рамках \textit{scp-процесса} могут создаваться дочерние
\textit{scp-процессы}, однако синхронизация между ними при необходимости осуществляется посредством введения дополнительных внутренних блокировок. Таким образом, каждый \textit{scp-процесс} с точки зрения \textit{процессов в sc-памяти} является атомарным и законченным актом деятельности некоторого \textit{sc-агента};
\item во избежание нежелательных изменений в самом теле \textit{scp-процесса}, вся конструкция, сгенерированная на основе некоторой \textit{scp-программы} (весь текст \textit{scp-процесса}), должна быть добавлена в \textit{полную блокировку}, соответствующую данному \textit{scp-процессу};
\item все конструкции, сгенерированные в процессе выполнения
\textit{scp-процесса}, автоматически попадают в \textit{полную 	блокировку}, соответствующую данному \textit{scp-процессу}. Дополнительно следует отметить, что знак самой этой структуры и вся метаинформация о ней также включаются в эту структуру;
\item при необходимости можно вручную разблокировать или заблокировать некоторую конструкцию каким-либо типом блокировки, используя соответствующие \textit{scp-операторы} класса \textit{scp-оператор управления блокировками};
\item после завершения выполнения некоторого \textit{scp-процесса} его текст как правило, удаляется из \textit{\mbox{sc-памяти}}, а все заблокированные конструкции освобождаются (разрушаются знаки структур, обозначавших блокировки);
\item несмотря на то, что каждый \textit{scp-оператор} представляет собой атомарное \textit{действие в sc-памяти}, дополнительные блокировки, соответствующие одному оператору не вводятся, чтобы избежать громоздкости и избытка дополнительных системных конструкций, создаваемых при выполнении некоторого \textit{scp-процесса}. Вместо этого используются блокировки, общие для всего \textit{scp-процесса}. Таким образом, агенты \textit{Абстрактной scp-машины} при интерпретации \textit{scp-операторов} работают только с учетом блокировок, общих для всего интерпретируемого \textit{scp-процесса};
\item как правило, частный \textit{класс действий}, соответствующий конкретной \textit{scp-программе} явно не вводится, а используется более общий класс \textit{scp-процесс}, за исключением тех случаев, когда введение специального \textit{класса действий} необходимо по каким-либо другим соображениям.
\end{itemize}

Под \textbf{\textit{scp-процессом}} понимается некоторое \textit{действие в sc-памяти}, однозначно описывающее конкретный акт выполнения некоторой \textit{scp-программы} для заданных исходных данных. Если \textit{scp-программа} описывает алгоритм решения какой-либо задачи в общем виде, то \textit{scp-процесс} обозначает конкретное действие, реализующее данный алгоритм для заданных входных параметров.

По сути, \textbf{\textit{scp-процесс}} представляет собой уникальную копию, созданную на основе \textit{scp-программы}, в которой каждой \textit{sc-переменной}, за исключением \textit{scp-переменных\scnrolesign}, соответствует сгенерированная \textit{sc-константа}.

Принадлежность некоторого действия множеству \textit{scp-процессов} гарантирует тот факт, что в декомпозиции данного действия будут присутствовать только знаки элементарных действий (\textit{scp-операторов}), которые может интерпретировать реализация \textit{Абстрактной scp-машины}.

Рассмотрим пример поэтапного выполнения scp-процесса (рис. \ref{fig:process_example} -- \ref{fig:process_example4}), соответствующего ранее рассмотренному примеру scp-программы. В приведенном примере последовательно показаны состояния \textit{scp-процесса}, соответствующего \textit{\mbox{scp-программе}}, добавляющей заданный элемент в заданное множество, если он там ранее не содержался. В примере предполагается, что рассматриваемый элемент (\textit{Element1}) изначально не содержится во множестве (\textit{Set1}).

\begin{figure}[H]
	\centering
	\includegraphics[scale=0.8]{images/part3/chapter_situation_management/process_example.png}
	\caption{Пример scp-процесса на начальной стадии выполнения}
	\label{fig:process_example}
\end{figure}

Осуществляется вызов \textit{scp-программы}. Генерируется соответствующий \textit{scp-процесс}. Происходит инициирование начального оператора scp-процесса \textit{Operator1}.

\begin{figure}[H]
	\centering
	\includegraphics[scale=0.8]{images/part3/chapter_situation_management/process_example2.png}
	\caption{Пример scp-процесса: безуспешно выполнен оператор поиска}
	\label{fig:process_example2}
\end{figure}

Оператор \textit{Operator1} оказался безуспешно выполненным. Производится инициирование \textit{\mbox{scp-оператора} генерации трёхэлементной конструкции} ~~~ \textit{Operator2}.

\begin{figure}[H]
	\centering
	\includegraphics[scale=0.8]{images/part3/chapter_situation_management/process_example3.png}
	\caption{Пример scp-процесса: выполнен оператор генерации, элемент добавлен во множество}
	\label{fig:process_example3}
\end{figure}

Оператор \textit{Operator2} выполнился. Производится инициирование \textit{scp-оператора завершения выполнения программы} ~~~ \textit{Operator3}.

\begin{figure}[H]
	\centering
	\includegraphics[scale=0.8]{images/part3/chapter_situation_management/process_example4.png}
	\caption{Пример scp-процесса: выполнение завершено}
	\label{fig:process_example4}
\end{figure}

Оператор \textit{Operator3} выполнился. Выполнение \textit{scp-процесса} завершается.

\begin{SCn}
\scnheader{scp-оператор}
\scnsubset{действие в sc-памяти}
\scnrelto{семейство подмножеств}{атомарный тип scp-оператора}
\end{SCn}

Каждый \textbf{\textit{scp-оператор}} представляет собой некоторое элементарное \textit{действие в sc-памяти}. Аргументы \textit{scp-оператора} будем называть операндами. Порядок операндов указывается при помощи соответствующих ролевых отношений (\textit{1\scnrolesign}, \textit{2\scnrolesign}, \textit{3\scnrolesign} и так далее). Операнд, помеченный ролевым отношением \textit{1\scnrolesign}, будем называть первым операндом, помеченный ролевым отношением \textit{2\scnrolesign} – вторым операндом, и т.д. Тип и смысл каждого операнда также уточняется при помощи различных подклассов отношения \textit{scp-операнд\scnrolesign}. В общем случае операндом может быть любой \textit{sc-элемент}, в том числе, знак какой-либо \textit{scp-программы}, в том числе самой программы, содержащей данный оператор.

Каждый \textbf{\textit{scp-оператор}} должен иметь один и более операнд, а также указание того \textbf{\textit{scp-оператора}} (или нескольких), который должен быть выполнен следующим. Исключение их данного правила составляет \textit{scp-оператор завершения выполнения программы}, который не содержит ни одного операнда и после выполнения которого никакие \textit{scp-операторы} в рамках данной программы выполняться не могут.

Каждый \textbf{\textit{атомарный тип scp-оператора}} представляет собой класс \textit{scp-операторов}, который не разбивается на более частные, и, соответственно, интерпретируется реализацией \textit{Aбстрактной scp-машины}.

\begin{SCn}
\scnheader{начальный оператор\scnrolesign}
\scnsubset{1\scnrolesign}
\end{SCn}

Ролевое отношение \textbf{\textit{начальный оператор\scnrolesign}} указывает в рамках декомпозиции соответствующего \textit{\mbox{scp-программе}} \textit{scp-процесса} те \textit{scp-операторы}, которые должны быть выполнены в первую очередь, т.е. те, с которых собственно начинается выполнение \textit{scp-процесса}.

\begin{SCn}
\scnheader{параметр scp-программы\scnrolesign}
\scnsubset{аргумент действия\scnrolesign}
\begin{scnrelfromset}{разбиение}
	\scnitem{in-параметр\scnrolesign}
	\scnitem{out-параметр\scnrolesign}
\end{scnrelfromset}
\end{SCn}

Ролевое отношение \textbf{\textit{параметр scp-программы\scnrolesign}} связывает знак соответствующего \textit{scp-программе} \textit{\mbox{scp-процесса}} с его аргументами.

Параметры типа \textbf{\textit{in-параметр\scnrolesign}} хоть и соответствуют \textit{переменным scp-программы\scnrolesign}, не могут менять значение в процессе ее интерпретации. Фиксированное значение переменной устанавливается при создании уникальной копии \textit{scp-программы} (\textit{scp-процесса}) для ее интерпретации, и, таким образом, соответствующая \textit{scp-переменная\scnrolesign} на момент начала ее интерпретации становится \textit{scp-константой\scnrolesign} в рамках каждого \textit{scp-оператора}, в котором встречалась данная \textit{scp-переменная\scnrolesign}. Использование \textit{in-параметров} можно рассматривать по аналогии с использованием варианта механизма передачи по значению в традиционных языках программирования, с тем условием, что значение локальной переменной в рамках дочерней программы не может быть изменено.

Параметры типа \textbf{\textit{out-параметр\scnrolesign}} соответствуют \textit{переменным scp-программы\scnrolesign} и обладают всеми теми же соответствующими свойствами. Чаще всего предполагается, что значение данного параметра необходимо родительской \textit{scp-программе}, содержащей оператор вызова текущей \textit{scp-программы}. При этом на момент начала интерпретации в качестве параметра дочернему процессу передается непосредственно узел, обозначающий переменную (а точнее, ее уникальную копию в рамках процесса) родительского процесса. Указанная переменная может при необходимости иметь значение, либо не иметь. После завершения и во время интерпретации дочернего процесса родительский процесс по-прежнему может работать с переменной, переданной в качестве \textit{out-параметра\scnrolesign}, при необходимости просматривая или изменяя ее значение. Использование out-параметра можно рассматривать по аналогии с использованием механизма передачи по ссылке в традиционных языках программирования.

Рассмотрим классификацию структур (фрагментов sc-памяти) с точки зрения Базовой модели обработки sc-текстов.

\begin{SCn}
\scnheader{структура}
\scnrelfrom{разбиение}{Классификация структур с точки зрения Базовой модели обработки sc-текстов}
\begin{scnindent}
	\begin{scneqtoset}
		\scnitem{sc-конструкция нестандартного вида}
		\scnitem{sc-конструкция стандартного вида}
	\end{scneqtoset}
\begin{scnindent}
\begin{scnrelfromset}{разбиение}
	\scnitem{одноэлементная sc-конструкция}
	\scnitem{трехэлементная sc-конструкция}
	\scnitem{пятиэлементная sc-конструкция}	
\end{scnrelfromset}
\end{scnindent}
\end{scnindent}
\end{SCn}

Каждая \textit{sc-конструкция нестандартного вида} состоит из произвольного количества \textit{sc-элементов} произвольного типа (Рис. \ref{fig:pic_ps4}).

\begin{figure}[H]
	\centering
	\includegraphics[scale=0.8]{images/part3/chapter_situation_management/pic_ps1.png}
	\caption{Пример sc-конструкции нестандартного вида}
	\label{fig:pic_ps1}
\end{figure}

В свою очередь, каждый элемент \textit{\mbox{sc-конструкции} стандартного вида} имеет свою условную строго фиксированную позицию в рамках этой \mbox{sc-конструкции} (первый элемент, второй элемент и т. д.). В зависимости от указанной позиции вводятся дополнительные ограничения на тип соответствующего \textit{sc-элемента}.

Каждая \textit{одноэлементная sc-конструкция} состоит из одного \textit{sc-элемента} произвольного типа (Рис. \ref{fig:pic_ps2}).

\begin{figure}[H]
	\centering
	\includegraphics[scale=0.8]{images/part3/chapter_situation_management/pic_ps2.png}
	\caption{Пример одноэлементных sc-конструкций в SCg-коде}
	\label{fig:pic_ps2}
\end{figure}

Каждая \textit{трехэлементная sc-конструкция} состоит из трех \textit{sc-элементов} (Рис. \ref{fig:pic_ps3}). Второй элемент всегда является \textit{sc-коннектором}, остальные элементы могут быть произвольного типа.

\begin{figure}[H]
	\centering
	\includegraphics[scale=0.8]{images/part3/chapter_situation_management/pic_ps3.png}
	\caption{Пример трехэлементной sc-конструкции в SCg-коде}
	\label{fig:pic_ps3}
\end{figure}

Каждая \textit{пятиэлементная sc-конструкция} состоит из пяти \textit{sc-элементов} (Рис. \ref{fig:pic_ps4}). Второй и четвертый элементы обязательно являются \textit{sc-коннекторами}, остальные элементы могут быть произвольного типа.

\begin{figure}[H]
	\centering
	\includegraphics[scale=0.8]{images/part3/chapter_situation_management/pic_ps4.png}
	\caption{Пример пятиэлементной sc-конструкции в SCg-коде}
	\label{fig:pic_ps4}
\end{figure}

\subsection{Синтаксис Базового языка программирования ostis-систем}

\subsection{Денотационная семантика Базового языка программирования ostis-систем}

\begin{SCn}
\scnheader{scp-оператор}
\scnsubset{действие в sc-памяти}
\scnrelto{семейство подмножеств}{атомарный тип scp-оператора}
\begin{scnrelfromset}{разбиение}
	\scnitem{scp-оператор генерации конструкций}
	\begin{scnindent}
		\begin{scnrelfromset}{разбиение}
			\scnitem{scp-оператор генерации конструкции по произвольному образцу}
			\scnitem{scp-оператор генерации пятиэлементной конструкции}
			\scnitem{scp-оператор генерации трехэлементной конструкции}
			\scnitem{scp-оператор генерации одноэлементной конструкции}
		\end{scnrelfromset}
	\end{scnindent}
	\scnitem{scp-оператор ассоциативного поиска конструкций}
	\begin{scnindent}
		\begin{scnrelfromset}{разбиение}
			\scnitem{scp-оператор поиска конструкции по произвольному образцу}
			\scnitem{scp-оператор поиска пятиэлементной конструкции с формированием множеств}
			\scnitem{scp-оператор поиска трехэлементной конструкции с формированием множеств}
			\scnitem{scp-оператор поиска пятиэлементной конструкции}
			\scnitem{scp-оператор поиска трехэлементной конструкции}
		\end{scnrelfromset}
	\end{scnindent}
	\scnitem{scp-оператор удаления конструкций}
	\begin{scnindent}
		\begin{scnrelfromset}{разбиение}
			\scnitem{scp-оператор удаления множества элементов трехэлементной конструкции}
			\scnitem{scp-оператор удаления одноэлементной конструкции}
			\scnitem{scp-оператор удаления пятиэлементной конструкции}
			\scnitem{scp-оператор удаления трехэлементной конструкции}
		\end{scnrelfromset}
	\end{scnindent}
	\scnitem{scp-оператор проверки условий}
	\begin{scnindent}
		\begin{scnrelfromset}{разбиение}
			\scnitem{scp-оператор сравнения числовых содержимых файлов}
			\scnitem{scp-оператор проверки равенства числовых содержимых файлов}
			\scnitem{scp-оператор проверки совпадения значений операндов}
			\scnitem{scp-оператор проверки наличия содержимого у файла}
			\scnitem{scp-оператор проверки наличия значения у переменной}
			\scnitem{scp-оператор проверки типа sc-элемента}
		\end{scnrelfromset}
	\end{scnindent}
	\scnitem{scp-оператор управления значениями операндов}
	\begin{scnindent}
		\begin{scnrelfromset}{разбиение}
			\scnitem{scp-оператор удаления значения переменной}
			\scnitem{scp-оператор присваивания значения переменной}
		\end{scnrelfromset}
	\end{scnindent}
	\scnitem{scp-оператор управления scp-процессами}
	\begin{scnindent}
		\begin{scnrelfromset}{разбиение}
			\scnitem{scp-оператор удаления значения переменной}
			\scnitem{scp-оператор завершения выполнения программы}
			\scnitem{конъюнкция предшествующих scp-операторов}
			\scnitem{scp-оператор ожидания завершения выполнения множества scp-программ}
			\scnitem{scp-оператор ожидания завершения выполнения scp-программы}
			\scnitem{scp-оператор асинхронного вызова подпрограммы}
		\end{scnrelfromset}
	\end{scnindent}
	\scnitem{scp-оператор управления событиями}
	\begin{scnindent}
		\scnsuperset{scp-оператор ожидания события}
	\end{scnindent}
	\scnitem{scp-оператор обработки содержимых файлов}
	\begin{scnindent}
		\begin{scnrelfromset}{разбиение}
			\scnitem{scp-оператор вычисления арксинуса числового содержимого файла}
			\scnitem{scp-оператор вычисления арккосинуса числового содержимого файла}
			\scnitem{scp-оператор деления числовых содержимых файлов}
			\scnitem{scp-оператор умножения числовых содержимых файлов}
			\scnitem{scp-оператор вычитания числовых содержимых файлов}
			\scnitem{scp-оператор сложения числовых содержимых файлов}
			\scnitem{scp-оператор вычисления тангенса числового содержимого файла}
			\scnitem{scp-оператор вычисления косинуса числового содержимого файла}
			\scnitem{scp-оператор вычисления синуса числового содержимого файла}
			\scnitem{scp-оператор вычисления логарифма числового содержимого файла}
			\scnitem{scp-оператор возведения числового содержимого файла в степень}
			\scnitem{scp-оператор удаления содержимого файла}
			\scnitem{scp-оператор копирования содержимого файла}
			\scnitem{scp-оператор нахождения остатка от деления числовых содержимых файлов}
			\scnitem{scp-оператор нахождения целой части от деления числовых содержимых файлов}
			\scnitem{scp-оператор вычисления арктангенса числового содержимого файла}
			\scnitem{scp-оператор перевода в верхний регистр строкового содержимого файла}
			\scnitem{scp-оператор перевода в верхний регистр строкового содержимого файла}
			\scnitem{scp-оператор замены определенной части строкового содержимого файла на содержимое указанной файла}
			\scnitem{scp-оператор проверки совпадения конца строкового содержимого файла со строковом содержимым другого файла}
			\scnitem{scp-оператор проверки совпадения начальной части строкового содержимого файла со строковом содержимым другого файла}
			\scnitem{scp-оператор получения части строкового содержимого файла по индексам}
			\scnitem{scp-оператор поиска строкового содержимого файла в строковом содержимом другого файла}
			\scnitem{scp-оператор вычисления длины строкового содержимого файла}
			\scnitem{scp-оператор разбиения строки на подстроки}
			\scnitem{scp-оператор лексикографического сравнения строковых содержимых файлов}
			\scnitem{scp-оператор проверки равенства строковых содержимых файлов}
		\end{scnrelfromset}
	\end{scnindent}
	\scnitem{scp-оператор управления блокировками}
	\begin{scnindent}
		\begin{scnrelfromset}{разбиение}
			\scnitem{scp-оператор снятия всех блокировок данного scp-процесса}
			\scnitem{scp-оператор снятия блокировки с sc-элемента}
			\scnitem{scp-оператор установки полной блокировки на sc-элемент}
			\scnitem{scp-оператор установки блокировки на изменение sc-элемента}
			\scnitem{scp-оператор установки блокировки на удаление sc-элемента}
			\scnitem{scp-оператор снятия блокировки со структуры}
			\scnitem{scp-оператор установки полной блокировки на структуру}
			\scnitem{scp-оператор установки блокировки на изменение структуры}
			\scnitem{scp-оператор установки блокировки на удаление структуры}
		\end{scnrelfromset}
	\end{scnindent}
\end{scnrelfromset}

\scnheader{scp-операнд\scnrolesign}
\scnsubset{аргумент действия\scnrolesign}
\scniselement{неосновное понятие}
\scniselement{ролевое отношение}
\begin{scnrelfromset}{разбиение}
	\scnitem{scp-константа\scnrolesign}
	\scnitem{scp-переменная\scnrolesign}
\end{scnrelfromset}	
\begin{scnrelfromset}{разбиение}
	\scnitem{scp-операнд с заданным значением\scnrolesign}
	\scnitem{scp-операнд со свободным значением\scnrolesign}
\end{scnrelfromset}	
\begin{scnrelfromset}{разбиение}
	\scnitem{константный sc-элемент\scnrolesign}
	\scnitem{переменный sc-элемент\scnrolesign}
\end{scnrelfromset}	
\begin{scnrelfromlist}{включение}
	\scnitem{формируемое множество\scnrolesign}
	\begin{scnindent}
		\begin{scnrelfromset}{разбиение}
			\scnitem{формируемое множество 1\scnrolesign}
			\scnitem{формируемое множество 2\scnrolesign}
			\scnitem{формируемое множество 3\scnrolesign}
			\scnitem{формируемое множество 4\scnrolesign}
			\scnitem{формируемое множество 5\scnrolesign}
		\end{scnrelfromset}	
	\end{scnindent}
	\scnitem{удаляемый sc-элемент\scnrolesign}
	\scnitem{тип sc-элемента\scnrolesign}
	\begin{scnindent}
	\begin{scnrelfromset}{разбиение}
		\scnitem{sc-узел\scnrolesign}
		\begin{scnindent}
			\begin{scnrelfromset}{разбиение}
				\scnitem{структура\scnrolesign}
				\scnitem{отношение\scnrolesign}
				\begin{scnindent}
					\scnsuperset{ролевое отношение\scnrolesign}
				\end{scnindent}
			\scnitem{класс\scnrolesign}
			\end{scnrelfromset}	
		\end{scnindent}
		\scnitem{sc-дуга\scnrolesign}
		\begin{scnindent}
			\begin{scnrelfromset}{разбиение}
				\scnitem{sc-дуга общего вида\scnrolesign}
				\scnitem{sc-дуга принадлежности\scnrolesign}
				\begin{scnindent}
					\scnsuperset{sc-дуга основного вида\scnrolesign}
					\begin{scnindent}
					\scneq{(константный sc-элемент\scnrolesign $\cap$ позитивная sc-дуга принадлежности\scnrolesign $\cap$ постоянная sc-дуга принадлежности\scnrolesign)}
					\end{scnindent}					
					\begin{scnrelfromset}{разбиение}
						\scnitem{позитивная sc-дуга принадлежности\scnrolesign}
						\scnitem{негативная sc-дуга принадлежности\scnrolesign}
						\scnitem{нечеткая sc-дуга принадлежности\scnrolesign}
					\end{scnrelfromset}
					\begin{scnrelfromset}{разбиение}
						\scnitem{временная sc-дуга принадлежности\scnrolesign}
						\scnitem{постоянная sc-дуга принадлежности\scnrolesign}
					\end{scnrelfromset}
				\end{scnindent}
			\end{scnrelfromset}
		\end{scnindent}
		\scnitem{sc-ребро\scnrolesign}
		\scnitem{файл\scnrolesign}
	\end{scnrelfromset}
	\end{scnindent}
\end{scnrelfromlist}
\end{SCn}

Ролевое отношение \textit{scp-операнд\scnrolesign} является неосновным понятием и указывает на принадлежность аргументов \textit{scp-оператору}. Помимо указания какого-либо класса \textit{scp-операндов\scnrolesign} порядок аргументов \textit{scp-оператора} дополнительно уточняется \textit{ролевыми отношениями 1\scnrolesign}, \textit{2\scnrolesign} и т. д.

В рамках \textit{scp-программы} \textbf{\textit{scp-константы\scnrolesign}} явно участвуют в \textit{\mbox{scp-операторах}} в качестве элементов (в теоретико-множественном смысле) и напрямую обрабатываются при интерпретации \textit{scp-программы}. Константами в рамках \textit{scp-программы} могут быть \textit{sc-элементы} любого типа, как \textit{\mbox{sc-константы}}, так и \textit{\mbox{sc-переменные}}. Константа в рамках \textit{scp-программы} остается неизменной в течение всего срока интерпретации. Константа \textit{\mbox{scp-программы}} может быть рассмотрена как переменная, значение которой совпадает с самой переменной в каждый момент времени, и изменено быть не может. Таким образом, далее будем считать, что \textit{scp-константа\scnrolesign} и ее значение это одно и то же. Каждый \textit{in-параметр\scnrolesign} при интерпретации каждой конкретной копии \textit{scp-программы} становится \textit{scp-константой\scnrolesign} в рамках всех ее операторов, хотя в исходном теле данной программы в каждом из этих операторов он является \textit{scp-переменной\scnrolesign}.

В рамках \textit{scp-программы} \textbf{\textit{scp-переменные\scnrolesign}} не обрабатываются явно при интерпретации, обрабатываются значения переменных. Каждая переменная \textit{scp-программы} может иметь одно значение в каждый момент времени, т. е. представляет собой ситуативный \textit{синглетон}, элементом которого является текущее значение \textit{scp-переменной\scnrolesign}. Значение каждой \textit{scp-переменной\scnrolesign} может меняться в ходе интерпретации \textit{scp-программы}. При этом интерпретатор при обработке \textit{scp-оператора} работает непосредственно со значениями \textit{\mbox{scp-переменных\scnrolesign}}, а не самими \textit{scp-переменными\scnrolesign} (которые также являются узлами той же семантической сети).

Значение операндов, помеченных ролевым отношением \textbf{\textit{scp-операнд с заданным значением\scnrolesign}}, считается заданным в рамках текущего \textit{scp-оператора}. Данное значение учитывается при выполнении \textit{scp-оператора} и остается неизменным после окончания выполнения \textit{scp-оператора}. Каждая \textit{scp-константа\scnrolesign} по умолчанию рассматривается как \textit{scp-операнд с заданным значением\scnrolesign}, в связи с чем явное использование данного ролевого отношения в таком случае является избыточным. В таком случае в качестве значения рассматривается непосредственно сам операнд. В случае если отношением \textit{\mbox{scp-операнд} с заданным значением\scnrolesign} помечена \textit{scp-переменная\scnrolesign}, то осуществляется попытка поиска значения для данной \textit{scp-переменной\scnrolesign} (ее элемента). Если попытка оказалась безуспешной, то возникает ошибка времени выполнения, которая должна быть обработана соответствующим образом.
	
Любой \textit{scp-операнд с заданным значением\scnrolesign} независимо от конкретного типа \textit{scp-оператора} может быть \textit{scp-переменной\scnrolesign}.

Значение операндов, помеченных ролевым отношением \textit{scp-операнд со свободным значением\scnrolesign}, считается свободным (не заданным заранее) в рамках текущего \textit{scp-оператора}. В начале выполнения \textit{scp-оператора} связь между \textit{scp-переменной\scnrolesign}, помеченной данным ролевым отношением, и ее элементом (значением) всегда удаляется. В результате выполнения данного оператора может быть либо сгенерировано новое значение \textit{scp-переменной\scnrolesign}, либо не сгенерировано, тогда \textit{scp-переменная\scnrolesign} будет считаться не имеющей значения. Ни одна \textit{scp-константа\scnrolesign} не может быть помечена как \textit{scp-операнд со свободным значением\scnrolesign}, поскольку константа не может изменять свое значение в ходе интерпретации \textit{scp-программы}.

Ролевое отношение \textit{тип \mbox{sc-элемента\scnrolesign}} используется для уточнения типа \textit{sc-элемента}, выступающего в роли значения некоторого операнда. \textit{тип \mbox{sc-элемента\scnrolesign}} имеет смысл указывать только для операндов, помеченных как \textit{scp-операнд со свободным значением\scnrolesign}, тогда данное уточнение типа \textit{\mbox{sc-элемента}} будет использовано для сужения области поиска либо уточнения параметров генерации каких-либо конструкций. Значением \textit{scp-операндов с заданным значением\scnrolesign} является конкретный, известный на момент начала выполнения \textit{scp-оператора sc-элемент} с конкретным типом, не зависящим от указания \textit{типа sc-элемента\scnrolesign}, в связи с чем использование ролевого отношения \textit{тип sc-элемента\scnrolesign} в данном случае является некорректным.

Допускается использование комбинаций семантически непротиворечащих друг другу подмножеств указанного отношения. Например, допускается комбинация \textit{константный sc-элемент\scnrolesign} и \textit{sc-дуга общего вида\scnrolesign}, но не допускается комбинация \textit{sc-узел\scnrolesign} и \textit{sc-дуга\scnrolesign}.

Ролевое отношение \textbf{\textit{формируемое множество\scnrolesign}} используется для указания того факта, что в результате выполнения \textit{scp-оператора} должно быть сформировано либо дополнено некоторое множество \textit{sc-элементов}, являющееся значением одного из операндов данного \textit{scp-оператора}. При этом если данный операнд помечен как \textit{scp-операнд со свободным значением\scnrolesign}, то множество будет сформировано с нуля (сгенерирован новый \textit{sc-элемент}, обозначающий данное множество), в противном случае уже существующее множество может быть дополнено. Использование данного ролевого отношения предполагает, что при его отсутствии множество бы не формировалось, а значением указанного операнда стал бы произвольный \textit{sc-элемент} из данного множества. 

Ролевое отношение \textit{формируемое множество\scnrolesign} без уточнения порядкового номера используется только в \textit{scp-операторах обработки произвольных конструкций}. Для явного указания номера операнда, которому соответствует \textit{формируемое множество\scnrolesign}, используются подмножества данного ролевого отношения, аналогичные ролевым отношениям, задающим порядок элементов в кортеже (\textit{1\scnrolesign, 2\scnrolesign, 3\scnrolesign} и т. д.), например \textit{формируемое множество 1\scnrolesign}, \textit{формируемое множество 2\scnrolesign} и т. д. Указанные ролевые отношения используются только в \textit{scp-операторах поиска конструкций с формированием множеств}.

Ролевое отношение \textbf{\textit{удаляемый sc-элемент\scnrolesign}} используется для указания тех операндов, значение которых должно быть удалено в процессе выполнения \textit{scp-операторов удаления}. Данным ролевым отношением может быть помечен как \textit{scp-операнд с заданным значением\scnrolesign}, так и \textit{scp-операнд со свободным значением\scnrolesign}. При этом удаляемым \textit{sc-элементом} может быть как \textit{scp-константа\scnrolesign}, так и \textit{scp-переменная\scnrolesign} (в случае \textit{scp-переменной\scnrolesign} удаляется не только связка принадлежности между этой \textit{scp-переменной\scnrolesign} и ее значением, но и непосредственно сам \textit{sc-элемент}, являющийся значением).

\begin{SCn}
\scnheader{следует отличать*}
\begin{scnhaselementset}
	\scnitem{scp-переменная\scnrolesign}
	\scnitem{sc-переменная}	
\end{scnhaselementset}
\begin{scnhaselementset}
	\scnitem{scp-константа\scnrolesign}
	\scnitem{sc-константа}	
\end{scnhaselementset}
\end{SCn}

На рисунках \ref{fig:genElStr5_fafaa} -- \ref{fig:genElStr5_fafaa_2} показан пример работы scp-оператора генерации пятиэлементной конструкции. В приведённом примере выполняется генерация пятиэлементной конструкции, которая имеет два scp-операнда с заданным значением. В примере предполагается, что рассматриваемые элементы (some\_node1 и some\_node2) изначально никак не связаны между собой.

\begin{figure}[H]
	\centering
	\includegraphics[scale=0.8]{images/part3/chapter_situation_management/genElStr5_fafaa.png}
	\caption{Пример выполнения scp-оператора генерации пятиэлементной конструкции (вызов scp-оператора)}
	\label{fig:genElStr5_fafaa}
\end{figure}

\begin{figure}[H]
	\centering
	\includegraphics[scale=0.8]{images/part3/chapter_situation_management/genElStr5_fafaa_2.png}
	\caption{Пример выполнения scp-оператора генерации пятиэлементной конструкции (результат выполнения scp-оператора)}
	\label{fig:genElStr5_fafaa_2}
\end{figure}

На рисунках \ref{fig:searchElStr3_faf} -- \ref{fig:searchElStr3_faf_2} приведён пример scp-оператора поиска трехэлементной конструкции, которая имеет два scp-операнда с заданным значением. В примере предполагается, что рассматриваемые элементы (some\_node1 и some\_node2) изначально связаны между собой константной постоянной sc-дугой.

\begin{figure}[H]
	\centering
	\includegraphics[scale=0.8]{images/part3/chapter_situation_management/searchElStr3_faf.png}
	\caption{Пример выполнения scp-оператора поиска трехэлементной конструкции (вызов scp-оператора)}
	\label{fig:searchElStr3_faf}
\end{figure}

\begin{figure}[H]
	\centering
	\includegraphics[scale=0.8]{images/part3/chapter_situation_management/searchElStr3_faf_2.png}
	\caption{Пример выполнения scp-оператора поиска трехэлементной конструкции (результат выполнения scp-оператора)}
	\label{fig:searchElStr3_faf_2}
\end{figure}

На рисунках \ref{fig:erase_edge} -- \ref{fig:erase_edge_2} показан пример scp-оператора удаления одноэлементной конструкции. В примере предполагается, что рассматриваемые элементы (node1 и node2) изначально связаны между собой базовой sc-дугой принадлежности.

\begin{figure}[H]
	\centering
	\includegraphics[scale=0.8]{images/part3/chapter_situation_management/searchElStr3_faf.png}
	\caption{Пример выполнения scp-оператора удаления одноэлементной конструкции (вызов scp-оператора)}
	\label{fig:erase_edge}
\end{figure}

\begin{figure}[H]
	\centering
	\includegraphics[scale=0.8]{images/part3/chapter_situation_management/searchElStr3_faf_2.png}
	\caption{Пример выполнения scp-оператора удаления одноэлементной конструкции (результат выполнения scp-оператора)}
	\label{fig:erase_edge_2}
\end{figure}

\subsection{Операционная семантика Базового языка программирования ostis-систем}

\begin{SCn}
\scnheader{Абстрактная scp-машина}
\begin{scnrelfromset}{декомпозиция абстрактного sc-агента}
	\scnitem{Абстрактный sc-агент создания scp-процессов}
	\scnitem{Абстрактный sc-агент интерпретации scp-операторов}
	\scnitem{Абстрактный sc-агент синхронизации процесса интерпретации scp-программ}
	\scnitem{Абстрактный sc-агент уничтожения scp-процессов}
	\scnitem{Абстрактный sc-агент синхронизации событий в sc-памяти и ее реализации}
	\begin{scnindent}
	\begin{scnrelfromset}{декомпозиция абстрактного sc-агента}		
		\scnitem{Абстрактный sc-агент трансляции сформированной спецификации события в sc-памяти во внутреннее представление}
		\scnitem{Абстрактный sc-агент обработки события в sc-памяти, инициирующего агентную scp-программу}
	\end{scnrelfromset}
	\end{scnindent}
\end{scnrelfromset}
\end{SCn}

Задачей \textit{Абстрактного sc-агента создания scp-процессов} является создание \textit{scp-процессов}, соответствующих заданной \textit{scp-программе}. Данный \textit{\mbox{sc-агент}} активируется при появлении в \textit{sc-памяти} \textit{инициированного действия}, принадлежащего классу \textit{действие интерпретации scp-программы}.  После проверки \textit{sc-агентом} условия инициирования выполняется создание \textit{scp-процесса} с учетов конкретных параметров интерпретации \textit{\mbox{scp-программы}}, после чего осуществляется поиск \textit{начального оператора\scnrolesign \mbox{scp-процесса}} и добавление его во множество \textit{настоящих сущностей}.

Задачей \textit{Абстрактного sc-агента интерпретации scp-операторов} является собственно интерпретация операторов \textit{scp-программы}, то есть выполнение в \textit{sc-памяти} действий, описываемых конкретным \textit{\mbox{scp-оператором}}. Данный \textit{sc-агент} активируется при появлении в \textit{sc-памяти} \textit{scp-оператора}, принадлежащего классу \textit{настоящих сущностей}. После выполнения действия, описываемого \textit{scp-оператором}, \textit{scp-оператор} добавляется во множество \textit{прошлых сущностей}. В случае когда семантика действия, описываемого \textit{\mbox{scp-оператором}}, предполагает возможность ветвления \textit{scp-программы} после выполнения данного \textit{\mbox{scp-оператора}}, то используется одно из подмножеств класса \textit{выполненных действий -- безуспешно выполненное действие} или \textit{успешно выполненное действие}.

Задачей \textit{Абстрактного sc-агента синхронизации процесса интерпретации scp-программ} является обеспечение переходов между \textit{scp-операторами} в рамках одного \textit{scp-процесса}. Данный \textit{sc-агент} активизируется при добавлении некоторого \textit{scp-оператора} во множество \textit{прошлых сущностей}. Далее осуществляется переход по \textit{sc-дуге}, принадлежащей отношению \textit{последовательность действий*} (или более частным отношениям, в случае, если \textit{\mbox{scp-оператор}} был добавлен во множество \textit{успешно выполненных действий} или \textit{безуспешно выполненных действий}). При этом очередной \textit{scp-оператор} становится \textit{настоящей сущностью} (активным \textit{scp-оператором}) в том случае, если хотя бы один \textit{scp-оператор}, связанный с ним входящими \textit{sc-дугами}, принадлежащими отношению \textit{последовательность действий*} (или более частным отношениям), стал \textit{прошлой сущностью} (или, соответственно, подмножеством прошлых сущностей). В случае, когда необходимо дождаться завершения выполнения всех предыдущих операторов, для синхронизации используется оператор класса \textit{конъюнкция предшествующих операторов}.

Задачей \textit{Абстрактного sc-агента уничтожения scp-процессов} является уничтожение \textit{scp-процесса}, т. е. удаление из \textit{sc-памяти} всех \textit{sc-элементов}, его составляющих. Данный \textit{sc-агент} активируется при появлении в \textit{sc-памяти} \textit{scp-процесса}, принадлежащего множеству \textit{прошлых сущностей}.
При этом уничтожаемый \textit{scp-процесс} необязательно должен быть полностью сформирован. Необходимость уничтожения не до конца сформированного \textit{scp-процесса} может возникнуть в случае, если при создании \textit{scp-процесса} возникли проблемы, не позволяющие продолжить создание \textit{scp-процесса} и его выполнение.

Задачей \textit{Абстрактного sc-агента синхронизации событий в sc-памяти и ее реализации} является обеспечение работы \textit{неатомарных sc-агентов}, реализованных на \textit{языке SCP}.

Задачей \textit{\textbf{Абстрактного sc-агента трансляции сформированной спецификации события в sc-памяти во внутреннее представление}} является трансляция ориентированных пар, описывающих \textit{первичное условие инициирования*} некоторого \textit{\mbox{sc-агента}} во внутреннее представление элементарных событий на уровне \textit{\mbox{sc-хранилища}}, при условии, что этот \textit{sc-агент} реализован на платформенно-независимом уровне (с использованием \textit{языка SCP}). Условием инициирования данного \textit{sc-агента} является появление в \textit{\mbox{sc-памяти}} нового элемента множества \textit{активных sc-агентов}, для которого будет найдена и протранслирована соответствующая ориентированная пара.

Задачей \textit{Абстрактного sc-агента обработки события в sc-памяти, инициирующего агентную \mbox{scp-программу}}, является поиск \textit{агентной scp-программы}, входящей во множество \textit{программ sc-агента*} для каждого \textit{sc-агента}, первичное условие инициирования которого соответствует событию, произошедшему в \textit{sc-памяти}, а также генерация и инициирование действия, направленного на интерпретацию этой программы. В результате работы данного \textit{sc-агента} в \textit{sc-памяти} появляется \textit{инициированное действие}, принадлежащее классу \textit{действие} \textit{интерпретации scp-программы}.

%\input{author/references}
\chapauthor{Самодумкин С.А.\\Шункевич Д.В.}
\chapter{Язык вопросов для интеллектуальных компьютерных систем нового поколения}
\chapauthortoc{Самодумкин С.А.\\Шункевич Д.В.}
\label{chapter_requests}

\abstract{Аннотация к главе.}

\section{Синтаксис языка вопросов для ostis-систем}
\section{Денотационная семантика языка вопросов для ostis-систем}
\section{Операционная семантика языка вопросов для ostis-систем}

%\input{author/references}
\chapauthor{Василевская А.П.\\Орлов М.К.\\Шункевич Д.В.}
\chapter{Логические, продукционные и функциональные модели решения задач в ostis-системах}
\chapauthortoc{Василевская А.П.\\Орлов М.К.\\Шункевич Д.В.}
\label{chapter_logic_productions}

\abstract{Аннотация к главе.}

\section{Операционная семантика логических языков, используемых ostis-системами}

Логика решает задачи доказательства истинности высказываний, аргументации того или иного высказывания, задачу генерации и опровержения гипотез. Некоторые гипотезы могут быть опровергнуты, однако извлекая причины того, почему гипотеза опровергнута, можно изменить посылку гипотезы так, чтобы создать новую гипотезу, которая впоследствии может стать теоремой.

Технология OSTIS позволяет интегрировать любые модели решения задач и принципы логического вывода для решения задач в интеллектуальных системах на основе общей формальной модели. Для того, чтобы использовать какую-либо новую или существующую модель, необходимо привести ее к предлагаемому формализму, что позволит интегрировать и синхронизировать ее с уже имеющимися в соответствующей библиотеке совместимых компонентов.

\begin{SCn}
	\scnheader{Предметная область логических формул, высказываний и формальных теорий}
	\begin{scnrelfromlist}{дочерняя предметная область}
		\scnitem{Предметная область логических языков}
		\scnitem{Предметная область логического вывода}
	\end{scnrelfromlist}
	
	\scnheader{Предметная область логических языков}
	\scnrelfrom{дочерняя предметная область}{Предметная область языка логики высказываний}
	
	\scnheader{Предметная область языка логики высказываний}
	\scnrelfrom{дочерняя предметная область}{Предметная область языка логики предикатов}
	
	\scnheader{Предметная область логических моделей решения задач}
	\begin{scnreltolist}{дочерняя предметная область}
		\scnitem{Предметная область логических языков}
		\scnitem{Предметная область логического вывода}
	\end{scnreltolist}
\end{SCn}

Наследование предметных областей позволяет использовать описанные логики и их компоненты при описании любых логик. Базовые понятия позволяют разработчикам интеллектуальной системы добавлять новые логики. Для реализации конкретной логической модели решения задач необходимо создать предметную область, которая будет дочерней по отношению к \textit{Предметной области логических моделей решения задач} и предметной области некоторого \textit{логического языка}, например, языка логики высказываний, языка логики предикатов, языка нечёткой логики и других.

\textit{Предметная область логических формул, высказываний и формальных теорий} задаёт денотационную семантику логических формул, высказываний и формальных теорий и содержит формальную спецификацию понятий, необходимых для формирования логических формул и высказываний любых логик, в том числе традиционных, нечётких, правдоподбных, темпоральных, логик умолчания и любых других. Логические формулы и высказывания интерпретирубтся с помощью понятий, описанных в \textit{Предметной области логических моделей решения задач}, включающую модель и реализацию абстрактных агентов, необходимых для решения логических задач. Эта предметная область включает в себя спецификацию таких понятий, как логический вывод, правила вывода, равносильные преобразования и аксиомные схемы.

Современная логика изучает формальные языки, служащие для выражения логических рассуждений. Логический язык — формальный язык, предназначенный для воспроизведения логических форм контекстов естественного языка, а также выражения логических законов и способов правильных рассуждений в логических теориях, строящихся в данном языке. Логика не изучает то, как были получены знания, она позволяет представлять знания, а также из существующих знаний вывести новые (то есть из имеющихся формул логики вывести новые формулы этой же логики), установить правильность рассуждений.

% Источник этого текста и используемых в нём ссылок здесь https://libeldoc.bsuir.by/bitstream/123456789/30629/1/Golenkov_Grapho.pdf
\textbf{Язык SCL} — подъязык SC-кода для записи логических утверждений. Язык SCL является логическим языком графового типа, используемым ostis-системами. Тексты языка SCL представляют собой однородные семантические сети, являющиеся текстами языка SC. Алфавит языка SCL отдельно не выделяется, так как используется алфавит SC-кода, на котором можно описать любые утверждения, явления, закономерности, программы и любые другие знания. Язык SCL позволяет записывать тексты языка логики высказываний, языка логики предикатов и любых других логических языков. SC-код является метаязыком как для языка SCL, так и для самого себя, то есть он позволяет описывать смысл формул, записанных на SCL. Многие формальные языки, в отличие от SC, недостаточно богаты, чтобы быть метаязыком для самих себя. Специфика выделения языка SCL в том, что тексты этого языка могут обрабатываться особым образом. Над высказываниями языка SCL можно проводить логический вывод.

Одной из важных особенностей SCL является его способность представления текстов языка логики предикатов с учётом семантики этих текстов (высказываний). Язык SCL естественным образом ориентирован на работу в формальной системе языка логики предикатов. Язык SC позволяет записать любые отношения и соответствия в графовом представлении. Значению предиката от некоторого набора sc-переменных соответствует результат операции поиска по шаблону некоторой sc-конструкции (найдена или не найдена), в которую входят sc-константы и/или sc-переменные с соответствующей конфигурацией связей между ними. Подход, основанный на языке SCL для представления формул предоставляет возможность явно не записывать кванторы общности и существования (это не запрещается, однако является излишним). Квантор существования является "встроенным"{} понятием в том смысле, что если некоторый sc-элемент входит в некоторую sc-структуру, то соответствующее понятие существует в этой sc-структуре. Таким образом, квантор существования накладывается автоматически (если иной квантор не наложен явно) на те sc-переменные, которые входят в атомарные логические формулы. Квантор всеобщности накладывается по умолчанию (если иной квантор не наложен явно) на переменные, входящие в связки эквиваленции и импликации в соответствии с денотационной семантикой логических языков.

% Сказать, что поиск по шаблону реализован в базовой платформе
Такие особенности упрощают логический вывод в логике предикатов на языке SCL, так как это избавляет от необходимости приводить высказывание в Сколемовскую нормальную форму за счёт встроенных кванторов и от необходимости процедуры унификации за счёт операции поиска sc-конструкции по шаблону, в которой происходят необходимые подстановки переменных.

Выводом в формальной системе называется любая последовательность формул такая, что любая формула либо аксиома этой формальной системы, либо непосредственное следствие каких-либо предыдущих формул по одному из правил вывода. Идея выводимости центральна в логике: в любой формальной аксиоматической теории `теорема' – это формула, которая выводится из аксиом. Правильность умозаключений вводится и проверяется совершенно формально, без какой-либо связи с истинностью входящих в него посылок, т.е. исключительно с точки зрения структуры рассуждения. С практической точки зрения самое важное свойство такой формальной правильности рассуждений заключается в следующем: если нам удалось доказать, пользуясь методами формальной логики, правильность рассуждения, и нам известно из опыта, что все используемые посылки истинны, то мы можем быть уверены в истинности заключения \cite{vagin}. Истинность используемых посылок задаётся состоянием базы знаний.

Различные логические подходы позволяют проектировать решатели задач для интеллектуальных систем в разных предметных областях, учитывая их специфику. \textit{Машина обработки знаний} каждой конкретной системы во многом зависит от назначения данной системы, множества решаемых задач, предметной области и других факторов. Некоторые операции, необходимые в одной предметной области будут избыточными в другой. Например, в системе, решающей задачи по геометрии, химии и другим естественным наукам обоснованным будет использование дедуктивных методов вывода, поскольку решение задач в таких предметных областях основывается только на достоверных правилах. В системах же медицинской диагностики, к примеру, постоянно возникает ситуация, когда диагноз может быть поставлен только с некоторой долей уверенности и абсолютно достоверным ответ на поставленный вопрос быть не может. В связи с этим возникает необходимость использования различных машин обработки знаний в различных системах, при этом состав и возможности машины обработки знаний в конкретной системе определяется не только непосредственно разработчиком, а требует консультаций с экспертами в данной предметной области. Тем не менее основой для всех видов логик является классическая логика и наиболее общие её методы распространяются на другие логики с некоторыми модификациями, уточнениями и ограничениями.

Приведем краткую классификацию существующих логических методов решения задач:
\begin{itemize}
	\item{\textbf{Классический дедуктивный вывод.} Классический дедуктивный вывод является наиболее популярным при построении автоматических решателей задач, так как всегда дает достоверный результат. Дедуктивный вывод включает в себя прямой и обратный и логический вывод (принцип резолюции, процедуру Эрбрана и др.) \cite{vagin}, все виды силлогизмов \cite{syllogism} и т.д. Основной проблемой дедуктивного вывода является невозможность его использования в ряде случаев, когда отсутствуют достоверные знания.}
	\item{\textbf{Индуктивный вывод.} Индуктивный вывод предоставляет возможность в процессе решения использовать различные предположения, что делает его удобным для использования в слабоформализованных и трудноформализуемых предметных областях, например при построении систем медицинской диагностики. Подробно принципы индуктивного вывода рассмотрены в \cite{inductive_incompleteness}, \cite{inductive}.}
	\item{\textbf{Абдуктивный вывод.} Под абдуктивным выводом в искусственном интеллекте, как правило, понимается вывод наилучшего абдуктивного объяснения, т.е. объяснения некоторого события, ставшего неожиданным для системы. Причем «наилучшим»	считается такое объяснение, которое удовлетворяет специальным критериям, определяемым в зависимости от решаемой задачи и используемой	формализации. Абдуктивный вывод подробно рассматривается в \cite{abductive}, \cite{abductive_diagnistic}.}
	\item{\textbf{Нечеткая логика.} Теория нечетких множеств и, соответственно, нечетких логик, также применяется в системах, связанных с трудноформализуемыми предметными областями \cite{fuzzy_automobile} \cite{fuzzy_logic_picture}. Подробнее теория нечетких логик рассматривается в \cite{fuzzy_inference} и других изданиях.}
	\item{\textbf{Логика умолчаний.} Логика умолчаний применяется, в том числе, для того, чтобы оптимизировать процесс рассуждений,	дополняя процесс достоверного вывода вероятностными  предположениями в тех случаях, когда вероятность ошибки крайне мала. Подробнее логика умолчаний рассмотрена в статьях \cite{default_logic} \cite{default_logic2}.}
	\item{\textbf{Темпоральная логика.} Применение темпоральной логики является очень актуальным для нестатичных предметных областей, в которых истинность того или иного утверждения меняется со временем, что существенно влияет на ход решения какой-либо задачи \cite{temporal_logic2} \cite{temporal_logic}. Следует отметить, что используемый в данной работе язык представления знаний предоставляет все необходимые возможности для описания таких динамических предметных областей.}
\end{itemize}

Формальным уточнением различных моделей обработки информации в графодинамической ассоциативной памяти являются \textbf{абстрактные графодинамические ассоциативные машины}. К числу моделей обработки информации, в частности, относятся модели параллельной переработки знаний, соответствующие различным логикам, различным стратегиям решения задач \cite{sc_machines}. 

Преимущество использования графодинамических ассоциативных машин в качестве инструмента для создания интеллектуальных компьютерных систем нового поколения обусловлено следующими обстоятельствами:

\begin{itemize}
	\item{Принципиально проще реализуется ассоциативный метод доступа к перерабатываемой
		информации;}
	\item{Существенно проще поддерживается открытый характер как самих машин, так и реализуемых на них формальных моделей;}
	\item{Являются удобной основой для интеграции различных моделей обработки информации.}
\end{itemize}

Остальные достоинства графодинамических ассоциативных машин обусловлены достоинствами графовых текстов и графовых языков.

База знаний интеллектуальной системы включает в себя как модель фактографических знаний о предметной области, для которой предназначена система, так и модель знаний, включающая в себя логические формулы об этой предметной области (аксиомы, теоремы и правила вывода).

\textbf{Абстрактная scl-машина} является машиной логического вывода и относится к классу абстрактных sc-машин \cite{scl}. Внутренним языком scl-машины является указанный выше графовый логический язык SCL, её операции соответствуют правилам логического вывода. Семейство специализированных абстрактных графодинамических машин обработки знаний является формальным уточнением операционной семантики указанных выше специализированных графовых языков представления знаний, каждому из которых соответствует одна или несколько абстрактных машин. Эти абстрактные машины соответствуют различным моделям решения задач, различным логикам, различным моделям правдоподобных рассуждений. 
Агент из семейства агентов логического вывода может представлять собой какое-либо правило вывода, которое можно применять для решения логической задачи. Кроме того, необходимы агенты для выполнения равносильных преобразований логической формулы (например, записать формулу эквиваленции как конъюнкцию двух дизъюнкций) и другие агенты, помогающие применять правила вывода на множестве формул языка логики.

% Показать разные scl-машины в разных логиках. Дедуктивный вывод это база
\begin{SCn}
	\scnheader{Абстрактная scl-машина}
	\begin{scnrelfromset}{декомпозиция абстрактного sc-агента}
		\scnitem{Абстрактный sc-агент применения правила вывода}
		\scnitem{Абстрактный sc-агент эквиалентных преобразований логической формулы}
		\scnitem{Абстрактный sc-агент прямого логического вывода}
		\scnitem{Абстрактный sc-агент обратного логического вывода}	
	\end{scnrelfromset}
\end{SCn}

% Проверить во всей статье корректность использования "логическое правило" и "логическая формула"
Задачей Абстрактного sc-агента применения правила вывода является применение заданного правила вывода с заданными логическими формулами. Данный sc-агент активируется при появлении в sc-памяти инициированного действия, принадлежащего классу \textit{действие применение правила вывода}. После проверки sc-агентом условия инициирования выполняется процесс применения правила вывода, который заключается в  проверке, существует ли в sc-памяти структуры, соответствующие условию применения данного правила и генерации sc-конструкций в соответствии с применяемым правилом. Агент применения правила вывода зачастую используется в процессе работы агентов прямого логического вывода, обратного логического вывода и других агентов. Примером правила вывода может быть правило Modus ponens, представленное на рисунке \ref{fig:modus_ponens}.

\begin{figure}[http]
	\includegraphics[scale=0.8]{author/part3/figures/Modus_ponens.png}
	\caption{Формализация правила вывода Modus ponens}
	\label{fig:modus_ponens}
\end{figure}

Задачей Абстрактного sc-агента эквиалентных преобразований логической формулы является применение некоторых правил, которые приводят логическую формулу в определённый вид. Данный sc-агент активируется при появлении в sc-памяти инициированного действия, принадлежащего классу \textit{действие эквиалентное преобразование логической формулы}. После проверки sc-агентом условия инициирования выполняется процесс преобразования формулы из одной формы в другую, при этом никакие новые знания в sc-памяти с точки зрения исследуемой предметной области не генерируются. Ответом данного агента является множество формул, эквивалентных по смыслу, но различных по форме представления. Такими формами могут быть, например, конъюнктивная нормальная форма или дизъюнктивная нормальная форма. Агент эквивалентных преобразований зачастую вызывается в процессе работы агента применения правила вывода, так как логические формулы не всегда находятся в той форме, которая доступна для применения того или иного правила вывода, однако может быть приведена к нужной форме.

Задачей Абстрактного sc-агента прямого логического вывода является генерации новых знаний на основе некоторых логических утверждений. Данный sc-агент активируется при появлении в sc-памяти инициированного действия, принадлежащего классу \textit{действие прямого логического вывода}. После проверки sc-агентом условия инициирования выполняется процесс прямого логического вывода, который состоит из циклических операций применения правил вывода, генерации новых знаний в sc-памяти и проверки некоторого условия, например, появление в памяти sc-элементов из целевой sc-структуры \cite{gavrilova}. Входными аргументами такого агента является целевая структура, множество формул, которые используются в ходе вывода агентом применения правил вывода, множество правил вывода, входная структура и выходная структура. В результате выполнения агентом логического вывода действия, в sc-памяти формируется sc-структура, представляющая собой дерево решения. Это дерево состоит из последовательности узлов, представляющих собой применённые правила, которые привели к появлению в sc-памяти требуемых знаний. Такое дерево может быть пустым в случае, если требуемую структуру не удалось сгенерировать в ходе логического вывода. На рисунке \ref{fig:direct_inference_agent} приведён пример спецификации агента прямого логического вывода.

\begin{figure}[http]
	\includegraphics[scale=0.8]{author/part3/figures/direct_inference_agent.png}
	\caption{Спецификация агента прямого логического вывода}
	\label{fig:direct_inference_agent}
\end{figure}

Задачей Абстрактного sc-агента обратного логического вывода является проверка гипотез. Некоторые гипотезы могут быть опровергнуты, однако извлекая причины того, почему гипотеза опровергнута, можно изменить посылку гипотезы так, чтобы создать новую гипотезу, которая впоследствии может стать полезной теоремой. Данный sc-агент активируется при появлении в sc-памяти инициированного действия, принадлежащего классу \textit{действие обратного логического вывода}. После проверки sc-агентом условия инициирования выполняется процесс обратного логического вывода, который схож с процессом прямого логического вывода за исключением того, что поиск правил основывается не на посылках формул, а на их следствиях \cite{gavrilova}. Ответом данного агента будет также дерево вывода, которое показывает, с использованием каких правил можно доказать или опровергнуть выдвинутую гипотезу.

\begin{SCn}
	\scnheader{Абстрактный sc-агент эквиалентных преобразований логической формулы}
	\begin{scnrelfromset}{декомпозиция абстрактного sc-агента}
		\scnitem{Абстрактный sc-агент преобразования формулы в конъюнктивную нормальную форму}
		\scnitem{Абстрактный sc-агент преобразования формулы в дизъюнктивную нормальную форму}
		\scnitem{Абстрактный sc-агент применения законов Де Моргана}
		\scnitem{Абстрактный sc-агент эквиалентных преобразований логической формулы по определению} % ???
		\scnitem{Абстрактный sc-агент применения свойств отрицания логических формул}
		\scnitem{Абстрактный sc-агент применения закона идемпотентности логических формул}
		\scnitem{Абстрактный sc-агент применения закона коммутативности логических формул}
		\scnitem{Абстрактный sc-агент применения закона ассоциативности логических формул}
		\scnitem{Абстрактный sc-агент применения закона поглощения логических формул}
		\scnitem{Абстрактный sc-агент применения закона противоречия логических формул}
		\scnitem{Абстрактный sc-агент применения закона двойного отрицания логических формул}
		\scnitem{Абстрактный sc-агент применения закона расщепления логических формул}
	\end{scnrelfromset}
\end{SCn}

% Отсебятина
%В логике выделяют два подхода: содержательный и формальный подход. В содержательном подходе используются истинностные таблицы для анализа логической формулы. Формальный подход основывается на понятии логического вывода без интерпретации таблиц истинности. Для вычисления истинностных таблиц существует тривиальный алгоритм, однако при большом количестве переменных и подформул этот метод является неэффективным. В свою очередь при формальном подходе решение может быть более изящным (коротким), но не очевидно, какую аксиому применить и с какими значениями переменных этой аксиомы с тем или иным правилом вывода. В связи с этим больший практический и научный интерес представляет формальный подход.

Также можно использовать стратегии решения задач путём упрощения задачи (переход от формулировки в терминах предметной области к формулировке на логическом языке):
\begin{itemize}
	\item{операция обобщения;}
	\item{вывод обобщенного логического высказывания;}
	\item{фаззификация;}
	\item{дефаззификация;}
	\item{применение аналогий;}
	\item{и другие.}
\end{itemize}

Применение аналогий:
\begin{itemize}
	\item{генерация логического утверждения по аналогии;}
	\item{восстановление решения после применения аналогии;}
	\item{генерация фактов по аналогии;}
	\item{и другие.}
\end{itemize}

Используя такие агенты можно решать следующие задачи:
\begin{itemize}
	\item{генерация знаний на основании определения (эквиваленции)}
	\item{генерация определения (эквиваленции) на основании двух импликаций}
	\item{получение значения некоторой продукции}
	\item{вывод обобщенного высказывания}
	\item{генерация связки отношения на основании логического утверждения}
\end{itemize}
% Я не знаю что это
% \item{получение значения некоторой продукции;}
% \item{Агент генерации связки отношения на основании логического утверждения}

% А где шаблон того, что мы хотим получить в результате?
Под операцией логического вывода понимается некоторый sc-агент, который получает на вход теоретико-множественную пару {S, O}, где S - логическое утверждение произвольной конфигурации, O - совокупность объектов, в семантической окрестности которых необходимо применить утверждение S. Целью такого агента является генерация в памяти новых знаний на основании уже имеющихся, т.е. по сути, применение утверждения S. Указанный процесс поиска ответа можно разделить на следующие этапы:

\begin{itemize}
	\item{\textbf{Этап работы поисковых операций.} Вне зависимости от типа поставленного вопроса всегда имеется вероятность того, что данная задача уже была решена системой ранее или системе уже 	откуда-либо известен ответ на поставленный вопрос. На данном этапе работу осуществляет коллектив поисковых операций, каждая из которых, как правило, соответствует некоторому классу	решаемых задач. Если ответ найден, подсистема обработки знаний прекращает свою работу. В противном случае происходит переход на следующий этап решения.}
	\item{\textbf{Этап применения стратегий решения задач.}	На данном этапе осуществляется выбор между
	различными стратегиями решения задач, и, при необходимости, параллельный запуск различных стратегий. Целью работы каждой из стратегий является получение набора пар, связывающих некоторое множество объектов и логическое утверждение из базы знаний, которое справедливо для классов, которым принадлежат эти объекты в рамках некоторой теории. Впоследствии при	рассмотрении каждого утверждения осуществляется	попытка применить его в рамках некоторой семантической окрестности рассматриваемых объектов, для чего осуществляется переход на следующий этап решения.}
	\item{\textbf{Этап применения правил логического вывода.} На данном этапе происходит попытка применения утверждения, полученного на предыдущем шаге, с целью генерации в системе	новых знаний. Если такое применение справедливо (например, посылка истинна) и имеет смысл (в результате применения будут сгенерированы новые знания), то осуществляется генерация новых знаний на основе одного из правил логического вывода. При этом применение происходит в контексте объекта, рассматриваемого на предыдущем этапе (в общем случае – ряда объектов). Если в данном контексте вывод на основе данного утверждения	невозможен или нецелесообразен, решение возвращается на предыдущий этап. В случае успешного применения утверждения происходит переход к следующему этапу решения.}
	\item{\textbf{Этап верификации и оптимизации сгенерированных знаний и сборки мусора.} На данном этапе происходит интерпретация арифметических отношений, сгенерированных в процессе решения на предыдущем этапе, то есть попытка вычисления недостающих значений компонентов связок арифметических отношений
	(например, сложение величин и произведение величин) на основе имеющихся значений. Если вычислить все недостающие значения не представляется возможным, то все знания, сгенерированные на предыдущем этапе,
	уничтожаются и решение переходит на этап применения стратегий. В таком случае применение логического вывода для рассматриваемого на предыдущем шаге утверждения считается не	целесообразным. Также на данном этапе происходит устранение синонимии, если таковая появилась на предыдущем этапе решения,
	например, сгенерирована связка отношения совпадения между некоторыми объектами. В конечном итоге происходит удаление конструкций, ставших ненужными и по каким-либо причинам не удаленных на предыдущих этапах решения. Если все этапы решения выполнены успешно, то решение возвращается к первому этапу, и в случае, если ответ не получен, процесс повторяется еще раз. Стоит отметить, что в процессе решения один и тот же объект или одно и тоже высказывание могут быть использованы многократно, если это целесообразно. Однако, очевидно, что применение одного и того же утверждения для одного объекта несколько раз не имеет смысла, при условии, что нужные знания из памяти не удаляются в процессе решения какими-либо сторонними операциями. Следует учитывать тот факт, что агенты сборки мусора, устранения синонимии и верификации знаний могут оказаться полезными и необходимыми не только на завершающем этапе работы интеллектуального решателя задач. В этом смысле 4-ый этап может быть частично интегрирован с какими-либо из предыдущих.}
\end{itemize}

Таким образом, в структуре описываемой модели можно выделить 4 логических уровня, на каждом из которых возможно использование методов параллельной обработки информации. 

Любая формула эквивалентна некоторой формуле в конъюнктивной нормальной форме, в связи с этим иногда удобно применять правило резолюции. Используя законы Де Моргана можно также получить формулы, пригодные для использования правила резолюции.
С помощью правила резолюции можно эффективно доказывать формулы языка логики высказываний.

Однако ничего принципиально нового правильно резолюции не привносит, поскольку формула $A \Rightarrow B$  равносильно $\neg A \lor B$ и из выводимости A и $A \rightarrow B$ следует выводимость B.

\begin{figure}[H]
	\includegraphics[scale=0.8]{author/part3/figures/conjunction_implication_rule.png}
	\caption{Формализация конъюнктивной нормальной формы для импликации}
	\label{fig:conjunction_implication_rule}
\end{figure}

Если в любых двух дизъюнктах $C_1$ и $C_2$ имеется пара формул $A$ и $\neg A$, то можно сформировать новый дизъюнкт из оставшихся частей изначальных дизъюнктов.

\begin{figure}[H]
	\includegraphics[scale=0.6]{author/part3/figures/resolution.png}
	\caption{Формализация правила резолюции}
	\label{fig:resolution}
\end{figure}

Приведём пример вывода формулы из множества посылок принципом резолюции.
Если команда A выигрывает в футбол, то город A' торжествует, а если выигрывает команда B, то торжествовать будет город B'. Выиграть может или только город A', или только город B'. Однако, если выигрывает команда A, то город B' не торжествует, а если выигрывает команда B, то не торжествует город A'. Следовательно, город B' торжествует тогда и только тогда, когда не будет торжествовать город A'. Цель логического вывода - удостовериться, что город B' торжествует тогда и только тогда, когда не будет торжествовать город A'. Доказать вывода формулы равносильно доказательству противоречивости вывода отрицания этой формулы. При использовании правила резолюции это особенно удобно использовать.
Формализация логических формул, соответствующих примеру приведена на рисунке ниже. Каждая неатомарная формула на рисунке принадлежит некоторой формальной теории, то есть считается истинной.

\begin{figure}[H]
	\includegraphics[scale=0.8]{author/part3/figures/resolution_formulas_example.png}
	\caption{Формализация правил для применения правила резолюции}
	\label{fig:resolution_formulas}
\end{figure}

Структура A представляет собой атомарную логическую формулу, которая обозначает победу команды A, структура A' представляет формулу, обозначающую торжество города A'. Соответственно, то же самое для структур B и B'.
Прежде всего необходимо привести импликацию в конъюнктивную нормальную форму по формуле \ref{fig:conjunction_implication_rule} и эквиваленцию по определению. А также применим отрицание к формуле, которую необходимо вывести (эвиваленция). В результате получим следующие формулы:

\begin{figure}[H]
	\includegraphics[scale=0.8]{author/part3/figures/resolution_prepared_formulas_example.png}
	\caption{Формализация правил для применения правила резолюции после преобразования в конъюнктивную нормальную форму}
	\label{fig:resolution_formulas}
\end{figure}

Далее применяя правило резолюции для преобразованных формул получаем пустой дизъюнкт, что говорит о противоречивости множества формул и доказывает формулу эквиваленции о том, что город B' торжествует тогда и только тогда, когда не будет торжествовать город A'.

\begin{figure}[H]
	\includegraphics[scale=0.7]{author/part3/figures/resolution_inference.png}
	\caption{Применение принципа резолюции}
	\label{fig:resolution_inference}
\end{figure}

% Привести сравнение формального вывода этой же формулы через MP и аксиомные схемы, сделать сравнение

\section{Языки продукционного программирования, используемые ostis-системами}
\subsection{Синтаксис языков продукционного программирования, используемых ostis-системами}
\subsection{Денотационная семантика языков продукционного программирования, используемых ostis-системами}
\subsection{Операционная семантика языков продукционного программирования, используемых ostis-системами}

%\input{author/references}

\chapter{Конвергенция и интеграция искусственных нейронных сетей с базами знаний в интеллектуальных компьютерных системах нового поколения}
\label{chapter_ann}

\abstract{Аннотация к главе.}

\section{Заголовок параграфа}
%\label{sec}
Текст параграфа

%\input{author/references}
%
\begin{partbacktext}
\part{Онтологические модели интерфейсов интеллектуальных компьютерных систем нового поколения}
\noindent Описание к главе
\end{partbacktext}

\chapauthor{Садовский М.Е.}
\chapter{Общие принципы организации интерфейсов ostis-систем}
\chapauthortoc{Садовский М.Е.}
\label{chapter_interfaces}

\bigskip
\begin{SCn}
\scntext{аннотация}{В главе рассмотрены принципы организации \textit{интерфейсов интеллектуальных компьютерных систем нового поколения}. Ключевыми свойствами \textit{интерфейсов интеллектуальных компьютерных систем нового поколения} являются адаптивность и мультимодальность, что обеспечивает переход от парадигмы грамотного пользователя к парадигме равноправного сотрудничества пользователя с интеллектуальной системой, что позволяет повысить эффективность человеко-машинного взаимодействия. В главе рассматривается подход к обеспечению указанных свойств на основе онтологической модели интерфейса и онтологической модели процесса проектирования интерфейсов.}

\bigskip

\begin{scnrelfromlist}{подраздел}
	\scnitem{\ref{sec_analysis}~\nameref{sec_analysis}}	
	\scnitem{\ref{sec_proposed_ui_approach}~\nameref{sec_proposed_ui_approach}}
	\scnitem{\ref{sec_interface_user_actions}~\nameref{sec_interface_user_actions}}
	\scnitem{\ref{sec_messages}~\nameref{sec_messages}}
	\scnitem{\ref{sec_interfaces_actions_and_agents}~\nameref{sec_interfaces_actions_and_agents}}
\end{scnrelfromlist}

\bigskip

\begin{scnrelfromlist}{ключевое понятие}
	\scnitem{}
	\scnitem{}
	\scnitem{}
\end{scnrelfromlist}

\bigskip

\begin{scnrelfromlist}{ключевое знание}
	\scnitem{}
	\scnitem{}
	\scnitem{}
\end{scnrelfromlist}

\bigskip

\begin{scnrelfromlist}{ключевой знак}
	\scnitem{}
	\scnitem{}
	\scnitem{}
\end{scnrelfromlist}

\bigskip

\begin{scnrelfromlist}{библиографическая ссылка}
	\scnitem{}
\end{scnrelfromlist}
\end{SCn}

\section*{Введение в Главу \ref{chapter_interfaces}}

Организация взаимодействия пользователей с компьютерными системами (в том числе и с интеллектуальными компьютерными системами) оказывает существенное влияние на эффективность автоматизации человеческой деятельности, пользовательский опыт и уровень удовлетворенности пользователей. 

Одним из ключевых свойств интеллектуальных компьютерных систем нового поколения является их \myuline{интероперабельность} - способность к эффективному взаимодействию. Такие системы являются автономными и самодостаточными субъектами деятельности наравне с человеком. Однако, в основе современной организации взаимодействия пользователя с компьютерной системой лежит парадигма \myuline{грамотного пользователя}, который знает, как управлять системой и несёт полную ответственность за качество взаимодействия с ней. Многообразие форм и видов интерфейсов приводит к необходимости пользователя  адаптироваться к каждой конкретной системе, обучаться принципам взаимодействия с ней для решения необходимых ему задач.

На современном этапе развития Искусственного интеллекта для повышения эффективности взаимодействия необходим переход от парадигмы грамотного управления используемым инструментом к парадигме \textbf{равноправного сотрудничества, партнёрскому взаимодействию}  интеллектуальной компьютерной системы со своим пользователем. Дружественность пользовательского интерфейса должна заключаться в адаптивности системы к особенностям и квалификации пользователя, исключении любых проблем для пользователя в процессе диалога с интеллектуальной компьютерной системой, в перманентной заботе о совершенствовании коммуникационных навыков пользователя. Следовательно, необходимо отойти от привычной адаптации пользователя к системе (путем обучения ее использованию) в сторону адаптации самого интерфейса под цели, задачи и характеристики конкретного пользователя в режиме реального времени (см. \scncite{fomina}).

\section{Анализ и проблемы существующих принципов организации интерфейсов}
\label{sec_analysis}

\textit{Интерфейс} - совокупность технических, программных и методических (протоколов, правил, соглашений) средств, обеспечивающих обмен информацией между пользователем и устройствами и программами, а также между устройствами и другими устройствами и программами. В широком смысле слова, это способ (стандарт) взаимодействия между объектами. Интерфейс в техническом смысле слова задаёт параметры, процедуры и характеристики взаимодействия объектов.

Интерфейсы бывают разных видов. Они отличаются по характеру систем, которые взаимодействуют между собой; реализацией и функциями.

Вне зависимости от типа интерфейса, взаимодействие компьютерной системы с окружающей средой происходит при помощи \textit{сенсоров} и \textit{эффекторов}.
Ключевая задача интерфейса - обеспечение эффективного взаимодействия с внешними субъектами (пользователями, другими ostis-системами, другими традиционными компьютерными системами).

Принято выделять следующие виды интерфейсов:
\begin{textitemize}
	\item физический интерфейс;
	\item программный интерфейс;
	\item пользовательский интерфейс.
\end{textitemize}

\textit{Физический интерфейс} позволяет преобразовать сигналы и передать их от одного компонента оборудования к другому и определяется набором электрических связей и характеристиками сигналов.

\textit{Программный интерфейс} предназначен для обмена информацией между компонентами вычислительной системы и задает набор необходимых процедур, их параметров и способов обращения.

\textit{Пользовательский интерфейс} - один из наиболее важных компонентов компьютерной системы. Представляет собой совокупность аппаратных и программных средств, обеспечивающих обмен информацией между пользователем и компьютерной системой.

В контексте данной главы основное внимание будет уделено \textit{пользовательским интерфейсам}, другие виды интерфейсов являются объектами будущих исследований.

Ключевыми проблемами современных пользовательских интерфейсов являются:
\begin{textitemize}
	\item \myuline{необходимость пользователя обучаться принципам взаимодействия} с каждой конкретной системой;
	\item \myuline{отсутствие партнерского взаимодействия} между пользователем и системой (система является объектом управления со стороны пользователя), что приводит к необходимости пользователя быть постоянным инициатором взаимодействия;
	\item \myuline{отсутствие адаптации системы} к особенностям каждого конкретного пользователя и окружающей среды для максимально комфортного взаимодействия пользователя с системой.
\end{textitemize}

\textit{Интерфейс интеллектуальных компьютерных систем нового поколения} должен обеспечивать взаимодействие с пользователем на \myuline{равноправной основе, уметь адаптироваться к его особенностям,} а также \myuline{воспринимать различные типы} \myuline{ввода информации}. Для организации такого взаимодействия используются термины адаптивного, интеллектуального и мультимодального интерфейса.

\textit{Адаптивный интерфейс} - пользовательский интерфейс, который изменяется на основе потребностей пользователя или контекста использования.

Как правило, контекст использования состоит из знаний о \myuline{пользователе, платформе и среде}, как показано на рисунке \nameref{fig:use_context} (см. \scncite{jamil_hussain}).

\begin{figure}[H]
	\centering
	\includegraphics[scale=0.5]{author/part4/figures/user-context.png}
	\caption{Контекст использования системы}
	\label{fig:use_context}
\end{figure}

Настройка функциональных возможностей и параметров интерфейса может осуществляться либо вручную самим пользователем, либо автоматически системой на основании имеющейся информации о пользователе. Таким образом, следует различать адаптивные и адаптируемые системы, эти термины не являются синонимами, хотя в литературе довольно часто можно встретить подмену данных понятий (см. \scncite{valeev2}).

В адаптируемых системах любая адаптация является предопределенной и может изменяться пользователями перед запуском системы. В адаптивных же системах, напротив, любая адаптация является динамической, то есть происходит в то же время, когда пользователь взаимодействует с системой, и зависит от поведения пользователя. Но система также может быть адаптируемой и адаптивной одновременно (см. \scncite{Montero}).
Недостаток ручного редактирования интерфейса заключается в необходимости пользователя быть достаточно хорошо знакомым, как с самой системой, так и со средствами, позволяющими изменять ее интерфейс.

Также в литературе можно встретить термин ``адаптированный интерфейс''. 

\textit{Адаптированный интерфейс} - это пользовательский интерфейс, который адаптирован к конечному пользователю при проектировании и не изменяется во время эксплуатации системы (см. \scncite{Schlungbaum1997IndividualUI}).

\textit{Интеллектуальный интерфейс} - пользовательский интерфейс, который может предположить дальнейшие действия пользователей и представить информацию на основе этого предположения. Можно заметить, что понятия интеллектуальный и адаптивный интерфейс имеют отличия. Однако, в различных статьях эти понятия рассматриваются как синонимы.

\textit{Мультимодальный интерфейс} - пользовательский интерфейс, предназначенный для обработки двух или более комбинированных режимов пользовательского ввода, таких как речь, перо, касание, ручные жесты и взгляд, скоординированным образом с выводом мультимедийной системы.

Взаимодействие с большей частью традиционных компьютерных систем происходит с помощью клавиатуры и мыши (тачпада, стилуса). Пользовательский интерфейс таких систем как правило не хранит информацию о модели пользователя, историю его действий и так далее. Традиционный пользовательский интерфейс также не содержит модуль адаптации. На рисунке \nameref{fig:traditional_ui} представлена структура традиционного пользовательского интерфейса (см.\scncite{OSTIS2022Sadovski}).

\begin{figure}[H]
	\centering
	\includegraphics[scale=0.4]{author/part4/figures/traditional_ui.png}
	\caption{Структура традиционного пользовательского интерфейса}
	\label{fig:traditional_ui}
\end{figure}

Общая архитектура адаптивного интеллектуального мультимодального пользовательского интерфейса в свою очередь как правило выглядит так, как представлено на рисунке \nameref{fig:adaptive_ui}.

\begin{figure}[H]
	\centering
	\includegraphics[scale=0.4]{author/part4/figures/adaptive_ui.jpeg}
	\caption{Структура адаптивного интеллектуального мультимодального пользовательского интерфейса}
	\label{fig:adaptive_ui}
\end{figure}

Среди современных средств создания адаптивных пользовательских интерфейсов можно выделить следующие средства, представленные на рисунке \nameref{fig:adaptive_ui_tools} (см. \scncite{context}).

\begin{figure}[H]
	\centering
	\includegraphics[scale=0.4]{author/part4/figures/adaptive_ui_tools.png}
	\caption{Существующие средства создания адаптивных пользовательских интерфейсов}
	\label{fig:adaptive_ui_tools}
\end{figure}

Вне зависимости от средств создания адаптивных интеллектуальных мультимодальных пользовательских интерфейсов такие системы должны эффективно хранить и обрабатывать знания о пользователе, взаимодействии с ним и другую необходимую информацию. Большинство таких систем используют онтологическую модель для хранения информации для адаптации пользовательского интерфейса. Именно онтологический подход позволяет:
\begin{textitemize}
	\item создать наиболее полное \myuline{унифицированное описание} различных аспектов пользовательского интерфейса;
	\item \myuline{легко интегрировать} различные аспекты пользовательского интерфейса;
	\item \myuline{упростить повторное использование} модели интерфейса.
\end{textitemize}

В рамках онтологического подхода принято выделять онтологии и предметные области. База знаний адаптивного интеллектуального мультимодального интерфейса должна включать как минимум следующие предметные области:
\begin{textitemize}
	\item Предметную область и онтологию модели пользователя; 
	\item Предметную область и онтологию компонентов интерфейса;
	\item Предметную область и онтологию интерфейсных действий;
	\item Предметную область и онтологию контекста использования.
\end{textitemize}

Среди существующих \textit{онтологий модели пользователя} можно выделить онтологию GUMO (см. \scncite{Heckmann}), в рамках которой выделяют:
\begin{textitemize}
	\item физиологическое состояние - может измениться за секунды;
	\item психическое состояние - может измениться за минуты;
	\item эмоциональное состояние - может измениться за часы;
	\item характер - может измениться за месяцы;
	\item личность - может измениться за годы;
	\item демография - обычно не может измениться.
\end{textitemize}

В работе \scncite{paulheim} рассматривается \textit{онтология компонентов интерфейса}, на верхнем уровне которой рассматриваются следующие типы компонентов:
\begin{textitemize}
	\item компонент пользовательского интерфейса для отображения;
	\item декоративный компонент пользовательского интерфейса;
	\item интерактивный компонент пользовательского интерфейса;
	\item компонент ввода данных;
	\item компонент для манипуляции отображением;
	\item компонент для запуска операций;
	\item контейнер;
	\item окно;
	\item модальное окно;
	\item немодальное окно.
\end{textitemize}

В данную онтологию также можно включить классы свойств компонента, которые определяют  оформление внешнего вида интерфейсных элементов, начиная от простых, таких как шрифт, цвет, размер элементов, до составных, содержащих наборы интерфейсных решений (см. \cite{gribova_2022}).

Классификация \textit{интерфейсных действий} представлена в \scncite{paulheim} и содержит следующие основные классы:
\begin{textitemize}
	\item действие мышью;
	\item действие речью;
	\item действие осязания;
	\item действие прикосновения.
\end{textitemize}


\textit{Онтология контекста использования} рассматривается в работе \scncite{dream} и описывает:
\begin{textitemize}
	\item Статус пользователей:
	\begin{textitemize}
		\item Движение (стояние, сидение, ходьба);
		\item Возможность слушать (да, нет);
		\item Возможность печатать (да, нет);
		\item Возможность говорить (да, нет);
		\item Возможность читать (да, нет);
	\end{textitemize}
	\item Естественная среда:
	\begin{textitemize}
		\item Освещение (яркий, умеренно освещенный, темный);
		\item Шум (шумный, тихий);
		\item Ветер (сильный, слабый, безветрие);
		\item Погода (солнечно, облачно, дождливо);
		\item Температура (жарко, тепло, холодно);
		\item Местоположение (в офисе, в аэропорту, на улице, в библиотеке, дома, в торговом центре);
	\end{textitemize}
	\item Особенности устройства:
	\begin{textitemize}
		\item Размер экрана (большой, маленький);
		\item Тип экрана (монохромный, цветной);
		\item Клавиатура (большая, маленькая, виртуальная).
	\end{textitemize}
\end{textitemize}

Для управления взаимодействием пользователя с системой принято использовать \myuline{интеллектуальные агенты}.

Интеллектуальный агент способен выполнять гибкое автономное действие для достижения своих целей. Согласно данному определению, гибкость означает:
\begin{textitemize}
	\item Реактивность: интеллектуальные агенты способны воспринимать свою среду и реагировать в своевременном режиме на изменения в ней, чтобы удовлетворить свои цели проектирования;
	\item Проактивность: интеллектуальные агенты способны проявлять целенаправленное поведение, инициируя действия для достижения своих целей проектирования;
	\item Социальная способность: интеллектуальные агенты способны взаимодействовать с другими агентами (и, возможно, людьми) с целью удовлетворения своих целей проектирования.
\end{textitemize}
Интеллектуальные агенты направлены на единственную цель, но обладают большим знанием о рассуждении в пределах своей деятельности. Умение использовать другие ресурсы (других агентов), предпочтения пользователя или клиента и другие способности являются признаками интеллектуального агента.

В результате проведенного анализа можно сделать следующие выводы:
\begin{textitemize}
	\item Для перехода к парадигме равноправного сотрудничества пользователя и системы интерфейсы должны быть адаптивными, интеллектуальными и мультимодальными. Существующие решения позволяют проектировать такие интерфейсы, однако имеют ряд недостатков, которые будут представлены далее.
	\item Структура интеллектуальных интерфейсов включает базу знаний, модуль управления взаимодействием пользователя с системой.
	\item При проектировании базы знаний активно применяется онтологический подход и уже реализованы некоторые онтологии, которые используются при проектировании интеллектуальных интерфейсов.
	\item Модуль управления взаимодействием пользователя с системой, как правило, реализуется на основе многоагентного подхода.
\end{textitemize} 

Среди недостатков существующих решений можно выделить:
\begin{textitemize}
	\item Существующие решения, как правило, предусматривают \myuline{вопросно-ответный принцип взаимодействия}.
	\item Актуальной остается \myuline{проблема совместимости интеллектуального интерфейса с интеллектуальной системой}, для которой он создается, в силу различий используемых средств и методов при проектировании и реализации.
	\item Актуальной остается \myuline{проблема совместимости компонентов интеллектуального интерфейса} (база знаний и модуль управления взаимодействием) между собой.
\end{textitemize}

Для устранения недостатков существующих решений предлагается использовать онтологический подход на основе семантической модели при проектировании и реализации адаптивного интеллектуального мультимодального пользовательского интерфейса, принципы которого будут рассмотрены далее.

\section{Предлагаемый подход к организации интерфейсов ostis-систем}
\label{sec_proposed_ui_approach}


\textit{Пользовательский интерфейс ostis-системы} представляет собой специализированную \textit{ostis-систему}, ориентированную на решение интерфейсных задач, и имеющую в своем составе базу знаний и решатель задач пользовательского интерфейса ostis-системы.
Для решения задачи построения пользовательского интерфейса в базе знаний \textit{пользовательского интерфейса ostis-системы} необходимо наличие sc-модели \textit{компонентов пользовательского интерфейса}, \textit{интерфейсных действий пользователей}, а также классификации \textit{пользовательских интерфейсов} вцелом. При проектировании интерфейса используется компонентный подход,который предполагает представление всего интерфейса приложения в виде отдельных специфицированных компонентов, которые могут разрабатываться и совершенствоваться независимо.

Описание модели базы знаний и решателя предлагается осуществлять на основе универсального унифицированного языка представления знаний, что обеспечит совместимость между этими компонентами.

Интеллектуальная система, для которой будет создаваться интеллектуальный интерфейс, должна иметь модель, описанную на том же унифицированном языке, что и сам интеллектуальный интерфейс. Это обеспечит совместимость интеллектуальной системы и ее интеллектуального интерфейса.

Решатель задач интеллектуального интерфейса должен быть основан на многоагентном подходе, а сами агенты - иметь возможность инициирования действий и сообщений пользователю и другим агентам.

\textit{Пользовательский интерфейс ostis-системы} является адаптивным интеллектуальным мультимодальным пользовательским интерфейсом, структура которого представлена на рисунке \nameref{fig:ostis_ui_structure}.

\begin{figure}[H]
	\centering
	\includegraphics[scale=0.3]{author/part4/figures/ui_model.png}
	\caption{Структура пользовательского интерфейса ostis-системы}
	\label{fig:ostis_ui_structure}
\end{figure}

\textit{Предметная область пользовательских интерфейсов} представляет собой формализованную типологию пользовательских интерфейсов. Пример фрагмента данной предметной области в базе знаний пользовательского интерфейса будет выглядеть следующим образом.

\begin{SCn}
	\scnheader{интерфейс}
	\scnidtf{совокупность технических, программных и методических (протоколов, правил, соглашений) средств, обеспечивающих обмен информацией между пользователем и устройствами и программами, а также между устройствами и другими устройствами и программами. В широком смысле слова, это способ (стандарт) взаимодействия между объектами. Интерфейс в техническом смысле слова задаёт параметры, процедуры и характеристики взаимодействия объектов}
\end{SCn}

\begin{SCn}
	
	\scnheader{интерфейс}
	\begin{scnrelfromset}{разбиение}
			\scnitem{пользовательский интерфейc}
			\begin{scnindent}
					\scnidtf{один из наиболее важных компонентов компьютерной системы. Представляет собой совокупность аппаратных и программных средств, обеспечивающих обмен информацией между пользователем и компьютерной системой.}
				\end{scnindent}
			\scnitem{программный интерфейс}
			\begin{scnindent}
					\scnidtf{система унифицированных связей, предназначенных для обмена информацией между компонентами вычислительной системы. Программный интерфейс задает набор необходимых процедур, их параметров и способов обращения.}
				\end{scnindent}
			\scnitem{физический интерфейс}
			\begin{scnindent}
					\scnidtf{устройство, преобразующее сигналы и передающее их от одного компонента оборудования к другому. Физический интерфейс определяется набором электрических связей и характеристиками сигналов.}
				\end{scnindent}
		\end{scnrelfromset}
	
\end{SCn}

\begin{SCn}
	\scnheader{адаптивный интерфейс}
	\scnidtf{пользовательский интерфейс, который изменяется на основе потребностей пользователя или контекста}
\end{SCn}

\begin{SCn}
	\scnheader{интеллектуальный интерфейс}
	\scnidtf{пользовательский интерфейс, который может предположить дальнейшие действия пользователей и представить информацию на основе этого предположения}
\end{SCn}

\begin{SCn}
	\scnheader{мультимодальный интерфейс}
	\scnidtf{пользовательский интерфейс, предназначенный для обработки двух или более комбинированных режимов пользовательского ввода, таких как речь, перо, касание, ручные жесты и взгляд, скоординированным образом с выводом мультимедийной системы}
\end{SCn}

\begin{SCn}
	
	\scnheader{пользовательский интерфейс}
	\scnsuperset{командный пользовательский интерфейс}
	\begin{scnindent}
			\scnidtf{пользовательский интерфейс, при котором обмен информацией между компьютерной системой и пользователем осуществляется путем написания текстовых инструкций или команд}
		\end{scnindent}		
	\scnsuperset{WIMP-интерфейс}
	\begin{scnindent}
			\scnidtf{Window, Image, Menu, Pointer - интерфейс}
			\scnidtf{Окно, Образ, Меню, Указатель - интерфейс}
			\scnidtf{пользовательский интерфейс, при котором обмен информацией между компьютерной системой и пользователем осуществляется в форме диалога при помощью окон, меню и других элементов управления}
			\begin{scnindent}
					\scnsuperset{пользовательский интерфейс ostis-системы}
				\end{scnindent}	
		\end{scnindent}
	\scnsuperset{SILK-интерфейс}
	\begin{scnindent}
			\scnidtf{Speech, Image, Language, Knowledge - интерфейс}
			\scnidtf{Речь, Образ, Язык, Знание - интерфейс}
			\scnidtf{пользовательский интерфейс, наиболее приближенный к естественной для человека форме общения. Компьютерная система находит для себя команды, анализируя человеческую речь и находя в ней ключевые фразы. Результат выполнения команд преобразуется в понятную человеку форму, например, в естественно-языковую форму или изображение.}
			\scnsuperset{естественно-языковой интерфейc}
			\begin{scnindent}
					\scnidtf{SILK-интерфейс, обмен информацией между компьютерной системой и пользователем в котором происходит за счёт диалога. Диалог ведётся на одном из естественных языков}
				\end{scnindent}			
			\begin{scnindent}
					\scnsuperset{речевой интерфейc}
					\begin{scnindent}
							\scnidtf{SILK-интерфейс, обмен информацией в котором происходит за счёт диалога, в процессе которого компьютерная система и пользователь общаются с помощью речи. Данный вид интерфейса наиболее приближен к естественному общению между людьми}
						\end{scnindent}
				\end{scnindent}
		\end{scnindent}
	
\end{SCn}

\bigskip

\textit{Предметная область пользователя}, \textit{Предметная область контекста использования}, формализована в соотвествии с материалами, рассмотренными в подразделе \nameref{sec_analysis}.

\textit{Предметная область компонентов интерфейса} содержит классификацию компонентов пользовательского интерфейса, пример приведен далее.

\begin{SCn}

\scnheader{Компонент пользовательского интерфейса}
\scnidtf{знак фрагмента базы знаний, имеющий определённую форму внешнего представления на экране}
\begin{scnrelfromset}{разбиение}
	\scnitem{атомарный компонент пользовательского интерфейса}
	\begin{scnindent}
		\scnidtf{компонент пользовательского интерфейса, не содержащий в своём составе других компонентов пользовательского интерфейса}
	\end{scnindent}
	\scnitem{неатомарный компонент пользовательского интерфейса}
		\begin{scnindent}
		\scnidtf{компонент пользовательского интерфейса, состоящий из других компонентов пользовательского интерфейса}
	\end{scnindent}	
\end{scnrelfromset}

\end{SCn}

\bigskip

\begin{SCn}

\scnheader{визуальная часть пользовательского интерфейса ostis-системы}
\scnidtf{часть базы знаний пользовательского интерфейса ostis-системы, содержащая необходимые для отображения пользовательского интерфейса компоненты}
\scnsubset{неатомарный компонент пользовательского интерфейса}

\end{SCn}

\bigskip
Компоненты пользовательского интерфейса могут быть отнесены к одной из трех категорий: \textit{компонент пользовательского интерфейса для отображения}, \textit{декоративный компонент пользовательского интерфейса}, \textit{интерактивный компонент пользовательского интерфейса}.

Полная классификация компонентов пользовательского интерфейса приведена далее:
\begin{textitemize}
	\item интерактивный компонент пользовательского интерфейса
	\begin{textitemize}
		\item компонент ввода данных
		\begin{textitemize}
			\item компонент ввода данных с прямой ответной реакцией
			\begin{textitemize}
				\item область рисования
				\item ползунок
				\item компонент ввода текста с прямой ответной реакцией (однострочное текстовое поле, многострочное текстовое поле)
				\item компонент выбора (компонент выбора одного значения, компонент выбора нескольких значений)
				\item компонент выбора данных (выбираемый элемент, радиокнопка, переключатель, флаговая кнопка)
			\end{textitemize}
			\item компонент ввода данных без прямой ответной реакции
			\begin{textitemize}
				\item кнопка-счётчик
				\item компонент ввода движений
				\item компонент речевого ввода
			\end{textitemize}
		\end{textitemize}
		\item компонент для представления и взаимодействия с пользователем
		\begin{textitemize}
			\item активирующий компонент
			\item компонент непрерывной манипуляции
			\begin{textitemize}
				\item компонент редактирования размера
				\item полоса прокрутки
			\end{textitemize}
		\end{textitemize}
		\item компонент запроса действий
		\begin{textitemize}
			\item компонент выбора команд
			\begin{textitemize}
				\item пункт меню
				\item кнопка
			\end{textitemize}
			\item компонент ввода команд
		\end{textitemize}
	\end{textitemize}
	\item компонент пользовательского интерфейса для отображения
	\begin{textitemize}
		\item компонент вывода
		\begin{textitemize}
			\item компонент вывода видео
			\item компонент вывода звука
			\item компонент вывода изображения
			\item компонент вывода графической информации
			\begin{textitemize}
				\item индикатор выполнения
				\item диаграмма
				\item карта
			\end{textitemize}
			\item компонент вывода текста
			\begin{textitemize}
				\item сообщение
				\item заголовок
				\item параграф
			\end{textitemize}
		\end{textitemize}
		\item декоративный компонент пользовательского интерфейса
		\begin{textitemize}
			\item пустое пространство
			\item разделитель
		\end{textitemize}
		\item контейнер
		\begin{textitemize}
			\item списковый контейнер
			\item древовидный контейнер
			\item узловой контейнер
			\item таблично-строковый контейнер
			\item таблично-клеточный контейнер
			\item панель вкладок
			\item панель вращения
			\item меню
			\item строка меню
			\item панель инструментов
			\item строка состояния
			\item панель прокрутки
			\item окно
			\begin{textitemize}
				\item модальное окно
				\item немодальное окно
			\end{textitemize}
		\end{textitemize}
	\end{textitemize}
\end{textitemize}

В рамках \textit{Предметной области методик проектирования интерфейсов} предлагается формализовать различные существующие методы проектирования интерфейсов, такие как:
\begin{textitemize}
	\item проектирование пользовательских интерфейсов на основе онтологий(концепция разработки пользовательских интерфйсов на основе онтологий);
	\item методика эргономического проектирования;
	\item методика целеориентированного проектирования.
\end{textitemize}

В рамках \textit{Предметной области средств проектирования интерфейсов} предлагается формализовать существующие средства проектирования интерфейсов (см. \scncite{BradMyers}), такие как:
\begin{textitemize}
	\item средства для поддержки создания интерфейса написанием кода;
	\item интерактивные инструментальные средства;
	\item средства, основанные на создании интерфейса путем связывания отдельно созданных компонентов.
\end{textitemize}

В рамках \textit{Предметной области логических правил адаптации интерфейса} предлагается формализовать типологию логических правил, на основе которых будет происходить адаптация интерфейса к пользователю.


\section{Интерфейсные действия пользователей ostis-систем}
\label{sec_interface_user_actions}


Действие, выполняемое пользователем над некоторым \textit{компонентом пользовательского интерфейса}, называется интерфейсным действием. Для связи данного действия с \textit{компонентом пользовательского интерфейса} и необходимым к выполнению \textit{внутренним действием системы} используется отношение \textit{инициируемое пользовательским интерфейсом действие*}.

Классификация интерфейсных действий:

\begin{SCn}

\scnheader{интерфейсное действие пользователя}
\scnsuperset{действие мышью}
\begin{scnindent}
\scnsuperset{прокрутка мышью}
\begin{scnindent}
\scnsuperset{прокрутка мышью вверх}
\scnsuperset{прокрутка мышью вниз}
\end{scnindent}
\scnsuperset{наведение мышью}
\scnsuperset{отпускание мышью}
\scnsuperset{нажатие мыши}
\begin{scnindent}
\scnsuperset{одиночное нажатие мыши}
\scnsuperset{двойное нажатие мыши}			
\end{scnindent}
\scnsuperset{жест мышью}
\scnsuperset{отведение мышью}		
\scnsuperset{перетаскивание мышью}
\end{scnindent}
\scnsuperset{действие голосом}
\scnsuperset{действие клавиатурой}
\begin{scnindent}
\scnsuperset{нажатие функциональной клавиши}
\scnsuperset{нажатие клавиши набора текста}
\end{scnindent}
\scnsuperset{действие осязанием}	
\scnsuperset{действие сенсором}	
\begin{scnindent}
\scnsuperset{нажатие сенсора}
\begin{scnindent}
\scnsuperset{одиночное нажатие сенсора}
\scnsuperset{двойное нажатие сенсора}
\end{scnindent}
\scnsuperset{жест по сенсору}
\begin{scnindent}
\scnsuperset{жест по сенсору одним пальцем}
\scnsuperset{жест по сенсору несколькими пальцами}
\end{scnindent}
\scnsuperset{отпускание сенсором}
\scnsuperset{перетаскивание сенсором}
\end{scnindent}
\scnsuperset{действие пером}	
\begin{scnindent}
\scnsuperset{нажатие функциональной клавиши пером}
\scnsuperset{рисование пером}
\scnsuperset{написание текста пером}
\end{scnindent}

\end{SCn}

\bigskip
\textit{Прокрутка мышью} -- интерфейсное действие пользователя, соответствующее прокрутке содержимого некоторого компонента пользовательского интерфейса при помощи мыши.

\textit{Наведение мышью} -- интерфейсное действие пользователя, соответствующее появлению курсора мыши в рамках компонента пользовательского интерфейса.

\textit{Отпускание мышью} -- интерфейсное действие пользователя, соответствующее отпусканию некоторого компонента пользовательского интерфейса в рамках другого компонента пользовательского интерфейса при помощи мыши.

\textit{Нажатие мыши} -- интерфейсное действие пользователя, соответствующее выполнению нажатия мыши в рамках некоторого компонента пользовательского интерфейса.

\textit{Отведение мышью} -- интерфейсное действие пользователя, соответствующее выходу курсора мыши за рамки компонента пользовательского интерфейса.

\textit{перетаскивание мышью} -- интерфейсное действие пользователя, соответствующее перетаскиванию компонента пользовательского интерфейса при помощи мыши.

\textit{Нажатие сенсора} -- интерфейсное действие пользователя, соответствующее выполнению нажатия сенсора в рамках некоторого компонента пользовательского интерфейса.

\textit{Жест по сенсору} -- интерфейсное действие пользователя, соответствующее выполнению некоторого жеста, выполняемого при помощи движения пальцев на экране сенсора.

\textit{отпускание сенсором} -- интерфейсное действие пользователя, соответствующее отпусканию некоторого компонента пользовательского интерфейса в рамках другого компонента пользовательского интерфейса при помощи сенсора.

\textit{перетаскивание сенсором} -- интерфейсное действие пользователя, соответствующее перетаскиванию компонента пользовательского интерфейса при помощи сенсора.

\textit{действие пером} -- интерфейсное действие пользователя, осуществляемое при помощи пера на графическом планшете.

\textit{Класс интерфейсных действий пользователя} -- множество, элементами которого являются классы \textit{интерфейсных действий пользователя}.

При взаимодействии пользователя с \textit{компонентом пользовательского интерфейса} могут быть произведены различные интерфейсные действия. В зависимости от выполненного интерфейсного действия и компонента, над которым оно было выполнено, происходит инициирование некоторого \textit{внутреннего действия системы}. Для задания такого инициируемого при взаимодействии с пользовательским интерфейсом действия и используется указанное отношение. Первым компонентом связки отношения \textit{инициируемое пользовательским интерфейсом действие*} является связка, элементами которой являются элемент множества компонентов пользовательского интерфейса и и элемент множества \textit{класс интерфейсных действий пользователя}. Вторым компонентом является элемент множества \textit{класс внутренних действий системы}.

\section{Сообщения, входящие в ostis-систему и выходящие из неё}
\label{sec_messages}

Сообщение -- дискретная информационная конструкция, используемая в процессе передачи от отправителя к получателю.

В качестве отправителя сообщения может выступать как пользователь системы, так и сама система. В случае ostis-системы сообщение может быть эффекторным либо рецепторным.

Эффекторное сообщение ostis-системы -- сообщение ostis-системы, формируемое самой ostis-системой при возникновении некоторых ситуаций. К ситуациям, инициирующим возникновение эффекторных сообщений, можно отнести:
\begin{textitemize}
	\item ситуации, возникающие при анализе деятельности самого пользователя. Например, задание аргументов, не соответствующих типу инициируемого действия или появление подсказок при использовании компонентов пользовательского интерфейса;
	\item ситуации, возникающие при анализе синтаксиса текстов внешних языков. Например, неполнота сформированного предложения на внешнем языке или использование конструкций, нехарактерных или некорректно использованных в контексте отдельно взятого внешнего языка/
\end{textitemize}

Рецепторное сообщение ostis-системы -- сообщение ostis-системы, являющееся реакцией на императивное сообщение (сообщение, побуждающее к какому-либо действию). Возможными реакциями ostis-системы на императивное сообщение пользователя являются:
\begin{textitemize}
	\item указание факта завершения выполнения некоторой задачи, что, например, характерно для поведенческих действий;
	\item получение ответа на поставленную задачу, формируемого либо в результате анализа базы знаний	пользовательского интерфейса, либо в результате анализа предметной части базы знаний самой ostis-системы.
\end{textitemize}

Сообщение пользователя ostis-системы может быть сформировано как на внешнем языке (языке, внешнем по отношению к ostis-системе, который не используется для коммуникации внутри системы), так и на внутреннем языке (SC-коде).

Любое сообщение может быть атомарным (не содержащем в своем составе другие сообщения) либо неатомарным (сообщение, в состав которого входят другие сообщения).

Типология сообщений представлена в следующем фрагменте:

\begin{SCn}

\scnheader{сообщение}
\begin{scnrelfromset}{разбиение}
\scnitem{сообщение пользователя системы}
\begin{scnindent}
	\scnsubset{сообщение пользователя ostis-системы}
\end{scnindent}
\scnitem{сообщение системы}
\end{scnrelfromset}
\begin{scnrelfromset}{разбиение}
\scnitem{атомарное сообщение}
\scnitem{атомарное сообщение}
\end{scnrelfromset}
\begin{scnrelfromset}{разбиение}
\scnitem{сообщение на естественном языке}
\scnitem{сообщение на искусственном языке}
\end{scnrelfromset}
\scnsubset{графическое сообщение}
\begin{scnindent}
	\scnidtf{сообщение, содержащее графическую информацию}
	\scnsubset{видео-сообщение}
	\begin{scnindent}
		\scnidtf{сообщение, содержащее видео-информацию}
	\end{scnindent}	
\end{scnindent}
\scnsubset{аудио-сообщение}
\begin{scnindent}
	\scnidtf{сообщение, представленное в звуковом формате}
\end{scnindent}
\scnsubset{обонятельное сообщение}
\begin{scnindent}
	\scnidtf{сообщение, содержащее информацию о запахах}
\end{scnindent}
\scnsubset{текстовое сообщение}
\begin{scnindent}
	\scnidtf{сообщение, содержащее текстовую информацию}
\end{scnindent}
\scnsubset{сообщение, требующее трансляции}
\begin{scnindent}
	\scnidtf{сообщение, которое необходимо сформировать системой для дальнейшей передачи его пользователю}
\end{scnindent}
\scnsubset{протранслированное сообщение}
\begin{scnindent}
	\scnidtf{сообщение, которое было сформировано системой для дальнейшей передачи его пользователю}
\end{scnindent}

\end{SCn}

\section{Действия и внутренние агенты пользовательского интерфейса ostis-системы}
\label{sec_interfaces_actions_and_agents}

Интерфейсное действие пользователя, как правило, инициирует некоторое внутреннее действие системы. 

\begin{SCn}

\scnheader{внутреннее действие системы}
\scnsuperset{внутреннее действие ostis-системы}

\scnheader{внутреннее действие ostis-системы}
\scnidtf{действие в sc-памяти}
\scnidtf{действие, выполняемое в sc-памяти}

\end{SCn}
	
Каждое \textit{внутреннее действие ostis-системы} обозначает некоторое преобразование, выполняемое некоторым \textit{sc-агентом} (или коллективом \textit{sc-агентов}) и ориентированное на преобразование \textit{sc-памяти}.

\begin{SCn}

\scnheader{действие в sc-памяти}
\scnsuperset{действие в sc-памяти, инициируемое вопросом}
\scnsuperset{действие редактирования базы знаний ostis-системы}
\scnsuperset{действие установки режима ostis-системы}
\scnsuperset{действие редактирования файла, хранимого в sc-памяти}
\scnsuperset{действие интерпретации программы, хранимой в sc-памяти}

\end{SCn}

\bigskip
Среди агентов интепретации модели пользовательского интерфейса ostis-систем можно выделить агент генерации интерфейса на основе его модели и агент обработки пользовательских действий.

В качестве входного параметра агент генерации интерфейса на основе его модели принимает экземпляр компонента пользовательского интерфейса для отображения. Результатом работы является графическое представление указанного компонента с учетом используемой реализации платформы интерпретации семантических моделей ostis-систем.

Агент обработки пользовательских действий реагирует на появление в базе знаний системы экземпляра интерфейсного действия пользователя, находит связанный с ним класс внутреннего действия и генерирует экземпляр данного внутреннего действия для последующей обработки.


%\input{author/references}
\chapauthor{Никифоров С.А.\\Гойло А.}
\chapter{Естественно-языковые интерфейсы интеллектуальных компьютерных систем нового поколения}
\chapauthortoc{Никифоров С.А.\\Гойло А.}
\label{chapter_nl_interfaces}

\abstract{Аннотация к главе.}

\section{Заголовок параграфа}
%\label{sec}
Текст параграфа

%\input{author/references}
\chapauthor{Азаров И.С.\\Вашкевич М.И.\\Захарьев В.А.\\Лихачев Д.С.\\Петровский Н.А.}
\chapter{Аудиоинтерфейс ostis-систем}
\chapauthortoc{Азаров И.С.\\Вашкевич М.И.\\Захарьев В.А.\\Лихачев Д.С.\\Петровский Н.А.}
\label{chapter_audio_interfaces}

\abstract{Аннотация к главе.}

\section{Анализ существующих подходов к разработке аудиоинтерфейсов интеллектуальных компьютерных систем}
\section{Применение принципов онтологического проектирования при разработке аудиоинтерфейсов}
\section{Предметная область и онтология аудиоинтерфейса ostis-систем}
\section{Предметная область и онтология задач аудиоинтерфейса ostis-систем}
\section{Предметная область и онтология моделей параметрического представления сигнала}
\section{Предметная область и онтология моделей классификации параметров сигнала}

%\input{author/references}
\chapauthor{Головатый А.И.\\Головатая Е.А.}
\chapter{3D-модель внешней среды в интеллектуальных компьютерных системах нового поколения}
\chapauthortoc{Головатый А.И.\\Головатая Е.А.}
\label{chapter_3d_models}

\abstract{Аннотация к главе.}

%\input{author/references}

\chapter{Модели, методы и средства адаптации пользовательских интерфейсов к носителям китайского языка}
\label{chapter_chinese_interfaces}

\abstract{Аннотация к главе.}

\section{Заголовок параграфа}
%\label{sec}
Текст параграфа

%\input{author/references}
%\begin{partbacktext}
\part{Методы и средства проектирования интеллектуальных компьютерных систем нового поколения}
\label{part_design}

\begin{SCn}
	\scnidtf{Часть 5. Компонентное проектирование интеллектуальных компьютерных систем нового поколения}
	\scnidtf{Часть 5. Компонентное проектирование баз знаний, решателей задач и интерфейсов ostis-систем}
	\scnidtf{Часть 5. Принципы, виды и структура библиотек многократно используемых компонентов ostis-систем}
	
	\scntext{аннотация}{В данной части рассматриваются методы и средства, позволяющие повысить качество и удобство проектирования современных интеллектуальных систем. Принципы, лежащие в основе компонентного проектирования баз знаний, решателей задач и интерфейсов ostis-систем. Приведена типология многократно используемых компонентов ostis-систем, а также модель библиотек многократно используемых компонентов ostis-систем.}

	\bigskip

	\begin{scnrelfromlist}{подраздел}
		\scnitem{Глава~\ref{chapter_library}~\nameref{chapter_library}}
		\scnitem{Глава~\ref{chapter_kb_design}~\nameref{chapter_kb_design}}
		\scnitem{Глава~\ref{chapter_ps_design}~\nameref{chapter_ps_design}}
		\scnitem{Глава~\ref{chapter_ui_design}~\nameref{chapter_ui_design}}
	\end{scnrelfromlist}
\end{SCn}

\section*{Введение в часть \ref{part_design}}

Основным результатом искусственного интеллекта являются не сами интеллектуальные системы, а мощные и эффективные технологии их разработки. Анализ современных технологий проектирования интеллектуальных компьютерных систем показывает, что наряду с весьма впечатляющими достижениями имеют место следующие серьезные проблемы:
\begin{textitemize}
	\item Высокие требования к начальной квалификации пользователей и разработчиков. Технологии искусственного интеллекта не ориентированы на широкий круг разработчиков и пользователей интеллектуальных систем и, следовательно, не получили массового распространения.
	\item Длительные сроки разработки интеллектуальных компьютерных систем и высокий уровень трудоемкости их сопровождения и расширения.
	\item Высока степень зависимости интеллектуальных компьютерных систем и их компонентов друг от друга, отсутствие их автоматической синхронизации. Отсутствие самодостаточности систем и компонентов, способности их функционировать отдельно друг от друга без утраты целесообразности их использования.
	\item Возрастание времени решения задачи при расширении функционала решателя задач и при расширении базы знаний системы.
	\item Отсутствие методик проектирования интеллектуальных компьютерных систем. Обновление компьютерных систем часто сводится к разработке различного рода "заплаток"{}, которые устраняют не причины выявленных недостатков обновляемых компьютерных систем, а только некоторые следствия этих причин.
	\item Отсутствие инструментальных средств проектирования интеллектуальных компьютерных систем, в том числе интеллектуальных обучающих подсистем, подсистем компонентного проектирования компьютерных систем.
\end{textitemize}
Детальнее перечисленные проблемы рассматриваются в работах \scncite{Golenkov2011}, \scncite{Ivashenko2011}, \scncite{Zalivako2011}, \scncite{Koronchik2011}, \scncite{Golenkov2013} и \scncite{Golenkov2014a}.

Для устранения возникших недостатков предлагается использовать технологию, в основе которой лежат следующие принципы, описанные в работах \scncite{Golenkov2011}, \scncite{Zalivako2011}, \scncite{Zalivako2012}, \scncite{Golenkov2013}, \scncite{Davydenko2013}, \scncite{Shunkevich2013} и \scncite{Golenkov2014a}:
\begin{textitemize}
	\item Доступность. Сравнительно неподготовленный пользователь может использовать технологию искусственного интеллекта.
	\item Полнота. Технология должна обеспечивать решение как можно большего набора задач.
	\item Модульность. Ориентация на компонентное (модульное, сборочное) проектирование интеллектуальных систем.
	\item Поэтапное эволюционное проектирование на основе быстрого прототипирования.
	\item Высокий уровень гибкости интеллектуальных систем, простота их модификации (расширяемость). Возможность свободного добавления новых компонентов самого различного вида в интеллектуальную систему. Для этого необходимо достичь максимального уровня независимости эволюции баз знаний от эволюции решателей задач и интеллектуальных интерфейсов этих систем.
	\item Полная совместимость инструментальных средств проектирования, а также инфраструктура проектирования с проектируемыми системами --- инструментальные средства строятся как интеллектуальные системы на основе тех же принципов.
	\item Включение в состав интеллектуальных систем обучающей подсистемы как для разработчиков, так и для конечных пользователей.
	\item Включение подсистем самоанализа, самотестирования, верификации, отладки и самосовершенствования в состав разрабатываемых интеллектуальных систем.
\end{textitemize}

Интеллектуальная система должна быть ориентирована не столько на разработку определенного класса систем, сколько на их постоянное совершенствование на основе указанных принципов.

В состав технологии проектирования интеллектуальных систем входят (см. \scncite{Zalivako2012}, \scncite{Davydenko2013} и \scncite{Shunkevich2013}):
\begin{textitemize}
	\item Модель интеллектуальной компьютерной системы.
	\item Библиотека многократно используемых компонентов интеллектуальных компьютерных систем.
	\item Интеллектуальная интегрированная система автоматизации коллективного проектирования интеллектуальных компьютерных систем, включая подсистемы редактирования, отладки, оценки производительности и визуализации разработанных компонентов, а также подсистему имитационного моделирования.
	\item Методика проектирования интеллектуальных компьютерных систем.
	\item Интеллектуальный пользовательский интерфейс.
	\item Обучающие подсистемы проектирования интеллектуальных компьютерных систем, включая подсистему ведения диалога с разработчиком и пользователем.
	\item Подсистема тестирования и верификации интеллектуальных компьютерных систем, включая подсистему проверки совместимости разработанной системы с другими системами.
\end{textitemize}

\end{partbacktext}

\chapauthor{Орлов М.К.}
\chapter{Комплексная библиотека многократно используемых семантически совместимых компонентов интеллектуальных компьютерных систем нового поколения}
\chapauthortoc{Орлов М.К.}
\label{chapter_library}

\abstract{Аннотация к главе.}

\section{Комплексная библиотека многократно используемых семантически совместимых компонентов ostis-систем}
\section{Многократно используемые компоненты баз знаний ostis-систем}
\section{Многократно используемые встраиваемые ostis-системы}

%\input{author/references}

\chapter{Методика и средства проектирования и анализа качества баз знаний интеллектуальных компьютерных систем нового поколения}
\label{chapter_kb_design}

\abstract{Аннотация к главе.}

\section{Заголовок параграфа}
%\label{sec}
Текст параграфа

%\input{author/references}
\chapauthor{Шункевич Д.В.\\Марковец В.С.}
\chapter{Методика и средства компонентного проектирования решателей задач ostis-систем}
\chapauthortoc{Шункевич Д.В.\\Марковец В.С.}
\label{chapter_ps_design}

\abstract{Аннотация к главе.}

В области разработки \textit{решателей задач} существует большое количество конкретных реализаций, однако вопросы совместимости различных решателей при решении одной задачи практически не рассматриваются. Гипотетически возможно существование универсального решателя задач, объединяющего в себе все известные способы и методы решения задач. Однако использование такого решателя в прикладных целях не является целесообразным. Таким образом, наиболее приемлемым вариантом становится создание библиотеки совместимых между собой компонентов, из которых впоследствии может быть скомпилирован решатель, удовлетворяющий необходимым требованиям.
% \section{Анализ подходов к решению проблем в области построения решателей задач}

% Можно выделить два основных исторически сложившихся подхода к построению решателей задач интеллектуальных систем.

% Первый подход предполагает наличие в системе фиксированного решателя (например, машины логического вывода), к которому впоследствии добавляется база знаний, наполнение которой определяется предметной областью, в которой должна работать система. Такие системы получили название «пустых» экспертных систем \cite{AIRefBookP11990} или «оболочек» (expert system shells, \cite{Jackson1998}). Данный подход, как правило, использовался для разработки относительно несложных систем и не имеет широкого применения.

% Второй подход, широко используемый в настоящее время, предполагает наличие программных средств доступа к информации, хранящейся в некоторой базе, совместимых с различными популярными языками программирования. Данный подход широко используется, например, в системах, построенных на основе стандартов W3C \cite{W3C2017}, которые будут подробнее рассмотрены ниже. Структура решателя, построенного на базе таких средств, определяется разработчиком в каждом конкретном случае и не фиксируется какими-либо стандартами. Такой подход обладает большей гибкостью, но отсутствие унификации в структуре и процессе разработки решателей приводит к отсутствию совместимости компонентов решателей, созданных разными разработчиками, большому количеству дублирований одних и тех же решений, повышению накладных расходов в процессе разработки и поддержки решателя.

% \subsection{Подходы к построению решателей задач, предлагаемые консорциумом W3C}

% При построении интеллектуальных систем различного назначения широко используются стандарты консорциума W3C, одной из задач которого является собственно разработка стандартов в области семантических технологий.

% Достаточно сложно выделить конкретные примеры широко используемых решателей, ориентированных на семантическое представление обрабатываемых знаний на основе стандартов W3C, поскольку отсутствуют общие принципы разработки таких решателей, и в каждом конкретном случае, как правило, разрабатывается свой частный решатель. В связи с этим имеет смысл перейти сразу к рассмотрению подходов к обработке знаний, связанных с таким представлением, выделению их достоинств и недостатков.

% Широко используется стандарт OWL 2 \cite{OWL} для семантического представления знаний и стандарт RDF \cite{RDF} для представления знаний в виде семантических сетей. Для непосредственно хранения RDF-троек используются дополнительно специфицируемые форматы, например, RDF/XML, RDF/JSON, N3 и др. Существует большое количество эффективных хранилищ, обеспечивающих хранение и доступ к данным в форматах RDF.

% Для осуществления доступа к данным, представленных в модели RDF, используется протокол и одноименный язык SPARQL \cite{SPARQL}, который в многом схож с языком SQL. Следующим шагом по отношению к языку SPARQL можно считать декларативный язык запросов Cypher, разработанный создателями хранилища Neo4j \cite{CYPHER}. В запросе на языке Cypher может задаваться граф-образец (the graph pattern to match), на основе которого будет осуществляться изоморфный поиск в хранилище.

% Рассмотренные языки SPARQL и Cypher обеспечивают исключительно доступ к хранимым знаниям, непосредственно обработка осуществляется выше, на уровне приложения, работающего с хранилищем знаний. Кроме того, язык SPARQL не предполагает средств редактирования уже имеющихся в системе знаний.

% Существует большое количество реализаций так называемых ризонеров (semantic reasoners), осуществляющих логический вывод на онтологиях, представленных в формате OWL 2, а также средств разработки и редактирования таких онтологий. Полный список таких средств, признанных консорциумом W3C, можно найти на сайте \cite{OWLImplementations}. Как видно из приведенной на нем таблицы, подавляющее большинство средств способно осуществлять только прямой логический вывод на основе отношений, описанных в онтологии.

% Можно сделать вывод о том, что в рамках консорциума разработаны эффективные средства представления знаний, доступа к ним и механизмы логического вывода на онтологиях, представленных в таком виде. Однако отсутствуют общие принципы построения решателей задач, ориентированных на такое представление, что порождает большое количество дублирований и значительно повышает трудоемкость разработки основанных на таком представлении приложений. 

% \subsection{Комплексные подходы к проектированию интеллектуальных систем различных классов и их компонентов}

% Одним из важнейших способов снижения сроков разработки компьютерных систем любых классов и устранения дублирований при разработке аналогичных решений является обеспечение возможности повторного использования различных компонентов таких систем и накопление библиотек таких компонентов. Отдельное внимание стоит также уделить средствам, обеспечивающим комплексный подход к проектированию интеллектуальных систем какого-либо класса и интеграции различных подходов к решению задач в рамках одной системы.

% Среди комплексных подходов к построению решателей задач, разрабатываемых русскоязычными авторами, можно выделить проект IACPaaS \cite{Gribova2015a,Gribova2011}, активно развивающийся в настоящее время. Целью данного проекта является разработка облачной платформы для построения на ее основе интеллектуальных сервисов различного назначения. В данном проекте активно используются библиотеки многократно используемых компонентов интеллектуальных систем. Конкретно для построения решателей задач, а также пользовательских интерфейсов таких систем используется многоагентный подход. Несмотря на близость некоторых технологических решений, реализуемых в проекте IACPaaS и в данной диссертационной работе, основной целью указанного проекта является предоставление пользователю большого числа разнородных сервисов, выбор которых осуществляется самим пользователем, в то время как в рамках данной диссертационной работы предполагается разработать общую формальную основу для интеграции различных моделей решения задач с целью их комбинирования при решении одной и той же комплексной задачи. Так, в данном проекте осуществляется построение отдельных решателей, ориентированных на решение конкретных классов задач и предполагается возможность последующего их использования клиентами. Однако данный проект не ставит целью решение таких проблем, как разработка комплексной модели решателя, удовлетворяющей изложенным выше требованиям; унификация различных подходов к решению задач на некоторой общей формальной основе; унификация деятельности разработчиков решателей и разработка единых принципов выделения многократно используемых компонентов решателей.

% Комплексные подходы к построению интеллектуальных систем различных классов активно развивались и развиваются в Новосибирске Ю. А. Загорулько и коллегами. Ряд работ по данному направлению был опубликован еще в конце 1980-х гг., например, \cite{Zhagorulko1987,Zhagorulko1988}, но имеются и современные разработки авторов в сфере проектирования интеллектуальных систем \cite{Zhagorulko2013b,Zhagorulko2016}. В частности, в работе \cite{Zhagorulko2016} высказывается тезис о том, что в процессе реализации интеллектуальных систем различных классов большую помощь разработчикам может оказать репозитарий – библиотека готовых к использованию методов принятия решений вместе с методикой и средствами их исполнения и композиции. Кроме этого, в работе \cite{Zhagorulko2013a} формулируется тезис о необходимости интеграции различных методов поддержки принятия решений для решения сложных задач.

% Задачи интеграции различных подходов, в том числе связанных с решением задач, исследуются также в работе \cite{Phylyppov2016} и других работах тех же авторов. В данной работе предлагается подход к построению единой платформы проведения интеллектуального анализа данных, включающей в себя объектно-ориентированные, а также нечеткие модели и программные инструментальные средства работы с ними, с использованием онтологического системного анализа.

% Компонентному проектированию интеллектуальных систем, основанных на знаниях, посвящена работа \cite{Borisov2014}, в которой обосновывается необходимость накопления и повторного использования различных компонентов интеллектуальных систем, предлагаются возможные решения данной проблемы с использованием онтологий.

% Состояние работ англоязычных авторов, посвященных вопросам решения задач в системах, основанных на знаниях, и актуальных на момент начала 1990-х гг., отражено в обзорных публикациях \cite{Dutta1993,Pau1990}. Более поздние англоязычные работы в данной области в основном ориентированы на решение конкретных частных задач в системах, построенных на основе стандартов W3C, о которых более подробно было сказано выше.

% Таким образом, можно сказать, что существует ряд конкретных разработок в направлении построения комплексных технологий разработки интеллектуальных систем различных классов, в том числе с использованием библиотек многократно используемых компонентов, однако проблемы, сформулированные в данной диссертационной работе, не решались или в полной мере не решены ни в одном из этих подходов. Во многом это обусловлено отсутствием унифицированной формальной основы для представления любых видов знаний, в том числе различного рода программ, отсутствием строгих принципов, регламентирующих процесс построения решателей задач, а также средств поддержки разработчиков таких решателей и их компонентов.

\section{Действия и методики проектирования решателей задач ostis-систем}

Методика построения и модификации решателей задач включает несколько этапов. 
\begin{enumerate}
    \item формирование требований и спецификация  решателя задач;
    \item формирование коллектива sc-агентов, входящих в состав разрабатываемого решателя;
    \item разработка алгоритмов атомарных sc-агентов;
    \item реализация scp-программ;
    \item верификация разработанных компонентов;
    \item отладка разработанных компонентов, исправление ошибок.
\end{enumerate}

Методика может быть применена как при разработке объединенных решателей, так и при разработке решателей частного вида, поскольку с формальной точки зрения все они трактуются как неатомарный абстрактный sc-агент.

\textbf{\textit{Этап 1. Формирование требований и спецификация  решателя задач}}

На данном этапе необходимо:
\begin{itemize}
\item четко выделить задачи, решение которых должен обеспечивать решатель задач;
\item продумать предполагаемые способы их решения и на основе данного анализа определить место будущего решателя в общей иерархии решателей.
\end{itemize}

Важность данного этапа заключается в том, что при правильной классификации существует вероятность того, что в составе библиотека многократно используемых компонентов ostis-систем уже есть реализованный вариант требуемого решателя. В противном случае, у разработчика появляется возможность включить разработанный решатель в библиотека многократно используемых компонентов ostis-систем для последующего использования. Данные факты обусловлены тем, что структура библиотеки компонентов решателей задач основана на семантической классификации таких решателей и, соответственно, их компонентов.

При недостаточно четкой спецификации и классификации разрабатываемого решателя повышается вероятность того, что подходящий решатель не будет найден в библиотеке компонентов даже в случае, если он там есть, а вновь разработанный решатель не сможет быть включен в библиотеку. Таким образом, идея многократного использования уже разработанных компонентов будет нарушена, что существенно повысит затраты на разработку такого решателя.

\textbf{\textit{Этап 2. Формирование коллектива sc-агентов, входящих в состав разрабатываемого решателя}}

В случае, когда найти в библиотеке готовый решатель, удовлетворяющий всем предъявляемым требованиям, не представляется возможным, необходимо выделить и специфицировать все компоненты такого решателя.

Результатом данного этапа является перечень полностью специфицированных \textit{sc-агентов}, которые войдут в состав разрабатываемого решателя, с их иерархией вплоть до \textit{атомарных sc-агентов}. В рамках данного этапа очень важно проектировать коллектив агентов таким образом, чтобы максимально задействовать уже имеющиеся в библиотеке многократно используемые компоненты ostis-систем, а при отсутствии нужного компонента –- иметь возможность включить его в библиотеку после реализации.

При разработке перечня агентов (в том числе их спецификаций) необходимо соблюдать ряд принципов:

\begin{itemize}
\item каждый разрабатываемый sc-агент должен быть по возможности предметно независим, т. е. во множество ключевых узлов данного sc-агента не должны входить понятия, имеющие отношение непосредственно к рассматриваемой предметной области. Исключение составляют понятия из общих предметных областей, которые носят междисциплинарный характер (например, отношение \textit{включение*} или понятие \textit{действие}). Данное правило также может быть нарушено в случае, если sc-агент является вспомогательным и ориентирован на обработку какого-либо конкретного класса объектов (например, sc-агенты, выполняющие арифметические вычисления, могут напрямую работать с конкретными отношениями \textit{сложение*} и \textit{умножение*} и т. п.). Всю необходимую для решения задачи информацию sc-агент должен извлекать из семантической окрестности соответствующего инициированного действия. Очевидно, что sc-агент, разработанный с учетом указанных требований, может быть использован при проектировании большего числа ostis-систем, чем в случае, если бы он был реализован с ориентацией на конкретную частную предметную область. После завершения разработки и отладки такой sc-агент должен быть включен в \textit{Библиотеку многократно используемых абстрактных sc-агентов};
\item не стоит путать понятия sc-агент и агентная программа (в том числе агентная scp-программа). Взаимодействие sc-агентов осуществляется исключительно посредством спецификации информационных процессов в общей памяти, каждый sc-агент реагирует на некоторый класс событий в sc-памяти. Таким образом, каждому sc-агенту соответствует некоторое условие инициирования и одна агентная программа, которая запускается автоматически при возникновении в sc-памяти соответствующего условия инициирования. При этом в рамках данной программы могут сколько угодно раз вызываться различные подпрограммы. Однако не стоит путать инициирование sc-агента, которое осуществляется при появлении в sc-памяти соответствующей конструкции, и вызов подпрограммы другой программой, который предполагает явное указание вызываемой подпрограммы и перечня ее параметров;
\item каждый sc-агент должен самостоятельно проверять полноту соответствия своего условия инициирования текущему состоянию sc-памяти. В процессе решения какой-либо задачи может возникнуть ситуация, когда на появление одной и той же структуры среагировали несколько sc-агентов. В таком случае выполнение продолжают только те из них, условие инициирования которых полностью соответствует сложившейся ситуации. Остальные sc-агенты в данном случае прекращают выполнение и возвращаются в режим ожидания. Выполнение данного принципа достигается за счет тщательного уточнения спецификаций разрабатываемых sc-агентов. В общем случае условия инициирования у нескольких sc-агентов могут совпадать, например, в случае, когда одна и та же задача может быть решена разными способами и заранее неизвестно, какой из них приведет к желаемому результату;
% нужен наглядный пример
\item необходимо помнить, что неатомарный sc-агент с точки зрения других sc-агентов, не входящих в его состав, должен функционировать как целостный sc-агент (выполнять логически атомарные действия), что накладывает определенные требования на спецификации атомарных sc-агентов, входящих в его состав: как минимум, необходимо, чтобы в составе неатомарного sc-агента присутствовал хотя бы один атомарный sc-агент, условие инициирования которого полностью совпадает с условием инициирования данного неатомарного sc-агента;
\item при необходимости реализации нового sc-агента следует руководствоваться следующими принципами выделения атомарных абстрактных \textit{sc-агентов}:

\begin{SCn}
\begin{scnindent}
    \begin{scnitemizeii}
        \item проектируемый sc-агент должен быть максимально независим от предметной области, что позволит в дальшейшем использовать его при разработке решателей максимально возможного числа ostis-систем. При этом универсальность предполагает не только минимизацию числа ключевых узлов sc-агента, но и выделение класса действий, выполняемых данным sc-агентом таким образом, чтобы имело смысл включить данный sc-агент в \textit{Библиотеку многократно используемых абстрактных sc-агентов} и использовать его при разработке решателей других ostis-систем. Не следует искусственно увязывать ряд действий в один sc-агент и, наоборот, расчленять одно самодостаточное действие на поддействия: это вызовет сложности восприятия принципов работы sc-агента разработчиками и не позволит использовать sc-агент в ряде систем (например, в обучающих системах, которые должны объяснять ход решения пользователю);
        % а как быть с неатомарными агентами? ведь в них по сути мы вызываем по порядку несколько sc-агентов 
        \item выполняемое данным sc-агентом действие должно быть логически целостным и завершенным. Следует помнить, что все sc-агенты взаимодействуют исключительно через общую sc-память и избегать ситуаций, в которых инициирование одного sc-агента осуществляется путем явной генерации известного условия инициирования другим \textit{sc-агентом} (т. е., по сути, явным непосредственным вызовом одного sc-агента другим);
        \item имеет смысл выделять в отдельные sc-агенты относительно крупные фрагменты реализации некоторого общего алгоритма, которые могут выполняться независимо друг от друга.
    \end{scnitemizeii}
\end{scnindent}
\end{SCn}

\item при объединении sc-агентов в коллективы рекомендуется проектировать их таким образом, чтобы они могли быть использованы не только как часть рассматриваемого неатомарного абстрактного sc-агента. В случае, если это не представляется возможным и некоторые sc-агенты, будучи отделенными от коллектива, теряют смысл, необходимо указать данный факт при документировании рассматриваемых sc-агентов;
\item фактическим инициатором запуска sc-агента посредством общей памяти (автором соответствующей конструкции) может быть как непосредственно пользователь системы, так и другой sc-агент, что никак не должно отражаться в работе самого sc-агента.
\end{itemize}

\textbf{\textit{Этап 3. Разработка алгоритмов атомарных sc-агентов}}

В рамках данного этапа необходимо продумать алгоритм работы каждого разрабатываемого \textit{атомарного sc-агента}. Разработка алгоритма подразумевает выделение в нем логически целостных фрагментов, которые могут быть реализованы как отдельные \textit{scp-программы}, в том числе выполняемые параллельно. Таким образом, появляется необходимость говорить не только о \textit{Библиотеке многократно используемых абстрактных sc-агентов}, но и о \textit{Библиотеке многократно используемых программ обработки sc-текстов} на различных языках программирования, в том числе \textit{Библиотеке многократно используемых scp-программ}. Благодаря этому часть scp-программ, реализующих алгоритм работы некоторого sc-агента, может быть заимствована из соответствующей библиотеки.

Важно помнить, что если в процессе работы \textit{sc-агент} генерирует в памяти какие-либо временные структуры, то при завершении работы он обязан удалять всю информацию, использование которой в системе более нецелесообразно (убрать за собой информационный мусор). Исключение составляют ситуации, когда подобная информация необходима нескольким \textit{sc-агентам} для решения одной задачи, однако после решения задачи информация становится бесполезной или избыточной и требует удаления. В данном случае может возникнуть ситуация, когда ни один из \textit{sc-агентов} не в состоянии удалить информационный мусор. В таком случае возникает необходимость говорить о включении в состав решателя специализированных \textit{sc-агентов}, задачей которых является выявление и уничтожение информационного мусора.

\textbf{\textit{Этап 4. Реализация scp-программ}}

Конечным этапом непосредственно разработки является реализация специфицированных ранее \textit{scp-программ} или при необходимости программ, реализуемых на уровне платформы.

\textbf{\textit{Этап 5. Верификация разработанных компонентов}}

Верификация разработанных компонентов может осуществляться как вручную, так и с использованием специфицированных средств, входящих в состав системы автоматизации проектирования решателей задач по Технологии OSTIS.

\textbf{\textit{Этап 6. Отладка разработанных компонентов. Исправление ошибок}}
Этап отладки разработанных компонентов, в свою очередь, можно также условно разделить на более частные этапы:

\begin{itemize}
    \item отладка отдельных scp-программ или программ, реализуемых на уровне платформы;
    \item отладка отдельных атомарных sc-агентов;
    \item неатомарных sc-агентов, входящих в состав решателя задач;
    \item отладка всего решателя задач.
\end{itemize}

\textit{Этап 5} и \textit{Этап 6} могут выполняться параллельно и повторяются до тех пор, пока разработанные компоненты не будут удовлетворять необходимым требованиям.

Методика построения и модификации решателей задач основана на онтологии деятельности разработчиков таких решателей.

Решатель задач представляет собой \textit{абстрактный sc-агент}, в связи с чем разработка решателя сводится к разработке такого агента.

Фрагмент онтологии деятельности, направленной на построение и модификацию решателей задач:
\begin{SCn}
\scnheader{действие. разработать решатель задач ostis-системы}
\scneq{действие. разработать абстрактный sc-агент}
\begin{scnreltoset}{разбиение}

\scnitem{действие. разработать атомарный абстрактный sc-агент}
\begin{scnindent}
\begin{scnrelfrom}{включение} {действие. разработать платформенно-независимый атомарный абстрактный sc-агент}
\end{scnrelfrom}
\end{scnindent}
\scnitem{действие. разработать неатомарный абстрактный sc-агент}
\end{scnreltoset}

\begin{scnrelfromlist}{абстрактное поддействие}
\scnitem{действие. специфицировать абстрактный sc-агент}
\scnitem{действие. найти в библиотеке абстрактный sc-агент, удовлетворяющий заданной спецификации}
\scnitem{действие. верифицировать sc-агент}
\scnitem{действие. отладить sc-агент}
\end{scnrelfromlist}

\scnheader{действие. разработать платформенно-независимый атомарный абстрактный sc-агент}

\begin{scnrelfromlist}{абстрактное поддействие}
\scnitem{действие. декомпозировать платформенно-независимый атомарный абстрактный sc-агент на scp-программы}
\scnitem{действие. разработать scp-программу}
\end{scnrelfromlist}

\scnheader{действие. разработать неатомарный абстрактный sc-агент}

\begin{scnrelfromlist}{абстрактное поддействие}
    \scnitem{действие. декомпозировать неатомарный абстрактный sc-агент на более частные}
    \scnitem{действие. разработать абстрактный sc-агент}
\end{scnrelfromlist}

\scnheader{действие. разработать scp-программу}

\begin{scnrelfromlist}{абстрактное поддействие}
    \scnitem{действие. специфицировать scp-программу}
    \scnitem{действие. найти в библиотеке scp-программу, удовлетворяющую заданной спецификации}
    \scnitem{действие. реализовать специфицированную scp-программу}
    \scnitem{действие. верифицировать scp-программу}
    \scnitem{действие. отладить scp-программу}
\end{scnrelfromlist}

\scnheader{действие. верифицировать sc-агент}
\begin{scnreltoset}{разбиение}
    \scnitem{действие. верифицировать атомарный sc-агент}
    \scnitem{действие. верифицировать неатомарный sc-агент}
\end{scnreltoset}

\scnheader{действие. отладить sc-агент}

\begin{scnreltoset}{разбиение}
    \scnitem{действие. отладить атомарный sc-агент}
    \scnitem{действие. отладить неатомарный sc-агент}
\end{scnreltoset}
\end{SCn}

Наличие такой онтологии позволяет:
\begin{itemize}
    \item частично автоматизировать процесс построения и модификации решателей;
    \item повысить эффективность информационной поддержки разработчиков, поскольку данная онтология может быть включена в базу знаний интеллектуальной метасистемы IMS.
\end{itemize}

\section{Логико-семантическая модель комплекса ostis-систем автоматизации проектирования решателей задач ostis-систем}

К числу задач системы автоматизации проектирования решателей задач относится техническая поддержка разработчиков решателей, в том числе -- обеспечение корректного и эффективного выполнения этапов, предусмотренных методикой проектирования решателей задач ostis-систем.

При разработке любых компонентов ostis-систем используются схожие принципы. Одним из основных принципов является принцип использования готовых компонентов различного рода, уже имеющихся в библиотеке многократно используемых компонентов ostis-систем, входящей в состав Метасистемы IMS.

Система автоматизации проектировани решателей задач сама по себе также является ostis-системой и имеет соответствующую структуру. Таким образом, модель данной системы включает sc-модель базы знаний, sc-модель объединенного решателя задач и sc-модель пользовательского интерфейса.

В рамках системы условно выделяются две подсистемы -- подсистема автоматизации проектирования агентов обработки знаний и подсистема автоматизации проектирования scp-программ.

% Формально модель такой системы задается следующим образом:

% \begin{equation}
%     M_{SYS} = \{M_{KB}, M_{IPS}, M_{UI}\},	
% \end{equation}
% где $M_{KB}$ – sc-модель базы знаний системы;

% $M_{IPS}$ – sc-модель объединенного решателя задач;

% $M_{UI}$ – sc-модель пользовательского интерфейса системы.

Cистема может использоваться тремя способами:
\begin{itemize}
    \item Как подсистема в рамках метасистемы поддержки проектирования компьютерных систем, управляемых знаниями (IMS). Данный вариант использования предполагает отладку необходимых компонентов в рамках метасистемы с последующим переносом их в дочернюю систему.
    \item Как самостоятельная ostis-система, предназначенная исключительно для разработки и отладки компонентов решателей задач. В этом случае проектируемые компоненты отлаживаются в рамках такой системы, а затем должны быть перенесены в дочернюю ostis-систему.
    \item Как подсистема в рамках дочерней ostis-системы. В таком варианте отладка компонентов осуществляется непосредственно в той же системе, в которой предполагается их использование, и дополнительного переноса не требуется.
\end{itemize}

Независимо от выбранного способа использования системы, разработанные компоненты впоследствии могут быть включены в состав библиотеки многократно используемых компонентов ostis-систем.

Выделяются два принципиально разных уровня отладки решателя задач:
\begin{itemize}
    \item отладка на уровне sc-агентов;
    \item отладка на уровне scp-программ.
\end{itemize}

В случае отладки на уровне sc-агентов акт выполнения каждого агента считается неделимым и не может быть прерван. При этом может выполняться отладка как атомарных sc-агентов, так и неатомарных. Инициирование того или иного агента, в том числе входящего в состав неатомарного, осуществляется путем создания соответствующих конструкций в sc-памяти, таким образом, отладка может осуществляться на разных уровнях детализации агентов, вплоть до атомарных.

Система поддержки проектирования агентов может служить основой для моделирования систем агентов, использующих другие принципы коммуникации, например, непосредственный обмен сообщениями между агентами.

Отладка на уровне scp-программ осуществляется аналогично существующим современным подходам к отладке процедурных программ и предполагает возможность установки точек останова, пошагового выполнения программы и т. д.

Система автоматизации проектирования решателей задач и, соответственно, ее sc-модель, разделяется на две более частные:

\begin{SCn}
\scnheader{Система автоматизации проектирования решателей задач по Технологии OSTIS}

\begin{scnreltoset}{базовая декомпозиция}
\scnitem{Система автоматизации проектирования агентов обработки знаний}
\begin{scnindent}
    \begin{scnreltoset}{базовая декомпозиция}
        \scnitem{База знаний системы автоматизации проектирования агентов обработки знаний}
        \scnitem{Решатель задач системы автоматизации проектирования агентов обработки знаний}
        \scnitem{Пользовательский интерфейс системы автоматизации проектирования агентов обработки знаний}
    \end{scnreltoset}
\end{scnindent}
\scnitem{Система автоматизации проектирования scp-программ}
\begin{scnindent}
    \begin{scnreltoset}{базовая декомпозиция}
        \scnitem{База знаний системы автоматизации проектирования scp-программ}
        \scnitem{Решатель задач системы автоматизации проектирования scp-программ}
        \scnitem{Пользовательский интерфейс системы автоматизации проектирования scp-программ}
    \end{scnreltoset}
\end{scnindent}
\end{scnreltoset}
\end{SCn}

\subsection{Семантическая модель базы знаний системы автоматизации проектирования решателей задач}

База знаний системы автоматизации проектирования решателей задач включает в себя кроме Ядра и расширений ядра sc-моделей баз знаний, предоставляемых на уровне Технологии OSTIS, и моделей предметных областей scp-программ и scp-интерпретатора также описание ключевых понятий, связанных с верификацией и отладкой scp-программ.

Основные понятия, специфичные для базы знаний системы автоматизации проектирования scp-программ.
\begin{SCn}
\scnheader{точка останова*}
\scniselement{квазибинарное отношение}
\end{SCn}

Связки отношения \textit{точки останова*} связывают \textit{scp-программу} с некоторым множеством sc-переменных, соответствующих \textit{scp-операторам} в рамках этой программы. При генерации каждого \textit{scp-процесса}, соответствующего этой \textit{scp-программе}, все \textit{scp-операторы}, соответствующие таким переменным, будут добавлены во множество \textit{точка останова}, т. е. выполнение данного scp-процесса будет прерываться при достижении каждого из этих \textit{scp-операторов}.
Использование данного отношения приводит к указанию точек останова для всех \textit{scp-процессов}, формируемых на основе заданной \textit{\mbox{scp-программы}}. Для указания точки останова в рамках отдельно взятого \textit{scp-процесса} нужный scp-оператор явно включается во множество \textit{точка останова}.

\begin{SCn}
\scnheader{точка останова}
\scnrelto{включение}{scp-оператор}
\end{SCn}

Во множество \textit{точка останова} входят все \textit{scp-операторы}, являющиеся точками останова в рамках какого-либо \textit{scp-процесса}. Это означает, что в момент, когда в соответствии с переходами между \textit{scp-операторами} по связкам отношения \textit{последовательность действий*} указанный \textit{scp-оператор} должен стать \textit{активным действием}, он становится \textit{отложенным действием}, и, соответственно, выполнение всего \textit{scp-процесса} по данной ветке приостанавливается. Чтобы продолжить выполнение, необходимо удалить указанный \textit{\mbox{scp-оператор}} из множества \textit{отложенных действий} и добавить его во множество \textit{активных действий}.

\begin{SCn}
\scnheader{некорректность в scp-программе}
\scnexplanation{Под \textit{некорректностью в scp-программе} понимается \textit{некорректная структура}, описывающая некорректность (не обязательно делающую невозможным выполнение соответствующих данной \textit{scp-программе scp-процессов}), выявленную в рамках какой-либо конкретной \textit{scp-программы}.}
\scnrelto{включение}{некорректная структура}
\scnrelfrom{включение}{ошибка в scp-программе}
\begin{scnrelfromlist}{включение}
    \scnitem{недостижимый scp-оператор}
    \scnitem{потенциально бесконечный цикл}
\end{scnrelfromlist}

\scnheader{ошибка в scp-программе}
\begin{scnindent}
    \scnexplanation{Под \textit{ошибкой в scp-программе} понимается такая \textit{некорректность в scp-программе}, которая делает невозможным успешное выполнение любого \textit{scp-процесса}, соответствующего данной \textit{scp-программе}, или даже создание такого \textit{scp-процесса}.}
\end{scnindent}
\begin{scnreltoset}{разбиение}
    \scnitem{синтаксическая ошибка в scp-программе}
    \begin{scnindent}
        \scnexplanation{Под \textit{синтаксической ошибкой в scp-программе} понимается \textit{ошибка в scp-программе}, в состав которой входит некоторая конструкция, не соответствующая синтаксису \textit{scp-программы} или какой-либо ее части, например, конкретного \textit{scp-оператора}.}
    \end{scnindent}
    \scnitem{семантическая ошибка в scp-программе}
    \begin{scnindent}
        \scnexplanation{Под \textit{семантической ошибкой в scp-программе} понимается \textit{ошибка в scp-программе}, в состав которой входит некоторая конструкция, корректная с точки зрения синтаксиса, но некорректная с семантической точки зрения, например, нарушающая логическую целостность \textit{scp-программы}.}
    \end{scnindent}
\end{scnreltoset}
\begin{scnreltoset}{разбиение}
    \scnitem{ошибка в scp-программе на уровне программы}
    \scnitem{ошибка в scp-программе на уровне множества параметров}
    \scnitem{ошибка в scp-программе на уровне множества операторов}
    \scnitem{ошибка в scp-программе на уровне оператора}
    \scnitem{ошибка в scp-программе на уровне операнда}
\end{scnreltoset}
\end{SCn}

Каждая \textit{ошибка в scp-программе на уровне программы} описывает некорректный фрагмент, выявление которого требует анализа всей \textit{scp-программы} как единого целого, и не может быть выполнено путем анализа ее отдельных частей, например, конкретных \textit{scp-операторов}.

\begin{SCn}
\scnheader{ошибка в scp-программе на уровне программы}
\begin{scnrelfromlist}{включение}
    \scnitem{отсутствует scp-процесс, соответствующий данной scp-программе}
    \begin{scnindent}
        \scniselement{синтаксическая ошибка в scp-программе}
    \end{scnindent}
    \scnitem{не указана декомпозиция scp-процесса, соответствующего данной scp-программе}
    \begin{scnindent}
        \scniselement{синтаксическая ошибка в scp-программе}
    \end{scnindent}
\end{scnrelfromlist}
\end{SCn}

Каждая \textit{ошибка в scp-программе на уровне множества параметров} описывает некорректный фрагмент, для выявления которого достаточно анализа параметров некоторой \textit{scp-программы}, т. е. явным образом выделенных аргументов \textit{действия (scp-процессе)}, соответствующего данной scp-программе. К такого рода ошибкам относятся, например, неверное указание ролей этих аргументов в рамках данного действия.

\begin{SCn}
\scnheader{ошибка в scp-программе на уровне множества параметров}
\begin{scnrelfromlist}{включение}
    \scnitem{не указан тип параметра scp-программы}
    \begin{scnindent}
        \scniselement{синтаксическая ошибка в scp-программе}
    \end{scnindent}
    \scnitem{не указан порядковый номер параметра scp-программы}
    \begin{scnindent}
        \scniselement{синтаксическая ошибка в scp-программе}
    \end{scnindent}
\end{scnrelfromlist}
\end{SCn}

Каждая \textit{ошибка в scp-программе на уровне множества операторов} описывает некорректный фрагмент, для выявления которого достаточно анализа множества операторов некоторой \textit{scp-программы}, т. е. элементов декомпозиции \textit{действия (scp-процесса)}, соответствующего данной \textit{scp-программе}. К таким ошибкам относится, например, факт отсутствия \textit{начального оператора' scp-программы} или факт отсутствия в программе \textit{scp-оператора завершения выполнения программы}.

\begin{SCn}
\scnheader{ошибка в scp-программе на уровне множества операторов}
\begin{scnrelfromlist}{включение}
    \scnitem{декомпозиция scp-процесса не содержит ни одного элемента}
    \begin{scnindent}
        \scniselement{синтаксическая ошибка в scp-программе}
    \end{scnindent}
    \scnitem{отсутствует scp-оператор завершения выполнения программы}
    \begin{scnindent}
        \scniselement{синтаксическая ошибка в scp-программе}
    \end{scnindent}
    \scnitem{scp-оператор, к которому осуществляется переход, не является частью текущего scp-процесса}
    \begin{scnindent}
        \scniselement{синтаксическая ошибка в scp-программе}
    \end{scnindent}
    \scnitem{не указана последовательность действий после выполнения текущего scp-оператора}
    \begin{scnindent}
        \scniselement{синтаксическая ошибка в scp-программе}
    \end{scnindent}
    \scnitem{отсутствует начальный оператор scp-программы}
    \begin{scnindent}
        \scniselement{синтаксическая ошибка в scp-программе}
    \end{scnindent}
\end{scnrelfromlist}
\end{SCn}

Каждая \textit{ошибка в scp-программе на уровне оператора} описывает некорректный фрагмент, для выявления которого достаточно анализа одного конкретного \textit{scp-оператора}, при этом не важно, в состав какой \textit{scp-программы} он входит. К такого рода ошибкам относится, например, факт указания количества операндов \textit{scp-оператора}, не соответствующего спецификации соответствующего класса \textit{scp-операторов}.

\begin{SCn}
\scnheader{ошибка в scp-программе на уровне оператора}
\begin{scnrelfromlist}{включение}
    \scnitem{scp-оператор не принадлежит ни одному из атомарных классов scp-операторов}
    \begin{scnindent}
        \scniselement{синтаксическая ошибка в scp-программе}
    \end{scnindent}
    \scnitem{ни один операнд scp-оператора удаления не помечен как удаляемый sc-элемент}
    \begin{scnindent}
        \scniselement{синтаксическая ошибка в scp-программе}
    \end{scnindent}
    \scnitem{в scp-операторе поиска пятиэлементной конструкции совпадает второй и четвертый операнд}
    \begin{scnindent}
        \scniselement{синтаксическая ошибка в scp-программе}
    \end{scnindent}
    \scnitem{scp-оператор поиска не содержит ни одного операнда с заданным значением}
    \begin{scnindent}
        \scniselement{синтаксическая ошибка в scp-программе}
    \end{scnindent}
    \scnitem{scp-оператор поиска с формированием множеств не содержит ни одного операнда с атрибутом формируемое множество}
    \begin{scnindent}
        \scniselement{синтаксическая ошибка в scp-программе}
    \end{scnindent}
    \scnitem{атрибутом формируемое множество отмечен операнд, которому соответствует операнд с заданным значением}
    \begin{scnindent}
        \scniselement{синтаксическая ошибка в scp-программе}
    \end{scnindent}
    \scnitem{количество операндов scp-оператора не совпадает со спецификацией}
    \begin{scnindent}
        \scniselement{синтаксическая ошибка в scp-программе}
    \end{scnindent}
\end{scnrelfromlist}
\end{SCn}

Каждая \textit{ошибка в scp-программе на уровне оператора} описывает некорректный фрагмент, для выявления которого достаточно анализа одного конкретного операнда в рамках scp-программы, точнее sc-дуги принадлежности, связывающей указанный операнд и соответствующий \textit{scp-оператор}, при этом не важно, какой именно \textit{scp-оператор}. К такого рода ошибкам относится, например, факт отсутствия ролевого отношения, указывающего на номер операнда в рамках \textit{scp-оператора}.

\begin{SCn}
\scnheader{ошибка в scp-программе на уровне операнда}
\begin{scnrelfromlist}{включение}
    \scnitem{не указан номер операнда в рамках scp-оператора}
    \begin{scnindent}
        \scniselement{синтаксическая ошибка в scp-программе}
    \end{scnindent}
\end{scnrelfromlist}

\scnheader{некорректность в scp-программе*}
\scniselement{бинарное отношение}
\scnrelfrom{первый домен}{некорректность в scp-программе}
\scnrelfrom{второй домен*}{scp-программа}
\end{SCn}

Отношение \textit{scp-программа поиска некорректности в scp-программе*} связывает \textit{класс некорректностей в scp-программе} и \textit{scp-программу}, которая может использоваться для выявления соответствующей некорректности в какой-либо другой \textit{scp-программе}. 

Указанная \textit{scp-программа} должна иметь единственный параметр, который является \textit{in-параметром’} и, в зависимости от соответствующего класса некорректностей в \textit{scp-программе}, обозначает:

\begin{itemize}
    \item саму \textit{scp-программу} в случае выявления \textit{некорректности в \mbox{scp-программе}} вообще или \textit{ошибки в scp-программе на уровне программы};
    \item \textit{scp-процесс}, являющийся \textit{ключевым sc-элементом} данной \textit{\mbox{scp-программы}} в случае выявления ошибки в \textit{scp-программе на уровне множества параметров};
    \item \textit{множество операторов} данной \textit{\mbox{scp-программы}} в случае выявления \textit{ошибки в scp-программе на уровне множества операторов};
    \item \textit{знак конкретного scp-оператора} в случае выявления ошибки в \textit{\mbox{scp-программе} на уровне оператора};
    \item \textit{sc-дугу принадлежности} в случае выявления \textit{ошибки в scp-программе на уровне операнда}.
\end{itemize}

Если в результате верификации \textit{scp-программы} выявлена некорректность, то формируется соответствующая \textit{структура} и генерируется связка отношения \textit{некорректность в scp-программе*}.

\subsection{Семантическая модель решателя задач системы автоматизации проектирования решателей задач}

\textit{\textbf{Семантическая модель решателя задач системы автоматизации проектирования агентов обработки знаний}}

\begin{SCn}
\scnheader{Решатель задач системы автоматизации проектирования агентов обработки знаний}
\begin{scnreltoset}{декомпозиция sc-агента}
    \scnitem{Абстрактный sc-агент верификации sc-агентов}
    \begin{scnindent}
        \begin{scnreltoset}{декомпозиция sc-агента}
            \scnitem{Абстрактный sc-агент верификации спецификации sc-агента}
            \scnitem{Абстрактный sc-агент проверки неатомарного sc-агента на непротиворечивость его спецификации спецификациям более частных sc-агентов в его составе}
        \end{scnreltoset}
    \end{scnindent}
    \scnitem{Абстрактный sc-агент отладки коллективов sc-агентов}
    \begin{scnindent}
        \begin{scnreltoset}{декомпозиция sc-агента}
            \scnitem{Абстрактный sc-агент поиска всех выполняющихся процессов, соответствующих заданному sc-агенту}
            \scnitem{Абстрактный sc-агент инициирования заданного sc-агента на заданных аргументах}
            \scnitem{Абстрактный sc-агент активации заданного sc-агента}
            \scnitem{Абстрактный sc-агент деактивации заданного sc-агента}
            \scnitem{Абстрактный sc-агент установки блокировки заданного типа для заданного процесса на заданный sc-элемент}
            \scnitem{Абстрактный sc-агент снятия всех блокировок заданного процесса}
            \scnitem{Абстрактный sc-агент снятия всех блокировок с заданного sc-элемента}
        \end{scnreltoset}
    \end{scnindent}
\end{scnreltoset}
\end{SCn}

Единственным аргументом класса действий, соответствующего \textit{Абстрактному sc-агенту поиска всех выполняющихся процессов, соответствующих заданному sc-агенту, Абстрактному sc-агенту активации заданного sc-агента, Абстрактному sc-агенту деактивации заданного sc-агента}, является знак этого \textit{sc-агента}.

Класс действий, соответствующий \textit{Абстрактному sc-агенту инициирования заданного sc-агента на заданных аргументах}, имеет два аргумента. Первый аргумент является знаком инициируемого sc-агента, второй -- знаком связки, в которую под соответствующими атрибутами входят sc-элементы, которые станут аргументами соответствующего \textit{действия в sc-памяти}.

Класс действий, соответствующий \textit{Абстрактному sc-агенту установки блокировки заданного типа на заданный sc-элемент}, имеет три аргумента. Первый аргумент является знаком класса блокировок, второй -- знаком процесса в sc-памяти, третий -- sc-элементом, на который должна быть установлена блокировка.

Единственным аргументом класса действий, соответствующего \textit{Абстрактному sc-агенту снятия всех блокировок заданного процесса}, является знак этого \textit{процесса в sc-памяти}.

Единственным аргументом класса действий, соответствующего \textit{Абстрактному sc-агенту снятия всех блокировок с заданного sc-элемента}, является знак этого \textit{sc-элемента}.


\textit{\textbf{Семантическая модель решателя задач системы автоматизации проектирования scp-программ}}

\begin{SCn}
\scnheader{Решатель задач системы автоматизации проектирования scp-программ}
\begin{scnreltoset}{декомпозиция sc-агента}
    \scnitem{Абстрактный sc-агент верификации scp-программ}
    \scnitem{Абстрактный sc-агент отладки scp-программ}
    \begin{scnindent}
        \begin{scnreltoset}{декомпозиция sc-агента}
            \scnitem{Абстрактный sc-агент запуска заданной scp-программы для заданного множества входных данных}
            \scnitem{Абстрактный sc-агент запуска заданной scp-программы для заданного множества входных данных в режиме пошагового выполнения}
            \scnitem{Абстрактный sc-агент поиска всех scp-операторов в рамках scp-программы}
            \scnitem{Абстрактный sc-агент поиска всех точек останова в рамках scp-процесса}
            \scnitem{Абстрактный sc-агент добавления точки останова в scp-программу}
            \scnitem{Абстрактный sc-агент удаления точки останова из scp-программы}
            \scnitem{Абстрактный sc-агент добавления точки останова в scp-процесс}
            \scnitem{Абстрактный sc-агент удаления точки останова из scp-процесса}
            \scnitem{Абстрактный sc-агент продолжения выполнения scp-процесса на один шаг}
            \scnitem{Абстрактный sc-агент продолжения выполнения scp-процесса до точки останова или завершения}
            \scnitem{Абстрактный sc-агент просмотра информации об scp-процессе}
            \scnitem{Абстрактный sc-агент просмотра информации об scp-операторе}
        \end{scnreltoset}
    \end{scnindent}
\end{scnreltoset}
\end{SCn}

Алгоритм работы \textit{Абстрактного sc-агента верификации scp-программ} сводится к поиску некорректностей в рамках \textit{scp-программы} на основе определений, соответствующих различным классам таких некорректностей, а также посредством запуска соответствующих данным классам некорректностей \textit{scp-программ поиска некорректности в scp-программе*}.

Результатом работы \textit{Абстрактного sc-агента верификации scp-программ} является формирование в \textit{sc-памяти структур}, описывающих некорректности в исследуемой \textit{scp-программе}, если таковые имеются.

Единственным аргументом класса действий, соответствующего \textit{Абстрактному sc-агенту верификации scp-программ}, является знак верифицируемой \textit{scp-программы}.

Классы действий, соответствующие \textit{Абстрактному sc-агенту запуска заданной scp-программы для заданного множества входных данных} и \textit{Абстрактному sc-агенту запуска заданной scp-программы для заданного множества входных данных в режиме пошагового выполнения}, имеют два аргумента. Первый аргумент является знаком запускаемой scp-программы, второй -- знаком связки, в которую под соответствующими атрибутами входят sc-элементы, которые станут аргументами соответствующего scp-процесса.

В режиме пошагового выполнения предполагается, что на каждом шаге инициируется выполнение всех scp-операторов в рамках заданного \mbox{scp-процесса}, для которых предыдущий scp-оператор стал прошлой сущностью (выполнился). В свою очередь, шаг заканчивается, когда все инициированные таким образом операторы закончат выполнение. Таким образом, в случае, если в рамках scp-программы есть параллельные ветви, то на одном шаге могут одновременно инициироваться два и более scp-оператора.

Классы действий, соответствующие \textit{Абстрактному sc-агенту добавления точки останова в scp-программу, Абстрактному sc-агенту удаления точки останова из scp-программы, Абстрактному sc-агенту добавления точки останова в scp-процесс} и \textit{Абстрактному sc-агенту удаления точки останова из scp-процесса}, имеют два аргумента. Первый аргумент является знаком \mbox{scp-программы} или scp-процесса соответственно, второй -- знаком \mbox{scp-оператора}, входящего в состав этой scp-программы или scp-процесса.

Единственным аргументом классов действий, соответствующих \textit{Абстрактному sc-агенту поиска всех точек останова в рамках scp-процесса, Абстрактному sc-агенту продолжения выполнения scp-процесса на один шаг, Абстрактному sc-агенту продолжения выполнения scp-процесса до точки останова или завершения} и \textit{Абстрактному sc-агенту просмотра информации об scp-процессе}, является знак scp-процесса, с которым будет выполнено соответствующее действие.

Единственным аргументом класса действий, соответствующего \textit{Абстрактному sc-агенту поиска всех scp-операторов в рамках scp-программы}, является знак этой scp-программы.

Единственным аргументом класса действий, соответствующего \textit{Абстрактному sc-агенту просмотра информации об scp-операторе}, является знак scp-оператора, входящего в состав некоторого scp-процесса. Результатом работы данного агента является структура, описывающая значения операндов данного scp-оператора, его атомарный тип и другую служебную информацию, полезную для разработчика.

\subsection{Семантическая модель пользовательского интерфейса системы автоматизации проектирования решателей задач}

Пользовательский интерфейс системы автоматизации проектирования решателей задач представлен набором интерфейсных команд, позволяющих пользователю инициировать деятельность нужного агента, входящего в состав этой системы.

\begin{SCn}
\scnheader{команда пользовательского интерфейса системы автоматизации проектирования решателей задач}
\begin{scnreltoset}{разбиение}
    \scnitem{команда пользовательского интерфейса системы автоматизации проектирования агентов обработки знаний}
    \scnitem{команда пользовательского интерфейса системы автоматизации проектирования программ языка SCP}
\end{scnreltoset}

\scnheader{команда пользовательского интерфейса системы автоматизации проектирования агентов обработки знаний}
\begin{scnreltoset}{разбиение}
    \scnitem{команда верификации sc-агентов}
    \begin{scnindent}
        \begin{scnreltoset}{разбиение}
            \scnitem{команда верификации спецификации sc-агента}
            \scnitem{команда верификации неатомарного sc-агента на непротиворечивость его спецификации спецификациям более частных sc-агентов в его составе}
        \end{scnreltoset}
    \end{scnindent}
    \scnitem{команда отладки коллективов sc-агентов}
    \begin{scnindent}
        \begin{scnreltoset}{разбиение}
            \scnitem{команда поиска всех выполняющихся процессов, соответствующих заданному sc-агенту}
            \scnitem{команда инициирования заданного sc-агента на заданных аргументах}
            \scnitem{команда активации заданного sc-агента}
            \scnitem{команда деактивации заданного sc-агента}
            \scnitem{команда установки блокировки заданного типа для заданного процесса на заданный sc-элемент}
            \scnitem{команда снятия всех блокировок заданного процесса}
            \scnitem{команда снятия всех блокировок с заданного sc-элемента}
        \end{scnreltoset}
    \end{scnindent}
\end{scnreltoset}

\scnheader{команда пользовательского интерфейса системы автоматизации проектирования scp-программ}
\begin{scnreltoset}{разбиение}
    \scnitem{команда верификации scp-программ}
    \scnitem{команда отладки scp-программ}
    \begin{scnindent}
        \begin{scnreltoset}{разбиение}
            \scnitem{команда запуска заданной scp-программы для заданного множества входных данных}
            \scnitem{команда запуска заданной scp-программы для заданного множества входных данных в режиме пошагового выполнения}
            \scnitem{команда поиска всех scp-операторов в рамках scp-программы}
            \scnitem{команда поиска всех точек останова в рамках scp-процесса}
            \scnitem{команда добавления точки останова в scp-программу}
            \scnitem{команда удаления точки останова из scp-программы}
            \scnitem{команда добавления точки останова в scp-процесс}
            \scnitem{команда удаления точки останова из scp-процесса}
            \scnitem{команда продолжения выполнения scp-процесса на один шаг}
            \scnitem{команда продолжения выполнения scp-процесса до точки останова или завершения}
            \scnitem{команда просмотра информации об scp-процессе}
            \scnitem{команда просмотра информации об scp-операторе}
        \end{scnreltoset}
    \end{scnindent}
\end{scnreltoset}

\end{SCn}

% \section{Логико-семантическая модель ostis-системы автоматизации проектирования программ Базового языка программирования ostis-систем}
% \section{Логико-семантическая модель ostis-системы автоматизации проектирования внутренних агентов ostis-систем, а также коллективов таких агентов}
\section{Многократно используемые компоненты решателей задач ostis-систем}

Рассмотрим классификацию многократно используемых компонентов решателей задач ostis-систем.

\begin{SCn}
\scnheader{многократно используемый компонент решателей задач}
\scnsuperset{программа}
\scnsuperset{пакет программ}
\scnsuperset{абстрактный sc-агент}
\scnsuperset{решатель задач ostis-системы}
\begin{scnindent}
    \scnnote{Целые решатели задач могут быть многократно используемыми компонентами в случае разработки интеллектуальных систем, назначение которых совпадает.}
\end{scnindent}

\scnheader{метод}
\scnidtf{программа}
\scnsuperset{программа на основе нейросетевых моделей}
\scnsuperset{программа на основе генетических алгоритмов}
\scnsuperset{императивная программа}
\begin{scnindent}
    \scnsuperset{процедурная программа}
    \scnsuperset{объектно-ориентированная программа}
\end{scnindent}
\scnsuperset{декларативная программа}
\begin{scnindent}
    \scnsuperset{логическая программа}
    \scnsuperset{функциональная программа}
\end{scnindent}
\scnsuperset{программа sc-агента}

\scnheader{абстрактный sc-агент}
\begin{scnsubdividing}
    \scnitem{неатомарный абстрактный sc-агент}
    \scnitem{атомарный абстрактный sc-агент}
\end{scnsubdividing}
\begin{scnsubdividing}
    \scnitem{внутренний абстрактный sc-агент}
    \scnitem{эффекторный абстрактный sc-агент}
    \scnitem{рецепторный абстрактный sc-агент}
\end{scnsubdividing}
\begin{scnsubdividing}
    \scnitem{абстрактный sc-агент, не реализуемый на Языке SCP}
    \scnitem{абстрактный sc-агент, реализуемый на Языке SCP}
\end{scnsubdividing}
\begin{scnsubdividing}
    \scnitem{абстрактный sc-агент интерпретации scp-программ}
    \scnitem{абстрактный программный sc-агент}
    \scnitem{абстрактный sc-метаагент}
\end{scnsubdividing}
\begin{scnsubdividing}
    \scnitem{платформенно-зависимый абстрактный sc-агент}
    \begin{scnindent}
        \scnsuperset{абстрактный sc-агент, не реализуемый на Языке SCP}
    \end{scnindent}
    \scnitem{платформенно-независимый абстрактный sc-агент}
\end{scnsubdividing}

\scnheader{решатель задач ostis-системы}
\scnsuperset{решатель задач с использованием хранимых методов}
\begin{scnindent}
    \scnsuperset{решатель задач на основе нейросетевых моделей}
    \scnsuperset{решатель задач на основе генетических алгоритмов}
    \scnsuperset{решатель задач на основе императивных программ}
    \begin{scnindent}
        \scnsuperset{решатель задач на основе процедурных программ}
        \scnsuperset{решатель задач на основе объектно-ориентированных программ}
    \end{scnindent}
    \scnsuperset{решатель задач на основе декларативных программ}
    \begin{scnindent}
        \scnsuperset{решатель задач на основе логических программ}
        \scnsuperset{решатель задач на основе функциональных программ}
    \end{scnindent}
\end{scnindent}
\scnsuperset{решатель задач в условиях, когда метод решения задач данного класса в текущий момент времени не известен}
\begin{scnindent}
    \scnsuperset{решатель, реализующий стратегию поиска путей решения задачи в глубину}
    \scnsuperset{решатель, реализующий стратегию поиска путей решения задачи в ширину}
    \scnsuperset{решатель, реализующий стратегию проб и ошибок}
    \scnsuperset{решатель, реализующий стратегию разбиения задачи на подзадачи}
    \scnsuperset{решатель, реализующий стратегию решения задач по аналогии}
    \scnsuperset{решатель, реализующий концепцию интеллектуального пакета программ}
\end{scnindent}
% \scnsuperset{машина информационного поиска}
% \begin{scnindent}
%     \scnsuperset{машина информационного поиска информации, удовлетворяющей заданной спецификации}
%     \scnsuperset{машина информационного поиска информации, не удовлетворяющей заданной спецификации}
%     \scnsuperset{машина, выявляющая отсутствие информации, удовлетворяющей заданной спецификации}
% \end{scnindent}
% \scnsuperset{решатель явно сформулированных задач}
% \begin{scnindent}
%     \scnsuperset{машина поиска значений заданного множества величин}
%     \scnsuperset{машина установления истинности заданного логического высказывания в рамках заданной формальной теории}
%     \scnsuperset{машина формирования способа решения указанной задачи}
%     \begin{scnindent}
%         \scnsuperset{машина формирования доказательства заданного высказывания в рамках заданной формальной теории}
%     \end{scnindent}
%     \scnsuperset{машина верификации ответа на указанную задачу}
%     \scnsuperset{машина верификации способа решения указанной задачи}
%     \begin{scnindent}
%         \scnsuperset{машина верификации доказательства заданного высказывания в рамках заданной формальной теории}
%     \end{scnindent}
% \end{scnindent}
% \scnsuperset{машина классификации сущностей}
% \begin{scnindent}
%     \scnsuperset{машина соотнесения сущности с одним из заданного множества классов}
%     \scnsuperset{машина разделения множества сущностей на классы по заданному множеству признаков}
% \end{scnindent}
% \scnsuperset{машина синтеза естественно-языковых текстов}
% \scnsuperset{машина анализа естественно-языковых текстов}
% \begin{scnindent}
%     \scnsuperset{машина распознавания естественно-языковых текстов}
%     \scnsuperset{машина верификации естественно-языковых текстов}
% \end{scnindent}
% \scnsuperset{машина синтеза сигналов}
% \begin{scnindent}
%     \scnsuperset{машина синтеза речи}
% \end{scnindent}
% \scnsuperset{машина анализа сигналов}
% \begin{scnindent}
%     \scnsuperset{машина анализа речи}
%     \begin{scnindent}
%         \scnsuperset{машина распознавания речи}
%     \end{scnindent}
% \end{scnindent}
% \scnsuperset{машина обработки мультимедийных данных}
% \begin{scnindent}
%     \scnsuperset{машина анализа изображений}
%     \begin{scnindent}
%         \scnsuperset{машина распознавания изображений}
%     \end{scnindent}
% \end{scnindent}
% \scnsuperset{решатель задач IMS}
% \scnsuperset{решатель задач вспомогательной компьютерной системы}
% \begin{scnindent}
%     \scnsuperset{решатель задач интерфейса компьютерной системы}
%     \begin{scnindent}
%         \scnsuperset{решатель задач пользовательского интерфейса компьютерной системы}
%         \scnsuperset{решатель задач интерфейса компьютерной системы с другими компьютерными системами}
%         \scnsuperset{решатель задач интерфейса компьютерной системы с окружающей средой}
%     \end{scnindent}
%     \scnsuperset{решатель задач подсистемы поддержки проектирования компонентов определенного класса}
%     \begin{scnindent}
%         \scnsuperset{решатель задач подсистемы поддержки проектирования баз знаний}
%         \begin{scnindent}
%             \scnsuperset{машина повышения качества базы знаний}
%             \begin{scnindent}
%                 \scnsuperset{машина верификации базы знаний}
%                 \begin{scnindent}
%                     \scnsuperset{машина поиска и устранения некорректностей}
%                     \scnsuperset{машина поиска и устранения неполноты}
%                 \end{scnindent}
%                 \scnsuperset{машина оптимизации базы знаний}
%                 \scnsuperset{машина выявления и устранения информационного мусора}
%             \end{scnindent}
%         \end{scnindent}
%         \scnsuperset{решатель задач подсистемы поддержки проектирования решателей}
%         \begin{scnindent}
%             \scnsuperset{решатель задач подсистемы поддержки проектирования программ обработки знаний}
%             \scnsuperset{решатель задач подсистемы поддержки проектирования агентов обработки знаний}
%         \end{scnindent}
%     \end{scnindent}
%     \scnsuperset{решатель задач подсистемы управления проектирования компьютерных систем и их компонентов}
% \end{scnindent}
% \scnsuperset{решатель задач самостоятельной компьютерной системы}
\end{SCn}

Рассмотрим отношения необходимые для спецификации многократно используемого компонента решателя задач, его поиска и установки в дочернюю ostis-систему.

\begin{SCn}
\scnheader{отношение, специфицирующее многократно используемый компонент решателей задач ostis-систем}
\scnsubset{отношение, специфицирующее многократно используемый компонент ostis-систем}
\scnhaselement{первичное условие инициирования*}
\begin{scnindent}
    \scnexplanation{Связки отношения первичное условие инициирования* связывают между собой sc-узел, обозначающий абстрактный sc-агент и бинарную ориентированную пару, описывающую первичное условие инициирования данного абстрактного sc-агента, т.е. такой ситуации в sc-памяти, которая побуждает sc-агента перейти в активное состояние и начать проверку наличия своего полного условия инициирования.
    
    Первым компонентом данной ориентированной пары является знак некоторого подмножества понятия событие, например событие добавления выходящей sc-дуги, т.е. по сути конкретный тип события в sc-памяти.
    
    Вторым компонентом данной ориентированной пары является произвольный в общем случае sc-элемент, с которым непосредственно связан указанный тип события в sc-памяти, т.е., например, sc-элемент, из которого выходит либо в который входит генерируемая либо удаляемая sc-дуга, либо sc-ссылка, содержимое которой было изменено.
    
    После того, как в sc-памяти происходит некоторое событие, активизируются все активные sc-агенты, первичное условие инициирования* которых соответствует произошедшему событию.}
    \scnrelfrom{первый домен}{абстрактный sc-агент}
    \scnrelfrom{второй домен}{бинарная ориентированная пара}
\end{scnindent}
\scnhaselement{условие инициирования и результат*}
\begin{scnindent}
    \scnexplanation{Связки отношения условие инициирования и результат* связывают между собой sc-узел, обозначающий абстрактный sc-агент и бинарную ориентированную пару, связывающую условие инициирования данного абстрактного sc-агента и результаты выполнения данного экземпляров данного sc-агента в какой-либо конкретной системе.
    
    Указанную ориентированную пару можно рассматривать как логическую связку импликации, при этом на sc-переменные, присутствующие в обеих частях связки, неявно накладывается квантор всеобщности, на sc-переменные, присутствующие либо только в посылке, либо только в заключении неявно накладывается квантор существования.
    
    Первым компонентом указанной ориентированной пары является логическая формула, описывающая условие инициирования описываемого абстрактного sc-агента, то есть конструкции, наличие которой в sc-памяти побуждает sc-агент начать работу по изменению состояния sc-памяти. Данная логическая формула может быть как атомарной, так и неатомарной, в которой допускается использование любых связок логического языка.
    
    Вторым компонентом указанной ориентированной пары является логическая формула, описывающая возможные результаты выполнения описываемого абстрактного sc-агента, то есть описание произведенных им изменений состояния sc-памяти. Данная логическая формула может быть как атомарной, так и неатомарной, в которой допускается использование любых связок логического языка.}
    \scnrelfrom{первый домен}{абстрактный sc-агент}
    \scnrelfrom{второй домен}{бинарная ориентированная пара}
\end{scnindent}
\scnhaselement{эквивалентный компонент*}
\begin{scnindent}
    \scnhaselement{неориентированное отношение}
    \scnexplanation{Бинарное отношение связывающее функционально эквивалентные многократно используемые компоненты решателей задач.}
    \scnrelfrom{первый домен}{многократно используемый компонент решателей задач}
    \scnrelfrom{второй домен}{многократно используемый компонент решателей задач}
\end{scnindent}
\end{SCn}

% \section{Многократно используемые методы, хранимые в памяти ostis-систем и интерпретируемые их внутренними агентами}

\section{Многократно используемые внутренние агенты ostis-систем}

\begin{SCn}
\scnheader{внутренний абстрактный sc-агент}
\scnexplanation{Каждый \textbf{\textit{внутренний абстрактный sc-агент}} обозначает класс \textit{sc-агентов}, которые реагируют на события в \textit{sc-памяти} и осуществляют преобразования исключительно в рамках этой же \textit{sc-памяти}.}

% \scnheader{многократно используемый внутренний агент ostis-системы}
% \scnidtf{внутренний абстрактный sc-агент, являющийся многократно используемым компонентом ostis-системы}
\end{SCn}

%\input{author/references}
\chapter{Методика и средства компонентного проектирования интерфейсов ostis-систем}
\chapauthortoc{Садовский М.~Е.\\Жмырко А.~В.}
\label{chapter_ui_design}

\vspace{-7\baselineskip}

\begin{SCn}
\begin{scnrelfromlist}{автор}
	\scnitem{Садовский М.~Е.}
	\scnitem{Жмырко А.~В.}
\end{scnrelfromlist}
\bigskip
	
\begin{scnrelfromlist}{подраздел}
	\scnitem{\ref{sec_analysis_UI_design_methodologies}~\nameref{sec_analysis_UI_design_methodologies}}
	\scnitem{\ref{sec_reusable_UI_components}~\nameref{sec_reusable_UI_components}}
\end{scnrelfromlist}	
	
	
\scntext{аннотация}{Проектирование интерфейса – это один из наиболее важных  этапов разработки любой системы.
Пользователь при обращении с интерфейсом должен представить себе, какая информация о выполняемой задаче у него существует, и в каком состоянии находятся средства, с помощью которых он будет решать данную задачу. Эффективность работы пользователя и его интерес обеспечивает правильно сформулированная методика разработки и проектирования пользовательского интерфейса. \newline
В настоящее время организация взаимодействия пользователя с компьютерной системой лежит парадигма \textbf{грамотного пользователя},	который знает, как управлять системой и несёт полную ответственность за качество взаимодействия с ней.
Многообразие форм и видов интерфейсов приводит к необходимости пользователя  адаптироваться к каждой конкретной системе, обучаться принципам взаимодействия с ней для решения необходимых ему задач. \newline
Проектирование пользовательских интерфейсов включает в себя ряд последовательных этапов.
В рамках главы будут рассмотрены этапы проектирования традиционных пользовательских интерфейсов и этапы проектирования адаптивных интеллектуальных мультимодальных пользовательских интерфейсов.}
\end{SCn}

% ПОСМОТРЕТЬ \scncite{Koronchik2011}

\section{Анализ методик проектирования пользовательских интерфейсов}
\label{sec_analysis_UI_design_methodologies}

Среди существующих методик проектирования адаптивных интеллектуальных мультимодальных пользовательских интерфейсов можно выделить методики,
предложенные в \scncite{Ehlert2003} и  \scncite{Kong2011}.

В рамках работы \scncite{Ehlert2003} выделяется 4 основных этапа проектирования:
\begin{textitemize}
    \item анализ;
    \item разработка интерфейса;
    \item оценка интерфейса;
    \item доработка и усовершенствование.
\end{textitemize}

Этап анализа является, вероятно, самой важной фазой в любом процессе проектирования, но тем более в проектировании интерфейсов ostis-систем. В
процессе проектирования обычного неинтеллектуального интерфейса
необходимо проанализировать, кто является обычным пользователем, какие задачи интерфейс должен поддерживать. 

В пользовательском интерфейсе часто нет среднего пользователя.
В идеале, пользовательский интерфейс должен быть способен адаптироваться к любому пользователю в любой среде. Поэтому используемая техника адаптации должна быть разработана таким образом, чтобы она могла поддерживать все типы пользователей.

Этап \textit{анализа} включает выполнение четырех взаимосвязанных видов анализа:
\begin{textitemize}
    \item функциональный анализ;
    \item анализ данных;
    \item анализ пользователей;
    \item анализ среды.
\end{textitemize}

В рамках \textit{функционального анализа} необходимо дать ответ на вопрос: "каковы \uline{основные функции системы}?".
В рамках \textit{анализа данных} необходимо определить \uline{значение и структуру данных}, используемых в приложении.
В рамках \textit{анализа пользователей} необходимо выделить \uline{типы пользователей и их возможности} в интеллектуальном
и когнитивном плане.
В рамках \textit{анализа среды} необходимо определить \uline{требования, предъявляемые к среде}, в которой будет работать система.

Результатом данного этапа является \uline{cпецификация целей и информационных потребностей пользователя}, а также
\uline{спецификация функций и информации}, которые требуются системе.

\textit{Разработка интерфейса} включает следующие шаги:
\begin{textitemize}
	\item \textit{создание модели интерфейса} в соответствии с этапом анализа;
	\item реализация модели интерфейса.
\end{textitemize}

Результатом данного этапа является \uline{пользовательский интерфейс}, который, по мнению разработчика, удовлетворяет требованиям пользователей и соответствует требованиям, сформулированным на этапе анализа.

\textit{Оценка интерфейса} предполагает, что:
\begin{textitemize}
	\item требования, которые были сформулированы на этапе анализа, должны быть удовлетворены;
	\item эффективность модели интерфейса должна быть исследована.
\end{textitemize}

На этапе \textit{оценки интерфейса} необходимо вернуться к требованиям \textit{этапа анализа}. Требования, которые
были сформулированы на \textit{этапе анализа}, должны быть выполнены, а также должна быть исследована эффективность модели интерфейса.
Чтобы определить эту эффективность, необходимо определить критерии эффективности.

Очень важным, но субъективным критерием является удовлетворенность пользователя. Поскольку пользователь должен работать с интерфейсом, он имеет право голоса в вопросе о том, удобно ли работать с интерфейсом и т.п.

Критериями эффективности могут выступать различные показатели, такие как:
\begin{textitemize}
	\item количество ошибок;
	\item время выполнения задачи;
	\item отношение пользователя к интерфейсу;
	\item т.д.
\end{textitemize}

\textit{Доработка и усовершенствование} осуществляется на основе проблем, выявленых на этапе оценки. В рамках данного этапа вносится ряд улучшений в модель интерфейса. Затем начинается новый цикл проектирования. Этот итеративный процесс будет продолжаться до тех пор, пока результат оценки не будет удовлетворять обозначенным требованиям. 

Методика, предложенная в \scncite{Kong2011} включает 6 этапов:
\begin{textitemize}
	\item моделирование пользовательского интерфейса (описание абстрактного пользовательского интерфейса);
	\item проектирование пользовательского интерфейса по умолчанию (стандартная версия, конкретный пользовательский интерфейс);
	\item разработка пользовательского интерфейса (расширение или замена пользовательского интерфейса по умолчанию) - этот шаг опускается, когда система генерирует пользовательский интерфейс по умолчанию автоматически;
	\item создание контекста использования (идентификация и создание контекста использования - модели пользователя, модель устройства и модель среды/платформы);
	\item адаптация пользовательского интерфейса - автоматически - (адаптация пользовательского интерфейса во время выполнения для соответствия конкретного контекста использования);
	\item кастомизация пользовательского интерфейса - настройка пользовательского интерфейса самим пользователем (адаптируемость).
\end{textitemize}

На основе рассмотренных методик проектирования интерфейсов можно выделить следующие общие этапы:
\begin{textitemize}
\item анализ контекста использования и задач пользователей;
\item проектирование и разработка интерфейса;
\item оценка качества спроектированного интерфейса.
\end{textitemize}

Среди недостатков предложенных подходов можно выделить:
\begin{textitemize}
	\item знания по каждому этапу проектирования находятся у разных специалистов в неформализорованном неунифицированном виде;
	\item отсутствие этапа формализованного документирования этапов проектирования приводит в дальнейшем в необходимости создания отдельных help-систем для пользователей, разработчиков и т.д.
\end{textitemize}

Предлагается использовать онтологический подход на основе семантической модели в процессе проектирования и реализации адаптивного интеллектуального мультимодального пользовательского интерфейса. Такой интерфейс предлагается рассматривать как специ-
ализированную подсистему для решения интерфейсных задач пользователя, состоящую из базы знаний и решателя интерфейсных задач. 

Описание модели базы знаний и решателя предлагается осуществлять на основе универсального унифицированного языка представления знаний, что обеспечит совместимость между этими компонентами.

Архитектура интерфейса такой системы была рассмотрена на рисунке "Архитектура интеллектуального интерфейса"{}.

\textbf{Архитектура интеллектуального интерфейса}

\begin{figure}[h]
	\centering
	\includegraphics[scale=0.15]{images/part5/sc-model-ui.png}
\end{figure}

Таким образом, предлагаемая методика проектирования интерфейсов ostis-систем будет включать:
\begin{textitemize}
\item анализ пользователя, его задач и целей, а также контекста использования;
\item анализ требований к пользовательскому интерфейсу;
\item проектирование пользовательского интерфейса по умолчанию;
\item разработка пользовательского интерфейса;
\item анализ пользовательского интерфейса и его адаптации.
\end{textitemize}

Поскольку знания о конкретном этапе обычно находятся у разных экспертов, особенностью предлагаемого подхода является обязательное формализованное документирование знаний в унифицированном виде и применение на каждом из этапов компонентного подхода.

Для применения компонентного подхода предлагается использовать \textit{библиотеку многократно используемых компонентов} базы знаний, решателя задач и интерфейса.

\textbf{Анализ пользователя, его задач и целей, а также контекста использования}

Результаты первого этапа, такие как: модель конкретного пользователя, его потребности и контекст использования системы (устройство, окружающая среда) должны быть формализованы в рамках соответствующих онтологий базы знаний интерфейса. 
При этом в процессе формализации по необходимости должны быть переиспользованы компоненты базы знаний из библиотеки многократно используемых компонентов, а новые компоненты могут пополнить эту же библиотеку.

\textbf{Анализ требований к пользовательскому интерфейсу}

Результатом второго этапа являются конечные требования к интерфейсу, которые должны быть сформулированы относительно модели пользователя и его цели, а также относительно контекста использования.

Результаты должны быть также формализованы, а в процессе выполнения могут быть использованы существующие компонент базы знаний из библиотеки многократно используемых компонентов.

\textbf{Проектирование пользовательского интерфейса по умолчанию}

В соответствии с требованиями к пользовательскому интерфейсу, строится модель адаптивного интеллектуального мультимодального пользовательского интерфейса, которая является результатом третьего этапа.

Такая модель будет включать в себя формализованную модель базы знаний и решателя задач.

При проектировании могут быть использованы компоненты интерфейса, базы знаний и решателя задач. 
Компоненты будут записаны в унифицированном виде, что позволит обеспечить их автоматическую совместимость.

\textbf{Разработка пользовательского интерфейса}

Результатом четвертого этапа является реализация спроектированного пользовательского интерфейса. В данном случае следует использовать готовые компоненты интерфейса из библиотеки многократно используемых компонентов интерфейса.

\textbf{Анализ пользовательского интерфейса и его адаптации}

На данном этапе используются готовые компоненты решателя задач.

Таким образом будет сформирована база знаний проектируемого интерфейса, которая автоматически может быть использована в качестве help-системы для пользователей, разработчиков и т.д.

\section{Многократно используемые компоненты интерфейсов ostis-систем}
\label{sec_reusable_UI_components}

%\input{author/references}

%
\begin{partbacktext}
\part{Платформы реализации интеллектуальных компьютерных систем нового поколения}
\noindent Описание к главе
\end{partbacktext}
%
\begin{partbacktext}
\part{Экосистема интеллектуальных компьютерных систем нового поколения и их пользователей}
\noindent Описание к главе
\end{partbacktext}

%\backmatter%%%%%%%%%%%%%%%%%%%%%%%%%%%%%%%%%%%%%%%%%%%%%%%%%%%%%%%

\Extrachap{Предметный указатель OSTIS}

Предлагаемый вашему вниманию предметный указатель представляет собой алфавитный перечень всех основных терминов, используемых в данной монографии, которые \underline{взаимно-однозначно} соответствуют элементам рафинированной семантической сети, представляющей собой базу знаний \textit{Метасистемы OSTIS}, основная часть которой семантически эквивалентна тексту данной монографии.

В рамках внутреннего языка \textit{ostis-систем} (в рамках \textit{SC-кода}) указанные основные термины называются \textit{основными sc-идентификаторами} (основными внешними идентификаторами sc-элементов -- элементов рафинированных семантических сетей).

В данный предметный указатель включаются как русскоязычные, так и англоязычные термины (если нет аналогичного русскоязычного), непереводимые интернациональные названия (например, названия различных программных систем, такие как \textit{Neo4j}, \textit{MySQL} и т.д.), а также различные используемые сокращения.

При этом для каждого \underline{основного} русскоязычного термина указывается синонимичный ему \underline{основной} англоязычный термин, например:

\bigskip

\begin{SCn}

\begin{scnset}
\vspace{-\baselineskip}
\scnheader{sc-элемент}
\scnidtf{\textit{sc-element}}
\begin{scnindent}
	\scniselement{основной англоязычный sc-идентификатор}
	\begin{scnindent}
		\scnidtf{основной sc-идентификатор для англоязычного режима}
	\end{scnindent}
\end{scnindent}
\scnidtf{\textit{sc-элемент}}
\begin{scnindent}
	\scniselement{основной русскоязычный sc-идентификатор}
	\begin{scnindent}
		\scnidtf{основной sc-идентификатор для русскоязычного режима}
	\end{scnindent}
\end{scnindent}
\end{scnset}
\scnexplanation{Для каждого ключевого термина в предметном указателе указывается его основной русскоязычный идентификатор и основной англоязычный идентификатор. При этом по умолчанию считается, что русскоязычный термин является \underline{основным} русскоязычным идентификатором, а соответствующий ему англоязычный -- \underline{основным} англоязычным идентификатором.}

\end{SCn}

\bigskip

Для каждого неосновного, но часто используемого русскоязычного термина указывается синонимичный ему основной русскоязычный термин. Кроме того, для каждого основного русскоязычного термина указывается ссылка на соответствующую главу, параграф или пункт монографии, где сущность, обозначаемая этим термином, является ключевым знаком.

\bigskip

\begin{SCn}

\scnheader{\scnfilelong{\textbf{\textit{SC-код}}}}
\scnidtf{\textit{SC-code}}
\scnrelto{часто используемый sc-идентификатор}{sc-структура}
\begin{scnreltolist}{ключевой знак}
	\scnitem{\nameref{chapter_new_generation_systems}}
\end{scnreltolist}

\scnheader{sc-идентификатор}
\scnidtf{\textit{sc-identifier}}
\begin{scnreltolist}{ключевой знак}
	\scnitem{\nameref{chapter_ext_lang}}
\end{scnreltolist}	

\bigskip
\bigskip
\bigskip
	
\scnheader{sc-элемент}
\scnidtf{\textit{sc-element}}
\begin{scnreltolist}{ключевой знак}
	\scnitem{\nameref{chapter_new_generation_systems}}
\end{scnreltolist}

\end{SCn}	
\Extrachap{Библиография}

Работа \scncite{Golenkov2019} посвящена основным принципам Технологии OSTIS.

Работа \scncite{Wooldridge2009} описывает основные принципы многоагентных систем.

\begin{SCn}
	
\scnciteheader{Golenkov2019}
\scnfullcite{Golenkov2019}
\scnciteannotation{Golenkov2019}
\scnrelfrom{автор}{В.В. Голенков}

\scnciteheader{Wooldridge2009}
\scnfullcite{Wooldridge2009}
\scnciteannotation{Wooldridge2009}
	
\end{SCn}


%%%%%%%%%%%%%%%%%%%%%%%%%%%%%%%%%%%%%%%%%%%%%%%%%%%%%%%%%%%%%%%%%%%%%%

\end{document}





