%TODO Раздел для уточнения
\newpage

Текущее описание с указанием текущих имен программных компонентов:

\begin{SCn}
\scnheader{Программный вариант реализации ostis-платформы}
\begin{scnrelfromset}{декомпозиция программной системы}
	\scnitem{Реализация sc-памяти}
	\begin{scnindent}
		\scnidtf{sc-machine}
	\end{scnindent}
	\scnitem{Реализация scp-интерпретатора}
	\scnitem{Реализация интерпретатора sc-моделей пользовательских интерфейсов}
	\begin{scnindent}
		\scnidtf{sc-web}
	\end{scnindent}
	\scnitem{Реализация базового набора платформенно-зависимых sc-агентов и их общих компонентов}
	\begin{scnindent}
		\scnidtf{sc-kpm}
	\end{scnindent}
\end{scnrelfromset}
\scnrelfrom{архитектура}{\scnfileimage{author/part6/images/platform\_arch.pdf}}
\end{SCn}
	
\textit{Реализация базового набора платформенно-зависимых sc-агентов и их общих компонентов} представляет собой реализацию множества базовых sc-агентов (поиск сем. окрестности, по выходящим, входящим и т.д.), реализованных на уровне платформы.
	
\begin{SCn}
\scnheader{Реализация sc-памяти}
\scnidtf{sc-machine}
\scnidtf{Реализация sc-машины}
\scnrelto{программная модель}{sc-память}
\scniselement{программная модель sc-памяти на основе линейной памяти}
\begin{scnrelfromlist}{компонент программной системы}
	\scnitem{Реализация sc-хранилища}
	\begin{scnindent}
		\scnidtf{sc-memory}
	\end{scnindent}		
	\scnitem{Реализация файлового хранилища}	
	\scnitem{Реализация подсистемы взаимодействия с внешней средой с использованием сетевых протоколов}
	\begin{scnindent}
		\scnidtf{sc-server}
	\end{scnindent}		
	\scnitem{Реализация вспомогательных инструментальных средств для работы с sc-памятью}
	\begin{scnindent}
		\scnidtf{sc-builder}
	\end{scnindent}		
\end{scnrelfromlist}
\end{SCn}

\textit{Реализация sc-хранилища} обеспечивает
\begin{itemize}
	\item хранение sc-конструкций
	\item доступ к ним через программный интерфейс (функции поиска, генерации и удаления конструкций на уровне языка платформы)
	\item возможность реализации sc-агентов на уровне платформы (возможность при возникновении определенных событий запустить определенные процедуры, реализованные на языке платформы)
	\item сохранение состояния памяти во внешний файл (дамп) и загрузку из него
\end{itemize}

\textit{Реализация файлового хранилища} обеспечивает хранение внешних файлов.

\textit{Реализация подсистемы взаимодействия с внешней средой с использованием сетевых протоколов} по сути является отдельным приложением, которое использует \textit{Реализацию sc-памяти} как библиотеку, и после запуска ожидает соединения по сети по определенному порту, принимает запросы и отвечает на них согласно сетевому протоколу.

\textit{Реализация вспомогательных инструментальных средств для работы с sc-памятью} на данный момент включает сборщик базы знаний из исходных текстов в sc-память. Сборщик это отдельное приложение, которое тоже используется sc-память как библиотеку, транслирует туда базу знаний из исходников (через программный интерфейс), затем инициирует выгрузку состояния памяти на файловую систему (в дамп) и завершает работу. Затем при старте \textit{sc-server} (\textit{Реализация подсистемы взаимодействия с внешней средой с использованием сетевых протоколов}) состояние памяти из дампа загружается в sc-память и система работает уже с обновленной базой знаний.

\textbf{\uline{Комментарий:}}
Таким образом \textit{Реализация sc-памяти} (или sc-хранилища, смотря как назовем) представляет собой программную библиотеку, а не самостоятельное приложение, это плохо. То есть сама по себе sc-память не может работать, только как часть какого-то приложения.

\textit{Реализация интерпретатора sc-моделей пользовательских интерфейсов} (\textit{sc-web}) представляет собой отдельное приложение, реализованное в виде web-сервера, которое, с одной стороны, взаимодействует с \textit{sc-server} через сетевой протокол, с другой стороны, общается с пользователем через web-браузер (не важно, через интернет с локального компьютера). То есть, при попытке открыть систему в браузере sc-web отображает часть интерфейса, реализованную современными средствами (вообще без привязки к базе знаний), а за содержимым некоторых окон (список команд меню, собственно содержимое окна с SCn) обращается к sc-server, откуда получает нужные знания, записанные в виде сетевого протокола и отображает их в SCn, SCg или как-то по-другому (как команды меню например).
При взаимодействии пользователя с каким-то компонентом интерфейса sc-web формирует нужный запрос к sc-server, ожидает ответа и отображает результат пользователю.

\uline{\textbf{Вопросы:}}
\begin{itemize}
\item В чем разница sc-памяти, sc-хранилища, sc-машины?
\item Что входит в sc-память?
\item Считать ли sc-server обязательным компонентом sc-памяти? ostis-платформы? Без него система не может общаться с внешним миром пока что.
\item Что в текущей реализации считать частью платформы (не памяти, а вообще платформы)?
\item Что в общем случае должно быть частью платформы (с учетом возможной аппаратной и программной реализации)?
\end{itemize}

\newpage
%%%%%%%%%%%%%%%%%%%%%%%%%%%%%%%%%%%%%%%%%%%%%%%%%%%%