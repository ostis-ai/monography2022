\chapauthor{Голенков В.В.\\Шункевич Д.В.\\Гулякина Н.А.\\Захарьев В.А.}
\chapter{Ассоциативные семантические компьютеры для ostis-систем}
\chapauthortoc{Голенков В.В.\\Шункевич Д.В.\\Гулякина Н.А.\\Захарьев В.А.}
\label{chapter_computers}

\section{Современное состояние в области разработки компьютеров для интеллектуальных систем}
\section{Общие принципы, лежащие в основе ассоциативных семантических компьютеров для ostis-систем}
\section{Архитектура ассоциативных семантических компьютеров для ostis-систем}


\abstract{В главе рассмотрены принципы реализации аппаратной платформы для реализации систем, построенных на основе Технологии OSTIS, -- ассоциативного семантического компьютера.}

Применение для разработки \textit{ostis-систем} современных программно-аппаратных платформ, ориентированных на адресный доступ к хранящимся в памяти данным, не всегда оказывается эффективным, поскольку при разработке интеллектуальных систем фактически приходится моделировать нелинейную память на базе линейной. Повышение эффективности решения задач интеллектуальными системами требует разработки специализированных платформ, в том числе аппаратных, ориентированных на унифицированные семантические модели представления и обработки информации. Таким образом, основной целью создания \textit{ассоциативных семантических компьютеров} является повышение производительности ostis-систем.

Рассмотрим более детально особенности логической организации традиционной архитектуры компьютерных систем, существенно затрудняющие эффективную реализацию \textit{ostis-систем} на ее основе:
\begin{itemize}
	\item низкий уровень доступа к памяти, т.е. сложность и громоздкость выполнения процедуры ассоциативного поиска нужного фрагмента знаний; 
	\item линейная организация памяти и чрезвычайно простой вид конструктивных объектов, непосредственно хранимых в памяти. Это приводит к тому, что в интеллектуальных системах, построенных на базе современных компьютеров, манипулирование знаниями осуществляется с большим трудом. Во-первых, приходится оперировать не самими структурами, а их громоздкими линейными представлениями (списками, матрицами смежности, матрицами инцидентности); во-вторых, линеаризация сложных структур разрушает локальность их преобразований;
	\item представление информации в памяти современных компьютеров имеет уровень весьма далекий от смыслового, что делает переработку знаний довольно громоздкой, требующей учета большого количества деталей, касающейся не смысла перерабатываемой информации, а способа ее представления в памяти;
	\item в современных компьютерах имеет место весьма низкий уровень аппаратно реализуемых операций над нечисловыми данными и полностью отсутствует аппаратная поддержка логических операций над фрагментами знаний, имеющих сложную структуру, что делает манипулирование такими фрагментами весьма сложным.
\end{itemize}

Перечисленные особенности, по существу, не устраняются также и в развиваемых в настоящее время подходах к построению нетрадиционных высокопроизводительных компьютеров (например, компьютеров, предназначенных для обучения и интерпретации искусственных нейронных сетей \cite{Neurocomputers,USB_Accelerator}), ибо, в основном, все эти подходы (даже если они достаточно далеко отходят от предложенных фон Нейманом базовых принципов организации вычислительных машин) неявно сохраняют точку зрения на компьютер как на большой арифмометр и тем самым сохраняют ее ориентацию на задачи числового характера.

Существует ряд статей \cite{Tran2018,Shi2018,Lu2021,Afanasyev2021,Zhang2017,Hu2021,Minati2019,Song2016} и патентов \cite{Somsubhra_2006,Allen_1989,Moussa_2013}, направленных на разработку аппаратных архитектур, предназначенных для обработки информации, представленной в более сложных формах, чем в традиционных архитектурах, однако они не получили широкого распространения и применения, по причине, во-первых, частности предлагаемых решений, и во-вторых из-за отсутствия общего универсального и унифицированного языка кодирования любой информации, в роли которого в рамках Технологии OSTIS выступает SC-код.

\textit{SC-код}, являющийся формальной основой \textit{Технологии OSTIS} изначально разрабатывался как язык кодирования информации в памяти \textit{ассоциативных семантических компьютеров}, таким образом в нем изначально заложены такие принципы, как универсальность (возможность представить знания любого рода) и унификация (единообразие) представления, а также минимизация \textit{Алфавита SC-кода}, которая, в свою очередь, позволяет облегчить создание аппаратной платформы, позволяющей хранить и обрабатывать тексты \textit{SC-кода}.

\begin{SCn}
	\scnheader{ассоциативный семантический компьютер}
	\scnidtf{аппаратно реализованный интерпретатор семантических моделей (sc-моделей) компьютерных систем}
	\scnidtf{семантический ассоциативный компьютер, управляемый знаниями}
	\scnidtf{компьютер с нелинейной структурно перестраиваемой (графодинамической) ассоциативной памятью, переработка информации в которой сводится не к изменению состояния элементов памяти, а к изменению конфигурации связей между ними}
	\scnidtf{sc-компьютер}
	\scnidtf{scp-компьютер}
	\scnidtf{универсальный компьютер нового поколения, специально предназначенный для реализации семантически совместимых гибридных интеллектуальных компьютерных систем}
	\scnidtf{универсальный компьютер нового поколения, ориентированный на аппаратную интерпретацию логико-семантических моделей интеллектуальных компьютерных систем}
	\scnidtf{универсальный компьютер нового поколения, ориентированный на аппаратную интерпретацию ostis-систем}
	\scnidtf{ostis-компьютер}
	\scnidtf{компьютер для реализации ostis-систем}
	\scnidtf{компьютер, управляемый знаниями, представленными в SC-коде}
	\scnidtf{компьютер, ориентированный на обработку текстов SC-кода}
\end{SCn}

Рассмотрим принципы, лежащие в основе реализации \textit{ассоциативных семантических компьютеров}:
\begin{itemize}
	\item нелинейная память -- каждый элементарный фрагмент хранимого в памяти текста может быть инцидентен неограниченному числу других элементарных фрагментов этого текста. Таким образом, ячейки памяти, в отличие от обычной памяти, связываются не фиксированными условными связями, задающими фиксированную последовательность (порядок) ячеек в памяти, a реально (физически) проводимыми связями произвольной конфигурации. Эти связи соответствуют дугам, ребрам, гиперребрам записанного в памяти графа (sc-текста);
	\item структурно-перестраиваемая (реконфигурируемая) память -- процесс отработки хранимой в памяти информации сводится не только к изменению состояния элементов, но и к реконфигурации связей между ними. То есть, в ходе переработки информации в структурно-перестраиваемой памяти меняются на только и даже не столько состояния ячеек памяти, как это имеет место в обычной памяти, сколько конфигурация связей между этими ячейками. Т.е. в структурно-перестраиваемой памяти в ходе переработки информации не только перераспределяются метки на вершинах записанного в памяти графа, но и меняется структура самого этого графа;	
	\item в качестве внутреннего способа кодирования знаний, хранимых в памяти \textit{ассоциативного семантического компьютера}, используется универсальный (!) способ нелинейного (графоподобного) смыслового представления знаний -- SC-код;
	\item обработка информации осуществляется коллективом агентов, работающих над общей памятью. Каждый из них реагирует на соответствующую ему ситуацию или событие в памяти (компьютер, управляемый хранимыми знаниями);
	\item есть программно реализуемые агенты, поведение которых описывается хранимыми в памяти агентно-ориентированными программами, которые интерпретируются соответствующими коллективами агентов;
	\item есть базовые агенты, которые не могут быть реализованы программно (в частности, это агенты интерпретации агентных программ, базовые рецепторные агенты-датчики, базовые эффекторные агенты);
	\item все агенты работают над общей памятью одновременно. Более того, если для какого-либо агента в некоторый момент времени в различных частях памяти возникает сразу несколько условий его применения, разные информационные процессы, соответствующие указанному агенту в разных частях памяти могут выполняться одновременно;
	\item для того, чтобы информационные процессы агентов, параллельно выполняемые в общей памяти не "мешали"{} друг другу, для каждого информационного процесса в памяти фиксируется и постоянно актуализируется его текущее состояние. То есть каждый информационный процесс сообщает всем остальным о своих намерениях и пожеланиях, которым остальные информационные процессы не должны препятствовать. Реализация такого подхода может выполняться, например, на основе механизма блокировок элементов семантической памяти, рассмотренного в главе \nameref{chapter_situation_management};
	\item процессор и память \textit{ассоциативного семантического компьютера} глубоко интегрированы и составляют единую процессоро-память. Процессор \textit{ассоциативного семантического компьютера} равномерно "распределен"{} по его памяти так, что процессорные элементы одновременно являются и элементами памяти компьютера. То есть каждая ячейка дополняется функциональным (процессорным) элементом, a перестраиваемые связи между ячейками становятся коммутируемыми каналами связи между функциональными элементами. Каждый функциональный элемент при этом имеет свою специальную внутреннюю регистровую память, отражающую важные для данного функционального элемента аспекты текущего состояния процесса выполнения элементарных операций внутреннего языка.
	
	Обработка информации в \textit{ассоциативном семантическом компьютере} сводится к реконфигурации каналов связи между процессорными элементами,  следовательно память такого компьютера есть не что иное, как \uline{коммутатор} (!) указанных каналов связи. Таким образом, текущее состояние конфигурации этих каналов связи и есть текущее состояние обрабатываемой информации. Этот принцип обеспечивает значительное ускорение переработки информации благодаря исключению этапов передачи информации из памяти в процессор и обратно, но оплачивается ценой большой избыточности функциональных (процессорных) средств, равномерно распределяемых по памяти.
	
\end{itemize}