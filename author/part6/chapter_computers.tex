\chapter{Ассоциативные семантические компьютеры для ostis-систем}
\chapauthortoc{Голенков В.В.\\Шункевич Д.В.\\Гулякина Н.А.\\Захарьев В.А.}
\label{chapter_computers}

\vspace{-7\baselineskip}

\begin{SCn}
\begin{scnrelfromlist}{автор}
	\scnitem{Голенков В.В.}
	\scnitem{Шункевич Д.В.}
	\scnitem{Гулякина Н.А.}	
	\scnitem{Захарьев В.А.}	
\end{scnrelfromlist}

\bigskip

\scntext{аннотация}{В главе рассмотрены принципы реализации аппаратной платформы для реализации систем, построенных на основе Технологии OSTIS, -- ассоциативного семантического компьютера.}

\bigskip

\begin{scnrelfromlist}{подраздел}
	\scnitem{***}
\end{scnrelfromlist}

\bigskip

\begin{scnrelfromlist}{ключевое понятие}
	\scnitem{***}
\end{scnrelfromlist}

\bigskip

\begin{scnrelfromlist}{библиографическая ссылка}
	\scnitem{***}
\end{scnrelfromlist}

\end{SCn}

\section*{Введение в Главу \ref{chapter_computers}}

Применение для разработки \textit{ostis-систем} современных программно-аппаратных платформ, ориентированных на адресный доступ к хранящимся в памяти данным, не всегда оказывается эффективным, поскольку при разработке интеллектуальных систем фактически приходится моделировать нелинейную память на базе линейной. Повышение эффективности решения задач интеллектуальными системами требует разработки специализированных платформ, в том числе аппаратных, ориентированных на унифицированные семантические модели представления и обработки информации. Таким образом, основной целью создания \textit{ассоциативных семантических компьютеров} является повышение производительности ostis-систем.

\section{Современное состояние в области разработки компьютеров для интеллектуальных систем}

Подавляющее большинство современных программно-аппаратных платформ, применяемых при разработке современных компьютерных систем, и, в частности, интеллектуальных компьютерных систем, основаны на принципах \textit{Абстрактной машины фон-Неймана} [ссылка]. Рассмотрим основные принципы, лежащие в основе \textit{Абстрактной машины фон-Неймана}.

%TODO Принципы машины фон-Неймана
%TODO Добавить картинку?
%TODO Добавить материал из литературы и ссылки
\begin{SCn}
	\scnheader{машина фон-Неймана}
	\scnsubset{абстрактная машина обработки информации}
	\begin{scnrelfromvector}{принципы, лежащие в основе}
		\scnfileitem{Информация в памяти представляется в виде последовательности строк символов.}
		\scnfileitem{Память машины представляет собой последовательность \underline{адресуемых} ячеек памяти.}
		\scnfileitem{В каждую ячейку может быть записана любая строка символов в бинарном алфавите (``0'' или ``1''). При этом длина строк для всех адресуемых ячеек одинакова (в текущем стандарте ячеек, называемых байтами, равна 8 бит).}
		\scnfileitem{Каждой ячейке памяти взаимно однозначно соответствует битовая строка, обозначающая эту ячейку и являющаяся ее адресом.}
		\scnfileitem{каждому типу элементарных действий (операций), выполняемых в памяти машины фон-Неймана, взаимно однозначно ставится ее идентификатор, который в памяти представляется в виде битовой строки}
		\scnfileitem{каждая конкретная операция (команда), выполняемая в памяти, представляется (специфицируется) в памяти в виде строки, состоящей
			\begin{scnitemize}
				\item из кода соответствущего типа операции;
				\item из последовательности адресов фрагментов памяти, в которых находятся операнды, над которыми выполняются операции -- исходные аргументы и результаты. Любой такой фрагмент задается адресом первого байта и количеством байт. Количество операндов \underline{однозначно} задается кодом типа операции;
		\end{scnitemize}}
		\scnfileitem{Программа, выполняемая в памяти, хранится в памяти в виде последовательности спецификаций конкретных операций (команд).}
	\end{scnrelfromvector}
\end{SCn}

%TODO Сослаться на Кузмицкого обязательно, там анализ большой

Рассмотрим более детально особенности логической организации традиционной (фон-Неймановской) архитектуры компьютерных систем, существенно затрудняющие эффективную реализацию \textit{ostis-систем} на ее основе:
\begin{textitemize}
	\item последовательная обработка, ограничивающая эффективность компьютеров физическими возможностями элементной базы;
	\item низкий уровень доступа к памяти, т.е. сложность и громоздкость выполнения процедуры ассоциативного поиска нужного фрагмента знаний; 
	\item линейная организация памяти и чрезвычайно простой вид конструктивных объектов, непосредственно хранимых в памяти. Это приводит к тому, что в интеллектуальных системах, построенных на базе современных компьютеров, манипулирование знаниями осуществляется с большим трудом. Во-первых, приходится оперировать не самими структурами, а их громоздкими линейными представлениями (списками, матрицами смежности, матрицами инцидентности); во-вторых, линеаризация сложных структур разрушает локальность их преобразований;
	\item представление информации в памяти современных компьютеров имеет уровень весьма далекий от смыслового, что делает переработку знаний довольно громоздкой, требующей учета большого количества деталей, касающейся не смысла перерабатываемой информации, а способа ее представления в памяти;
	\item в современных компьютерах имеет место весьма низкий уровень аппаратно реализуемых операций над нечисловыми данными и полностью отсутствует аппаратная поддержка логических операций над фрагментами знаний, имеющих сложную структуру, что делает манипулирование такими фрагментами весьма сложным.
\end{textitemize}

Попытки преодоления ограничений традиционных фон-Неймановских ЭВМ привели к возникновению множества подходов, связанных с отдельными изменениями принципов логической организации ЭВМ, прежде всего в зависимости от классов задач и предметных областей, на которые ориентируется тот или иной класс ЭВМ. Все эти тенденции, рассмотренные в совокупности, позволяют очертить некоторые ключевые принципы логической организации ЭВМ, ориентированных на переработку знаний (машин переработки знаний -- МПЗ). Перечислим основные из этих тенденций:
%TODO Ссылки взять у Кузмицкого
\begin{textitemize}
\item Переход к нелинейной организации памяти [41, 19] и аппаратная интерпретация сложных структур данных [139, 136,67].
\item Аппаратная реализация ассоциативного доступа к информации [89, 97, 41, 79, 65, 26, 18].
\item Реализация параллельных асинхронных процессов над памятью [41, 88] и, в частности, разработка вычислительных машин, управляемых потоком данных [101, 63, 132, 87].
\item Аппаратная интерпретация языков высокого уровня [43, 42, 104].
\item Разработка аппаратных средств ведения баз данных (процессоров баз данных) [61, 150, 113].
\end{textitemize}
	
На пересечении этих тенденций в разное время возникали различные классы вычислительных устройств. Перечислим некоторые из них:
%TODO Ссылки взять у Кузмицкого
\begin{textitemize}
\item машины с аппаратной интерпретацией сложных структур данных [134, 136, 135];
\item машины с развитой ассоциативной памятью [79, 80, 123];
\item ассоциативные параллельные матричные процессоры [20];
\item однородные параллельные структуры для решения комбинаторно-логических задач на графах и гиперграфах [27];
\item всевозможные устройства переработки графов [35], [107], [ссылка на свежую статью про графовый компьютер], в частности, на основе FPGA [Zhang2017,Hu2021, Song2016] и векторных процессоров [Afanasyev2021];
\item системы, осуществляющие переработку информации непосредственно в памяти путем равномерного распределения функциональных средств по памяти и, в частности, предложенная М.Н.Вайнцвайгом процессоро-память, ориентированная на решение задач искусственного интеллекта [32, 33];
\item машины, управляемые потоком данных [20, 101, 87], и, в частности, процессоры, реконфигурируемые с учетом семантики входного потока данных \scncite{Somsubhra_2006};
\item рекурсивные вычислительные машины [41];
\item процессоры реляционных баз данных [61, 87, 97];
\item вычислительные машины со структурно-перестраиваемой памятью [103, 105, 40, 39];
\item активные семантические сети (М-сети) [17];
\item ассоциативные однородные среды [64];
\item нейроподобные структуры [92, 38, 125].
\item машины для интерпретации логических правил \scncite{Allen_1989}.
\end{textitemize}

В последние годы активное развитие теории искусственных нейронных сетей привело к развитию различных подходов к построению высокопроизводительных компьютеров, предназначенных для обучения и интерпретации искусственных нейронных сетей \cite{Neurocomputers,USB_Accelerator,Moussa_2013} и их внедрению в различные программно-аппаратные комплексы. 

Кроме того, развитие графических процессоров (graphics processing unit, GPU) привело к возможности организации параллельных вычислений непосредственно на GPU, для чего разрабатываются специализированные программно-аппаратные архитектуры, например CUDA [CUDA] и OpenCL [OpenCL]. Преимуществом GPU в данном случае выступает наличие в рамках одного GPU большого (по сравнению с центральным процессором) числа ядер, что позволяет эффективно решать на такой архитектуре задачи, обладающие естественным параллелизмом (например, операции с  матрицами). Развиваются также работы, посвященные принципам обработки графовых структур на GPU \cite{Tran2018,Shi2018,Lu2021}. 

В то же время, большинство перечисленных подходов (даже если они достаточно далеко отходят от предложенных фон Нейманом базовых принципов организации вычислительных машин) неявно сохраняют точку зрения на компьютер как на большой арифмометр и тем самым сохраняют ее ориентацию на задачи числового характера. Работы же направленные на разработку аппаратных архитектур, предназначенных для обработки информации, представленной в более сложных формах, чем в традиционных архитектурах не получили широкого распространения и применения, 
по причине, во-первых, частности предлагаемых решений, и во-вторых из-за отсутствия общего универсального и унифицированного языка кодирования любой информации, в роли которого в рамках \textit{Технологии OSTIS} выступает \textit{SC-код}.

\textit{SC-код}, являющийся формальной основой \textit{Технологии OSTIS} изначально разрабатывался как язык кодирования информации в памяти \textit{ассоциативных семантических компьютеров}, таким образом в нем изначально заложены такие принципы, как универсальность (возможность представить знания любого рода) и унификация (единообразие) представления, а также минимизация \textit{Алфавита SC-кода}, которая, в свою очередь, позволяет облегчить создание аппаратной платформы, позволяющей хранить и обрабатывать тексты \textit{SC-кода}.

Основная методологическая особенность предлагаемого подхода к разработке средств аппаратной реализации поддержки интеллектуальных систем заключается в том, что такие средства должны разрабатываться не до, а \underline{после того, как} основные положения соответствующей \underline{технологии} проектирования и эксплуатации интеллектуальных систем будут апробированы на современных технических средствах. Более того, в рамках \textit{\textit{Технологии OSTIS}} четко продумана методика перехода на новые аппаратные средства, которая затрагивает только самый нижний уровень технологии -- уровень реализации базовой машины обработки семантических сетей (интерпретатора \textit{Языка SCP}).

Проект \textit{ассоциативного семантического компьютера} имеет давнюю историю, основными этапами которой являются:
\begin{textitemize}
\item 1984 год -- в Московском  институте электронной техники В.В. Голенковым защищена кандидатского диссертация на тему ``Структурная организация и переработка информации в электронных математических машинах, управляемых потоком сложноструктурированных данных'', в которой были сформулированы и рассмотрены основные принципы семантических ассоциативных компьютеров [ссылка];
\item 1993 год -- комиссия Госкомпрома провела успешные испытания прототипа \textit{семантического ассоциативного компьютера}, разработанного на базе транспьютеров в рамках научно-исследовательского проекта ``Параллельная графовая вычислительная система, ориентированная на решение задач искусственного интеллекта'' [ссылка на препринты];
\item 1996 год -- В.В. Голенковым защищена докторская диссертация на тему ``Графодинамические модели и методы параллельной асинхронной переработки информации в интеллектуальных системах'' [ссылка]. 
\item 2000 год -- в Институте проблем управления РАН П.А. Гапоновым защищена кандидатская диссертация на тему ``Модели и методы параллельной асинхронной переработки информации в графодинамической ассоциативной памяти'' [ссылка];
\item 2000 год -- в Институте программных систем РАН В.М. Кузьмицким защищена кандидатская диссертация на тему ``Принципы построения графодинамического параллельного компьютера, ориентированного на решение задач искусственного интеллекта'' [ссылка];
\item 2004 год -- в Белорусском государственном университете информатики и радиоэлектроники Р.Е. Сердюковым защищена кандидатская диссертация на тему ``Базовые алгоритмы и инструментальные средства обработки информации в графодинамических ассоциативных машинах'', в которой было рассмотрено базовое программное обеспечение семантических ассоциативных компьютеров [ссылка].
\end{textitemize}

В тоже время, несмотря на наличие действующего прототипа \textit{семантического ассоциативного компьютера} на базе транспьютеров, основное внимание в рамках соответствующего проекта и других перечисленных работ уделялось принципам организации распределенной параллельной обработки конструкций SC-кода, в частности, был разработан язык SCD (Semantic Code Distributed) для распределенного хранения конструкций SC-кода и язык SCPD для распределенной параллельной их обработки. Однако, общие принципы хранения информации и общая архитектура каждого из процессорных элементов (транспьютеров) оставались фон-Неймановскими. В частности, для кодирования конструкций SC-кода в традиционной адресной памяти были разработаны соответствующие структуры данных, близкие к тем, что описаны в \textit{\ref{sec_soft_platform_sc_storage}~\nameref{sec_soft_platform_sc_storage}}.

Таким образом, можно сказать, что обоснованность и необходимость разработки \textit{семантического ассоциативного компьютера}, а также компетентность авторов в данной области подтверждается более чем 30-летним опытом работы и рядом успешных проектов в данном направлении, однако, в тоже время, в предшествующих работах в полной мере не устранены все недостатки фон-Неймановской архитектуры, рассмотренные выше, и разработка и реализация проекта \textit{семантического ассоциативного компьютера}, устраняющего перечисленные недостатки, остается актуальной.

\section{Общие принципы, лежащие в основе ассоциативных семантических компьютеров для ostis-систем}

В основу предлагаемого подхода к разработке \textit{семантического ассоциативного компьютера} положены идеи, предложенные в работах В.В. Голенкова [ссылка] и получившие развитие в работе В.М. Кузьмицкого [ссылка].

При формализации предметных областей, имеющих достаточно сложную семантическую организацию, перерабатываемые данные естественным образом группируются в некоторые сложные структуры. Эффективность решения задач, связанных с переработкой сложноструктурированных данных, на многопроцессорных вычислительных системах значительно возрастает в том случае, когда структура связей между процессорными элементами вычислительной системы, решающей эту задачу, совпадает со структурой данных, перерабатываемых в ходе её решения (или, в более общем случае -- отображается в структуру перерабатываемых данных простым и естественным образом). При переходе к переработке данных все более сложной структурной и семантической организации (а затем и к переработке знаний) сохранение высокой эффективности вычислительной системы обеспечивается главным образом путем увеличения числа одновременно работающих процессорных элементов и усложнения структуры связей между ними [Кузьмицкий].

Такую тенденцию развития технических средств ЭВМ мы и рассмотрим в качестве основной линии эволюции, создающей предпосылки ддя появления \textit{ассоциативных семантических компьютеров}. К ней относятся параллельные регулярные спецпроцессоры (векторные, матричные), спецвычислители для решения задач на графах и средства аппаратной поддержки семантических и нейронных сетей. К этой линии примыкают также и ассоциативные процессоры (в которых в роли процессорных элементов выступают ячейки ассоциативной памяти), процессоры баз данных и вычислительные системы, эффективно решающие те или иные классы задач за счет совпадения структуры связей между процессорными элементами со структурой информационного графа алгоритма (систолические вычислители, машины потоков данных) [Кузьмицкий].

Закономерным результатом развития вычислительных систем является переход к системам, изменяющим структуру связей между процессорными элементами в процессе функционирования. Такие системы настраивают свою внутреннюю структуру на структуру перерабатываемых данных и информационные графы алгоритмов решаемых задач и могут решать разные классы задач, сохраняя при этом высокую эффективность.

Так образом развитая ЭВМ, ориентированная на переработку знаний, должна представлять собой в общем случае коллектив спецпроцессоров, ориентированных на максимально эффективное решение тех или иных классов задач, и обладать следующими свойствами:
\begin{textitemize}
\item Спецпроцессоры представляют собой многопроцессорную вычислительную систему;
\item Структура связей между процессорными элементами спецпроцессоров совпадает со структурой данных или (реже) со структурой информационного графа алгоритма решения задачи;
\item Связи между процессорными элементами спецпроцессоров имеют перестраиваемую структуру;
\item Набор и функции спецпроцессоров определяются для каждой машины переработки знаний конкретно в зависимости от набора предметных областей, на которые эта машина ориентирована, и специфики задач, решаемых в этих областях;
\item Выявленный для некоторого семантического процессора набор механизмов переработки знаний должен быть "погружен"{} в язык представления и переработки знаний. При этом наиболее удобными для этой цели представляются языки семантических сетей;
\item Процессорные элементы соответствуют вершинам или фрагментам семантической сети;
\item Переработка информации сводится к изменению структуры связей между процессорными элементами, соответствующему изменению конфигурации семантической сети.
\end{textitemize}

В качестве семантического спецпроцессора можно предложить нелинейную (графовую) структурно перестраиваемую (динамическую) процессоропамять, аппаратно реализующую некоторый язык переработки семантических сетей, а саму ЭВМ такого рода можно, таким образом, назвать графодинамическим параллельным ассоциативным компьютером или \textit{ассоциативным семантическим компьютером}.

%TODO Уточнить про современные коммутаторы

%TODO Принципы уточнить
С учетом сказанного выше, а также общих принципов обработки информации в ostis-системах, описанных в \textit{\ref{subsec_ps_proposed_approach}~\nameref{subsec_ps_proposed_approach}}, рассмотрим более конкретно принципы, лежащие в основе реализации \textit{ассоциативных семантических компьютеров}:
\begin{textitemize}
	\item нелинейная память -- каждый элементарный фрагмент хранимого в памяти текста может быть инцидентен неограниченному числу других элементарных фрагментов этого текста. Таким образом, ячейки памяти, в отличие от обычной памяти, связываются не фиксированными условными связями, задающими фиксированную последовательность (порядок) ячеек в памяти, a реально (физически) проводимыми связями произвольной конфигурации. Эти связи соответствуют дугам, ребрам, гиперребрам записанного в памяти графа (sc-текста);
	\item структурно-перестраиваемая (реконфигурируемая) память -- процесс отработки хранимой в памяти информации сводится не только к изменению состояния элементов, но и к реконфигурации связей между ними. То есть, в ходе переработки информации в структурно-перестраиваемой памяти меняются на только и даже не столько состояния ячеек памяти, как это имеет место в обычной памяти, сколько конфигурация связей между этими ячейками. Т.е. в структурно-перестраиваемой памяти в ходе переработки информации не только перераспределяются метки на вершинах записанного в памяти графа, но и меняется структура самого этого графа;	
	\item в качестве внутреннего способа кодирования знаний, хранимых в памяти \textit{ассоциативного семантического компьютера}, используется универсальный (!) способ нелинейного (графоподобного) смыслового представления знаний -- SC-код;
	\item обработка информации осуществляется коллективом агентов, работающих над общей памятью. Каждый из них реагирует на соответствующую ему ситуацию или событие в памяти (компьютер, управляемый хранимыми знаниями);
	\item есть программно реализуемые агенты, поведение которых описывается хранимыми в памяти агентно-ориентированными программами, которые интерпретируются соответствующими коллективами агентов;
	\item есть базовые агенты, которые не могут быть реализованы программно (в частности, это агенты интерпретации агентных программ, базовые рецепторные агенты-датчики, базовые эффекторные агенты);
	\item все агенты работают над общей памятью одновременно. Более того, если для какого-либо агента в некоторый момент времени в различных частях памяти возникает сразу несколько условий его применения, разные информационные процессы, соответствующие указанному агенту в разных частях памяти могут выполняться одновременно;
	\item для того, чтобы информационные процессы агентов, параллельно выполняемые в общей памяти не "мешали"{} друг другу, для каждого информационного процесса в памяти фиксируется и постоянно актуализируется его текущее состояние. То есть каждый информационный процесс сообщает всем остальным о своих намерениях и пожеланиях, которым остальные информационные процессы не должны препятствовать. Реализация такого подхода может выполняться, например, на основе механизма блокировок элементов семантической памяти, рассмотренного в главе \nameref{chapter_situation_management};
	\item процессор и память \textit{ассоциативного семантического компьютера} глубоко интегрированы и составляют единую процессоро-память. Процессор \textit{ассоциативного семантического компьютера} равномерно "распределен"{} по его памяти так, что процессорные элементы одновременно являются и элементами памяти компьютера. То есть каждая ячейка дополняется функциональным (процессорным) элементом, a перестраиваемые связи между ячейками становятся коммутируемыми каналами связи между функциональными элементами. Каждый функциональный элемент при этом имеет свою специальную внутреннюю регистровую память, отражающую важные для данного функционального элемента аспекты текущего состояния процесса выполнения элементарных операций внутреннего языка.
	
	Обработка информации в \textit{ассоциативном семантическом компьютере} сводится к реконфигурации каналов связи между процессорными элементами,  следовательно память такого компьютера есть не что иное, как \uline{коммутатор} (!) указанных каналов связи. Таким образом, текущее состояние конфигурации этих каналов связи и есть текущее состояние обрабатываемой информации. Этот принцип обеспечивает значительное ускорение переработки информации благодаря исключению этапов передачи информации из памяти в процессор и обратно, но оплачивается ценой большой избыточности функциональных (процессорных) средств, равномерно распределяемых по памяти.
\end{textitemize}

\section{Архитектура ассоциативных семантических компьютеров для ostis-систем}

\begin{SCn}
	\scnheader{ассоциативный семантический компьютер}
	\scnsubset{компьютер с графодинамической ассоциативной памятью}
	\scnidtf{associative semantic computer}
	\scnidtf{sc-компьютер}	
	\scnidtf{аппаратно реализованный базовый интерпретатор семантических моделей (sc-моделей) компьютерных систем}
	\scnidtf{аппаратно реализованная ostis-платформа}
	\scnidtf{аппаратный вариант ostis-платформы}
	\scnidtf{семантический ассоциативный компьютер, управляемый знаниями}
	\scnidtf{компьютер с нелинейной структурно перестраиваемой (графодинамической) ассоциативной памятью, переработка информации в которой сводится не к изменению состояния элементов памяти, а к изменению конфигурации связей между ними}
	\scnidtf{универсальный компьютер нового поколения, специально предназначенный для реализации семантически совместимых гибридных интеллектуальных компьютерных систем}
	\scnidtf{универсальный компьютер нового поколения, ориентированный на аппаратную интерпретацию логико-семантических моделей интеллектуальных компьютерных систем}
	\scnidtf{универсальный компьютер нового поколения, ориентированный на аппаратную интерпретацию ostis-систем}
	\scnidtf{ostis-компьютер}
	\scnidtf{компьютер для реализации ostis-систем}
	\scnidtf{компьютер, управляемый знаниями, представленными в SC-коде}
	\scnidtf{компьютер, ориентированный на обработку текстов SC-кода}
	\scnidtf{компьютер, внутренним языком которого является SC-код}
	\scnidtf{компьютер, осуществляющий реализацию sc-памяти и интерпретацию scp-программ}
	\scnidtf{предлагаемый нами компьютер нового поколения, ориентированный на реализацию интеллектуальных компьютерных систем и использующий SC-код в качестве внутреннего языка}
	\begin{scnsubdividing}
		\scnitem{scp-компьютер} %TODO Провести аналогию с платформами
		\begin{scnindent}
			\scnidtf{sc-компьютер с минимальным набором аппаратно реализованных sc-агентов}
		\end{scnindent}
		\scnitem{sc-компьютер с расширенным набором аппаратно реализуемых sc-агентов}
	\end{scnsubdividing}
	
	\scnheader{scp-компьютер}
	%TODO Уточнить, что такое базовые агенты
	\scnidtf{минимальная конфигурация аппаратно реализованной ostis-платформы, в рамках которой все небазовые sc-агенты реализованы в виде агентных scp-программ}
	\scnidtf{минимальная конфигурация аппаратно реализованной ostis-платформы, в рамках которой аппаратно реализуются только базовые sc-агенты}
	\scnexplanation{В рамках scp-компьютера аппаратно реализуется (1) sc-память, (2) базовые sc-агенты, обеспечивающие интерпретацию scp-программ, (3) элементарные рецепторные sc-агенты, (4) элементарные эффекторные sc-агенты}
\end{SCn}

Для уточнения архитектуры \textit{ассоциативных семантических компьютеров} необходимо уточнить:
\begin{textitemize}
	\item Базовую структуру компьютера, и, в частности, его процессоро-памяти;
	\item Алфавит элементов, хранимых в процессоро-памяти компьютера;
	\item Систему команд, интерпретируемых компьютером;
	\item Принципы управления процессом интерпретации указанных команд;
	\item Систему микропрограмм, обеспечивающих реализацию принципов управления указанным процессом.
\end{textitemize}

Поскольку внутренним языком кодирования информации для \textit{ассоциативного семантического компьютера} является SC-код, то алфавит элементов, хранимых в процессоро-памяти компьютера, совпадает с \textit{Алфавитом SC-кода}, рассмотренным в \textit{Главе \ref{chapter_sc_code}~\nameref{chapter_sc_code}}. При этом алфавит физически кодируемых  синтаксических меток может быть расширен, например, из соображений производительности, по аналогии с тем, как это делается в программном варианте реализации \textit{ostis-платформы} (см. \textit{Главу \ref{chapter_soft_platform}~\nameref{chapter_soft_platform}}).

В качестве системы команд для \textit{ассоциативного семантического компьютера} предлагается \textit{Язык SCP}, подробно рассмотренный в \textit{\ref{sec_ps_scp}~\nameref{sec_ps_scp}}. Таким образом, как уже сказано в указанном параграфе, \textit{Язык SCP} представляет собой ассемблер для \textit{ассоциативного семантического компьютера}.

Для определения базовой структуры \textit{ассоциативного семантического компьютера} уточним варианты такой структуры, предложенные в работах В.В. Голенкова и В.М. Кузьмицкого [ссылки]. В частности, в работе В.М. Кузмицкого предложен переход от крупноструктурных архитектур графодинамических машин к мелкоструктурным [Кузьмицкий].

Модели крупноструктурных архитектур имеют в своем составе параллельно функционирующие модули, обладающие следующими свойствами: 
\begin{textitemize}
	\item Каждый модуль имеет строго фиксированное функциональное назначение в рамках архитектуры графодинамической машины в целом (так называемое глобальное функциональное назначение).
	\item Каждый модуль имеет относительно большой объем памяти (количество элементов памяти много больше общего количества модулей).
	\item Над памятью каждого типа модуля определен свой неэлементарный набор операций, выполняющих некоторые законченные преобразования над достаточно большими фрагментами памяти.
\end{textitemize}

Основным формальным отличием для моделей мелкоструктурных архитектур выступает иное соотношение между общим количеством модулей и количеством элементов памяти каждого модуля (емкостью памяти модуля), которое стремится к единице, и уровнем сложности операций модели. Соответственно свойствами моделей мелкоструктурных архитектур можно считать следующие [Кузьмицкий]:
\begin{textitemize}
\item Для каждого модуля по отдельности может не просматриваться его функциональное назначение в рамках графодинамической машины в целом. Вместе с тем каждый отдельный модуль в конкретный момент времени может иметь некоторое так называемое локальное функциональное назначение, соответствующее множество которых может уже рассматриваться как имеющее определенное так называемое глобальное функциональное назначение в рамках графодинамической машины в целом.
\item Объем (количество элементов) памяти каждого модуля минимален и стремится к единице. Как следствие, общее количество модулей сопоставимо с общим количеством элементов памяти всех модулей.
\item Для каждого модуля (в общем случае) набор операций, выполняемых над его памятью, элементарен и ограничен (конечен), так как воздействует лишь на один элемент (или лишь на несколько элементов) памяти и определяется очевидной ограниченностью (конечностью) семантики интерпретации содержимого элемента графовой памяти графодинамической машины.
\end{textitemize}

Целесообразность перехода от крупноструктурных архитектур к мелкоструктурным обусловлена  соответствующим увеличением степени потенциального параллелизма в процедурах переработки знаний. При этом максимально возможный параллелизм, очевидно, будет иметь место при предельной реализации мелкоструктурных архитектур, в которых один структурный модуль процессоро-памяти будет соответствовать одному элементу памяти, то есть в нашем случае -- одному sc-элементу.

%TODO Уточнить про метки

С учетом сказанного рассмотрим более детально принципы, лежащие в основе базовой структуры \textit{ассоциативного семантического компьютера}:
\begin{textitemize}
	\item процессоро-память \textit{ассоциативного семантического компьютера} состоит из однотипных модулей, которые будем называть процессорными элементами sc-памяти или просто \textit{процессорными элементами}. Каждый \textit{процессорный элемент} соответствует одному sc-элементу (хранит один sc-элемент). При этом в каждый момент времени каждый \textit{процессорный элемент} может быть пустым (не хранить никакой sc-элемент) или заполненным, то есть иметь взаимно однозначно соответствующий ему хранимый sc-элемент. На физическом уровне для описания этого факта вводится соответствующий признак, имеющим два значения.
	\item каждый процессорный элемент имеет память, в которой хранится
		\begin{textitemize}
		\item синтаксическая метка, задающая тип соответствующего sc-элемента;
		\item содержимое sc-файла или ссылка на внешнюю файловую систему (если данный процессорный элемент соответствует sc-файлу);
		\item перечень логических связей данного процессорного элемента с другими, то есть перечень адресов процессорных элементов, связанных с данным процессорным элементом \textit{логическими каналами связи} с указанием типа связи (подробнее о \textit{логических каналах связи} см. ниже);
		\item метка блокировки sc-элементов с указанием метки соответствующего процесса;
		\item волновые микропрограммы, выполняемые данным процессорным элементом в данный момент (подробнее о \textit{волновых микропрограммах} см. ниже), и временные данные для этих микропрограмм, а также очередь микропрограмм при необходимости.
	\end{textitemize}
	\item процессорные элементы связаны между собой двумя типами каналов связи -- \textit{физическими каналами связи} и \textit{логическими каналами связи}. 
	\begin{textitemize}
		\item В общем случае число \textit{физических каналов связи} у каждого \textit{процессорного элемента} может быть произвольным, кроме того теоретически \textit{физические каналы связи} между процессорными элементами могут перестраиваться (перекоммутироваться) с течением времени, например, с целью оптимизации времени передачи сообщений между процессорными элементами. Конфигурация \textit{физических каналов связи} не учитывается на уровне логической обработки знаний, как на уровне Языка SCP, так и на уровне языка микропрограмм, обеспечивающих интерпретацию команд Языка SCP, то есть \textit{scp-операторов}. Для упрощения в рамках данной работы будем рассматривать вариант физической реализации sc-памяти, в котором каждый процессорный элемент имеет фиксированное и одинаковое для всех процессорных элементов число \textit{физических каналов связи} (N), при этом конфигурация таких каналов связи с течением времени не меняется. Очевидно, что минимальным значением N является 2, в этом случае мы получим линейную цепочку \textit{процессорных элементов}. При N равном 4 мы получим двумерную "матрицу"{} \textit{процессорных элементов}, При N равном 6 -- трехмерную "матрицу"{} \textit{процессорных элементов} и т.д. Будем называть ``смежными'' \textit{процессорные элементы}, непосредственно связанные \textit{физическим каналом связи}.
		\item В таком случае можно сказать, что каждый процессорный элемент имеет свой \underline{адрес} в некотором многомерном пространстве, число измерений (признаков) которого определяется числом N \textit{физических каналов связи}, связанных с одним \textit{процессорным элементом}. В приведенных выше примерах размерность такого пространства равна N/2, что позволяет предположить, что число N целесообразно делать четным.
		\item \textit{логические каналы связи} между процессорными элементами формируются динамически и соответствуют \textit{связям инцидентности} между sc-элементами. Таким образом, \textit{логические каналы связи} могут описывать два типа связей инцидентности -- \textit{инцидентность обозначений sc-пар с их компонентами} и \textit{инцидентность обозначений ориентированных sc-пар с их вторыми компонентами} (см. \textit{\ref{sec_syntactic_core_sc_code}~\nameref{sec_syntactic_core_sc_code}}). При этом конфигурация \textit{логических каналов связи} в общем случае никак не связана с конфигурацией \textit{физических каналов связи} -- инцидентные sc-элементы могут физически храниться в процессорных элементах, не являющихся смежными.
		\item Кроме связей инцидентности \textit{логические каналы связи} могут соответствовать и другим типам связей между sc-элементами, по аналогии с тем, как это сделано в программном варианте реализации ostis-платформы (см. \textit{Главу \ref{chapter_soft_platform}~\nameref{chapter_soft_platform}}). Например, для упрощения реализации алгоритмов поиска в базе знаний и уменьшения объема памяти, которым должен обладать каждый \textit{процессорный элемент}, целесообразно хранить в памяти процессорного элемента адрес только первого sc-коннектора, инцидентного соответствующему sc-элементу соответствующим типом инцидентности, а в рамках процессорного элемента, соответствующего данному sc-коннектору, адрес следующего sc-коннектора, инцидентного тому же sc-элементу тем же типом инцидентности и т.д. При таком подходе количество памяти процессорного элемента, хранящей логические связи между процессорными элементами, можно сделать фиксированным.
	\end{textitemize}
	\item Каждый процессорный элемент может отправлять сообщения (микропрограммы) другим процессорным элементам и принимать сообщения от других процессорных элементов по \textit{логическим каналам связи} и имеет соответствующие рецепторно-эффекторные подмодули. На физическом уровне передача сообщений осуществляется, в свою очередь, по \textit{физическим каналам связи}, конфигурация которых, как было сказано выше, фиксируется и в общем случае не зависит от конфигурации логических каналов связи.
	\item Таким образом, процессорные элементы формируют однородную процессоро-память, в которой нет отдельно выделяемых модулей, предназначенных только для хранения информации и отдельно выделяемых модулей, предназначенных только для ее обработки. 
	\item Для связи такой процессоро-памяти с внешней средой вводится \textit{терминальный модуль} [Голенков], который в общем случае может быть реализован по разному и задачами которого являются:
	\begin{textitemize}
		\item подготовка (генерация) информации, поступающей из внешней среды для ее последующей загрузки в процессорные модули;
		\item передача (использование, реализация) информации, подготовленной (полученной, представленной) в процессорных модулях, во внешнюю среду.
	\end{textitemize}
	\item Для хранения содержимых sc-файлов большого размера может оказаться целесообразным иметь отдельную файловую память, связанную с процессоро-памятью и построенную по традиционным фон-Неймановским принципам. Это обусловлено тем, что основная цель построения процессоро-памяти -- обеспечение как можно большей параллельности при обработке конструкций SC-кода, в случае же с хранением и обработкой содержимых sc-файлов, которые по определению являются внешними по отношению к SC-коду информационными конструкциями, целесообразно использовать современные традиционные подходы.
\end{textitemize}

Перечисленные принципы позволяют сформулировать ключевую особенность обработки информации, хранимой в рамках такой процессоро-памяти. В отличие от архитектуры фон-Неймана (и других архитектур, разработанных примерно в то же время, например, Гарвардской архитектуры) и даже от \textit{программного варианта ostis-платформы} в предлагаемой архитектуре процессоро-памяти \underline{отсутствует общая память}, доступная для всех модулей, осуществляющих обработку информации. Благодаря этому значительно упрощается параллельная обработка информации, однако усложняется реализации набора микропрограмм интерпретации команд обработки информации в такой памяти, поскольку каждый процессорный элемент становится очень "близоруким"{} и "видит"{} только те процессорные элементы, которые связаны с ним \textit{логическими каналами связи}. 

Таким образом, язык описания микропрограмм интерпретации команд \textit{ассоциативного семантического компьютера} не может быть построен как традиционный язык программирования, например, процедурного типа, поскольку все такие языки предполагают возможность непосредственного адресного или ассоциативного доступа к произвольным элементам памяти. Предлагаемый язык описания микропрограмм предлагается строить по принципам \textit{волновых языков программирования} [ссылки].

Рассмотрим более детально принципы интерпретации команд в рамках рассмотренной процессоро-памяти (уточнить и дополнить). %TODO
\begin{textitemize}
	\item Каждый процессорный элемент может интерпретировать некоторый ограниченный набор микропрограмм
	\item Каждый процессорный элемент может порождать и хранить в памяти временные данные для микропрограмм (Подумать, что делать, если память переполняется, у Кузьмицкого что-то написано)
	\item Каждый процессорный элемент может сформировать микропрограмму и отправить ее в виде волны для выполнения другими процессорными элементами. Передача сообщений происходит по физическим каналам связи. Поскольку конфигурация физических каналов связи в общем случае не связана конфигурацией логических каналов связи, то каждый процессорный элемент самостоятельно принимает решение о необходимости выполнения микропрограммы и передачи ее дальше. Можно провести аналогию с волновым алгоритмом поиска пути в графе (вариант поиска в ширину)
	\item Часто процессорные элементы будут не выполнять микропрограмму, а передавать ее дальше, таким образом, сами процессорные элементы выполняют также и роль коммутационных элементов
	\item В рамках каждого процессорного элемента тоже можно что-то параллелить, например, выполнение микропрограммы и проверку условия выполнения для других микропрограмм и отправку сообщения дальше, если условие не выполнено. Но тут надо подумать.
	\item ***
\end{textitemize}

Типология микропрограмм (уточнить и дополнить). %TODO
\begin{textitemize}
	\item Переслать указанную микропрограмму для исполнения из данного процессорного элемента по всем \underline{указанным} каналам (инцидентным sc-коннекторам указываемого типа) всем \underline{смежным} sc-элементам указываемого типа;
	\item Выполнить указанное преобразование содержимого данного sc-узла;
	\item Заменить метку типа sc-элемента (возможно стоит запретить);
	\item Заменить блокировку данного sc-элемента для указанного процесса (в том числе, отменить метку);
	\item Удалить sc-элемент;
	\item Сгенерировать инцидентный sc-коннектор (и новый \textit{логический канал связи}), возможно, вместе со смежным sc-элементом;
	\item Сгенерировать оба или один sc-элемент, соединяемые данным sc-коннектором
	\item ***
\end{textitemize}

Иерархия языков программирования (уточнить и дополнить). %TODO
\begin{textitemize}
	\item Язык SCP
	\item Язык микропрограмм, которыми обмениваются процессорные элементы между собой, и которые исполняются этими процессорными элементами
	\item Язык, на котором пишутся программы обмена сообщениями (микропрограммами)
\end{textitemize}

Достоинства (дописать): %TODO
\begin{textitemize}
	\item параллелизм и расширяемость
	\item возможность реализовать параллельность А-систем [Котов, Нариньяни]
\end{textitemize}