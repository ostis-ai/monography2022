\chapter{Ассоциативные семантические компьютеры для ostis-систем}
\chapauthortoc{Голенков В.В.\\Шункевич Д.В.\\Гулякина Н.А.\\Захарьев В.А.}
\label{chapter_computers}

\vspace{-7\baselineskip}

\begin{SCn}
\begin{scnrelfromlist}{автор}
	\scnitem{Голенков В.В.}
	\scnitem{Шункевич Д.В.}
	\scnitem{Гулякина Н.А.}	
	\scnitem{Захарьев В.А.}	
\end{scnrelfromlist}

\bigskip

\scntext{аннотация}{В главе рассмотрены принципы реализации аппаратной платформы для реализации систем, построенных на основе Технологии OSTIS, -- ассоциативного семантического компьютера.}

\bigskip

\begin{scnrelfromlist}{подраздел}
	\scnitem{***}
\end{scnrelfromlist}

\bigskip

\begin{scnrelfromlist}{ключевое понятие}
	\scnitem{***}
\end{scnrelfromlist}

\bigskip

\begin{scnrelfromlist}{библиографическая ссылка}
	\scnitem{***}
\end{scnrelfromlist}

\end{SCn}

\section*{Введение в Главу \ref{chapter_computers}}

Применение для разработки \textit{ostis-систем} современных программно-аппаратных платформ, ориентированных на адресный доступ к хранящимся в памяти данным, не всегда оказывается эффективным, поскольку при разработке интеллектуальных систем фактически приходится моделировать нелинейную память на базе линейной. Повышение эффективности решения задач интеллектуальными системами требует разработки специализированных платформ, в том числе аппаратных, ориентированных на унифицированные семантические модели представления и обработки информации. Таким образом, основной целью создания \textit{ассоциативных семантических компьютеров} является повышение производительности ostis-систем.

\section{Современное состояние в области разработки компьютеров для интеллектуальных систем}

Рассмотрим основные принципы, лежащие в основе \textit{Абстрактной машины фон-Неймана}.

%TODO Принципы машины фон-Неймана
%TODO Добавить картинку?
%TODO Добавить материал из литературы и ссылки
\begin{SCn}
	\scnheader{машина фон-Неймана}
	\scnsubset{абстрактная машина обработки информации}
	\begin{scnrelfromvector}{принципы, лежащие в основе}
		\scnfileitem{Информация в памяти представляется в виде последовательности строк символов.}
		\scnfileitem{Память машины представляет собой последовательность \underline{адресуемых} ячеек памяти.}
		\scnfileitem{В каждую ячейку может быть записана любая строка символов в бинарном алфавите (``0'' или ``1''). При этом длина строк для всех адресуемых ячеек одинакова (в текущем стандарте ячеек, называемых байтами, равна 8 бит).}
		\scnfileitem{Каждой ячейке памяти взаимно однозначно соответствует битовая строка, обозначающая эту ячейку и являющаяся ее адресом.}
		\scnfileitem{каждому типу элементарных действий (операций), выполняемых в памяти машины фон-Неймана, взаимно однозначно ставится ее идентификатор, который в памяти представляется в виде битовой строки}
		\scnfileitem{каждая конкретная операция (команда), выполняемая в памяти, представляется (специфицируется) в памяти в виде строки, состоящей
			\begin{scnitemize}
				\item из кода соответствущего типа операции;
				\item из последовательности адресов фрагментов памяти, в которых находятся операнды, над которыми выполняются операции -- исходные аргументы и результаты. Любой такой фрагмент задается адресом первого байта и количеством байт. Количество операндов \underline{однозначно} задается кодом типа операции;
		\end{scnitemize}}
		\scnfileitem{Программа, выполняемая в памяти, хранится в памяти в виде последовательности спецификаций конкретных операций (команд).}
	\end{scnrelfromvector}
\end{SCn}

%TODO Сослаться на Кузмицкого обязательно, там анализ большой

Рассмотрим более детально особенности логической организации традиционной (фон-Неймановской) архитектуры компьютерных систем, существенно затрудняющие эффективную реализацию \textit{ostis-систем} на ее основе:
\begin{textitemize}
	\item последовательная обработка, ограничивающая эффективность компьютеров физическими возможностями элементной базы;
	\item низкий уровень доступа к памяти, т.е. сложность и громоздкость выполнения процедуры ассоциативного поиска нужного фрагмента знаний; 
	\item линейная организация памяти и чрезвычайно простой вид конструктивных объектов, непосредственно хранимых в памяти. Это приводит к тому, что в интеллектуальных системах, построенных на базе современных компьютеров, манипулирование знаниями осуществляется с большим трудом. Во-первых, приходится оперировать не самими структурами, а их громоздкими линейными представлениями (списками, матрицами смежности, матрицами инцидентности); во-вторых, линеаризация сложных структур разрушает локальность их преобразований;
	\item представление информации в памяти современных компьютеров имеет уровень весьма далекий от смыслового, что делает переработку знаний довольно громоздкой, требующей учета большого количества деталей, касающейся не смысла перерабатываемой информации, а способа ее представления в памяти;
	\item в современных компьютерах имеет место весьма низкий уровень аппаратно реализуемых операций над нечисловыми данными и полностью отсутствует аппаратная поддержка логических операций над фрагментами знаний, имеющих сложную структуру, что делает манипулирование такими фрагментами весьма сложным.
\end{textitemize}

Попытки преодоления ограничений традиционных фон-Неймановских ЭВМ привели к возникновению множества подходов, связанных с отдельными изменениями принципов логической организации ЭВМ, прежде всего в зависимости от классов задач и предметных областей, на которые ориентируется тот или иной класс ЭВМ. Все эти тенденции, рассмотренные в совокупности, позволяют очертить некоторые ключевые принципы логической организации ЭВМ, ориентированных на переработку знаний (машин переработки знаний -- МПЗ). Перечислим основные из этих тенденций:
\begin{textitemize}
\item Переход к нелинейной организации памяти и аппаратная интерпретация сложных структур данных.
\item Аппаратная реализация ассоциативного доступа к информации.
\item Реализация параллельных асинхронных процессов над памятью и, в частности, разработка вычислительных машин, управляемых потоком данных.
\item Аппаратная интерпретация языков высокого уровня.
\item Разработка аппаратных средств ведения баз данных (процессоров баз данных).
\end{textitemize}
	
На пересечении этих тенденций в разное время возникали различные классы вычислительных устройств. Перечислим некоторые из них:
\begin{textitemize}
\item машины с аппаратной интерпретацией сложных структур данных;
\item машины с развитой ассоциативной памятью;
\item ассоциативные параллельные матричные процессоры типа STARAN;
\item однородные параллельные структуры для решения комбинаторно-логических задач на графах и гиперграфах; %TODO Добавить ссылки на статью про графовый компьютер
\item всевозможные устройства переработки графов [ссылка на статью про графовый компьютер];
\item системы, осуществляющие переработку информации непосредственно в памяти путем равномерного распределения функциональных средств по памяти и, в частности, предложенная М.Н.Вайнцвайгом процессоро-память, ориентированная на решение задач искусственного интеллекта [ссылка]; %TODO ссылка
\item машины, управляемые потоком данных;
\item рекурсивные вычислительные машины;
\item процессоры реляционных баз данных;
\item вычислительные машины со структурно-перестраиваемой памятью;
\item активные семантические сети (М-сети);
\item ассоциативные однородные среды;
\item нейроподобные структуры.
\end{textitemize}

Перечисленные особенности фон-Неймановской архитектуры, по существу, не устраняются также и в развиваемых в настоящее время подходах к построению нетрадиционных высокопроизводительных компьютеров (например, компьютеров, предназначенных для обучения и интерпретации искусственных нейронных сетей \cite{Neurocomputers,USB_Accelerator}), ибо, в основном, все эти подходы (даже если они достаточно далеко отходят от предложенных фон Нейманом базовых принципов организации вычислительных машин) неявно сохраняют точку зрения на компьютер как на большой арифмометр и тем самым сохраняют ее ориентацию на задачи числового характера.

Существует ряд современных статей  \cite{Tran2018,Shi2018,Lu2021,Afanasyev2021,Zhang2017,Hu2021,Minati2019,Song2016} и патентов \cite{Somsubhra_2006,Allen_1989,Moussa_2013}, направленных на разработку аппаратных архитектур, предназначенных для обработки информации, представленной в более сложных формах, чем в традиционных архитектурах, однако они не получили широкого распространения и применения, по причине, во-первых, частности предлагаемых решений, и во-вторых из-за отсутствия общего универсального и унифицированного языка кодирования любой информации, в роли которого в рамках Технологии OSTIS выступает SC-код.

\textit{SC-код}, являющийся формальной основой \textit{Технологии OSTIS} изначально разрабатывался как язык кодирования информации в памяти \textit{ассоциативных семантических компьютеров}, таким образом в нем изначально заложены такие принципы, как универсальность (возможность представить знания любого рода) и унификация (единообразие) представления, а также минимизация \textit{Алфавита SC-кода}, которая, в свою очередь, позволяет облегчить создание аппаратной платформы, позволяющей хранить и обрабатывать тексты \textit{SC-кода}.

Основная методологическая особенность предлагаемого подхода к разработке средств аппаратной реализации поддержки интеллектуальных систем заключается в том, что такие средства должны разрабатываться не до, а \underline{после того, как} основные положения соответствующей \underline{технологии} проектирования и эксплуатации интеллектуальных систем будут апробированы на современных технических средствах. Более того, в рамках \textit{\textit{Технологии OSTIS}} четко продумана методика перехода на новые аппаратные средства, которая затрагивает только самый нижний уровень технологии -- уровень реализации базовой машины обработки семантических сетей (интерпретатора \textit{Языка SCP}).



Проект \textit{ассоциативного семантического компьютера} имеет давнюю историю, основными этапами которой являются:
\begin{textitemize}
\item 1984 год -- в Московском  институте электронной техники В.В. Голенковым защищена кандидатского диссертация на тему ``Структурная организация и переработка информации в электронных математических машинах, управляемых потоком сложноструктурированных данных'', в которой были сформулированы и рассмотрены основные принципы семантических ассоциативных компьютеров;
\item 1993 год -- комиссия Госкомпрома провела успешные испытания прототипа семантического ассоциативного компьютера, разработанного на базе транскомпьютеров в рамках научно-исследовательского проекта ``Параллельная графовая вычислительная система, ориентированная на решение задач искусственного интеллекта'';
%TODO Докторская Голенкова

\item 2000 год -- в Институте проблем управления РАН П.А. Гапоновым защищена кандидатская диссертация на тему ``Модели и методы параллельной асинхронной переработки информации в графодинамической ассоциативной памяти'';
\item 2000 год -- в Институте программных систем РАН В.М. Кузьмицким защищена кандидатская диссертация на тему ``Принципы построения графодинамического параллельного компьютера, ориентированного на решение задач искусственного интеллекта'';
\item 2004 год -- в Белорусском государственном университете информатики и радиоэлектроники Р.Е. Сердюковым защищена кандидатская диссертация на тему ``Базовые алгоритмы и инструментальные средства обработки информации в графодинамических ассоциативных машинах'', в которой было рассмотрено базовое программное обеспечение семантических ассоциативных компьютеров.
\end{textitemize}

\section{Общие принципы, лежащие в основе ассоциативных семантических компьютеров для ostis-систем}

При формализации предметных областей, имеющих достаточно сложную семантическую организацию, перерабатываемые данные естественным образом группируются в некоторые сложные структуры. Эффективность решения задач, связанных с переработкой сложноструктурированных данных, на многопроцессорных вычислительных системах значительно возрастает в том случае, когда структура связей между процессорными элементами вычислительной системы, решающей эту задачу, совпадает со структурой данных, перерабатываемых в ходе её решения (или, в более общем случае -- отображается в структуру перерабатываемых данных простым и естественным образом). При переходе к переработке данных все более сложной структурной и семантической организации (а затем и к переработке знаний) сохранение высокой эффективности вычислительной системы обеспечивается главным образом путем увеличения числа одновременно работающих процессорных элементов и усложнения структуры связей между ними.

Такую тенденцию развития технических средств ЭВМ мы и рассмотрим в качестве основной линии эволюции, создающей предпосылки ддя появления \textit{ассоциативных семантических компьютеров}. К ней относятся параллельные регулярные спецпроцессоры (векторные, матричные), спецвычислители для решения задач на графах и средства аппаратной поддержки семантических и нейронных сетей. К этой линии примыкают также и ассоциативные процессоры (в которых в роли процессорных элементов выступают ячейки ассоциативной памяти), процессоры баз данных и вычислительные системы, эффективно решающие те или иные классы задач за счет совпадения структуры связей между процессорными элементами со структурой информационного графа алгоритма (систолические вычислители, машины потоков данных).

Закономерным результатом развития вычислительных систем является переход к системам, изменяющим структуру связей между процессорными элементами в процессе функционирования. Такие системы настраивают свою внутреннюю структуру на структуру перерабатываемых данных и информационные графы алгоритмов решаемых задач и могут решать разные классы задач, сохраняя при этом высокую эффективность.

Так образом развитая ЭВМ, ориентированная на переработку знаний, должна представлять собой в общем случае коллектив спецпроцессоров, ориентированных на максимально эффективное решение тех или иных классов задач, и обладать следующими свойствами:
\begin{textitemize}
\item Спецпроцессоры представляют собой многопроцессорную вычислительную систему;
\item Структура связей между процессорными элементами спецпроцессоров совпадает со структурой данных или (реже) со структурой информационного графа алгоритма решения задачи;
\item Связи между процессорными элементами спецпроцессоров имеют перестраиваемую структуру;
\item Набор и функции спецпроцессоров определяются для каждой машины переработки знаний конкретно в зависимости от набора предметных областей, на которые эта МПЗ ориентирована, и специфики задач, решаемых в этих областях;
\item Выявленный для некоторого семантического процессора набор механизмов переработки знаний должен быть "погружен"{} в язык представления и переработки знаний. При этом наиболее удобными для этой цели представляются языки семантических сетей;
\item Процессорные элементы соответствуют вершинам или фрагментам семантической сети;
\item Переработка информации сводится к изменению структуры связей между процессорными элементами, соответствующему изменению конфигурации семантической сети.
\end{textitemize}

В качестве семантического спецпроцессора можно предложить нелинейную (графовую) структурно перестраиваемую (динамическую) процессоропамять, аппаратно реализующую некоторый язык переработки семантических сетей, а саму ЭВМ такого рода можно, таким образом, назвать графодинамическим параллельным ассоциативным компьютером или \textit{ассоциативным семантическим компьютером}.

%TODO Переписать и дополнить
Необходимо описать абстрактную sc-машину по аналогии с тем, как описывается \textit{абстрактная машина фон Неймана}, взяв за основу принципы \textit{Языка SCP}. Необходимо уточнить:
\begin{textitemize}
	\item как кодируется машинная команда
	\item что происходит в памяти в результате ее выполнения
\end{textitemize}

Память sc-машины представляет собой динамический sc-текст. % Или sc-конструкция? Сама память или ее содержимое?

Перечень команд sc-машины (уточнить):
\begin{textitemize}
	\item команда-продукция
	\begin{textitemize}
		\item область действия команды
		\item для всех найденных структур/для одной найденной структуры
		\item образец (шаблон) поиска (с указанием констант и переменных)
		\item образец (шаблон) генерируемой структуры, в состав которой могут входить
		\begin{textitemize}
			\item команды удаления sc-элементов
			\item команды переноса инцидентных связей от одного sc-элемента к другому
		\end{textitemize}
	\end{textitemize}
	\item команда удаления sc-элемента
	\item команда удаления sc-элемента в рамках указываемой структуры %?
	\item команда переноса инцидентных связей от одного sc-элемента к другому
\end{textitemize}

Склеивание двух sc-элементов реализуется в несколько этапов:
\begin{textitemize}
	\item генерация нового sc-элемента, который станет результатом склеивания;
	\item для каждой связи первого склеиваемого sc-элемента генерируется аналогичная связь с новым sc-элементом, после чего эти перенесенные связи удаляются (команда переноса инцидентных связей от одного sc-элемента к другому);
	\item аналогично осуществляется перенос всех связей второго склеиваемого sc-элемента (команда переноса инцидентных связей от одного sc-элемента к другому); %Зачем делать второй перенос, почему нельзя переносить просто на первый sc-элемент
	\item после этого оба склеиваемых sc-элемента удаляются;
\end{textitemize}

Рассмотрим более конкретно принципы, лежащие в основе реализации \textit{ассоциативных семантических компьютеров}:
\begin{textitemize}
	\item нелинейная память -- каждый элементарный фрагмент хранимого в памяти текста может быть инцидентен неограниченному числу других элементарных фрагментов этого текста. Таким образом, ячейки памяти, в отличие от обычной памяти, связываются не фиксированными условными связями, задающими фиксированную последовательность (порядок) ячеек в памяти, a реально (физически) проводимыми связями произвольной конфигурации. Эти связи соответствуют дугам, ребрам, гиперребрам записанного в памяти графа (sc-текста);
	\item структурно-перестраиваемая (реконфигурируемая) память -- процесс отработки хранимой в памяти информации сводится не только к изменению состояния элементов, но и к реконфигурации связей между ними. То есть, в ходе переработки информации в структурно-перестраиваемой памяти меняются на только и даже не столько состояния ячеек памяти, как это имеет место в обычной памяти, сколько конфигурация связей между этими ячейками. Т.е. в структурно-перестраиваемой памяти в ходе переработки информации не только перераспределяются метки на вершинах записанного в памяти графа, но и меняется структура самого этого графа;	
	\item в качестве внутреннего способа кодирования знаний, хранимых в памяти \textit{ассоциативного семантического компьютера}, используется универсальный (!) способ нелинейного (графоподобного) смыслового представления знаний -- SC-код;
	\item обработка информации осуществляется коллективом агентов, работающих над общей памятью. Каждый из них реагирует на соответствующую ему ситуацию или событие в памяти (компьютер, управляемый хранимыми знаниями);
	\item есть программно реализуемые агенты, поведение которых описывается хранимыми в памяти агентно-ориентированными программами, которые интерпретируются соответствующими коллективами агентов;
	\item есть базовые агенты, которые не могут быть реализованы программно (в частности, это агенты интерпретации агентных программ, базовые рецепторные агенты-датчики, базовые эффекторные агенты);
	\item все агенты работают над общей памятью одновременно. Более того, если для какого-либо агента в некоторый момент времени в различных частях памяти возникает сразу несколько условий его применения, разные информационные процессы, соответствующие указанному агенту в разных частях памяти могут выполняться одновременно;
	\item для того, чтобы информационные процессы агентов, параллельно выполняемые в общей памяти не "мешали"{} друг другу, для каждого информационного процесса в памяти фиксируется и постоянно актуализируется его текущее состояние. То есть каждый информационный процесс сообщает всем остальным о своих намерениях и пожеланиях, которым остальные информационные процессы не должны препятствовать. Реализация такого подхода может выполняться, например, на основе механизма блокировок элементов семантической памяти, рассмотренного в главе \nameref{chapter_situation_management};
	\item процессор и память \textit{ассоциативного семантического компьютера} глубоко интегрированы и составляют единую процессоро-память. Процессор \textit{ассоциативного семантического компьютера} равномерно "распределен"{} по его памяти так, что процессорные элементы одновременно являются и элементами памяти компьютера. То есть каждая ячейка дополняется функциональным (процессорным) элементом, a перестраиваемые связи между ячейками становятся коммутируемыми каналами связи между функциональными элементами. Каждый функциональный элемент при этом имеет свою специальную внутреннюю регистровую память, отражающую важные для данного функционального элемента аспекты текущего состояния процесса выполнения элементарных операций внутреннего языка.
	
	Обработка информации в \textit{ассоциативном семантическом компьютере} сводится к реконфигурации каналов связи между процессорными элементами,  следовательно память такого компьютера есть не что иное, как \uline{коммутатор} (!) указанных каналов связи. Таким образом, текущее состояние конфигурации этих каналов связи и есть текущее состояние обрабатываемой информации. Этот принцип обеспечивает значительное ускорение переработки информации благодаря исключению этапов передачи информации из памяти в процессор и обратно, но оплачивается ценой большой избыточности функциональных (процессорных) средств, равномерно распределяемых по памяти.
\end{textitemize}

\section{Архитектура ассоциативных семантических компьютеров для ostis-систем}

В разработке компьютера выделим три этапа:
\begin{textitemize}
	\item Согласование интерфейса между платформенно-независимой частью технологии и платформенно-зависимой частью. Разработка частной спецификации системы команд (желательно очень ограниченной) для компьютера.
	\item Обоснование системы ограничений. Разработка архитектуры компьютера (разработка альтернативных вариантов, экспресс-оценка и выбор и детализация одного из вариантов). Моделирование архитектуры. Верификация проекта. Оценка эффективности архитектуры.
	\begin{textitemize}%TODO См. картинку на фото
		\item Алфавит компьютера
		\item Система команд
		\item Формат команд
		\item Способ программного управления		
		\item Базовая структура компьютера
	\end{textitemize}
	\item Выбор элементной базы, аппаратной платформы, обоснование системы ограничений. Разработка структуры и схемотехники аппаратного прототипа компьютера. Верификация проекта. Испытания и оценка эффективности компьютера.
\end{textitemize}

\begin{SCn}
	\scnheader{ассоциативный семантический компьютер}
	\scnsubset{компьютер с графодинамической ассоциативной памятью}
	\scnidtf{associative semantic computer}
	\scnidtf{sc-компьютер}	
	%TODO Уточнить насчет SCP
	\scnidtf{scp-компьютер}
	\scnidtf{аппаратно реализованный базовый интерпретатор семантических моделей (sc-моделей) компьютерных систем}
	\scnidtf{аппаратно реализованная ostis-платформа}
	\scnidtf{аппаратный вариант ostis-платформы}
	\scnidtf{семантический ассоциативный компьютер, управляемый знаниями}
	\scnidtf{компьютер с нелинейной структурно перестраиваемой (графодинамической) ассоциативной памятью, переработка информации в которой сводится не к изменению состояния элементов памяти, а к изменению конфигурации связей между ними}
	\scnidtf{универсальный компьютер нового поколения, специально предназначенный для реализации семантически совместимых гибридных интеллектуальных компьютерных систем}
	\scnidtf{универсальный компьютер нового поколения, ориентированный на аппаратную интерпретацию логико-семантических моделей интеллектуальных компьютерных систем}
	\scnidtf{универсальный компьютер нового поколения, ориентированный на аппаратную интерпретацию ostis-систем}
	\scnidtf{ostis-компьютер}
	\scnidtf{компьютер для реализации ostis-систем}
	\scnidtf{компьютер, управляемый знаниями, представленными в SC-коде}
	\scnidtf{компьютер, ориентированный на обработку текстов SC-кода}
	\scnidtf{компьютер, внутренним языком которого является SC-код}
	\scnidtf{компьютер, осуществляющий реализацию sc-памяти и интерпретацию scp-программ}
	\scnidtf{предлагаемый нами компьютер нового поколения, ориентированный на реализацию интеллектуальных компьютерных систем и использующий SC-код в качестве внутреннего языка}
	\begin{scnsubdividing}
		\scnitem{scp-компьютер} %TODO Провести аналогию с платформами
		\begin{scnindent}
			\scnidtf{sc-компьютер с минимальным набором аппаратно реализованных sc-агентов}
		\end{scnindent}
		\scnitem{sc-компьютер с расширенным набором аппаратно реализуемых sc-агентов}
	\end{scnsubdividing}
	
	\scnheader{scp-компьютер}
	%TODO Уточнить, что такое базовые агенты
	\scnidtf{минимальная конфигурация аппаратно реализованной ostis-платформы, в рамках которой все небазовые sc-агенты реализованы в виде агентных scp-программ}
	\scnidtf{минимальная конфигурация аппаратно реализованной ostis-платформы, в рамках которой аппаратно реализуются только базовые sc-агенты}
	\scnexplanation{В рамках scp-компьютера аппаратно реализуется (1) sc-память, (2) базовые sc-агенты, обеспечивающие интерпретацию scp-программ, (3) элементарные рецепторные sc-агенты, (4) элементарные эффекторные sc-агенты}
\end{SCn}

%TODO привести в порядок

Компоненты sc-компьютера:

\begin{textitemize}
	\item коммутатор
	\begin{textitemize}
		\item узлы которого -- sc-элементы
		\item коммутируемые каналы связи -- дуги инцидентности sc-элементов
		\item узлы -- процессорные элементы, которые обмениваются между собой фрагментами микропрограмм, обеспечивающих интерпретацию хранимых в памяти scp-программ.
		\item волновой язык микропрограмм
	\end{textitemize}
	\item процессорный элемент sc-памяти. В его памяти хранится:
	\begin{textitemize}
		\item содержимое (если это узел с содержимым)
		\item метка типа sc-элемента
		\item метка блокировки sc-элементов с указанием метки соответствующего процесса
		\item микропрограммы с микрокомандами типа "переслать по всем выходящим sc-дугам указанного типа и с указанными метками блокировки данную микропрограмму" (волновые микропрограммы)
		\item уникальные имена (метки) процессов, к которым относятся передаваемые микропрограммы (с описанием иерархии \underline{над}процессов).
	\end{textitemize}
\end{textitemize}

\textbf{Типология микрокоманд sc-компьютера}
\begin{textitemize}
	\item Переслать указанную микропрограмму для исполнения из данного процессорного элемента по всем \underline{указанным} каналам (инцидентным sc-коннекторам указываемого типа) всем \underline{смежным} sc-элементам указываемого типа;
	\item Выполнить указанное преобразование содержимого данного sc-узла;
	\item Заменить метку типа sc-элемента;
	\item Заменить блокировку данного sc-элемента для указанного процесса (в том числе, отменить метку);
	\item Удалить sc-элемент;
	\item Сгенерировать инцидентный sc-коннектор (новый канал связи), возможно, вместе со смежным sc-элементом;
	\item Сгенерировать оба или один sc-элемент, соединяемые данным sc-коннектором
\end{textitemize}

Языки для sc-компьютера:
\begin{textitemize}
	\item SC-код;
	\item Язык SCP;
	\item Язык описания состояния процессорного элемента (sc-элемента);
	\item Язык микропрограмм, которым обмениваются процессорные элементы между собой, и которые исполняются этими процессорными элементами.
\end{textitemize}

Кроме \textit{процессорных элементов} есть \textit{коммутационные элементы}. Множество \textit{коммутационных элементов} -- это связующая среда, обеспечивающая формирование коммутируемых каналов связи между \textit{процессорными элементами}. 

В докторской В.В. Голенкова описано, каким образом обрабатываемая информационная конструкция распределяется между процессорными элементами (SCD-код).

Любой \textit{коммутационный элемент} может входить в не более чем в один канал связи. %TODO Уточнить, не слишком ли сильное требование

Связь между \textit{коммутационными элементами} имеет два состояния:
\begin{textitemize}
	\item свободна -- не входит в состав коммутируемого канала связи;
	\item занята -- является звеном в канале связи.
\end{textitemize}