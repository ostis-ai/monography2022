%TODO Переписать и дополнить
Необходимо описать абстрактную sc-машину по аналогии с тем, как описывается \textit{абстрактная машина фон Неймана}, взяв за основу принципы \textit{Языка SCP}. Необходимо уточнить:
\begin{textitemize}
	\item как кодируется машинная команда
	\item что происходит в памяти в результате ее выполнения
\end{textitemize}

Память sc-машины представляет собой динамический sc-текст. % Или sc-конструкция? Сама память или ее содержимое?

Перечень команд sc-машины (уточнить):
\begin{textitemize}
	\item команда-продукция
	\begin{textitemize}
		\item область действия команды
		\item для всех найденных структур/для одной найденной структуры
		\item образец (шаблон) поиска (с указанием констант и переменных)
		\item образец (шаблон) генерируемой структуры, в состав которой могут входить
		\begin{textitemize}
			\item команды удаления sc-элементов
			\item команды переноса инцидентных связей от одного sc-элемента к другому
		\end{textitemize}
	\end{textitemize}
	\item команда удаления sc-элемента
	\item команда удаления sc-элемента в рамках указываемой структуры %?
	\item команда переноса инцидентных связей от одного sc-элемента к другому
\end{textitemize}

Склеивание двух sc-элементов реализуется в несколько этапов:
\begin{textitemize}
	\item генерация нового sc-элемента, который станет результатом склеивания;
	\item для каждой связи первого склеиваемого sc-элемента генерируется аналогичная связь с новым sc-элементом, после чего эти перенесенные связи удаляются (команда переноса инцидентных связей от одного sc-элемента к другому);
	\item аналогично осуществляется перенос всех связей второго склеиваемого sc-элемента (команда переноса инцидентных связей от одного sc-элемента к другому); %Зачем делать второй перенос, почему нельзя переносить просто на первый sc-элемент
	\item после этого оба склеиваемых sc-элемента удаляются;
\end{textitemize}

В разработке компьютера выделим три этапа:
\begin{textitemize}
	\item Согласование интерфейса между платформенно-независимой частью технологии и платформенно-зависимой частью. Разработка частной спецификации системы команд (желательно очень ограниченной) для компьютера.
	\item Обоснование системы ограничений. Разработка архитектуры компьютера (разработка альтернативных вариантов, экспресс-оценка и выбор и детализация одного из вариантов). Моделирование архитектуры. Верификация проекта. Оценка эффективности архитектуры.
	\begin{textitemize}%TODO См. картинку на фото
		\item Алфавит компьютера
		\item Система команд
		\item Формат команд
		\item Способ программного управления		
		\item Базовая структура компьютера
	\end{textitemize}
	\item Выбор элементной базы, аппаратной платформы, обоснование системы ограничений. Разработка структуры и схемотехники аппаратного прототипа компьютера. Верификация проекта. Испытания и оценка эффективности компьютера.
\end{textitemize}

Кроме \textit{процессорных элементов} есть \textit{коммутационные элементы}. Множество \textit{коммутационных элементов} -- это связующая среда, обеспечивающая формирование коммутируемых каналов связи между \textit{процессорными элементами}. 

В докторской В.В. Голенкова описано, каким образом обрабатываемая информационная конструкция распределяется между процессорными элементами (SCD-код).

Любой \textit{коммутационный элемент} может входить в не более чем в один канал связи. %TODO Уточнить, не слишком ли сильное требование

Связь между \textit{коммутационными элементами} имеет два состояния:
\begin{textitemize}
	\item свободна -- не входит в состав коммутируемого канала связи;
	\item занята -- является звеном в канале связи.
\end{textitemize}

Компоненты sc-компьютера:

\begin{textitemize}
	\item коммутатор
	\begin{textitemize}
		\item узлы которого -- sc-элементы
		\item коммутируемые каналы связи -- дуги инцидентности sc-элементов
		\item узлы -- процессорные элементы, которые обмениваются между собой фрагментами микропрограмм, обеспечивающих интерпретацию хранимых в памяти scp-программ.
		\item волновой язык микропрограмм
	\end{textitemize}
	\item процессорный элемент sc-памяти. В его памяти хранится:
\end{textitemize}

	\item каждый процессорный элемент имеет память, в которой хранится
		\begin{textitemize}
		\item уникальные имена (метки) процессов, к которым относятся передаваемые микропрограммы (с описанием иерархии \underline{над}процессов).
	\end{textitemize}