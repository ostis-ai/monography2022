\chapter{Представление формальных онтологий базовых классов сущностей в ostis-системах}
\chapauthortoc{Бутрин С.В.\\Шункевич Д.В.}
\label{chapter_top_ontologies}

\vspace{-7\baselineskip}

\begin{SCn}
	\begin{scnrelfromlist}{автор}
		\scnitem{Бутрин С.В.}
	\end{scnrelfromlist}
	
\bigskip

\scntext{аннотация}{В данной главе рассматриваются актуальные проблемы текущего состояния технологий разработки формальных онтологий сущностей. Предложен список онтологий базовых классов сущностей в рамках Технологии OSTIS.}

\bigskip

\begin{scnrelfromlist}{подраздел}
	\scnitem{\ref{sec_top_ontologies_set}~\nameref{sec_top_ontologies_set}}
	\scnitem{\ref{sec_top_ontologies_rel}~\nameref{sec_top_ontologies_rel}}	\scnitem{\ref{sec_top_ontologies_params}~\nameref{sec_top_ontologies_params}}
	\scnitem{\ref{sec_top_ontologies_numbers}~\nameref{sec_top_ontologies_numbers}}
	\scnitem{\ref{sec_top_ontologies_temp}~\nameref{sec_top_ontologies_temp}}
	\scnitem{\ref{sec_top_ontologies_dynamic}~\nameref{sec_top_ontologies_dynamic}}
\end{scnrelfromlist}

\begin{scnrelfromlist}{ключевое понятие}
	\scnitem{онтология}
	\scnitem{предметная область}
	\scnitem{знание}
	\scnitem{база знаний}
	\scnitem{понятие}
	\scnitem{базовый класс сущностей}
	\scnitem{сущность}
	\scnitem{Ядро базы знаний}
	\scnitem{онтология верхнего уровня}
\end{scnrelfromlist}

\bigskip

\begin{scnrelfromlist}{библиографическая ссылка}
	\scnitem{\scncite{Davydenko2017a}}
\end{scnrelfromlist}

\end{SCn}


\section*{Введение в Главу \ref{chapter_top_ontologies}}

Для обеспечения совместного использования различных видов знаний, входящих в состав базы знаний, необходимо обеспечить их совместимость с указанной базой знаний, которая включает семантическую совместимость, что подразумевает однозначную и единую для всех фрагментов базы знаний трактовку используемых понятий.

Среди многообразия средств представления знаний к наиболее эффективным относятся онтологии (см. \scncite{Davydenko2017a}). Суть такого подхода при проектировании базы знаний состоит в рассмотрении базы знаний как иерархической системы выделенных предметных областей и соответствующих им онтологий. Однако онтологически можно по разному специфицировать знания. Чтобы решить эту проблему проектируются онтологии верхнего уровня.

Применение современных онтологий верхнего уровня при разработке баз знаний интеллектуальных компьютерных систем сопряжено с проблемами обеспечения их совместимости. Поскольку изначальной целью создания онтологий верхнего уровня являлось обеспечение  совместимости онтологий предметных областей и прикладных онтологий, а не самих интеллектуальных систем. 

Такими проблемами являются:
\begin{textitemize}
    \item свобода трактовки понятий, вызванная отсутствием их четкого определения;
    \item отсутствие единой технологии проектирования баз знаний на основе онтологий верхнего уровня;
    \item отсутствие принадлежности онтологий верхнего уровня к какой-либо технологии, что не позволяет использовать их в качестве многократно используемых компонентов;
\end{textitemize}

Поэтому возникает необходимость в разработке такой системы онтологии верхнего уровня, которая смогла бы обеспечить семантическую совместимость между большим количеством онтологий различных предметных областей.

Предлагаемый подход подразумевает разработку семейств Предметных областей и онтологий, которые бы содержали описания всех необходимых базовых классов сущностей для построения базы знаний интеллектуальной компьютерной системы.

К таким Предметным областям и онтологиям относятся:

\begin{textitemize}
\item Предметная область и онтология множеств
\item Предметная область и онтология связок и отношений
\item Предметная область и онтология параметров, величин и шкал
\item Предметная область и онтология чисел и числовых структур
\item Предметная область и онтология структур
\item Предметная область и онтология темпоральных сущностей
\item Предметная область и онтология темпоральных сущностей баз знаний ostis-систем
\item Предметная область и онтология семантических окрестностей
\item Предметная область и онтология предметных областей
\item Предметная область и онтология онтологий
\item Предметная область и онтология логических формул, высказываний и формальных %теорий
\item Предметная область и онтология внешних информационных конструкций и файлов ostis-систем
\item Глобальная предметная область действий и задач и соответствующая ей онтология методов и технологий
\end{textitemize}

Данные предметные области являются часть Ядра базы знаний, которое должно быть в каждой интеллектуальной системе. Это ядро гарантирует совместимость интеллектуальных компьютерных систем за счет общего понятийного аппарата. В зависимости от специфики конкретных систем могут выделяться различные Ядра базы знаний, но неизменным должна оставаться наличие базовая части, включающей в себя предметные области и онтологии указанные выше.

\section{Формальная онтология множеств}
\label{sec_top_ontologies_set}

\begin{SCn}

\scnheader{множество}

\begin{scnrelfromset}{разбиение}
\scnitem{конечное множество}
\scnitem{бесконечное множество}
\end{scnrelfromset}

\begin{scnrelfromset}{разбиение}
	\scnitem{множество без кратных элементов}
	\scnitem{мультимножество}
\end{scnrelfromset}


\begin{scnrelfromset}{разбиение}
	\scnitem{связка}
	\scnitem{класс}
	\begin{scnindent}
		\scnidtf{sc-элемент, обозначающий класс sc-элементов}
		\scnidtf{sc-знак множества sc-элементов, эквивалентных в том или ином смысле}
	\end{scnindent}
	\scnitem{структура}
	\begin{scnindent}
		\scnidtf{sc-знак множества sc-элементов, в состав которого входят sc-связки или структуры, связывающие эти sc-элементы}
	\end{scnindent}
\end{scnrelfromset}

\begin{scnrelfromset}{разбиение}
	\scnitem{четкое множество}
	\scnitem{нечеткое множество}
\end{scnrelfromset}

\begin{scnrelfromset}{разбиение}
	\scnitem{множество первичных сущностей}
	\scnitem{множество множеств}
	\scnitem{множество первичных сущностей и множеств}
\end{scnrelfromset}

\begin{scnrelfromset}{разбиение}
	\scnitem{рефлексивное множество}
	\scnitem{нерефлексивное множество}
\end{scnrelfromset}

\begin{scnrelfromset}{разбиение}
	\scnitem{сформированное множество}
	\scnitem{несформированное множество}
\end{scnrelfromset}

\begin{scnrelfromset}{разбиение}
	\scnitem{кортеж}
	\scnitem{неориентированное множество}
\end{scnrelfromset}

\end{SCn}

Под \textbf{\textit{множеством}} понимается соединение в некое целое M определённых хорошо различимых предметов m нашего созерцания или нашего мышления (которые будут называться «элементами» множества M). 
	
\textbf{\textit{множество}} – мысленная сущность, которая связывает одну или несколько сущностей в целое.
	
Более формально \textbf{\textit{множество}} – это абстрактный математический объект, состоящий из элементов. Связь множеств с их элементами задается бинарным ориентированным отношением \textbf{\textit{принадлежность*}}.

\textbf{\textit{Множество}} может быть полностью задано следующими тремя способами:

\begin{textitemize}
		\item путем перечисления (явного указания) всех его элементов (очевидно, что таким способом можно задать только конечное множество)
		\item с помощью определяющего высказывания, содержащего описание общего характеристического свойства, которым обладают все те и только те объекты, которые являются элементами (т.е. принадлежат) задаваемого множества.
		\item с помощью теоретико-множественных операций, позволяющих однозначно задавать новые множества на основе уже заданных (это операции объединения, пересечения, разности множеств и др.)
\end{textitemize}

Для любого семантически ненормализованного \textbf{\textit{множества}} существует единственное семантически нормализованное \textbf{\textit{множество}}, в котором все элементы, не являющиеся знаками множеств, заменены на знаки множеств.

\begin{SCn}
\scnheader{принадлежность*}
\scnidtf{принадлежность элемента множеству*}
\scnidtf{отношение принадлежности элемента множеству*}
\scniselement{бинарное отношение}
\scniselement{ориентированное отношение}
\end{SCn}

\textbf{\textit{принадлежность*}} – это бинарное ориентированное отношение, каждая связка которого связывает множество с одним из его элементов. Элементами отношения \textbf{\textit{принадлежность*}} по умолчанию являются \textit{позитивные sc-дуги принадлежности}.

% \begin{SCn}
% \scnheader{мультимножество}
% \scnidtf{множество, имеющее кратные вхождения хотя бы одного элемента}
% \scnidtf{множество, по крайней мере один элемент которого входит в его состав многократно}
% \end{SCn}

% \textbf{\textit{мультимножество}} - это \textit{множество}, для которого существует хотя бы одна кратная пара принадлежности, выходящая из знака этого множества.

% \begin{SCn}
% \scnheader{кратность принадлежности}
% \scnidtf{кратность принадлежности элемента}
% \scnidtf{кратность вхождения элемента во множество}
% \scniselement{параметр}
% \end{SCn}

% \textbf{\textit{кратность принадлежности}} - \textit{параметр}, значением которого являются числовые величины, показывающие сколько раз входит тот или иной элемент в рассматриваемое множество. Элементами данного параметра являются классы \textit{позитивных sc-дуг принадлежности}, связывающих данное множество с элементом, кратность вхождения которого в данное множество мы хотим задать.
	
% 	Таким образом, кратное вхождение элемента в мультимножество может быть задано как явным указанием \textit{позитивных sc-дуг принадлежности} этого элемента данному \textit{множеству}, так и «склеиванием» этих дуг в одну и включением ее в некоторый класс \textbf{\textit{кратности принадлежности}}.
%%\scnrelfrom{описание примера}{
%%	\scnfilescg{figures/sd_sets/multiplicityOfMembership.png}
%%}

\begin{SCn}
\scnheader{класс}
\scnidtf{класс sc-элементов}
\begin{scnrelfromset}{разбиение}
	\scnitem{класс первичных sc-элементов}
	\scnitem{класс множеств}
\end{scnrelfromset}
\end{SCn}

\textbf{\textit{класс}} – множество элементов, обладающих какими-либо явно указываемыми общими свойствами.

\begin{SCn}
\scnheader{кортеж}
\scnidtf{вектор}
\end{SCn}

\textbf{\textit{кортеж}} – это множество, представляющее собой упорядоченный набор элементов, т.е. такое множество, порядок элементов в котором имеет значение. Пары принадлежности элементов \textbf{\textit{кортежу}} могут дополнительно принадлежать каким-либо \textit{ролевым отношениям}, при этом, в рамках каждого \textbf{\textit{кортежа}} должен существовать хотя бы один элемент, роль которого дополнительно уточнена \textit{ролевым отношением}.


\begin{SCn}
\scnheader{включение*}
\scnidtf{включение множеств*}
\scnidtf{быть подмножеством*}
\scniselement{бинарное отношение}
\scniselement{ориентированное отношение}
\scniselement{транзитивное отношение}
\scnrelfrom{область определения}{множество}
\scnsuperset{строгое включение*}
\end{SCn}

\textbf{\textit{включение*}} – это бинарное ориентированное отношение, каждая связка которого связывает два множества. Будем говорить, что \textit{Множество Si} \textbf{\textit{включает*}} в себя \textit{Множество Sj} в том и только том случае, если каждый элемент \textit{Множества Sj} является также и элементом \textit{Множества Si}
%%\scnrelfrom{описание примера}{
%%	\scnfilescg{figures/sd_sets/inclusion.png}}
%%\scnaddlevel{1}
%%\scntext{пояснение}{Множество {Sj} включается во множество \textit{Si}.}
%%\scnaddlevel{-1}

\begin{SCn}
\scnheader{объединение*}
\scnidtf{объединение множеств*}
\scniselement{квазибинарное отношение}
\scniselement{ориентированное отношение}
\end{SCn}
	
\textbf{\textit{объединение*}} – это \textit{квазибинарное ориентированное отношение}, областью определения которого является семейство всевозможных множеств. Будем говорить, что \textit{Множество Si} является объединением \textit{Множество Sj} и \textit{Множество Sk} тогда и только тогда, когда любой элемент \textit{Множество Si} является элементом или \textit{Множество Sj} или \textit{Множество Sk}.
%%\scnrelfrom{описание примера}
%%	\scnfilescg {figures/sd_sets/union.png}}
%%\scnaddlevel{1}
%%\scntext{пояснение}{Множество \textit{Si} является объединением %%множеств \textit{Sj}, \textit{Sk} и \textit{Sm}.}
%%\scnaddlevel{-1}
%%\scnrelfrom{изображение}{
%%	\scnfileimage{\includegraphics[width=0.6\linewidth]{figures/sd_sets/union2.png}}}

\begin{SCn}
\scnheader{разбиение*}
\scnidtf{разбиение  множества*}
\scnidtf{объединение попарно непересекающихся множеств*}
\scnidtf{декомпозиция множества*}
\scniselement{квазибинарное отношение}
\scniselement{ориентированное отношение}
\scniselement{отношение декомпозиции}
\end{SCn}
	
\textbf{\textit{разбиение*}} – это \textit{квазибинарное ориентированное отношение}, областью определения которого является семейство всевозможных множеств. В результате разбиения множества получается множество попарно непересекающихся множеств, объединение которых есть исходное множество.\\
Семейство множеств \{\textit{S1…, Sn}\} является разбиением множества \textit{Si} в том и только том случае, если:
\begin{textitemize}
		\item семейство \{\textit{S1…, Sn}\} является семейством \textit{попарно непересекающихся множеств};
		\item семейство \{\textit{S1…, Sn}\} является покрытием множества \textit{Si} (или другими словами, множество \textit{Si} является \textit{объединением} множеств, входящих в указанное выше семейство)
\end{textitemize}
%%\scnrelfrom{описание примера}{
%%	\scnfilescg{figures/sd_sets/split.png}}
%\scnaddlevel{1}
%%\scnexplanation{Множество \textit{Si} разбивается на множества \textit{Sj}, \textit{Sk} и \textit{Sm}.}
%%\scnaddlevel{-1}
%%\scnrelfrom{изображение}{
%%	\scnfileimage{\includegraphics[width=0.5\linewidth]{figures/sd_sets/split2.png}}}

\begin{SCn}
\scnheader{пересечение*}
\scnidtf{пересечение множеств*}
\scniselement{квазибинарное отношение}
\scniselement{ориентированное отношение}
\end{SCn}

\textbf{\textit{пересечение*}} – это операция над множествами, аргументами которой являются два или большее число множеств, а результатом является множество, элементами которого являются все те и только те сущности, которые одновременно принадлежат каждому множеству, которое входит в семейство аргументов этой операции.\\
	Будем говорить, что \textit{Множество Si} является пересечением \textit{Множество Sj} и \textit{Множество Sk} тогда и только тогда, когда любой элемент \textit{Множество Si} является элементом \textit{Множество Sj} и элементом \textit{Множество Sk}.
%%\scnrelfrom{описание примера}{
%%	\scnfilescg{figures/sd_sets/intersection.png}}
%%\scnaddlevel{1}
%%\scntext{пояснение}{Множество \textit{Si} является результатом пересечения множеств \textit{Sj}, \textit{Sk} и \textit{Sm}.}
%%\scnaddlevel{-1}
%%\scnrelfrom{изображение}{
%%	\scnfileimage{\includegraphics[width=0.5\linewidth]{figures/sd_sets/intersection2.png}}}

\begin{SCn}
\scnheader{разность множеств*}
\scniselement{бинарное отношение}
\scniselement{ориентированное отношение}
\end{SCn}

\textbf{\textit{разность множеств*}} – это \textit{бинарное ориентированное отношение}, связывающее между собой \textit{ориентированную пару}, первым элементом которой является уменьшаемое множество, вторым - вычитаемое множество, и множество, являющееся результатом вычитания вычитаемого из уменьшаемого, в которое входят все элементы первого множества, не входящие во второе множество.
%%\scnrelfrom{описание примера}{
%%	\scnfilescg{figures/sd_sets/setDifference.png}}
%%\scnaddlevel{1}
%%\scnexplanation{Множество \textit{Si} является результатом разности множеств \textit{Sj} и \textit{Sk}.}
%%\scnaddlevel{-1}
%%\scnrelfrom{изображение}{\scnfileimage{\includegraphics[width=0.5\linewidth]{figures/sd_sets/setDifference2.png}}}

\begin{SCn}
\scnheader{симметрическая разность множеств*}
\scniselement{бинарное отношение}
\scniselement{ориентированное отношение}
\end{SCn}

\textbf{\textit{симметрическая разность множеств*}} – это \textit{бинарное ориентированное отношение}, связывающее между собой \textit{пару} множеств и множество, являющееся результатом симметрической разности элементов указанной пары. Будем называть \textit{Множество Si} результатом симметрической разности \textit{Множества Sj} и \textit{Множества Sk} тогда и только тогда, когда любой элемент \textit{Множества Si} является или элементом \textit{Множества Sj} или \textit{Множества Sk}, но не является элементом обоих множеств.
%%\scnrelfrom{описание примера}{
%%	\scnfilescg{figures/sd_sets/symmetricDifferenceOfSets.png}
%%	\scnexplanation{Множество \textit{Si} является результатом симметрической разности множеств \textit{Sj} и \textit{Sk}.}}
%%\scnrelfrom{изображение}{
%%	\scnfileimage{\includegraphics[width=0.5\linewidth]{figures/sd_sets/symmetricDifferenceOfSets2.png}}}

\begin{SCn}
\scnheader{декартово произведение*}
\scnidtf{декартово произведение множеств*}
\scnidtf{прямое произведение множеств*}
\scniselement{бинарное отношение}
\scniselement{ориентированное отношение}
\end{SCn}

\textbf{\textit{декартово произведение*}} – это \textit{бинарное ориентированное отношение} между \textit{ориентированной парой} множеств и \textit{множеством}, элементами которого являются всевозможные упорядоченные пары, первыми элементами которых являются элементы первого из указанных множеств, вторыми – элементы второго из указанных множеств.
%%\scnrelfrom{описание примера}{
%%	\scnfilescg{figures/sd_sets/cartesianMultiplication.png}}
%%\scnaddlevel{1}
%%\scnexplanation{Множество \textit{Si} является результатом декартова произведения множеств \textit{Sj} и \textit{Sk}.}
%%\scnaddlevel{-1}

\begin{SCn}
\scnheader{булеан*}
\scnidtf{булеан множества*}
\scnidtf{семейство всевозможных подмножеств заданного множества*}
\scniselement{бинарное отношение}
\scniselement{ориентированное отношение}
\end{SCn}
	
\textbf{\textit{булеан*}} – это \textit{бинарное ориентированное отношение} между множеством и некоторым семейством множеств, каждое из которых является подмножеством первого множества.
%%\scnrelfrom{описание примера}{
%%	\scnfilescg{figures/sd_sets/boulean.png}
%%}

\begin{SCn}
\scnheader{мощность множества}
\scnidtf{кардинальное число}
\scnidtf{общее число вхождений элементов в заданное множество}
\scnidtf{класс эквивалентности, элементами которого являются знаки всех тех и только тех множеств, которые имеют одинаковую мощность}
\scnidtf{класс эквивалентности, соответствующий отношению быть парой множеств, имеющих одинаковую мощность (равномощность множеств)}
\scnidtf{величина мощности множеств}
\scnidtf{трансфинитное число}
\scnidtf{мощность по Кантору}
\scniselement{параметр}
\end{SCn}

\textbf{\textit{мощность множества}} – это \textit{параметр}, элементами которых являются \textit{множества}, имеющие одинаковое количество элементов. Значением данного параметра является числовая величина, задающая количество элементов, входящих в данный класс множеств, т.е. по сути, количество \textit{позитивных sc-дуг принадлежности}, выходящих из данного \textit{множества}.
	
	Для двух множеств, имеющих одинаковую мощность, существует взаимно-однозначное соответствие между ними (между множествами вхождений элементов в эти множества – на случай мультимножеств).
%%\scnrelfrom{описание примера}{
%%	\scnfilescg{figures/sd_sets/power.png}
%%}

\section{Формальная онтология связок и отношений}
\label{sec_top_ontologies_rel}
%% Введение в пердметную область
%%%%%%%%%%%%%%%%%%%%%%

\begin{SCn}
\scnheader{Предметная область связок и отношений}
\scniselement{предметная область}
	% \scnsdmainclasssingle{связь}
	% \scnsdclass{бинарная связь;sc-коннектор;неатомарная бинарная связь;небинарная связь;неориентированная связь;ориентированная связь;отношение;класс равномощных связок;класс связок разной мощности;унарное отношение;бинарное отношение;квазибинарное отношение;тернарное отношение;небинарное отношение;ориентированное отношение;неориентированное отношение;рефлексивное отношение;антирефлексивное отношение;частично рефлексивное отношение;симметричное отношение;антисимметричное отношение;частично симметричное отношение;транзитивное отношение;антитранзитивное отношение;частично транзитивное отношение;связанное отношение;отношение порядка;отношение строгого порядка;отношение нестрогого порядка;отношение толерантности;отношение эквивалентности;ролевое отношение;числовой атрибут;неролевое отношение;неролевое бинарное отношение;арность;метаотношение;отношение декомпозиции;отношение интеграции}
	% \scnsdrelation{область определения*;атрибут отношения*;домен*;первый домен*;второй домен*;композиция отношений*;фактор-множество*;соответствие*;отношение соответствия*;область отправления\scnrolesign;область прибытия\scnrolesign;образ\scnrolesign;прообраз\scnrolesign;всюду определенное соответствие*;частично определенное соответствие*;сюръективное соответствие*;несюръективное соответствие*;однозначное соответствие*;обратное соответствие*;обратимое соответствие*;неоднозначное соответствие*;инъективное соответствие*;взаимно однозначное соответствие*;множество сочетаний*;множество размещений*;множество перестановок*}
\scnheader{связь}
\scnidtf{связка sc-элементов}
\scnidtf{sc-связка}
\end{SCn}

\textit{связь} -- множество, являющееся абстрактной моделью связи между описываемыми сущностями, которые или знаки которых являются элементами этого множества.
	
Напомним, что все элементы множества, представленного в SC-коде, являются знаками, но описываемыми сущностями могут быть не только сущности, обозначаемые sc-элементами, но и сами эти sc-элементы.
	
\begin{SCn}
\begin{scnsubdividing}
	\scnitem{бинарная связь}
	\scnitem{небинарная связь}
\end{scnsubdividing}

\begin{scnsubdividing}
	\scnitem{неориентированная связь}
	\scnitem{ориентированная связь}
\end{scnsubdividing}
	
\scnheader{бинарная связь}
\begin{scnsubdividing}
	\scnitem{sc-коннекторь}
	\scnitem{неатомарная бинарная связь}
\end{scnsubdividing}
\end{SCn}

Данное разбиение осуществляется на основе синтаксического признака, а не семантического, поскольку каждый \textit{sc-коннектор} может быть записан в памяти при помощи семантически эквивалентной конструкции, содержащей знак самой связи и пары принадлежности, ведущие к ее элементам, уточненные, при необходимости ролевыми отношениями.
	
\begin{SCn}
\scnheader{sc-коннектор}
\scnidtf{атомарная бинарная связь}
\end{SCn}

Каждый \textbf{\textit{sc-коннектор}} представлен в \textit{sc-памяти} одним \textit{sc-элементом} и семантически эквивалентен конструкции, содержащей знак некоторой \textit{бинарной связи} и пары принадлежности, ведущие к элементам этой связи, уточненные, при необходимости ролевыми отношениями.

\begin{SCn}
Такая конструкция может быть обозначена \textbf{\textit{sc-коннектором}} только в случае, когда роли компонентов соответствующей бинарной связи указываются только при помощи \textit{числовых атрибутов 1\scnrolesign} и \textit{2\scnrolesign} или не уточняются вообще.
\end{SCn}
	
\begin{SCn}
\scnheader{неатомарная бинарная связь}
\end{SCn}

\begin{SCn}
\textbf{\textit{неатомарная бинарная связь}} -- \textit{бинарная связь}, роли компонентов которой не могут быть заданы только при помощи \textit{ролевых отношений 1\scnrolesign} и \textit{2\scnrolesign}, или не заданы совсем, а требуют дополнительного уточнения при помощи более частных ролевых отношений.
\end{SCn}

\begin{SCn}
\scnheader{небинарная связь}
\textbf{\textit{небинарная связь}} -- связь, имеющая больше двух элементов.
	
	\scnheader{неориентированная связь}
	\scnsuperset{неориентированное множество}
	\scnexplanation{\textbf{\textit{неориентированная связь}} -- связь, все элементы которой имеют одинаковые роли (при этом соответствующее ролевое отношение, как правило, явно не указывается).}
	
	\scnheader{ориентированная связь}
	\scnsuperset{кортеж}
	\scnexplanation{\textbf{\textit{ориентированная связь}} -- связь, в которой с помощью ролевых отношений, указываются роли компонентов этой связи.}
	
	\scnheader{отношение}
	\scnidtf{класс связей}
	\scnidtf{класс sc-связок}
	\scnidtf{множество отношений}
	\scnidtf{Множество всевозможных отношений}
	\scntext{определение}{\textbf{\textit{отношение}}, \textit{заданное на множестве M} -- это подмножество \textit{декартового произведения} этого множества самого на себя некоторое количество раз}
\end{SCn}
	
В более широком смысле \textbf{\textit{отношение}} -- это математическая структура, которая формально определяет свойства различных объектов и их взаимосвязи.

\begin{SCn}
\begin{scnsubdividing}
	\scnitem{класс равномощных связок}
	\scnitem{класс связок разной мощности}
\end{scnsubdividing}
\begin{scnsubdividing}
	\scnitem{бинарное отношение}
	\scnitem{небинарное отношение}
\end{scnsubdividing}
\begin{scnsubdividing}
	\scnitem{ориентированное отношение}
	\scnitem{неориентированное отношение}
\end{scnsubdividing}
\begin{scnsubdividing}
	\scnitem{ролевое отношение}
	\scnitem{неролевое отношение}
\end{scnsubdividing}

\scnheader{класс равномощных связок}
\scnidtf{класс связок фиксированной арности}
\scnidtf{отношение, обладающее свойством арности}
\scnsuperset{унарное отношение}
\scnsuperset{бинарное отношение}
\scnsuperset{тернарное отношение}
\scntext{определение}{\textbf{\textit{класс равномощных связок}} -- класс связок, имеющих одинаковую мощность.}
	
\scnheader{класс связок разной мощности}
\scnidtf{отношение нефиксированной арности}
\scnsubset{небинарное отношение}
\scntext{определение}{\textbf{\textit{класс связок разной мощности}} -- класс связок, имеющих разную мощность.}
	
\scnheader{унарное отношение}
\scnidtf{отношение арности один}
\scnidtf{одноместное отношение}
\scnidtf{множество синглетонов}
\scntext{определение}{\textbf{\textit{унарное отношение}} -- это множество таких отношений на множестве M, являющихся любым подмножеством множества M.}
	
\scnheader{бинарное отношение}
\scnidtf{отношение арности два}
\scnidtf{двухместное отношение}
\scnsuperset{квазибинарное отношение}
\scnsuperset{отношение порядка}
\scnsuperset{отношение толерантности}
\begin{scnsubdividing}
	\scnitem{рефлексивное отношение}
	\scnitem{антирефлексивное отношение}
	\scnitem{частично рефлексивное отношение}
\end{scnsubdividing}
\begin{scnsubdividing}
	\scnitem{симметричное отношение}
	\scnitem{антисимметричное отношение}
	\scnitem{частично симметричное отношение}
\end{scnsubdividing}
\begin{scnsubdividing}
	\scnitem{транзитивное отношение}
	\scnitem{антитранзитивное отношение}
	\scnitem{частично транзитивное отношение}
\end{scnsubdividing}
\begin{scnsubdividing}
	\scnitem{ролевое отношение}
	\scnitem{неролевое бинарное отношение}
\end{scnsubdividing}

\scntext{определение}{\textbf{\textit{бинарное отношение}} -- это множество таких отношений на множестве \textbf{\textit{M}}, являющихся подмножеством \textit{декартова произведения} множества \textbf{\textit{M}}.}
\end{SCn}

Если \textbf{\textit{бинарное отношение R}} задано на \textit{множестве} \textbf{\textit{М}} и два элемента этого множества \textbf{\textit{a}} и \textbf{\textit{b}} связаны данным отношением, то будем обозначать такую связь как \textbf{\textit{aRb}}.
	
\begin{SCn}
\scnheader{квазибинарное отношение}
\scnexplanation{\textbf{\textit{квазибинарное отношение}} -- множество ориентированных пар, первые компоненты которых являются связками.}
\end{SCn}

Таким образом, \textit{sc-дуги}, принадлежащие \textbf{\textit{квазибинарным отношениям}}, всегда выходят из связок.

\begin{SCn}
\scntext{sc-утверждение}{В область определения квазибинарного отношения будем включать:
\begin{scnitemize}
	\item вторые компоненты ориентированных пар, принадлежащих этому отношению;
	\item элементы первых компонентов ориентированных пар, принадлежащих этому отношению;
	\item других элементов область определения квазибинарного отношения не содержит.
\end{scnitemize}

}
	
\scnheader{небинарное отношение}
\scnexplanation{\textbf{\textit{небинарное отношение}} -- это множество отношений, хотя бы одна из связок каждого из которых имеет значение мощности больше двух.}
	
\scnheader{ориентированное отношение}
\scntext{определение}{\textbf{\textit{ориентированное отношение}} -- это множество таких отношений, каждая связка которых является кортежем.}
	
\scnheader{неориентированное отношение}
\scntext{определение}{\textbf{\textit{неориентированное отношение}} -- это множество таких отношений, каждая связка которых является неориентированным множеством.}
	
	% \scnheader{рефлексивное отношение}
	% \scntext{определение}{\textbf{\textit{рефлексивное отношение}} -- это \textit{бинарное отношение}, любая пара которого есть канторовское множество.}
	
	% \scnheader{антирефлексивное отношение}
	% \scntext{определение}{\textbf{\textit{антирефлексивное отношение R}} на \textit{множестве} \textbf{\textit{A}} -- это \textit{бинарное отношение}, в котором все элементы множества \textbf{\textit{A}} не находятся в отношении \textbf{\textit{R}} к самому себе.}
	
	% \scnheader{частично рефлексивное отношение}
	% \scntext{определение}{\textbf{\textit{частично рефлексивное отношение R}} на \textit{множестве} \textbf{\textit{A}} -- это \textit{бинарное отношение},  в котором хотя бы один (но не все) элемент множества \textbf{\textit{A}} находится в отношении \textbf{\textit{R}} к самому себе.}
	
	% \scnheader{симметричное отношение}
	% \scntext{определение}{\textbf{\textit{симметричное отношение R}} на \textit{множестве} \textbf{\textit{A}} -- это \textit{бинарное отношение}, в котором для каждой пары элементов \textbf{\textit{а}} и \textbf{\textit{b}} этого множества выполнение отношения \textbf{\textit{aRb}} влечёт выполнение \textbf{\textit{bRa}}.}
	
	% \scnheader{антисимметричное отношение}
	% \scntext{определение}{\textbf{\textit{антисимметричное отношение R}} на \textit{множестве} \textbf{\textit{A}} -- это \textit{бинарное отношение}, в котором для каждой пары элементов \textbf{\textit{а}} и \textbf{\textit{b}} этого множества выполнение отношений \textbf{\textit{aRb}} и \textbf{\textit{bRa}} влечёт равенство \textbf{\textit{a}} и \textbf{\textit{b}}.}
	
	% \scnheader{частично симметричное отношение}
	% \scntext{определение}{\textbf{\textit{частично симметричное отношение R}} на \textit{множестве} \textbf{\textit{A}} -- это \textit{бинарное отношение}, в котором для каждой пары элементов \textbf{\textit{а}} и \textbf{\textit{b}} (но не для всех таких пар) этого множества выполнение отношения \textbf{\textit{aRb}} влечёт выполнение \textbf{\textit{bRa}}.}
	
	% \scnheader{транзитивное отношение}
	% \scntext{определение}{\textbf{\textit{транзитивное отношение R}} на \textit{множестве} \textbf{\textit{A}} -- это \textit{бинарное отношение}, в котором для любых трёх элементов этого множества \textbf{\textit{a, b, c}} выполнение отношений \textbf{\textit{aRb}} и \textbf{\textit{bRc}} влечёт выполнение отношения \textbf{\textit{aRc}}.}
	
	% \scnheader{антитранзитивное отношение}
	% \scntext{определение}{\textbf{\textit{антитранзитивное отношение R}} на \textit{множестве} \textbf{\textit{A}} -- это \textit{бинарное отношение}, в котором для любых трёх элементов этого множества \textbf{\textit{a, b, c}} выполнение отношений \textbf{\textit{aRb}} и \textbf{\textit{bRc}} не влечёт выполнение отношения \textbf{\textit{aRc}}.}
	
	% \scnheader{частично транзитивное отношение}
	% \scntext{определение}{\textbf{\textit{частично транзитивное отношение R}} на \textit{множестве} \textbf{\textit{A}} -- это \textit{бинарное отношение}, в котором для каждых трёх элементов этого множества \textbf{\textit{a, b, c}} (но не для всех таких троек) выполнение отношений \textbf{\textit{aRb}} и \textbf{\textit{bRc}} влечёт выполнение отношения \textbf{\textit{aRc}}.}
	
	\scnheader{связанное отношение*}
	\scniselement{бинарное отношение}
	\scntext{определение}{\textbf{\textit{связанное отношение* R}} на \textit{множестве} \textbf{\textit{A}} -- это \textit{бинарное отношение}, в котором для каждой пары элементов \textbf{\textit{а}} и \textbf{\textit{b}} этого множества выполняется одно из двух отношений: \textbf{\textit{aRb}} или \textbf{\textit{bRa}}.}
	
	\scnheader{отношение порядка}
	\begin{scnsubdividing}
		\scnitem{отношение строгого порядка}
		\scnitem{отношение нестрогого порядка}
	\end{scnsubdividing}
	
	\scntext{определение}{\textbf{\textit{отношение порядка}} -- это \textit{бинарное отношение}, обладающее свойством транзитивности и антисимметричности.}
	
	\scnheader{отношение строгого порядка}
	\scntext{определение}{\textbf{\textit{отношение строгого порядка}} -- это \textit{отношение порядка}, обладающее свойством антирефлексивности.}
	
	\scnheader{отношение нестрогого порядка}
	\scntext{определение}{\textbf{\textit{отношение нестрогого порядка}} -- это \textit{отношение порядка}, обладающее свойством рефлексивности.}
	
	\scnheader{отношение толерантности}
	\scntext{определение}{\textbf{\textit{отношение толерантности}} -- это \textit{бинарное отношение}, принадлежащее классам \textit{симметричное отношение} и \textit{рефлексивное отношение}.}
	
	\scnheader{отношение эквивалентности}
	\scnidtf{максимальное семейство отношений эквивалентности}
	\scnsubset{отношение толерантности}
	\scntext{определение}{\textbf{\textit{отношение эквивалентности}} -- это \textit{отношение толерантности}, принадлежащее классу \textit{транзитивных отношений}}
	\scntext{примечание}{Каждое отношение эквивалентности уточняет то, что мы считаем эквивалентными сущностями, т.е. то, на какие сходства этих сущностей мы обращаем внимание и какие их отличия мы игнорируем (не учитываем).}
	
	\scnheader{ролевое отношение}
	\scnidtf{атрибут}
	\scnidtf{атрибутивное отношение}
	\scnidtf{отношение, которое задает роль элементов в рамках некоторого множества}
	\scnidtf{отношение, являющееся подмножеством отношения принадлежности}
	\scnrelto{семейство подмножеств}{принадлежность*}
	\scnsubset{бинарное отношение}
	\scnsuperset{числовой атрибут}
	\scnexplanation{\textbf{\textit{ролевое отношение}} -- это отношение, являющееся подмножеством отношения принадлежности.}
	\scntext{правило идентификации экземпляров}{В конце каждого \textit{идентификатора}, соответствующего экземплярам класса \textbf{\textit{ролевое отношение}}, не являющегося системным, ставится знак ``\scnrolesign''.
	
	Например:\\
	\textit{ключевой экземпляр\scnrolesign}
	
	Из-за ограничений в разрешенном алфавите символов, в системном идентификаторе не может быть использовать знак ``\scnrolesign'', поэтому в начале каждого \textit{системного идентификатора}, соответствующего экземплярам класса \textbf{\textit{ролевое отношение}} ставится префикс ``rrel\_''.
	
	Например:\\
	\textit{rrel\_key\_sc\_element}}
	
	\scnheader{числовой атрибут}
	\scnidtf{порядковый номер}
	\scnidtf{номер компонента ориентированной связки}
	\scnhaselement{\textbf{1\scnrolesign}; \textbf{2\scnrolesign}; \textbf{3\scnrolesign}; \textbf{4\scnrolesign}; \textbf{5\scnrolesign}; \textbf{6\scnrolesign}; \textbf{7\scnrolesign}; \textbf{8\scnrolesign}; \textbf{9\scnrolesign}; \textbf{10\scnrolesign}}
	\scnexplanation{\textbf{\textit{числовой атрибут}} -- \textit{ролевое отношение}, задающее порядковый номер элемента некоторой ориентированной связки, не уточняя при этом семантику такой принадлежности. Во многих случаях бывает достаточно использовать числовые атрибуты, чтобы различать компоненты связки, семантика каждого из которых дополнительно оговаривается, например, при определении отношения, которому данная связка принадлежит.}
	
	\scnheader{неролевое отношение}
	\begin{scnsubdividing}
		\scnitem{небинарное отношение}
		\scnitem{неролевое бинарное отношение}
	\end{scnsubdividing}
	\scnexplanation{\textbf{\textit{неролевое отношение}} -- отношение, не являющееся подмножеством отношения принадлежности.}
	\scntext{правило идентификации экземпляров}{В конце каждого \textit{идентификатора}, соответствующего экземплярам класса \textbf{\textit{неролевое отношение}}, не являющегося системным, ставится знак ``*''.
	
	Например:\\
	\textit{включение*}
	
	Из-за ограничений в разрешенном алфавите символов, в системном идентификаторе не может быть использовать знак ``*'', поэтому в начале каждого \textit{системного идентификатора}, соответствующего экземплярам класса \textbf{\textit{неролевое отношение}} ставится префикс ``nrel\_''.
	
	Например:\\
	\textit{nrel\_inclusion}}
	
	\scnheader{неролевое бинарное отношение}
	\scnexplanation{\textbf{\textit{неролевое бинарное отношение}} -- \textit{бинарное отношение}, не являющееся \textit{ролевым отношением}.}
	
	\scnheader{арность}
	\scnidtf{арность отношения}
	\scniselement{параметр}
	\scnexplanation{\textbf{\textit{арность}} -- это параметр, каждый элемент которого представляет собой класс \textit{отношений}, каждая связка которых имеет одинаковую \textit{мощность}. Значение данного \textit{параметра} совпадает со значением \textit{мощности} каждой из таких связок.}
	\scnrelfrom{описание примера}
	
	
	\scnheader{область определения*}
	\scnidtf{область определения отношения*}
	\scniselement{бинарное отношение}
	\scnexplanation{\textbf{\textit{область определения*}} -- это \textit{бинарное отношение}, связывающее отношение со множеством, являющимся его областью определения.
	
	Областью определения отношения будем называть результат теоретико-множественного объединения всех связок этого отношения, или, другими словами, результат теоретико-множественного объединения всех множеств, являющихся доменами данного отношения.}
%	\scnrelfrom{описание примера}{
	% \scnfilescg{figures/sd_relations/domain.png}}
	
	\scnheader{атрибут отношения*}
	\scnidtf{ролевой атрибут, используемый в связках заданного отношения*}
	\scniselement{бинарное отношение}
	\scnexplanation{\textbf{\textit{атрибут отношения*}} -- это \textit{бинарное отношение}, связывающее заданное отношение с \textit{ролевым отношением}, используемым в данном отношении для уточнения роли того или иного элемента связок данного отношения.}
%	\scnrelfrom{описание примера}{
	% \scnfilescg{figures/sd_relations/relationshipAttribute.png}}
	
	\scnheader{домен*}
	\scnidtf{домен отношения по заданному атрибуту*}
	\scniselement{бинарное отношение}
	\scnexplanation{\textbf{\textit{домен*}} -- это \textit{бинарное отношение}, связывающее связку отношения \textit{атрибут отношения*} со множеством, являющимся доменом заданного отношения по заданному атрибуту. Множество \textbf{\textit{di}} является доменом отношения \textbf{\textit{ri}} по атрибуту \textbf{\textit{ai}} в том и только том случае, если элементами этого множества являются все те и только те элементы связок отношения \textbf{\textit{ri}}, которые имеют в рамках этих связок атрибут \textbf{\textit{ai}}.}
%	\scnrelfrom{описание примера}{
	% \scnfilescg{figures/sd_relations/domen.png}}
	
	
	%% Зачем? Мы их используем?
	% \scnheader{первый домен*}
	% \scniselement{бинарное отношение}
	% \scntext{определение}{\textbf{\textit{первый домен*}} -- это \textit{бинарное отношение}, связывающее отношение с множеством, являющимся доменом по атрибуту \textbf{\textit {1\scnrolesign}} данного отношения.}
	% \scnrelfrom{описание примера}{
	% \scnfilescg{figures/sd_relations/firstDomen.png}}
	
	% \scnheader{второй домен*}
	% \scniselement{бинарное отношение}
	% \scntext{определение}{\textbf{\textit{второй домен*}} -- это \textit{бинарное отношение}, связывающее отношение с множеством, являющимся доменом по атрибуту \textbf{\textit{2\scnrolesign}} данного отношения.}
	% \scnrelfrom{описание примера}{
	% \scnfilescg{figures/sd_relations/secondDomen.png}}
	
	\scnheader{композиция отношений*}
	\scniselement{квазибинарное отношение}
	\scntext{определение}{\textbf{\textit{композиция отношений*}} -- это \textit{квазибинарное отношение}, связывающее два бинарных отношения с отношением, являющимся их композицией. Под композицией бинарных отношений \textbf{\textit{R}} и \textbf{\textit{S}} будем понимать множество $\{(x, y) | \exists z(xSz \wedge zRy)\}$}
%	\scnrelfrom{описание примера}{
%	\scnfilescg{figures/sd_relations/relationshipComposition.png}}
	
	% \scnheader{фактор-множество*}
	% \scnidtf{быть фактор-множеством*}
	% \scnidtf{множество всевозможных максимальных множеств из попарно эквивалентных элементов*}
	% \scnidtf{множество всевозможных классов эквивалентности для заданного отношения эквивалентности*}
	% \scniselement{бинарное отношение}
	% \scntext{определение}{\textbf{\textit{фактор множество*}} -- это бинарное ориентированное отношение, каждая связка которого связывает некоторое отношение эквивалентности со множеством всех соответствующих этому отношению классов эквивалентности. Каждый такой класс представляет собой максимальное множество сущностей, каждая пара которых принадлежит указанному выше отношению эквивалентности.}
	% \scnrelfrom{описание примера}{
	% \scnfilescg{figures/sd_relations/factor_set.png}}
	
	\scnheader{метаотношение}
	\scntext{определение}{метаотношение -- это \textit{отношение}, в каждой связке которого есть по крайней мере один компонент, являющийся знаком некоторого \textit{отношения}.}
	
	\scnheader{отношение декомпозиции}
	\scnhaselement{разбиение*}
	\scnhaselement{декомпозиция раздела*}
	\scnhaselement{декомпозиция абстрактного объекта*}
	
	% \scnheader{отношение интеграции}
	% \scnhaselement{объединение*}
	
	\scnheader{соответствие*}
	\scnidtf{наличие соответствия*}
	\scniselement{бинарное отношение}
	\begin{scnsubdividing}
		\scnitem{соответствие между непересекающимися множествами*}
		\scnitem{соответствие между строго пересекающимися множествами*}
		\scnitem{соответствие, область отправления и область прибытия которого совпадают*}
	\end{scnsubdividing}
\begin{scnsubdividing}
	\scnitem{всюду определенное соответствие*}
	\scnitem{частично определенное соответствие*}
\end{scnsubdividing}
\begin{scnsubdividing}
	\scnitem{сюръекция*}
	\scnitem{несюръективное соответствие*}
\end{scnsubdividing}
\begin{scnsubdividing}
	\scnitem{однозначное соответствие*}
	\scnitem{неоднозначное соответствие*}
\end{scnsubdividing}
	\scntext{определение}{\textbf{\textit{соответствие*}} -- \textit{бинарное ориентированное отношение}, каждая пара которого связывает два множества и указывает на наличие некоторого отношения, связывающего элементы этих двух множеств.}
%	\scnrelfrom{описание примера}{
%	\scnfilescg{figures/sd_relations/conformity.png}}
	
	\scnheader{отношение соответствия*}
	\scniselement{бинарное отношение}
	\scntext{определение}{\textbf{\textit{отношение соответствия*}} -- \textit{бинарное отношение}, связывающее ориентированную пару множеств, на которых задано \textit{соответствие*} и некоторое подмножество \textit{декартова произведения*} этих \textit{множеств}.}
%	\scnrelfrom{описание примера}{
%	\scnfilescg{figures/sd_relations/relationshipConformity.png}}
	
	\scnheader{область отправления\scnrolesign}
	\scnidtf{область отправления соответствия\scnrolesign}
	\scnidtf{область определения соответствия\scnrolesign}
	\scnidtf{первый компонент пары в отношении соответствия\scnrolesign}
	\scniselement{ролевое отношение}
	\scntext{определение}{\textbf{\textit{область отправления\scnrolesign}} -- \textit{ролевое отношение}, указывающее на первый компонент пары в рамках отношения \textit{соответствие*}.}
%	\scnrelfrom{описание примера}{
%	\scnfilescg{figures/sd_relations/departureArea.png}}
	
	\scnheader{область прибытия\scnrolesign}
	\scnidtf{область прибытия соответствия\scnrolesign}
	\scnidtf{область значений соответствия\scnrolesign}
	\scniselement{ролевое отношение}
	\scntext{определение}{\textbf{\textit{область прибытия\scnrolesign}} -- \textit{ролевое отношение}, указывающее на второй компонент пары в рамках отношения \textit{соответствие*}.}
%	\scnrelfrom{описание примера}{
%	\scnfilescg{figures/sd_relations/arrivalArea.png}}
	
	\scnheader{образ\scnrolesign}
	\scnidtf{образ соответствия\scnrolesign}
	\scniselement{ролевое отношение}
	\scntext{определение}{\textbf{\textit{образ\scnrolesign}} -- \textit{ролевое отношение}, указывающее на второй компонент каждой пары в рамках множества пар, которое является вторым компонентом \textit{отношения соответствия*}.}
%	\scnrelfrom{описание примера}{
%	\scnfilescg{figures/sd_relations/form.png}}
	
	\scnheader{прообраз\scnrolesign}
	\scnidtf{прообраз соответствия\scnrolesign}
	\scniselement{ролевое отношение}
	\scntext{определение}{\textbf{\textit{прообраз\scnrolesign}} -- \textit{ролевое отношение}, указывающее на первый компонент каждой пары в рамках множества пар, которое является первым компонентом \textit{отношения соответствия*}.}
%	\scnrelfrom{описание примера}{
%	\scnfilescg{figures/sd_relations/prototype.png}}
	
	\scnheader{всюду определенное соответствие*}
	\scnidtf{полное соответствие*}
	\scnidtf{наличие всюду определенного соответствия*}
	\scntext{определение}{\textbf{\textit{всюду определенное соответствие*}} -- это \textit{соответствие*}, при котором существует \textit{образ\scnrolesign} для каждого элемента \textit{области отправления\scnrolesign} данного \textit{соответствия*}.}
%	\scnrelfrom{описание примера}{
%	\scnfilescg{figures/sd_relations/surjection.png}}
%	\scnrelfrom{изображение}{
%	\scnfileimage{\includegraphics[width=0.5\linewidth]{figures/sd_relations/surjection2.png}}}
	
	
	\scnheader{частично определенное соответствие*}
	\scnidtf{наличие частично определенного соответствия*}
	\scntext{определение}{\textbf{\textit{частично определенное соответствие*}} -- это \textit{соответствие*}, при котором существует \textit{образ\scnrolesign} для некоторых, но не всех элементов \textit{области отправления\scnrolesign} данного \textit{соответствия*}.}
%	\scnrelfrom{описание примера}{
%	\scnfilescg{figures/sd_relations/partiallyDefinedConformity.png}}
%	\scnrelfrom{изображение}{
%	\scnfileimage{\includegraphics[width=0.5\linewidth]{figures/sd_relations/partiallySurjection.png}}}
	
	\scnheader{сюръективное соответствие*}
	\scnidtf{наличие сюръективного соответствия*}
	\scnidtf{сюръекция*}
	\scntext{определение}{\textbf{\textit{сюръективное соответствие*}} -- это \textit{соответствие*}, при котором существует \textit{прообраз\scnrolesign} для каждого элемента \textit{области прибытия\scnrolesign} данного \textit{соответствия*}.}
%	\scnrelfrom{описание примера}{
%	\scnfilescg{figures/sd_relations/surjectiveConformity.png}}
%	\scnrelfrom{изображение}{
%	\scnfileimage{\includegraphics[width=0.5\linewidth]{figures/sd_relations/surjectiveConformity2.png}}}
	
	\scnheader{несюръективное соответствие*}
	\scnidtf{наличие несюръективного соответствия*}
	\scntext{определение}{\textbf{\textit{несюръективное соответствие*}} -- это \textit{соответствие*}, при котором не для каждого элемента \textit{области прибытия\scnrolesign} данного \textit{соответствия*} существует \textit{прообраз\scnrolesign}.}
%	\scnrelfrom{описание примера}{
%	\scnfilescg{figures/sd_relations/nonSurjectiveConformity.png}}
%	\scnrelfrom{изображение}{
%	\scnfileimage{\includegraphics[width=0.5\linewidth]{figures/sd_relations/nonSurjectiveConformity2.png}}}
	
	\scnheader{однозначное соответствие*}
	\scnidtf{наличие однозначного соответствия*}
	\scnidtf{функциональное соответветствие*}
	\scnidtf{функция*}
	\scntext{определение}{\textbf{\textit{однозначное соответствие*}} -- это \textit{соответствие*}, при котором каждому элементу из \textit{области отправления\scnrolesign} соответствия ставится не более, чем один элемент из \textit{области прибытия\scnrolesign} соответствия.}
%	\scnrelfrom{описание примера}{
%	\scnfilescg{figures/sd_relations/singleConformity.png}}
%	\scnrelfrom{изображение}{
%	\scnfileimage{\includegraphics[width=0.5\linewidth]{figures/sd_relations/singleConformity2.png}}}
	

	%% Подумать что оставить
	% \scnheader{обратное соответствие*}
	% \scniselement{бинарное отношение}
	% \scnrelfrom{область определения}{соответствие*}
	% \scntext{определение}{\textbf{\textit{обратное соответствие*}} -- \textit{бинарное отношение}, связывающее два \textit{соответствия*}, при этом выполняются следующие условия:
	% \begin{scnitemize}
	% 	\item \textit{область отправления\scnrolesign} первого соответствия является \textit{областью прибытия\scnrolesign} второго;
	% 	\item \textit{область прибытия\scnrolesign} первого соответствия является \textit{областью отправления\scnrolesign} второго;
	% 	\item для каждой пары, входящей в состав отношения первого соответствия, существует пара, входящая в состав отношения второго соответствия, при этом \textit{образ\scnrolesign} и \textit{прообраз\scnrolesign} в рамках первой указанной пары являются соответственно \textit{прообразом\scnrolesign} и \textit{образом\scnrolesign} в рамках второй.
	% \end{scnitemize}
	% }
	
	% \scnheader{обратимое соответствие*}
	% \scnsubset{однозначное соответствие*}
	% \scntext{определение}{\textbf{\textit{обратимое соответствие*}} -- такое \textit{однозначное соответствие*}, для которого \textit{обратное соответствие*} также является \textit{однозначным соответствием*}.}
	
	% \newpage
	% \scnheader{неоднозначное соответствие*}
	% \scntext{определение}{\textbf{\textit{неоднозначное соответствие*}} -- это \textit{соответствие*}, при котором хотя бы одному элементу из \textit{области отправления\scnrolesign} соответствия ставится более, чем один элемент из \textit{области прибытия\scnrolesign} соответствия.}
	% \scnrelfrom{описание примера}{
	% \scnfilescg{figures/sd_relations/nonSingleConformity.png}}
	% \scnrelfrom{изображение}{
	% \scnfileimage{\includegraphics[width=0.5\linewidth]{figures/sd_relations/nonSingleConformity2.png}}}
	
	% \scnheader{инъективное соответствие*}
	% \scnidtf{инъекция*}
	% \scnsubset{однозначное соответствие*}
	% \scntext{определение}{\textbf{\textit{инъективное соответствие*}} -- это \textit{соответствие*}, при котором разным элементам из \textit{области отправления\scnrolesign} соответствия всегда соответствуют разные элементы из \textit{области прибытия\scnrolesign} соответствия и наоборот.}
	% \scnrelfrom{описание примера}{
	% \scnfilescg{figures/sd_relations/injectiveConformity.png}}
	% \scnrelfrom{изображение}{
	% \scnfileimage{\includegraphics[width=0.5\linewidth]{figures/sd_relations/injectiveConformity2.png}}}
	
	% \scnheader{взаимно однозначное соответствие*}
	% \scnidtf{биекция*}
	% \scnsubset{всюду определенное соответствие*}
	% \scnsubset{сюръективное соответствие*}
	% \scnsubset{инъективное соответствие*}
	% \scntext{определение}{\textbf{\textit{взаимно однозначное соответствие*}} -- это \textit{инъективное соответствие*}, являющееся всюду определенным и сюръективным.}
	% \scnrelfrom{описание примера}{
	% \scnfilescg{figures/sd_relations/bijectiveConformity.png}}
	% \scnrelfrom{изображение}{
	% \scnfileimage{\includegraphics[width=0.5\linewidth]{figures/sd_relations/bijectiveConformity2.png}}}
	
	
	\scnheader{множество сочетаний*}
	\scnidtf{множество всевозможных сочетаний*}
	\scnidtf{множество всевозможных сочетаний заданной арности из элементов заданного множества*}
	\scnidtf{множество всех неориентированных связок заданной арности*}
	\scnidtf{множество всех подмножеств заданной мощности*}
	\scnidtf{семейство всевозможных сочетаний*}
	\scntext{определение}{\textbf{\textit{множество сочетаний*}} -- \textit{отношение}, связывающее некоторое множество и семейство всевозможных множеств, имеющих значение мощности, меньше либо равное мощности исходного множества и состоящих из тех же элементов, что и это множество.}
	\scntext{утверждение}{Мощность \textbf{\textit{множества сочетаний*}} может быть вычислена как \textbf{n!/(k!(n-k)!)}, где \textbf{\textit{n}} -- мощность исходного множества, \textbf{\textit{k}} -- мощность элементов множества сочетаний.}
%	\scnrelfrom{описание примера}{
%	\scnfilescg{figures/sd_relations/setsOfCombinations.png}
%	\scnexplanation{Для Множества \textbf{\textit{Si}} представлено множество сочетаний по 2 элемента.}}
	
	\scnheader{множество размещений*}
	\scntext{определение}{\textbf{\textit{множество размещений*}} -- \textit{отношение}, связывающее некоторое множество и семейство всевозможных кортежей, имеющих значение мощности, меньше либо равное мощности исходного множества и состоящих из тех же элементов, что и это множество.}
	\scntext{утверждение}{Мощность \textbf{\textit{множества размещений*}} может быть вычислена как \textbf{n!/(n-k)!}, где \textbf{\textit{n}} -- мощность исходного множества, \textbf{\textit{k}} -- мощность элементов множества сочетаний.}
%	\scnrelfrom{описание примера}{
%	\scnfilescg{figures/sd_relations/setsOfPlacements.png}
%	\scnexplanation{Для Множества \textbf{\textit{Si}} представлено множество размещений по 2 элемента.}}
	
	\scnheader{множество перестановок*}
	\scnsubset{множество размещений*}
	\scntext{определение}{\textbf{\textit{множество перестановок*}} -- \textit{отношение}, связывающее некоторое множество и семейство всевозможных кортежей, равномощных исходному множеству и состоящих из тех же элементов, что и это множество.}
	\scntext{утверждение}{Мощность \textbf{\textit{множества перестановок*}} может быть вычислена как \textbf{n!}, где \textbf{\textit{n}} -- мощность исходного множества.}
%	\scnrelfrom{описание примера}{
%	\scnfilescg{figures/sd_relations/setsOfPermutations.png}
%	\scnexplanation{Для Множества \textbf{\textit{Si}} представлено его множество перестановок.}}
\end{SCn}

%%%%%%%%%%%%%%%%%

\section{Формальная онтология параметров, величин и шкал}
\label{sec_top_ontologies_params}

\begin{SCn}
\scnheader{Предметная область параметров, величин и шкал}
\scnidtf{Предметная область параметров и классов эквивалентности, являющихся их элементами (значениями, величинами)}
\scniselement{предметная область}
\scnsdmainclasssingle{параметр}
%\scnsdclass{измеряемый параметр;неизмеряемый параметр;уровень класса эквивалентности;величина;точная величина;неточная величина;интервальная величина;параметрическая модель;измерение с фиксированной единицей измерения ;измерение по шкале;арифметическое выражение на величинах;арифметическая операция на величинах;действие. измерение;задача. измерение}
%\scnsdrelation{область определения параметра*;эталон\scnrolesign;измерение*;точность*;единица измерения*;нулевая отметка*;единичная отметка*;сумма величин*;произведение величин*;возведение величин в степень*;большая величина*;равенство величин*;большая или равная величина*}
	
\scnheader{параметр}
\scnidtf{характеристика}
\scnidtf{свойство}
%\scnidtf{признак}
\scnidtf{класс классов}
%\scnidtf{измеряемое свойство}
%\scnidtf{признак классификации или покрытия некоторого класса сущностей}
%\scnidtf{признак разбиения или покрытия некоторого класса сущностей}
%\scnidtf{семейство множеств, разбивающих или покрывающих некоторый класс сущностей}
%\scnidtf{семейство классов сущностей, обладающих одинаковым соответствующим свойством}
%\scnidtf{фактор-множество, соответствующее некоторому отношению эквивалентности, или аналог фактор-множества, соответствующий некоторому отношению толерантности}
\begin{scnsubdividing}
	\scnitem{измеряемый параметр}
	\scnitem{неизмеряемый параметр}
\end{scnsubdividing}
\end{SCn}
%\scnsuperset{ориентированный параметр}
Каждый \textbf{\textit{параметр}} представляет собой класс, являющийся семейством всевозможных классов эквивалентности или толерантности, задаваемых либо \textit{отношением эквивалентности}, либо \textit{отношением толерантности} (симметричным, рефлексивным, но частично транзитивным). Так, например, элементами (значениями, величинами) \textbf{\textit{параметра}} \textit{длина} являются либо классы эквивалентности, задаваемые отношением эквивалентности ``иметь точно одинаковую длину*'', либо классы толерантности, задаваемые отношением вида ``иметь приблизительно одинаковую длину с указываемой точностью*'', либо интервальные классы, задаваемые бинарными отношениями вида ``иметь длину, находящуюся в одном и том же указываемом интервале*'' (например, от 1 метра до 2 метров).

Примерами параметров как отношений эквивалентности являются:

\begin{textitemize}
		\item равновеликость геометрических фигур (по длине, площади, объему -- в зависимости от размерности этих фигур);
		\item иметь одинаковый цвет (быть эквивалентными по цвету);
		\item эквивалентность, по вкусу, запаху, твердости и т.д.
\end{textitemize}
		
Заметим, что среди элементов (значений, величин) параметра могут встречаться пересекающиеся множества (классы), но объединение всех элементов каждого параметра есть не что иное, как класс всевозможных сущностей, обладающих этим параметром (свойством, характеристикой). Например, класс всех сущностей, имеющих длину, класс всех сущностей, обладающих цветом.
		
Каждый конкретный параметр (характеристика), т.е. каждый элемент класса всевозможных параметров (характеристик) есть, по сути, признак классификации сущностей, обладающих это характеристикой, по принципу эквивалентности (одинаковости значения) этой характеристики. Например, параметр \textit{цвет} разбивает множество сущностей имеющих цвет на классы, каждый из которых включает в себя сущности, имеющие одинаковый цвет. Параметр может разбиваться на классы для уточнения некоторого свойства, например элементами параметра цвет будут классы, соответствующие конкретным цветам (синий, красный и т.д.), в свою очередь каждый конкретный цвет может включать более частные классы, уточняющие данное свойство, например, темно-синий, светло-красный и т.д.
		
Другими словами, каждому множеству сущностей может ставиться в соответствие набор (семейство) параметров (параметрическое пространство), которыми обладают сущности этого множества -- аналог семейства отношений, определенных (заданных) на этом множестве. Часто бывает важно построить такое параметрическое пространство, "точки"{} которого взаимнооднозначно соответствуют параметризуемым сущностям (например, набор параметров, позволяющих однозначно идентифицировать, установить личность каждого человека). 
		
Таким образом, для каждого используемого элемента (значения) какого-либо параметра, необходимо явно указывать спецификацию этого значения (точное значение, неточное значение, интервальное значение, точность, интервал).
		
%	\scnrelfrom{описание примера}{
%		\scnfilescg{figures/sd_parameters_and_quantities/parameterDescription.png}
%	}
	
\begin{SCn}
\scnheader{область определения параметра*}
\scnidtf{множество всех тех и только тех сущностей, которые являются компонентами значений заданного параметра*}
\scnidtf{множество всех тех и только тех сущностей, которые обладают заданным параметром*}
\scnrelto{включение}{объединение*}
	
\scnheader{измеряемый параметр}
\scnidtf{количественный параметр}
\scnidtf{семейство измеряемых величин}
\scnidtf{семейство классов эквивалентности, каждому из которых может быть поставлено в соответствие числовое значение}
\end{SCn}

Каждый \textbf{\textit{измеряемый параметр}} представляет собой \textit{параметр}, значение (элемент, экземпляр) которого трактуется как \textit{величина}, которой можно поставить в соответствие ее числовое значение на основании выбранной единицы измерения и/или точки отсчета (нулевой отметки выбранной шкалы).

\begin{SCn}
\scnsuperset{параметр, измеряемый по шкале}
	
\scnheader{неизмеряемый параметр}
\scnidtf{качественный параметр}
	
\scnheader{ориентированный параметр}
\scnidtf{упорядоченный параметр}
\scnsuperset{параметр, измеряемый по шкале}
\scnidtf{параметр, на значениях которого может быть задано некоторое отношение порядка, семантика которого уточняется в зависимости от семантики параметра}
	
%	\scnheader{порядок величин*}
%	\scnsubset{отношение строгого порядка}
%	\scnsuperset{большая величина*}
%	\scnnote{Связки отношения \textit{порядок величин*} могут связывать только величины \uline{одного и того же} \textit{ориентированного параметра}.}
	
%% Test
%	\scnheader{соединение значений ориентированного параметра*}
%	\scnexplanation{Связки отношения \textit{соединение значений ориентированного параметра*} связывают связки отношения \textit{порядок величин*} и величину (возможно, интервальную) того же параметра, элементами которой являются все сущности, значение данного параметра для которых лежит в интервале, границы которого задаются величинами, являющимися компонентами указанной связки отношения \textit{порядок величин*}.}
%	\scnrelfrom{описание примера}{
%		\scnfilescg{figures/sd_parameters_and_quantities/values_join.png}
%	}
%	\scnaddlevel{1}
%	\scnnote{В приведенном примере множество сущностей, имеющий длину 1м и множество сущностей, имеющих длину 3м, при помощи отношения \textit{соединение значений ориентированного параметра*} образуют множество сущностей, имеющих длину от 1 до 3м.}
%	\scnaddlevel{-1}
	
	
%% 
%	\scnheader{уровень класса эквивалентности}
%	\scnidtf{уровень параметра}
%	\scniselement{параметр}
%	\scnexplanation{Параметр \textbf{\textit{уровень класса эквивалентности}} задает уровень некоторого значения некоторого \textit{параметра} в иерархии значений этого параметра. Уровень класса эквивалентности равен 1, если значение параметра не является частным по отношению к другому значению этого параметра, равен 2, если значение параметра является частным по отношению к значению этого параметра с уровнем 1 и т.д.}
%	\scnrelfrom{описание примера}{
%		\scnfilescg{figures/sd_parameters_and_quantities/color.png}
%	}
	
\scnheader{величина}
\scnidtf{значение количественного параметра}
\scnidtf{значение измеряемого параметра}
\scnidtf{класс сущностей, имеющих одинаковое значение соответствующего параметра}
\begin{scnrelfromlist}{включение}
	\scnitem{точная величина}
	\scnitem{неточная величина}
	\scnitem{интервальная величина}
\end{scnrelfromlist}
\end{SCn}

Каждая \textbf{\textit{величина}} представляет собой однозначный и независящий от шкалы измерения результат измерения некоторой характеристики у некоторой сущности.
		
Каждой \textbf{\textit{величине}} можно поставить в соответствие ее числовое значение на основании выбранной единицы измерения и точки отсчета (нулевой отметки выбранной шкалы, в случае, если измерение осуществляется по шкале).
		
Нельзя путать значение параметра (\textbf{\textit{величину}}) и значение величины по некоторой шкале, которое может быть скалярным и векторным.
	
\begin{SCn}
\scnheader{точная величина}
\scnidtf{точное значение параметра}
\scnidtf{множество всех точных значений параметра}
\scnidtf{значение параметра, являющееся семейством классов эквивалентности, соответствующим некоторому отношению эквивалентности}
\scnidtf{класс эквивалентности}
\end{SCn}

Каждая \textbf{\textit{точная величина}} имеет одно фиксированное значение в некоторой единице измерения или по какой-либо шкале. При этом считается, что все элементы такого класса имеют одинаковое значение данного параметра и отклонениями можно пренебречь.
		
Каждой \textbf{\textit{точной величине}} можно поставить в соответствие группу \textit{неточных величин}, являющихся не разбиениями, а покрытиями того же множества, но с разной степенью точности.
%	\scnrelfrom{описание примера}{
%		\scnfilescg{figures/sd_parameters_and_quantities/exactLength.png}
%		\scnexplanation{В данном примере \textbf{\textit{ki}} обозначает класс сущностей, имеющих длину ровно 5 метров. Конкретный пример такой сущности - \textbf{\textit{bi}}.}}
\begin{SCn}
\scnheader{неточная величина}
\scnidtf{множество неточных значений параметра}
\scnidtf{приблизительная величина}
\scnidtf{приблизительное значение параметра}
\scnidtf{значение параметра в интервале с нефиксированными границами}
\end{SCn}

Каждой \textbf{\textit{неточной величине}} ставится в соответствие ее значение в некоторой единице измерения или по какой-либо шкале, а также дополнительно указывается \textit{точность*}, т.е. возможное отклонение от данного значения.
%	\scnrelfrom{описание примера}{
%		\scnfilescg{figures/sd_parameters_and_quantities/approximateLength.png}
%		\scnexplanation{В данном примере \textbf{\textit{ki}} обозначает класс сущностей, имеющих длину примерно 25 метров, при этом максимально возможное отклонение от этого значения составляет \textbf{\textit{kj}}, то есть 2 метра. Конкретный пример такой сущности - \textbf{\textit{bi}}.}}
\begin{SCn}
\scnheader{интервальная  величина}
\scnidtf{интервальное значение параметра}
\scnidtf{значение параметра в интервале с фиксированными границами}
\scnidtf{интервал значения параметра из множества пересекающихся интервалов разной длины, имеющих нефиксированные границы}
\end{SCn}

Каждая \textbf{\textit{интервальная величина}} представляет собой класс сущностей, находящихся в рамках точно заданного интервала, минимальная и максимальная точка которого являются \textit{точными величинами}. Результатом \textit{измерения*} такой величины является ориентированная пара, первым компонентом которой является левая (меньшая) граница интервала, вторым компонентом -- правая (большая) граница интервала.

%	\scnrelfrom{описание примера}{
%		\scnfilescg{figures/sd_parameters_and_quantities/intervalLength.png}
%		\scnexplanation{В данном примере \textbf{\textit{ki}} обозначает класс сущностей, имеющих длину, которая лежит в интервале от \textbf{\textit{kj}} до \textit{kl}, то есть в интервале от 4 до 5 метров, а \textbf{\textit{bi}} -- конкретный пример такой сущности.}}
\begin{SCn}
\scnheader{эталон\scnrolesign}
\scnidtf{образец\scnrolesign}
\scniselement{ролевое отношение}
\end{SCn}

\begin{SCn}
Ролевое отношение \textit{эталон\scnrolesign} указывает на тот элемент значения некоторого параметра, который в рамках данного класса эквивалентности считается эталонным, то есть он используется как образец при определении данного параметра.
\end{SCn}
		
\begin{SCn}
\textit{эталон\scnrolesign} может задаваться как для измеряемых, так и для неизмеряемых параметров, например, эталон метра или эталон красоты.
\end{SCn}
	
\begin{SCn}
\scnheader{измерение*}
\scnidtf{значение параметра*}
\scnidtf{значение заданной величины заданного параметра*}
\scnidtf{измерение как соответствие*}
\scnidtf{результат измерения заданной величины в заданной единице измерения и по заданной шкале*}
\scnidtf{бинарное ориентированное отношение, связывающее различные величины с результатами их измерения в различных единицах измерения и по различным шкалам*}
\end{SCn}

Связки отношения \textit{измерение*} связывают величину и ее значение в некоторой единице измерения (в том числе, в интервале) или по некоторой шкале. Конкретная единица измерения или шкала указывается дополнительно при помощи соответствующего отношения. Одной величине может соответствовать только одно значение в каждой возможной единице измерения или одна точка на некоторой шкале.
	
\begin{SCn}
\scnheader{точность*}
\scnidtf{отклонение*}
\scnidtf{степень точности неточного значения параметра*}
\scniselement{бинарное отношение}
\end{SCn}

Связки отношения \textbf{\textit{точность*}} связывают \textit{неточную величину} и \textit{точную величину} того же класса, задающую максимальное возможное отклонение указанной \textit{неточной величины} от своего значения.
	
%	\scnheader{параметрическая модель}
%	\scnidtf{параметрическая спецификация}
%	\scnidtf{параметрическое sc-описание заданной сущности}
%	\scnidtf{описание сущности как точки в некотором параметрическом (признаковом) пространстве}
%	\scnrelto{включение}{семантическая окрестность}
%	\scnexplanation{Каждая \textbf{\textit{параметрическая модель}} представляет собой описание заданной сущности в некотором параметрическом пространстве количественных и качественных \textit{параметров}, т.е. указание того, какие значения заданных параметров (характеристик) соответствуют описываемой (заданной) сущности.}
%	
\begin{SCn}
\scnheader{единица измерения*}
\scniselement{бинарное отношение}
\scnidtf{единица по шкале*}
\scnidtf{единичная отметка по шкале*}
\end{SCn}

Связки отношения \textbf{\textit{единица измерения*}} связывают знак конкретного \textbf{\textit{измерения с фиксированной единицей измерения}} и некоторую \textit{точную величину}, входящую в тот же конкретный \textit{параметр}, что и первый компонент связок этого конкретного измерения, и которая используется в данном случае в качестве единицы измерения.
	
\begin{SCn}
\scnheader{измерение с фиксированной единицей измерения }
\scnrelto{семейство подмножеств}{измерение*}
\end{SCn}

Каждая \textbf{\textit{измерение с фиксированной единицей измерения}} представляет собой подмножество отношения \textit{измерение*} и характеризуется некоторой \textit{единицей измерения*}, которая является элементом того же параметра (семейством сущностей, имеющих значение данного параметра, совпадающее с этой единицей измерения).
	
\begin{SCn}
\scnheader{измерение по шкале}
\scnidtf{шкала}
\scnrelto{семейство подмножеств}{измерение*}
\end{SCn}

Каждая \textbf{\textit{измерение по шкале}} представляет собой подмножество отношения \textit{измерение*} и характеризуется не единицей измерения, а некоторой точкой отсчета для данной \textbf{\textit{шкалы}}. Результатом \textbf{\textit{измерения по шкале}} будет некоторая точка шкалы, отстоящая от точки отсчета на определенное расстояние в нужную сторону (меньшую или большую). Понятно, что это расстояние может быть измерено любыми единицами измерения, но его величина при этом останется неизменной.
		
Не стоит путать измерение по \textbf{\textit{измерение по шкале}}, которое зависит от \textit{нулевой отметки*}, с измерением изменения того же \textit{параметра}, которое характеризуется единицей измерения и не зависит от точки отсчета. Например, не стоит путать дату по некоторому календарю, соответствующую \textit{началу} какого-либо процесса, и \textit{длительность} этого процесса, которая не зависит от выбранного календаря.

%	\scnrelfrom{описание примера}{
%		\scnfilescg{figures/sd_parameters_and_quantities/scale.png}
%		\scnexplanation{В данном примере \textbf{\textit{ki}} обозначает класс сущностей, имеющих точную температуру в 330 К, а \textbf{\textit{bi}} -- конкретный пример такой сущности.}}
	
%	\scnheader{нулевая отметка*}
%	\scnidtf{нуль по шкале*}
%	\scnidtf{начало отсчета*}
%	\scnidtf{точка отсчета*}
%	\scniselement{бинарное отношение}
%	\scnexplanation{Связки отношения \textbf{\textit{нулевая отметка*}} связывают знак некоторого \textit{измерения по шкале} со знаком \textit{точной величины} того же \textit{параметра}, которая в рамках данной шкалы принимается за точку отсчета.}
%	

Каждое \textbf{\textit{арифметическое выражение на величинах}} представляет собой \textit{связку}, компонентами которой являются элементы или подмножества некоторого \textit{количественного параметра}.

\begin{SCn}
\scnheader{арифметическая операция на величинах}
\scnrelto{семейство подмножеств}{арифметическое выражение на величинах}
\end{SCn}

Каждая \textbf{\textit{арифметическая операция на величинах}} представляет собой \textit{отношение}, элементами которого являются \textit{арифметические выражения на величинах}, то есть множество \textit{арифметических выражений на величинах} какого-либо одного вида.
	
\begin{SCn}
\scnheader{сумма величин*}
\scnidtf{сложение величин*}
\scniselement{арифметическая операция на величинах}
\scniselement{квазибинарное отношение}
\end{SCn}

\textbf{\textit{сумма величин*}} -- это \textit{арифметическая операция на величинах}, аналогичная \textit{арифметической операции сумма*} для \textit{чисел}.
		
Первым компонентом связки отношения \textbf{\textit{сумма величин*}} является подмножество некоторого \textit{количественного параметра} (слагаемые \textit{величины}), содержащее два или более элемента, вторым компонентом -- элемент этого же \textit{количественного параметра}, значение которого в любой \textit{единице измерения*} является результатом сложения значений всех слагаемых \textit{величин} в той же \textit{единице измерения*}. При несовпадении \textit{единиц измерения} слагаемых величин необходимо воспользоваться соотношениями между \textit{единицами измерения}, которые задаются при помощи операций \textit{произведение величин*} и \textit{возведение величин в степень*}.
	
\begin{SCn}
\scnheader{произведение величин*}
\scnidtf{умножение величин*}
\scniselement{арифметическая операция на величинах}
\scniselement{квазибинарное отношение}
\end{SCn}
	
\textbf{\textit{произведение величин*}} -- это \textit{арифметическая операция на величинах}, аналогичная \textit{арифметической операции} \textit{произведения*} для \textit{чисел}.
		
Первым компонентом связки отношения \textbf{\textit{произведение величин*}} является \textit{связка}, элементами которой являются либо \textit{величины количественных параметров}, либо \textit{числа}, но при этом хотя бы один элемент должен быть \textit{величиной}. Вторым компонентов является \textit{величина количественного параметра}.
		
Операция \textbf{\textit{произведение величин*}} может быть использована для задания соотношения между \textit{единицами измерения*} в рамках одного \textit{количественного параметра}.
%	\scnrelfrom{описание примера}{
%		\scnfilescg{figures/sd_parameters_and_quantities/multiplicationOfQuantities.png}}
%	
%	\scnrelfrom{описание примера}{
%		\scnfilescg{figures/sd_parameters_and_quantities/multiplicationOfQuantities2.png}}
\begin{SCn}
\scnheader{возведение величин в степень*}
\scniselement{арифметическая операция на величинах}
\scniselement{бинарное отношение}
\end{SCn}

\textbf{\textit{возведение величин в степень*}} -- это \textit{арифметическая операция на величинах}, аналогичная \textit{арифметической операции возведение в степень*} для \textit{чисел}.
		
Первым компонентом связки отношения \textbf{\textit{возведение величин в степень*}} является ориентированная пара, первым компонентом которой является \textit{величина количественного параметра} (основание степени), вторым -- \textit{число} (показатель степени); Вторым компонентом связки отношения \textbf{\textit{возведение величин в степень*}} является \textit{величина количественного параметра} (результат возведения в степень).
%	\scnrelfrom{описание примера}{
%		\scnfilescg{figures/sd_parameters_and_quantities/exponentiation.png}}
%	
%	\scnrelfrom{описание примера}{
%		\scnfilescg{figures/sd_parameters_and_quantities/exponentiationTo2.png}}
	
%	\scnheader{порядок величин*}
%	\scnidtf{большая величина*}
%	\scnidtf{сравнение величин*}
%	\scniselement{бинарное отношение}
%	\scniselement{отношение строгого порядка}
%	\scnexplanation{\textbf{\textit{порядок величин*}} -- это \textit{отношение на величинах}, аналогичное отношению \textit{больше*} для \textit{чисел}.\\
%		Из двух величин большей является та, \textit{значение} которой в любой \textit{единице измерения*} \textit{больше*} значения другой \textit{величины} в той же \textit{единице измерения*}. В ряде случаев нет возможности говорить о большей или меньшей величине, но можно говорить об их упорядоченности.}
\begin{SCn}
\scnheader{действие. измерение}
\scnidtf{измерение как действие}
\scnidtf{действие, направленное на установление связи, принадлежащей отношению измерение* и связывающей величину, которая принадлежит заданному параметру, и которой принадлежит заданная сущность, и соответствующее значение этой величины на некоторой шкале}
\scnidtf{действие, направленное на решение задачи измерения заданного параметра у заданной сущности}
\scnrelto{включение}{действие}
	
\scnheader{задача. измерение}
\scnidtf{спецификация действия измерения}
\scnidtf{спецификация действия, целью которого является измерение заданного параметра у заданной сущности}
\scnrelto{включение}{задача}
\end{SCn}	

\section{Формальная онтология чисел и числовых структур}
\label{sec_top_ontologies_numbers}
%% Введение в пердметную область

\begin{SCn}
\scnheader{Предметная область чисел и числовых структур}
\scniselement{предметная область}
\scnsdmainclasssingle{число}
	
\scnheader{число}
\scnidtf{множество чисел}
\scnsubset{абстрактная терминальная сущность}
\end{SCn}

\textbf{\textit{число}} -- это основное понятие математики, используемое для количественной характеристики, сравнения, нумерации объектов и их частей. Письменными знаками для обозначения чисел служат \textit{цифры}.

\begin{SCn}
\scnheader{цифра}
\scnidtf{множество цифр}
\scnsubset{внутренний файл ostis-системы}

\begin{scnrelfromlist}{включение}
	\scnitem{арабская цифра}
	\scnitem{римская цифра}
\end{scnrelfromlist}
\end{SCn}

\textbf{\textit{цифра}} -- это множество файлов, обозначающих вхождение этой цифры во всевозможные записи чисел с помощью этой цифры.

\begin{SCn}
\scnheader{натуральное число}
\scnidtf{множество натуральных чисел}
\scnsubset{целое число}
\end{SCn}

\textbf{\textit{натуральное число}} -- это подмножество множества \textit{целых чисел}, которые используются при счете предметов.

\begin{SCn}
\scnheader{целое число}
\scnidtf{множество целых чисел}
\scnsubset{рациональное число}
\scnsubset{целое число}
\end{SCn}

\textbf{\textit{целое число}} -- это подмножество множества \textit{рациональных чисел}, получаемых объединением \textit{натуральных чисел} с множеством чисел, \textit{противоположных* натуральным} и \textit{нулём}.

\begin{SCn}
\scnheader{рациональное число}
\scnidtf{множество рациональных чисел}
\scnsubset{действительное число}
\scnsubset{целое число}
\end{SCn}

\textbf{\textit{рациональное число}} -- это число, представляемое \textit{обыкновенной дробью}, где числитель — \textit{целое число}, а знаменатель — \textit{натуральное число}.

\begin{SCn}
\scnheader{дробь}
\scnidtf{множество дробей}
\begin{scnrelfromlist}{включение}
	\scnitem{обыкновенная дробь}
	\scnitem{десятичная дробь}
\end{scnrelfromlist}
\end{SCn}

\textbf{\textit{дробь}} — это число, состоящее из одной или нескольких равных частей (долей) единицы

\begin{SCn}
\scnheader{обыкновення дробь}
\scnidtf{множество обыкновенных дробей}
\scnidtf{множество простых дробей}
\end{SCn}

\textbf{\textit{обыкновенная дробь}} - запись \textit{рационального числа} в виде ${\displaystyle \pm {\frac {m}{n}}}$ или ${\pm m/n}$, где ${n\neq 0}$.Горизонтальная или косая черта обозначает знак деления, в результате которого получается частное. Делимое называется числителем дроби, а делитель — знаменателем.

\begin{SCn}
\scnheader{десятичная дробь}
\scnidtf{множество десятичных дробей}
\end{SCn}

\textbf{\textit{десятичная дробь}} — разновидность дроби, которая представляет собой способ представления действительных чисел в виде ${\pm d_m \ldots d_1 d_0{,} d_{-1} d_{-2} \ldots}$, где , — десятичная запятая, служащая разделителем между целой и дробной частью числа, ${d_{k}}$m — десятичные цифры.

\begin{SCn}
\scnheader{иррациональное число}
\scnidtf{множество иррациональных чисел}
\scnsubset{действительное число}
\end{SCn}

\textbf{\textit{иррациональное число}} -- это \textit{вещественное число}, которое не является рациональным, то есть не может быть представлено в виде дроби, где числитель — \textit{целое число}, знаменатель — \textit{натуральное число}. Любое \textbf{\textit{иррациональное число}} может быть представлено в виде бесконечной непериодической десятичной дроби.

\begin{SCn}
\scnheader{действительное число}
\scnidtf{вещественное число}
\scnidtf{множество вещественных чисел}
\begin{scnreltoset}{объединение}
	\scnitem{рациональное число}
	\scnitem{иррациональное число}
\end{scnreltoset}
\scnsubset{комплексное число}
\begin{scnsubdividing}
	\scnitem{положительное число}
	\scnitem{отрицательное число}
	\scnitem{$\{$Нуль$\}$}
\end{scnsubdividing}
\end{SCn}

\textbf{\textit{действительное число}} -- это множество чисел, получаемое в результате объединения иррациональных и \textit{рациональных чисел}.

\begin{SCn}
\scnheader{арифметическое выражение}
\scnidtf{множество арифметических выражений}
\end{SCn}

Каждое \textbf{\textit{арифметическое выражение}} представляет собой \textit{связку}, компонентами которой являются \textit{числа} или множества \textit{чисел}.

\begin{SCn}
\scnheader{арифметическая операция}
\scnidtf{множество арифметических операций}
\scnrelto{семейство подмножеств}{арифметическое выражение}
\end{SCn}

Каждая \textbf{\textit{арифметическая операция}} представляет собой \textit{отношение}, элементами которого являются \textit{арифметические выражения}, то есть множество \textit{арифметических выражений} какого-либо одного вида.
	
\begin{SCn}
\scnheader{сумма*}
\scnidtf{сложение*}
\scniselement{арифметическая операция}
\scniselement{квазибинарное отношение}
\end{SCn}

\textbf{\textit{сумма*}} -- это арифметическая операция, в результате которой по данным числам (слагаемым) находится новое число (сумма), обозначающее столько единиц, сколько их содержится во всех слагаемых.
		
Первым компонентом связки отношения \textbf{\textit{сумма*}} является \textit{множество чисел} (слагаемых), содержащее два или более элемента, вторым компонентом -- \textit{число}, являющееся результатом сложения.
		
Отдельно отметим, что каждая связка отношения \textbf{\textit{сумма*}} вида a = b+c может также трактоваться и как запись о вычитании чисел, например b = a-c, в связи с чем \textit{арифметическая операция} разности чисел отдельно не вводится.

%\scnrelfrom{описание примера}{
%\scnfilescg{figures/sd_numbers/sum.png}}
\begin{SCn}
\scnheader{произведение*}
\scnidtf{умножение*}
\scniselement{арифметическая операция}
\scniselement{квазибинарное отношение}
\end{SCn}

\textbf{\textit{произведение*}} -- это \textit{арифметическая операция}, в результате которой один аргумент складывается столько раз, сколько показывает другой, затем результат складывается столько раз, сколько показывает третий и т.д.
		
Первым компонентом связки отношения \textbf{\textit{произведение*}} является \textit{множество чисел} (множителей), содержащее два или более элемента, вторым компонентом -- \textit{число}, являющееся результатом произведения.
		
Отдельно отметим, что каждая связка отношения \textbf{\textit{произведение*}} вида a = b*c может также трактоваться и как запись о делении чисел, например b = a/c, в связи с чем \textit{арифметическая операция} деления чисел отдельно не вводится.
%	\scnrelfrom{описание примера}{
%		\scnfilescg{figures/sd_numbers/multiplication.png}}
	
%	\scnheader{возведение в степень*}
%	\scniselement{арифметическая операция}
%	\scniselement{бинарное отношение}
%	\scnexplanation{\textbf{\textit{возведение в степень*}} -- это \textit{арифметическая операция}, в результате которой число, называемое основанием степени, умножается само на себя столько раз, каков показатель степени.
%		
%		Первым компонентом связки отношения \textbf{\textit{возведение в степень*}} является ориентированная пара, первым компонентом которой является \textit{число}, которое является основанием степени, вторым -- \textit{число}, которое является показателем степени; Вторым компонентом связки отношения \textbf{\textit{возведение в степень*}} является \textit{число}, которое является результатом возведения в степень.
%		
%		Отдельно отметим, что каждая связка отношения \textbf{\textit{возведение в степень*}} вида a = $b^c$ может также трактоваться и как запись об извлечении корня или взятии логарифма, в связи с чем \textit{арифметические операции} извлечения корня и взятия логарифма отдельно не вводится.}
%	\scnrelfrom{описание примера}{
%		\scnfilescg{figures/sd_numbers/pow.png}}
	
%	\scnheader{больше*}
%	\scniselement{бинарное отношение}
%	\scniselement{отношение строгого порядка}
%	\scnexplanation{\textbf{\textit{больше*}} -- это \textit{бинарное отношение} сравнения чисел. Из двух чисел на координатной прямой больше то, которое расположено правее. Соответственно, первым компонентом связки \textit{отношения} \textbf{\textit{больше*}} является большее из двух \textit{чисел}.}
%	\scnrelfrom{описание примера}{
%		\scnfilescg{figures/sd_numbers/more.png}}
%	
%	\scnheader{больше или равно*}
%	\scniselement{бинарное отношение}
%	\scniselement{отношение нестрогого порядка}
%	\scnexplanation{\textbf{\textit{больше или равно*}} -- это \textit{бинарное отношение} сравнения чисел, при которой первое \textit{число} (первый компонент связки) может быть \textit{больше*} второго или \textit{равняться*} ему.}
%	\scnnote{Отношение \textit{равенство*} явно не вводится, поскольку в рамках SC-кода равные числа трактуются как совпадающие числа, то есть обозначаемые одним и тем же \textit{sc-элементом}.}
%	\scnrelfrom{описание примера}{
%		\scnfilescg{figures/sd_numbers/more_or_equal.png}}
	
%	\scnheader{Число Пи}
%	\scniselement{иррациональное число}
%	\scnexplanation{\textbf{\textit{Число Пи}} -- это  математическая константа, равная отношению длины окружности к длине её диаметра.}
%	
%	\scnheader{Нуль}
%	\scnidtf{0}
%	\scniselement{целое число}
%	\scnexplanation{\textbf{\textit{Нуль}} -- это \textit{целое число}, разделяющее на числовой прямой \textit{положительные числа} и \textit{отрицательные числа}.}
%	
%	\scnheader{Единица}
%	\scnidtf{1}
%	\scniselement{целое число}
%	\scniselement{натуральное число}
%	\scnexplanation{\textbf{\textit{Единица}} -- это наименьшее \textit{натуральное число}.}
%	
%	\scnheader{Мнимая единица}
%	\scnidtf{i}
%	\scniselement{комплексное число}
%	\scnexplanation{\textbf{\textit{Мнимая единица}} -- это \textit{число}, при возведении которого в степень 2 результатом будет число, противоположное \textit{Единице}.}
\begin{SCn}
\scnheader{числовая структура}
\scnsubset{структура}
\end{SCn}

\textbf{\textit{числовая структура}} -- \textit{структура}, в состав которой входят знаки \textit{арифметических выражений}, а также знаки их элементов и связи между выражениями и их элементами.
	
\begin{SCn}
\scnheader{система счисления}
\scniselement{параметр}
\end{SCn}

Каждая \textbf{\textit{система счисления}} представляет собой класс синтаксически эквивалентных файлов, хранимых в sc-памяти, каждый из которых может являться идентификатором какого-либо \textit{числа}.
		
Каждая \textbf{\textit{система счисления}} характеризуется алфавитом, т.е. конечным множеством символов (цифр), которые допускается использовать при построении файлов принадлежащих данной \textbf{\textit{системе счисления}}.
	
%	\scnheader{десятичная система счисления}
%	\scniselement{система счисления}
%	
%	\scnheader{двоичная система счисления}
%	\scniselement{система счисления}
%	
%	\scnheader{шестнадцатеричная система счисления}
%	\scniselement{система счисления}
	
\section{Формальная онтология темпоральных сущностей}
\label{sec_top_ontologies_temp}
%% Введение в пердметную область
		
\begin{SCn}
\scnheader{Предметная область темпоральных сущностей}
\scnidtf{Предметная область темпоральных связей и отношений}
\scnidtf{Предметная область временных сущностей}
\scniselement{предметная область}
\scnsdmainclasssingle{временная сущность}
%\scnsdclass{прошлая сущность;настоящая сущность;будущая сущность;временная связь;ситуация;процесс;процесс в sc-памяти;процесс во внешней среде ostis-системы;материальная сущность;воздействие;отношение;класс временных связей;класс временных и постоянных связей;множество;ситуативное множество;неситуативное множество;частично ситуативное множество;темпоральная связь;темпоральное отношение;начало\scnsupergroupsign;завершение\scnsupergroupsign;длительность\scnsupergroupsign;тысячелетие;век;год;месяц;сутки;час;минута;секунда}
%\scnsdrelation{воздействующая сущность*;объект воздействия*;начальная ситуация*;причинная ситуация*;конечная ситуация*;событие*;последний добавленный sc-элемент\scnrolesign;темпоральное включение*;темпоральная часть*;начальный этап*;конечный этап*;промежуточный этап*;темпоральное включение без совпадения начальных и конечных моментов*;темпоральное включение с совпадением начальных моментов*;темпоральное включение с совпадением конечных моментов*;темпоральное совпадение*;темпоральное объединение*;темпоральная декомпозиция*;темпоральная смежность*;темпоральная последовательность с промежутком*;темпоральная последовательность с пересечением*;номер тысячелетия\scnrolesign;номер века\scnrolesign;номер года\scnrolesign;номер месяца в году\scnrolesign;номер суток в месяце\scnrolesign;номер часа в дне\scnrolesign;номер минуты в часе\scnrolesign;номер секунды в минуте\scnrolesign}
		
\scnheader{временная сущность}
\scnidtf{временно существующая сущность}
\scnidtf{нестационарная сущность}
\scnidtf{сущность, имеющая и/или начало, и/или конец своего существования}
\scnidtf{sc-элемент, являющийся знаком некоторой временно существующей сущности}
\scnidtf{сущность, обладающая темпоральными характеристиками (длительностью, начальным моментом, конечным моментом и т.д.)}
\begin{scnsubdividing}
	\scnitem{прошлая сущность}
	\scnitem{настоящая сущность}
	\scnitem{будущая сущность}
\end{scnsubdividing}
\begin{scnsubdividing}
	\scnitem{временная связь}
	\scnitem{темпоральная структура}
	\begin{scnindent}
		\scnidtf{структура, содержащая хотя бы одну временную сущность}
		\scnrelfrom{включение}{структура}
		\scnnote{Следует отличать:
			\begin{scnitemize}
				\item временный характер самой структуры как sc-элемента;
				\item временный характер sc-элементов, принадлежащих данной структуре, и сущностей, обозначаемых этими sc-элементами;
				\item временный характер пар принадлежности, связывающих структуру с ее элементами.
		\end{scnitemize}}
		\scnidtf{структура, описывающая темпоральные свойства (свойства, связанные со временем) окружающей среды, частью которой являются также и различные базы знаний кибернетических систем (в том числе и собственная база знаний).}
		\begin{scnsubdividing}
			\scnitem{ситуация}
			\begin{scnindent}
			\scnidtf{статическая темпоральная структура}
			\end{scnindent}
			\scnitem{процесс}
			\begin{scnindent}
			\scnidtf{динамическая структура}
			\scnidtf{динамическая темпоральная структура}
			\end{scnindent}
		\end{scnsubdividing}
		\end{scnindent}
	\scnitem{материальная сущность}	
\end{scnsubdividing}


\begin{scnsubdividing}
	\scnitem{непрерывная временная сущность}
	\begin{scnindent}
	\begin{scnsubdividing}
		\scnitem{точечная временная сущность}
		\begin{scnindent}
			\scnidtf{атомарная временная сущность}
			\scnidtf{условно мгновенная временная сущность}
			\scnidtf{временная сущность, длительность существования которой в данном контексте считается несущественной (пренебрежительно малой)}
		\end{scnindent}
		\scnitem{длительная непрерывная временная сущность}
	\end{scnsubdividing}
	\end{scnindent}	
		\scnitem{дискретная временная сущность}
	\begin{scnindent}
		\scnidtf{временная сущность, которая может быть декомпозирована на последовательность точечных временных сущностей}
		\scnidtf{временная сущность, которой соответствует некоторый временной ряд параметров (состояний) точечных временных сущностей, на которые декомпозируется исходная временная сущность}
	\end{scnindent}	
		\scnitem{прерывистая временная сущность}
	\begin{scnindent}
		\scnidtf{временная сущность, являющаяся результатом соединения нескольких не только точечных временных сущностей}
		\scnidtf{временная сущность с прерываниями}
	\end{scnindent}
\end{scnsubdividing}

\begin{scnindent}
	\scnnote{Следует отметить, что приведенная классификация \textit{временных сущностей} характеризует не столько сами \textit{временные сущности}, сколько наши знания о них и степень детализации знаний об этих сущностях, с которой они описаны в базе знаний. Так, если для решения конкретных задач не важно, как изменялась некоторая \textit{временная сущность} в рамках какого-либо периода времени, а важно только ее начальное и конечное состояние, то она может рассматриваться как \textit{точечная временная сущность}. Впоследствии же та же \textit{временная сущность} может быть рассмотрена и описана с большей степенью детализации, и таким образом, уже не будет точечной.}
\end{scnindent}
\end{SCn}	

Следует отличать:
	\begin{textitemize}
		\item временный характер сущности, обозначаемой \textit{sc-элементом};
		\item временный характер существования самого \textit{sc-элемента} в рамках \textit{sc-памяти}, поскольку в ходе обработки информации каждый \textit{sc-элемент} может быть удален из \textit{sc-памяти}; 
		\item временный характер описываемых ситуаций, событий и самих процессов;
		\item временный характер хранения в sc-памяти тех sc-конструкций, которые являются самими описаниями соответствующих ситуаций, событий и процессов.
	\end{textitemize}

		
Следует отличать, например, \textit{материальную сущность} (некоторый физический или, в частности, биологический объект) от различных динамических структур (\textit{процессов}), которые с той или иной степенью детализации и в том или ином ракурсе отражают (описывают) динамику изменений этой \textit{материальной сущности}. 
			
При этом сам \textit{процесс} как уточнение динамики некоторой последовательности ситуаций и событий, также является сущностью, принадлежащей к классу \textit{временных сущностей}.
		
\begin{SCn}
\scnheader{прошлая сущность}
\scnidtf{сущность, существовавшая в прошлом времени}
\scnidtf{сущность прошлого времени}
\scnidtf{сущность, завершившая свое существование}
		
\scnheader{настоящая сущность}
\scnidtf{сущность, существующая в текущий момент времени}
\scnidtf{сущность, существующая сейчас}
\scnidtf{сущность настоящего времени}
		
\scnheader{будущая сущность}
\scnidtf{возможно будущая сущность}
\scnidtf{прогнозируемая временная сущность}
\scnidtf{временная сущность, которая может существовать в будущем}
\scnidtf{сущность, которая может или должна начать свое существование в будущем времени}
\scnrelfrom{включение}{инициированное действие}
\end{SCn}

Каждой \textbf{\textit{будущей сущности}} можно поставить в соответствие вероятность ее возникновения.
	
\begin{SCn}
\scnheader{временная связь}
\scnidtf{нестационарная связь}
\scnidtf{временно существующая связь}
\end{SCn}

Каждая \textbf{\textit{временная связь}} представляет собой \textit{связку}, принадлежащую множеству \textit{временных сущностей}.
			
Понятие \textbf{\textit{временной связи}} не следует путать с понятием \textit{темпоральной связи}, которая сама является \textit{постоянной сущностью}, описывающей то, как связаны во времени некоторые \textit{временные сущности}.

\begin{SCn}
\scnheader{ситуация}
\scnidtf{состояние}
\scnidtf{временная структура}
\scnidtf{временно существующая структура}
\scnidtf{квазистационарная структура}
\scnidtf{состояние некоторой динамической системы, описываемое с некоторой степенью детализации (подробности)}
\scnidtf{квазистационарная структура, существующая временно (в течение некоторого отрезка времени)}
\end{SCn}

Под ситуацией понимается \textit{структура}, содержащая, по крайней мере, один элемент, который является \textit{временной сущностью}. Наличие в рамках ситуации нескольких \textit{временных сущностей} говорит о том, что существует момент времени (в прошлом, настоящем или будущем), в который все они существуют одновременно.

\begin{SCn}
\scnheader{процесс}
\scnidtf{процесс преобразования некоторой временной сущности из одного состояния в другое}
\scnidtf{процесс перехода от одной ситуации к другой}
\scnidtf{абстрактный процесс}
\scnidtf{информационная модель некоторых преобразований}
\scnidtf{динамическая sc-модель}
\scnidtf{динамическая структура}
\scnrelfrom{включение}{воздействие}
\end{SCn}

Каждый \textbf{\textit{процесс}} определяется (задается) либо возникновением или исчезновением связей, связывающих заданную \textit{временную сущность} с другими сущностями, либо возникновением или исчезновением связей, связывающих части указанной \textit{временной сущности} с другими сущностями. 
			
Многим \textbf{\textit{процессам}} можно поставить в соответствие \textit{ситуацию}, которая является его \textit{начальной ситуацией*} и \textit{ситуацию}, которая является его \textit{конечной ситуацией*}, то есть указать \textit{ситуации}, переход между которыми осуществляется в ходе \textbf{\textit{процесса}}.
			
При этом знаки тех \textit{временных сущностей}, с которыми связаны изменения, описываемые некоторым \textbf{\textit{процессом}}, входят в данный \textbf{\textit{процесс}} как элементы и при необходимости уточняются дополнительными \textit{ролевыми отношениями}.

\begin{SCn}
\begin{scnsubdividing}
	\scnitem{процесс в sc-памяти}
	\scnitem{процесс во внешней среде ostis-системы}
\end{scnsubdividing}
\end{SCn}

Каждой \textbf{\textit{материальной сущности}} можно поставить в соответствие различные \textit{процессы}, описывающие ее преобразование из одного состояния в другое.

Поскольку \textit{процесс} представляет собой изменяющуюся во времени динамическую структуру, то полностью представить процесс в базе знаний в общем случае не представляется возможным. Однако, можно ввести sc-элемент, обозначающий конкретный процесс, с необходимой степенью детализации описать его декомпозицию на более частные подпроцессы и/или описать ситуации, соответствующие состояниям процесса в разные моменты времени. В данном случае можно провести некоторую аналогию с \textit{бесконечными множествами}, все элементы которых физически не могут быть представлены в базе знаний одновременно, тем не менее, само множество и некоторые из его элементов могут быть описаны с необходимой степенью детализации.
	
\begin{SCn}
\scnheader{воздействие}
\scnidtf{процесс, осуществляющийся на основе (в результате) воздействия одной сущности на другую}
\scnrelfrom{включение}{действие}
\end{SCn}

Каждому \textbf{\textit{воздействию}} может быть поставлена в соответствие (1) некоторая \textit{воздействующая сущность*}, т.е. сущность, осуществляющая \textbf{\textit{воздействие}} (в частности, это может быть некоторое физическое поле), и (2) некоторый \textit{объект воздействия*}, т.е. сущность, на которую воздействие направлено. Если \textbf{\textit{воздействие}} связано с \textit{материальной сущностью}, то его объектом воздействия является либо сама эта \textit{материальная сущность}, либо некоторая ее пространственная часть.

\begin{SCn}
\scnheader{исходная ситуация*}
\scnidtf{начальная ситуация процесса*}
\scnidtf{начальная ситуация*}
\scniselement{бинарное отношение}
\scnrelfrom{первый домен}{процесс}
\scnrelfrom{второй домен}{ситуация}
\end{SCn}

Связки отношения \textbf{\textit{исходная ситуация*}} связывают некоторый \textit{процесс} и некоторую ситуацию, являющуюся начальной для этого \textit{процесса}, и, как правило, изменяемой в течение выполнения этого \textit{процесса}.
			
Первым компонентом каждой связки отношения \textbf{\textit{исходная ситуация*}} является знак \textit{процесса}, вторым -- знак начальной \textit{ситуации}.
		
\begin{SCn}
\scnheader{причинная ситуация*}
\scnsubset{начальная ситуация*}
\end{SCn}

Под причинной ситуацией понимается такая \textit{начальная ситуация*}, которая обладает достаточной полнотой для однозначного задания инициируемого \textit{процесса}.
		
\begin{SCn}
\scnheader{конечная ситуация*}
\scnidtf{конечная ситуация процесса*}
\scnidtf{результирующая ситуация*}
\scniselement{бинарное отношение}
\scnrelfrom{первый домен}{процесс}
\scnrelfrom{второй домен}{ситуация}
\end{SCn}

Связки отношения \textbf{\textit{конечная ситуация*}} связывают некоторый \textit{процесс} и некоторую \textit{ситуацию}, ставшую результатом выполнения этого \textit{процесса}, то есть его следствием.
			
Первым компонентом каждой связки отношения \textbf{\textit{конечная ситуация*}} является знак \textit{процесса}, вторым -- знак конечной \textit{ситуации}.
		
\begin{SCn}
\scnheader{точечный процесс}
\scnidtf{атомарный процесс}
\scnidtf{условно мгновенный процесс}
\scnidtf{процесс, длительность которого в данном контексте считается несущественной (пренебрежимо малой)}
\scnsubset{точечная временная сущность}
		
\scnheader{элементарный процесс}
\scnidtf{процесс, детализация которого на входящие в него подпроцессы в текущем контексте не производится}
\scnsuperset{точечный процесс}
\end{SCn}

Элементарные процессы могут иметь длительность и, следовательно, не обязательно являются атомарными процессами.

Понятия \textit{точечного процесса} и \textit{элементарного процесса}, как и понятие \textit{точечной временной сущности} в целом, характеризуют не столько характеристики самого \textit{процесса}, сколько степень наших знаний о нем и степень детализации описания процесса в базе знаний. Так, очевидно, что любой процесс, протекающий в компьютерной системе, может быть при необходимости детализирован до уровня команд процессора, затем до уровня микропрограмм и даже до уровня физических процессов (изменения физических характеристик сигналов), однако чаще всего такая детализация не требуется.
%		\scnaddlevel{1}
%		\scnrelto{примечание}{точечный процесс}
%		\scnaddlevel{-1}

\begin{SCn}
\scnheader{событие}
\scnsubset{точечная временная сущность}
\scnidtf{точечная временная сущность, являющаяся началом и/или завершением какой-либо временной сущности (например, процесса)}
\scnidtf{граничная точка временной сущности}
%\scnrelfrom{описание примера}{
%		\scnfilescg{figures/sd_temp_entities/event.png}
%		}
%		\scnaddlevel{1}
%		\scnnote{Одно и то же событие может быть одновременно завершением одной временной сущности и началом другой. В приведенном примере событие $\bm{ei}$ является завершением временной сущности $\bm{si}$ и началом временной сущности $\bm{sj}$.}
%		\scnaddlevel{-1}
		
\scnheader{начало*}
\scnidtf{быть начальным событием заданной временной сущности*}
\scnrelfrom{первый домен}{временная сущность}
\scnrelfrom{второй домен}{событие}
\scnidtf{быть начальной точечной временной частью заданной временной сущности*}
		
\scnheader{завершение*}
\scnidtf{конец*}
\scnidtf{быть конечным событием заданной временной сущности*}
\scnidtf{быть конечной точечной временной частью заданной временной сущности*}
\scnrelfrom{первый домен}{временная сущность}
\scnrelfrom{второй домен}{событие}
		
\scnheader{событие*}
\scniselement{бинарное отношение}
\scnexplanation{Связки отношения \textbf{\textit{событие*}} связывают знак процесса и ориентированную пару, первым компонентом которой является знак \textit{начальной ситуации*} данного процесса, вторым компонентом -- знак \textit{конечной ситуации*} данного процесса.}
%\scnrelfrom{описание примера}{
%			\scnfilescg{figures/sd_temp_entities/nrel_event.png}
%		}
%		
\scnheader{детализация процесса*}
\scnidtf{Бинарное ориентированное отношение, каждая связка которого связывает некоторый процесс с более детальным его описанием, что предполагает представление декомпозиции этого процесса на систему взаимосвязанных его подпроцессов (в том числе элементарных).}
\scnrelfrom{пример}{Переход от процесса, соответствующего какой-либо программе, к рассмотрению декомпозиции этого процесса (протокола) в терминах языка программирования высокого уровня, затем переход для каждого из полученных подпроцессов (операторов языка высокого уровня) к детализации выполнения этих подпроцессов на уровне машинных операций, выполняемых процессором компьютера (на уровне ассемблера), и далее к детализации выполнения подпроцессов уровня машинных операций к подпроцессам на уровне языка микропрограммирования. Таким образом, детализация процесса может быть иерархической, вплоть до уровня \textit{элементарных процессов}.}
		
\scnheader{отношение}
\begin{scnsubdividing}
	\scnitem{класс временных связей}
	\scnitem{класс постоянных связей}
	\scnitem{класс временных и постоянных связей}
\end{scnsubdividing}
		
\scnheader{класс временных связей}
\scnidtf{отношение, все связки которого являются нестационарными}
\end{SCn}

В общем случае \textbf{\textit{класс временных связей}} не является \textit{ситуативным множеством}, поскольку факт принадлежности некоторой \textit{временной связи} такому классу следует считать постоянным, а не временным, поскольку временность/постоянство связи и ее семантический тип, задаваемый классом (отношением), это принципиально разные параметры (характеристики, признаки) любой связи.
		
\begin{SCn}
\scnheader{класс постоянных связей}
\scnidtf{отношение, все связки которого являются стационарными}
		
\scnheader{класс временных и постоянных связей}
\scnidtf{отношение, некоторые (но не все) связки которого являются нестационарными}
		
\scnheader{множество}
\begin{scnsubdividing}
	\scnitem{ситуативное множество}
	\scnitem{неситуативное множество}
	\scnitem{частично ситуативное множество}
\end{scnsubdividing}
		
\scnheader{ситуативное множество}
\scnidtf{полностью ситуативное множество}
\end{SCn}

Под \textbf{\textit{ситуативным множеством}} понимается постоянное множество, у которого все выходящие из него связи принадлежности являются \textit{временными сущностями}.
			
В частности, ситуативное множество может использоваться как вспомогательная динамическая структура, которая содержит элементы некоторых структур, обрабатываемые в данный момент, например, это может быть копия некоторого множества, из которой постепенно удаляются элементы по мере их просмотра и обработки. В случае, когда такая структура содержит всего один элемент, ее можно считать \underline{указателем} на данный элемент, при этом в разные моменты времени это могут быть разные элементы.

\begin{SCn}
\scnheader{последний добавленный sc-элемент\scnrolesign}
\scniselement{ролевое отношение}
		
\scnheader{неситуативное множество}
\scnexplanation{Под \textbf{\textit{неситуативным множеством}} понимается постоянное множество, у которого все выходящие из него связи принадлежности являются \textit{постоянными сущностями}.}
		
\scnheader{частично ситуативное множество}
\scnexplanation{Под \textbf{\textit{частично ситуативным множеством}} понимается постоянное множество, у которого некоторые (но не все) выходящие из него связи принадлежности являются \textit{временными сущностями}.}
		
\scnheader{темпоральная связь}
\scnidtf{связь во времени}
\scnidtf{\uline{постоянная} связь, описывающая связь во времени между временными сущностями}
		
\scnheader{темпоральное отношение}
\scnrelto{семейство подмножеств}{темпоральная связь}
\scnidtf{класс темпоральных связей}
\scnidtf{отношение, задающее темпоральные связи между временными сущностями}
\scnhaselement{темпоральное включение*}
\scnhaselement{темпоральное объединение*}
\scnhaselement{темпоральная декомпозиция*}
\scnhaselement{темпоральная последовательность*}
\begin{scnindent}
	\begin{scnsubdividing}
	\scnitem{темпоральная смежность*}
	\scnitem{темпоральная последовательность с промежутком*}
	\scnitem{темпоральная последовательность с пересечением*}
	\end{scnsubdividing}
\end{scnindent}
		
\scnheader{темпоральное включение*}
\scnexplanation{Связки отношения \textbf{\textit{темпоральное включение*}} связывают две \textit{временные сущности}, период существования одной из которых полностью включается в период существования второй.\\
Первым компонентом каждой связки отношения \textbf{\textit{темпоральное включение*}} является знак \textit{временной сущности}, \textit{длительность} существования которой больше.}
\scnsuperset{темпоральная часть*}
\scnsuperset{темпоральное включение без совпадения начальных и конечных моментов*}
\scnsuperset{темпоральное совпадение*}
\scnsuperset{темпоральное включение с совпадением начальных моментов*}
\scnsuperset{темпоральное включение с совпадением конечных моментов*}
		
\scnheader{темпоральная часть*}
\scnidtf{этап (период) заданной временной сущности*}
\scnidtf{этап процесса существования временной сущности*}
\scnsuperset{начальный этап*}
\scnsuperset{конечный этап*}
\scnsuperset{промежуточный этап*}
\scnsuperset{подпроцесс*}
\begin{scnindent}
		\scnrelfrom{первый домен}{процесс}
		\scnrelfrom{второй домен}{процесс}
\end{scnindent}
\end{SCn}
%		\scnrelfrom{описание примера}{
%			\scnfilescg{figures/sd_temp_entities/temporal_part.png}
%		}
%		\scnrelfrom{иллюстрация}{
%			\scnfileimage{\includegraphics[width=1\linewidth]{figures/sd_temp_entities/img_temporal_part.png}}}
Связки отношения \textbf{\textit{темпоральная часть*}} связывают две \textit{временные сущности}, одна из которых является частью другой, например, действие и одно из его поддействий. Соответственно, период существования одной из этих сущностей всегда будет включаться в период существования другой (большей).
			
В отличие от более общего отношения \textit{темпоральное включение*}, связки которого могут связывать любые \textit{временные сущности}, связки отношения \textbf{\textit{темпоральная часть*}} связывают только \textit{временные сущности}, одна из которых является частью другой.

\begin{SCn}
\scnheader{следует отличать*}
\begin{scnhaselementset}
	\scnitem{темпоральная часть*}
	\begin{scnindent}
		\scnsuperset{подпроцесс*}
	\end{scnindent}
	\scnitem{темпоральное включение*}
	\begin{scnindent}
		\scnnote{Связь \textit{темпорального включения*} может связывать абсолютно разные \textit{временные сущности}, существующие в общем случае в разных местах, а не только \textit{временные сущности}, одна из которых является частью другой. Хотя формально и можно объединить любые разные \textit{временные сущности} в одну общую \textit{временную сущность}, далеко не всегда имеет смысл это делать.}
	\end{scnindent}
\end{scnhaselementset}
		
\scnheader{темпоральное включение без совпадения начальных и конечных моментов*}
\scnidtf{строгое темпоральное включение*}
%		\scnrelfrom{описание примера}{
%			\scnfilescg{figures/sd_temp_entities/strict_temporal_inclusion.png}}
%		\scnrelfrom{иллюстрация}{
%			\scnfileimage{\includegraphics[width=1\linewidth]{figures/sd_temp_entities/img_strict_temporal_inclusion.png}}}
		
%\scnheader{темпоральное включение с совпадением начальных моментов*}
%\scnrelfrom{описание примера}{
%			\scnfilescg{figures/sd_temp_entities/temporal_include_with_match_start_points.png}}
%		\scnrelfrom{иллюстрация}{
%			\scnfileimage{\includegraphics[width=1\linewidth]{figures/sd_temp_entities/img_temporal_include_with_match_start_points.png}}}
		
%		\scnheader{темпоральное включение с совпадением конечных моментов*}
%		\scnrelfrom{описание примера}{
%			\scnfilescg{figures/sd_temp_entities/temporal_include_with_terminal_point_match.png}}
%		\scnrelfrom{иллюстрация}{
%			\scnfileimage{\includegraphics[width=1\linewidth]{figures/sd_temp_entities/img_temporal_include_with_terminal_point_match.png}}}
		
\scnheader{темпоральное совпадение*}
\scnidtf{совпадение начала и завершения*}
\scniselement{отношение эквивалентности}
		
\scnheader{темпоральное объединение*}
\scnidtf{преобразование нескольких временных сущностей в одну общую временную сущность, которая может оказаться прерывистой или даже дискретной*}
\scnrelboth{аналог}{объединение множеств*}
\scnnote{С формальной точки зрения объединять можно любые временные сущности. Но делать это надо только тогда, когда это имеет смысл, точно так же, как и в случае объединения множеств.}
%		\scnrelfrom{описание примера}{
%			\scnfilescg{figures/sd_temp_entities/temporal_union.png}}
%		\scnrelfrom{иллюстрация}{
%			\scnfileimage{\includegraphics[width=1\linewidth]{figures/sd_temp_entities/img_temporal_union.png}}}
		
\scnheader{темпоральная декомпозиция*}
\scnidtf{Темпоральное отношение, связывающее временную сущность и множество смежных во времени временных сущностей, которые являются темпоральными частями исходной сущности и результатом темпорального объединения которых является исходная сущность*}
\scnrelboth{аналог}{разбиение*}
%\scnrelfrom{описание примера}{
%	\scnfilescg{figures/sd_temp_entities/temporal_decomposition.png}}
%\scnrelfrom{иллюстрация}{
%	\scnfileimage{\includegraphics[width=1\linewidth]{figures/sd_temp_entities/img_temporal_decomposition.png}
%		}}
		
\scnheader{темпоральная смежность*}
\scnidtf{сразу позже*}
\scnidtf{смежность во времени*}
\scnidtf{строгая темпоральная последовательность (без темпорального промежутка)*}
\scnidtf{темпоральная последовательность без промежутка*}
%\scnrelfrom{описание примера}{
%			\scnfilescg{figures/sd_temp_entities/temporal_adjacency.png}}
%		\scnrelfrom{иллюстрация}{
%			\scnfileimage{\includegraphics[width=1\linewidth]{figures/sd_temp_entities/img_temporal_adjacency.png}
%		}}
		
		\scnheader{темпоральная последовательность с промежутком*}
		\scnidtf{позже*}
%		\scnrelfrom{описание примера}{
%			\scnfilescg{figures/sd_temp_entities/temporal_sequence_with_intermediate.png}}
%		\scnrelfrom{иллюстрация}{
%			\scnfileimage{\includegraphics[width=1\linewidth]{figures/sd_temp_entities/img_temporal_sequence_with_intermediate.png}}}
		
		\scnheader{темпоральная последовательность с пересечением*}
%		\scnrelfrom{описание примера}{
%			\scnfilescg{figures/sd_temp_entities/temporal_sequence_with_intersection.png}
%		}
%		\scnrelfrom{иллюстрация}{
%			\scnfileimage{\includegraphics[width=1\linewidth]{figures/sd_temp_entities/img_temporal_cross_sequence.png}
%		}}
		
\scnheader{начало\scnsupergroupsign}
\scnidtf{одновременность начинаний\scnsupergroupsign}
\scnidtf{класс одновременно начавшихся сущностей\scnsupergroupsign}
\scniselement{параметр}
\end{SCn}

Каждый элемент множества \textbf{начало} представляет собой класс \textit{временных сущностей}, у которых совпадает момент начала их существования. Конкретное значение данного \textit{параметра} может быть как \textit{точной величиной}, так и \textit{неточной величиной} или \textit{интервальной величиной}.

%		\scnrelfrom{описание примера}{
%			\scnfilescg{figures/sd_temp_entities/start.png}}
%		\scnaddlevel{1}
%		\scnexplanation{В данном примере \textbf{\textbf{\textit{ki}}} обозначает класс сущностей, начавших свое существование 19 февраля 2015 года по григорианскому календарю. Конкретные примеры таких сущностей -- \textbf{\textit{bi}} и \textbf{\textit{bj}}. \textbf{\textit{ti}} обозначает временную точку григорианского календаря, соответствующую 19 февраля 2015 года.}
%		\scnaddlevel{-1}

\begin{SCn}
\scnheader{завершение\scnsupergroupsign}
\scnidtf{конец\scnsupergroupsign}
\scnidtf{одновременность завершений\scnsupergroupsign}
\scnidtf{класс одновременно завершившихся сущностей\scnsupergroupsign}
\scniselement{параметр}
\end{SCn}

Каждый элемент множества \textbf{\textit{завершение}} представляет собой класс \textit{временных сущностей}, у которых совпадает конечный момент их существования (момент завершения существования). Конкретное значение данного \textit{параметра} может быть как \textit{точной величиной}, так и \textit{неточной величиной} или \textit{интервальной величиной}.


%\scnrelfrom{описание примера}{
%	\scnfilescg{figures/sd_temp_entities/completion.png}}
%\scnaddlevel{1}
%	\scnexplanation{В данном примере \textbf{\textit{ki}} обозначает класс сущностей, завершивших свое существование 21 февраля 2015 года по григорианскому календарю. Конкретные примеры таких сущностей -- \textbf{\textit{bi}} и \textbf{\textit{bj}}. \textbf{\textit{ti}} обозначает временную точку григорианского календаря, соответствующую 21 февраля 2015 года.}
%		\scnaddlevel{-1}

\begin{SCn}
\scnheader{одновременность\scnsupergroupsign}
\scnidtf{параметр, значениями (элементами) которого являются классы либо одновременно существующих (происходящих) \textit{точечных временных сущностей}, одновременность которых рассматривается с заданной степенью точности, либо одновременно начинающихся и заканчивающихся длительных процессов}
\end{SCn}

Важно отметить, что элементами некоторого значения параметра \textit{одновременности} с заданной точностью могут быть только те временные сущности, которые и начались, и завершились в течение периода времени, заданного указанным значением этого параметра, но при этом начало и завершение этих временных сущностей не обязательно должно совпадать с началом и завершением указанного периода времени. Так, например, можно ввести значение параметра \textit{одновременности} ``\textit{2022 год по Григорианскому календарю}'', элементами которого будут все временные сущности, начавшие и закончившие свое существовавшие в рамках 2022 года. При этом не обязательно, чтобы эти временные сущности начались именно в полночь 1 января 2022 года и закончились в полночь 1 января 2023 года, это могут быть временные сущности, существовавшие, например, в течение июля 2022 года.
%		\scnrelfrom{описание примера}{
%			\textit{}\scnfilescg{figures/sd_temp_entities/simultaneity.png}}
%		\scnaddlevel{1}
%		\scnnote{Некоторые значения параметра одновременности могут быть подмножествами других значений того же параметра. Семантика такой связи будет выражаться в том, что первое из указанных значений описывает \textit{одновременность} \textit{временных сущностей} с большей точностью. Так, в приведенном примере величина ``\textit{2002 год}'' описывает одновременность временных сущностей с точностью до года, а величина ``\textit{июль 2022 года}'' описывает одновременность временных сущностей с точностью до месяца. При этом очевидно, что сущности, входящие во величину ``\textit{июль 2022 года}'' будут также входить и в величину ``\textit{2022 год}'' (как например временная сущность $\bm{sk})$. В приведенном примере для простоты предполагается, что все измерения производятся по Григорианскому календарю.}
%		\scnaddlevel{-1}
		
%\scnheader{соединение значений ориентированного параметра*}
%\scnrelfrom{описание примера}{
%	\scnfilescg{figures/sd_temp_entities/temporal_values_join.png}
%	}
%	\scnaddlevel{1}
%		\scnnote{В приведенном примере множество сущностей, существовавших 10.01.2022, и множество сущностей, существовавших 12.01.2022, при помощи отношения \textit{соединение значений ориентированного параметра*} образуют множество сущностей, существовавших в период 10-12.02.2022.}
%		\scnaddlevel{-1}
\begin{SCn}
\scnheader{следует отличать*}
\begin{scnhaselementset}
\scnitem{темпоральное совпадение*}
	\begin{scnindent}
	\scniselement{отношение эквивалентности}
	\end{scnindent}
	\scnitem{одновременность\scnsupergroupsign}
	\begin{scnindent}
	\scnidtf{фактор-множество для отношения темпоральное совпадение*}
	\end{scnindent}
\end{scnhaselementset}
		
\scnheader{длительность\scnsupergroupsign}
\scnidtf{класс временных сущностей, имеющих одинаковую длительность\scnsupergroupsign}
\scniselement{параметр}
\scnhaselement{тысячелетие}
\scnhaselement{век}
\scnhaselement{год}
\scnhaselement{месяц}
\scnhaselement{день}
\scnhaselement{час}
\scnhaselement{минута}
\scnhaselement{секунда}
\end{SCn}

Каждый элемент множества \textbf{\textit{длительность}} представляет собой класс \textit{временных сущностей}, у которых совпадает длительность их существования. Конкретное значение данного \textit{параметра} может быть как \textit{точной величиной}, так и \textit{неточной величиной} или \textit{интервальной величиной}.

%\scnrelfrom{описание примера}{
%	\scnfilescg{figures/sd_temp_entities/duration.png}}
%	\scnaddlevel{1}
%	\scnexplanation{В данном примере \textbf{\textit{ki}} обозначает класс сущностей, существовавших в течение 2 месяцев. Конкретный пример такой сущности -- \textbf{\textit{bi}}.}
%	\scnaddlevel{-1}
		
	
\section{Формальная онтология ситуаций и событий, описывающих динамику баз знаний ostis-систем}
\label{sec_top_ontologies_dynamic}
%% Введение в пердметную область

Обработка информации в \textit{sc-памяти} (т.е. динамика базы знаний, хранимой в \textit{sc-памяти}) в конечном счете сводится:
\begin{textitemize}
	\item к появлению в \textit{sc-памяти} новых актуальных \textit{sc-узлов} и \textit{sc-коннекторов};
	\item к логическому удалению актуальных \textit{sc-элементов}, т.е. к переводу их в неактуальное состояние (это необходимо для хранения протокола изменения состояния базы знаний, в рамках которого могут описываться действия по удалению \textit{sc-элементов});
	\item к возврату логически удаленных \textit{sс-элементов} в статус актуальных (необходимость в этом может возникнуть при откате базы знаний к какой-нибудь ее прошлой версии);
	\item к физическому удалению \textit{sc-элементов};
	\item к изменению состояния актуальных (логически не удаленных \textit{sc-элементов}) -- \textit{sc-узел} может превратиться в \textit{sc-ребро}, \textit{sc-ребро} может превратиться в \textit{sc-дугу}, \textit{sc-дуга} может поменять направленность, \textit{sc-дуга} общего вида может превратиться в \textit{константную стационарную sc-дугу принадлежности}, и т.д.;
\end{textitemize}

Подчеркнем, что временный характер самого \textit{sc-элемента} (т.к. он может появиться или исчезнуть) никак не связан с возможно временным характером сущности, обозначаемой этим \textit{sc-элементом}. Т.е. временный характер самого sc-элемента и временный характер сущности, которую он обозначает -- абсолютно разные вещи.

Таким образом, следует четко отличать динамику внешнего мира, описываемого базой знаний, а динамику самой базы знаний (динамику внутреннего мира). При этом очень важно, чтобы описание динамики базы знаний также входило в состав каждой базы знаний.

К числу понятий, используемых для описания динамики базы знаний относятся:
\begin{textitemize}
	\item логически удаленный sc-элемент;
	\item сформированное множество;
	\item вычисленное число;
	\item сформированное высказывание;
\end{textitemize}

\begin{SCn}
\scnheader{Предметная область ситуаций и событий, описывающих динамику баз знаний ostis-систем}
\scnidtf{Предметная область, описывающая динамику базы знаний, хранимой в sc-памяти}
\scniselement{предметная область}
\scnsdmainclasssingle{ситуация}
%\scnsdclass{sc-элемент;наcтоящий sc-элемент;логически удаленный sc-элемент;число;невычисленное число;вычисленное число;понятие;основное понятие;неосновное понятие;понятие, переходящее из основного в неосновное;понятие, переходящее из неосновного в основное;специфицированная сущность;недостаточно специфицированная сущность;достаточно специфицированная сущность;средне специфицированная сущность;структура;файл;событие в sc-памяти*;элементарное событие в sc-памяти*;событие добавления sc-дуги, выходящей из заданного sc-элемента*;событие добавления sc-дуги, входящей в заданный sc-элемент*;событие добавления sc-ребра, инцидентного заданному sc-элементу*;событие удаления sc-дуги, выходящей из заданного sc-элемента*;событие удаления sc-дуги, входящей в заданный sc-элемент*;событие удаления sc-ребра, инцидентного заданному sc-элементу*;событие удаления sc-элемента*;событие изменения содержимого файла*}

\scnheader{sc-элемент}
\begin{scnreltoset}{разбиение}
	\scnitem{наcтоящий sc-элемент}
	\scnitem{логически удаленный sc-элемент}
\end{scnreltoset}

\scnheader{наcтоящий sc-элемент}
\scniselement{ситуативное множество}

\scnheader{логически удаленный sc-элемент}
\scniselement{ситуативное множество}

\scnheader{число}
\begin{scnsubdividing}
	\scnitem{невычисленное число}
	\scnitem{вычисленное число}
\end{scnsubdividing}

\scnheader{невычисленное число}
\scniselement{ситуативное множество}

\scnheader{вычисленное число}

%\scnheader{понятие}
%\scnsubdividing{основное понятие;неосновное понятие;понятие, переходящее из основного в неосновное;понятие, переходящее из неосновного в основное}

\scnheader{основное понятие}
\scnidtf{основное понятие для данной ostis-системы}
\scniselement{ситуативное множество}
\end{SCn}

К \textbf{\textit{основным понятиям}} относятся те понятия, которые активно используются в системе и могут быть ключевыми элементами sc-агентов. К \textbf{\textit{основным понятиям}} относятся также все неопределяемые понятия.

\begin{SCn}
\scnheader{неосновное понятие}
\scnidtf{дополнительное понятие}
\scnidtf{вспомогательное понятие}
\scnidtf{неосновное понятие для данной ostis-системы}
\scniselement{ситуативное множество}
\end{SCn}

Каждое \textbf{\textit{неосновное понятие}} должно быть строго определено на основе \textit{основных понятий}. Такие \textbf{\textit{неосновные понятия}} используются только для понимания и правильного восприятия вводимой информации, в том числе, для выравнивания онтологий. Ключевым элементом \textit{sc-агентов} \textbf{\textit{неосновные понятия}} быть не могут.

\begin{SCn}
\scntext{правило идентификации экземпляров}{В случае, когда некоторое понятие полностью перешло из \textit{основных понятий} в неосновные, то есть стало \textbf{\textit{неосновным понятием}}, и соответствующее ему \textit{основное понятие} (через которое оно определяется) в рамках некоторого внешнего языка имеет одинаковый с ним основной идентификатор, то к идентификатору \textbf{\textit{неосновного понятия}} спереди добавляется знак \#. Если при этом соответствуюшее \textit{основное понятие} имеет в идентификаторе знак \$, добавленный в процессе перехода, то этот знак удаляется. Если указанные понятия имеют разные основные идентификаторы в рамках этого внешнего языка, то никаких дополнительных средств идентификации не используется.

Например:\\
\textit{\#трансляция sc-текста}\\
\textit{\#scp-программа}}

\scnheader{понятие, переходящее из основного в неосновное}
\scniselement{ситуативное множество}

\scnheader{понятие, переходящее из неосновного в основное}
\scniselement{ситуативное множество}
\scntext{правило идентификации экземпляров}{В случае, когда текущее \textit{основное понятие} и соответствующее ему \textbf{\textit{понятие, переходящее из неосновного в основное}} в рамках некоторого внешнего языка имеют одинаковый основной идентификатор, то к идентификатору понятия, переходящего из неосновного в основное спереди добавляется знак \$. Если указанные понятия имеют разные основные идентификаторы в рамках этого внешнего языка, то никаких дополнительных средств идентификации не используется.

Например:\\
\textit{\$трансляция sc-текста}\\
\textit{\$scp-программа}}

\scnheader{специфицированная сущность}
\begin{scnsubdividing}
	\scnitem{недостаточно специфицированная сущность}
	\scnitem{достаточно специфицированная сущность}
	\scnitem{средне специфицированная сущность} %% Нужно ли это? Мне кажется нет
\end{scnsubdividing}
\end{SCn}

К \textbf{\textit{достаточно специфицированным сущностям}} предъявляются следующие требования:
\begin{textitemize}
\item если сущность не является понятием, то для нее должны быть указаны
\begin{textitemize}
	\item различные варианты обозначающих ее внешних знаков;
	\item классы, которым она принадлежит;
	\item связки, которыми она связана с другими сущностями (с указанием соответствующего отношения);
	\item значения параметров, которыми она обладает;
	\item те разделы базы знаний, в которых указанная сущность является ключевой;
	\item предметные области, в которые данная сущность входит.
\end{textitemize}
\item если специфицированная сущность является понятием, то для нее должны быть указаны:
\begin{textitemize}
	\item различные варианты внешних обозначений этого понятия;
	\item предметные области, в которых это понятие исследуется;
	\item определение понятия;
	\item пояснения
	\item разделы базы знаний, в которых это понятие является ключевым;
	\item описание примера -- пример экземпляра понятия.
\end{textitemize}
\end{textitemize}

%\scnheader{структура}
%\scnsubdividing{сформированная структура;несформированная структура}
%\scnsubdividing{недостаточно сформированная структура;достаточно сформированная структура;структура, имеющая средний уровень сформированности}
%% НИЧЕГО НЕ ПОНЯТНО

%\scnheader{файл}
%\scnsubdividing{недостаточно сформированный внутренний файл;достаточно сформированный внутренний файл;внутренний файл, имеющий средний уровень сформированности}

\begin{SCn}
\scnheader{событие в sc-памяти}
\scnsuperset{событие}

\scnheader{элементарное событие в sc-памяти}
\scnsubset{событие в sc-памяти}
\begin{scnsubdividing}
	\scnitem{событие добавления sc-дуги, выходящей из заданного sc-элемента}
	\scnitem{событие добавления sc-дуги, входящей в заданный sc-элемент}
	\scnitem{событие добавления sc-ребра, инцидентного заданному sc-элементу}
	\scnitem{событие удаления sc-дуги, выходящей из заданного sc-элемента}
	\scnitem{событие удаления sc-дуги, входящей в заданный sc-элемент}
	\scnitem{событие удаления sc-ребра, инцидентного заданному sc-элементу}
	\scnitem{событие удаления sc-элемента}
	\scnitem{событие изменения содержимого файла}
\end{scnsubdividing}
\end{SCn}

Под \textbf{\textit{элементарным событием в sc-памяти}} понимается такое \textit{событие}, в результате выполнения которого изменяется состояние только одного \textit{sc-элемента}.

\begin{SCn}
\scnheader{точечный процесс}
\scnidtf{атомарный процесс}
\scnidtf{условно мгновенный процесс}
\scnidtf{процесс, длительность которого в данном контексте считается несущественной (пренебрежимо малой)}

\scnheader{элементарный процесс}
\scnidtf{процесс, детализация которого на входящие в него подпроцессы в текущем контексте не производится}
\end{SCn}

%\section{Формальная онтология пространственных сущностей}
%% Введение в пердметную область
%\section{Формальная онтология материальных сущностей}
%% Введение в предметную область
%\section{Иерархическая система онтологий верхнего уровня}
%% Введение в пердметную область

%\input{author/references}
