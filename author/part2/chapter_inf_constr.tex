\chapter{Информационные конструкции и языки}
\chapauthortoc{Никифоров С.А.\\Гойло А.А.}
\label{chapter_inf_constr}

\vspace{-7\baselineskip}

\begin{SCn}
    \begin{scnrelfromlist}{автор}
        \scnitem{Никифоров С.А.}
        \scnitem{Гойло А.А.}
    \end{scnrelfromlist}

    \bigskip

    \begin{scnrelfromlist}{подраздел}
        \scnitem{\ref{section_information_construction_formalization}~\nameref{section_information_construction_formalization}}
        \scnitem{\ref{sec_external_information_constructs_external_lang}~\nameref{sec_external_information_constructs_external_lang}}
    \end{scnrelfromlist}

    \bigskip

    \begin{scnrelfromlist}{ключевое понятие}
        \scnitem{знак}
        \scnitem{знаковая конструкция}
        \scnitem{информационная конструкция}
        \scnitem{дискретная информационная конструкция}
        \scnitem{алфавит}
        \scnitem{язык}
        \scnitem{язык ostis-системы}
        \scnitem{смысловое представление информации}
        \scnitem{синтаксис языка}
        \scnitem{денотационная семантика языка}
        \scnitem{операционная семантика языка}
        \scnitem{идентификатор}
        \scnitem{файл}
        \scnitem{файл ostis-системы}
    \end{scnrelfromlist}

    \begin{scnrelfromlist}{ключевой класс параметров}
        \scnitem{параметр, заданный на множестве знаковых конструкций}
        \scnitem{параметр, заданный на множестве дискретных информационных конструкций}
        \scnitem{параметр, заданный на множестве дискретных информационных конструкций}
        \scnitem{параметр, заданный на множестве дискретных информационных конструкций}
        \scnitem{параметр, заданный на множестве языков}
    \end{scnrelfromlist}

    \begin{scnrelfromlist}{ключевой класс отношений}
        \scnitem{отношение, заданное на множестве знаков}
        \scnitem{отношение, заданное на множестве знаковых конструкций}
        \scnitem{отношение, заданное на множестве элементов дискретных информационных конструкций}
        \scnitem{отношение, заданное на множестве дискретных информационных конструкций}
        \scnitem{отношение, заданное на множестве языков}
        \scnitem{отношение, заданное на естественно-языковых файлах}
    \end{scnrelfromlist}

    \bigskip

    \begin{scnrelfromlist}{библиографическая ссылка}
        \scnitem{\scncite{Goylo2022}}
        \scnitem{\scncite{Golenkov2019b}}
        \scnitem{\scncite{Golenkov2014}}
        \scnitem{\scncite{Standart2021}}
        \scnitem{\scncite{Golenkov2019}}
    \end{scnrelfromlist}

\end{SCn}

\section{Формализация понятия информационной конструкции}
\label{section_information_construction_formalization}

В начале определим понятия знака и знаковой конструкции.
Ранее данные понятия рассматривались в работах (см.~\textit{\scncite{Goylo2022}}).

\begin{SCn}

    \scnheader{знак}
    \begin{scnrelfromset}{разбиение}
        \scnitem{знак, являющийся элементом \textit{дискретной информационной конструкции}}
        \scnitem{знак, являющийся неатомарным фрагментом \textit{дискретной информационной конструкции}}
    \end{scnrelfromset}

\end{SCn}

Под \textbf{\textit{знаком}} понимается фрагмент \textit{информационной конструкции}, который условно представляет (изображает) некоторую описываемую сущность, которую называют \textbf{\textit{денотатом знака}}.
При этом отсутствие \textit{знака}, обозначающего некоторую сущность, не означает отсутствие самой этой сущности.
Это означает только то, что мы даже не догадываемся о ее существовании и, следовательно, не приступили к ее исследованию.

Поскольку все \textit{знаки} являются \textit{дискретными информационными конструкциями}, множество \textit{знаков} является областью задания всех \textit{отношений, заданных на множестве дискретных \textit{информационных конструкций}}.
Тем не менее есть как минимум одно \textit{отношение}, специфичное для множества \textit{знаков}.

\begin{SCn}

    \scnheader{отношение, заданное на множестве знаков}
    \scnhaselement{синонимия знаков*}

\end{SCn}

\textit{знаки} являются \textbf{\textit{синонимичными}} в том и только в том случае, если они обозначают одну и ту же сущность.
При этом \textit{синонимичные знаки} могут быть \textit{синтаксически эквивалентными}, а могут и не быть таковыми.

\begin{SCn}

    \scnheader{знаковая конструкция}
    \scnsubset{дискретная информационная конструкция}

\end{SCn}

\textbf{\textit{знаковая конструкция}} --- \textit{дискретная информационная конструкция} которая в общем случае представляет собой конфигурацию \textit{знаков} и специальных фрагментов \textit{информационной конструкции}, обеспечивающих структуризацию конфигурации \textit{знаков} --- различного вида \textbf{\textit{разделителей}} и \textbf{\textit{ограничителей}}.
Для некоторых \textit{знаковых конструкций} и даже для некоторых \textit{языков} необходимость в \textit{разделителях} и \textit{ограничителях} может отсутствовать.

\begin{SCn}

    \scnheader{отношение, заданное на множестве знаковых конструкций}
    \scnhaselement{знак*}
    \scnhaselement{разделитель знаковой конструкции*}
    \scnhaselement{разделители знаковой конструкции*}
    \scnhaselement{ограничитель знаковой конструкции*}
    \scnhaselement{ограничители знаковой конструкции*}
    \scnhaselement{семантическая смежность знаковых конструкций*}
    \scnhaselement{конкатенация знаковых конструкций*}
        \begin{scnindent}
        \scnidtf{декомпозиция заданной знаковой конструкции на последовательность знаковых конструкций*}
        \end{scnindent}

\end{SCn}

\textbf{\textit{знак*}} --- \textit{бинарное ориентированное отношение}, связывающее \textit{знаковую конструкцию} со множеством всех \textit{знаков}, входящих в ее состав.

\textbf{\textit{семантическая смежность знаковых конструкций*}} --- \textit{бинарное отношение}, связывающее \textit{семантически смежные знаковые конструкции}.
При этом \textit{знаковые конструкции} \textit{$\bm{Ti}$} и \textit{$\bm{Tj}$} являются \textit{смежными} в том и только в том случае, если существуют \textit{синонимичные знаки} \textit{$\bm{Ti}$} и \textit{$\bm{Tj}$}, один из которых входит в состав \textit{$\bm{Ti}$}, а второй --- в состав \textit{$\bm{Tj}$}

\begin{SCn}

    \scnheader{класс знаковых конструкций\scnsupergroupsign}
    \scnhaselement{семантически элементарная знаковая конструкция}
    \scnhaselement{семантически связная знаковая конструкция}

\end{SCn}

\textbf{\textit{семантически элементарная знаковая конструкция}} --- \textit{знаковая конструкция}, описывающая некоторую (одну) связь между некоторыми \textit{знаками} сущностей.

\textbf{\textit{семантически связная знаковая конструкция}} --- \textit{знаковая конструкция}, которую можно представить в виде конкатенации \textit{семантически элементарных знаковых конструкций}, каждая из которых \textit{семантически смежна} предшествующей и последующей \textit{семантически элементарной знаковой конструкции}.

\begin{SCn}

    \scnheader{параметр, заданный на множестве знаковых конструкций\scnsupergroupsign}
    \scnhaselement{семантическая связность знаковых конструкций\scnsupergroupsign}
    \begin{scnindent}
        \scnhaselement{семантически связная знаковая конструкция}
        \scnhaselement{семантически несвязная знаковая конструкция}
    \end{scnindent}
    \scnhaselement{наличие разделителей и ограничителей\scnsupergroupsign}
    \begin{scnindent}
        \scnhaselement{знаковая конструкция, содержащая разделители и/или ограничители}
        \scnhaselement{знаковая конструкция без разделителей и ограничителей}
    \end{scnindent}

\end{SCn}

\textbf{\textit{информационная конструкция}} --- \textit{знаковая конструкция} (структура), содержащая некоторые сведения о некоторых сущностях.
Форма представления ("изображения"{}, "материализации"{}), форма структуризации (синтаксическая структура), а также \textit{смысл*} (денотационная семантика) \textit{\textit{информационных конструкций}} могут быть самыми различными.

\begin{SCn}

    \scnheader{информационная конструкция}
    \begin{scnrelfromset}{разбиение}
        \scnitem{sc-конструкция}
            \begin{scnindent}
            \scnidtf{информационная конструкция, принадлежащая внутреннему языку ostis-системы}
            \end{scnindent}
        \scnitem{информационна конструкция, не являющаяся sc-конструкцией}
            \begin{scnindent}
            \scnidtf{инородная (внешняя) для sc-памяти информационная конструкция}
            \scnidtf{информационная конструкция, представленная на внешнем языке ostis-систем}
            \begin{scnrelfromset}{разбиение}
                \scnitem{файл}
                \scnitem{информационная конструкция. не являющаяся ни sc-конструкцией, ни файлом}
            \end{scnrelfromset}
            \end{scnindent}
    \end{scnrelfromset}

\end{SCn}

\textbf{\textit{дискретная информационная конструкция}} --- \textit{информационная конструкция}, \textit{смысл} которой задается:
\begin{textitemize}
    \item множеством элементов (синтаксически атомарных фрагментов) этой \textit{информационной конструкции},
    \item \textit{алфавитом} этих элементов --- семейством классов \textit{синтаксически эквивалентных элементов информационной конструкции},
    \item принадлежностью каждого элемента \textit{информационной конструкции} соответствующему классу \textit{синтаксически эквивалентных элементов информационной конструкции},
    \item конфигурацией связей инцидентности между элементами \textit{информационной конструкции}.
\end{textitemize}

Следствием этого является то, что форма представления элементов \textit{дискретной информационной конструкции} для анализа ее смысла не требует уточнения.
Главным является:
\begin{textitemize}
    \item наличие простой процедуры выделения (сегментации) элементов \textit{\textit{дискретной информационной конструкции}},
    \item наличие простой процедуры установления синтаксической эквивалентности разных элементов \textit{дискретной информационной конструкции},
    \item наличие простой процедуры установления принадлежности каждого элемента \textit{дискретной информационной конструкции} соответствующему классу синтаксически эквивалентных элементов (то есть соответствующему элементу алфавита).
\end{textitemize}

\textbf{\textit{элементом дискретной информационной конструкции}} является синтаксически атомарный фрагмент (символ), входящий в состав \textit{дискретной информационной конструкции}.
Поскольку \textit{дискретные информационные конструкции} могут иметь общие элементы (атомарные фрагменты) и даже некоторые из них могут быть частями других \textit{информационных конструкций}, элемент \textit{дискретной информационной конструкции} может входить в состав сразу нескольких \textit{информационных конструкций}.

Далее перейдем к рассмотрению понятий \textbf{\textit{синтаксиса информационной конструкции}} и \textbf{\textit{денотационной семантики}}.

\begin{SCn}

    \scnheader{синтаксис}
    \scnsuperset{синтаксис информационной конструкции}
        \begin{scnindent}
        \scnidtf{описание структуры информационной конструкции, являющееся минимально достаточным для ее семантически эквивалентное смысловое представление}
        \end{scnindent}
    \scnsuperset{синтаксис языка}
        \begin{scnindent}
        \scnidtf{синтаксические правила языка}
        \end{scnindent}

    \scnheader{денотационная семантика}
    \scnsuperset{денотационная семантика информационной конструкции}
    \begin{scnindent}
        %оно нашим правлам не противоречит?
        \scnidtf{отображение синтаксической структуры информацинной конструкции в ее семантически эквивалентное смысловое представление}
    \end{scnindent}
    \scnsuperset{денотационная семантика языка}
    \begin{scnindent}
        \scnidtf{объединение семантических окрестностей основных понятий заданного языка и конъюнкция кванторных высказываний, являющихся семантическими правилами заданного языка, то есть правилами, которым должны удовлетворять семантически правильные смысловые информационные конструкции, соответствующие (семантические эквивалентные) синтаксически правильным конструкциям (текстам) заданного языка}
    \end{scnindent}

    \bigskip
    \bigskip
    \scnheader{понимание}
    \scnsubset{действие}
    \scnidtf{трансляция информационной конструкции на язык смыслового представления знаний}
    \scnidtf{процесс построения семантически эквивалентного данной информационной конструкции смыслового представления}
    \scnidtf{описание структуры информационной конструкции, являющееся минимально достаточным для ее семантически эквивалентное смысловое представление}

    \scnheader{семантический анализ}
    \scnsubset{действие}
    \scnidtf{построение синтаксического описания}

    \scnheader{трансляция}
    \scnsubset{действие}
    \scnidtf{перевод}
    \scnidtf{построение семантически эквивалентной информационной конструкции}

\end{SCn}

Далее рассмотрим отношения, заданные на множестве элементов \textit{дискретных информационных конструкций}.

\begin{SCn}

    \scnheader{отношение, заданное на множестве элементов \textit{дискретных информационных конструкций}}
    \scnhaselement{элемент дискретной информационной конструкции*}
    \scnhaselement{синтаксическая эквивалентность элементов дискретных \textit{информационных конструкций}*}
    \scnhaselement{инцидентность элементов дискретных \textit{информационных конструкций}*}

\end{SCn}

\textbf{\textit{элемент дискретной информационной конструкции*}} --- \textit{бинарное ориентированное отношение}, каждая пара которого связывает (1) \textit{знак} некоторой \textit{дискретной информационной конструкции} и (2) \textit{знак} одного из элементов этой \textit{дискретной информационной конструкции}*.

\textbf{\textit{cинтаксическая эквивалентность элементов дискретных информационных конструкций*}} --- \textit{отношение}, связывающее \textit{синтаксически эквивалентные элементы} (атомарные фрагменты) одной и той же или разных \textit{дискретных информационных конструкций}, то есть элементы, принадлежащие одному и тому же классу \textit{синтаксически эквивалентных элементов дискретных информационных конструкций*}.

\textbf{\textit{инцидентность элементов дискретных информационных конструкций*}} для \textit{линейных информационных конструкций} --- это последовательность \textit{элементов} (символов), входящих в состав этих \textit{информационных конструкций}.
Для \textit{дискретных информационных конструкций}, конфигурация которых имеет нелинейный характер, \textit{отношение инцидентности} их элементов может быть разбито на несколько частных \textit{отношений инцидентности}, каждое из которых является \myuline{подмножеством} объединенного \textit{отношения инцидентности}.
Например, для двухмерных дискретных \textit{информационных конструкций} это (1) \textit{инцидентность элементов информационных конструкций} "по горизонтали"{} и (2) \textit{инцидентность элементов информационных конструкций} "по вертикали"{}.

Далее рассмотрим отношения, заданные на множестве \textit{дискретных информационных конструкций}.

\begin{SCn}

\scnheader{отношение, заданное на множестве \textit{дискретных информационных конструкций}}
\scnhaselement{неэлементарный фрагмент \textit{дискретной информационной конструкции}*}
\scnhaselement{алфавит \textit{дискретной информационной конструкции}*}
\scnhaselement{первичная синтаксическая структура \textit{дискретной информационной конструкции}*}
\scnhaselement{синтаксическая эквивалентность дискретных \textit{информационных конструкций}*}
\scnhaselement{копия \textit{дискретной информационной конструкции}*}
\scnhaselement{семантическая эквивалентность дискретных \textit{информационных конструкций}*}
\scnhaselement{семантическое расширение \textit{дискретной информационной конструкции}*}
\scnhaselement{синтаксис информационной конструкции*}
\scnhaselement{смысл*}
\scnhaselement{операционная семантика информационной конструкции*}

\end{SCn}

\textbf{\textit{неэлементарный фрагмент дискретной информационной конструкции*}} --- \textit{бинарное ориентированное отношение}, связывающее заданную \textit{дискретную информационную конструкцию} с \textit{дискретной информационной конструкцией}, которая является \myuline{подструктурой} для нее, в состав которой входит (1) подмножество элементов заданной \textit{информационной конструкции} и, соответственно, (2) подмножество пар инцидентности \textit{элементов заданной информационной конструкции}.

\textbf{\textit{алфавит дискретной информационной конструкции*}} --- \textit{бинарное отношение}, связывающее \textit{дискретную информационную конструкцию} с семейством попарно непересекающихся \textit{\myuline{классов} синтаксически эквивалентных элементов} заданной \textit{дискретной информационной конструкции}*

\textbf{\textit{первичная синтаксическая структура дискретной информационной конструкции*}} --- \textit{бинарное ориентированное отношение}, связывающее \textit{дискретную информационную конструкцию} с \textit{графовой структурой}, которая полностью описывает ее конфигурацию и которая включает: (1) \textit{знаки} всех тех классов \textit{синтаксически эквивалентных элементов}, которым принадлежат элементы описываемой \textit{дискретной информационной конструкции}, (2) \textit{знаки} всех \textit{элементов} (атомарных фрагментов) описываемой \textit{информационной конструкции}, (3) пары, описывающие \textit{инцидентность элементов} описываемой \textit{информационной конструкции}, (4) пары, описывающие принадлежность \textit{элементов} описываемой \textit{информационной конструкции} соответствующим \textit{классам синтаксически эквивалентных элементов} этой \textit{информационной конструкции}.

\textbf{\textit{синтаксическая эквивалентность дискретных информационных конструкций*}}: \textit{дискретные информационные конструкции} $\bm{Ti}$ и $\bm{Tj}$ являются синтаксически эквивалентными в том и только в том случае, если между конструкцией \textit{$\bm{Ti}$} и конструкцией \textit{$\bm{Tj}$} существует \myuline{изоморфизм}, в рамках которого каждому элементу конструкции \textit{$\bm{Ti}$} соответствует \textit{синтаксически эквивалентный элемент информационной конструкции} \textit{$\bm{Tj}$}, то есть элемент, принадлежащий тому же классу \textit{синтаксически эквивалентных} \textit{элементов дискретных информационных конструкций}.
Обратное утверждение также является верным.

\textbf{\textit{копия дискретной информационной конструкции*}} --- \textit{бинарное ориентированное отношение}, которое связывает \textit{дискретную информационную конструкцию} с \textit{дискретной информационной конструкцией}, которая не только \textit{синтаксически эквивалентна} ей, но и содержит информацию о форме представления \textit{элементов} данной копируемой \textit{информационной конструкции*}.

\begin{SCn}
    \scnheader{копия \textit{дискретной информационной конструкции}*}
    \scnsubset{синтаксическая эквивалентность дискретных \textit{информационных конструкций}*}
\end{SCn}

\textbf{\textit{семантическая эквивалентность дискретных информационных конструкций*}}: \textit{информационная конструкция} \textit{$\bm{Ti}$} и \textit{информационная конструкция} \textit{$\bm{Tj}$} являются \myuline{семантически эквивалентными} в том и только в том случае, если \myuline{каждая} сущность (в том числе, и каждая связь между сущностями), описываемая в \textit{информационной конструкции} \textit{$\bm{Ti}$} описывается также и в \textit{информационной конструкции} \textit{$\bm{Tj}$}.
Обратное утверждение также является верным.

\textbf{\textit{cемантическое расширение дискретной информационной конструкции*}}: \textit{информационная конструкция} \textit{$\bm{Tj}$} является \textit{семантическим расширением дискретной информационной конструкции} \textit{$\bm{Ti}$} в том и только в том случае, если \myuline{каждая} сущность, описываемая в \textit{$\bm{Ti}$}, описывается также и в \textit{$\bm{Tj}$}, но обратное неверно.

\textbf{\textit{синтаксис информационной конструкции*}} --- описание того, из каких частей состоит заданная \textit{информационная конструкция} и как эти части (фрагменты) связаны между собой.

\textbf{\textit{смысл*}} (\textit{денотационная семантика информационной конструкции*}) --- \textit{бинарное ориентированное отношение}, каждая пара которого связывает некоторую \textit{информационную конструкцию} с явным (формальным) представлением того, какие сущности описывает данная \textit{информационная конструкция} и как эти сущности связаны между собой.

\textbf{\textit{операционная семантика информационной конструкции*}} --- \textit{бинарное ориентированное отношение}, каждая пара которого связывает \textit{знак} некоторой \textit{информационной конструкции} со множеством правил ее трансформации --- описанием того, на основании каких правил можно осуществлять действия по преобразования (обработке, трансформации) заданной \textit{информационной конструкции}, оставляя ее в рамках класса синтаксически и семантически правильных \textit{информационных конструкций}.

\begin{SCn}

    \scnheader{операционная семантика информационной конструкции*}
    \scnrelfrom{второй домен}{операционная семантика информационной конструкции}

\end{SCn}

Далее рассмотрим заданные на множестве дискретных \textit{информационных конструкций} соответствия.

\begin{SCn}

    \scnheader{соответствие, заданное на множестве дискретных \textit{информационных конструкций}}
    \scnhaselement{соответствие между синтаксической структурой информационной конструкции и смыслом этой конструкции*}
    \begin{scnindent}
        \scnsubset{соответствие*}
    \end{scnindent}

\end{SCn}

\textbf{\textit{cоответствие, заданное на множестве дискретных информационных конструкций}} --- множество ориентированных пар, первым компонентом которых является ориентированная пара, состоящая из (1) \textit{знака} \textit{синтаксической структуры} некоторой \textit{информационной конструкции} и (2) \textit{знака} смысловой структуры этой \textit{конструкции}, а вторым компонентом которых является множество ориентированных пар, связывающих фрагменты синтаксической структуры заданной \textit{информационной конструкции} (которые описывают либо структуру фрагментов заданной конструкции, либо связи между фрагментами этой конструкции) с теми фрагментами смысловой структуры заданной \textit{информационной конструкции}, которые семантически эквивалентны либо синтаксически представленным фрагментам заданной \textit{информационной конструкции}, либо синтаксически представленным связям между такими фрагментами.

\begin{SCn}

    \scnheader{параметр, заданный на множестве дискретных \textit{информационных конструкций}\scnsupergroupsign}
    \scnhaselement{размерность дискретных \textit{информационных конструкций}\scnsupergroupsign}
    \begin{scnindent}
        \scnidtf{типология дискретных \textit{информационных конструкций}, определяемая их размерностью}
        \scnhaselement{линейная информационная конструкция}
        \scnhaselement{двухмерная информационная конструкция}
        \scnhaselement{трехмерная информационная конструкция}
        \scnhaselement{четырехмерная информационная конструкция}
        \scnhaselement{графовая информационная конструкция}
    \end{scnindent}

\end{SCn}

\textbf{\textit{линейная информационная конструкция}} --- \textit{дискретная информационная конструкция}, каждый \textit{элемент} которой может иметь не более двух инцидентных ему элементов (один слева, другой справа).

\textbf{\textit{двухмерная информационная конструкция}} --- \textit{дискретная информационная конструкция}, каждый \textit{элемент} которой может иметь не более четырех инцидентных ему элементов (слева-справа, сверху-снизу).

\textbf{\textit{трехмерная информационная конструкция}} --- \textit{дискретная информационная конструкция}, каждый \textit{элемент} которой может иметь не более шести инцидентных ему элементов (слева-справа, сверху-снизу, сзади-спереди).

\textbf{\textit{четырехмерная информационная конструкция}} --- \textit{дискретная информационная конструкция}, каждый \textit{элемент} которой может иметь не более восьми инцидентных ему элементов (например, слева-справа, сверху-снизу, сзади-спереди, раньше-позже).

\textbf{\textit{графовая информационная конструкция}} --- \textit{дискретная информационная конструкция}, множество элементов которой разбивается на два подмножества --- связки и узлы.
При этом узлы могут иметь \myuline{неограниченное} количество инцидентных им связок.
В некоторых \textit{графовых информационных конструкциях} и связки могут иметь неограниченное количество инцидентных им других связок.

\begin{SCn}

    \scnheader{параметр, заданный на множестве дискретных \textit{информационных конструкций}\scnsupergroupsign}
    \scnhaselement{типология дискретных \textit{информационных конструкций}, определяемая их носителем\scnsupergroupsign}
    \begin{scnindent}
        \scnhaselement{некомпьютерная форма представления дискретных \textit{информационных конструкций}}
        \begin{scnindent}
            \scnsuperset{аудио-сообщение}
            \scnsuperset{информационная конструкция, представленная на языке жестов}
            \scnsuperset{информационная конструкция, представленная в письменной форме}
        \end{scnindent}
        \scnhaselement{файл}
    \end{scnindent}

\end{SCn}

Представление \textit{информационных конструкций} в виде \textit{файлов} ориентировано на представление \textit{\myuline{дискретных}} (!) \textit{информационных конструкций}.
Поэтому "файловое"{} представление недискретных \textit{информационных конструкций} (например, различного рода сигналов) предполагает "дискретизацию"{} таких конструкций, то есть преобразование их в дискретные.
Так преобразуются аудио-сигналы (в частности, речевые сообщения), изображения, видео-сигналы и другие.

\begin{SCn}

    \scnheader{параметр, заданный на множестве дискретных \textit{информационных конструкций}\scnsupergroupsign}
    \scnhaselement{уровень унификации представления синтаксически эквивалентных элементов дискретных \textit{информационных конструкций}\scnsupergroupsign}
    \begin{scnindent}
        \scnhaselement{дискретная информационная конструкция с низким уровнем унификации представления элементов}
        \begin{scnindent}
            \scnsuperset{аудио-сообщение}
            \scnsuperset{информационная конструкция, представленная на языке жестов}
            \scnsuperset{рукопись или ее копия}
        \end{scnindent}
        \scnhaselement{дискретная информационная конструкция с высоким уровнем унификации представления элементов}
        \begin{scnindent}
            \scnsuperset{печатный текст}
            \scnsuperset{файл}
        \end{scnindent}
    \end{scnindent}

\end{SCn}

Уровень \textbf{\textit{унификации представления синтаксически эквивалентных элементов дискретных информационных конструкций\scnsupergroupsign}} --- уровень "членораздельности"{} \textit{дискретных информационных конструкций}.

Чем выше \textit{уровень унификации представления элементов дискретных информационных конструкций}, тем проще реализуется:
\begin{textitemize}
    \item процедура выделения (сегментации) элементов \textit{дискретной информационной конструкции},
    \item процедура установления синтаксической эквивалентности этих элементов,
    \item процедура их распознавания, то есть процедура установления их принадлежности соответствующим классам синтаксически эквивалентных элементов.
\end{textitemize}

Уточнив понятия \textit{знака}, \textit{знаковой конструкции}, \textit{информационной конструкции}, \textit{дискретной информационной конструкции} и рассмотрев соответствующие отношения, можно перейти к формализации понятия \textit{язык}.

\textbf{\textit{язык}} --- класс \textit{знаковых конструкций}, для которого существуют:
\begin{textitemize}
    \item общие правила их построения,
    \item общие правила их соотнесения с теми сущностями и конфигурациями сущностей, которые описываются (отражаются) указанными \textit{знаковыми конструкциями}.
\end{textitemize}

\begin{SCn}

    \scnheader{язык}
    \begin{scnrelfromset}{разбиение}
        \scnitem{язык, у которого все знаки, входящие в состав его знаковых конструкций, являются элементарными фрагментами этих конструкций}
        \scnitem{язык, у которого знаки, входящие в состав его знаковых конструкций, в общем случае не являются элементарными фрагментами этих конструкций}
        \begin{scnindent}
	        \begin{scnrelfromset}{разбиение}
	            \scnitem{язык, знаковые конструкции которого содержат разделители и ограничители}
	            \scnitem{язык, знаковые конструкции которого не содержат разделителей и ограничителей}
    	    \end{scnrelfromset}
	    \end{scnindent}
    \end{scnrelfromset}

\end{SCn}

Для \textit{языков, у которого все знаки, входящие в состав его знаковых конструкций, являются элементарными фрагментами этих конструкций} существенно упрощаются методы обработки их текстов.

\textbf{\textit{язык, знаковые конструкции которого не содержат разделителей и ограничителей}} --- \textit{язык}, \textit{знаковые конструкции} такого языка состоят \myuline{только} из \textit{знаков}.

\textbf{\textit{отношение, заданное на множестве языков}} --- \textit{отношение}, область определения которого включает в себя множество всевозможных \textit{языков}.

\textbf{\textit{текст заданного языка*}} --- \textit{бинарное отношение}, связывающее \textit{язык} и синтаксически правильную (правильно построенную) \textit{знаковую конструкцию} данного языка.
\textbf{\textit{синтаксически корректная знаковая конструкция для заданного языка*}} --- \textit{бинарное отношение}, связывающее \textit{язык} и \textit{знаковую конструкцию}, не содержащая синтаксических ошибок для данного \textit{языка}.

\begin{SCn}
    \scnheader{отношение, заданное на множестве языков}
    \scnidtf{отношение, область определения которого включает в себя множество всевозможных языков}
    \scnhaselement{текст заданного языка*}
    \begin{scnindent}
        \scneq{{\normalfont(}синтаксически корректная знаковая конструкция для заданного языка* $\cap$ синтаксически целостная знаковая конструкция для заданного языка*{\normalfont)}}
    \end{scnindent}
    \scnhaselement{синтаксически корректная знаковая конструкция для заданного языка*}
    \scnhaselement{синтаксически целостная знаковая конструкция для заданного языка*}
    \scnhaselement{синтаксически неправильная знаковая конструкция для заданного языка*}
    \begin{scnindent}
        \scneq{{\normalfont(}синтаксически некорректная знаковая конструкция для заданного языка* $\cup$ синтаксически нецелостная знаковая конструкция для заданного языка*{\normalfont)}}
        \scnsuperset{синтаксически некорректная знаковая конструкция для заданного языка*}
        \scnsuperset{синтаксически нецелостная знаковая конструкция для заданного языка*}
    \end{scnindent}
    \scnhaselement{знание, представленное на заданном языке*}
    \begin{scnindent}
        \scnidtf{семантически правильный текст заданного языка*}
        \scneq{(семантически корректный текст заданного языка* $\cap$ семантически целостный текст заданного языка*)}
        \scnidtf{истинный текст заданного языка*}
        \scnidtf{истинное высказывание, представленное на заданном языке*}
    \end{scnindent}
    \scnhaselement{семантически корректный текст заданного языка*}
    \begin{scnindent}
        \scnidtf{текст заданного языка, не содержащий семантических ошибок, противоречащих признанным закономерностям и фактам*}
    \end{scnindent}
    \scnhaselement{семантически целостный текст заданного языка*}
    \begin{scnindent}
        \scnidtf{текст заданного языка, содержащий достаточную информацию для установления его истинности*}
    \end{scnindent}
    \scnhaselement{семантически неправильный текст для заданного языка*}
    \begin{scnindent}
        \scneq{(семантически некорректный текст для заданного языка* $\cup$ семантически нецелостный текст для заданного языка*)}
        \scnsuperset{семантически некорректный текст для заданного языка*}
        \scnsuperset{семантически нецелостный текст для заданного языка*}
    \end{scnindent}
    \scnhaselement{алфавит*}
    \begin{scnindent}
        \scnidtf{алфавит заданной информационной конструкции или заданного языка*}
        \scnidtf{семейство классов синтаксически эквивалентных элементов (элементарных фрагментов) заданной информационной конструкции или \textit{информационных конструкций} заданного языка*}
    \end{scnindent}
    \scnhaselement{семейство классов синтаксически эквивалентных разделителей*}
    \begin{scnindent}
        \scnidtf{семейство классов синтаксически эквивалентных разделителей, входящих в состав заданной информационной конструкции или в состав \textit{информационных конструкций} заданного языка*}
    \end{scnindent}
    \scnhaselement{семейство классов синтаксически эквивалентных ограничителей*}
    \begin{scnindent}
        \scnidtf{семейство классов синтаксически эквивалентных ограничителей, входящих в состав заданной информационной конструкции или в состав \textit{информационных конструкций} заданного языка*}
    \end{scnindent}
    \scnhaselement{синтаксис языка*}
    \begin{scnindent}
        \scnidtf{быть теорией правильно построенных \textit{информационных конструкций}, принадлежащих заданному языку*}
        \scnidtf{определение понятия правильно построенной информационной конструкции заданного языка*}
        \scnidtf{синтаксические правила заданного языка*}
        \scnidtf{быть синтаксическими правилами заданного языка*}
        \scnidtf{\textit{бинарное ориентированное отношение}, каждая пара которого связывает \textit{знак} некоторого \textit{языка} с описанием синтаксически выделяемых классов фрагментов конструкций заданного языка, с описанием отношений, заданных на этих классах и с конъюнкцией кванторных высказываний, являющихся \textit{синтаксическими правилами заданного языка}, то есть правилами, которым должны удовлетворять все \textit{синтаксические правильные (правильно построенные) конструкции (тексты)} указанного языка*}
        \scntext{примечание}{При представлении синтаксиса (синтаксических правил) заданного \textit{языка} используются все те понятия, которые вводятся для представления синтаксических структур \textit{информационных конструкций}, принадлежащих указанному \textit{языку}. Это и синтаксически выделяемые классы фрагментов указанных \textit{информационных конструкций}, и \textit{отношения}, заданные на множестве таких фрагментов.}
        \scnrelfrom{второй домен}{синтаксис языка}
    \end{scnindent}
    \scnhaselement{описание синтаксических понятий языка*}
    \begin{scnindent}
        \scnidtf{описание синтаксически выделяемых классов фрагментов конструкций заданного языка*}
        \scnrelfrom{второй домен}{описание синтаксических понятий языка}
        \begin{scnindent}
            \scnrelto{обобщенное включение}{синтаксис языка}
        \end{scnindent}
    \end{scnindent}
    \scnhaselement{синтаксические правила языка*}
    \begin{scnindent}
        \scnidtf{синтаксические правила заданного языка*}
        \scnrelfrom{второй домен}{синтаксические правила языка}
    \end{scnindent}
    \scnhaselement{денотационная семантика языка*}
    \begin{scnindent}
        \scnidtf{быть теорией морфизмов, связывающих правильно построенные \textit{информационные конструкции} заданного языка с описываемыми конфигурациями описываемых сущностей*}
    \end{scnindent}

    \scnhaselement{денотационная семантика языка*}
    \begin{scnindent}
        \scnidtf{семантические правила заданного языка*}
        \scnidtf{быть семантическими правилами заданного языка*}
        \scnidtf{\textit{бинарное ориентированное отношение}, каждая пара которого связывает \textit{знак} некоторого \textit{языка} с описанием основных семантических понятий заданного \textit{языка} и конъюнкцией кванторных высказываний, являющихся \textit{семантическими правилами} заданного \textit{языка}, то есть правилами, которым должны удовлетворять семантически правильные \myuline{смысловые} \textit{информационные конструкции}, соответствующие (\textit{семантические эквивалентные}) синтаксически правильным конструкциям (текстам) заданного \textit{языка}*}
        \scntext{примечание}{При формулировке семантических правил заданного \textit{языка} используются понятия, введенные в рамках базовых онтологий (онтологий самого высокого уровня, в которых рассматриваются самые общие классы описываемых сущностей, самые общие \textit{отношения} и \textit{параметры}).}
        \scnrelfrom{второй домен}{денотационная семантика языка}
    \end{scnindent}
    \scnhaselement{описание семантических понятий языка*}
    \begin{scnindent}
        \scnrelfrom{второй домен}{описание семантических понятий языка}
    \end{scnindent}
    \scnhaselement{семантические правила языка*}
    \begin{scnindent}
        \scnrelfrom{второй домен}{семантические правила языка}
    \end{scnindent}
    \bigskip
    \scnhaselement{семантическая эквивалентность языков*}
    \begin{scnindent}
        \scnidtf{быть семантически эквивалентными языками*}
        \scntext{определение}{\textit{язык} \textit{$\bm{Li}$} и \textit{язык} \textit{$\bm{Lj}$} будем считать \textit{семантически эквивалентными языками*} в том и только в том случае, если для каждого текста, принадлежащего \textit{$\bm{Li}$}, существует \textit{семантически эквивалентный текст*}, принадлежащий \textit{$\bm{Lj}$}, и наоборот.}
    \end{scnindent}
    \scnhaselement{семантическое расширение языка*}
    \begin{scnindent}
        \scnrelboth{обратное отношение}{семантическое сужение языка*}
        \scntext{определение}{\textit{язык} \textit{$\bm{Lj}$} будем считать \textit{семантическим расширением*} \textit{языка} \textit{$\bm{Li}$} в том и только в том случае, есть ли для каждого текста, принадлежащего \textit{$\bm{Li}$}, существует \textit{семантически эквивалентный текст*}, принадлежащий \textit{$\bm{Lj}$}, но обратное неверно.}
    \end{scnindent}
    \scnhaselement{синтаксическое расширение языка*}
    \begin{scnindent}
        \scnidtf{быть семантически эквивалентным надмножеством заданного языка*}
        \scntext{определение}{\textit{язык} \textit{$\bm{Lj}$} будем считать \textit{синтаксическим расширением*} \textit{языка} \textit{$\bm{Li}$} в том и только в том случае, если:
            \begin{scnitemize}
                \item \textit{$\bm{L_j} \supset \bm{Li}$} (то есть все тексты языка \textit{$\bm{Li}$} являются также и текстами языка \textit{$\bm{Lj}$}, но обратное неверно);
                \item \textit{язык} \textit{$\bm{Lj}$} и \textit{язык} \textit{$\bm{Li}$} являются \textit{семантически эквивалентными языками*}.
            \end{scnitemize}
        }
    \end{scnindent}
    \scnhaselement{синтаксическое ядро языка*}
    \begin{scnindent}
        \scnidtf{быть синтаксическим ядром заданного языка*}
        \scnidtf{быть семантически эквивалентным подмножеством заданного \textit{языка}, имеющим минимальную синтаксическую сложность*}
    \end{scnindent}
    \scnhaselement{направление синтаксического расширения ядра заданного языка*}
    \begin{scnindent}
        \scnidtf{быть правилом трансформации \textit{информационных конструкций}, принадлежащих заданному \textit{языку}, которое описывает одно из направлений перехода от множества конструкций ядра этого языка ко множеству всех принадлежащих ему \textit{информационных конструкций}*}
    \end{scnindent}
    \scnhaselement{операционная семантика языка*}
    \begin{scnindent}
        \scnidtf{\textit{бинарное ориентированное отношение}, каждая пара которого связывает знак некоторого \textit{языка} со множеством правил трансформации текстов этого языка*}
        \scnrelfrom{второй домен}{операционная семантика языка}
    \end{scnindent}
    \scnhaselement{внутренний язык*}
    \begin{scnindent}
        \scnidtf{быть внутренним \textit{языком} для заданной системы, основанной на обработке информации, или заданного множества таких систем*}
        \scnidtf{быть \textit{языком} внутреннего представления информации в памяти заданной системы, основанной на обработке информации или заданного класса таких систем*}
    \end{scnindent}
    \scnhaselement{внешний язык*}
    \begin{scnindent}
        \scnidtf{быть внешним \textit{языком} для заданной системы, основанной на обработке информации, или заданного множества таких систем*}
        \scnidtf{быть \textit{языком}, используемым для обмена информацией заданной системы, основанной на обработке информации, или заданного множества таких систем с другими системами, основанными на обработке информации, (в том числе, и с себе подобными)*}
    \end{scnindent}
    \scnhaselement{используемый язык*}
    \begin{scnindent}
        \scneq{{\normalfont(}внутренний язык* $\cup$ внешний язык*{\normalfont)}}
        \scnidtf{\textit{язык}, используемый заданной системой, основанной на обработке информации или заданного множества таких систем*}
        \scnidtf{\textit{язык}, носителем которого является (которым владеет) данная система, основанная на обработке информации}
    \end{scnindent}

\end{SCn}

\textit{параметр, заданный на множестве языков\scnsupergroupsign} --- \textit{семейство классов эквивалентности} \textit{языков}, трактуемой в контексте того или иного свойства (характеристики), присущего \textit{языкам}.

\begin{SCn}

    \scnheader{параметр, заданный на множестве языков\scnsupergroupsign}
    \scnhaselement{семантическая мощность языка\scnsupergroupsign}
    \begin{scnindent}
        \scnidtf{класс \textit{языков}, семантически эквивалентных друг другу}
        \scnhaselement{универсальный язык}
        \begin{scnindent}
            \scnidtf{класс всевозможных универсальных языков}
            \scntext{примечание}{Очевидно, что все универсальные \textit{языки} (если они действительно таковыми являются, а не только претендуют на это) семантически эквивалентны друг другу, то есть имеют одинаковую семантическую мощность.}
        \end{scnindent}
    \end{scnindent}
    \scnhaselement{уровень синтаксической сложности представления знаков в текстах языка\scnsupergroupsign}
    \begin{scnindent}
        \scnhaselement{язык, в текстах которого все знаки представлены синтаксически элементарными фрагментами}
        \scnhaselement{язык, в текстах которого знаки в общем случае представлены синтаксически неэлементарными фрагментами}
    \end{scnindent}
    \scnhaselement{использование разделителей и ограничителей в текстах языка\scnsupergroupsign}
    \begin{scnindent}
        \scnhaselement{язык, в текстах которого не используются разделители и ограничители}
        \scnhaselement{язык, в текстах которого используются разделители и ограничители}
    \end{scnindent}
    \scnhaselement{уровень сложности процедуры установления синонимии знаков в текстах языка\scnsupergroupsign}
    \begin{scnindent}
        \scnhaselement{язык, в рамках каждого текста которого синонимичные знаки отсутствуют}
        \begin{scnindent}
            \scntext{пояснение}{В текстах такого языка знак каждой описываемой сущности входит \myuline{однократно}.}
        \end{scnindent}
        \scnhaselement{язык, в рамках которого синонимичные знаки представлены синтаксически эквивалентными фрагментами текстов}
        \scnhaselement{флективный язык}
        \begin{scnindent}
            \scnidtf{язык, в рамках которого синонимичные знаки могут быть представлены синтаксически неэквивалентными фрагментами, но фрагментами, являющимися модификациями некоторого "ядра"{} этих фрагментов (при склонении и спряжении этих знаков).}
        \end{scnindent}
        \scnhaselement{язык, в рамках которого синонимичные знаки в общем случае могут быть представлены синтаксически неэквивалентными текстовыми фрагментами, структура которых носит непредсказуемый характер}
    \end{scnindent}
    \scnhaselement{наличие омонимии в текстах языка\scnsupergroupsign}
    \begin{scnindent}
        \scnhaselement{язык, в текстах которого присутствует омонимия знаков}
        \begin{scnindent}
            \scnidtf{язык, в текстах которого присутствуют синтаксически эквивалентные, не синонимичные знаки}
        \end{scnindent}
        \scnhaselement{язык, в текстах которого омонимия знаков отсутствует}
    \end{scnindent}

    \scnheader{семантически выделяемый класс дискретных \textit{информационных конструкций}}
    \scnhaselement{синтаксическая структура информационной конструкции}
    \begin{scnindent}
        \scnrelto{второй домен}{синтаксис информационной конструкции*}
        \scnsuperset{первичная синтаксическая структура информационной конструкции}
        \scnsuperset{вторичная синтаксическая структура информационной конструкции}
    \end{scnindent}
    \scnhaselement{синтаксис языка}
    \scnhaselement{описание синтаксических понятий языка}
    \scnhaselement{синтаксические правила языка}
    \scnhaselement{денотационная семантика языка}
    \scnhaselement{описание семантических понятий языка}
    \scnhaselement{семантические правила языка}
    \scnhaselement{операционная семантика языка}
    \scnhaselement{смысловое представление информации}
    \begin{scnindent}
        \scnrelto{второй домен}{смысл*}
    \end{scnindent}

\end{SCn}

\textbf{\textit{смысловое представление информации}} --- явное (формальное) представление описываемых сущностей и связей между ними (см.~\scncite{Golenkov2019b}, \scncite{Golenkov2014}).
Для явного представления описываемых сущностей и связей между ними требуется существенное упрощение синтаксической структуры \textit{информационных конструкций} (см.~\textit{\ref{sec_ngics_sense_principles}~\nameref{sec_ngics_sense_principles}}).

\textbf{\textit{язык ostis-системы}} --- \textit{язык} представления \textit{информационных конструкций} в ostis-системах.

\begin{SCn}

    \scnheader{язык ostis-системы}
    \scnsubset{формальный язык}
    \scnsubset{универсальный язык}
    \scnrelto{используемые языки}{ostis-система}

    \scnhaselement{SC-код}
    \begin{scnindent}
        \scnrelto{внутренний язык}{ostis-система}
        \scniselement{универсальный язык}
    \end{scnindent}

\end{SCn}

Для формального описания различного рода \textit{языков} (включая \textit{SCg-код}, \textit{SCs-код}, \textit{SCn-код}, рассматриваемые в Главе \ref{chapter_ext_lang}) используется целый ряд метаязыковых понятий.

Перечислим некоторые из них: \textit{идентификатор}, \textit{класс синтаксически эквивалентных идентификаторов}, \textit{имя}, \textit{простое имя}, \textit{выражение}, \textit{внешний идентификатор*}, \textit{алфавит*}, \textit{разделители*}, \textit{ограничители*}, \textit{предложения*}

Синтаксис \textit{языков представления знаний в ostis-системах} может быть формально описан различными способами.
Так, например, можно использовать метаязык Бэкуса-Наура для описания синтаксиса \textit{SCs-кода} или его расширение для описания синтаксиса \textit{SCn-кода}.
Однако значительно более логично и целесообразно описывать синтаксис всех форм внешнего отображения sc-текстов с помощью самого \textit{SC-кода}.
Такой подход позволит \textit{ostis-системам} самостоятельно понимать, анализировать и генерировать тексты указанных \textit{языков} на основе принципов, общих для любых форм внешнего представления информации, в том числе нелинейных.

\textbf{\textit{алфавит*}} --- \textit{бинарное отношение}, связывающее множество текстов с  семейством максимальных множеств синтаксически однотипных элементарных (атомарных) фрагментов текстов, принадлежащих заданному множеству текстов.

\begin{SCn}

\scnheader{ограничители*}
\scnidtf{\textit{отношение}, связывающее заданный класс \textit{информационных конструкций} с соответствующим классом их \textit{ограничителей}}
\scnidtf{быть \textit{ограничителями}, используемыми в заданном множестве \textit{информационных конструкций}*}

\scnheader{разделители*}
\scnidtf{быть \textit{разделителями}, используемыми в заданном множестве \textit{информационных конструкций}*}
\scnrelfrom{второй домен}{разделитель}

\end{SCn}

\textbf{\textit{идентификатор}} --- структурированный \textit{знак} соответствующей (обозначаемой) сущности, который чаще всего представляет собой строку (цепочку символов), которую будем называть именем соответствующей сущности.
В формальных текстах (в том числе текстах \textit{SC-кода}, \textit{SCg-кода}, \textit{SCs-кода}, \textit{SCn-кода}) основные используемые \textit{идентификаторы} не должны быть омонимичными, то есть должны \myuline{однозначно} соответствовать идентифицируемым сущностям.
Следовательно, каждая пара идентификаторов, имеющих \myuline{одинаковую} структуру, должны обозначать одну и ту же сущность.

\begin{SCn}

    \scnheader{имя}
    \scnsubset{идентификатор}
    \scnidtf{строковый идентификатор}
    \scnidtf{идентификатор, представляющий собой строку (цепочку) символов}
    \begin{scnrelfromset}{декомпозиция действия}
        \scnitem{простое имя}
        \begin{scnindent}
            \scnidtf{атомарное имя}
            \scnidtf{имя, в состав которого другие имена не входят}
        \end{scnindent}
        \scnitem{выражение}
        \begin{scnindent}
            \scnidtf{неатомарное имя}
        \end{scnindent}
    \end{scnrelfromset}

\end{SCn}

\textbf{\textit{внешний идентификатор*}} --- \textit{бинарное ориентированное отношение}, каждая связка (sc-дуга) которого связывает некоторый элемент с \textit{файлом}, содержимым которого является \textit{внешний идентификатор} (чаще всего, имя), соответствующий указанному элементу.
Понятие внешнего идентификатора является понятием относительным и важным для ostis-систем, поскольку внутреннее для \textit{ostis-систем} представление информации (в виде текстов \textit{SC-кода}) оперирует не идентификаторами описываемых сущностей, а \textit{знаками}, структура которых никакого значения не имеет.

\begin{SCn}

\scnheader{предложения*}
\scnidtf{быть множеством всех предложений заданного текста, не являющихся встроенными предложениями, то есть предложениями, входящими в состав других предложений*}
\scnrelfrom{второй домен}{предложение}

\scnheader{предложение}
\scnexplanation{минимальный семантически целостный фрагмент текста, представляющий собой конфигурацию \textit{знаков}, входящих в этот фрагмент и связываемых между собой \textit{отношениями инцидентности} (в частности, отношением непосредственной последовательности в строке), а также различного вида \textit{разделителями} и \textit{ограничителями}}

\end{SCn}

\section{Внешние информационные конструкции и внешние языки ostis-систем}
\label{sec_external_information_constructs_external_lang}

Для представления \textit{баз знаний ostis-систем} используется целый ряд как \textit{универсальных языков}, так и \textit{специализированных языков}, как \textit{формальных языков}, так и \textit{естественных языков}, как \textit{внутренних языков}, обеспечивающих представление информации в памяти \textit{ostis-систем}, так и \textit{внешних языков}, обеспечивающих представление информации, вводимой в память \textit{ostis-систем}, либо выводимой из этой памяти. \textit{Естественные языки} используются исключительно для представления \textit{файлов}, хранимых в памяти \textit{ostis-системы} и формально специфицируемых в рамках \textit{базы знаний} этой \textit{ostis-системы}.

Для эксплуатации \textit{интеллектуальных компьютерных систем}, построенных на основе \textit{SC-кода} (см. \textit{Главу~\ref{chapter_sc_code}~\nameref{chapter_sc_code}}, а также \scncite{Standart2021}, \scncite{Golenkov2019}), кроме способа абстрактного внутреннего представления баз знаний (\textit{SC-кода}) потребуются несколько способов внешнего изображения абстрактных \textit{sc-текстов}, удобных для пользователей и используемых при оформлении исходных текстов \textit{баз знаний} указанных интеллектуальных компьютерных систем и исходных текстов фрагментов этих \textit{баз знаний}, а также используемых для отображения пользователям различных фрагментов \textit{баз знаний} по пользовательским запросам.
В качестве таких способов внешнего отображения \textit{sc-текстов} и применяются рассмотренные в \textit{Главе~\ref{chapter_ext_lang}~\nameref{chapter_ext_lang}} внешние языки ostis-систем (\textit{SCg-код}, \textit{SCs-код} и \textit{SCn-код}).

\textbf{\textit{SCg-код}} --- \textit{внешний язык*} \textit{ostis-систем}, тексты которого представляют собой графовые структуры общего вида с точно заданной \textit{денотационной семантикой*}.

\begin{SCn}

    \scnheader{SCg-код}
    \scniselement{язык ostis-системы}
    \begin{scnindent}
        \scnidtf{Semantic Code graphical}
        \scnrelto{внешний язык}{ostis-система}
        \scniselement{универсальный язык}
    \end{scnindent}

\end{SCn}

\textbf{\textit{SCs-код}} --- \textit{внешний язык*} \textit{ostis-систем}, тексты которого представляют собой строки (цепочки) символов.

\begin{SCn}

    \scnheader{SCs-код}
    \scniselement{язык ostis-системы}
    \begin{scnindent}
        \scnidtf{Semantic Code string}
        \scnrelto{внешний язык}{ostis-система}
        \scniselement{универсальный язык}
    \end{scnindent}

\end{SCn}

\textbf{\textit{SCn-код}} --- \textit{внешний язык*} \textit{ostis-систем}, тексты которого представляют собой двухмерные матрицы символов, являющиеся результатом форматирования, двухмерной структуризации текстов \textit{SCs-кода}.

\begin{SCn}

    \scnheader{SCn-код}
    \scniselement{язык ostis-системы}
    \begin{scnindent}
        \scnidtf{Semantic Code natural}
        \scnrelto{внешний язык}{ostis-система}
        \scniselement{универсальный язык}
    \end{scnindent}

\end{SCn}

Все основные внешние \textit{формальные языки}, используемые \textit{ostis-системами} (\textit{SCg-код}, \textit{SCs-код}, \textit{SCn-код}) являются различными вариантами внешнего представления текстов \textit{внутреннего языка} \textit{ostis-систем} --- \textit{SC-кода}.
Указанные языки являются универсальными и, следовательно, \textit{семантически эквивалентными языками*}.

При этом, каждая \textit{ostis-система} может приобрести способность использовать любой \textit{внешний язык} (как универсальный, так и специализированный, как естественный, так и искусственный), если \textit{синтаксис} и \textit{денотационная семантика} этого \textit{языка} будут описаны в памяти \textit{ostis-системы} на ее \textit{внутреннем языке} (\textit{SC-коде}).

\textbf{\textit{файл}} --- \textit{информационная конструкция}, представленная в "цифровой" форме, хранимой в какой-либо компьютерной памяти, но не являющийся \textit{sc-конструкцией}, хранимой в памяти \textit{ostis-системы}.

\begin{SCn}

    \scnheader{файл}
    \begin{scnrelfromset}{разбиение}
        \scnitem{файл ostis-системы}
            \begin{scnindent}
            \scnidtf{файл, хранимый в памяти ostis-системы}
                \begin{scnrelfromset}{разбиение}
                    \scnitem{сформированный файл ostis-системы}
                    \scnitem{несформированный файл ostis-системы}
                \end{scnrelfromset}
            \end{scnindent}
        \scnitem{файл компьютерной системы, не явлющейся ostis-системой}
    \end{scnrelfromset}
    \scnsuperset{текстовый файл}
    \begin{scnindent}
        % в рукописях Владимира Васильевича это было разбиением, но побоялся
        \scnsuperset{естественно-языковой файл}
        \begin{scnindent}
            \begin{scnrelfromset}{разбиение}
                \scnitem{размеченный естественно-языковой файл}
                \scnitem{неразмеченный естественно-языковой файл}
            \end{scnrelfromset}
            \scnsuperset{русскоязычный естественно-языковой файл}
            \scnsuperset{англоязычный естественно-языковой файл}
        \end{scnindent}
    \end{scnindent}
    \scnsuperset{файл изображения}
    \scnsuperset{видео-файл}
    \scnsuperset{аудио-файл}

\end{SCn}

\textbf{\textit{файл ostis-системы}} --- инородная для \textit{sc-кода} \textit{информационная конструкция}, которая может как храниться в памяти \textit{ostis-системы}, так и вне ее.


\begin{SCn}

    \scnheader{файл ostis-систем}
    \begin{scnrelfromset}{разбиение}
        \scnitem{хранимый файл ostis-систем}
        \scnitem{пустой файл ostis-систем}
    \end{scnrelfromset}
    \begin{scnrelfromset}{разбиение}
        \scnitem{файл-экземпляр}
            \begin{scnindent}
            \scnidtf{обозначение одного из вхождений (однорго из экземпляров) информационной конструкции]}
            \end{scnindent}
        \scnitem{файл-класс}
            \begin{scnindent}
            \scnidtf{обозначение всевозможнных \textit{информационных конструкций}, каждая из которых эквивалентна той, что представлена содержимым данного sc-узла}
            \end{scnindent}
    \end{scnrelfromset}
    \scnsuperset{sc.g-файл}
        \begin{scnindent}
        % по доброму, это навреное ключевой знак, но в той главе он так не обозначен, пусть будет просто ссылка
        \scntext{примечание}{См.~\textit{\ref{sec_scg}~\nameref{sec_scg}}.}
        \end{scnindent}
    \scnsuperset{sc.s-файл}
        \begin{scnindent}
        \scntext{примечание}{См.~\textit{\ref{sec_scg}~\nameref{sec_scs}}.}
        \end{scnindent}
    \scnsuperset{sc.n-файл}
        \begin{scnindent}
        \scntext{примечание}{См.~\textit{\ref{sec_scn}~\nameref{sec_scn}}.}
        \end{scnindent}
    \scnsuperset{естественно-языковой файл ostis-системы}
        \begin{scnindent}
        \scnsubset{естественно-языковой файл}
        \scnidtf{естественно-языковой файл, в котором выделены фрагменты, являющиеся sc-идентификаторами и, соответственно, ссылками на соответствующие им sc-элементы}
        \end{scnindent}


    \scnheader{файл-образец}
    \scnidtf{файл-эталон}
    \scnidtf{файл-оригинал}

\end{SCn}

\textbf{\textit{файл-образец}} --- \textit{файл}, объявленный как оригинал в рамках \textit{экоситсемы OSTIS}.
\textbf{\textit{файл-копия}} --- \textit{файл}, являющийся \textit{копией} соответствующего образца и \myuline{хранимый в одном месте}. \textit{файл-копия} может быть \textit{копией} как и \textit{файла-класса}, так и \textit{файла-экземпляра}.
\textbf{\textit{файл-копия}} --- \textit{файл}, являющийся \textit{копией} соответствующего образца и \myuline{хранимый в одном месте}. \textit{файл-копия} может быть \textit{копией} как и \textit{файла-класса}, так и \textit{файла-экземпляра}.

\begin{SCn}

    \scnheader{файл-копия}
    \begin{scnrelfromset}{разбиение}
        \scnitem{файл-копия, образец которого хранится в памяти ostis-системы}
        \scnitem{файл-копия, образец которого хранится в памяти компьютерной системы, которая ostis-системой не является}
    \end{scnrelfromset}

\end{SCn}

\textit{файлом ostis-системы} может стать любой \textit{файл} современной компьютерной системы.
Некоторые \textit{текстовые файлы} (в первую очередь \textit{естественно-языковые файлы} --- тексты \textit{естественных языков}) могут быть некоторым образом размечены путем указания связей размеченного \textit{файла} с другими \textit{файлами} \textit{ostis-системы}, а также с различными \textit{sc-элементами}.
Так, например:
\begin{textitemize}
    \item в любом месте размеченного текста можно сделать ссылку на другой \textit{файл}, трактуемый как примечание к данному месту размеченного текста,
    \item в размеченном тексте можно выделить используемые в нем \textit{основные идентификаторы} и \textit{неосновные идентификаторы} (курсивом),
    \item для выделенных \textit{неосновных sc-идентификаторов} можно выделить соответствующие им \textit{основные sc-идентификаторы}.
\end{textitemize}

Eсли в естественно-языковом тексте (например в цитате) используется \textit{неосновной sc-идентификатор}, то при разметке этого текста после указанного \textit{sc-идентификатора} приводится соответствующий (синонимичный) основной \textit{sc-идентификатор}.
Eсли в естественно-языковом тексте приводится подряд несколько \textit{основных sc-идентификаторов}, выделенных курсивом, то при разметке этого текста после указанных идентификаторов приводится перечень (через точку с запятой) соответствующих \textit{sc-идентификаторов}.

В состав \textit{текстового файла} \textit{ostis-систем} могут входить выделенные курсивом \textit{основные sc-идентификаторы}, являющиеся \textit{внешними sc-идентификаторами} соответствующих \textit{sc-элементов}.
В частности, это могут быть \textit{sc-идентификаторы}, обозначающие другие текстовые \textit{файлы}, являющиеся примечаниями и(или) пояснениями к соответствующим местам заданного \textit{файла}, а также библиографические ссылки.
Такого рода \textit{sc-идентификаторы} в исходном тексте \textit{файла} \textit{ограничителями} ссылок на \textit{информационные конструкции} (квадратными скобками).

\begin{SCn}

    \scnheader{отношение, заданное на естественно-языковых файлах}
    \scnhaselement{ссылка*}
        \begin{scnindent}
        \scnidtf{\textit{бинарное ориентированное отношение}, любая пара которого связывает \textit{естественно-языковой файл} с другим \textit{файлом}, на который указанный \textit{файл} ссылается. При этом (1) второй \textit{файл} может быть не только \textit{естественно-языковым файлом} (2) в тексте первого \textit{файла} может быть явно указано место с которого ссылка осуществляется, (3) частным видом ссылки может быть \textit{примечание}.}
        \end{scnindent}
    \scnhaselement{ключевой знак*}
    \scnhaselement{семантическая смежность*}
    \scnhaselement{перевод*}
    \scnhaselement{авторская ссылка*}
    \scnhaselement{авторское примечание*}

\end{SCn}

%\input{author/references}