\chapauthor{Ивашенко В.П.\\Голенков В.В.}
\chapter{Смысловое представление знаний в
	памяти ostis-систем}
\chapauthortoc{Ивашенко В.П.\\Голенков В.В.}
\label{chapter_sc_code}

\abstract{Аннотация к главе.}

\begin{SCn}
\begin{scnrelfromlist}{подраздел}
	\scnitem{\ref{sec_sr_sccode}~\nameref{sec_sr_sccode}}
	\scnitem{\ref{sec_sr_scdsemantics}~\nameref{sec_sr_scdsemantics}}
	\scnitem{\ref{sec_sr_scsyntax}~\nameref{sec_sr_scsyntax}}
	\scnitem{\ref{sec_sr_ostisfiles}~\nameref{sec_sr_ostisfiles}}
	\scnitem{\ref{sec_sr_semspace}~\nameref{sec_sr_semspace}}
\end{scnrelfromlist}
\end{SCn}

\section{SC-код (Semantic Computer Code)}
\label{sec_sr_sccode}
Общие положения

Ниже (в \ref{sec_sr_scdsemantics}~\nameref{sec_sr_scdsemantics} и в \ref{sec_sr_scsyntax}~\nameref{sec_sr_scsyntax}) приведено
формальное описание денотационной семантики
и синтаксиса SC-кода. Но сделано это будет не совсем обычно. Поскольку все элементы конструкции являются
обозначениями описываемых сущностей и, в том
числе, обозначениями различных выделяемых
классов sc-элементов, (!)нам ничего не стоит(!) явно
ввести различные семантически значимые и
синтаксически выделяемые классы sc-элементов и на основе этого явно описать
средствами SC-кода базовую денотационную
семантику и синтаксис SC-кода. 
Синтаксис SC-кода задаётся семейством классов
синтаксический задаваемых (выделяемых)
sc-элементов. 

Элементы, принадлежащие каждому
синтаксически выделяемому классу sc-элементов
должны иметь одинаковые синтаксические
признаки (синтаксические метки). 
При этом очевидно, что синтаксис SC-кода
существенно упростится, если синтаксически
выделяемые классы sс-элементов будут
одновременно иметь и чёткую семантическую
интерпретацию. Таким образом, формализацию
синтаксиса SC-кода целесообразно осуществлять
после формализации базовой денотационной
семантики SC-кода. Путём синтаксического
выделения тех семантически выделенных
классов sc-элементов, которые, во-первых,
необходимы для кодирования sc-конструкций
памяти ostis-систем (в sc-памяти) и во-вторых,
обеспечивают максимально возможное
упрощение обработки sc-конструкций (например,
путём упрощения анализа часто проверяемых
семантических характеристик обрабатываемых
sc-элементов).

Особенности SC-кода как одного из возможных
вариантов смыслового представления знаний 
(см. \$ 1.2.2) 

Понятие sc-элемента 

Понятие sc-конструкции 

Понятие внутреннего файла ostis-систем 

Понятие sc-идентификатора

\begin{SCn}
\section{Базовая денотационная семантика SC-кода}
\label{sec_sr_scdsemantics}
\scntext{содержание}{
Структурная классификация sc-элементов с
пояснением смысла понятий, используемых в
структурной классификации элементов 

Логико-семантическая классификация
sc-элементов с пояснением смысла вводимых
понятий 

Классификация sc-элементов по темпоральным
характеристикам обозначаемых ими сущностей с
пояснением смысла вводимых понятий 

Понятие базовой спецификации sc-элементов 

Онтологическая формализация базовой
денотационной семантики SC-кода 
}
В основе базовой денотационной семантики
SC-кода лежит:
\begin{itemize} 
\item семантическая классификация sc-элементов по
основным семантически значимым признакам 
\item уточнение структуры базовой семантической
спецификации элементов
\item …
\end{itemize}

\scnheader{Структурная классификация sc-элементов}
\scnstartstruct

\scnheader{sc-элемент}
\begin{scnrelfromset}{разбиение}
	\scnitem{обозначение sc-множества}
	\begin{scnindent}
		\begin{scnrelfromset}{разбиение}
			\scnitem{обозначение sc-связки}
			\begin{scnindent}
				\begin{scnrelfromset}{разбиение}
					\scnitem{обозначение sc-синглетона}
					\scnitem{обозначение sc-пары}
					\begin{scnindent}
						\begin{scnrelfromset}{разбиение}
							\scnitem{обозначение неориентированной sc-пары}
							\scnitem{обозначение ориентированной sc-пары}
							\begin{scnindent}
								\begin{scnrelfromset}{разбиение}
									\scnitem{\scnkeyword{обозначение sc-пары принадлежности}}
									\scnitem{обозначение sc-пары, не являющейся парой принадлежности}
								\end{scnrelfromset}
							\end{scnindent}
						\end{scnrelfromset}
					\end{scnindent}
					\scnitem{обозначение sc-связки, не являющаяся синглетоном и парой}
				\end{scnrelfromset}
			\end{scnindent}
			\scnitem{обозначение sc-класса}
			\scnitem{обозначение sc-структуры}
		\end{scnrelfromset}
	\end{scnindent}
	\scnitem{обозначение внешней сущности}
	\begin{scnindent}
		\begin{scnrelfromset}{разбиение}
			\scnitem{обозначение внутреннего файла}
			\scnitem{обозначение внешней сущности, не являющейся внутренним файлом}
		\end{scnrelfromset}
	\end{scnindent}	
\end{scnrelfromset}


\scnheader{sc-элемент}
\begin{scnrelfromset}{разбиение}
	\scnitem{sc-множество}
	\begin{scnindent}
		\begin{scnrelfromset}{разбиение}
			\scnitem{sc-связка}
			\begin{scnindent}
				\begin{scnrelfromset}{разбиение}
					\scnitem{sc-синглетон}
					\scnitem{sc-пара}
					\begin{scnindent}
						\begin{scnrelfromset}{разбиение}
							\scnitem{неориентированная sc-пара}
							\scnitem{ориентированная sc-пара}
							\begin{scnindent}
								\begin{scnrelfromset}{разбиение}
									\scnitem{\scnkeyword{sc-пара принадлежности}}
									\scnitem{sc-пара, не являющаяся парой принадлежности}
								\end{scnrelfromset}
							\end{scnindent}
						\end{scnrelfromset}
					\end{scnindent}
					\scnitem{sc-связка, не являющаяся синглетоном и парой}
				\end{scnrelfromset}
			\end{scnindent}
			\scnitem{sc-класс}
			\scnitem{sc-структура}							
		\end{scnrelfromset}
	\end{scnindent}
	\scnitem{внешняя сущность}
	\begin{scnindent}
		\begin{scnrelfromset}{разбиение}
			\scnitem{внутренний файл}
			\scnitem{внешняя сущность, не являющаяся внутренним файлом}
		\end{scnrelfromset}
	\end{scnindent}	
\end{scnrelfromset}


\bigskip

\scnheader{обозначение sc-пары принадлежности}
\begin{scnrelfromset}{разбиение}
	\scnitem{обозначение sc-пары нечеткой принадлежности}
	\scnitem{обозначение sc-пары позитивной принадлежности}
	\begin{scnindent}
		\scnsuperset{константное обозначение sc-пары постоянной позитивной принадлежности}
		\begin{scnindent}
			\begin{scnreltoset}{пересечение}
				\scnitem{sc-константа}
				\scnitem{постоянная (!)сущность}
				\scnitem{обозначение sc-пары позитивной принадлежности}	
			\end{scnreltoset}
		\end{scnindent}
		\scnsuperset{константное обозначение sc-пары временной позитивной принадлежности}
		\begin{scnindent}
			\begin{scnreltoset}{пересечение}
				\scnitem{sc-константа}
				\scnitem{временная (!)сущность}
				\scnitem{обозначение sc-пары позитивной принадлежности}	
			\end{scnreltoset}
		\end{scnindent}
	\end{scnindent}
	\scnitem{обозначение sc-пары негативной принадлежности}
\end{scnrelfromset}

\scnheader{sc-пара принадлежности}
\begin{scnrelfromset}{разбиение}
	\scnitem{sc-пара нечеткой принадлежности}
	\scnitem{sc-пара позитивной принадлежности}
	\begin{scnindent}
		\scnsuperset{sc-пара константной постоянной позитивной принадлежности}
		\begin{scnindent}
			\begin{scnreltoset}{пересечение}
				\scnitem{sc-константа}
				\scnitem{постоянная сущность}
				\scnitem{sc-пара позитивной принадлежности}	
			\end{scnreltoset}
		\end{scnindent}
		\scnsuperset{sc-пара константной временной позитивной принадлежности}
		\begin{scnindent}
			\begin{scnreltoset}{пересечение}
				\scnitem{sc-константа}
				\scnitem{временная сущность}
				\scnitem{sc-пара позитивной принадлежности}	
			\end{scnreltoset}
		\end{scnindent}
	\end{scnindent}
	\scnitem{sc-пара негативной принадлежности}
\end{scnrelfromset}

\bigskip

\scnendstruct \scnsourcecommentinline{Завершили представление \textit{Структурной классификации sc-элементов}}
\end{SCn}
	
\bigskip

Поясним смысл понятий, структурной
классификации sc-элементов 


\begin{SCn}
\scnheader{sc-элемент}
\scnidtf{sc-элемент, обозначающий множество}
\scnidtf{sc-обозначение множества}
\scnidtf{множество, представимое в SC-коде}
\scnidtf{множество} 

\bigskip

\scnsourcecommentinline{так как любое множество можно
представить в виде sc-множества}

\begin{scnrelfromlist}{примечание}
\scnfileitem{Каждый sc-элемент является обозначением
соответствующего множества}
\scnfileitem{Строго говоря, не каждое множество может быть
обозначено соответствующим sc-элементом. 
К таким множествам относятся либо множества,
элементами которых являются sc-элементы
(sc-множества), либо синглетоны элементами
которых являются сущности, не являющиеся
элементами (синглетоны внешних сущностей). Но
каждое множество, не являющийся sc-множеством
или синглетоном указанного вида, может быть
однозначно преобразовано в sc-множество и
описано средствами SC-кода. При этом
теоретико-множественные свойства
"нестандартных" для SC-кода множеств совпадают
со свойствами соответствующих им
"стандартных" для SC-кода множеств.}
\end{scnrelfromlist}
\end{SCn}

\begin{SCn}
\scnheader{sc-элемент}
\scnidtf{обозначение множества}
\scnnote{Тот факт, что каждый sc-элемент является
обозначением соответствующего множества
(частым случаем которых являются синглетоны
внешних описываемых сущностей), означает то,
что базовым видом объектов, которыми
оперирует SC-код на синтаксическом,
семантическом и логическом уровне являются
множества знаков, обозначающих различные
множества. 
В этом смысле SC-код имеет базовую
теоретико-множественную основу.}
\end{SCn}

\begin{SCn}
\scnheader{sc-элемент}
\begin{scnrelfromset}{разбиение}
	\scnitem{sc-множество}
	\begin{scnindent}
	\scnidtf{обозначение множества sc-элементов}
	\scnidtf{обозначение множества, все элементы
	которого являются sc-элементами}
	\end{scnindent}
	\scnitem{внешняя сущность}
	\begin{scnindent}
	\scnidtf{обозначение синглетона внешней сущности}
	\scnidtf{терминальный sc-элемент}
	\scnidtf{синглетон внешней сущности}
	\end{scnindent}
\end{scnrelfromset}	
\end{SCn}

\begin{SCn}
\scnheader{sc-множество}
\scnidtf{sc-элемент, обозначающий множество
sc-элементов}
\scnidtf{sc-обозначение множества sc-элементов}
\scnidtf{множество sc-элементов}
\scnidtf{множество, каждый элемент которого
является sc-элементом}
\scnsubset{sc-элемент}
\begin{scnindent}
\scnidtf{множество, представимое в SC-коде}

\scneq{\normalfont{(}sc-множество $\cup$ синглетон внешней сущности\normalfont{)}}

\scnidtf{информационная конструкция SC-кода}
\scnidtftext{часто используемый sc-идентификатор}
{sc-конструкция}

\end{scnindent}
\end{SCn}

\begin{SCn}
\scnheader{sc-связка}
\scnidtf{sc-элемент, обозначающий связку sc-элементов}
\scnidtf{sc-обозначение связки sc-элементов}
\scnidtf{обозначение sc-связки}
\end{SCn}

\begin{SCn}
\scnheader{sc-связка}
\scnidtf{обозначение связи между sc-элементами}
\scnsuperset{отображение связи между сущностями, которые
обозначаются sc-элементами, связанными sc-связкой}
\end{SCn}

\begin{SCn}
\scnheader{sc-синглетон}
\scnidtf{sc-множество, являющееся синглетоном}
\scnidtf{одномощное sс-множество}
\scnidtf{sc-множество, имеющее мощность, равную
единице}
\scnidtf{sc-элемент, обозначающий унарную sc-связку}
\scnidtf{sc-обозначение унарной sc-связки}
\scnidtf{унарная sc-связка}
\scnidtf{обозначение sc-синглетона}
\scnidtf{обозначение одномощного множества,
единственный элемент которого является}
sc-элементом]
\end{SCn}

\begin{SCn}
\scnheader{sc-пара}
\end{SCn}

\begin{SCn}
\scnheader{неориентированная sc-пара}
\end{SCn}

\begin{SCn}
\scnheader{ориентированная sc-пара}
\end{SCn}

\begin{SCn}
\scnheader{sc-пара принадлежности}
\end{SCn}

\begin{SCn}
\scnheader{sc-пара нечёткой принадлежности}
\end{SCn}

\begin{SCn}
\scnheader{sc-пара позитивной принадлежности}
\end{SCn}

\begin{SCn}
\scnheader{sc-пара константной постоянной позитивной
принадлежности}
\scnidtf{константная позитивная постоянная sc-пара
принадлежности}
\end{SCn}

\begin{SCn}
\scnheader{sc-пара константной временной позитивной
принадлежности}
\end{SCn}

\begin{SCn}
\scnheader{sc-пара негативной принадлежности}
\end{SCn}

\begin{SCn}
\scnheader{sc-пара, не являющаяся парой принадлежности}
\end{SCn}

\begin{SCn}
\scnheader{sc-связка, не являющаяся синглетоном и парой}
\end{SCn}

\begin{SCn}
\scnheader{sc-класс}
\scnnote{Требованиями, предъявляемыми к каждому
sc-классу являются:
\begin{itemize}
	\item бесконечность этого sc-множества 
	\item наличие общего свойства, присущего всем
	элементам этого sc-множества, в частности,
	наличие его определения
\end{itemize} }
\end{SCn}

\begin{SCn}
\scnheader{sc-класс}
\scnsuperset{sc-отношение}

\begin{scnindent}
\scnidtf{sc-класс sc-связок}
\scnsuperset{бинарное sc-отношение}

\begin{scnindent}
\begin{scnrelfromset}{разбиение}
\scnitem{бинарное неориентированное sc-отношение}
\scnitem{бинарное ориентированное sc-отношение}
\begin{scnindent}
		\scnsuperset{ролевое sc-отношение}
\end{scnindent}		
\end{scnrelfromset}	
\end{scnindent}
\end{scnindent}

\scnsuperset{sc-класс sc-классов}

\scnsuperset{sc-параметр}


\scnsuperset{sc-класс sc-структур}


\scnsuperset{sc-класс эквивалентности}
\begin{scnindent}
\scnidtf{фактор-множество соответствующего
отношения эквивалентности}
\end{scnindent}

\scnsuperset{sc-класс внешних сущностей}

\scnsuperset{sc-класс внутренних файлов}


\begin{scnrelfromset}{следует отличать}
\scnitem{sc-класс}
\scnitem{sc-связка}
\end{scnrelfromset}	


\scntext{сравнение}
{В отличие от sc-связки принципом формирования
является наличие общего свойства, присущего
всем элементам этого sc-класса и только им (или
присущего всем сущностям, которые
обозначаются указанными sc-элементами). Таким
общим свойством может быть определение
sc-класса либо принадлежность одному из
значений некоторого параметра, то есть одному
из элементов фактор-множества,
соответствующего некоторому отношению
эквивалентности или толерантности}

\scntext{сравнение}{Примерами sc-классов являются:
	\begin{itemize}
		\item конкретная окружность (множество всех
		точек, равноудалённых от некоторой заданной
		точки)
		\item конкретный отрезок (множество всех точек,
		лежащих между двумя заданными точками, с
		включением этих точек)
		\item конкретный линейный треугольник (множество
		всех точек, лежащих между каждыми двумя из
		трёх заданных точек, с включением этих точек)
	\end{itemize} 
и, соответственно этому, примерами sc-связок
являются: 
	\begin{itemize}
	\item пары граничных точек различных отрезков
	\item тройки вершин различных треугольников
	\end{itemize} 
}
\begin{scnrelfromset}{следует отличать}
	\scnitem{sc-связка попарно эквивалентных сущностей}
	\scnitem{sc-класс эквивалентности}
	
\begin{scnindent}
		\scntext{пояснение}{В sc-класс входит не просто некоторые количество попарно эквивалентных между собой сущностей, а абсолютно все такие сущности}
\end{scnindent}		
\end{scnrelfromset}	

Приведём пример: 
\begin{scnrelfromset}{следует отличать}
	\scnitem{множество всех треугольников, подобных
	одному из них}
\begin{scnindent}
		\scnsubset{sc-класс}
\end{scnindent}
	\scnitem{конечное множество подобных треугольников}
\begin{scnindent}
		\scnsubset{sc-связка попарно эквивалентных треугольников}		
\end{scnindent}
\end{scnrelfromset}	
\end{SCn}

\begin{SCn}
\scnheader{sc-структура}

\begin{scnrelfromset}{следует отличать}
\scnitem{sc-структура}
\scnitem{sc-связка}
\end{scnrelfromset}	

\scntext{сравнение}{В отличие от sc-связок в каждую sc-структуру
должна входить по крайней мере одна sc-связка
вместе с компонентами этой sc-связки}
\end{SCn}

\begin{SCn}
\scnheader{внешняя сущность}
\scnidtf{синглетон внешней сущности}
\scnidtf{обозначение синглетона внешней сущности}
\scnidtf{sc-элемент, обозначающий синглетон,
элементом которого является некоторая
внешняя описываемое сущность}
\scnidtf{множество, обозначаемое sc-элементом,
являющиеся одномощным множеством,
единственным элементом которого является
сущность, внешняя по отношению к
sc-конструкции, то есть сущность, не являющиеся
sc-элементом}
\scnnote{обозначение синглетона внешний сущности, то
есть sc-элемент, обозначающий этот синглетон,
можно также трактовать как sc-элемент,
обозначающий соответствующую внешнюю
описываемую сущность, которую, в свою очередь,
можно считать денотатом указанного
sc-элемента}

\scnnote{Очевидно, что пара принадлежности,
связывающая sc-элемент, обозначающий
синглетон внешней сущности, не может быть
непосредственно представлена в виде
соответствующей sc-дуги принадлежности, так
как второй компонент этой sc-дуги не находится
в sc-памяти}

\begin{scnrelfromset}{следует отличать}
	\scnitem{синглетон внешней сущности}
	\scnitem{sc-синглетон}
\begin{scnindent}	
	\scnidtf{sc-синглетон, единственным элементом
	которого является некоторый sc-элемент}

	\scnsubset{sc-множество}
\begin{scnindent}
	\scnidtf{sc-элемент, обозначающий множество,
	элементами которого являются только
	sc-элементы}
	\scnidtf{множество sc-элементов}
\end{scnindent}
\end{scnindent}
\end{scnrelfromset}	
\end{SCn}

\begin{SCn}
\scnheader{внутренний файл}
\scnidtf{внутренний файл ostis-системы}
\scnidtf{внутренний образ (копия), внешней
информационной конструкции, хранимый в
файловой памяти ostis-системы}
\scnnote{Файловая память ostis-системы, хранящая
различного рода информационные конструкции
(образы, модели), не являющиеся
sc-конструкциями, должна быть тесно связана с
sc-памятью этой же ostis-системы. Как минимум
каждый файл ostis-системы должен быть связан с
тем sc-узлом, который является знаком этого
файла (точнее, знаком синглетона, элементом
которого является указанный файл)}
\end{SCn}

\begin{SCn}
\scnheader{внешняя сущность, не являющаяся внутренним
файлом}
\end{SCn}

\section{Синтаксис SC-кода}
\label{sec_sr_scsyntax}

\begin{SCn}
\scnheader{sc-элемент}
\begin{scnrelfromset}{разбиение}
	\scnitem{sc-коннектор}
	\begin{scnindent}
		\begin{scnrelfromset}{разбиение}
			\scnitem{sc-дуга}
			\begin{scnindent}
				\begin{scnrelfromset}{разбиение}
					\scnitem{базовая sc-дуга}
					\scnitem{sc-дуга общего вида}
					\begin{scnindent}
%							\scneq{\normalfont{(}sc-синглетон $\cup$ sc-связка не являющаяся синглетоном и парой\normalfont{)}}
					\end{scnindent}	
				\end{scnrelfromset}
			\end{scnindent}	
			\scnitem{sc-ребро}
			\begin{scnindent}
			\scneq{неориентированная sc-пара}
			\end{scnindent}	
		\end{scnrelfromset}
	\end{scnindent}	
	\scnitem{sc-узел общего вида}
	\begin{scnindent}
	\scneq{\normalfont{(}sc-синглетон $\cup$ sc-связка не являющаяся синглетоном и парой\normalfont{)}}
	\end{scnindent}	
\end{scnrelfromset}
\end{SCn}


\section{Файлы ostis-системы}
\label{sec_sr_ostisfiles}
\section{Смысловое пространство ostis-систем}
\label{sec_sr_semspace}

%\input{author/references}