\chapauthor{Ивашенко В.П.\\Голенков В.В.}
\chapter{Универсальный язык смыслового представления знаний в памяти ostis-систем}
\chapauthortoc{Ивашенко В.П.\\Голенков В.В.}
\label{chapter_sc_code}

\abstract{Аннотация к главе.}

\section{SC-код (Semantic Computer Code)}
\section{Базовая денотационная семантика SC-кода}

\scnheader{Структурная классификация sc-элементов}
\scnstartstruct

\scnheader{sc-элемент}
\begin{scnrelfromset}{разбиение}
	\scnitem{sc-множество}
	\begin{scnindent}
	\begin{scnrelfromset}{разбиение}
		\scnitem{sc-связка}
		\begin{scnindent}
		\begin{scnrelfromset}{разбиение}
			\scnitem{sc-синглетон}
			\scnitem{sc-пара}
			\begin{scnindent}
			\begin{scnrelfromset}{разбиение}
				\scnitem{неориентированная sc-пара}
				\scnitem{ориентированная sc-пара}
				\begin{scnindent}
				\begin{scnrelfromset}{разбиение}
					\scnitem{\scnkeyword{sc-пара принадлежности}}
					\scnitem{sc-пара, не являющаяся парой принадлежности}
				\end{scnrelfromset}
				\end{scnindent}
			\end{scnrelfromset}
			\end{scnindent}
			\scnitem{sc-связка, не являющаяся синглетоном и парой}
		\end{scnrelfromset}
		\end{scnindent}
		\scnitem{sc-класс}
		\scnitem{sc-структура}							
	\end{scnrelfromset}
	\end{scnindent}
	\scnitem{sc-внешняя сущность}
	\begin{scnindent}
	\begin{scnrelfromset}{разбиение}
		\scnitem{внутренний файл}
		\scnitem{внешняя сущность, не являющаяся внутренним файлом}
	\end{scnrelfromset}
	\end{scnindent}	
\end{scnrelfromset}

\bigskip

\scnheader{sc-пара принадлежности}
\begin{scnrelfromset}{разбиение}
\scnitem{sc-пара нечеткой принадлежности}
\scnitem{sc-пара позитивной принадлежности}
\begin{scnindent}
	\scnsuperset{sc-пара константной постоянной позитивной принадлежности}
	\begin{scnindent}
		\begin{scnreltoset}{пересечение}
			\scnitem{sc-константа}
			\scnitem{постоянная сущность}
			\scnitem{sc-пара позитивной принадлежности}	
		\end{scnreltoset}
	\end{scnindent}
	\scnsuperset{sc-пара константной временной позитивной принадлежности}
	\begin{scnindent}
		\begin{scnreltoset}{пересечение}
			\scnitem{sc-константа}
			\scnitem{временная сущность}
			\scnitem{sc-пара позитивной принадлежности}	
		\end{scnreltoset}
	\end{scnindent}
\end{scnindent}
\scnitem{sc-пара негативной принадлежности}
\end{scnrelfromset}

\bigskip

\scnendstruct \scnsourcecommentinline{Завершили представление \textit{Структурной классификации sc-элементов}}

\section{Синтаксис SC-кода}
\section{Смысловое пространство ostis-систем}

%\input{author/references}