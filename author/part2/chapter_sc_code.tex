%\chapauthor{Ивашенко В.П.\\Голенков В.В.}
\chapter{Смысловое представление знаний в памяти ostis-систем}
\chapauthortoc{Ивашенко В.П.\\Голенков В.В.}
\label{chapter_sc_code}

\begin{SCn}
	\begin{scnrelfromlist}{автор}
		\scnitem{Голенков В.~В.}
		\scnitem{Ивашенко В.~П.}
	\end{scnrelfromlist}
\end{SCn}

\bigskip

\abstract{Аннотация к главе.}

%разбиение
%пересечение множеств
\begin{SCn}
	\begin{scnrelfromlist}{ключевое понятие}
		\scnitem{sc-элемент}
%		\scnitem{Структурный признак классификации sc-элементов}
		\scnitem{обозначение sc-множества}
		\scnitem{обозначение sc-связки}
		\scnitem{обозначение sc-синглетона}
		\scnitem{обозначение sc-пары}		
		\scnitem{обозначение неориентированной sc-пары}		
		\scnitem{обозначение ориентированной sc-пары}		
		\scnitem{обозначение sc-пары принадлежности}		
		\scnitem{обозначение sc-пары нечёткой принадлежности}		
		\scnitem{обозначение sc-пары позитивной принадлежности}
		\scnitem{обозначение sc-пары негативной принадлежности}		
		\scnitem{обозначение ориентированной sc-пары, не являющейся sc-парой принадлежности}		
		\scnitem{обозначение sc-связки, не являющейся синглетоном и парой}		
		\scnitem{обозначение sc-класса}		
		\scnitem{обозначение sc-структуры}		
		\scnitem{обозначение внешней сущности}		
		\scnitem{обозначение внутреннего файла}		
		\scnitem{обозначение внешней сущности, не являющейся внутренним файлом}		
		\scnitem{обозначение постоянной сущности}
		\scnitem{обозначение временной сущности}				
		\scnitem{обозначение статической сущности}
		\scnitem{обозначение динамической сущности}				
		\scnitem{sc-константа}
		\scnitem{sc-переменная}				
		\scnitem{sc-множество}
		\scnitem{sc-связка}
		\scnitem{sc-синглетон}
		\scnitem{sc-пара}
		\scnitem{неориентированная sc-пара}
		\scnitem{ориентированная sc-пара}
		\scnitem{sc-пара принадлежности}
		\scnitem{sc-пара нечёткой принадлежности}
		\scnitem{sc-пара позитивной принадлежности}
		\scnitem{sc-пара негативной принадлежности}
		\scnitem{ориентированная sc-пара принадлежности, не являющаяся sc-парой принадлежности}
		\scnitem{sc-связка,не являющаяся синглетоном и парой}
		\scnitem{sc-класс}						
		\scnitem{sc-структура}						
		\scnitem{внешняя сущность}				
		\scnitem{внутренний файл}				
		\scnitem{внешняя сущность, не являющаяся внутренним файлом}				
	\end{scnrelfromlist}
\end{SCn}
%ключевые понятия:

%sc-элемент

\begin{SCn}
	\begin{scnrelfromlist}{подраздел}
		\scnitem{\ref{sec_sr_sccode}~\nameref{sec_sr_sccode}}
		\scnitem{\ref{sec_sr_scdsemantics}~\nameref{sec_sr_scdsemantics}}
		\scnitem{\ref{sec_sr_scsyntax}~\nameref{sec_sr_scsyntax}}
		\scnitem{\ref{sec_sr_ostisfiles}~\nameref{sec_sr_ostisfiles}}
		\scnitem{\ref{sec_sr_semspace}~\nameref{sec_sr_semspace}}
	\end{scnrelfromlist}
\end{SCn}


\section{SC-код (Semantic Computer Code)}
\label{sec_sr_sccode}
Общие положения

Ниже (в \textit{\ref{sec_sr_scdsemantics}~\nameref{sec_sr_scdsemantics}} и в \textit{\ref{sec_sr_scsyntax}~\nameref{sec_sr_scsyntax}}) приведено формальное описание денотационной семантики и синтаксиса SC-кода. Но сделано это будет не совсем обычно. Поскольку все элементы конструкции являются обозначениями описываемых сущностей и, в том числе, обозначениями различных выделяемых классов \textit{sc-элементов}, (!)нам ничего не стоит(!) явно ввести различные семантически значимые и синтаксически выделяемые классы \textit{sc-элементов} и на основе этого явно описать средствами SC-кода базовую денотационную семантику и синтаксис SC-кода. Синтаксис SC-кода задаётся семейством классов синтаксический задаваемых (выделяемых) \textit{sc-элементов}. 

Элементы, принадлежащие каждому синтаксически выделяемому классу \textit{sc-элементов} должны иметь одинаковые синтаксические признаки (синтаксические метки). При этом очевидно, что синтаксис SC-кода существенно упростится, если синтаксически выделяемые классы sс-элементов будут одновременно иметь и чёткую семантическую интерпретацию. Таким образом, формализацию синтаксиса SC-кода целесообразно осуществлять после формализации базовой денотационной семантики SC-кода. Путём синтаксического выделения тех семантически выделенных классов \textit{sc-элементов}, которые, во-первых, необходимы для кодирования sc-конструкций памяти ostis-систем (в sc-памяти) и во-вторых, обеспечивают максимально возможное упрощение обработки sc-конструкций (например, путём упрощения анализа часто проверяемых семантических характеристик обрабатываемых \textit{sc-элементов}).

Особенности SC-кода как одного из возможных вариантов смыслового представления знаний (см.\textit{~\ref{sec_ngics_sense_principles}~\nameref{sec_ngics_sense_principles}}) 

Понятие \textit{sc-элемента} 

Понятие sc-конструкции 

Понятие внутреннего файла ostis-систем 

Понятие sc-идентификатора

\newpage
\section{Базовая денотационная семантика SC-кода}
\label{sec_sr_scdsemantics}

\begin{SCn}
	\begin{scnrelfromlist}{подраздел}
		\scnitem{\ref{sec_semantic_classification_sc-elements}~\nameref{sec_semantic_classification_sc-elements}}
		\scnitem{\ref{sec_meaning_selected_classes_sc-elements}~\nameref{sec_meaning_selected_classes_sc-elements}}
		\scnitem{\ref{sec_structure_basic_semantic_specification_sc-element}~\nameref{sec_structure_basic_semantic_specification_sc-element}}
		\scnitem{\ref{sec_ontological_formalization_basic_denotational_semantics_sc-code}~\nameref{sec_ontological_formalization_basic_denotational_semantics_sc-code}}
	\end{scnrelfromlist}
\end{SCn}
	
%\begin{SCn}
%	\scntext{содержание}{
%		Структурная классификация sc-элементов с пояснением смысла понятий, используемых в структурной классификации элементов 
%		
%		Логико-семантическая классификация sc-элементов с пояснением смысла вводимых понятий 
%		
%		Классификация sc-элементов по темпоральным характеристикам обозначаемых ими сущностей с пояснением смысла вводимых понятий 
%		
%		Понятие базовой спецификации sc-элементов 
%		
%		Онтологическая формализация базовой денотационной семантики SC-кода 
%	}

%	В основе базовой денотационной семантики SC-кода лежит:
%	\begin{itemize} 
%		\item семантическая классификация sc-элементов по основным семантически значимым признакам 
%		\item уточнение структуры базовой семантической спецификации элементов
%		\item …
%	\end{itemize}
%\end{SCn}

\newpage
\subsection{Семантическая классификация sc-элементов по базовым признакам}
\label{sec_semantic_classification_sc-elements}
К числу базовых признаков классификации \textit{sc-элементов} относятся:

\begin{textitemize}
	\item структурный признак;
	\item логико-семантический признак;
	\item темпоральная характеристика сущностей, обозначаемых \textit{sc-элементами}, которая в свою очередь, включает в себя:
	\begin{textitemize}
		\item постоянство или временность существования обозначаемой сущности;
		\item статичность (стационарность) или динамичность (изменчивость) обозначаемой сущности.
	\end{textitemize}
\end{textitemize}

\begin{SCn}
	\scnheader{Структурная классификация sc-элементов}
	\scnstartstruct
	
	\scnheader{sc-элемент}
	\scnrelfrom{разбиение}{Структурный признак классификации sc-элементов}
	\begin{scnindent}
		\begin{scneqtoset}
			\scnitem{обозначение sc-множества}
			\begin{scnindent}
			\begin{scnsubdividing}
				\scnitem{обозначение sc-связки}
				\begin{scnindent}
				\begin{scnsubdividing}
					\scnitem{обозначение sc-синглетона}
					\scnitem{обозначение sc-пары}
					\begin{scnindent}
					\begin{scnsubdividing}
						\scnitem{обозначение неориентированной sc-пары}
						\scnitem{обозначение ориентированной sc-пары}
						\begin{scnindent}
							\begin{scnsubdividing}
								\scnitem{обозначение sc-пары принадлежности}
								\begin{scnindent}
								\begin{scnsubdividing}
									\scnitem{обозначение sc-пары нечеткой принадлежности}
									\scnitem{обозначение sc-пары позитивной принадлежности}
									\scnitem{обозначение sc-пары негативной принадлежности}
								\end{scnsubdividing}
								\end{scnindent}
								\scnitem{обозначение ориентированной sc-пары, не являющейся парой принадлежности}
							\end{scnsubdividing}
						\end{scnindent}
					\end{scnsubdividing}
					\end{scnindent}
					\scnitem{обозначение sc-связки, не являющейся синглетоном и парой}
				\end{scnsubdividing}
				\end{scnindent}
				\scnitem{обозначение sc-класса}
				\scnitem{обозначение sc-структуры}
			\end{scnsubdividing}
			\end{scnindent}
			\scnitem{обозначение внешней сущности}
			\begin{scnindent}
			\begin{scnsubdividing}
					\scnitem{внутренний файл}
					\scnitem{внешняя сущность, не являющаяся внутренним файлом}
					\scnitem{обозначение файла}
					\scnitem{обозначение информационной конструкции, не являющейся ни sc-множеством, ни файлом}
					\scnitem{обозначение внешней сущности, не являющейся информационной конструкцией}
			\end{scnsubdividing}
			\end{scnindent}
		\end{scneqtoset}
	\end{scnindent} 

	\scnendstruct \scnsourcecommentinline{Завершили представление \textit{Структурной классификации sc-элементов}}
\end{SCn}

\begin{comment}
Поясним смысл понятий, структурной
классификации sc-элементов 


\begin{SCn}
	\scnheader{sc-элемент}
	\scnidtf{sc-элемент, обозначающий множество}
	\scnidtf{sc-обозначение множества}
	\scnidtf{множество, представимое в SC-коде}
	\scnidtf{множество} 
	
	\bigskip
	
	\scnsourcecommentinline{так как любое множество можно
		представить в виде sc-множества}
	
	\begin{scnrelfromlist}{примечание}
		\scnfileitem{Каждый sc-элемент является обозначением
			соответствующего множества}
		\scnfileitem{Строго говоря, не каждое множество может быть
			обозначено соответствующим sc-элементом. 
			К таким множествам относятся либо множества,
			элементами которых являются sc-элементы
			(sc-множества), либо синглетоны элементами
			которых являются сущности, не являющиеся
			элементами (синглетоны внешних сущностей). Но
			каждое множество, не являющийся sc-множеством
			или синглетоном указанного вида, может быть
			однозначно преобразовано в sc-множество и
			описано средствами SC-кода. При этом
			теоретико-множественные свойства
			"нестандартных" для SC-кода множеств совпадают
			со свойствами соответствующих им
			"стандартных" для SC-кода множеств.}
	\end{scnrelfromlist}
\end{SCn}

\begin{SCn}
	\scnheader{sc-элемент}
	\scnidtf{обозначение множества}
	\scnnote{Тот факт, что каждый sc-элемент является
		обозначением соответствующего множества
		(частым случаем которых являются синглетоны
		внешних описываемых сущностей), означает то,
		что базовым видом объектов, которыми
		оперирует SC-код на синтаксическом,
		семантическом и логическом уровне являются
		множества знаков, обозначающих различные
		множества. 
		В этом смысле SC-код имеет базовую
		теоретико-множественную основу.}
\end{SCn}

\begin{SCn}
	\scnheader{sc-элемент}
	\begin{scnrelfromset}{разбиение}
		\scnitem{обозначение sc-множества}
		\begin{scnindent}
			\scnidtf{обозначение множества sc-элементов}
			\scnidtf{обозначение множества, все элементы
				которого являются sc-элементами}
		\end{scnindent}
		\scnitem{обозначение внешней сущности}
		\begin{scnindent}
			\scnidtf{обозначение синглетона внешней сущности}
			\scnidtf{терминальный sc-элемент}
			\scnidtf{синглетон внешней сущности}
		\end{scnindent}
	\end{scnrelfromset}	
\end{SCn}

\begin{SCn}
	\scnheader{sc-множество}
	\scnidtf{sc-элемент, обозначающий множество
		sc-элементов}
	\scnidtf{sc-обозначение множества sc-элементов}
	\scnidtf{множество sc-элементов}
	\scnidtf{множество, каждый элемент которого
		является sc-элементом}
	\scnsubset{sc-элемент}
	\begin{scnindent}
		\scnidtf{множество, представимое в SC-коде}
		
		\scneq{\normalfont{(}sc-множество $\cup$ синглетон внешней сущности\normalfont{)}}
		
		\scnidtf{информационная конструкция SC-кода}
		\scnidtftext{часто используемый sc-идентификатор}
		{sc-конструкция}
		
	\end{scnindent}
\end{SCn}

\begin{SCn}
	\scnheader{sc-связка}
	\scnidtf{sc-элемент, обозначающий связку sc-элементов}
	\scnidtf{sc-обозначение связки sc-элементов}
	\scnidtf{обозначение sc-связки}
\end{SCn}

\begin{SCn}
	\scnheader{sc-связка}
	\scnidtf{обозначение связи между sc-элементами}
	\scnsuperset{отображение связи между сущностями, которые
		обозначаются sc-элементами, связанными sc-связкой}
\end{SCn}

\begin{SCn}
	\scnheader{sc-синглетон}
	\scnidtf{sc-множество, являющееся синглетоном}
	\scnidtf{одномощное sс-множество}
	\scnidtf{sc-множество, имеющее мощность, равную
		единице}
	\scnidtf{sc-элемент, обозначающий унарную sc-связку}
	\scnidtf{sc-обозначение унарной sc-связки}
	\scnidtf{унарная sc-связка}
	\scnidtf{обозначение sc-синглетона}
	\scnidtf{обозначение одномощного множества,
		единственный элемент которого является}
	sc-элементом]
\end{SCn}

\begin{SCn}
	\scnheader{sc-пара}
\end{SCn}

\begin{SCn}
	\scnheader{неориентированная sc-пара}
\end{SCn}

\begin{SCn}
	\scnheader{ориентированная sc-пара}
\end{SCn}

\begin{SCn}
	\scnheader{sc-пара принадлежности}
\end{SCn}

\begin{SCn}
	\scnheader{sc-пара нечёткой принадлежности}
\end{SCn}

\begin{SCn}
	\scnheader{sc-пара позитивной принадлежности}
\end{SCn}

\begin{SCn}
	\scnheader{sc-пара константной постоянной позитивной
		принадлежности}
	\scnidtf{константная позитивная постоянная sc-пара
		принадлежности}
\end{SCn}

\begin{SCn}
	\scnheader{sc-пара константной временной позитивной
		принадлежности}
\end{SCn}

\begin{SCn}
	\scnheader{sc-пара негативной принадлежности}
\end{SCn}

\begin{SCn}
	\scnheader{sc-пара, не являющаяся парой принадлежности}
\end{SCn}

\begin{SCn}
	\scnheader{sc-связка, не являющаяся синглетоном и парой}
\end{SCn}

\begin{SCn}
	\scnheader{sc-класс}
	\scnnote{Требованиями, предъявляемыми к каждому
		sc-классу являются:
		\begin{itemize}
			\item бесконечность этого sc-множества 
			\item наличие общего свойства, присущего всем
			элементам этого sc-множества, в частности,
			наличие его определения
	\end{itemize} }
\end{SCn}

\begin{SCn}
	\scnheader{sc-класс}
	\scnsuperset{sc-отношение}
	
	\begin{scnindent}
		\scnidtf{sc-класс sc-связок}
		\scnsuperset{бинарное sc-отношение}
		
		\begin{scnindent}
			\begin{scnrelfromset}{разбиение}
				\scnitem{бинарное неориентированное sc-отношение}
				\scnitem{бинарное ориентированное sc-отношение}
				\begin{scnindent}
					\scnsuperset{ролевое sc-отношение}
				\end{scnindent}		
			\end{scnrelfromset}	
		\end{scnindent}
	\end{scnindent}
	
	\scnsuperset{sc-класс sc-классов}
	
	\scnsuperset{sc-параметр}
	
	
	\scnsuperset{sc-класс sc-структур}
	
	
	\scnsuperset{sc-класс эквивалентности}
	\begin{scnindent}
		\scnidtf{фактор-множество соответствующего
			отношения эквивалентности}
	\end{scnindent}
	
	\scnsuperset{sc-класс внешних сущностей}
	
	\scnsuperset{sc-класс внутренних файлов}
	
	
	\begin{scnrelfromset}{следует отличать}
		\scnitem{sc-класс}
		\scnitem{sc-связка}
	\end{scnrelfromset}	
	
	
	\scntext{сравнение}
	{В отличие от sc-связки принципом формирования
		является наличие общего свойства, присущего
		всем элементам этого sc-класса и только им (или
		присущего всем сущностям, которые
		обозначаются указанными sc-элементами). Таким
		общим свойством может быть определение
		sc-класса либо принадлежность одному из
		значений некоторого параметра, то есть одному
		из элементов фактор-множества,
		соответствующего некоторому отношению
		эквивалентности или толерантности}
	
	\scntext{сравнение}{Примерами sc-классов являются:
		\begin{itemize}
			\item конкретная окружность (множество всех
			точек, равноудалённых от некоторой заданной
			точки)
			\item конкретный отрезок (множество всех точек,
			лежащих между двумя заданными точками, с
			включением этих точек)
			\item конкретный линейный треугольник (множество
			всех точек, лежащих между каждыми двумя из
			трёх заданных точек, с включением этих точек)
		\end{itemize} 
		и, соответственно этому, примерами sc-связок
		являются: 
		\begin{itemize}
			\item пары граничных точек различных отрезков
			\item тройки вершин различных треугольников
		\end{itemize} 
	}
	\begin{scnrelfromset}{следует отличать}
		\scnitem{sc-связка попарно эквивалентных сущностей}
		\scnitem{sc-класс эквивалентности}
		
		\begin{scnindent}
			\scntext{пояснение}{В sc-класс входит не просто некоторые количество попарно эквивалентных между собой сущностей, а абсолютно все такие сущности}
		\end{scnindent}		
	\end{scnrelfromset}	
	
	Приведём пример: 
	\begin{scnrelfromset}{следует отличать}
		\scnitem{множество всех треугольников, подобных
			одному из них}
		\begin{scnindent}
			\scnsubset{sc-класс}
		\end{scnindent}
		\scnitem{конечное множество подобных треугольников}
		\begin{scnindent}
			\scnsubset{sc-связка попарно эквивалентных треугольников}		
		\end{scnindent}
	\end{scnrelfromset}	
\end{SCn}

\begin{SCn}
	\scnheader{sc-структура}
	
	\begin{scnrelfromset}{следует отличать}
		\scnitem{sc-структура}
		\scnitem{sc-связка}
	\end{scnrelfromset}	
	
	\scntext{сравнение}{В отличие от sc-связок в каждую sc-структуру
		должна входить по крайней мере одна sc-связка
		вместе с компонентами этой sc-связки}
\end{SCn}

\begin{SCn}
	\scnheader{внешняя сущность}
	\scnidtf{синглетон внешней сущности}
	\scnidtf{обозначение синглетона внешней сущности}
	\scnidtf{sc-элемент, обозначающий синглетон,
		элементом которого является некоторая
		внешняя описываемое сущность}
	\scnidtf{множество, обозначаемое sc-элементом,
		являющиеся одномощным множеством,
		единственным элементом которого является
		сущность, внешняя по отношению к
		sc-конструкции, то есть сущность, не являющиеся
		sc-элементом}
	\scnnote{обозначение синглетона внешний сущности, то
		есть sc-элемент, обозначающий этот синглетон,
		можно также трактовать как sc-элемент,
		обозначающий соответствующую внешнюю
		описываемую сущность, которую, в свою очередь,
		можно считать денотатом указанного
		sc-элемента}
	
	\scnnote{Очевидно, что пара принадлежности,
		связывающая sc-элемент, обозначающий
		синглетон внешней сущности, не может быть
		непосредственно представлена в виде
		соответствующей sc-дуги принадлежности, так
		как второй компонент этой sc-дуги не находится
		в sc-памяти}
	
	\begin{scnrelfromset}{следует отличать}
		\scnitem{синглетон внешней сущности}
		\scnitem{sc-синглетон}
		\begin{scnindent}	
			\scnidtf{sc-синглетон, единственным элементом
				которого является некоторый sc-элемент}
			
			\scnsubset{sc-множество}
			\begin{scnindent}
				\scnidtf{sc-элемент, обозначающий множество,
					элементами которого являются только
					sc-элементы}
				\scnidtf{множество sc-элементов}
			\end{scnindent}
		\end{scnindent}
	\end{scnrelfromset}	
\end{SCn}

\begin{SCn}
	\scnheader{внутренний файл}
	\scnidtf{внутренний файл ostis-системы}
	\scnidtf{внутренний образ (копия), внешней
		информационной конструкции, хранимый в
		файловой памяти ostis-системы}
	\scnnote{Файловая память ostis-системы, хранящая
		различного рода информационные конструкции
		(образы, модели), не являющиеся
		sc-конструкциями, должна быть тесно связана с
		sc-памятью этой же ostis-системы. Как минимум
		каждый файл ostis-системы должен быть связан с
		тем sc-узлом, который является знаком этого
		файла (точнее, знаком синглетона, элементом
		которого является указанный файл)}
\end{SCn}

\begin{SCn}
	\scnheader{внешняя сущность, не являющаяся внутренним
		файлом}
\end{SCn}
\end{comment}

\vskip 5cm

\begin{SCn}
	\scnheader{Логико-семантическая классификация sc-элементов}
	\scnstartstruct
	
	\scnheader{sc-элемент}
	\begin{scnsubdividing}
		\scnitem{sc-константа}
		\begin{scnindent}
			\scnidtf{sc-элемент, логико-семантическим значением которого является он сам}
		\end{scnindent}
		\scnitem{sc-переменная}
		\begin{scnindent}
			\begin{scnsubdividing}
				\scnitem{sc-переменная 1-го уровня}
				\begin{scnindent}
					\scnidtf{sc-элемент, областью возможных значений которого является множество sc-констант}
				\end{scnindent}
				\scnitem{sc-переменная 2-го уровня}
				\begin{scnindent}
					\scnidtf{sc-элемент, возможными значениями которого являются переменные 1-го уровня}
				\end{scnindent}
				\scnnote{такие переменные (метапеременные) необходимы для описания логических языков}
				\scnitem{sc-переменная универсального типа}
				\begin{scnindent}
					\scnidtf{sc-переменная, на значения которой не накладывается никаких ограничений}
				\end{scnindent}
			\end{scnsubdividing}
		\end{scnindent}
	\end{scnsubdividing}	
	
	\scnendstruct \scnsourcecommentinline{Завершили представление \textit{Логико-семантической классификации sc-элементов}}
\end{SCn}

\begin{SCn}
	\scnheader{Классификация sc-элементов по темпоральным характеристикам обозначаемых ими сущностей}
	\scnstartstruct
	
	\scnheader{sc-элемент}
	\scnrelfrom{разбиение}{Признак постоянства существования сущностей, обозначаемых sc-элементами}
	\begin{scnindent}
		\begin{scneqtoset}
			\scnitem{обозначение постоянной сущности}
			\begin{scnindent}
				\scnidtf{обозначение постоянно существующей сущности}
			\end{scnindent}
			\scnitem{обозначение временной сущности}
			\begin{scnindent}
				\begin{scnsubdividing}
					\scnitem{обозначение внешней временной сущности}
					\begin{scnindent}
						\scnsuperset{обозначение внешней ситуации}
						\scnsuperset{обозначение внешнего события}
						\scnsuperset{обозначение внешнего процесса}
					\end{scnindent}
					\scnitem{обозначение внутренней временной сущности в sc-памяти}
					\begin{scnindent}
						\begin{scnsubdividing}
							\scnitem{обозначение ситуации в sc-памяти}
							\begin{scnindent}
								\scnidtf{обозначение ситуации, которая возникла или возникает в процессе обработки информации в sc-памяти}
							\end{scnindent}
							\scnitem{обозначение события в sc-памяти}
							\begin{scnindent}
								\scnidtf{обозначение события, которое произошло или произойдет в процессе обработки информации в sc-памяти}
							\end{scnindent}
							\scnitem{обозначение информационного процесса в sc-памяти}
							\begin{scnindent}
								\scnidtf{обозначение внутреннего процесса в sc-памяти, который происходит, произошёл или будет происходить}
							\end{scnindent}
						\end{scnsubdividing}
					\end{scnindent}
				\end{scnsubdividing}
			\end{scnindent}
		\end{scneqtoset}
	\end{scnindent} 
	
	\scnrelfrom{разбиение}{Признак статичности сущностей, обозначаемых sc-элементами}
	\begin{scnindent}
		\begin{scneqtoset}
			\scnitem{обозначение статической сущности}
			\begin{scnindent}
				\scnidtf{обозначение статичной сущности}
				\scnidtf{обозначение стационарной сущности}
				\scnidtf{обозначение сущности, изменения которой в рамках соответствующего отрезка времени считаются несущественными}
				\scnsuperset{обозначение статического sc-множества}
			\end{scnindent}
			\scnitem{обозначение динамической сущности}
			\begin{scnindent}
				\scnidtf{обозначение сущности изменяющейся во времени}
				\scnsuperset{обозначение динамического sc-множетсва}
			\end{scnindent}
		\end{scneqtoset}
	\end{scnindent} 
	
	\scnendstruct \scnsourcecommentinline{Завершили представление \textit{Классификации sc-элементов по темпоральным характеристикам обозначаемых ими сущностей}}
\end{SCn}

Когда речь идёт о темпоральных свойствах sc-элементов, следует чётко отличать:
\begin{textitemize}
	\item временный характер присутствия любого sc-элемента в составе той базы знаний (в той sc-памяти) ostis-системы, в которой он находится (когда-то он появляется, когда-то может быть удалён из sc-памяти;
	\item временный характер присутствия в sc-памяти всей заданной sc-конструкции (заданного множества sc-элементов) --- такую sc-конструкцию будем называть ситуацией в sc-памяти;
	\item временный характер существования внешней сущности, которую sc-элемент обозначает;
	\item статичный или динамичный (изменчивый) характер внешней сущности, обозначаемой sc-элементом динамический характер внешней сущности, предполагает наличие в sc-памяти описания процесса изменения состояния или конфигурации указанной внешней сущности;
	\item динамическое sc-множество (динамическую sc-конструкцию), являющееся отражением (динамической модели) соответствующего внешнего процесса (процесса, происходящего во внешней среде);
	\item динамическое sc-множество (динамическую sc-конструкцию), являющуюся отражением (динамической моделью) соответствующего внутреннего процесса (информационного процесса, происходящего в той же sc-памяти, в которой находится sc-элемент, обозначающий указанное динамическое sc-множество)
\end{textitemize}

\begin{SCn}
	\scnheader{Структурная классификация sc-констант}
	\scnnote{Данная классификация полностью аналогична Структурной классификации sc-элементов, в отличие от которой она описывает структурную классификацию только константных sc-элементов (sc-констант)}
	\scniselement{sc-структура}
	\scnrelboth{аналог}{Структурная классификация sc-элементов}
\end{SCn}
	
\begin{SCn}
	\scnheader{Структурная классификация sc-констант}
	\scnstartstruct
	
	\scnheader{sc-константа}
	\begin{scnsubdividing}
		\scnitem{sc-множество}
		\begin{scnindent}
			\begin{scnsubdividing}
				\scnitem{sc-связка}
				\begin{scnindent}
					\begin{scnsubdividing}
						\scnitem{sc-синглетон}
						\scnitem{sc-пара}
						\begin{scnindent}
							\begin{scnsubdividing}
								\scnitem{неориентированная sc-пара}
								\scnitem{ориентированная sc-пара}
								\begin{scnindent}
									\begin{scnsubdividing}
										\scnitem{sc-пара принадлежности}
										\begin{scnindent}
											\begin{scnsubdividing}
												\scnitem{sc-пара нечёткой принадлежности}
												\scnitem{sc-пара позитивной принадлежности}
												\begin{scnindent}
												\scnsuperset{sc-пара постоянной позитивной принадлежности}
												\begin{scnindent}
													\begin{scnreltoset}{пересечение множеств}
														\scnitem{sc-константа}
														\scnitem{постоянная сущность}
														\scnitem{статическая сущность}
														\scnitem{sc-пара позитивной принадлежности}	
														\begin{scnindent}
															\scnsuperset{sc-пара постоянной позитивной принадлежности}
															\begin{scnindent}
																\begin{scnreltoset}{пересечение множеств}
																	\scnitem{sc-константа}
																	\scnitem{постоянная сущность}
																	\scnitem{статическая сущность}
																	\scnitem{sc-пара позитивной принадлежности}	
																\end{scnreltoset}
															\end{scnindent}
															\scnsuperset{sc-пара временной позитивной принадлежности}
															\begin{scnindent}
																\begin{scnreltoset}{пересечение множеств}
																	\scnitem{sc-константа}
																	\scnitem{временная сущность}
																	\scnitem{динамическая сущность}
																	\scnitem{sc-пара позитивной принадлежности}	
																\end{scnreltoset}
															\end{scnindent}
														\end{scnindent}
													\end{scnreltoset}
												\end{scnindent}
												\scnsuperset{sc-пара временной позитивной принадлежности}
												\begin{scnindent}
													\begin{scnreltoset}{пересечение множеств}
														\scnitem{sc-константа}
														\scnitem{временная сущность}
														\scnitem{динамическая сущность}
														\scnitem{sc-пара позитивной принадлежности}	
													\end{scnreltoset}
												\end{scnindent}
												\end{scnindent}
												\scnitem{sc-пара негативной принадлежности}
											\end{scnsubdividing}
										\end{scnindent}
										\scnitem{ориентированнная sc-пара, не являющаяся sc-парой принадлежности}	
									\end{scnsubdividing}
								\end{scnindent}
							\end{scnsubdividing}
						\end{scnindent}
						\scnitem{sc-связка, не являющаяся синглетоном и парой}
					\end{scnsubdividing}
				\end{scnindent}
				\scnitem{sc-класс}
				\scnitem{sc-структура}
			\end{scnsubdividing}
		\end{scnindent}
		\scnitem{внешняя сущность}
		\begin{scnindent}
			\scnidtf{sc-элемент, являющийся знаком внешней сущности}
			\scnidtf{знак внешней сущности}
			\scnidtf{знак сущности, не являющейся sc-множеством (sc-конструкцией)}
		\end{scnindent} 
		\begin{scnindent}
			\begin{scnsubdividing}
				\scnitem{файл}
				\scnitem{внутренний файл}
				\scnitem{внешняя сущность, не являющаяся внутренним файлом}
				\scnitem{внешняя сущность, не являющаяся информационной конструкцией}
				\scnitem{информационная конструкция, не являющаяся ни sc-множеством, ни файлом}
			\end{scnsubdividing}
		\end{scnindent}
	\end{scnsubdividing}
	
	\scnendstruct\scnsourcecommentinline{Завершили представление \textit{Структурной классификация sc-констант}}
\end{SCn}

\newpage
\subsection{Уточнение смысла выделенных классов sc-элементов и связей между этими классами}
\label{sec_meaning_selected_classes_sc-elements}
Перейдём к детальному рассмотрению смысла классов \textit{sc-элементов} (sc-классов), введенных в представленных выше классификациях.

Указанные sc-классы рассматриваются в порядке их введения в представленных выше классификациях \textit{sc-элементов}.

Сначала поясним смысл понятий (sc-классов), введенных в Структурной классификации \textit{sc-элементов} и в Структурной классификации sc-констант.

\begin{SCn}
	\scnheader{sc-элемент}
	\scnidtf{обозначение множества}
	\scnidtf{sc-обозначение множества, представимого в SC-коде}
	\begin{scnsubdividing}
		\scnitem{sc-множество}
		\begin{scnindent}
			\scnidtf{обозначение множества \textit{sc-элементов}}
			\scnidtf{обозначение множества, все элементы которого являются \textit{sc-элементами}}
			\scnidtf{обозначение внутренней для sc-памяти сущности, то есть сущности, хранимой в sc-памяти}
		\end{scnindent}	
		\scnitem{обозначение внешней сущности}
		\begin{scnindent}
			\scnidtf{обозначение синглетона внешней сущности}
			\scnidtf{терминальный \textit{sc-элемент}}
		\end{scnindent}	
	\end{scnsubdividing}
	\begin{scnrelfromlist}{примечание}
			\scnfileitem{Каждый \textit{sc-элемент} является обозначением соответствующего множества.}
			\scnfileitem{Ко множествам, представимым в SC-коде, относятся либо множества, элементами которых являются \textit{sc-элементы} (sc-множества), либо синглетоны, элементами которых являются сущности, не являющиеся \textit{sc-элементами} (синглетоны внешних сущностей). Таким образом, строго говоря, не каждое множество может быть обозначено соответствующим \textit{sc-элементом}. Но каждое множество, не являющееся sc-множеством или синглетоном указанного выше вида может быть однозначно преобразовано в sc-множество и описано средствами \textit{SC-кода}. При этом теоретико-множественные свойства "нестандартных" для \textit{SC-кода} множеств совпадают со свойствами соответствующих им "стандартных" для SC-кода множеств.}
			\scnfileitem{Тот факт, что \underline{каждый} \textit{sc-элемент} является обозначением соответствующего множества (частным случае которых являются синглетоны \underline{внешних} описываемых сущностей), означает то, что базовым видом объектов, которыми оперирует \textit{SC-код} на синтаксической, семантическом и логическом уровне являются множества знаков, обозначающих различные множества. В этом смысле \textit{SC-код} имеет базовую теоретико-множественную основу.}
	\end{scnrelfromlist}

	\scnrelfrom{правила построения внешних идентификаторов sc-элементов заданного класса}{Общие правила построения внешних идентификаторов sc-элементов}
	\begin{scnindent}
		\scnidtf{Общие правила идентификации sc-элементов}
		\begin{scneqtoset}
			\scnfileitem{Принадлежность идентифицируемого \textit{sc-элемента} некоторым классам \textit{sc-элементов} (sc-классам) явно указывается во внешнем идентификаторе этого \textit{sc-элемента} (в SC-идентификаторе) с помощью соответствующих условных признаков:
			\begin{scnitemize}
				\item если первым символом sc-идентификатора является знак подчеркивания, то идентифицируемый \textit{sc-элемент} принадлежит Классу sc-переменных. По умолчанию считается, что идентифицируемый \textit{sc-элемент} принадлежит Классу sc-констант;
				\item если последним символом sc-идентификатора является символ ''звёздочка'', то идентифицируемый \textit{sc-элемент} принадлежит Классу обозначений неролевых отношений;
				\item если последним символом sc-идентификатора является апостроф, то идентифицируемый \textit{sc-элемент} принадлежит Классу обозначений ролевых отношений, каждое из которых является подмножеством Отношения принадлежности, то есть Класса всех константных позитивных пар принадлежности;
				\item если последним символом sc-идентификатора является символ ''\textasciicircum'', то идентифицируемый \textit{sc-элемент} принадлежит Классу обозначений параметров.
			\end{scnitemize}} 
			\scnfileitem{Слово ''обозначение'' в sc-идентификаторе означает то, что обозначаемая сущность может быть как константной, так и переменной.}
			\scnfileitem{В sc-идентификаторах можно делать следующие сокращения:
			\begin{scnitemize}
				\item ''sc-элемент, обозначающий \ldots '' --- ''обозначение''
				\item ''обозначение константного'' --- ''знак константного''
				\item ''знак константного'' --- ''константный''
				\item слово "константный" в sc-идентификаторах можно опускать, так как константность подразумевается по умолчанию
			\end{scnitemize}}
			\scnfileitem{Для каждого sc-элемента можно построить sc-идентификатор, являющийся именем собственным, которое всегда начинается с большой буквы (заглавной) буквы.}
			\scnfileitem{Если sc-элемент является обозначением некоторого класса sc-элементов, то этому sc-элементу можно поставить в соответствие не только имя собственное, но и имя нарицательное, которое начинается маленькой (строчной) буквы. Приведём пример семейства эквивалентных (синонимичных) sc-идентификатор, среди которых есть как имена собственные, так и имена нарицательные:
			\begin{scnitemize}
				\item Множество всевозможных обозначений sc-множеств
				\item Класс обозначений sc-множеств
				\item обозначение sc-множества
			\end{scnitemize}}
		\end{scneqtoset}
	\end{scnindent} 
\end{SCn}

\begin{SCn}
	\scnheader{обозначение sc-множества}
	\scnidtf{sc-элемент, являющийся знаком множества всевозможных обозначений sc-множеств}
	\begin{scnindent}
		\scniselement{имя собственное}
	\end{scnindent} 
	\scnidtf{Знак множества всевозможных обозначений sc-множеств}
	\begin{scnindent}
		\scniselement{имя собственное}
	\end{scnindent} 
	\scnidtf{Множество всевозможных обозначений sc-множеств}
	\begin{scnindent}
		\scniselement{имя собственное}
	\end{scnindent} 
	\scnidtf{Класс обозначений sc-множеств}
	\begin{scnindent}
		\scniselement{имя собственное}
	\end{scnindent} 
	\scnidtf{sc-элемент, являющийся обозначением множества sc-элементов}
	\begin{scnindent}
		\scniselement{имя нарицательное}
	\end{scnindent} 
	\scnidtf{sc-обозначение множества sc-элементов}
	\begin{scnindent}
		\scniselement{имя нарицательное}
	\end{scnindent}
	\scnidtf{обозначение множества, каждый элемент которого является sc-элементом}
	\scnidtf{обозначение информационной конструкции, принадлежащей SC-коду}
	\scnidtftext{часто используемый sc-идентификатор}{обозначение sc-конструкции}
	\begin{scnsubdividing}
		\scnitem{sc-множество}
		\begin{scnindent}
			\scnidtf{знак константного sc-множества}
			\scneq{\textup{(}обозначение sc-множества $ \bigcap $ sc-константа\textup{)}}
		\end{scnindent} 
		\scnitem{переменное sc-множество}
		\begin{scnindent}
			\scneq{\textup{(}обозначение sc-множества $ \bigcap $ sc-переменная\textup{)}}
		\end{scnindent}
	\end{scnsubdividing}
\end{SCn}

\begin{SCn}
	\scnheader{следует отличать*}
	\begin{scnhaselementset}
		\scnitem{обозначение sc-множества}
		\begin{scnindent}
			\scnidtf{обозначение sc-множества, которое может быть как константным sc-множеством, так и переменным sc-множеством}
			\scnidtf{обозначение внутренней для sc-памяти сущности}
			\scnidtf{обозначение внутренней sc-памяти информационной конструкции (sc-конструкции)}
			\begin{scnsubdividing}
			\scnitem{sc-множество}
			\begin{scnindent}
				\scnidtf{обозначение конкретного множества}
				\scnidtf{знак множества}
				\scneq{\textup{(}sc-константа $ \bigcap $ обозначение sc-множества\textup{)}}
				\scnidtf{конкретное sc-множество}
				\scnidtf{знак константного sc-множества}
				\scnidtf{константное sc-множество}
			\end{scnindent}
			\scnitem{переменное sc-множество}
			\begin{scnindent}
				\scnidtf{произвольное sc-множества}
				\scnidtf{обозначение произвольного sc-множества}
				\scneq{\textup{(}sc-переменная $ \bigcap $ обозначение sc-множества\textup{)}}
			\end{scnindent} 
			\end{scnsubdividing}
		\end{scnindent}
		\scnitem{sc-множество}
		\scnitem{переменное sc-множество}
		\scnitem{обозначение внешней сущности}
		\begin{scnindent}
			\scnidtf{обозначение сущности, не являющейся множеством sc-элементов (sc-множеством)}
			\scnsuperset{обозначение файла}
			\begin{scnindent}
				\scnidtf{обозначение файла, хранимого либо в файловой памяти той же ostis-системы, в sc-памяти которой хранится знак этого файла, либо в файловой памяти другой дополнительно указываемой компьютерной системы}
			\end{scnindent} 
			\scnsuperset{обозначение информационной конструкции, не являющейся ни sc-множеством, ни файлом}
			\begin{scnindent}
				\scnnote{Примером такой информационной конструкции является напечатанный текст, речевое сообщение, которой следует отличать от его записи в виде аудио-файла \ldots}
			\end{scnindent}
			\scnsuperset{обозначение внешней сущности, не являющейся информационной конструкцией}
			\begin{scnindent}
				\scnnote{Примером такой внешней сущности \ldots}
			\end{scnindent}
		\end{scnindent} 
	\end{scnhaselementset}
\end{SCn}

\begin{SCn}
	\scnheader{sc-множество}
	\scnidtf{sc-конструкция} 
	\scnidtf{информационная конструкция, принадлежащая SC-коду} 
	\scnidtftext{часто используемый sc-идентификатор}{SC-код} 
	\begin{scnindent}
		\scniselement{имя собственное}
	\end{scnindent} 
	\scnidtf{Множество всевозможных sc-конструкций}
	
	\scnheader{обозначение sc-связки}
	\begin{scnsubdividing}
		\scnitem{sc-связка}
		\scnitem{переменная sc-связка}
	\end{scnsubdividing}
\end{SCn}

\begin{SCn}
	\scnheader{sc-связка}
	\scnidtf{знак связи (связки) между sc-элементами} 
	\scnnote{Если элементами sc-связки являются знаки внешних сущностей, то sc-связка является отображением (моделью) некоторой связи, которая связывает указанные внешние сущности}
	\scntext{пояснение}{Понятие sc-связки --- это попытка формализации понятия \underline{целостности}, понятия перехода некоторой совокупности сущности в некоторое новое качество, которое не сводится к свойсвам каждой сущности, входящей в эту совокупность.
		Таким образом, связками следует считать:
		\begin{textitemize}
			\item множество всех чисел, являющихся слагаемыми для заданного числа;
			\item множество всех сотрудников заданной организации, в заданный момент времени;
			\item множество всех сотрудников заданной организации, которые работают в ней;
			\item множество всех точек некоторого отрезка;
			\item множество всех точек некоторого треугольника.
		\end{textitemize}}
	\scntext{примеры}{Примерами sc-связок являются:
		\begin{textitemize}
			\item конкретная окружность, (множество \underline{всех} точек, равноудаленных от некоторой заданной точки);
			\item конкретный отрезок (множество \underline{всех} точек, лежащих между двумя заданными точками с включением этих точек);
			\item конкретный линейный треугольник (множество \underline{всех} точек, лежащих между каждыми двумя из трёх заданных точек с включением этих точек);
			\item пары граничных точек различных отрезков;
			\item тройки вершин различных треугольников.
		\end{textitemize}}
\end{SCn}

\begin{SCn}
	\scnheader{обозначение sc-синглетона}
	\begin{scnsubdividing}
		\scnitem{sc-синглетон}
		\scnitem{переменный sc-синглетон}
	\end{scnsubdividing}
\end{SCn}

\begin{SCn}
	\scnheader{sc-синглетон}
	\scnidtf{sc-множество, являющиеся синглетоном}
	\scnidtf{одномощное sc-множество}
	\scnidtf{sc-множество, имеющее мощность, равную единице}
	\scnidtf{sc-элемент, обозначающий унарную sc-связку}
	\scnidtf{sc-обозначение унарной sc-связки}
	\scnidtf{унарная sc-связка}
	\scnidtf{обозначение sc-синглетона}
	\scnidtf{обозначение одномощного множества, единственный элемент которого является sc-элементом}
\end{SCn}

\begin{SCn}	
	\scnheader{обозначение sc-пары}
	\scniselement{sc-константа}
	\scniselement{sc-класс}
	\begin{scnindent} 
		\begin{scnsubdividing}
			\scnitem{sc-пара}
			\begin{scnindent}
				\scnidtf{константная sc-пара}
				\scnsubset{sc-константа}
				\scniselement{sc-константа}
				\scniselement{sc-класс}
			\end{scnindent} 
			\scnitem{переменная sc-пара}
			\begin{scnindent}
				\scniselement{sc-переменная}
				\scniselement{sc-константа}
				\scniselement{sc-класс}
			\end{scnindent} 
		\end{scnsubdividing}
	\end{scnindent}
\end{SCn}

\begin{SCn}
	\scnheader{sc-пара}
\end{SCn}

\begin{SCn}
	\scnheader{обозначение неориентированной sc-пары}
	\scnidtf{неориентированная sc-пара}
\end{SCn}

\begin{SCn}
	\scnheader{обозначение ориентированной sc-пары}
	\scnidtf{ориентированная sc-пара}
\end{SCn}

\begin{SCn}
	\scnheader{обозначение sc-пары принадлежности}
	\scnidtf{sc-пара принадлежности}
\end{SCn}

\begin{SCn}
	\scnheader{обозначение sc-пары принадлежности}
	\scnidtf{sc-пара принадлежности}
\end{SCn}

\begin{SCn}
	\scnheader{обозначение sc-пары нечёткой принадлежности}
	\scnidtf{sc-пара нечёткой принадлежности}
\end{SCn}

\begin{SCn}
	\scnheader{обозначение sc-пары  позитивной принадлежности}
	\scnidtf{sc-пара позитивной принадлежности}
\end{SCn}

\begin{SCn}
	\scnheader{sc-пара константной постоянной позитивной принадлежности}
	\scnidtf{константная позитивная постоянная sc-пара принадлежности}
\end{SCn}

\begin{SCn}
	\scnheader{sc-пара константной временной позитивной принадлежности}
\end{SCn}

\begin{SCn}
	\scnheader{обозначение sc-пары негативной принадлежности}
	\scnidtf{sc-пара негативной принадлежности}
\end{SCn}

\begin{SCn}
	\scnheader{обозначение sc-пары, не являющейся парой принадлежности}
	\scnidtf{sc-пара, не являющаяся парой принадлежности}
\end{SCn}

\begin{SCn}
	\scnheader{обозначение sc-связки, не являющейся синглетоном и парой}
	\scnidtf{sc-пара, не являющаяся синглетоном и парой}
\end{SCn}

\begin{SCn}
	\scnheader{обозначение sc-класса}
	\scnidtf{sc-класс}
	\begin{scnsubdividing}
		\scnitem{sc-класс}
		\scnitem{переменный sc-класс}
	\end{scnsubdividing}
	\begin{scnsubdividing}
		\scnitem{обозначение sc-класса обозначений sc-связок}
		\scnitem{обозначение sc-класса обозначений sc-классов}
		\scnitem{обозначение sc-класса обозначений sc-структор}
		\scnitem{обозначение sc-классов обозначений внешних сущностей}
	\end{scnsubdividing}
\end{SCn}

\begin{SCn}
	\scnheader{sc-класс}
	\begin{scnsubdividing}
		\scnitem{sc-класс sc-связок}
		\begin{scnindent}
			\scnsuperset{sc-отношение}
		\end{scnindent}
		\scnitem{sc-класс sc-классов}
		\begin{scnindent}
			\scnsuperset{sc-параметр}
		\end{scnindent}
		\scnitem{sc-класс sc-структур}
		\scnitem{sc-класс внешних сущностей}
		\begin{scnindent}
			\scnidtf{sc-класс sc-элементов, являющихся знаками внешних сущностей}
		\end{scnindent} 
		\scnitem{sc-класс sc-элементов разного структурного типа}
		\begin{scnindent}
			\scnhaselementrole{пример}{\scnfilelong{sc-константа}}
		\end{scnindent} 
	\end{scnsubdividing}
\end{SCn}

Перечислим основные основные виды sc-классов

\begin{SCn}
	\scnheader{sc-класс}
	\scnsuperset{sc-отношение}
	\begin{scnindent}
		\scnidtf{sc-класс sc-связок}
		\scnsuperset{бинарное sc-отношение}
		\begin{scnindent} 
			\begin{scnsubdividing}
				\scnitem{бинарное неориентированное sc-отношение}
				\scnitem{бинарное ориентированное sc-отношение}
				\begin{scnindent}
					\scnsuperset{ролевое sc-отношение}
				\end{scnindent} 
			\end{scnsubdividing}
		\end{scnindent}
	\end{scnindent} 
	\scnsuperset{sc-класс sc-классов}
	\begin{scnindent}
		\scnsuperset{sc-параметр}
	\end{scnindent}
	\scnsuperset{sc-класс sc-структур}
	\scnsuperset{sc-класс внешних сущностей}
	\begin{scnindent}
		\scnsuperset{sc-класс внутренних файлов}
	\end{scnindent}
	\scnsuperset{sc-класс эквивалентности}
	\begin{scnindent}
		\scnidtf{фактор-множество соответствующего отношения эквивалентности}
	\end{scnindent} 
	\scntext{пояснение}{Требованиями, предъявляемыми к каждому sc-классу являются:
		\begin{textitemize}
			\item бесконечность этого sc-множества;
			\item наличие общего свойства, присущего \underline{всем} элементам этого sc-множества, в частности, наличие его определения.
		\end{textitemize}
	}
\end{SCn}

\vskip 5cm

\begin{SCn}
	\scnheader{следует отличать*}
	\begin{scnhaselementset}
		\scnitem{sc-связка}
		\scnitem{sc-класс}
	\end{scnhaselementset}
	\begin{scnindent}
		\scntext{сравнение}{В отличие от sc-связки принципом формирования sc-класса является наличие общего свойства, присущего \underline{всем} элементам этого sc-класса \underline{и только им}, (или присущего всем сущностям, которые обозначаются указанными sc-элементами). Таким общим свойством может быть \underline{определение sc-класса} либо принадлежность одному из значений некоторого параметра, то есть одному из элементов фактор-множества, соответствующего некоторому отношению эквивалентности или толерантности.}
		\scntext{пояснение}{Примерами связок являются:
		\begin{textitemize}
			\item множество людей живущих сейчас (динамическое множество);
			\item множество сотрудников некоторой организации (динамическое множество);
			\item отрезок, треугольник.
		\end{textitemize}
		Здесь речь не идёт об эквивалентности свойств самих людей и геометрических точек безотносительно к тому, в состав чего они входят. Поэтому это не является sc-классом.
		}
	\end{scnindent}
	\begin{scnhaselementset}
		\scnitem{sc-класс эквивалентности}
		\begin{scnindent}
			\scnexplanation{В sc-класс входит не просто некоторое количество попарно эквивалентных между собой сущностей, а абсолютно \underline{все} такие сущности}
		\end{scnindent}
		\scnitem{sc-связка попарно эквивалентных сущностей}
	\end{scnhaselementset}
\end{SCn}

Приведём пример:

\begin{SCn}
	\scnheader{следует отличать*}
	\begin{scnhaselementset}
		\scnitem{множество \underline{всех} треугольников, подобных одному из них}
		\begin{scnindent}
			\scnsubset{sc-класс}
		\end{scnindent}
		\scnitem{конечное множество подобных треугольников}
		\begin{scnindent}
			\scnsubset{sc-связка попарно эквивалентных треугольников}
		\end{scnindent}
	\end{scnhaselementset}
	\begin{scnhaselementset}
		\scnitem{параметр}
		\begin{scnindent}
			\scnidtftext{часто используемый sc-идентификатор}{sc-параметр}
			\scnsubset{класс классов}
		\end{scnindent}
		\scnitem{признак различия}
		\begin{scnindent}
			\scnidtf{признак классификации}
		\end{scnindent}
	\end{scnhaselementset}
	\scnrelfrom{пояснение}{
	\scnstartset
	\scnheaderlocal{параметр}
	\scnsubset{бесконечное множество}
	\bigskip
	\scnheaderlocal{признак различия}
	\scnsubset{конечное множество}
	\begin{scnhaselementrolelist}{пример}
		\scnitem{Признак конечности множеств}
		\begin{scnindent}
			\begin{scneqtoset}
				\scnitem{конечное множество}
				\scnitem{бесконечное множество}				
			\end{scneqtoset}
		\end{scnindent}
		\scnitem{Признак наличия кратных элементов}
		\begin{scnindent}
			\begin{scneqtoset}
				\scnitem{мультимножество}
				\scnitem{множество без кратных вхождений элементов}
			\end{scneqtoset}
		\end{scnindent}
	\end{scnhaselementrolelist}
	
	\scnendstruct
}
\end{SCn}

\vskip 5cm

\begin{SCn}
	\scnheader{sc-класс}
	\scnrelfrom{правила построения внешних идентификаторов sc-элементов заданного класса}{Правила построения внешних идентификаторов sc-элементов, являющихся знаками sc-классов}
	\begin{scnindent}
		\begin{scneqtoset}
			\scnfileitem{Слово ''обозначение'' в начале идентификатора используется тогда, когда в идентифицируемый класс sc-элементов включаются знаки как константных, так и переменных сущностей соответствующего вида}
			\scnfileitem{Слово ''переменный'' в начале идентификатора и ...}
		\end{scneqtoset}
	\end{scnindent}
\end{SCn}

\begin{SCn}
	\scnheader{обозначение sc-структуры}
\end{SCn}

\begin{SCn}
	\scnheader{sc-структура}
\end{SCn}

\begin{SCn}
	\scnheader{следует отличать*}
	\begin{scnhaselementset}
		\scnitem{sc-структура}
		\scnitem{sc-связка}
	\end{scnhaselementset}
	\begin{scnindent} 
		\begin{scnrelfromset}{сравнение}
			\scnfileitem{В отличие от sc-связок в каждую sc-структуру должна входить по крайней мере одна sc-связка вместе с компонентами этой sc-связки}
		\end{scnrelfromset}
	\end{scnindent}
\end{SCn}

\begin{SCn}
	\scnheader{обозначение внешняя сущность}
\end{SCn}

\begin{SCn}
	\scnheader{внешняя сущность}
	\scnidtf{синглетон внешней сущности}
	\scnidtf{обозначение синглетона внешней сущности}
	\scnidtf{sc-элемент, обозначающий синглетон, элементом которого является некоторая внешняя описываемая сущность}
	\scnidtf{множество обозначаемой sc-элементом, являющееся 1-мощным множеством, единственным элементом которого является сущность, внешняя по отношению к sc-конструкции, то есть сущность, не являющаяся sc-элементом}
	\begin{scnrelfromlist}{примечание}
		\scnfileitem{обозначение внешней сущности, то есть sc-элемент, обозначающий этот синглетон, можно также трактовать как sc-элемент, обозначающий соответсвтующую внешнюю описываемую сущность, которую в свою очередь можно считать денодом указанного sc-элемента}
		\scnfileitem{очевидно, что пара принадлежности, связывающая sc-элемент, обозначающий синглетон внешней сущности, не может быть непосредственно представлена в виде соответствующей sc-дуги принадлежности, так как второй компонент этой sc-дуги не находится в sc-памяти}
	\end{scnrelfromlist}
\end{SCn}

\begin{SCn}
	\scnheader{следует отличать*}
	\begin{scnhaselementset}
		\scnitem{синглетон внешней сущности}
		\scnitem{sc-синглетон}
		\begin{scnindent}
			\scnidtf{синглетон, единственным элементом которого является некоторый sc-элемент}
			\scnsubset{sc-множество}
			\begin{scnindent}
				\scnidtf{sc-элемент, обозначающий множество, элементами которого являются \underline{только} sc-элементы}
				\scnidtf{множество sc-элементов}
			\end{scnindent} 
		\end{scnindent}
	\end{scnhaselementset}
\end{SCn}

\begin{SCn}
	\scnheader{обозначение файла}
\end{SCn}

\begin{SCn}
	\scnheader{файл}
\end{SCn}

\vskip 5cm

\begin{SCn}
	\scnheader{внутренний файл}
	\scnidtf{внутренний образ (копия), внутренней информационной конструкции, хранимый в файловой памяти ostis-системы}
	\scnidtf{файл ostis-системы}
	\begin{scnrelfromlist}{примечание}
		\scnfileitem{файловая память ostis-системы, хранящая различного рода информационные конструкции (образы, модели), не являющиеся sc-конструкциями, должна быть тесно связана с sc-памятью этой же ostis-системы. Как минимум каждый файл ostis-системы должен быть связан с тем sc-узлом, которых является знаком этого файла (точнее, знаком синглетона, элементом которого является указанный файл)}
	\end{scnrelfromlist}
\end{SCn}

Перейдем к пояснению смысла понятий используемых в \textit{Логической классификации sc-элементов}.

\begin{SCn}
	\scnheader{sc-константа}
	\scnidtf{sc-элемент, обозначающий константную сущность}
	\begin{scnindent}
		\scntext{сокращение}{обозначение константной сущности}
	\end{scnindent}
	\scnidtf{обозначение константной сущности}
	\scnidtf{знак константной сущности}
	\begin{scnindent}
		\scntext{сокращение}{константная сущность}
		\begin{scnindent}
			\scntext{сокращение}{сущность}
		\end{scnindent} 
	\end{scnindent}
	\scnidtf{константная сущность}
	\scnidtf{конкретная сущность}
	\scnidtf{сущность}
	\scnidtf{константный sc-элемент}
	\scnidtf{sc-элемент, имеющий одно логико-семантическое значение, каковым является он сам}
	\scnidtf{sc-элемент, являющийся знаком константной (конкретной, фиксированной) сущности}
	\scntext{сокращение}{знак константной (конкретной, фиксированной) сущности}
		\begin{scnindent} 
			\scntext{сокращение}{константная (конкретная, фиксированная) сущность}
			\begin{scnindent} 
				\scntext{сокращение}{константная сущность}
			\end{scnindent}
		\end{scnindent}
\end{SCn}

\begin{SCn}
	\scnheader{sc-переменная}
	\scnidtf{переменный sc-элемент}
	\scnidtf{sc-элемент, являющийся обозначением некоторой произвольной (нефиксируемой, переменной) сущности}
	\begin{scnindent}
		\scntext{сокращение}{обозначение произвольной (переменной) сущности}
		\begin{scnindent}
			\scntext{сокращение}{переменная сущность}
		\end{scnindent}
	\end{scnindent}
	\scniselement{sc-константа}
	\scniselement{sc-класс}
	\begin{scnrelfromlist}{примечание}
		\scnfileitem{Сам sc-элемент, имеющий внешний идентификатор ''sc-переменная'' является sc-константой (константным sc-элементом), которая является знаком одного из классов sc-элементов}
	\end{scnrelfromlist}
\end{SCn}

\begin{SCn}
	\scnheader{sc-элемент}
	\scnidtf{обозначение константной или переменной сущности}
	\scnidtf{константная или переменная сущность}
	\scnidtf{sc-константа или переменная сущность}
	\scnidtf{обозначение описываемой сущности, которая может быть как константой, так и переменной сущностью, как внутренней, так и внешней sc-конструкцией для заданной ostis-системы, как информационной конструкцией, так и сущностью которая информационной конструкцией не является, как временной сущностью, так и постоянной, как динамической, так и статической сущностью}
\end{SCn}

\vskip 5cm

\begin{SCn}
	\scnheader{обозначение sc-множества}
	\begin{scnsubdividing}
		\scnitem{sc-множество}
		\begin{scnindent}
			\scnidtf{часто используемый sc-идентификатор}
			\begin{scnindent}
				\scntext{сокращение}{множество sc-элементов}
			\end{scnindent} 
			\scnidtf{константное (конкретное) sc-множество}
			\scnidtf{обозначение (знак) конкретного множества}
			\scnsubset{sc-константа}
			\scniselement{sc-константа}
			\begin{scnsubdividing}
				\scnitem{множество констант}
				\begin{scnindent}
					\scnidtf{множество, каждый элемент которого является константой}
					\scnidtf{множество, являющееся подмножеством Множества всевозможных констант}
				\end{scnindent}
				\scnitem{множество переменных}
				\scnitem{множество констант и переменных}
				\begin{scnindent}
					\scnidtf{множество, элементами которого являются как константы, так и переменные}
				\end{scnindent}
				\scnitem{sc-множество sc-констант}
				\begin{scnindent}
					\scnidtf{sc-множество, элементами которого являются только sc-константы}
				\end{scnindent} 
				\scnitem{sc-множество sc-переменных}
				\begin{scnindent}
					\scnidtf{sc-множество, элементами которого являются только sc-переменные}
				\end{scnindent} 
				\scnitem{sc-множество sc-констант и sc-переменных}
				\begin{scnindent}
					\scnidtf{произвольное множество}
					\scnidtf{обозначение переменного (произвольного) sc-множества)}
					\scnsubset{sc-переменная}
					\scniselement{sc-константа}
					\scnrelboth{следует отличать}{sc-множество sc-переменных}
				\end{scnindent} 
			\end{scnsubdividing}
		\end{scnindent}
		\scnitem{переменное sc-множество}
		\begin{scnindent}
			\scnidtf{обозначение переменного (произвольного) sc-множества}
		\end{scnindent}
	\end{scnsubdividing}
\end{SCn}

Перейдем к пояснению смысла понятий, используемых в \textit{Классификации sc-элементов по темпоральным характеристикам обозначаемых ими сущностей}.

\begin{SCn}
	\scnheader{обозначение временной сущности}
	\begin{scnsubdividing}
		\scnitem{обозначение временной сущности существующей сейчас}
		\begin{scnindent}
			\scnidtf{обозначение временной сущности, существующей в текущий (настоящий) момент}
		\end{scnindent} 
		\scnitem{обозначение прошлой временной сущности}
		\begin{scnindent}
			\scnidtf{обозначение бывшей временной сущности}
			\scnidtf{обозначение временной сущности, которая уже перестала существовать, прекратила своё существование}
		\end{scnindent} 
		\scnitem{обозначение будущей временной сущности}
		\begin{scnindent}
			\scnidtf{обозначение временной сущности, появление которой прогнозируется (планируется, обеспечивается)}
			\scnnote{проектирование и производство новых, ранее не существующих полезных сущностей -- это основное направление человеческой деятельности}
		\end{scnindent} 
	\end{scnsubdividing}
	\begin{scnindent}
		\scnnote{ostis-системы должны постоянно мониторить состояние временных сущностей, так как в процессе их функционирования будущие сущности становятся настоящими, а настоящие -- прошлыми}
\end{scnindent} 
\end{SCn}

\vskip 5cm

\begin{SCn}
	\scnheader{динамическое sc-множество}
	\scnidtf{sc-процесс}
	\scnidtf{процесс}
	\scntext{определение}{sc-множество, у которого некоторые позитивные пары принадлежности, связывающие знак этого множества с его элементами, носят временный характер}
	\scnnote{Сами элементы динамического sc-множества, связанные с ним временными позитивными парами принадлежности, могут обозначать как временные, так и постоянные сущности. Но чаще всего такими временными элементами динамического sc-множества являются знаки временных связок.}
	\begin{scnsubdividing}
		\scnitem{внешний процесс}
		\scnitem{процесс в sc-памяти}
	\end{scnsubdividing}
\end{SCn}

\begin{SCn}
	\scnheader{темпоральная декомпозиция динамического sc-множества}
	\scnidtf{покадровая развертка динамического sc-множества}
	\scnidtf{представление sc-множества в виде кортежа (последовательности) ситуаций}
\end{SCn}

\begin{SCn}
	\scnheader{следует отличать*}
	\begin{scnhaselementset}
		\scnitem{временная сущность}
		\scnitem{обозначение временной сущности}
		\scnitem{переменная временная сущность}
	\end{scnhaselementset}
\end{SCn}

\begin{SCn}
	\scnheader{обозначение временной сущности}
	\begin{scnsubdividing}
		\scnitem{временная сущность}
		\begin{scnindent}
			\scnidtf{знак конкретной (константной) временной сущности}
		\end{scnindent} 
		\scnitem{переменная временная сущность}
		\begin{scnindent}
			\scnidtf{обозначение произвольной временной сущности}
		\end{scnindent} 
	\end{scnsubdividing}
\end{SCn}

\begin{SCn}
	\scnheader{сформированное sc-множество}
	\scnidtf{sc-множество, у которого в текущем состоянии sc-памяти перечислены все его элементы}
	\scniselement{динамическое sc-множество}
	\scnnote{Очевидно, что сформированным sc-множеством может стать только конечное sc-множество}
\end{SCn}

\begin{SCn}
	\scnheader{формируемое sc-множество}
\end{SCn}

\begin{SCn}
	\scnheader{sc-множество, элементы которого не известны}
\end{SCn}

\begin{SCn}
	\scnheader{сформированный файл}
\end{SCn}

\begin{SCn}
	\scnheader{формируемый файл}
\end{SCn}

\begin{SCn}
	\scnheader{файл, структура которого не известна}
\end{SCn}

Перейдем к рассмотрению семантически выделяемых классов sc-элементов, которые необходимо ввести \underline{дополнительно} к выше рассмотренным классам sc-элементов.

\begin{SCn}
	\scnheader{sc-элемент, не являющийся sc-синглетоном и sc-парой}
\end{SCn}

\begin{SCn}
	\scnheader{sc-элемент, копируемый в других компьютерных системах}
	\scnidtf{sc-элемент, имеющий в других компьютерных системах свои копии и/или копии обозначаемой им информационной конструкции}
\end{SCn}

\begin{SCn}
	\scnheader{отношение, заданное на множестве sc-элементов, копируемых в других компьютерных системах*}
	\scnhaselement{}
	\scnhaselement{}
	\scnhaselement{}
\end{SCn}

\begin{SCn}
	\scnheader{отношение, заданное на множестве sc-элементов, имеющих копии в других компьютерных системах*}
	\scnhaselement{ostis-система, в sc-памяти которой хранится копия заданного sc-элемента*}
	\scnhaselement{компьютерная система, в файловой памяти которой хранится заданный файл*}
	\begin{scnindent}
		\scnnote{указанная компьютерная система назначается хранителем файла}
	\end{scnindent}
	\scnhaselement{ostis-система, в sc-памяти которой хранится копия знака заданного sc-множества и все известные в текущий момент его элементы*}
	\begin{scnindent}
		\scnnote{указанная ostis-система назначается основным хранителем указанного sc-множества}
	\end{scnindent}
\end{SCn}

\begin{SCn}
	\scnheader{информационная конструкция}
	\begin{scnsubdividing}
		\scnitem{sc-множество}
		\begin{scnindent}
			\scnidtf{sc-конструкция}
			\scnidtf{информационная конструкция SC-кода}
			\scnidtf{внутренняя информационная конструкция ostis-системы, хранимая в её sc-памяти}
		\end{scnindent}
		\scnitem{файл}
		\begin{scnindent}
			\scnidtf{файл ostis-системы}
			\scnidtf{внутреняя информационная конструкция ostis-системы, хранимая в её файловой памяти}
			\scnnote{файл, может храниться в памяти другой компьютерной системы и, в частности, в файловой памяти другой ostis-системы}
		\end{scnindent}
		\scnitem{внешняя информационная конструкция, не являющаяся ни файлом, ни sc-конструкцией}
	\end{scnsubdividing}
\end{SCn}

\begin{SCn}
	\scnheader{sc-идентификатор}
	\scnidtf{внешний идентификатор sc-элемента}
	\scnsuperset{файл}
	\begin{scnsubdividing}
		\scnitem{основной идентификатор}
		\scnitem{часто используемый sc-идентификатор}
		\scnitem{дополнительный sc-идентификатор}
	\end{scnsubdividing}
\end{SCn}

\begin{SCn}
	\scnheader{sc-идентификатор*}
	\scnidtf{Бинарное ориентированное отношение, связывающее sc-элементы с их внешними идентификаторами}
\end{SCn}

\newpage
\subsection{Структура базовой семантической спецификации sc-элемента}
\label{sec_structure_basic_semantic_specification_sc-element}
\begin{SCn}
	\scnheader{базовая семантическая спецификация sc-элемента}
	\scnidtfexp{Класс sc-структур, каждая из которых описывает базовые семантические свойства (характеристики) соответствующего (описываемого, специфицируемого) sc-элемента}
	\scnsubset{sc-структура}
	\begin{scnindent}
		\scnsubset{sc-спецификация}
		\begin{scnindent}
			\scnidtf{представленная в SC-коде семантическая окрестность (спецификация) некоторого (специфицируемого) sc-элемента}
		\end{scnindent}
	\end{scnindent}
	\scnsubset{sc-спецификация}
	\scnrelto{второй домен}{базовая семантическая спецификация sc-элемента*}
	\begin{scnindent}
		\scnidtfexp{Бинарное ориентированное отношение, каждая пара которого связывает sc-элемент с его базовой семантической спецификацией*}
	\end{scnindent}
	\scnidtfexp{хранимая в sc-памяти ostis-система спецификация каждого sc-элемента, необходимая для эффективной обработки этого sc-элемента}
	\scnnote{базовая спецификация sc-элементов осуществляется как явно с помощью соответствующих sc-конструкций, так и неявно с помощью соответствующих семантических меток, приписываемых sc-элементам}
	\scntext{пояснение}{Базовая семантическая спецификация каждого sc-элемента включает в себя:
		\begin{scnitemize}
			\item перечисление всех тех \textit{базовых классов sc-элементов}, которым принадлежит специфицируемый sc-элемент;
			\item уточнение ''привязки'' рассматриваемых временных сущностей к текущему моменту времени и другим моментам времени;
			\item уточнение того, какие важные характеристики специфицируемого sc-элемента в текущем состоянии sc-памяти и файловой памяти ostis-системы не известны.
		\end{scnitemize}
	}
\end{SCn}

\textbf{Базовая семантическая спецификация sc-элемента, обозначающего временную сущность} включает в себя указание дополнительных темпоральных характеристик, позволяющих уточнить темпоральные ''координаты'' этих временных сущностей (то есть их ''координаты'' во времени), а также их основные темпоральные связи с другими временными сущностями. К числу понятий, обеспечивающих описание указанных темпоральных характеристик временных сущностей, относятся:
\begin{textitemize}
	\item момент времени\textasciicircum
	\item Текущий момент времени
	\item прошлая сущность
	\item будущая сущность 
	\item момент начала*
	\item момент завершения*
	\item внешняя ситуация
	\item ситуация в sc-памяти
	\item внешнее событие
	\item событие в sc-памяти
	\item внешний процесс
	\item процесс в sc-памяти
\end{textitemize}

\vskip 5cm

\begin{SCn}
	\scnheader{момент времени\textasciicircum}
	\scniselement{параметр}
	\scniselement{параметр, заданный на множестве временных сущностей}
	\scnidtf{глобальная приблизительно точная ситуация\textasciicircum}
	\scnidtf{глобальная ситуация пренебрежительно малого отрезка времени\textasciicircum}
	\scnidtf{множество (класс) \underline{всех} временных сущностей, существующих одновременно в соответствующий момент времени\textasciicircum}
	\scnnote{момент времени, соответствующий глобальной точечной ситуации может быть задан с различной и \underline{дополнительно} \underline{указываемой} степенью точности --- с точностью до секунды, до минуты, до часа, до даты, до календарного месяца, до календарного года и так далее. В том смысле корректнее говорить не о моменте времени, а об интервале времени, длительность которого считается пренебрежимо малой для рассмотрения описываемых процессов}
\end{SCn}

\begin{SCn}
	\scnheader{Текущий момент времени}
	\scnidtf{Глобальная ситуация текущего (настоящего) момента времени}
	\scnidtf{Глобальная ситуация, имеющая место сейчас}
	\scnidtf{Класс всех сущностей, существующих в настоящий момент времени}
	\scniselement{sc-синглетон}
	\scniselement{динамическое sc-множество}
	\scnrelto{включение множества}{момент времени}
	\scnexplanation{Из знака \textit{текущего момента времени} (который является также знаком \textit{sc-синглетона}) ''выходит'' sc-дуга (sc-пара) \underline{временной} принадлежности, представляющая собой, образно говоря, ''стрелку'' внутренних часов ostis-системы, которая всегда указывает только на один элемент множества моментов времени, но в разные моменты времени указывает на разные элементы этого множества}
\end{SCn}

\begin{SCn}
	\scnheader{прошлая сущность}
	\scnidtf{временная сущность, уже завершившая своё существование}
\end{SCn}

\begin{SCn}
	\scnheader{будущая сущность}
	\scnidtf{прогнозируемая, планируемая или создаваемая временная сущность}
\end{SCn}

\begin{SCn}
	\scnheader{момент начала*}
	\scnidtf{момент времени, соответствующий началу существования заданной временной сущности}
	\scnidtf{Бинарное ориентированное отношение, каждая пара которого, связывает (1) знак некоторой временной сущности и (2) глобальную точечную ситуацию (значение параметра ''\textit{момент времени}\textasciicircum''), элементом которой является условно точечная временная сущность, представляющая собой начальный этап существования временной сущности, указанной в первом компоненте рассматриваемой ориентированной пары}
	\scnnote{Начальный этап существования временной сущности (переходный процесс от небытия к реальному существованию) может рассматриваться с любой степенью детализации}
	\scnrelfrom{первый домен}{временная сущность}
	\scnrelfrom{второй домен}{момент времени\textasciicircum}
\end{SCn}

\begin{SCn}
	\scnheader{момент завершения*}
	\scnidtf{момент времени, соответствующий завершению существования заданной временной сущности}
\end{SCn}

\begin{SCn}
	\scnheader{ситуация}
	\begin{scnsubdividing}
		\scnitem{внешняя ситуация}
		\scnitem{ситуация в sc-памяти}
	\end{scnsubdividing}
\end{SCn}

\vskip 5cm

\begin{SCn}
	\scnheader{событие}
	\begin{scnsubdividing}
		\scnitem{внешнее событие}
		\scnitem{событие в sc-памяти}
	\end{scnsubdividing}
\end{SCn}

\begin{SCn}
	\scnheader{динамическое sc-множество}
	\begin{scnsubdividing}
		\scnitem{внешний процесс}
		\begin{scnindent}
			\scnidtf{процесс, происходящий в окружающей среде ostis-системы}
		\end{scnindent}
		\scnitem{процесс в sc-памяти}
	\end{scnsubdividing}
\end{SCn}

\begin{SCn}
	\scnheader{внешняя ситуация}
	\scnidtf{ситуация во внешней среде}
	\scnidtf{ситуация \underline{одновременного} существования (в соответствующий период времени) указанных временных внешних сущностей}
	\scnsubset{временная сущность}
	\scnsubset{sc-структура}
	\scnsubset{sc-константа}
	\scnsubset{обозначение внешней ситуации}
	\scniselement{sc-класс}
\end{SCn}

\begin{SCn}
	\scnheader{класс внешних ситуаций}
\end{SCn}

\begin{SCn}
	\scnheader{обобщенное описание}
\end{SCn}

\begin{SCn}
	\scnheader{класс внешних ситуаций}
	\scnnote{В простейшем случае внешние ситуации, входящие в класс внешних ситуаций являются изоморфными}
\end{SCn}

\begin{SCn}
	\scnheader{внешний процесс}
	\scnidtf{темпоральная детализация внешней динамической сущности}
\end{SCn}

\begin{SCn}
	\scnheader{внешнее событие}
	\scnidtf{факт появления (возникновения) некоторой внешней сущности (в том числе некоторой внешней ситуации) или факт завершения существования некоторой внешней сущности (в том числе некоторой внешней ситуации)}
\end{SCn}

\begin{SCn}
	\scnheader{ситуация в sc-памяти}
	\scnidtf{внутрення ситуация}
	\begin{scnindent}
		\scnidtf{sc-ситуация}
		\scnidtf{хранимый в sc-памяти фрагмент базы знаний, рассматриваемый в контексте его появления в sc-памяти или его исчезновения (из-за удаления некоторые sc-элементов)}
	\end{scnindent}
\end{SCn}

\begin{SCn}
	\scnheader{класс ситуаций в sc-памяти}
	\scnidtf{класс внутренних ситуаций}
\end{SCn}

\begin{SCn}
	\scnheader{обобщённое описание класса ситуаций в sc-памяти}
\end{SCn}

\begin{SCn}
	\scnheader{процесс в sc-памяти}
	\scnidtf{внутренний процесс}
	\scnidtf{информационный процесс, происходящий в sc-памяти}
	\scnidtf{sc-процесс}	
\end{SCn}

\begin{SCn}
	\scnheader{событие в sc-памяти}
\end{SCn}

Важной частью \textit{базовой семантической спецификации sc-элемента} является фиксация того, что ostis-система знает и чего она не знает о специфицируемом sc-элементе или об обозначенной им сущности:

\begin{textitemize}
	\item Если в спецификации sc-элемента указывается его принадлежность к некоторому классу sc-элементов, но не указывается его принадлежность \underline{одному} из подклассов, на которые \textit{разбивается} указанный выше класс, то это означает, что в текущий момент времени ostis-система этого \underline{не знает};
	\item Если специфицируемый sc-элемент является обозначением \underline{конечного} множества sc-элементов (в частности, пары sc-элементов), и если в текущий момент времени ostis-системе не известны \underline{все} этого множества (то есть специфицируемый sc-элемент не соединён соответствующими парами принадлежности со \underline{всеми} элементами обозначаемого им множества sc-элементов), то этот специфицируемый sc-элемент следует отнести к sc-классу ''\textbf{\textit{обозначение несформированного sc-множества}}'';
	\item Если специфицируемый sc-элемент является обозначением ориентированной sc-пары и если в текущий момент времени ostis-системе не известна \underline{направленность} этой ориентированной пары sc-элементов (то есть не известно, какой элемент этой пары является первым её компонентом, а какой её элемент является её вторым компонентом), то этот специфицируемый sc-элемент следует отнести к sc-классу ''\textbf{\textit{обозначение ориентированной sc-пары неизвестной направленности}}''.
\end{textitemize}

К числу понятий, используемых для описания полноты базовой спецификации sc-элементов, относятся:

\begin{textitemize}
	\item \textit{обозначение бесконечного sc-множества};
	\item \textit{обозначение конечного sc-множества};
	\item \textit{мощность обозначаемого sc-множества*};
	\item \textit{обозначение sc-множества неизвестной мощности};
	\item \textit{обозначение sc-множества, о котором не известно, является ли оно sc-парой};
	\item \textit{обозначение sc-пары, о которой не известно, является ли она ориентированной или нет};
	\item \textit{обозначение ориентированной sc-пары неизвестной направленности};
	\item \textit{обозначение сформированного sc-множества};
	\item \textit{обозначение \underline{частично} сформированного sc-множества};
	\item \textit{обозначение \underline{полностью} несформированного sc-множества};
	\item \textit{обозначение сформированного файла};
	\item \textit{обозначение частично сформированного файла};
	\item \textit{обозначение полностью несформированного файла};
\end{textitemize}

Подчеркнём то, что базовую семантическую спецификацию должны иметь абсолютно все \textit{sc-элементы}, хранимые в \textit{sc-памяти} в текущий момент времени, в том числе и все \textit{sc-элементы}, являющиеся ключевыми знаками в рамках \textit{Предметной области Базовой денотационной семантики SC-кода}. Приведём пример базовой семантической sc-спецификации одного из таких sc-элементов:

\begin{SCn}
	\scnheader{обозначение sc-множества}
	\scnidtf{Множество всевозможных sc-элементов, обозначающих sc-множества}
	\begin{scnindent}
		\scniselement{имя собственное}
	\end{scnindent}
	\scniselement{обозначение sc-множества}
	\scnnote{Одним из элементов данного множества является знак, обозначающий это множество. Это означает, это данное множество является рефлексивным множеством}
	\scniselement{обозначение множества sc-элементов разного структурного типа}
	\scntext{примечание}{Элементами данного множества являются:
		\begin{textitemize}
			\item обозначения sc-синглетонов;
			\item обозначения sc-пар;
			\item обозначения sc-связок, не являющихся sc-синглетонами или sc-парами;
			\item обозначения sc-классов;
			\item обозначения sc-структур.
		\end{textitemize}
	}
	\scniselement{обозначение множества sc-элементов, содержащего как константные, так и переменные sc-элементы}
	\scniselement{sc-константа}
	\scnnote{Само данное множество является константным, несмотря на то, что его элементами являются как sc-константы, так и sc-переменные}
	\scniselement{обозначение множества sc-элементов, содержащего sc-элементы, обозначающие как постоянные, так и временные сущности}
	\scniselement{постоянная сущность}
	\scnnote{Слудует отличать постоянство~/~временность сущности, обозначаемой sc-элементом и постоянство~/~временность sc-множества, одним из элементов которого указанный sc-элемент является}
	\scniselement{обозначение множества sc-элементов, содержащего sc-элементы, обозначающие как статические, так и динамические sc-множества}
	\scniselement{статическое sc-множество}
	\begin{scnrelfromlist}{примечание}
		\scnfileitem{Следует отличать статичность~/~динамичность sc-множества, обозначаемого соответствующим sc-элементом и статичность~/~динамичность sc-множества, одним из элементов которого указанный выше sc-элемент является}
		\scnfileitem{Напомним, что статический характер sc-множества означает отсутствие временных sc-пар принадлежности (временных sc-дуг принадлежности), выходящих из нака этого sc-множества}
	\end{scnrelfromlist}
	\scniselement{sc-класс}
	\scnidtf{Класс всевозможных sc-элементов, обозначающих sc-множества}
	\scnidtf{Класс обозначающий sc-множеств}
	\scnnote{Следует отличать разные sc-элементы, являющиеся обозначениями соответствующих sc-множеств, и класс, элементами которого являются \underline{всевозможные} такие sc-элементы}
\end{SCn}

\newpage
\subsection{Онтологическая формализация Базовой денотационной семантики SC-кода}
\label{sec_ontological_formalization_basic_denotational_semantics_sc-code}

Суть онтологической формализации различных областей знаний, различных фрагментов \textit{баз знаний} интеллектуальных компьютерных систем заключается в следующем:
\begin{textitemize}
	\item Выделяется достаточно большой \underline{семантически целостный} фрагмент \textit{баз знаний}, включающий в себя:
	\begin{textitemize}
		\item все элементы некоторого одного ключевого класса рассматриваемых объектов (объектов исследования) или \underline{конечного} числа таких ключевых классов объектов исследования;
		\item \underline{все связи} между выделенными объектами исследования, соответствующие заданному \underline{семейству} отношений, параметров и классов структур, которое условно будем называть предметом исследования. 
	\end{textitemize}
	\item Указанный семантически целостный фрагмент \textit{базы знаний}, являющийся чаще всего \underline{бесконечной} структурой, будем называть \textbf{\textit{предметной областью}}.
	\item Сама формальная \textbf{\textit{онтология}} представляет собой формальную спецификацию выделенной \textit{предметной области} и включает в себя следующие \textbf{\textit{частные онтологии}}:
	\begin{textitemize}
		\item \textbf{\textit{структурную спецификацию предметной области}}, в которой указываются роли всех ключевых элементов (ключевых знаков), входящих в состав \textit{предметной области}. К числу таких ролей относятся:
		\begin{textitemize}
			\item \textit{максимальный класс объектов исследования\scnrolesign}
			\item \textit{немаксимальный класс объектов исследования\scnrolesign}
			\item \textit{ключевой объект исследования\scnrolesign}
			\item \textit{исследуемый класс связок\scnrolesign}
			\item \textit{исследуемый класс классов\scnrolesign}
			\item \textit{исследуемый класс структур\scnrolesign}
			\item \textit{неисследуемый класс\scnrolesign}
			\begin{SCn}
				\scnidtf{sc-класс, исследуемый в другой (смежной) предметной области}
			\end{SCn}
		\end{textitemize}
		\item \textbf{\textit{теоретико-множественную онтологию}}, в которой описываются теоретико-множественные связи между всеми классами (\textit{sc-классами}), исследуемыми в рамках заданной (специфицируемой) \textit{предметной области}
		\item \textbf{\textit{логическую онтологию}}, которая включает в себя
		\begin{textitemize}
			\item определения исследуемых классов (исследуемых понятий);
			\item логическую иерархию исследуемых понятий, которая связывает каждое понятие со множеством тех понятий, которые явно используются в определении этого понятия;
			\item аксиомы и теоремы, описывающие свойства специфицируемой предметной области;
			\item тексты доказательств теорем;
			\item логическую иерархию теорем, которая связывает каждую теорему со множеством теорем, на основе которых она доказывается.
		\end{textitemize}
		\item \textbf{\textit{терминологическую спецификацию предметной области}}, в которой указывается \textit{sc-идентификаторы} всех ключевых \textit{sc-элементов} специфицируемой \textit{предметной области}, а также приводятся правила построения \textit{основных sc-идентификаторов} для всех \textit{sc-классов} (понятий), исследуемых в рамках специфицируемой \textit{предметной области};
		\item \textbf{\textit{дидактическую спецификацию предметной области}}, в которой приводится дополнительная информация, предназначенная для того, чтобы пользователи и разработчики (инженеры знаний), которые используют или совершенствуют специфицируемую предметную область и её онтологию, могли быстрее усвоить их особенности (см.~\textit{\ref{section_knowledge_control}~\nameref{section_knowledge_control}})
		\item \textbf{\textit{проектную спецификацию предметной области и соответствующей ей онтологии}}, в которой приводится информация об истории эволюции этой \textit{предметной области и онтологии}, а также о направлениях и планах организации дальнейшего их развития.
	\end{textitemize}
\end{textitemize}	 

Более подробно о предметных областях см. в~\textit{\ref{sec_sd}~\nameref{sec_sd}}, а более детальное рассмотрение формальных онтологий, представленных в SC-коде см. в~\textit{\ref{sec_ontology}~\nameref{sec_ontology}}.

Онтологическая формализация базовой денотационной семантики \textit{SC-кода} трактуется нами как \textit{формальная онтология}, представленная в \textit{SC-коде} и описывающая детонационную семантику \textit{семантически корректных sc-конструкций}. Указанную \textit{формальную онтологию} будем называть \textbf{\textit{Базовой денотационной семантикой SC-кода}}. Для того, чтобы уточнить \textit{предметную область}, специфицируемую этой \textit{онтологией}, введём следующие понятия:

\vskip 5cm

\begin{SCn}
	\scnheader{синонимия sc-элементов}
	\scnidtf{Бинарное ориентированное \textit{отношение эквивалентности}, каждая пара которого связывает два \textit{sc-элемента}, обозначающие одну и ту же сущность*}
	\scnnote{Синонимия двух \textit{sc-элементов} возможна только в том случае, если эти \textit{sc-элементы} хранятся в \textit{sc-памяти} (входят в состав \textit{базы знаний}) \underline{разных} \textit{ostis-систем}. В рамках каждой \textit{ostis-системы} синонимичные \textit{sc-элементы} совпадают (отождействляются, склеиваются, считаются одним и тем же \textit{sc-элементом}).}
\end{SCn}

\begin{SCn}
	\scnheader{отношение эквивалентности}
	\scnrelto{ключевое понятие}{\textsection~2.4.2. Формальная онтология связок и отношений}
\end{SCn}

\begin{SCn}
	\scnheader{sc-память}
\end{SCn}

\begin{SCn}
	\scnheader{база знаний ostis-системы}
\end{SCn}

\begin{SCn}
	\scnheader{ostis-система}
	\begin{scnsubdividing}
		\scnitem{индивидуальная ostis-система}
		\scnitem{коллективная ostis-система}
	\end{scnsubdividing}
\end{SCn}

\begin{SCn}
	\scnheader{sc-конструкция}
	\scnrelto{часто используемый sc-идентификатор}{sc-множество}
	\scnidtf{информационная конструкция, представляющая собой множество sc-элементов}
	\scnsuperset{sc-текст}
	\begin{scnindent}
		\scnidtf{текст SC-кода}
		\scnidtf{sc-конструкция, являющаяся семантически корректной по отношению к Базовой денотационной семантике SC-кода}
		\scnidtf{sc-конструкция, удовлетворяющая (соответствующая) правилам Базовой денотационной семантики SC-кода}
		\scnidtftext{часто используемый sc-идентификатор}{SC-код}
		\begin{scnindent}
			\scniselement{имя собственное}
			\scnidtf{Класс (Множество всевозможных) sc-текстов}
		\end{scnindent}
		\scnsuperset{sc-знание}
	\end{scnindent} 
\end{SCn}

\begin{SCn}
	\scnheader{sc-знание}
	\scnidtf{sc-текст, являющийся либо фрагментом (подструктурой) соответствующей \textit{предметной области}, либо \textit{высказыванием}, описывающим некоторое свойство (в частности, некоторую закономерность) этой \textit{предметной области}}
	\scnidtf{знание, представленное в SC-коде}
	\scnidtf{\textit{sc-текст}, обладающий истинным значением по отношению к соответствующей \textit{предметной области}}
	\scnsubset{связная sc-конструкция}
	\scnnote{Разные sc-знания могут противоречить друг другу, то есть отражать разные точки зрения на некоторую предметную область, но любое sc-знание должно быть sc-текстом, то есть не должно противоречить правилам Базовой денотационной семантики SC-кода}
\end{SCn}

\begin{SCn}
	\scnheader{интеграция sc-конструкций*}
	\scnidtf{объединение sc-конструкций*}
	\scnidtf{объединение sc-множеств*}
	\scnnote{При интеграции sc-конструкций sc-элементы, обозначающие одну и ту же сущность, то есть синонимичные sc-элементы, считаются одинаковыми (совпадающими, тождественными) и, следовательно, должны склеиваться (отождествляться)}
\end{SCn}

\begin{SCn}
	\scnheader{SC-пространство}
	\scnidtf{Результат интеграции \underline{всевозможных} sc-конструкций, \textit{семантически корректных} по отношению к \textit{Базовой денотационной семантики SC-кода}}
	\scnidtf{Предметная область, специфицируемая (описываемая) \textit{Базовой денотационной семантикой SC-кода}, которая является формальной онтологией, представленной средствами SC-кода}
	\scnidtf{Результат интеграции всевозможных sc-текстов (текстов SC-кода)}
	\scnidtf{Максимальный sc-текст}
	\scnidtf{Текст SC-кода, включающий в себя всевозможные sc-тексты}
	\scnidtf{Пространство sc-конструкций, семантически корректных по отношению к \textit{Базовой денотационной семантике SC-кода}}
	\begin{scnrelfromlist}{примечание}
		\scnfileitem{Особенностью \textit{SC-пространство} является то, что оно включает в себя и формальную онтологию, описывающую его свойства}
		\scnfileitem{очевидно, что \textit{SC-пространство} является \underline{бесконечным} \textit{sc-текстом}, то есть текстом, содержащим бесконечное количество \textit{sc-элементов}. В частности, в состав \textit{SC-пространства} входят \underline{все} \textit{sc-элементы}, являющиеся элементами \underline{всех} \textit{sc-множеств}, знаки которых входят в состав \textit{SC-пространства}}
		\scnfileitem{\textit{SC-пространство} является ''вместилищем'' семантически корректных (по отношению к \textit{Базовой денотационной семантике SC-кода}) частей баз знаний всевозможных ostis-систем и, в том числе, глобальной (объединенной) \textit{Базы знаний Экосистемы OSTIS}. Подчеркнём при этом, что \textit{Экосистема OSTIS} является примером распределённых иерархических \textit{ostis-систем}}
		\scnfileitem{Тот факт, что корректная (с точки зрения \textit{Базовой денотационной семантики SC-кода}) часть базы знаний \underline{каждой} \textit{ostis-системы} входит в состав \textit{SC-пространства}, позволяет трактовать описание соотношения между текущим состоянием \textit{базы знаний ostis-системы} и \textit{Sc-пространством} как описание того, что указанная \textit{ostis-система} в текущий момент времени не знает. Например, \textit{ostis-система} в некоторый момент времени может не знать (1) всех элементов некоторого конкретного \underline{конечного} \textit{sc-множества} (конечно sc-конструкции), (2) количества элементов указанного конечного \textit{sc-множества}, (3) какому подклассу заданного \textit{sc-класса} принадлежит указанный элемент этого \textit{sc-класса} и так далее}
		\scnfileitem{В \textit{памяти ostis-системы} каждый \textit{sc-элемент} считается в рамках этой памяти \underline{временной} сущностью (имеется в виду сам \textit{sc-элемент}, а не обозначаемая им сущность), поскольку он появляется в \textit{памяти ostis-системы} и удаляется из неё независимо от того, что он обозначает. В отличие от этого в \textit{SC-пространстве} все sc-элементы считаются постоянными (\underline{постоянно} присутствующими) в рамках этого пространства}
	\end{scnrelfromlist}
\end{SCn}

\begin{SCn}
	\scnheader{Базовая денотационная семантика SC-кода}
	\scnidtf{Онтология Базовой денотационной семантики SC-кода}
	\scnidtf{Формальная \textit{онтология}, представленная в \textit{SC-коде} и являющаяся материнской \textit{онтологией} (онтологией самого высокого уровня) для всех остальных \textit{формальных онтологий}, представленных в \textit{SC-коде}}
	\scnidtf{Онтология SC-пространства}
	\scnidtf{Описание (представление) системы \textit{правил построения семантически корректируемых sc-конструкций}, удовлетворяющих требованиям Базовой денотационной семантики SC-кода}
	\scniselement{sc-онтология}
	\begin{scnindent}
		\scnidtf{формальная онтология, представленная в SC-коде}
	\end{scnindent} 
\end{SCn}

В состав \textit{Базовой денотационной семантики SC-кода} включается:
\begin{textitemize}
	\item Приведенный выше текст \textit{Пункта 2.2.2.1. Семантическая классификация sc-элементов по базовым признакам}
	\item Приведенный выше текст \textit{Пункта 2.2.2.2. Уточнение смысла выделенных классов sc-элементов и связей между этими классами}
	\item Средства базовой семантической спецификации sc-элементов, рассмотренные в  \textit{Пункте 2.2.2.3. Структура базовой семантической спецификации sc-элемента}
\end{textitemize}

\begin{SCn}
	\scnheader{Логическая онтология SC-пространства}
	\scnrelto{логическая онтология}{Базовая денотационная семантика SC-пространства}
	\scntext{примечание}{Приведём пример правил, входящих в состав данной логической онтологии:
		\begin{textitemize}
			\item Вторыми компонентами sc-пар константной парой принадлежности могут быть sc-элементы \underline{любого}типа(в том числе, и sc-переменные), но первыми компонентами таких sc-пар могут быть только \underline{константные} sc-множества
			\item Знак \textit{sc-ситуации} связан с элементами этой ситуации sc-парами константной \underline{постоянной} позитивной принадлежности. То есть позитивная принадлежность считается постоянной в рамках интервала времени существования соответствующей ситуации. В этом смысле ситуацию можно считать квазистатической
			\item Знак атомарной логической формулы связан со всеми элементами этой формулы sc-парами \underline{константной} постоянной позитивной принадлежности, в том числе, и с теми элементами атомарной формулы, которые являются sc-переменными
			\item Из переменного sc-множества могут выходить только переменные sc-пары принадлежности
			\item Не существует sc-пар принадлежности выходящих из обозначений внешних сущностей и sc-пар
			\item и другие
		\end{textitemize}
	}
\end{SCn}

\newpage
\section{Синтаксис SC-кода}
\label{sec_sr_scsyntax}

\begin{SCn}
	\begin{scnrelfromlist}{подраздел}
		\scnitem{\ref{sec_syntactic_sc_code}~\nameref{sec_syntactic_sc_code}}
		\scnitem{\ref{sec_syntactic_core_sc_code}~\nameref{sec_syntactic_core_sc_code}}
		\scnitem{\ref{sec_concept_syntactically_correct_sc_construction}~\nameref{sec_concept_syntactically_correct_sc_construction}}
		\scnitem{\ref{sec_syntactic_extensions_core_sc_code}~\nameref{sec_syntactic_extensions_core_sc_code}}
	\end{scnrelfromlist}
\end{SCn}

\newpage
\subsection{Синтаксические особенности SC-кода}
\label{sec_syntactic_sc_code}
Алфавит SC-кода --- семейство синтаксических меток, приписываемых sc-элементам и указывающих факт принадлежности sc-элемента к соответствующему классу sc-элементов (sc-классу)

\begin{SCn}
	\scnheader{Минимальный алфавит SC-кода}
	\scnexplanation{Если нам известен смысл выделяемых классов sc-элементов (sc-классов), каждый из которых в sc-памяти представлен константным sc-элементом, обозначающим этот sc-класс, то для ''прочтения'' и понимания sc-конструкций, хранимых в sc-памяти, достаточно синтаксически выделить только Класс константных постоянных позитивных sc-пар принадлежности (Класс базовых sc-дуг), с помощью которых каждый sc-элемент будет \underline{явно} соединяться с sc-элементами, обозначающими те sc-классы, которым этот sc-элемент принадлежит. Очевидно, что таким явням способом выделить базовые sc-дуги с помощью самих этих базовых sc-дуг невозможно.}
\end{SCn}

Таким образом, любой класс sc-элементов можно выделить \underline{явно} путём:
\begin{textitemize}
	\item введения sc-элемента, являющегося знаком этого класса sc-элементов (sc-класса);
	\item проведение постоянных позитивных sc-пар принадлежности во все sc-элементы, являющиеся элементами выделяемого sc-класса и хранимые (присутствующие) в текущем состоянии sc-памяти.
\end{textitemize}

\begin{SCn}
	\scnheader{Минимальный алфавит SC-кода}
	\scnnote{Таким образом, Минимальным алфавитом SC-кода является Класс базовых sc-дуг и Класс всех остальных sc-элементов (по умолчанию)}
\end{SCn}

Тем не менее, если учитывать особенности обработки в \textit{sc-памяти} разных классов \textit{sc-элементов}, целесообразно сделать расширение Минимального алфавита SC-кода и, соответственно, ввести понятие \textbf{\textit{Синтаксического Ядра SC-кода}}.

\newpage
\subsection{Синтаксическое Ядро SC-кода}
\label{sec_syntactic_core_sc_code}

Синтаксическая структура линейных информационных конструкций (строк символов) задаётся:
\begin{textitemize}
	\item алфавитом используемых символов (элементарных, атомарных фрагментов информационной конструкции, каковыми, в частности, являются буквы), то есть семейством таких попарно непересекающихся классов синтаксически эквивалентных символов, для которых существует простая процедура, позволяющая для любого символа про его синтаксическим особенностям установить факт его принадлежности одному из указанных классов;
	\item бинарным ориентированным отношением, определяющим непосредственный порядок (последовательность) символов в строках символов.
\end{textitemize}

Аналогично этому, синтаксическая структура sc-конструкций задаётся:
\begin{textitemize}
	\item семейством классов \underline{синтаксически} эквивалентных sc-элементов, в каждый из которых входят sc-элементы с одинаковыми \underline{синтаксическими} характеристиками или, условно говоря, с одинаковыми \underline{синтаксическими} метриками;
	\item двумя \underline{бинарными ориентированными} отношениями инцидентности sc-элементами, заданными на множестве всех sc-элементов:
	\begin{textitemize}
		\item Отношением инцидентности обозначений ы-пар с их компонентами
		\item Отношением инцидентности обозначений \underline{ориентированных} sc-пар с их вторыми компонентами\\
		\scnnote{Данное отношение является подмножеством Отношения инцидентности обозначений sc-пар с их компонентами}
	\end{textitemize}
\end{textitemize}

\begin{SCn}
	\scnheader{Отношение инцидентности обозначений sc-пар с их компонентами*}
	\scnidtftext{часто используемый sc-идентификатор}{пара инцидентности обозначения sc-пары с её компонентом*}
	\begin{scnindent}
		\scniselement{имя нарицательное}
	\end{scnindent}
	\begin{scnrelfromlist}{примечание}
		\scnfileitem{Каждая пара, принадлежащая данному отношению семантически трактуется как \textit{обозначение sc-пары принадлежности}, но синтаксически оформляется не в виде самостоятельного sc-элемента, а в виде бинарной ориентированной связи между sc-элементами, что аналогично бинарным ориентированным связям, описывающим последовательность символов в строке символов. Заметим при этом, что конфигурация sc-конструкций в отличие от строк символов не является линейной. Заметим также, что уточнение семантической интерпретации пар инцидентности sc-элементов полностью определяется семантической типологией первых компонентов этих пар инцидентности, то есть семантической типологией \textit{обозначений sc-пар}, являющихся первым и компонентами рассматриваемых пар инцидентности:
			\begin{scnitemize}
				\item если указанное \textit{обозначение sc-пары} является \textit{sc-константой}, то соответствующая пара инцидентности трактуется как пара константной принадлежности;
				\item если указанное \textit{обозначение sc-пары} является \textit{sc-переменной}, то соответствующая пара инцидентности трактуется как пара переменной принадлежности;
				\item если указанное \textit{обозначение sc-пары} является \textit{обозначением постоянной сущности}, то соответствующая пара инцидентности трактуется как пара постоянной принадлежности;
				\item если указанное \textit{обозначение sc-пары} является \textit{обозначением временной сущности}, то соответствующая пара инцидентности трактуется как пара временной принадлежности
		\end{scnitemize}}
		\scnfileitem{Подчеркнём, что первыми компонентами пар инцидентности sc-элементов всегда являются \textit{обозначения sc-пар}, но вторыми компонентами пар инцидентности sc-элементов могут быть sc-элементы любого типа (в том числе, и обозначения sc-пар)}
	\end{scnrelfromlist}
\end{SCn}

\begin{SCn}
	\scnheader{пара инцидентности}
	\scnexplanation{Каждая sc-пара (константная пара sc-элементов), каждая переменная sc-пара и каждое обозначение sc-пары связывается со своими элементами не явно вводимыми константными или переменными sc-парами позитивной принадлежности, а реализуемыми на ''физическом'' уровне связями (парами) инцидентности. Таким образом пары инцидентности --- это специальным образом синтаксически выделенные константные или переменные sc-пары позитивной принадлежности, связывающие обозначения sc-пар с элементами этих пар. Соответственно этому синтаксические особенности имеют и все обозначения sc-пар, поскольку только из них могут выходить отриентированные пары инцидентности. Поэтому с сентаксической точки зрения обозначения sc-пар будем называть \textbf{\textit{sc-коннекторами}}, обозначения неориентированных sc-пар --- \textbf{\textit{sc-ребрами}}, а обозначения ориентированных sc-пар --- \textbf{\textit{sc-дугами}}. При этом из класса пар инцидентности выделим подкласс пар, связывающих обозначения sc-дуг с теми sc-элементами, вкоторые эти дуги входят. Такую пару инцидентности будем называть \textbf{\textit{парой инцидентности входящей sc-дуги}}.}
\end{SCn}

Какие \textit{семантически выделенные классы sc-элементов} следует также рассматривать, как и \textbf{\textit{семантически выделенные классы sc-элементов}}:
\begin{textitemize}
	\item \textit{обозначение неориентированной sc-пары}\\
	\scnnote{Каждый sc-элемент, принадлежащиц этому классу является \underline{первым} компонентом \underline{двух} пар принадлежности обозначений sc-пар с их компонентами (двух пар, принадлежащих Отношению инцидентности) обозначений sc-пар с их компонентами}
	\item \textit{обозначение ориентированной sc-пары не являющейся знаком костоянной позитивной sc-пары принадлежности}\\
	\scntext{примечание}
	{Каждый sc-элемент, принадлежащий этому классу, является:
		\begin{textitemize}
			\item \underline{первым} компонентом \underline{одной} пары инцидентности обозначения sc-пары с её компонентом;
			\item underline{первым} компонентом \underline{одной} пары инцидентности обозначения ориентированной sc-пары с её вторым компонентом
		\end{textitemize}
	}
	\item \textit{постоянная позитивная sc-пара принадлежности}\\
	\scnnote{Каждый элемент этого класса, как и люое другое обозначение ориентированной sc-пары, является первым компонентом пары инцидентности обозначения sc-пары с её компонентом, а также первым компонентом пары инцидентности обозначения ориентированной sc-пары с её вторым компонентом}
	\item \textit{файл}\\
	\scnidtf{знак файла}
	\scnnote{Для sc-элементов этого класса необходимо на ''синтаксическом'' уровне обеспечить возможность связи этого sc-элемента с обозначаемым им файлом, хранимым в файловой памяти этой же ostis-системы}
	\item класс всех остальных sc-элементов, не попавших в перечисленные выше синтаксически выделенные классы\\
	\scntext{примечание}
	{Данный класс включает в себя:
		\begin{textitemize}
			\item обозначение sc-синглетона;
			\item обозначение sc-связки, не являющейся синглетоном или парой;
			\item обозначение sc-класса;
			\item обозначение sc-структуры;
			\item обозначение внешней сущности, не являющейся файлом.
		\end{textitemize}
	}
\end{textitemize}

Какие классы sc-элементов требуют специального синтаксического оформления:
\begin{textitemize}
	\item неориентированные sc-пары (замена принадлежности на инцидентность)
	\item ориентированные sc-пары (замена принадлежности на инцидентность с дополнительным указанием направленности)
	\item константная постоянная позитивная sc-пара принадлежности
	\item константная временная позитивная sc-пара принадлежности
	\item константный внутренний файл\\
	\scnnote{Для sc-элементов данного класса необходимо обеспечить явную их связь с обозначаемыми ими файлами, хранимыми в памяти той же ostis-системы}
\end{textitemize}

\begin{SCn}
	\scnheader{sc-ребро}
	\scnidtf{Класс sc-элементов, имеющих в рамках Ядра SC-кода синтаксическую метку обозначений неориентированной sc-пары}
	\scnidtf{Синтаксическая метка обозначения неориентированной sc-пары, используемая в рамках Ядра SC-кода}
	\scniselement{синтаксически выделяемый sc-класс в рамках Ядра SC-кода}
\end{SCn}

\begin{SCn}
	\scnheader{sc-дуга общего вида}
\end{SCn}

\begin{SCn}
	\scnheader{базовая sc-дуга}
\end{SCn}

\begin{SCn}
	\scnheader{sc-узел, являющийся знаком файла}
\end{SCn}

\begin{SCn}
	\scnheader{sc-узел, не являющийся знаком файла}
	\scnidtf{sc-узел общего вида}
\end{SCn}

Перечисленное семейство синтаксически выделяемых классов sc-элементов составляет \textit{Алфавит Ядра SC-кода}.

\begin{SCn}
	\scnheader{Алфавит Ядра SC-кода}
	\begin{scneqtoset}
		\scnitem{sc-ребро}
		\scnitem{sc-дуга общего вида}
		\scnitem{базовая sc-дуга}
		\scnitem{sc-узел, являющийся знаком файла}
		\scnitem{sc-узел, не являющийся знаком файла}
		\begin{scnindent}
			\scnidtf{sc-узел общего вида}
		\end{scnindent}
	\end{scneqtoset}
\end{SCn}

Приведём \underline{синтаксическую} классификацию sc-элементов

\begin{SCn}
	\scnheader{sc-элемент}
	\begin{scnsubdividing}
		\scnitem{sc-коннектор}
		\begin{scnindent}
			\begin{scnsubdividing}
				\scnitem{sc-дуга}
				\begin{scnindent}
					\begin{scnsubdividing}
						\scnitem{базовая sc-дуга}
						\scnitem{sc-дуга общего вида}
					\end{scnsubdividing}
				\end{scnindent}	
				\scnitem{sc-ребро}
			\end{scnsubdividing}
		\end{scnindent}	
		\scnitem{sc-узел}
		\begin{scnsubdividing}
			\scnitem{sc-узел, являющийся знаком файла}
			\scnitem{sc-узел, неявляющийся знаком файла}
		\end{scnsubdividing}
	\end{scnsubdividing}
\end{SCn}

\begin{SCn}
	\scnheader{sc-коннектор}
	\scnidtf{sc-элемент, являющийся обозначением sc-пары и имеющий синтаксическую метку обозначения sc-пары}
	\scnidtf{sc-элемент, являющийся обозначением sc-пары и связанный с элементами обозначаемой им sc-пары не sc-парами, обозначающими принадлежность, то есть не с помощью связующих sc-элементов, а с помощью синтаксически реализуемых связей \underline{инцидентности}}
	\scnidtf{sc-элемент, имеющий синтаксическую метку, семантически эквивалентную sc-элементу, который является обозначением класса всех константных и переменных sc-пар (пар sc-элементов)}
\end{SCn}

\begin{SCn}
	\scnheader{sc-дуга}
\end{SCn}

\begin{SCn}
	\scnheader{sc-узел}
\end{SCn}

\begin{SCn}
	\scnheader{синтаксически выделяемый класс sc-элементов}
	\scnidtf{класс sc-элементов, имеющих общий (одинаковый) синтаксический признак, который будем называть синтаксической меткой, каждая из которых семантически эквивалентна (синонимична) соответствующему sc-элементу, обозначающему некоторый класс sc-элементов (sc-класс)}
	\begin{scnindent}
		\scnnote{Наличие у двух разных sc-элементов одной и той же синтаксической метки означает то, что оба эти sc-элемента принадлежат sc-классу, знаком которого является sc-элемент, семантически эквивалентный указанной метке}
	\end{scnindent}
	\scnidtfexp{sc-элемент, обозначающий sc-класс, принадлежность которому может быть представлена либо с помощью sc-пары позитивной принадлежности, либо с помощью соответствующей метки, приписываемой этому sc-элементу}
	\begin{scnindent}
		\scnnote{Приписывание sc-элементам меток ускоряет проверку принадлежности, только такие метки и надо вводить, этих элементов соответствующих классам}
	\end{scnindent}
\end{SCn}

\begin{SCn}
	\scnheader{синтаксически выделяемый sc-класс}
	\scnidtf{sc-кkасс, каждому sc-элементу которого приписывается соответствующая этому sc-классу синтаксическая метка, которая является неявной (синтаксической) формой указания факта принадлежности указанного sc-элемента указанному sc-классу}
	\begin{scnsubdividing}
		\scnitem{синтаксически выделяемый sc-класс в рамках Ядра SC-кода}
		\scnitem{синтаксически выделяемый sc-класс в рамках расширения Ядра SC-кода}
	\end{scnsubdividing}
	\scnnote{Каждый синтаксически выделяемый sc-класс можно считать элементом \textit{Алфавита SC-кода}. Но, в отличие от других языков, синтаксически выделяемые sc-классы могут \underline{пересекаться}}
\end{SCn}

Особенностью привычных нам языков является то, что каждый элементарный (атомарный) фрагмент информационных конструкций может иметь только \uline{одну} метку, то есть может быть элементом только одного синтаксически выделяемого класса элементарных фрагментов. Семейство таких синтаксически выделяемых классов элементарных фрагментов информационных конструкций (например, букв) называют алфавитом соответствующего языка.

\begin{SCn}
	\scnheader{синтаксически выделяемый sc-класс в рамках Ядра SC-кода}
	\scnidtf{синтаксически выделяемый в рамках Ядра SC-кода класс sc-элементов}
	\scnidtf{синтаксическая метка, приписываемая sc-элементам в рамках Ядра SC-кода}
	\scnidtf{синтаксическая метка sc-элементов, выделяющая в рамках Ядра SC-кода соответствующий класс \uline{синтакси\-чески} эквивалентных sc-элементов}
	\scnidtf{класс синтаксически эквивалентных sc-элементов в рамках Ядра SC-кода}
	\scnidtf{синтаксический тип sc-элементов, выделяемый в рамках Ядра SC-кода}
	\scnnote{В различных синтаксических расширениях Ядра SC-кода синтаксически выделяемые sc-классы могут пересекаться. То есть sc-элемент может принадлежать сразу несколькими синтаксически выделяемым sc-классом.}
\end{SCn}

При этом формулирование, семейства синтаксически выделяемых sc-классов (то есть семейства синтаксических меток, приписываемых sc-элементам) может осуществляться путём \uline{синтаксической классификации} sc-элементов по \uline{различным} признакам. Желательно при этом, чтобы такая синтаксическая классификация sc-элементов была согласована с семантической классификацией sc-элементов, которая рассмотрена в \ref{sec_semantic_classification_sc-elements}~\nameref{sec_semantic_classification_sc-elements}. Другими словами каждый синтаксически выделяемый класс sc-элементов (каждая синтаксическая метка) должен иметь чёткую семантическую интерпретацию, то есть должен быть одновременно и семантически выделяемым sc-классом.

\begin{SCn}
	\scnheader{следует отличать*}
	\begin{scnhaselementset}
		\scnitem{синтаксически выделяемый sc-класс в рамках Ядра SC-кода}
		\scnitem{синтаксически выделяемый sc-класс в рамках какого-либо расширения Ядра SC-кода}
		\scnitem{sc-класс}
		\begin{scnindent}
			\scnidtf{Класс sc-элементов, выделяемый (задаваемый) явно с помощью sc-конструкции, состоящей (1) из sc-элемента, являющего \uline{знаком} этого класса и (2) из константных постоянных позитивных sc-пар принадлежности, соединяющих указанный знак выделяемого класса sc-элементов со всеми sc-элементами, принадлежащими этому классу и хранимыми в текущем состоянии sc-памяти}
		\end{scnindent} 
		\scnitem{денотационную семантику каждого sc-элемента, то есть соотношение между sc-элементом и тем, что он обозначает (его денотатом) и соответствующую \uline{семантическую} классификацию всего множества sc-элементов}
		\scnitem{семантический \uline{тип} каждого sc-элемента, то есть синтаксическую метку (значение синтаксического признака-параметра), приписываемую каждому sc-элементу и соответствующую \uline{синтаксическую} классификацию всего множества sc-элементов (Алфавит sc-элементов))}
	\end{scnhaselementset}
\end{SCn}

\newpage
\subsection{Уточнение понятия синтаксически корректной sc-конструкции}
\label{sec_concept_syntactically_correct_sc_construction}

\begin{SCn}
	\scnheader{Синтаксис SC-кода}
	\scnidtf{Онтология синтаксиса SC-кода}
	\scnidtf{Описание правил построения \textit{синтаксически корректных sc-конструкций}}
	\scnidtf{Описание требований, предъявляемых к \textit{синтаксически корректных sc-конструкциями}}
	\scniselement{sc-онтология}
\end{SCn}

\begin{SCn}
	\scnheader{sc-множество}
	\scnidtftext{часто используемый sc-идентификатор}{sc-конструкция}
	\scnidtf{множество \textit{sc-элементов}, которые могут быть (но не обязательно) связаны между собой бинарными ориентированными связями \textit{инцидентности}, каждая из которых связывает некоторый \textit{sc-конструктор} с \textit{sc-элементами}, которые связываются этим \textit{sc-коннектором}}
	\scnidtf{информационная конструкция, каждый элемент (атомарный фрагмент) которой входит в состав некоторого текста, принадлежащего SC-коду, но при этом \textit{конфигурация} всей указанной информационной конструкции не всегда позволяет считать текстом SC-кода, удовлетворяющим целому ряду синтаксических и семантических требований}
	\begin{scnsubdividing}
		\scnitem{синтаксически корректная sc-конструкция}
		\scnitem{синтаксически некорректная sc-конструкция}
	\end{scnsubdividing}
\end{SCn}

\begin{SCn}
	\scnheader{синтаксически корректная sc-конструкция}
	\scnidtf{синтаксически правильно построенная sc-конструкция}
\end{SCn}

\begin{SCn}
	\scnheader{правило построения синтаксически корректных sc-конструкций}
	\scnidtf{синтаксическое правило SC-кода}
	\scnidtf{требование (одно из требований), предъявляемое к \textit{синтаксически корректным sc-конструкциям}}
\end{SCn}

Синтаксис SC-кода
\begin{textitemize}
	\item Каждая sc-пара принадлежности, связывающая sc-элемент, обозначающий пару sc-элементов, с компонентом этой пары (то есть с sc-элементом, связываемым этой sc-парой с другими sc-элементом) синтаксически ''преобразуется'' из \textit{sc-элемента}, обозначающего пару принадлежности в \textit{пару инцидентности}, которая синтаксически уже не является sc-элементом
	\item Поскольку для каждого обозначения sc-пары осуществляется \textit{синтаксическая} замена sc-пар принадлежности их элементов на пары инцидентности этих элементов соответствующие синтаксическое преобразование происходит и с самими обозначениями sc-пар --- они ''превращаются'' в sc-коннекторы. Соответственно этому обозначения неориентированных sc-пар ''преобразуется'' в sc-ребра обозначения ориентированных sc-пар --- в sc-дуги.
\end{textitemize}

\begin{SCn}
	\scnheader{синтаксически некорректная sc-конструкция}
	\scnidtf{\textit{sc-конструкция}, содержащая одну или несколько синтаксических ошибок}
	\scnsuperset{минимальная синтаксически некорректная sc-конструкция}
	\begin{scnindent}
		\scnidtf{\textit{sc-конструкция}, не содержащая подструктуру, являющихся синтаксически некорректными \textit{sc-конструкциями}}
		\scnnote{Каждой \textit{минимальной синтаксически некорректной sc-конструкции} ставится в соответствие одно из синтаксических правил \textit{SC-кода}, которому указанная \textit{sc-конструкция} противоречит}
	\end{scnindent}
	\scnnote{Строго говоря, \textit{синтаксически некорректные sc-конструкции} не являются \textit{sc-текстами}, то есть информационными конструкциями, принадлежащими \textit{SC-коду}}
	\scnrelto{невключение}{sc-текст}
	\begin{scnindent}
		\scnidtf{sc-конструкция принадлежащая SC-коду}
	\end{scnindent} 
\end{SCn}


\newpage
\subsection{Синтаксические расширения Ядра SC-кода}
\label{sec_syntactic_extensions_core_sc_code}

Способ кодирования \textit{sc-конструкций} в различных вариантах реализации \textit{sc-памяти} может быть различным. То есть каждому варианту реализации \textit{sc-памяти} может соответствовать своя синтаксическая модификация \textit{SC-кода}. При этом оно может касаться не только способа представления меток \textit{sc-элементов}, но и (кодирования) \textit{отношений инцидентности sc-элементов}. В любом случае каждая такая модификация должна быть чётко описана.

Зачет нужно расширить Алфавит sc-элементов.
\begin{textitemize}
	\item Для того, чтобы ускорить определение семантического типа каждого sc-элемента в ходе обработки sc-конструкций
	\item Чтобы быстро установить то, чего не хватает \textit{базовой спецификации sc-элемента}\\
	При этом уточнить содержание базовой sc-спецификации sc-элемента (что необходимо о нём знать, чтобы с ним эффективно работать). В частности, необходимо знать то, что о нём не известно.\\
	Но при этом число меток, приписываемых sc-элементу должно быть \uline{минимизировано}, то есть эти метки должны быть \uline{информативными}~(!). Для этого не надо бояться \textit{расширения Алфавита sc-элементов}
\end{textitemize}

Поскольку \textit{SC-код} является языком \uline{внутреннего} представления информации в \textit{sc-памяти} ostis-системы, \textit{Синтаксис SC-кода} является уточнением \uline{формы} такого представления и, как следствие уточнением того, как устроена \textbf{\textit{sc-память ostis-систем}}. Поскольку хранимая в \textit{sc-памяти} информационная конструкция представляет собой множество \textit{sc-элементов}, можно ввести понятие \textbf{\textit{ячейки sc-памяти}}, каждая из которых обеспечивает хранение одного из \textit{sc-элементов}.

\begin{SCn}
	\scnheader{ячейка sc-памяти}
	\scntext{пояснение}
	{фрагмент \textit{sc-памяти}, в котором может храниться один \textit{sc-элемент} и который должен содержать
		\begin{textitemize}
			\item набор синтаксических меток, приписываемых хранимому sc-элементу;
			\item уникальный (взаимнооднозначный) идентификатор хранимого sc-элемента (аналог адреса ячейки в адресной памяти)
			\item связи хранимого sc-элемента со смежными sc-элементами (связи инцидентности)
			\item ссылка на хранимый файл, если хранимый sc-элемент является знаком файла, хранимого в файловой памяти этой же индивидуальной ostis-системы
		\end{textitemize}
	}
\end{SCn}

Формы представления меток sc-элементов могут быть разными:
\begin{textitemize}
	\item приписывание sc-идентификатора того sc-класса которому принадлежит данный sc-элемент
	\item формирование вектора признаков в некотором пространстве признаков (Каждому признаку ставится в соответствии свое \uline{поле}, в которое записывается соответствующее значение признака --- в качестве этого значения тоже можно использовать sc-идентификатор соответствующие значения этого признака)
\end{textitemize}

\newpage
\section{Файлы ostis-системы}
\label{sec_sr_ostisfiles}
\section{Смысловое пространство ostis-систем}
\label{sec_sr_semspace}
\begin{SCn}
	\begin{scnrelfromlist}{ключевое понятие}
		\scnitem{sc-отношение}
		\scnitem{бинарное sc-отношение}
		\scnitem{слотовое sc-отношение}
		\scnitem{sc-структура*}
		\scnitem{элементарно представленный элемент'}
		\scnitem{полносвязно представленный элемент'}
		\scnitem{полностью представленный элемент'}
		\scnitem{sc-связка'}
		\scnitem{sc-отношение'}
		\scnitem{sc-класс'}
		\scnitem{сущностное замыкание*}
		\scnitem{содержательное замыкание*}
		\scnitem{sc-отношение сходства по слотовым отношениям*}
		\scnitem{sc-отношение семантического сходства по слотовым отношениям*}
		\scnitem{связная sc-структура*}
		\scnitem{семантическое сходство sc-структур*}
		\scnitem{семантическое непрерывное сходство sc-структур*}
		\scnitem{ключевой запрос’}
		\scnitem{минимальный ключевой запрос’}
		\scnitem{полная семантическая окрестность элемента*}
		\scnitem{интроспективный ключевой элемент’}
		\scnitem{топологическое пространство}
		\scnitem{топологическое пространство замыкания инцидентности коннекторов}
		\scnitem{топологическое пространство синтаксического замыкания}
		\scnitem{топологическое пространство сущностного замыкания}
		\scnitem{топологическое пространство содержательного замыкания}
		\scnitem{метрика}
		\scnitem{семантическая метрика}
		\scnitem{метрическое пространство}
		\scnitem{метрическое конечное синтаксическое пространство}
		\scnitem{метрическое конечное семантическое пространство}
		\scnitem{псевдометрика}
		\scnitem{псевдометрическое пространство}
		\scnitem{псевдометрическое конечное семантическое пространство}
	\end{scnrelfromlist}
\end{SCn}

Понятие SC-пространства наряду с понятием SC-кода является необходимым для уточнения и формализации понятия смысла информационных конструкций и в унификации смыслового представления информации. В SC-пространстве можно выделять структуры, связанные как с синтаксическими свойствами текстов SC-кода, так и с его семантикой. В последнем случае речь можно вести о «смысловом пространстве». Смысл информационной конструкции, в конечном счёте, определяется (1) внутренними связями всех элементарных фрагментов этой конструкции и (2) её внешними связями с элементами смыслового пространства (её положением в контексте). Смысловое пространство является результатом естественного становления знаний в процессе их интеграции.
Важнейшим достоинством SC-пространства является возможность уточнения таких понятий, как понятие аналогичности (сходства и отличия) различных описываемых «внешних» сущностей, аналогичности унифицированных семантических сетей (текстов SC-кода), понятие семантической близости описываемых сущностей (в том числе, и текстов SC-кода).

Следует отметить, что в силу абстрактности языков модели унифицированного семантического представления знаний и условности выбора меток элементов их текстов, нельзя исключить, что объединение двух произвольных текстов таких языков не будет текстом языка модели унифицированного семантического представления знаний. Чтобы избежать результатов подобных эклектических объединений с точки зрения синтаксиса или семантики, для абстрактных языков следует рассматривать множество «смысловых пространств». Однако, для конкретных (реальных) языков может оказаться достаточным рассмотрение одного «смыслового пространства».
Далее рассмотрим:

\begin{textitemize}
	\item возможность перехода от sc-текстов к графовым структурам и от них к топологическому пространству;
	\item возможность перехода от sc-текстов к графовым структурам и от них к многообразию (топологическому пространству);
	\item возможность перехода от sc-текстов к графовым структурам и от них к метрическому пространству.
\end{textitemize}

Чтобы исследовать структурные свойства SC-пространства, можно использовать уже разработанные модели пространств и связь их известными топологическими моделями например такими как графы. При этом изначально не будем принимать в расчёт динамические особенности, связанные с обработкой знаний, однако позже будет показано, что учёт динамики в процессах обработки и при становлении знаний является необходимым для вычисления семантической метрики, являющейся одним из определяющих признаков знаний.
Обратимся к исследованию структурно-топологических свойств пространства.

Структурно-топологические свойства могут свидетельствовать о синтаксических или семантических зависимостях обозначений текстов языка, позволяющих упростить работу со структурами за счёт перехода к более простым структурам на уровнях управления данными или знаниями в характерных задачах управления для библиотек компонентов.
На множестве элементов, образующих SC-пространство, можно изучать топологические свойства и рассматривать SC-пространство как топологическое пространство. Следует заметить, что, несмотря на то, что SC-код ориентирован на смысловое представление знаний, в силу наличия НЕ-факторов, не все смыслы могут быть представлены в некоторый момент времени и не будет известна структура соответствующего представления. Поэтому структурно-топологические свойства текстов языка представления знаний скорее определяют синтаксическое пространство, нежели семантическое (смысловое). Хотя оба могут приближаться друг к другу по мере устранения неопределённостей, вызванных НЕ-факторами.

Рассмотрим следующие виды топологических пространств:
\begin{textitemize}
	\item топологическое пространство замыкания инцидентности коннекторов;
	\item топологическое пространство синтаксического замыкания;
	\item топологическое пространство сущностного замыкания;
	\item топологическое пространство содержательного замыкания.
\end{textitemize}

\begin{SCn}
	\scnheader{топологическое пространство}
	\scnexplanation{Топологическое пространство -- множество с определённым над ним множеством (семейством) (открытых) подмножеств, включая само множество и пустое множество. Для любого счётного подмножества семейства результат объединения принадлежит семейству, а для любого подмножества семейства результат пересечения также принадлежит семейству. Дополнения множеств семейства до наибольшего из множеств называются замкнутыми множествами.}
\end{SCn}

Чтобы рассмотреть более детально некоторые виды топологических пространств введём следующие понятия.

\begin{SCn}
	\scnheader{sc-отношение}
	\scnexplanation{sc-отношение -- sc-множество sc-связок.}
\end{SCn}

\begin{SCn}
	\scnheader{бинарное sc-отношение}
	\scnexplanation{Бинарное sc-отношение -- sc-множество sc-пар.}
\end{SCn}

\begin{SCn}
	\scnheader{слотовое sc-отношение}
	\scnexplanation{Слотовое sc-отношение -- sc-множество (ориентированных) sc-пар, которые не являются узловыми sc‑парами.}
\end{SCn}

\begin{SCn}
	\scnheader{sc-структура*}
	\scnexplanation{sc-структура* -- sc-множество, в котором есть непустое sc-подмножество-носитель (множество первичных элементов sc-структуры*).}
\end{SCn}

\begin{SCn}
	\scnheader{элементарно представленный элемент’}
	\scnexplanation{Элементарно представленный элемент’ -- элемент sc-структуры*, sc‑множество, все элементы которого являются элементами sc-структуры*.}
\end{SCn}

\begin{SCn}
	\scnheader{полносвязно представленный элемент’}
	\scnexplanation{Полносвязно представленный элемент’ -- элемент sc-структуры*, sc‑множество, все элементы и все принадлежности которому являются элементами sc-структуры*, или sc‑дуга, являющаяся элементарно представленным элементом’ этой sc-структуры*.}
\end{SCn}

\begin{SCn}
	\scnheader{полностью представленный элемент’}
	\scnexplanation{Полностью представленный элемент’ -- полносвязно представленный элемент’ sc‑структуры*, с любым элементом, не являющимся sc-дугой, выходящей из него, связанный принадлежащей этой sc-структуре* sc-дугой принадлежности или sc-дугой непринадлежности.}
\end{SCn}

\begin{SCn}
	\scnheader{sc-связка’}
	\scnexplanation{sc-связка’ -- полносвязно представленный элемент’ sc-структуры*, принадлежащий sc‑отношению’ этой sc-структуры*, являющийся sc-связкой.}
\end{SCn}

\begin{SCn}
	\scnheader{sc-отношение’}
	\scnexplanation{sc-отношение’ -- полносвязно представленный элемент’ sc-структуры*, sc-отношение, все элементы которого являются sc-связками’ этой sc-структуры*.}
\end{SCn}

\begin{SCn}
	\scnheader{sc-класс’}
	\scnexplanation{sc-класс’ -- полносвязно представленный элемент’ sc-структуры*, все элементы которого являются элементами sc‑структуры*, не являющийся ни sc-отношением’, ни sc-связкой’ этой sc‑структуры*.}
\end{SCn}

\begin{SCn}
	\scnheader{сущностное замыкание*}
	\scnexplanation{Сущностное замыкание* -- наименьшая надмножество* (структура*), в котором каждый элемент является элементарно представленным’.}
\end{SCn}
◦ ;

\begin{SCn}
	\scnheader{содержательное замыкание*}
	\scnexplanation{Содержательное замыкание* -- наименьшее надмножество* (структура*), в котором каждый элемент является полносвязно представленным’}
\end{SCn}

\begin{SCn}
	\scnheader{sc-отношение сходства по слотовым отношениям*}
	\scnexplanation{sc-отношение сходства по слотовым sc-отношениям* -- sc-отношение, являющееся рефлексивным по этим слотовым отношениям, т.е. для любого элемента, входящего в связку этого sc-отношения под одним из слотовых sc-отношений, найдётся связка этого sc‑отношения, в которую он входит под каждым из этих слотовых sc-отношений.}
\end{SCn}

\begin{SCn}
	\scnheader{sc-отношение семантического сходства по слотовым отношениям*}
	\scnexplanation{sc-отношение семантического сходства по слотовым отношениям* -- sc-отношение сходства по слотовым sc-отношениям* si и sj, в котором каждый элемент под слотовым sc-отношением si, может быть преобразован к элементу синтаксического типа элемента под слотовым sc-отношением sj; два инцидентных sc-элемента под слотовым sc-отношением si, в рамках этого sc-отношения семантического сходства соответствуют инцидентным элементам соответственно под слотовым sc-отношением sj.}
\end{SCn}

\begin{SCn}
	\scnheader{связная sc-структура*}
	\scnexplanation{Связная sc-структура* -- sc-структура*, являющаяся связной.}
\end{SCn}

\begin{SCn}
	\scnheader{семантическое сходство sc-структур*}
	\scnidtf{семантическое подобие sc-структур*}
		\scnexplanation{Семантическое сходство sc-структур* -- связывает sc-множество sc-структур* с sc‑структурой* sc‑отношением семантического сходства по слотовым sc-отношениям si, sj так, что для каждой sc-структуры* из sc‑множества найдётся её элемент и связка этого sc‑отношения сходства, в которую он входит под слотовым sc‑отношением si, а под слотовым sc‑отношением sj входит элемент sc-структуры*, также для каждого элемента sc‑структуры найдётся связка этого sc-отношения сходства, в которую он входит под слотовым sc‑отношением sj, а под слотовым sc‑отношением si входит элемент sc-структуры* из sc‑множества.}
\end{SCn}

\begin{SCn}
	\scnheader{семантическое непрерывное сходство sc-структур*}
	\scnidtf{семантическое непрерывное подобие sc-структур*}
	\scnexplanation{Семантическое непрерывное сходство sc-структур* -- связывает sc-множество sc‑структур* со связной sc‑структурой* sc‑отношением семантического сходства по слотовым sc-отношениям si, sj так, что для каждой sc-структуры* из sc‑множества найдётся её элемент и связка этого sc‑отношения сходства, в которую он входит под слотовым sc‑отношением si, а под слотовым sc‑отношением sj входит элемент связной sc-структуры*, также для каждого элемента связной sc-структуры найдётся связка этого sc-отношения сходства, в которую он входит под слотовым sc‑отношением sj, а под слотовым sc‑отношением si входит элемент sc-структуры* из sc‑множества.}
\end{SCn}

\begin{SCn}
	\scnheader{ключевой запрос’}
	\scnexplanation{Ключевой запрос’ -- поисковый-проверочный запрос (от одного известного элемента), который выполняется хотя бы от одного элемента и не выполняется хотя бы от одного элемента.}
\end{SCn}

\begin{SCn}
	\scnheader{минимальный ключевой запрос’}
	\scnexplanation{Минимальный ключевой запрос’ -- ключевой запрос, который находит sc‑подмножества множеств элементов, находимых всеми другими ключевыми запросами, которые имеют те же области известных элементов выполнимости и невыполнимости.}
\end{SCn}

\begin{SCn}
	\scnheader{полная семантическая окрестность элемента*}
	\scnexplanation{Полная семантическая окрестность элемента* -- все элементы, находимые выполнимыми минимальными ключевыми запросами от этого элемента (c учётом дизъюнктивного поиска и отрицания поиска).}
\end{SCn}

\begin{SCn}
	\scnheader{интроспективный ключевой элемент’}
	\scnexplanation{Интроспективный (базовый) ключевой элемент’ -- элемент множества (из класса наименьших таких множеств) элементов такого, что любая полная семантическая окрестность любого элемента является sc-подмножеством объединения их полных семантических окрестностей}
\end{SCn}


\begin{SCn}
	\scnheader{топологическое пространство замыкания инцидентности коннекторов}
	\scnexplanation{Топологическое пространство замыкания инцидентности коннекторов на множестве sc-элементов -- топологическое пространство, все замкнутые множества которого содержат все sc-элементы этого множества, до которых есть маршрут по ориентированным связкам отношения инцидентности коннекторов.}
\scncomment{В общем случае не удовлетворяет аксиоме отделимости по Тихонову. Прагматика рассмотрения таких пространств обуславливается операциями удаления sc-элементов и коннекторов, которым они инцидентны. Удаление sc-элемента требует удаления всех коннекторов, замыканию любой открытой окрестности которых он принадлежит.}
\end{SCn}

\begin{SCn}
	\scnheader{топологическое пространство синтаксического замыкания}
	\scnexplanation{Топологическое пространство синтаксического замыкания на множестве sc-элементов -- топологическое пространство, все замкнутые множества которого содержат все sc-элементы этого множества, до которых есть маршрут по ориентированным связкам отношения инцидентности.}
\scncomment{В общем случае не удовлетворяет аксиоме отделимости по Колмогорову. В качестве основы замкнутых множеств топологического пространства можно выделить синтаксическое замыкание, однако в силу возможности проведения дуг из любого sc-узла в любой в итоговом случае (в итоге процесса устранения НЕ-факторов) такое пространство является тривиальным (антидискретным) пространством. Отношение объединения топологических пространств синтаксического замыкания алгебраически не замкнуто на множестве топологических пространств синтаксического замыкания. По той же причине для любого неполного топологического пространства синтаксического замыкания можно рассмотреть топологическое пространство синтаксического замыкания, носитель которого является надмножеством носителя первого и которое не сохраняет замкнутые множества. В этом смысле топология на основе синтаксического замыкания не является устойчивой относительно процессов становления знаний и её рассмотрение прагматически не оправдывается. Топология полного же топологического пространства синтаксического замыкания антидискретна (тривиальна). Таким образом, у полного топологического пространства синтаксического замыкания все топологические подпространства синтаксического замыкания обладают антидискретной (тривиальной) топологией.}
\end{SCn}
\begin{SCn}
	\scnheader{топологическое пространство сущностного замыкания}
	\scnexplanation{Топологическое пространство сущностного замыкания на множестве sc-элементов -- топологическое пространство, все замкнутые множества которого являются сущностными замыканиями.}
	\scncomment{В общем случае не удовлетворяет аксиоме отделимости по Тихонову. В качестве носителя топологического (под)пространства можно выделить сущностное замыкание. Топологическое пространство сущностного замыкания сохраняет замкнутые множества любых топологических пространств сущностного замыкания, носитель которых является подмножеством его носителя и сущностным замыканием. Такие пространства образуют структуру топологических подпространств‑топологических надпространств сущностного замыкания. Топология пространств в этой структуре разнообразна.}
\end{SCn}

\begin{SCn}
	\scnheader{топологическое пространство содержательного замыкания}
	\scnexplanation{Топологическое пространство содержательного замыкания на множестве sc-элементов -- топологическое пространство, все замкнутые множества которого являются содержательными замыканиями.}
	\scncomment{В общем случае не удовлетворяет аксиоме отделимости по Тихонову. В качестве носителя топологического (под)пространства можно выделить содержательное замыкание. Топологическое пространство содержательного замыкания сохраняет замкнутые множества любых топологических пространств содержательного замыкания, носитель которых является подмножеством его носителя и содержательным замыканием. Такие пространства образуют структуру топологических подпространств‑топологических надпространств содержательного замыкания. Топология пространств в этой структуре разнообразна.
	}
\end{SCn}

Возможен переход от sc-структур к многообразиям и топологическим пространствам путём сведения sc-структур к графовым структурам, подробно вопросы сведения sc-структур к графовым структурам и далее -- к многообразиям и топологическим пространствам рассмотрены в работе [].

Для более сложных структур таких, как полная семантическая окрестность, топологические свойства подлежат дальнейшему изучению.

Далее можно рассмотреть метрические пространства, в частности -- конечные подпространства с полностью представленными sc-элементами. 

\begin{SCn}
	\scnheader{метрика}
	\scnexplanation{Метрика -- функция двух аргументов, принимающая значения на (линейно) упорядоченном носителе группы, неотрицательна, равна нейтральному элмененту (нулю) только при равенстве аргументов, симметрична, удовлетворяет неравенству треугольника.}
\end{SCn}

\begin{SCn}
	\scnheader{метрическое пространство}
	\scnexplanation{Метрическое пространство -- множество, с определённой на нём функцией двух аргументов, являющейся метрикой, принимающей значения на упорядоченном носителе группы.}
\end{SCn}

\begin{SCn}
	\scnheader{семантическая метрика}
	\scnidtf{семантическая близость}
	\scnexplanation{Семантическая метрика -- метрика, определённая на знаках и выражающая количественно близость их значений.}
\end{SCn}

\begin{SCn}
	\scnheader{метрическое конечное синтаксическое пространство}
	\scnexplanation{Метрическое конечное синтаксическое пространство SC-кода -- метрическое пространство с конечным носителем, элементами которого являются обозначения (sc-элементы), а значение метрики может быть определено через отношения инцидентности элементов без учёта их семантического типа.}
\end{SCn}

\begin{SCn}
	\scnheader{метрическое конечное семантическое пространство}
	\scnexplanation{Метрическое конечное семантическое пространство SC-кода -- метрическое пространство с конечным носителем, элементами которого являются обозначения (sc-элементы), а значение метрики не может быть определено через отношения инцидентности элементов без учёта их семантического типа.}
\end{SCn}

\begin{SCn}
	\scnheader{псевдометрика}
	\scnexplanation{Псевдометрика -- функция двух аргументов, принимающая значения на (линейно) упорядоченном носителе группы, неотрицательна, симметрична, удовлетворяет неравенству треугольника.}
\end{SCn}

\begin{SCn}
	\scnheader{псевдометрическое пространство}
	\scnexplanation{Псевдометрическое пространство -- множество, с определённой на нём функцией двух аргументов, являющейся псевдометрикой, принимающей значения на упорядоченном носителе группы.}
\end{SCn}

\begin{SCn}
	\scnheader{псевдометрическое конечное семантическое пространство}
	\scnexplanation{Псевдометрическое конечное семантическое пространство SC-кода -- псевдометрическое пространство с конечным носителем, элементами которого являются обозначения (sc-элементы), а значение псевдометрики не может быть определено через отношения инцидентности элементов без учёта их семантического типа.}
\end{SCn}



В силу неполноты выразительных средств для представления изменяющихся со временем знаний, отсутствия определённой пространственно-временной модели, наличия семантически неопределённых или слабоопределённых обозначений в текстах да и наличия недоопределённости самих текстов описанного в предыдущих разделах языка, на данном этапе в этом описании затруднительно предложить какую-либо модель метрического пространства для более сложных структур, учитывающих НЕ‑факторы, связанные с пространством‑временем. Это станет возможным при проявлении желания идти навстречу, готовности к конвергенции, интероперабельности и после достижения консенсуса, достаточного для соответствующего описания развития предлагаемого стандарта и языка.

Тем не менее, некоторые такие модели были успешно предложены в работе []. Предложенные модели полагались на представление, способное выразить семантику переменных обозначений и операционную семантику расширенными средствами алфавита. Для построения подобных моделей, кроме расширенных средств алфавита, предлагается полагаться на модели, описывающие процессы интеграции и становления знаний [], на средства спецификации знаний [], ориентированные на рассмотрение финитных структур, что позволяет перейти к рассмотрению сложных метрических соотношений в рамках метамодели смыслового пространства.

В современных работах в технических науках [bManin], возможно, наиболее близкими понятиями являются понятия, выражающие смысл термина «семантическое пространство» (интериорный подход (Табл. 1)).
Общим во многих подходах к работе с «семантическим пространством» является рассмотрение словоформ или лексем (множеств словоформ) и их признаков (Табл. 1). В литературе [bManin] встречаются следующие подходы (Табл. 2):
\begin{textitemize}
	\item подход на основе семантических осей и пространства признаков (бинарных $\left\lbrace 0,1\right\rbrace ^{n}$, монополярных $\left[0;1\right]^{n}$, биполярных $\left[-1;1\right]^{n}$);
	\item подход на основе семантических осей и нейронного кодирования места в поле смыслов (слова и словосочетания имеют области (подмножества) значений, связываясь другими частями речи как включением и пересечением, тексты соответствуют пути связанных областей, бинарное кодирование групп нейронов, распознающих смыслы);
	\item подход на основе модели «смысл-текст» [bMeaningText] (отражение неполноты семантических шкал и анализ синтагм и поверхностно-синтаксической структуры);
	\item нейролингвистические данные отражает процессы синтеза и восприятия речи в нейронных сетях (сеть лексического синтеза), близка к модели «смысл‑текст»;
	\item модели, построенные на основе статического анализа (корпусов) текстов (модель векторного пространства).
\end{textitemize}
Статистический подход к обработке естественного языка противопоставляется интуиции и коммуникативному опыту учёных [bManin].

В основе подхода лежит семантическая статическая гипотеза, что смысл слов (лексем) определяется контекстом использования (его статистическим образом) в языке (с коммуникативной структурой) [bManin].

Модель векторного пространства семантики [bManin]. Модель рассматривается для двух случаев: большого словаря ($N\leq{M}$) и задачи информационного поиска ($M\leq{N}$). $M$ -- размер словаря, $N$ -- количество контекстов.

На основе статистики строится матрица размерности $M\times{N}$ частот $p_{ij}$ появления лексемы (слова) $w_{i}$ в документе (контексте, подтексты, которые могут перекрываться) $c_{j}$.

В знаменателе -- оценки вероятности слова и контекста соответственно.

В случае невырожденной матрицы $r=N$ каждая такая матрица задаёт точку в грассманиане $N$‑мерных подпространств $M$‑мерного пространства ($N\leq{M}$).

В случае невырожденной матрицы $r=M$ каждая такая матрица задаёт точку в грассманиане $M$‑мерных подпространств $N$‑мерного пространства ($M\leq{N}$).

Каждый текст -- точка в грассманиане [bGrassmanian], соответствующем проективному пространству , относительно одного выделенного контекста. Для всех контекстов получая ориентированную $N$-ку, в соответствии с порядком контекстов в текстах, можно построить маршрут (путь), соединяя геодезическими соседние точки в $N$-ке. Для двух текстов  и  это будут две ломанные, между которыми можно вычислить метрику Фреше [bFrechet], используя метрику Фубини-Штуди [bFubiniStudy] в , для этого следует параметризовать пути  и  через  (,): 
.

Другой способ задать линейный порядок -- это рассмотреть фильтрацию (флаг (флаговое многообразие)) [bFlagManifold] в , заданную расширяющимися контекстами. В итоге для текста получаем точки (флаги) во флаговом многообразии. Для флаговых многообразий тоже можно вычислить метрику Фубини-Штуди [bFubiniStudy].

Этот порядок соответствует временному измерению (процессу коммуникации во времени), что может быть существенным. Другой порядок может быть не зависимым от этого, например алфавитный или порядок в соответствии с законом Ципфа [bZipf], [bZipf2]. 

Сравнение подходов к построению «семантических пространств»

\begin{tabular}{|>{\centering\arraybackslash}m{3cm}|>{\centering\arraybackslash}m{2cm}|>{\centering\arraybackslash}m{2cm}|>{\centering\arraybackslash}m{3cm}|>{\centering\arraybackslash}m{3cm}|>{\centering\arraybackslash}m{3cm}|}
	\hline
& семанти-ческие оси и простран-ства признаков
& семанти-ческие оси и нейронное кодирование признаков
& модель <<смысл‑текст>>
& нейролингвисти-ческое кодирование
& статистическая модель (модель векторного пространства семантики)
\\
	\hline
	определённые семантические оси
	& +
	& +
	& -
	& -
	& -
	\\
	\hline
	динамическая (вычислительная) декомпозиция
	& -
	& +
	& -
	& +
	& -
	\\
	\hlineсеманти-ческие оси и простран-ства признаков
	семанти-ческие оси и нейронное кодирование признаков
	модель смысл‑текст
	нейролингвисти-ческое кодирование
	статистическая модель (модель векторного пространства семантики)
	
	определённые семантические оси
	+
	+
	-
	-
	-
	
	динамическая (вычислительная) декомпозиция
	-
	+
	-
	+
	-
	
	анализ когнитивных процессов (интроспекция)
	-
	+
	-
	+
	-
	
	учёт НЕ‑факторов (неполнота)
	-
	-
	+
	+
	+
анализ когнитивных процессов (интроспекция)
 & -
 & +
 & -
 & +
 & -
 \\
	\hline
учёт НЕ‑факторов (неполнота)
 & -
 & -
 & +
 & +
 & +
 \\
	\hline
\end{tabular}






Вопросы соотнесения смыслов, их формализации, развития языков в пространстве и времени рассмотрены в работах В.В. Мартынова [bMartynov], [bMartynov2], [bGordey].

\newpage
\section*{Заключение к Главе \ref{chapter_sc_code}}
Показано, что SC-код может быть использован в качестве метаязыка для описания \uline{собственной} денотационной семантики и синтаксиса.

Типология sc-конструкций выглядит следующим образом

\begin{SCn}
	\scnheader{sc-множество}
	\scnidtf{sc-конструкция}
	\scnidtf{информационная конструкция, принадлежащая SC-коду}
	\scnsuperset{sc-структура}
	\begin{scnindent}
		\scnsuperset{sc-текст}
		\begin{scnindent}
			\scnidtftext{часто используемый sc-идентификатор}{SC-код}
			\begin{scnindent}
				\scniselement{имя собственное}
			\end{scnindent} 
			\scnidtf{синтаксически целостная и синтаксически корректная (правильно построенная) информационная конструкция SC-кода}
			\scnidtf{Класс (Множество всевозможных) sc-текстов}
			\scnsuperset{sc-знание}
			\begin{scnindent}
				\scnidtf{семантически целостное и семантически корректный sc-текст, являющийся адекватным фрагментом соответствующей предметной области или её спецификации}
			\end{scnindent} 
		\end{scnindent} 
	\end{scnindent} 
\end{SCn}

Перечислим аспекты представления знаний в памяти интеллектуальных компьютерных систем, которые требуют особой аккуратности:
\begin{textitemize}
	\item константный и переменный характер денотационной семантики знаков, хранимых в памяти интеллектуальных компьютерных систем;
	\item динамический характер знаковых конструкций, хранимых в памяти, обусловленный либо выполняемыми в памяти информационными процессами, либо динамическим характером структур внешних объектов, описываемых этими знаковыми конструкциями;
	\item временный характер существования внешних описываемых объектов и временный характер существования различных конфигураций знаковых конструкций и даже самих \uline{знаков, хранимых в памяти};
	\item неизвестность (отсутствие в памяти) востребованной информации различного вида, неполнота знаний, их недостаточность для решения актуальных задач;
	\item синтаксическая и семантическая некорректность, неадекватность и,  в частности, противоречивость некоторых имеющихся знаний;
	\item наличие информационного мусора (излишеств) в имеющихся знаниях.
\end{textitemize}

%\input{author/references}