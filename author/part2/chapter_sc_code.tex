\chapauthor{Ивашенко В.П.\\Голенков В.В.}
\chapter{Смысловое представление знаний в памяти ostis-систем}
\chapauthortoc{Ивашенко В.П.\\Голенков В.В.}
\label{chapter_sc_code}

\abstract{Аннотация к главе.}

%разбиение
%пересечение множеств
\begin{SCn}
	\begin{scnrelfromlist}{ключевое понятие}
		\scnitem{sc-элемент}
%		\scnitem{Структурный признак классификации sc-элементов}
		\scnitem{обозначение sc-множества}
		\scnitem{обозначение sc-связки}
		\scnitem{обозначение sc-синглетона}
		\scnitem{обозначение sc-пары}		
		\scnitem{обозначение неориентированной sc-пары}		
		\scnitem{обозначение ориентированной sc-пары}		
		\scnitem{обозначение sc-пары принадлежности}		
		\scnitem{обозначение sc-пары нечёткой принадлежности}		
		\scnitem{обозначение sc-пары позитивной принадлежности}
		\scnitem{обозначение sc-пары негативной принадлежности}		
		\scnitem{обозначение ориентированной sc-пары, не являющейся sc-парой принадлежности}		
		\scnitem{обозначение sc-связки, не являющейся синглетоном и парой}		
		\scnitem{обозначение sc-класса}		
		\scnitem{обозначение sc-структуры}		
		\scnitem{обозначение внешней сущности}		
		\scnitem{обозначение внутреннего файла}		
		\scnitem{обозначение внешней сущности, не являющейся внутренним файлом}		
		\scnitem{обозначение постоянной сущности}
		\scnitem{обозначение временной сущности}				
		\scnitem{обозначение статической сущности}
		\scnitem{обозначение динамической сущности}				
		\scnitem{sc-константа}
		\scnitem{sc-переменная}				
		\scnitem{sc-множество}
		\scnitem{sc-связка}
		\scnitem{sc-синглетон}
		\scnitem{sc-пара}
		\scnitem{неориентированная sc-пара}
		\scnitem{ориентированная sc-пара}
		\scnitem{sc-пара принадлежности}
		\scnitem{sc-пара нечёткой принадлежности}
		\scnitem{sc-пара позитивной принадлежности}
		\scnitem{sc-пара негативной принадлежности}
		\scnitem{ориентированная sc-пара принадлежности, не являющаяся sc-парой принадлежности}
		\scnitem{sc-связка,не являющаяся синглетоном и парой}
		\scnitem{sc-класс}						
		\scnitem{sc-структура}						
		\scnitem{внешняя сущность}				
		\scnitem{внутренний файл}				
		\scnitem{внешняя сущность, не являющаяся внутренним файлом}				
	\end{scnrelfromlist}
\end{SCn}
%ключевые понятия:

%sc-элемент

\begin{SCn}
	\begin{scnrelfromlist}{подраздел}
		\scnitem{\ref{sec_sr_sccode}~\nameref{sec_sr_sccode}}
		\scnitem{\ref{sec_sr_scdsemantics}~\nameref{sec_sr_scdsemantics}}
		\scnitem{\ref{sec_sr_scsyntax}~\nameref{sec_sr_scsyntax}}
		\scnitem{\ref{sec_sr_ostisfiles}~\nameref{sec_sr_ostisfiles}}
		\scnitem{\ref{sec_sr_semspace}~\nameref{sec_sr_semspace}}
	\end{scnrelfromlist}
\end{SCn}


\section{SC-код (Semantic Computer Code)}
\label{sec_sr_sccode}
Общие положения

Ниже (в \textit{\ref{sec_sr_scdsemantics}~\nameref{sec_sr_scdsemantics}} и в \textit{\ref{sec_sr_scsyntax}~\nameref{sec_sr_scsyntax}}) приведено формальное описание денотационной семантики и синтаксиса SC-кода. Но сделано это будет не совсем обычно. Поскольку все элементы конструкции являются обозначениями описываемых сущностей и, в том числе, обозначениями различных выделяемых классов \textit{sc-элементов}, (!)нам ничего не стоит(!) явно ввести различные семантически значимые и синтаксически выделяемые классы \textit{sc-элементов} и на основе этого явно описать средствами SC-кода базовую денотационную семантику и синтаксис SC-кода. Синтаксис SC-кода задаётся семейством классов синтаксический задаваемых (выделяемых) \textit{sc-элементов}. 

Элементы, принадлежащие каждому синтаксически выделяемому классу \textit{sc-элементов} должны иметь одинаковые синтаксические признаки (синтаксические метки). При этом очевидно, что синтаксис SC-кода существенно упростится, если синтаксически выделяемые классы sс-элементов будут одновременно иметь и чёткую семантическую интерпретацию. Таким образом, формализацию синтаксиса SC-кода целесообразно осуществлять после формализации базовой денотационной семантики SC-кода. Путём синтаксического выделения тех семантически выделенных классов \textit{sc-элементов}, которые, во-первых, необходимы для кодирования sc-конструкций памяти ostis-систем (в sc-памяти) и во-вторых, обеспечивают максимально возможное упрощение обработки sc-конструкций (например, путём упрощения анализа часто проверяемых семантических характеристик обрабатываемых \textit{sc-элементов}).

Особенности SC-кода как одного из возможных вариантов смыслового представления знаний (см.\textit{~\ref{sec_ngics_sense_principles}~\nameref{sec_ngics_sense_principles}}) 

Понятие \textit{sc-элемента} 

Понятие sc-конструкции 

Понятие внутреннего файла ostis-систем 

Понятие sc-идентификатора

\newpage
\section{Базовая денотационная семантика SC-кода}
\label{sec_sr_scdsemantics}

\begin{SCn}
	\begin{scnrelfromlist}{подраздел}
		\scnitem{Пункт 2.2.2.1. Семантическая классификация sc-элементов по базовым признакам}
		\scnitem{Пункт 2.2.2.2. Уточнение смысла выделенных классов sc-элементов и связей между этими классами}
		\scnitem{Пункт 2.2.2.3. Структура базовой семантической спецификации sc-элемента}
		\scnitem{Пункт 2.2.2.4. Онтологическая формализация Базовой денотационной семантики SC-кода}
	\end{scnrelfromlist}
\end{SCn}
	
%\begin{SCn}
%	\scntext{содержание}{
%		Структурная классификация sc-элементов с пояснением смысла понятий, используемых в структурной классификации элементов 
%		
%		Логико-семантическая классификация sc-элементов с пояснением смысла вводимых понятий 
%		
%		Классификация sc-элементов по темпоральным характеристикам обозначаемых ими сущностей с пояснением смысла вводимых понятий 
%		
%		Понятие базовой спецификации sc-элементов 
%		
%		Онтологическая формализация базовой денотационной семантики SC-кода 
%	}

%	В основе базовой денотационной семантики SC-кода лежит:
%	\begin{itemize} 
%		\item семантическая классификация sc-элементов по основным семантически значимым признакам 
%		\item уточнение структуры базовой семантической спецификации элементов
%		\item …
%	\end{itemize}
%\end{SCn}

\newpage
\subsection{Семантическая классификация sc-элементов по базовым признакам}
К числу базовых признаков классификации \textit{sc-элементов} относятся:

\begin{textitemize}
	\item структурный признак;
	\item логико-семантический признак;
	\item темпоральная характеристика сущностей, обозначаемых \textit{sc-элементами}, которая в свою очередь, включает в себя:
	\begin{textitemize}
		\item постоянство или временность существования обозначаемой сущности;
		\item статичность (стационарность) или динамичность (изменчивость) обозначаемой сущности.
	\end{textitemize}
\end{textitemize}

\begin{SCn}
	\scnheader{Структурная классификация sc-элементов}
	\scnstartstruct
	
	\scnheader{sc-элемент}
	\scnrelfrom{разбиение}{Структурный признак классификации sc-элементов}
	\begin{scnindent}
		\begin{scneqtoset}
			\scnitem{обозначение sc-множества}
			\begin{scnindent}
			\begin{scnsubdividing}
				\scnitem{обозначение sc-связки}
				\begin{scnindent}
				\begin{scnsubdividing}
					\scnitem{обозначение sc-синглетона}
					\scnitem{обозначение sc-пары}
					\begin{scnindent}
					\begin{scnsubdividing}
						\scnitem{обозначение неориентированной sc-пары}
						\scnitem{обозначение ориентированной sc-пары}
						\begin{scnindent}
							\begin{scnsubdividing}
								\scnitem{обозначение sc-пары принадлежности}
								\begin{scnindent}
								\begin{scnsubdividing}
									\scnitem{обозначение sc-пары нечеткой принадлежности}
									\scnitem{обозначение sc-пары позитивной принадлежности}
									\scnitem{обозначение sc-пары негативной принадлежности}
								\end{scnsubdividing}
								\end{scnindent}
								\scnitem{обозначение ориентированной sc-пары, не являющейся парой принадлежности}
							\end{scnsubdividing}
						\end{scnindent}
					\end{scnsubdividing}
					\end{scnindent}
					\scnitem{обозначение sc-связки, не являющейся синглетоном и парой}
				\end{scnsubdividing}
				\end{scnindent}
				\scnitem{обозначение sc-класса}
				\scnitem{обозначение sc-структуры}
			\end{scnsubdividing}
			\end{scnindent}
			\scnitem{обозначение внешней сущности}
			\begin{scnindent}
			\begin{scnsubdividing}
					\scnitem{внутренний файл}
					\scnitem{внешняя сущность, не являющаяся внутренним файлом}
					\scnitem{обозначение файла}
					\scnitem{обозначение информационной конструкции, не являющейся ни sc-множеством, ни файлом}
					\scnitem{обозначение внешней сущности, не являющейся информационной конструкцией}
			\end{scnsubdividing}
			\end{scnindent}
		\end{scneqtoset}
	\end{scnindent} 

	\scnendstruct \scnsourcecommentinline{Завершили представление \textit{Структурной классификации sc-элементов}}
\end{SCn}

\begin{comment}
Поясним смысл понятий, структурной
классификации sc-элементов 


\begin{SCn}
	\scnheader{sc-элемент}
	\scnidtf{sc-элемент, обозначающий множество}
	\scnidtf{sc-обозначение множества}
	\scnidtf{множество, представимое в SC-коде}
	\scnidtf{множество} 
	
	\bigskip
	
	\scnsourcecommentinline{так как любое множество можно
		представить в виде sc-множества}
	
	\begin{scnrelfromlist}{примечание}
		\scnfileitem{Каждый sc-элемент является обозначением
			соответствующего множества}
		\scnfileitem{Строго говоря, не каждое множество может быть
			обозначено соответствующим sc-элементом. 
			К таким множествам относятся либо множества,
			элементами которых являются sc-элементы
			(sc-множества), либо синглетоны элементами
			которых являются сущности, не являющиеся
			элементами (синглетоны внешних сущностей). Но
			каждое множество, не являющийся sc-множеством
			или синглетоном указанного вида, может быть
			однозначно преобразовано в sc-множество и
			описано средствами SC-кода. При этом
			теоретико-множественные свойства
			"нестандартных" для SC-кода множеств совпадают
			со свойствами соответствующих им
			"стандартных" для SC-кода множеств.}
	\end{scnrelfromlist}
\end{SCn}

\begin{SCn}
	\scnheader{sc-элемент}
	\scnidtf{обозначение множества}
	\scnnote{Тот факт, что каждый sc-элемент является
		обозначением соответствующего множества
		(частым случаем которых являются синглетоны
		внешних описываемых сущностей), означает то,
		что базовым видом объектов, которыми
		оперирует SC-код на синтаксическом,
		семантическом и логическом уровне являются
		множества знаков, обозначающих различные
		множества. 
		В этом смысле SC-код имеет базовую
		теоретико-множественную основу.}
\end{SCn}

\begin{SCn}
	\scnheader{sc-элемент}
	\begin{scnrelfromset}{разбиение}
		\scnitem{обозначение sc-множества}
		\begin{scnindent}
			\scnidtf{обозначение множества sc-элементов}
			\scnidtf{обозначение множества, все элементы
				которого являются sc-элементами}
		\end{scnindent}
		\scnitem{обозначение внешней сущности}
		\begin{scnindent}
			\scnidtf{обозначение синглетона внешней сущности}
			\scnidtf{терминальный sc-элемент}
			\scnidtf{синглетон внешней сущности}
		\end{scnindent}
	\end{scnrelfromset}	
\end{SCn}

\begin{SCn}
	\scnheader{sc-множество}
	\scnidtf{sc-элемент, обозначающий множество
		sc-элементов}
	\scnidtf{sc-обозначение множества sc-элементов}
	\scnidtf{множество sc-элементов}
	\scnidtf{множество, каждый элемент которого
		является sc-элементом}
	\scnsubset{sc-элемент}
	\begin{scnindent}
		\scnidtf{множество, представимое в SC-коде}
		
		\scneq{\normalfont{(}sc-множество $\cup$ синглетон внешней сущности\normalfont{)}}
		
		\scnidtf{информационная конструкция SC-кода}
		\scnidtftext{часто используемый sc-идентификатор}
		{sc-конструкция}
		
	\end{scnindent}
\end{SCn}

\begin{SCn}
	\scnheader{sc-связка}
	\scnidtf{sc-элемент, обозначающий связку sc-элементов}
	\scnidtf{sc-обозначение связки sc-элементов}
	\scnidtf{обозначение sc-связки}
\end{SCn}

\begin{SCn}
	\scnheader{sc-связка}
	\scnidtf{обозначение связи между sc-элементами}
	\scnsuperset{отображение связи между сущностями, которые
		обозначаются sc-элементами, связанными sc-связкой}
\end{SCn}

\begin{SCn}
	\scnheader{sc-синглетон}
	\scnidtf{sc-множество, являющееся синглетоном}
	\scnidtf{одномощное sс-множество}
	\scnidtf{sc-множество, имеющее мощность, равную
		единице}
	\scnidtf{sc-элемент, обозначающий унарную sc-связку}
	\scnidtf{sc-обозначение унарной sc-связки}
	\scnidtf{унарная sc-связка}
	\scnidtf{обозначение sc-синглетона}
	\scnidtf{обозначение одномощного множества,
		единственный элемент которого является}
	sc-элементом]
\end{SCn}

\begin{SCn}
	\scnheader{sc-пара}
\end{SCn}

\begin{SCn}
	\scnheader{неориентированная sc-пара}
\end{SCn}

\begin{SCn}
	\scnheader{ориентированная sc-пара}
\end{SCn}

\begin{SCn}
	\scnheader{sc-пара принадлежности}
\end{SCn}

\begin{SCn}
	\scnheader{sc-пара нечёткой принадлежности}
\end{SCn}

\begin{SCn}
	\scnheader{sc-пара позитивной принадлежности}
\end{SCn}

\begin{SCn}
	\scnheader{sc-пара константной постоянной позитивной
		принадлежности}
	\scnidtf{константная позитивная постоянная sc-пара
		принадлежности}
\end{SCn}

\begin{SCn}
	\scnheader{sc-пара константной временной позитивной
		принадлежности}
\end{SCn}

\begin{SCn}
	\scnheader{sc-пара негативной принадлежности}
\end{SCn}

\begin{SCn}
	\scnheader{sc-пара, не являющаяся парой принадлежности}
\end{SCn}

\begin{SCn}
	\scnheader{sc-связка, не являющаяся синглетоном и парой}
\end{SCn}

\begin{SCn}
	\scnheader{sc-класс}
	\scnnote{Требованиями, предъявляемыми к каждому
		sc-классу являются:
		\begin{itemize}
			\item бесконечность этого sc-множества 
			\item наличие общего свойства, присущего всем
			элементам этого sc-множества, в частности,
			наличие его определения
	\end{itemize} }
\end{SCn}

\begin{SCn}
	\scnheader{sc-класс}
	\scnsuperset{sc-отношение}
	
	\begin{scnindent}
		\scnidtf{sc-класс sc-связок}
		\scnsuperset{бинарное sc-отношение}
		
		\begin{scnindent}
			\begin{scnrelfromset}{разбиение}
				\scnitem{бинарное неориентированное sc-отношение}
				\scnitem{бинарное ориентированное sc-отношение}
				\begin{scnindent}
					\scnsuperset{ролевое sc-отношение}
				\end{scnindent}		
			\end{scnrelfromset}	
		\end{scnindent}
	\end{scnindent}
	
	\scnsuperset{sc-класс sc-классов}
	
	\scnsuperset{sc-параметр}
	
	
	\scnsuperset{sc-класс sc-структур}
	
	
	\scnsuperset{sc-класс эквивалентности}
	\begin{scnindent}
		\scnidtf{фактор-множество соответствующего
			отношения эквивалентности}
	\end{scnindent}
	
	\scnsuperset{sc-класс внешних сущностей}
	
	\scnsuperset{sc-класс внутренних файлов}
	
	
	\begin{scnrelfromset}{следует отличать}
		\scnitem{sc-класс}
		\scnitem{sc-связка}
	\end{scnrelfromset}	
	
	
	\scntext{сравнение}
	{В отличие от sc-связки принципом формирования
		является наличие общего свойства, присущего
		всем элементам этого sc-класса и только им (или
		присущего всем сущностям, которые
		обозначаются указанными sc-элементами). Таким
		общим свойством может быть определение
		sc-класса либо принадлежность одному из
		значений некоторого параметра, то есть одному
		из элементов фактор-множества,
		соответствующего некоторому отношению
		эквивалентности или толерантности}
	
	\scntext{сравнение}{Примерами sc-классов являются:
		\begin{itemize}
			\item конкретная окружность (множество всех
			точек, равноудалённых от некоторой заданной
			точки)
			\item конкретный отрезок (множество всех точек,
			лежащих между двумя заданными точками, с
			включением этих точек)
			\item конкретный линейный треугольник (множество
			всех точек, лежащих между каждыми двумя из
			трёх заданных точек, с включением этих точек)
		\end{itemize} 
		и, соответственно этому, примерами sc-связок
		являются: 
		\begin{itemize}
			\item пары граничных точек различных отрезков
			\item тройки вершин различных треугольников
		\end{itemize} 
	}
	\begin{scnrelfromset}{следует отличать}
		\scnitem{sc-связка попарно эквивалентных сущностей}
		\scnitem{sc-класс эквивалентности}
		
		\begin{scnindent}
			\scntext{пояснение}{В sc-класс входит не просто некоторые количество попарно эквивалентных между собой сущностей, а абсолютно все такие сущности}
		\end{scnindent}		
	\end{scnrelfromset}	
	
	Приведём пример: 
	\begin{scnrelfromset}{следует отличать}
		\scnitem{множество всех треугольников, подобных
			одному из них}
		\begin{scnindent}
			\scnsubset{sc-класс}
		\end{scnindent}
		\scnitem{конечное множество подобных треугольников}
		\begin{scnindent}
			\scnsubset{sc-связка попарно эквивалентных треугольников}		
		\end{scnindent}
	\end{scnrelfromset}	
\end{SCn}

\begin{SCn}
	\scnheader{sc-структура}
	
	\begin{scnrelfromset}{следует отличать}
		\scnitem{sc-структура}
		\scnitem{sc-связка}
	\end{scnrelfromset}	
	
	\scntext{сравнение}{В отличие от sc-связок в каждую sc-структуру
		должна входить по крайней мере одна sc-связка
		вместе с компонентами этой sc-связки}
\end{SCn}

\begin{SCn}
	\scnheader{внешняя сущность}
	\scnidtf{синглетон внешней сущности}
	\scnidtf{обозначение синглетона внешней сущности}
	\scnidtf{sc-элемент, обозначающий синглетон,
		элементом которого является некоторая
		внешняя описываемое сущность}
	\scnidtf{множество, обозначаемое sc-элементом,
		являющиеся одномощным множеством,
		единственным элементом которого является
		сущность, внешняя по отношению к
		sc-конструкции, то есть сущность, не являющиеся
		sc-элементом}
	\scnnote{обозначение синглетона внешний сущности, то
		есть sc-элемент, обозначающий этот синглетон,
		можно также трактовать как sc-элемент,
		обозначающий соответствующую внешнюю
		описываемую сущность, которую, в свою очередь,
		можно считать денотатом указанного
		sc-элемента}
	
	\scnnote{Очевидно, что пара принадлежности,
		связывающая sc-элемент, обозначающий
		синглетон внешней сущности, не может быть
		непосредственно представлена в виде
		соответствующей sc-дуги принадлежности, так
		как второй компонент этой sc-дуги не находится
		в sc-памяти}
	
	\begin{scnrelfromset}{следует отличать}
		\scnitem{синглетон внешней сущности}
		\scnitem{sc-синглетон}
		\begin{scnindent}	
			\scnidtf{sc-синглетон, единственным элементом
				которого является некоторый sc-элемент}
			
			\scnsubset{sc-множество}
			\begin{scnindent}
				\scnidtf{sc-элемент, обозначающий множество,
					элементами которого являются только
					sc-элементы}
				\scnidtf{множество sc-элементов}
			\end{scnindent}
		\end{scnindent}
	\end{scnrelfromset}	
\end{SCn}

\begin{SCn}
	\scnheader{внутренний файл}
	\scnidtf{внутренний файл ostis-системы}
	\scnidtf{внутренний образ (копия), внешней
		информационной конструкции, хранимый в
		файловой памяти ostis-системы}
	\scnnote{Файловая память ostis-системы, хранящая
		различного рода информационные конструкции
		(образы, модели), не являющиеся
		sc-конструкциями, должна быть тесно связана с
		sc-памятью этой же ostis-системы. Как минимум
		каждый файл ostis-системы должен быть связан с
		тем sc-узлом, который является знаком этого
		файла (точнее, знаком синглетона, элементом
		которого является указанный файл)}
\end{SCn}

\begin{SCn}
	\scnheader{внешняя сущность, не являющаяся внутренним
		файлом}
\end{SCn}
\end{comment}

\vskip 5cm

\begin{SCn}
	\scnheader{Логико-семантическая классификация sc-элементов}
	\scnstartstruct
	
	\scnheader{sc-элемент}
	\begin{scnsubdividing}
		\scnitem{sc-константа}
		\begin{scnindent}
			\scnidtf{sc-элемент, логико-семантическим значением которого является он сам}
		\end{scnindent}
		\scnitem{sc-переменная}
		\begin{scnindent}
			\begin{scnsubdividing}
				\scnitem{sc-переменная 1-го уровня}
				\begin{scnindent}
					\scnidtf{sc-элемент, областью возможных значений которого является множество sc-констант}
				\end{scnindent}
				\scnitem{sc-переменная 2-го уровня}
				\begin{scnindent}
					\scnidtf{sc-элемент, возможными значениями которого являются переменные 1-го уровня}
				\end{scnindent}
				\scnnote{такие переменные (метапеременные) необходимы для описания логических языков}
				\scnitem{sc-переменная универсального типа}
				\begin{scnindent}
					\scnidtf{sc-переменная, на значения которой не накладывается никаких ограничений}
				\end{scnindent}
			\end{scnsubdividing}
		\end{scnindent}
	\end{scnsubdividing}	
	
	\scnendstruct \scnsourcecommentinline{Завершили представление \textit{Логико-семантической классификации sc-элементов}}
\end{SCn}

\begin{SCn}
	\scnheader{Классификация sc-элементов по темпоральным характеристикам обозначаемых ими сущностей}
	\scnstartstruct
	
	\scnheader{sc-элемент}
	\scnrelfrom{разбиение}{Признак постоянства существования сущностей, обозначаемых sc-элементами}
	\begin{scnindent}
		\begin{scneqtoset}
			\scnitem{обозначение постоянной сущности}
			\begin{scnindent}
				\scnidtf{обозначение постоянно существующей сущности}
			\end{scnindent}
			\scnitem{обозначение временной сущности}
			\begin{scnindent}
				\begin{scnsubdividing}
					\scnitem{обозначение внешней временной сущности}
					\begin{scnindent}
						\scnsuperset{обозначение внешней ситуации}
						\scnsuperset{обозначение внешнего события}
						\scnsuperset{обозначение внешнего процесса}
					\end{scnindent}
					\scnitem{обозначение внутренней временной сущности в sc-памяти}
					\begin{scnindent}
						\begin{scnsubdividing}
							\scnitem{обозначение ситуации в sc-памяти}
							\begin{scnindent}
								\scnidtf{обозначение ситуации, которая возникла или возникает в процессе обработки информации в sc-памяти}
							\end{scnindent}
							\scnitem{обозначение события в sc-памяти}
							\begin{scnindent}
								\scnidtf{обозначение события, которое произошло или произойдет в процессе обработки информации в sc-памяти}
							\end{scnindent}
							\scnitem{обозначение информационного процесса в sc-памяти}
							\begin{scnindent}
								\scnidtf{обозначение внутреннего процесса в sc-памяти, который происходит, произошёл или будет происходить}
							\end{scnindent}
						\end{scnsubdividing}
					\end{scnindent}
				\end{scnsubdividing}
			\end{scnindent}
		\end{scneqtoset}
	\end{scnindent} 
	
	\scnrelfrom{разбиение}{Признак статичности сущностей, обозначаемых sc-элементами}
	\begin{scnindent}
		\begin{scneqtoset}
			\scnitem{обозначение статической сущности}
			\begin{scnindent}
				\scnidtf{обозначение статичной сущности}
				\scnidtf{обозначение стационарной сущности}
				\scnidtf{обозначение сущности, изменения которой в рамках соответствующего отрезка времени считаются несущественными}
				\scnsuperset{обозначение статического sc-множества}
			\end{scnindent}
			\scnitem{обозначение динамической сущности}
			\begin{scnindent}
				\scnidtf{обозначение сущности изменяющейся во времени}
				\scnsuperset{обозначение динамического sc-множетсва}
			\end{scnindent}
		\end{scneqtoset}
	\end{scnindent} 
	
	\scnendstruct \scnsourcecommentinline{Завершили представление \textit{Классификации sc-элементов по темпоральным характеристикам обозначаемых ими сущностей}}
\end{SCn}

Когда речь идёт о темпоральных свойствах sc-элементов, следует чётко отличать:
\begin{textitemize}
	\item временный характер присутствия любого sc-элемента в составе той базы знаний (в той sc-памяти) ostis-системы, в которой он находится (когда-то он появляется, когда-то может быть удалён из sc-памяти;
	\item временный характер присутствия в sc-памяти всей заданной sc-конструкции (заданного множества sc-элементов) –- такую sc-конструкцию будем называть ситуацией в sc-памяти;
	\item временный характер существования внешней сущности, которую sc-элемент обозначает;
	\item статичный или динамичный (изменчивый) характер внешней сущности, обозначаемой sc-элементом динамический характер внешней сущности, предполагает наличие в sc-памяти описания процесса изменения состояния или конфигурации указанной внешней сущности;
	\item динамическое sc-множество (динамическую sc-конструкцию), являющееся отражением (динамической модели) соответствующего внешнего процесса (процесса, происходящего во внешней среде);
	\item динамическое sc-множество (динамическую sc-конструкцию), являющуюся отражением (динамической моделью) соответствующего внутреннего процесса (информационного процесса, происходящего в той же sc-памяти, в которой находится sc-элемент, обозначающий указанное динамическое sc-множество)
\end{textitemize}

\begin{SCn}
	\scnheader{Структурная классификация sc-констант}
	\scnnote{Данная классификация полностью аналогична Структурной классификации sc-элементов, в отличие от которой она описывает структурную классификацию только константных sc-элементов (sc-констант)}
	\scniselement{sc-структура}
	\scnrelboth{аналог}{Структурная классификация sc-элементов}
\end{SCn}
	
\begin{SCn}
	\scnheader{Структурная классификация sc-констант}
	\scnstartstruct
	
	\scnheader{sc-константа}
	\begin{scnsubdividing}
		\scnitem{sc-множество}
		\begin{scnindent}
			\begin{scnsubdividing}
				\scnitem{sc-связка}
				\begin{scnindent}
					\begin{scnsubdividing}
						\scnitem{sc-синглетон}
						\scnitem{sc-пара}
						\begin{scnindent}
							\begin{scnsubdividing}
								\scnitem{неориентированная sc-пара}
								\scnitem{ориентированная sc-пара}
								\begin{scnindent}
									\begin{scnsubdividing}
										\scnitem{sc-пара принадлежности}
										\begin{scnindent}
											\begin{scnsubdividing}
												\scnitem{sc-пара нечёткой принадлежности}
												\scnitem{sc-пара позитивной принадлежности}
												\begin{scnindent}
												\scnsuperset{sc-пара постоянной позитивной принадлежности}
												\begin{scnindent}
													\begin{scnreltoset}{пересечение множеств}
														\scnitem{sc-константа}
														\scnitem{постоянная сущность}
														\scnitem{статическая сущность}
														\scnitem{sc-пара позитивной принадлежности}	
														\begin{scnindent}
															\scnsuperset{sc-пара постоянной позитивной принадлежности}
															\begin{scnindent}
																\begin{scnreltoset}{пересечение множеств}
																	\scnitem{sc-константа}
																	\scnitem{постоянная сущность}
																	\scnitem{статическая сущность}
																	\scnitem{sc-пара позитивной принадлежности}	
																\end{scnreltoset}
															\end{scnindent}
															\scnsuperset{sc-пара временной позитивной принадлежности}
															\begin{scnindent}
																\begin{scnreltoset}{пересечение множеств}
																	\scnitem{sc-константа}
																	\scnitem{временная сущность}
																	\scnitem{динамическая сущность}
																	\scnitem{sc-пара позитивной принадлежности}	
																\end{scnreltoset}
															\end{scnindent}
														\end{scnindent}
													\end{scnreltoset}
												\end{scnindent}
												\scnsuperset{sc-пара временной позитивной принадлежности}
												\begin{scnindent}
													\begin{scnreltoset}{пересечение множеств}
														\scnitem{sc-константа}
														\scnitem{временная сущность}
														\scnitem{динамическая сущность}
														\scnitem{sc-пара позитивной принадлежности}	
													\end{scnreltoset}
												\end{scnindent}
												\end{scnindent}
												\scnitem{sc-пара негативной принадлежности}
											\end{scnsubdividing}
										\end{scnindent}
										\scnitem{ориентированнная sc-пара, не являющаяся sc-парой принадлежности}	
									\end{scnsubdividing}
								\end{scnindent}
							\end{scnsubdividing}
						\end{scnindent}
						\scnitem{sc-связка, не являющаяся синглетоном и парой}
					\end{scnsubdividing}
				\end{scnindent}
				\scnitem{sc-класс}
				\scnitem{sc-структура}
			\end{scnsubdividing}
		\end{scnindent}
		\scnitem{внешняя сущность}
		\begin{scnindent}
			\scnidtf{sc-элемент, являющийся знаком внешней сущности}
			\scnidtf{знак внешней сущности}
			\scnidtf{знак сущности, не являющейся sc-множеством (sc-конструкцией)}
		\end{scnindent} 
		\begin{scnindent}
			\begin{scnsubdividing}
				\scnitem{файл}
				\scnitem{внутренний файл}
				\scnitem{внешняя сущность, не являющаяся внутренним файлом}
				\scnitem{внешняя сущность, не являющаяся информационной конструкцией}
				\scnitem{информационная конструкция, не являющаяся ни sc-множеством, ни файлом}
			\end{scnsubdividing}
		\end{scnindent}
	\end{scnsubdividing}
	
	\scnendstruct\scnsourcecommentinline{Завершили представление \textit{Структурной классификация sc-констант}}
\end{SCn}

\newpage
\subsection{Уточнение смысла выделенных классов sc-элементов и связей между этими классами}
Перейдём к детальному рассмотрению смысла классов \textit{sc-элементов} (sc-классов), введенных в представленных выше классификациях.

Указанные sc-классы рассматриваются в порядке их введения в представленных выше классификациях \textit{sc-элементов}.

Сначала поясним смысл понятий (sc-классов), введенных в Структурной классификации \textit{sc-элементов} и в Структурной классификации sc-констант.

\begin{SCn}
	\scnheader{sc-элемент}
	\scnidtf{обозначение множества}
	\scnidtf{sc-обозначение множества, представимого в SC-коде}
	\begin{scnsubdividing}
		\scnitem{sc-множество}
		\begin{scnindent}
			\scnidtf{обозначение множества \textit{sc-элементов}}
			\scnidtf{обозначение множества, все элементы которого являются \textit{sc-элементами}}
			\scnidtf{обозначение внутренней для sc-памяти сущности, то есть сущности, хранимой в sc-памяти}
		\end{scnindent}	
		\scnitem{обозначение внешней сущности}
		\begin{scnindent}
			\scnidtf{обозначение синглетона внешней сущности}
			\scnidtf{терминальный \textit{sc-элемент}}
		\end{scnindent}	
	\end{scnsubdividing}
	\begin{scnrelfromlist}{примечание}
			\scnfileitem{Каждый \textit{sc-элемент} является обозначением соответствующего множества.}
			\scnfileitem{Ко множествам, представимым в SC-коде, относятся либо множества, элементами которых являются \textit{sc-элементы} (sc-множества), либо синглетоны, элементами которых являются сущности, не являющиеся \textit{sc-элементами} (синглетоны внешних сущностей). Таким образом, строго говоря, не каждое множество может быть обозначено соответствующим \textit{sc-элементом}. Но каждое множество, не являющееся sc-множеством или синглетоном указанного выше вида может быть однозначно преобразовано в sc-множество и описано средствами \textit{SC-кода}. При этом теоретико-множественные свойства "нестандартных" для \textit{SC-кода} множеств совпадают со свойствами соответствующих им "стандартных" для SC-кода множеств.}
			\scnfileitem{Тот факт, что \underline{каждый} \textit{sc-элемент} является обозначением соответствующего множества (частным случае которых являются синглетоны \underline{внешних} описываемых сущностей), означает то, что базовым видом объектов, которыми оперирует \textit{SC-код} на синтаксической, семантическом и логическом уровне являются множества знаков, обозначающих различные множества. В этом смысле \textit{SC-код} имеет базовую теоретико-множественную основу.}
	\end{scnrelfromlist}

	\scnrelfrom{правила построения внешних идентификаторов sc-элементов заданного класса}{Общие правила построения внешних идентификаторов sc-элементов}
	\begin{scnindent}
		\scnidtf{Общие правила идентификации sc-элементов}
		\begin{scneqtoset}
			\scnfileitem{Принадлежность идентифицируемого \textit{sc-элемента} некоторым классам \textit{sc-элементов} (sc-классам) явно указывается во внешнем идентификаторе этого \textit{sc-элемента} (в SC-идентификаторе) с помощью соответствующих условных признаков:
			\begin{scnitemize}
				\item если первым символом sc-идентификатора является знак подчеркивания, то идентифицируемый \textit{sc-элемент} принадлежит Классу sc-переменных. По умолчанию считается, что идентифицируемый \textit{sc-элемент} принадлежит Классу sc-констант;
				\item если последним символом sc-идентификатора является символ ''звёздочка'', то идентифицируемый \textit{sc-элемент} принадлежит Классу обозначений неролевых отношений;
				\item если последним символом sc-идентификатора является апостроф, то идентифицируемый \textit{sc-элемент} принадлежит Классу обозначений ролевых отношений, каждое из которых является подмножеством Отношения принадлежности, то есть Класса всех константных позитивных пар принадлежности;
				\item если последним символом sc-идентификатора является символ ''\textasciicircum'', то идентифицируемый \textit{sc-элемент} принадлежит Классу обозначений параметров.
			\end{scnitemize}} 
			\scnfileitem{Слово ''обозначение'' в sc-идентификаторе означает то, что обозначаемая сущность может быть как константной, так и переменной.}
			\scnfileitem{В sc-идентификаторах можно делать следующие сокращения:
			\begin{scnitemize}
				\item ''sc-элемент, обозначающий \ldots '' -- ''обозначение''
				\item ''обозначение константного'' -- ''знак константного''
				\item ''знак константного'' -- ''константный''
				\item слово "константный" в sc-идентификаторах можно опускать, так как константность подразумевается по умолчанию
			\end{scnitemize}}
			\scnfileitem{Для каждого sc-элемента можно построить sc-идентификатор, являющийся именем собственным, которое всегда начинается с большой буквы (заглавной) буквы.}
			\scnfileitem{Если sc-элемент является обозначением некоторого класса sc-элементов, то этому sc-элементу можно поставить в соответствие не только имя собственное, но и имя нарицательное, которое начинается маленькой (строчной) буквы. Приведём пример семейства эквивалентных (синонимичных) sc-идентификатор, среди которых есть как имена собственные, так и имена нарицательные:
			\begin{scnitemize}
				\item Множество всевозможных обозначений sc-множеств				\item Класс обозначений sc-множеств
				\item обозначение sc-множества
			\end{scnitemize}}
		\end{scneqtoset}
	\end{scnindent} 
\end{SCn}

\begin{SCn}
	\scnheader{обозначение sc-множества}
	\scnidtf{sc-элемент, являющийся знаком множества всевозможных обозначений sc-множеств}
	\begin{scnindent}
		\scniselement{имя собственное}
	\end{scnindent} 
	\scnidtf{Знак множества всевозможных обозначений sc-множеств}
	\begin{scnindent}
		\scniselement{имя собственное}
	\end{scnindent} 
	\scnidtf{Множество всевозможных обозначений sc-множеств}
	\begin{scnindent}
		\scniselement{имя собственное}
	\end{scnindent} 
	\scnidtf{Класс обозначений sc-множеств}
	\begin{scnindent}
		\scniselement{имя собственное}
	\end{scnindent} 
	\scnidtf{sc-элемент, являющийся обозначением множества sc-элементов}
	\begin{scnindent}
		\scniselement{имя нарицательное}
	\end{scnindent} 
	\scnidtf{sc-обозначение множества sc-элементов}
	\begin{scnindent}
		\scniselement{имя нарицательное}
	\end{scnindent}
	\scnidtf{обозначение множества, каждый элемент которого является sc-элементом}
	\scnidtf{обозначение информационной конструкции, принадлежащей SC-коду}
	\scnidtftext{часто используемый sc-идентификатор}{обозначение sc-конструкции}
	\begin{scnsubdividing}
		\scnitem{sc-множество}
		\begin{scnindent}
			\scnidtf{знак константного sc-множества}
			\scneq{\normalfont{(}обозначение sc-множества $ \bigcap $ sc-константа\normalfont{)}}
		\end{scnindent} 
		\scnitem{переменное sc-множество}
		\begin{scnindent}
			\scneq{\normalfont{(}обозначение sc-множества $ \bigcap $ sc-переменная\normalfont{)}}
		\end{scnindent}
	\end{scnsubdividing}
\end{SCn}

\begin{SCn}
	\scnheader{следует отличать*}
	\begin{scnhaselementset}
		\scnitem{обозначение sc-множества}
		\begin{scnindent}
			\scnidtf{обозначение sc-множества, которое может быть как константным sc-множеством, так и переменным sc-множеством}
			\scnidtf{обозначение внутренней для sc-памяти сущности}
			\scnidtf{обозначение внутренней sc-памяти информационной конструкции (sc-конструкции)}
			\begin{scnsubdividing}
			\scnitem{sc-множество}
			\begin{scnindent}
				\scnidtf{обозначение конкретного множества}
				\scnidtf{знак множества}
				\scneq{\normalfont{(}sc-константа $ \bigcap $ обозначение sc-множества\normalfont{)}}
				\scnidtf{конкретное sc-множество}
				\scnidtf{знак константного sc-множества}
				\scnidtf{константное sc-множество}
			\end{scnindent}
			\scnitem{переменное sc-множество}
			\begin{scnindent}
				\scnidtf{произвольное sc-множества}
				\scnidtf{обозначение произвольного sc-множества}
				\scneq{\normalfont{(}sc-переменная $ \bigcap $ обозначение sc-множества\normalfont{)}}
			\end{scnindent} 
			\end{scnsubdividing}
		\end{scnindent}
		\scnitem{sc-множество}
		\scnitem{переменное sc-множество}
		\scnitem{обозначение внешней сущности}
		\begin{scnindent}
			\scnidtf{обозначение сущности, не являющейся множеством sc-элементов (sc-множеством)}
			\scnsuperset{обозначение файла}
			\begin{scnindent}
				\scnidtf{обозначение файла, хранимого либо в файловой памяти той же ostis-системы, в sc-памяти которой хранится знак этого файла, либо в файловой памяти другой дополнительно указываемой компьютерной системы}
			\end{scnindent} 
			\scnsuperset{обозначение информационной конструкции, не являющейся ни sc-множеством, ни файлом}
			\begin{scnindent}
				\scnnote{Примером такой информационной конструкции является напечатанный текст, речевое сообщение, которой следует отличать от его записи в виде аудио-файла \ldots}
			\end{scnindent}
			\scnsuperset{обозначение внешней сущности, не являющейся информационной конструкцией}
			\begin{scnindent}
				\scnnote{Примером такой внешней сущности \ldots}
			\end{scnindent}
		\end{scnindent} 
	\end{scnhaselementset}
\end{SCn}

\begin{SCn}
	\scnheader{sc-множество}
	\scnidtf{sc-конструкция} 
	\scnidtf{информационная конструкция, принадлежащая SC-коду} 
	\scnidtftext{часто используемый sc-идентификатор}{SC-код} 
	\begin{scnindent}
		\scniselement{имя собственное}
	\end{scnindent} 
	\scnidtf{Множество всевозможных sc-конструкций}
	
	\scnheader{обозначение sc-связки}
	\begin{scnsubdividing}
		\scnitem{sc-связка}
		\scnitem{переменная sc-связка}
	\end{scnsubdividing}
\end{SCn}

\begin{SCn}
	\scnheader{sc-связка}
	\scnidtf{знак связи (связки) между sc-элементами} 
	\scnnote{Если элементами sc-связки являются знаки внешних сущностей, то sc-связка является отображением (моделью) некоторой связи, которая связывает указанные внешние сущности}
	\scntext{пояснение}{Понятие sc-связки -- это попытка формализации понятия \underline{целостности}, понятия перехода некоторой совокупности сущности в некоторое новое качество, которое не сводится к свойсвам каждой сущности, входящей в эту совокупность.
		Таким образом, связками следует считать:
		\begin{textitemize}
			\item множество всех чисел, являющихся слагаемыми для заданного числа;
			\item множество всех сотрудников заданной организации, в заданный момент времени;
			\item множество всех сотрудников заданной организации, которые работают в ней;
			\item множество всех точек некоторого отрезка;
			\item множество всех точек некоторого треугольника.
		\end{textitemize}}
	\scntext{примеры}{Примерами sc-связок являются:
		\begin{textitemize}
			\item конкретная окружность, (множество \underline{всех} точек, равноудаленных от некоторой заданной точки);
			\item конкретный отрезок (множество \underline{всех} точек, лежащих между двумя заданными точками с включением этих точек);
			\item конкретный линейный треугольник (множество \underline{всех} точек, лежащих между каждыми двумя из трёх заданных точек с включением этих точек);
			\item пары граничных точек различных отрезков;
			\item тройки вершин различных треугольников.
		\end{textitemize}}
\end{SCn}

\begin{SCn}
	\scnheader{обозначение sc-синглетона}
	\begin{scnsubdividing}
		\scnitem{sc-синглетон}
		\scnitem{переменный sc-синглетон}
	\end{scnsubdividing}
\end{SCn}

\begin{SCn}
	\scnheader{sc-синглетон}
	\scnidtf{sc-множество, являющиеся синглетоном}
	\scnidtf{одномощное sc-множество}
	\scnidtf{sc-множество, имеющее мощность, равную единице}
	\scnidtf{sc-элемент, обозначающий унарную sc-связку}
	\scnidtf{sc-обозначение унарной sc-связки}
	\scnidtf{унарная sc-связка}
	\scnidtf{обозначение sc-синглетона}
	\scnidtf{обозначение одномощного множества, единственный элемент которого является sc-элементом}
\end{SCn}

\begin{SCn}	
	\scnheader{обозначение sc-пары}
	\scniselement{sc-константа}
	\scniselement{sc-класс}
	\begin{scnindent} 
		\begin{scnsubdividing}
			\scnitem{sc-пара}
			\begin{scnindent}
				\scnidtf{константная sc-пара}
				\scnsubset{sc-константа}
				\scniselement{sc-константа}
				\scniselement{sc-класс}
			\end{scnindent} 
			\scnitem{переменная sc-пара}
			\begin{scnindent}
				\scniselement{sc-переменная}
				\scniselement{sc-константа}
				\scniselement{sc-класс}
			\end{scnindent} 
		\end{scnsubdividing}
	\end{scnindent}
\end{SCn}

\begin{SCn}
	\scnheader{sc-пара}
\end{SCn}

\begin{SCn}
	\scnheader{обозначение неориентированной sc-пары}
	\scnidtf{неориентированная sc-пара}
\end{SCn}

\begin{SCn}
	\scnheader{обозначение ориентированной sc-пары}
	\scnidtf{ориентированная sc-пара}
\end{SCn}

\begin{SCn}
	\scnheader{обозначение sc-пары принадлежности}
	\scnidtf{sc-пара принадлежности}
\end{SCn}

\begin{SCn}
	\scnheader{обозначение sc-пары принадлежности}
	\scnidtf{sc-пара принадлежности}
\end{SCn}

\begin{SCn}
	\scnheader{обозначение sc-пары нечёткой принадлежности}
	\scnidtf{sc-пара нечёткой принадлежности}
\end{SCn}

\begin{SCn}
	\scnheader{обозначение sc-пары  позитивной принадлежности}
	\scnidtf{sc-пара позитивной принадлежности}
\end{SCn}

\begin{SCn}
	\scnheader{sc-пара константной постоянной позитивной принадлежности}
	\scnidtf{константная позитивная постоянная sc-пара принадлежности}
\end{SCn}

\begin{SCn}
	\scnheader{sc-пара константной временной позитивной принадлежности}
\end{SCn}

\begin{SCn}
	\scnheader{обозначение sc-пары негативной принадлежности}
	\scnidtf{sc-пара негативной принадлежности}
\end{SCn}

\begin{SCn}
	\scnheader{обозначение sc-пары, не являющейся парой принадлежности}
	\scnidtf{sc-пара, не являющаяся парой принадлежности}
\end{SCn}

\begin{SCn}
	\scnheader{обозначение sc-связки, не являющейся синглетоном и парой}
	\scnidtf{sc-пара, не являющаяся синглетоном и парой}
\end{SCn}

\begin{SCn}
	\scnheader{обозначение sc-класса}
	\scnidtf{sc-класс}
	\begin{scnsubdividing}
		\scnitem{sc-класс}
		\scnitem{переменный sc-класс}
	\end{scnsubdividing}
	\begin{scnsubdividing}
		\scnitem{обозначение sc-класса обозначений sc-связок}
		\scnitem{обозначение sc-класса обозначений sc-классов}
		\scnitem{обозначение sc-класса обозначений sc-структор}
		\scnitem{обозначение sc-классов обозначений внешних сущностей}
	\end{scnsubdividing}
\end{SCn}

\begin{SCn}
	\scnheader{sc-класс}
	\begin{scnsubdividing}
		\scnitem{sc-класс sc-связок}
		\begin{scnindent}
			\scnsuperset{sc-отношение}
		\end{scnindent}
		\scnitem{sc-класс sc-классов}
		\begin{scnindent}
			\scnsuperset{sc-параметр}
		\end{scnindent}
		\scnitem{sc-класс sc-структур}
		\scnitem{sc-класс внешних сущностей}
		\begin{scnindent}
			\scnidtf{sc-класс sc-элементов, являющихся знаками внешних сущностей}
		\end{scnindent} 
		\scnitem{sc-класс sc-элементов разного структурного типа}
		\begin{scnindent}
			\scnhaselementrole{пример}{\scnfilelong{sc-константа}}
		\end{scnindent} 
	\end{scnsubdividing}
\end{SCn}

Перечислим основные основные виды sc-классов

\begin{SCn}
	\scnheader{sc-класс}
	\scnsuperset{sc-отношение}
	\begin{scnindent}
		\scnidtf{sc-класс sc-связок}
		\scnsuperset{бинарное sc-отношение}
		\begin{scnindent} 
			\begin{scnsubdividing}
				\scnitem{бинарное неориентированное sc-отношение}
				\scnitem{бинарное ориентированное sc-отношение}
				\begin{scnindent}
					\scnsuperset{ролевое sc-отношение}
				\end{scnindent} 
			\end{scnsubdividing}
		\end{scnindent}
	\end{scnindent} 
	\scnsuperset{sc-класс sc-классов}
	\begin{scnindent}
		\scnsuperset{sc-параметр}
	\end{scnindent}
	\scnsuperset{sc-класс sc-структур}
	\scnsuperset{sc-класс внешних сущностей}
	\begin{scnindent}
		\scnsuperset{sc-класс внутренних файлов}
	\end{scnindent}
	\scnsuperset{sc-класс эквивалентности}
	\begin{scnindent}
		\scnidtf{фактор-множество соответствующего отношения эквивалентности}
	\end{scnindent} 
	\scntext{пояснение}{Требованиями, предъявляемыми к каждому sc-классу являются:
		\begin{textitemize}
			\item бесконечность этого sc-множества;
			\item наличие общего свойства, присущего \underline{всем} элементам этого sc-множества, в частности, наличие его определения.
		\end{textitemize}
	}
\end{SCn}

\vskip 5cm

\begin{SCn}
	\scnheader{следует отличать*}
	\begin{scnhaselementset}
		\scnitem{sc-связка}
		\scnitem{sc-класс}
	\end{scnhaselementset}
	\begin{scnindent}
		\scntext{сравнение}{В отличие от sc-связки принципом формирования sc-класса является наличие общего свойства, присущего \underline{всем} элементам этого sc-класса \underline{и только им}, (или присущего всем сущностям, которые обозначаются указанными sc-элементами). Таким общим свойством может быть \underline{определение sc-класса} либо принадлежность одному из значений некоторого параметра, то есть одному из элементов фактор-множества, соответствующего некоторому отношению эквивалентности или толерантности.}
		\scntext{пояснение}{Примерами связок являются:
		\begin{textitemize}
			\item множество людей живущих сейчас (динамическое множество);
			\item множество сотрудников некоторой организации (динамическое множество);
			\item отрезок, треугольник.
		\end{textitemize}
		Здесь речь не идёт об эквивалентности свойств самих людей и геометрических точек безотносительно к тому, в состав чего они входят. Поэтому это не является sc-классом.
		}
	\end{scnindent}
	\begin{scnhaselementset}
		\scnitem{sc-класс эквивалентности}
		\begin{scnindent}
			\scnexplanation{В sc-класс входит не просто некоторое количесво попарно эквивалентных между собой сущностей, а абсолютно \underline{все} такие сущности}
		\end{scnindent}
		\scnitem{sc-связка попарно эквивалентных сущностей}
	\end{scnhaselementset}
\end{SCn}

Приведём пример:

\begin{SCn}
	\scnheader{следует отличать*}
	\begin{scnhaselementset}
		\scnitem{множество \underline{всех} треугольников, подобных одному из них}
		\begin{scnindent}
			\scnsubset{sc-класс}
		\end{scnindent}
		\scnitem{конечное множество подобных треугольников}
		\begin{scnindent}
			\scnsubset{sc-связка попарно эквивалентных треугольников}
		\end{scnindent}
	\end{scnhaselementset}
	\begin{scnhaselementset}
		\scnitem{параметр}
		\begin{scnindent}
			\scnidtftext{часто используемый sc-идентификатор}{sc-параметр}
			\scnsubset{класс классов}
		\end{scnindent}
		\scnitem{признак различия}
		\begin{scnindent}
			\scnidtf{признак классификации}
		\end{scnindent}
	\end{scnhaselementset}
	\scnrelfrom{пояснение}{
	\scnstartset
	\scnheaderlocal{параметр}
	\scnsubset{бесконечное множество}
	\bigskip
	\scnheaderlocal{признак различия}
	\scnsubset{конечное множество}
	\begin{scnhaselementrolelist}{пример}
		\scnitem{Признак конечности множеств}
		\begin{scnindent}
			\begin{scneqtoset}
				\scnitem{конечное множество}
				\scnitem{бесконечное множество}				
			\end{scneqtoset}
		\end{scnindent}
		\scnitem{Признак наличия кратных элементов}
		\begin{scnindent}
			\begin{scneqtoset}
				\scnitem{мультимножество}
				\scnitem{множество без кратных вхождений элементов}
			\end{scneqtoset}
		\end{scnindent}
	\end{scnhaselementrolelist}
	
	\scnendstruct
}
\end{SCn}

\vskip 5cm

\begin{SCn}
	\scnheader{sc-класс}
	\scnrelfrom{правила построения внешних идентификаторов sc-элементов заданного класса}{Правила построения внешних идентификаторов sc-элементов, являющихся знаками sc-классов}
	\begin{scnindent}
		\begin{scneqtoset}
			\scnfileitem{Слово ''обозначение'' в начале идентификатора используется тогда, когда в идентифицируемый класс sc-элементов включаются знаки как константных, так и переменных сущностей соответствующего вида}
			\scnfileitem{Слово ''переменный'' в начале идентификатора и ...}
		\end{scneqtoset}
	\end{scnindent}
\end{SCn}

\begin{SCn}
	\scnheader{обозначение sc-структуры}
\end{SCn}

\begin{SCn}
	\scnheader{sc-структура}
\end{SCn}

\begin{SCn}
	\scnheader{следует отличать*}
	\begin{scnhaselementset}
		\scnitem{sc-структура}
		\scnitem{sc-связка}
	\end{scnhaselementset}
	\begin{scnindent} 
		\begin{scnrelfromset}{сравнение}
			\scnfileitem{В отличие от sc-связок в каждую sc-структуру должна входить по крайней мере одна sc-связка вместе с компонентами этой sc-связки}
		\end{scnrelfromset}
	\end{scnindent}
\end{SCn}

\begin{SCn}
	\scnheader{обозначение внешняя сущность}
\end{SCn}

\begin{SCn}
	\scnheader{внешняя сущность}
	\scnidtf{синглетон внешней сущности}
	\scnidtf{обозначение синглетона внешней сущности}
	\scnidtf{sc-элемент, обозначающий синглетон, элементом которого является некоторая внешняя описываемая сущность}
	\scnidtf{множество обозначаемой sc-элементом, являющееся 1-мощным множеством, единственным элементом которого является сущность, внешняя по отношению к sc-конструкции, то есть сущность, не являющаяся sc-элементом}
	\begin{scnrelfromlist}{примечание}
		\scnfileitem{обозначение внешней сущности, то есть sc-элемент, обозначающий этот синглетон, можно также трактовать как sc-элемент, обозначающий соответсвтующую внешнюю описываемую сущность, которую в свою очередь можно считать денодом указанного sc-элемента}
		\scnfileitem{очевидно, что пара принадлежности, связывающая sc-элемент, обозначающий синглетон внешней сущности, не может быть непосредственно представлена в виде соответствующей sc-дуги принадлежности, так как второй компонент этой sc-дуги не находится в sc-памяти}
	\end{scnrelfromlist}
\end{SCn}

\begin{SCn}
	\scnheader{следует отличать*}
	\begin{scnhaselementset}
		\scnitem{синглетон внешней сущности}
		\scnitem{sc-синглетон}
		\begin{scnindent}
			\scnidtf{синглетон, единственным элементом которого является некоторый sc-элемент}
			\scnsubset{sc-множество}
			\begin{scnindent}
				\scnidtf{sc-элемент, обозначающий множество, элементами которого являются \underline{только} sc-элементы}
				\scnidtf{множество sc-элементов}
			\end{scnindent} 
		\end{scnindent}
	\end{scnhaselementset}
\end{SCn}

\begin{SCn}
	\scnheader{обозначение файла}
\end{SCn}

\begin{SCn}
	\scnheader{файл}
\end{SCn}

\vskip 5cm

\begin{SCn}
	\scnheader{внутренний файл}
	\scnidtf{внутренний образ (копия), внутренней информационной конструкции, хранимый в файловой памяти ostis-системы}
	\scnidtf{файл ostis-системы}
	\begin{scnrelfromlist}{примечание}
		\scnfileitem{файловая память ostis-системы, хранящая различного рода информационные конструкции (образы, модели), не являющиеся sc-конструкциями, должна быть тесно связана с sc-памятью этой же ostis-системы. Как минимум каждый файл ostis-системы должен быть связан с тем sc-узлом, которых является знаком этого файла (точнее, знаком синглетона, элементом которого является указанный файл)}
	\end{scnrelfromlist}
\end{SCn}

Перейдем к пояснению смысла понятий используемых в \textit{Логической классификации sc-элементов}.

\begin{SCn}
	\scnheader{sc-константа}
	\scnidtf{sc-элемент, обозначающий константную сущность}
	\begin{scnindent}
		\scntext{сокращение}{обозначение константной сущности}
	\end{scnindent}
	\scnidtf{обозначение константной сущности}
	\scnidtf{знак константной сущности}
	\begin{scnindent}
		\scntext{сокращение}{константная сущность}
		\begin{scnindent}
			\scntext{сокращение}{сущность}
		\end{scnindent} 
	\end{scnindent}
	\scnidtf{константная сущность}
	\scnidtf{конкретная сущность}
	\scnidtf{сущность}
	\scnidtf{константный sc-элемент}
	\scnidtf{sc-элемент, имеющий одно логико-семантическое значение, каковым является он сам}
	\scnidtf{sc-элемент, являющийся знаком константной (конкретной, фиксированной) сущности}
	\scntext{сокращение}{знак константной (конкретной, фиксированной) сущности}
		\begin{scnindent} 
			\scntext{сокращение}{константная (конкретная, фиксированная) сущность}
			\begin{scnindent} 
				\scntext{сокращение}{константная сущность}
			\end{scnindent}
		\end{scnindent}
\end{SCn}

\begin{SCn}
	\scnheader{sc-переменная}
	\scnidtf{переменный sc-элемент}
	\scnidtf{sc-элемент, являющийся обозначением некоторой произвольной (нефиксируемой, переменной) сущности}
	\begin{scnindent}
		\scntext{сокращение}{обозначение произвольной (переменной) сущности}
		\begin{scnindent}
			\scntext{сокращение}{переменная сущность}
		\end{scnindent}
	\end{scnindent}
	\scniselement{sc-константа}
	\scniselement{sc-класс}
	\begin{scnrelfromlist}{примечание}
		\scnfileitem{Сам sc-элемент, имеющий внешний идентификатор ''sc-переменная'' является sc-константой (константным sc-элементом), которая является знаком одного из классов sc-элементов}
	\end{scnrelfromlist}
\end{SCn}

\begin{SCn}
	\scnheader{sc-элемент}
	\scnidtf{обозначение константной или переменной сущности}
	\scnidtf{константная или переменная сущность}
	\scnidtf{sc-константа или переменная сущность}
	\scnidtf{обозначение описываемой сущности, которая может быть как константой, так и переменной сущностью, как внутренней, так и внешней sc-конструкцией для заданной ostis-системы, как информационной конструкцией, так и сущностью которая информационной конструкцией не является, как временной сущностью, так и постоянной, как динамической, так и статической сущностью}
\end{SCn}

\vskip 5cm

\begin{SCn}
	\scnheader{обозначение sc-множества}
	\begin{scnsubdividing}
		\scnitem{sc-множество}
		\begin{scnindent}
			\scnidtf{часто используемый sc-идентификатор}
			\begin{scnindent}
				\scntext{сокращение}{множество sc-элементов}
			\end{scnindent} 
			\scnidtf{константное (конкретное) sc-множество}
			\scnidtf{обозначение (знак) конкретного множества}
			\scnsubset{sc-константа}
			\scniselement{sc-константа}
			\begin{scnsubdividing}
				\scnitem{множество констант}
				\begin{scnindent}
					\scnidtf{множество, каждый элемент которого является константой}
					\scnidtf{множество, являющееся подмножеством Множества всевозможных констант}
				\end{scnindent}
				\scnitem{множество переменных}
				\scnitem{множество констант и переменных}
				\begin{scnindent}
					\scnidtf{множество, элементами которого являются как константы, так и переменные}
				\end{scnindent}
				\scnitem{sc-множество sc-констант}
				\begin{scnindent}
					\scnidtf{sc-множество, элементами которого являются только sc-константы}
				\end{scnindent} 
				\scnitem{sc-множество sc-переменных}
				\begin{scnindent}
					\scnidtf{sc-множество, элементами которого являются только sc-переменные}
				\end{scnindent} 
				\scnitem{sc-множество sc-констант и sc-переменных}
				\begin{scnindent}
					\scnidtf{произвольное множество}
					\scnidtf{обозначение переменного (произвольного) sc-множества)}
					\scnsubset{sc-переменная}
					\scniselement{sc-константа}
					\scnrelboth{следует отличать}{sc-множество sc-переменных}
				\end{scnindent} 
			\end{scnsubdividing}
		\end{scnindent}
		\scnitem{переменное sc-множество}
		\begin{scnindent}
			\scnidtf{обозначение переменного (произвольного) sc-множества}
		\end{scnindent}
	\end{scnsubdividing}
\end{SCn}

Перейдем к пояснению смысла понятий, используемых в \textit{Классификации sc-элементов по темпоральным характеристикам обозначаемых ими сущностей}.

\begin{SCn}
	\scnheader{обозначение временной сущности}
	\begin{scnsubdividing}
		\scnitem{обозначение временной сущности существующей сейчас}
		\begin{scnindent}
			\scnidtf{обозначение временной сущности, существующей в текущий (настоящий) момент}
		\end{scnindent} 
		\scnitem{обозначение прошлой временной сущности}
		\begin{scnindent}
			\scnidtf{обозначение бывшей временной сущности}
			\scnidtf{обозначение временной сущности, которая уже перестала существовать, прекратила своё существование}
		\end{scnindent} 
		\scnitem{обозначение будущей временной сущности}
		\begin{scnindent}
			\scnidtf{обозначение временной сущности, появление которой прогнозируется (планируется, обеспечивается)}
			\scnnote{проектирование и производство новых, ранее не существующих полезных сущностей -- это основное направление человеческой деятельности}
		\end{scnindent} 
	\end{scnsubdividing}
	\begin{scnindent}
		\scnnote{ostis-системы должны постоянно мониторить состояние временных сущностей, так как в процессе их функционирования будущие сущности становятся настоящими, а настоящие -- прошлыми}
\end{scnindent} 
\end{SCn}

\vskip 5cm

\begin{SCn}
	\scnheader{динамическое sc-множество}
	\scnidtf{sc-процесс}
	\scnidtf{процесс}
	\scntext{определение}{sc-множество, у которого некоторые позитивные пары принадлежности, связывающие знак этого множества с его элементами, носят временный характер}
	\scnnote{Сами элементы динамического sc-множества, связанные с ним временными позитивными парами принадлежности, могут обозначать как временные, так и постоянные сущности. Но чаще всего такими временными элементами динамического sc-множества являются знаки временных связок.}
	\begin{scnsubdividing}
		\scnitem{внешний процесс}
		\scnitem{процесс в sc-памяти}
	\end{scnsubdividing}
\end{SCn}

\begin{SCn}
	\scnheader{темпоральная декомпозиция динамического sc-множества}
	\scnidtf{покадровая развертка динамического sc-множества}
	\scnidtf{представление sc-множества в виде кортежа (последовательности) ситуаций}
\end{SCn}

\begin{SCn}
	\scnheader{следует отличать*}
	\begin{scnhaselementset}
		\scnitem{временная сущность}
		\scnitem{обозначение временной сущности}
		\scnitem{переменная временная сущность}
	\end{scnhaselementset}
\end{SCn}

\begin{SCn}
	\scnheader{обозначение временной сущности}
	\begin{scnsubdividing}
		\scnitem{временная сущность}
		\begin{scnindent}
			\scnidtf{знак конкретной (константной) временной сущности}
		\end{scnindent} 
		\scnitem{переменная временная сущность}
		\begin{scnindent}
			\scnidtf{обозначение произвольной временной сущности}
		\end{scnindent} 
	\end{scnsubdividing}
\end{SCn}

\begin{SCn}
	\scnheader{сформированное sc-множество}
	\scnidtf{sc-множество, у которого в текущем состоянии sc-памяти перечислены все его элементы}
	\scniselement{динамическое sc-множество}
	\scnnote{Очевидно, что сформированным sc-множеством может стать только конечное sc-множество}
\end{SCn}

\begin{SCn}
	\scnheader{формируемое sc-множество}
\end{SCn}

\begin{SCn}
	\scnheader{sc-множество, элементы которого не известны}
\end{SCn}

\begin{SCn}
	\scnheader{сформированный файл}
\end{SCn}

\begin{SCn}
	\scnheader{формируемый файл}
\end{SCn}

\begin{SCn}
	\scnheader{файл, структура которого не известна}
\end{SCn}

Перейдем к рассмотрению семантически выделяемых классов sc-элементов, которые необходимо ввести \underline{дополнительно} к выше рассмотренным классам sc-элементов.

\begin{SCn}
	\scnheader{sc-элемент, не являющийся sc-синглетоном и sc-парой}
\end{SCn}

\begin{SCn}
	\scnheader{sc-элемент, копируемый в других компьютерных системах}
	\scnidtf{sc-элемент, имеющий в других компьютерных системах свои копии и/или копии обозначаемой им информационной конструкции}
\end{SCn}

\begin{SCn}
	\scnheader{отношение, заданное на множестве sc-элементов, копируемых в других компьютерных системах*}
	\scnhaselement{}
	\scnhaselement{}
	\scnhaselement{}
\end{SCn}

\begin{SCn}
	\scnheader{отношение, заданное на множестве sc-элементов, имеющих копии в других компьютерных системах*}
	\scnhaselement{ostis-система, в sc-памяти которой хранится копия заданного sc-элемента*}
	\scnhaselement{компьютерная система, в файловой памяти которой хранится заданный файл*}
	\begin{scnindent}
		\scnnote{указанная компьютерная система назначается хранителем файла}
	\end{scnindent}
	\scnhaselement{ostis-система, в sc-памяти которой хранится копия знака заданного sc-множества и все известные в текущий момент его элементы*}
	\begin{scnindent}
		\scnnote{указанная ostis-система назначается основным хранителем указанного sc-множества}
	\end{scnindent}
\end{SCn}

\begin{SCn}
	\scnheader{информационная конструкция}
	\begin{scnsubdividing}
		\scnitem{sc-множество}
		\begin{scnindent}
			\scnidtf{sc-конструкция}
			\scnidtf{информационная конструкция SC-кода}
			\scnidtf{внутренняя информационная конструкция ostis-системы, хранимая в её sc-памяти}
		\end{scnindent}
		\scnitem{файл}
		\begin{scnindent}
			\scnidtf{файл ostis-системы}
			\scnidtf{внутреняя информационная конструкция ostis-системы, хранимая в её файловой памяти}
			\scnnote{файл, может храниться в памяти другой компьютерной системы и, в частности, в файловой памяти другой ostis-системы}
		\end{scnindent}
		\scnitem{внешняя информационная конструкция, не являющаяся ни файлом, ни sc-конструкцией}
	\end{scnsubdividing}
\end{SCn}

\begin{SCn}
	\scnheader{sc-идентификатор}
	\scnidtf{внешний идентификатор sc-элемента}
	\scnsuperset{файл}
	\begin{scnsubdividing}
		\scnitem{основной идентификатор}
		\scnitem{часто используемый sc-идентификатор}
		\scnitem{дополнительный sc-идентификатор}
	\end{scnsubdividing}
\end{SCn}

\begin{SCn}
	\scnheader{sc-идентификатор*}
	\scnidtf{Бинарное ориентированное отношение, связывающее sc-элементы с их внешними идентификаторами}
\end{SCn}


\newpage
\subsection{Структура базовой семантической спецификации sc-элемента}

\newpage
\subsection{Онтологическая формализация Базовой денотационной семантики SC-кода}

\section{Синтаксис SC-кода}
\label{sec_sr_scsyntax}

\begin{SCn}
	\scnheader{sc-элемент}
	\begin{scnsubdividing}
		\scnitem{sc-коннектор}
		\begin{scnindent}
			\begin{scnsubdividing}
				\scnitem{sc-дуга}
				\begin{scnindent}
					\begin{scnsubdividing}
						\scnitem{базовая sc-дуга}
						\scnitem{sc-дуга общего вида}
						\begin{scnindent}
							%							\scneq{\normalfont{(}sc-синглетон $\cup$ sc-связка не являющаяся синглетоном и парой\normalfont{)}}
						\end{scnindent}	
					\end{scnsubdividing}
				\end{scnindent}	
				\scnitem{sc-ребро}
				\begin{scnindent}
					\scneq{неориентированная sc-пара}
				\end{scnindent}	
			\end{scnsubdividing}
		\end{scnindent}	
		\scnitem{sc-узел общего вида}
		\begin{scnindent}
			\scneq{\normalfont{(}\textit{sc-синглетон} $\cup$ \textit{sc-связка, не являющаяся синглетоном и парой}\normalfont{)}}
		\end{scnindent}	
	\end{scnsubdividing}
\end{SCn}


\section{Файлы ostis-системы}
\label{sec_sr_ostisfiles}
\section{Смысловое пространство ostis-систем}
\label{sec_sr_semspace}
\begin{SCn}
	\begin{scnrelfromlist}{ключевое понятие}
		\scnitem{sc-отношение}
		\scnitem{бинарное sc-отношение}
		\scnitem{слотовое sc-отношение}
		\scnitem{sc-структура*}
		\scnitem{элементарно представленный элемент'}
		\scnitem{полносвязно представленный элемент'}
		\scnitem{полностью представленный элемент'}
		\scnitem{sc-связка'}
		\scnitem{sc-отношение'}
		\scnitem{sc-класс'}
		\scnitem{сущностное замыкание*}
		\scnitem{содержательное замыкание*}
		\scnitem{sc-отношение сходства по слотовым отношениям*}
		\scnitem{sc-отношение семантического сходства по слотовым отношениям*}
		\scnitem{связная sc-структура*}
		\scnitem{семантическое сходство sc-структур*}
		\scnitem{семантическое непрерывное сходство sc-структур*}
		\scnitem{ключевой запрос’}
		\scnitem{минимальный ключевой запрос’}
		\scnitem{полная семантическая окрестность элемента*}
		\scnitem{интроспективный ключевой элемент’}
		\scnitem{топологическое пространство}
		\scnitem{топологическое пространство замыкания инцидентности коннекторов}
		\scnitem{топологическое пространство синтаксического замыкания}
		\scnitem{топологическое пространство сущностного замыкания}
		\scnitem{топологическое пространство содержательного замыкания}
		\scnitem{метрика}
		\scnitem{семантическая метрика}
		\scnitem{метрическое пространство}
		\scnitem{метрическое конечное синтаксическое пространство}
		\scnitem{метрическое конечное семантическое пространство}
		\scnitem{псевдометрика}
		\scnitem{псевдометрическое пространство}
		\scnitem{псевдометрическое конечное семантическое пространство}
	\end{scnrelfromlist}
\end{SCn}

Понятие SC-пространства наряду с понятием SC-кода является необходимым для уточнения и формализации понятия смысла информационных конструкций и в унификации смыслового представления информации. В SC-пространстве можно выделять структуры, связанные как с синтаксическими свойствами текстов SC-кода, так и с его семантикой. В последнем случае речь можно вести о «смысловом пространстве». Смысл информационной конструкции, в конечном счёте, определяется (1) внутренними связями всех элементарных фрагментов этой конструкции и (2) её внешними связями с элементами смыслового пространства (её положением в контексте). Смысловое пространство является результатом естественного становления знаний в процессе их интеграции.
Важнейшим достоинством SC-пространства является возможность уточнения таких понятий, как понятие аналогичности (сходства и отличия) различных описываемых «внешних» сущностей, аналогичности унифицированных семантических сетей (текстов SC-кода), понятие семантической близости описываемых сущностей (в том числе, и текстов SC-кода).

Следует отметить, что в силу абстрактности языков модели унифицированного семантического представления знаний и условности выбора меток элементов их текстов, нельзя исключить, что объединение двух произвольных текстов таких языков не будет текстом языка модели унифицированного семантического представления знаний. Чтобы избежать результатов подобных эклектических объединений с точки зрения синтаксиса или семантики, для абстрактных языков следует рассматривать множество «смысловых пространств». Однако, для конкретных (реальных) языков может оказаться достаточным рассмотрение одного «смыслового пространства».
Далее рассмотрим:
\begin{scnitemize}
	\item возможность перехода от sc-текстов к графовым структурам и от них к топологическому пространству;
	\item возможность перехода от sc-текстов к графовым структурам и от них к многообразию (топологическому пространству);
	\item возможность перехода от sc-текстов к графовым структурам и от них к метрическому пространству.
\end{scnitemize}
Чтобы исследовать структурные свойства SC-пространства, можно использовать уже разработанные модели пространств и связь их известными топологическими моделями например такими как графы. При этом изначально не будем принимать в расчёт динамические особенности, связанные с обработкой знаний, однако позже будет показано, что учёт динамики в процессах обработки и при становлении знаний является необходимым для вычисления семантической метрики, являющейся одним из определяющих признаков знаний.
Обратимся к исследованию структурно-топологических свойств пространства.

Структурно-топологические свойства могут свидетельствовать о синтаксических или семантических зависимостях обозначений текстов языка, позволяющих упростить работу со структурами за счёт перехода к более простым структурам на уровнях управления данными или знаниями в характерных задачах управления для библиотек компонентов.
На множестве элементов, образующих SC-пространство, можно изучать топологические свойства и рассматривать SC-пространство как топологическое пространство. Следует заметить, что, несмотря на то, что SC-код ориентирован на смысловое представление знаний, в силу наличия НЕ-факторов, не все смыслы могут быть представлены в некоторый момент времени и не будет известна структура соответствующего представления. Поэтому структурно-топологические свойства текстов языка представления знаний скорее определяют синтаксическое пространство, нежели семантическое (смысловое). Хотя оба могут приближаться друг к другу по мере устранения неопределённостей, вызванных НЕ-факторами.

Рассмотрим следующие виды топологических пространств:
\begin{scnitemize}
	\item топологическое пространство замыкания инцидентности коннекторов;
	\item топологическое пространство синтаксического замыкания;
	\item топологическое пространство сущностного замыкания;
	\item топологическое пространство содержательного замыкания.
\end{scnitemize}

\begin{SCn}
	\scnheader{топологическое пространство}
	\scnnote{Топологическое пространство -- множество с определённым над ним множеством (семейством) (открытых) подмножеств, включая само множество и пустое множество. Для любого счётного подмножества семейства результат объединения принадлежит семейству, а для любого подмножества семейства результат пересечения также принадлежит семейству. Дополнения множеств семейства до наибольшего из множеств называются замкнутыми множествами.}
\end{SCn}

Чтобы рассмотреть более детально некоторые виды топологических пространств введём следующие понятия.

\begin{SCn}
	\scnheader{sc-отношение}
	\scnnote{sc-отношение -- sc-множество sc-связок.}
\end{SCn}

\begin{SCn}
	\scnheader{бинарное sc-отношение}
	\scnnote{Бинарное sc-отношение -- sc-множество sc-пар.}
\end{SCn}

\begin{SCn}
	\scnheader{слотовое sc-отношение}
	\scnnote{Слотовое sc-отношение -- sc-множество (ориентированных) sc-пар, которые не являются узловыми sc‑парами.}
\end{SCn}

\begin{SCn}
	\scnheader{sc-структура*}
	\scnnote{sc-структура* -- sc-множество, в котором есть непустое sc-подмножество-носитель (множество первичных элементов sc-структуры*).}
\end{SCn}

\begin{SCn}
	\scnheader{элементарно представленный элемент’}
	\scnnote{Элементарно представленный элемент’ -- элемент sc-структуры*, sc‑множество, все элементы которого являются элементами sc-структуры*.}
\end{SCn}

\begin{SCn}
	\scnheader{полносвязно представленный элемент’}
	\scnnote{Полносвязно представленный элемент’ -- элемент sc-структуры*, sc‑множество, все элементы и все принадлежности которому являются элементами sc-структуры*, или sc‑дуга, являющаяся элементарно представленным элементом’ этой sc-структуры*.}
\end{SCn}

\begin{SCn}
	\scnheader{полностью представленный элемент’}
	\scnnote{Полностью представленный элемент’ -- полносвязно представленный элемент’ sc‑структуры*, с любым элементом, не являющимся sc-дугой, выходящей из него, связанный принадлежащей этой sc-структуре* sc-дугой принадлежности или sc-дугой непринадлежности.}
\end{SCn}

\begin{SCn}
	\scnheader{sc-связка’}
	\scnnote{sc-связка’ -- полносвязно представленный элемент’ sc-структуры*, принадлежащий sc‑отношению’ этой sc-структуры*, являющийся sc-связкой.}
\end{SCn}

\begin{SCn}
	\scnheader{sc-отношение’}
	\scnnote{sc-отношение’ -- полносвязно представленный элемент’ sc-структуры*, sc-отношение, все элементы которого являются sc-связками’ этой sc-структуры*.}
\end{SCn}

\begin{SCn}
	\scnheader{sc-класс’}
	\scnnote{sc-класс’ -- полносвязно представленный элемент’ sc-структуры*, все элементы которого являются элементами sc‑структуры*, не являющийся ни sc-отношением’, ни sc-связкой’ этой sc‑структуры*.}
\end{SCn}

\begin{SCn}
	\scnheader{сущностное замыкание*}
	\scnnote{Сущностное замыкание* -- наименьшая надмножество* (структура*), в котором каждый элемент является элементарно представленным’.}
\end{SCn}
◦ ;

\begin{SCn}
	\scnheader{содержательное замыкание*}
	\scnnote{Содержательное замыкание* -- наименьшее надмножество* (структура*), в котором каждый элемент является полносвязно представленным’}
\end{SCn}

\begin{SCn}
	\scnheader{sc-отношение сходства по слотовым отношениям*}
	\scnnote{sc-отношение сходства по слотовым sc-отношениям* -- sc-отношение, являющееся рефлексивным по этим слотовым отношениям, т.е. для любого элемента, входящего в связку этого sc-отношения под одним из слотовых sc-отношений, найдётся связка этого sc‑отношения, в которую он входит под каждым из этих слотовых sc-отношений.}
\end{SCn}

\begin{SCn}
	\scnheader{sc-отношение семантического сходства по слотовым отношениям*}
	\scnnote{sc-отношение семантического сходства по слотовым отношениям* -- sc-отношение сходства по слотовым sc-отношениям* si и sj, в котором каждый элемент под слотовым sc-отношением si, может быть преобразован к элементу синтаксического типа элемента под слотовым sc-отношением sj; два инцидентных sc-элемента под слотовым sc-отношением si, в рамках этого sc-отношения семантического сходства соответствуют инцидентным элементам соответственно под слотовым sc-отношением sj.}
\end{SCn}

\begin{SCn}
	\scnheader{связная sc-структура*}
	\scnnote{Связная sc-структура* -- sc-структура*, являющаяся связной.}
\end{SCn}

\begin{SCn}
	\scnheader{семантическое сходство sc-структур*}
	\scnidtf{семантическое подобие sc-структур*}
		\scnnote{Семантическое сходство sc-структур* -- связывает sc-множество sc-структур* с sc‑структурой* sc‑отношением семантического сходства по слотовым sc-отношениям si, sj так, что для каждой sc-структуры* из sc‑множества найдётся её элемент и связка этого sc‑отношения сходства, в которую он входит под слотовым sc‑отношением si, а под слотовым sc‑отношением sj входит элемент sc-структуры*, также для каждого элемента sc‑структуры найдётся связка этого sc-отношения сходства, в которую он входит под слотовым sc‑отношением sj, а под слотовым sc‑отношением si входит элемент sc-структуры* из sc‑множества.}
\end{SCn}

\begin{SCn}
	\scnheader{семантическое непрерывное сходство sc-структур*}
	\scnidtf{семантическое непрерывное подобие sc-структур*}
	\scnnote{Семантическое непрерывное сходство sc-структур* -- связывает sc-множество sc‑структур* со связной sc‑структурой* sc‑отношением семантического сходства по слотовым sc-отношениям si, sj так, что для каждой sc-структуры* из sc‑множества найдётся её элемент и связка этого sc‑отношения сходства, в которую он входит под слотовым sc‑отношением si, а под слотовым sc‑отношением sj входит элемент связной sc-структуры*, также для каждого элемента связной sc-структуры найдётся связка этого sc-отношения сходства, в которую он входит под слотовым sc‑отношением sj, а под слотовым sc‑отношением si входит элемент sc-структуры* из sc‑множества.}
\end{SCn}

\begin{SCn}
	\scnheader{ключевой запрос’}
	\scnnote{Ключевой запрос’ -- поисковый-проверочный запрос (от одного известного элемента), который выполняется хотя бы от одного элемента и не выполняется хотя бы от одного элемента.}
\end{SCn}

\begin{SCn}
	\scnheader{минимальный ключевой запрос’}
	\scnnote{Минимальный ключевой запрос’ -- ключевой запрос, который находит sc‑подмножества множеств элементов, находимых всеми другими ключевыми запросами, которые имеют те же области известных элементов выполнимости и невыполнимости.}
\end{SCn}

\begin{SCn}
	\scnheader{полная семантическая окрестность элемента*}
	\scnnote{Полная семантическая окрестность элемента* -- все элементы, находимые выполнимыми минимальными ключевыми запросами от этого элемента (c учётом дизъюнктивного поиска и отрицания поиска).}
\end{SCn}

\begin{SCn}
	\scnheader{интроспективный ключевой элемент’}
	\scnnote{Интроспективный (базовый) ключевой элемент’ -- элемент множества (из класса наименьших таких множеств) элементов такого, что любая полная семантическая окрестность любого элемента является sc-подмножеством объединения их полных семантических окрестностей}
\end{SCn}


\begin{SCn}
	\scnheader{топологическое пространство замыкания инцидентности коннекторов}
	\scnnote{Топологическое пространство замыкания инцидентности коннекторов на множестве sc-элементов -- топологическое пространство, все замкнутые множества которого содержат все sc-элементы этого множества, до которых есть маршрут по ориентированным связкам отношения инцидентности коннекторов.}
\scncomment{В общем случае не удовлетворяет аксиоме отделимости по Тихонову. Прагматика рассмотрения таких пространств обуславливается операциями удаления sc-элементов и коннекторов, которым они инцидентны. Удаление sc-элемента требует удаления всех коннекторов, замыканию любой открытой окрестности которых он принадлежит.}
\end{SCn}

\begin{SCn}
	\scnheader{топологическое пространство синтаксического замыкания}
	\scnnote{Топологическое пространство синтаксического замыкания на множестве sc-элементов -- топологическое пространство, все замкнутые множества которого содержат все sc-элементы этого множества, до которых есть маршрут по ориентированным связкам отношения инцидентности.}
\scncomment{В общем случае не удовлетворяет аксиоме отделимости по Колмогорову. В качестве основы замкнутых множеств топологического пространства можно выделить синтаксическое замыкание, однако в силу возможности проведения дуг из любого sc-узла в любой в итоговом случае (в итоге процесса устранения НЕ-факторов) такое пространство является тривиальным (антидискретным) пространством. Отношение объединения топологических пространств синтаксического замыкания алгебраически не замкнуто на множестве топологических пространств синтаксического замыкания. По той же причине для любого неполного топологического пространства синтаксического замыкания можно рассмотреть топологическое пространство синтаксического замыкания, носитель которого является надмножеством носителя первого и которое не сохраняет замкнутые множества. В этом смысле топология на основе синтаксического замыкания не является устойчивой относительно процессов становления знаний и её рассмотрение прагматически не оправдывается. Топология полного же топологического пространства синтаксического замыкания антидискретна (тривиальна). Таким образом, у полного топологического пространства синтаксического замыкания все топологические подпространства синтаксического замыкания обладают антидискретной (тривиальной) топологией.}
\end{SCn}
\begin{SCn}
	\scnheader{топологическое пространство сущностного замыкания}
	\scnnote{Топологическое пространство сущностного замыкания на множестве sc-элементов -- топологическое пространство, все замкнутые множества которого являются сущностными замыканиями.}
	\scncomment{В общем случае не удовлетворяет аксиоме отделимости по Тихонову. В качестве носителя топологического (под)пространства можно выделить сущностное замыкание. Топологическое пространство сущностного замыкания сохраняет замкнутые множества любых топологических пространств сущностного замыкания, носитель которых является подмножеством его носителя и сущностным замыканием. Такие пространства образуют структуру топологических подпространств‑топологических надпространств сущностного замыкания. Топология пространств в этой структуре разнообразна.}
\end{SCn}

\begin{SCn}
	\scnheader{топологическое пространство содержательного замыкания}
	\scnnote{Топологическое пространство содержательного замыкания на множестве sc-элементов -- топологическое пространство, все замкнутые множества которого являются содержательными замыканиями.}
	\scncomment{В общем случае не удовлетворяет аксиоме отделимости по Тихонову. В качестве носителя топологического (под)пространства можно выделить содержательное замыкание. Топологическое пространство содержательного замыкания сохраняет замкнутые множества любых топологических пространств содержательного замыкания, носитель которых является подмножеством его носителя и содержательным замыканием. Такие пространства образуют структуру топологических подпространств‑топологических надпространств содержательного замыкания. Топология пространств в этой структуре разнообразна.
	}
\end{SCn}

Возможен переход от sc-структур к многообразиям и топологическим пространствам путём сведения sc-структур к графовым структурам, подробно вопросы сведения sc-структур к графовым структурам и далее -- к многообразиям и топологическим пространствам рассмотрены в работе [].

Для более сложных структур таких, как полная семантическая окрестность, топологические свойства подлежат дальнейшему изучению.

Далее можно рассмотреть метрические пространства, в частности -- конечные подпространства с полностью представленными sc-элементами. 

\begin{SCn}
	\scnheader{метрика}
	\scnnote{Метрика -- функция двух аргументов, принимающая значения на (линейно) упорядоченном носителе группы, неотрицательна, равна нейтральному элмененту (нулю) только при равенстве аргументов, симметрична, удовлетворяет неравенству треугольника.}
\end{SCn}

\begin{SCn}
	\scnheader{метрическое пространство}
	\scnnote{Метрическое пространство -- множество, с определённой на нём функцией двух аргументов, являющейся метрикой, принимающей значения на упорядоченном носителе группы.}
\end{SCn}

\begin{SCn}
	\scnheader{семантическая метрика}
	\scnidtf{семантическая близость}
	\scnnote{Семантическая метрика -- метрика, определённая на знаках и выражающая количественно близость их значений.}
\end{SCn}

\begin{SCn}
	\scnheader{метрическое конечное синтаксическое пространство}
	\scnnote{Метрическое конечное синтаксическое пространство SC-кода -- метрическое пространство с конечным носителем, элементами которого являются обозначения (sc-элементы), а значение метрики может быть определено через отношения инцидентности элементов без учёта их семантического типа.}
\end{SCn}

\begin{SCn}
	\scnheader{метрическое конечное семантическое пространство}
	\scnnote{Метрическое конечное семантическое пространство SC-кода -- метрическое пространство с конечным носителем, элементами которого являются обозначения (sc-элементы), а значение метрики не может быть определено через отношения инцидентности элементов без учёта их семантического типа.}
\end{SCn}

\begin{SCn}
	\scnheader{псевдометрика}
	\scnnote{Псевдометрика -- функция двух аргументов, принимающая значения на (линейно) упорядоченном носителе группы, неотрицательна, симметрична, удовлетворяет неравенству треугольника.}
\end{SCn}

\begin{SCn}
	\scnheader{псевдометрическое пространство}
	\scnnote{Псевдометрическое пространство -- множество, с определённой на нём функцией двух аргументов, являющейся псевдометрикой, принимающей значения на упорядоченном носителе группы.}
\end{SCn}

\begin{SCn}
	\scnheader{псевдометрическое конечное семантическое пространство}
	\scnnote{Псевдометрическое конечное семантическое пространство SC-кода -- псевдометрическое пространство с конечным носителем, элементами которого являются обозначения (sc-элементы), а значение псевдометрики не может быть определено через отношения инцидентности элементов без учёта их семантического типа.}
\end{SCn}



В силу неполноты выразительных средств для представления изменяющихся со временем знаний, отсутствия определённой пространственно-временной модели, наличия семантически неопределённых или слабоопределённых обозначений в текстах да и наличия недоопределённости самих текстов описанного в предыдущих разделах языка, на данном этапе в этом описании затруднительно предложить какую-либо модель метрического пространства для более сложных структур, учитывающих НЕ‑факторы, связанные с пространством‑временем. Это станет возможным при проявлении желания идти навстречу, готовности к конвергенции, интероперабельности и после достижения консенсуса, достаточного для соответствующего описания развития предлагаемого стандарта и языка.

Тем не менее, некоторые такие модели были успешно предложены в работе []. Предложенные модели полагались на представление, способное выразить семантику переменных обозначений и операционную семантику расширенными средствами алфавита. Для построения подобных моделей, кроме расширенных средств алфавита, предлагается полагаться на модели, описывающие процессы интеграции и становления знаний [], на средства спецификации знаний [], ориентированные на рассмотрение финитных структур, что позволяет перейти к рассмотрению сложных метрических соотношений в рамках метамодели смыслового пространства.

В современных работах в технических науках [bManin], возможно, наиболее близкими понятиями являются понятия, выражающие смысл термина «семантическое пространство» (интериорный подход (Табл. 1)).
Общим во многих подходах к работе с «семантическим пространством» является рассмотрение словоформ или лексем (множеств словоформ) и их признаков (Табл. 1). В литературе [bManin] встречаются следующие подходы (Табл. 2):
\begin{scnitemize}
	\item подход на основе семантических осей и пространства признаков (бинарных $\left\lbrace 0,1\right\rbrace ^{n}$, монополярных $\left[0;1\right]^{n}$, биполярных $\left[-1;1\right]^{n}$);
	\item подход на основе семантических осей и нейронного кодирования места в поле смыслов (слова и словосочетания имеют области (подмножества) значений, связываясь другими частями речи как включением и пересечением, тексты соответствуют пути связанных областей, бинарное кодирование групп нейронов, распознающих смыслы);
	\item подход на основе модели «смысл-текст» [bMeaningText] (отражение неполноты семантических шкал и анализ синтагм и поверхностно-синтаксической структуры);
	\item нейролингвистические данные отражает процессы синтеза и восприятия речи в нейронных сетях (сеть лексического синтеза), близка к модели «смысл‑текст»;
	\item модели, построенные на основе статического анализа (корпусов) текстов (модель векторного пространства).
\end{scnitemize}
Статистический подход к обработке естественного языка противопоставляется интуиции и коммуникативному опыту учёных [bManin].

В основе подхода лежит семантическая статическая гипотеза, что смысл слов (лексем) определяется контекстом использования (его статистическим образом) в языке (с коммуникативной структурой) [bManin].

Модель векторного пространства семантики [bManin]. Модель рассматривается для двух случаев: большого словаря ($N\leq{M}$) и задачи информационного поиска ($M\leq{N}$). $M$ -- размер словаря, $N$ -- количество контекстов.

На основе статистики строится матрица размерности $M\times{N}$ частот $p_{ij}$ появления лексемы (слова) $w_{i}$ в документе (контексте, подтексты, которые могут перекрываться) $c_{j}$.

В знаменателе -- оценки вероятности слова и контекста соответственно.

В случае невырожденной матрицы $r=N$ каждая такая матрица задаёт точку в грассманиане $N$‑мерных подпространств $M$‑мерного пространства ($N\leq{M}$).

В случае невырожденной матрицы $r=M$ каждая такая матрица задаёт точку в грассманиане $M$‑мерных подпространств $N$‑мерного пространства ($M\leq{N}$).

Каждый текст -- точка в грассманиане [bGrassmanian], соответствующем проективному пространству , относительно одного выделенного контекста. Для всех контекстов получая ориентированную $N$-ку, в соответствии с порядком контекстов в текстах, можно построить маршрут (путь), соединяя геодезическими соседние точки в $N$-ке. Для двух текстов  и  это будут две ломанные, между которыми можно вычислить метрику Фреше [bFrechet], используя метрику Фубини-Штуди [bFubiniStudy] в , для этого следует параметризовать пути  и  через  (,): 
.

Другой способ задать линейный порядок -- это рассмотреть фильтрацию (флаг (флаговое многообразие)) [bFlagManifold] в , заданную расширяющимися контекстами. В итоге для текста получаем точки (флаги) во флаговом многообразии. Для флаговых многообразий тоже можно вычислить метрику Фубини-Штуди [bFubiniStudy].

Этот порядок соответствует временному измерению (процессу коммуникации во времени), что может быть существенным. Другой порядок может быть не зависимым от этого, например алфавитный или порядок в соответствии с законом Ципфа [bZipf], [bZipf2]. 

Сравнение подходов к построению «семантических пространств»

\begin{tabular}{|>{\centering\arraybackslash}m{3cm}|>{\centering\arraybackslash}m{2cm}|>{\centering\arraybackslash}m{2cm}|>{\centering\arraybackslash}m{3cm}|>{\centering\arraybackslash}m{3cm}|>{\centering\arraybackslash}m{3cm}|}
	\hline
& семанти-ческие оси и простран-ства признаков
& семанти-ческие оси и нейронное кодирование признаков
& модель <<смысл‑текст>>
& нейролингвисти-ческое кодирование
& статистическая модель (модель векторного пространства семантики)
\\
	\hline
	определённые семантические оси
	& +
	& +
	& -
	& -
	& -
	\\
	\hline
	динамическая (вычислительная) декомпозиция
	& -
	& +
	& -
	& +
	& -
	\\
	\hlineсеманти-ческие оси и простран-ства признаков
	семанти-ческие оси и нейронное кодирование признаков
	модель смысл‑текст
	нейролингвисти-ческое кодирование
	статистическая модель (модель векторного пространства семантики)
	
	определённые семантические оси
	+
	+
	-
	-
	-
	
	динамическая (вычислительная) декомпозиция
	-
	+
	-
	+
	-
	
	анализ когнитивных процессов (интроспекция)
	-
	+
	-
	+
	-
	
	учёт НЕ‑факторов (неполнота)
	-
	-
	+
	+
	+
анализ когнитивных процессов (интроспекция)
 & -
 & +
 & -
 & +
 & -
 \\
	\hline
учёт НЕ‑факторов (неполнота)
 & -
 & -
 & +
 & +
 & +
 \\
	\hline
\end{tabular}






Вопросы соотнесения смыслов, их формализации, развития языков в пространстве и времени рассмотрены в работах В.В. Мартынова [bMartynov], [bMartynov2], [bGordey].

%\input{author/references}