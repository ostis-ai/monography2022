\chapauthor{Шункевич Д.В.}
\chapter{Ситуационное управление обработкой знаний в интеллектуальных компьютерных системах нового поколения}
\chapauthortoc{Шункевич Д.В.}
\label{chapter_situation_management}

\abstract{Аннотация к главе.}

\section{Решатели задач ostis-систем}

Одним из ключевых компонентов интеллектуальной системы, обеспечивающим возможность решать широкий круг задач, является решатель задач. Особенностью решателей задач интеллектуальных систем по сравнению с другими современными программными системами является необходимость решать задачи в условиях, когда сведения, необходимые для решения задачи, не локализованы явно в базе знаний интеллектуальной системы и должны быть найдены в процессе решения задачи на основании каких-либо критериев. Говоря другими словами, если в традиционных системах при решении задачи всегда подразумевается, что есть некоторые локализованные исходные данные (''дано'') и некоторое описание желаемого результата (''что требуется''), то в интеллектуальной системе в качестве исходных данных при решении большого числа задач выступает вся имеющаяся на текущий момент в системе информация, то есть вся база знаний. Кроме того, при невозможности решения задачи в текущем состоянии базы знаний интеллектуальная система должна иметь возможность понять, чего именно не хватает для решения задачи и попытаться добыть недостающие сведения во внешней среде (например, запросить у пользователя).

К настоящему времени в рамках различных направлений искусственного интеллекта разработано большое количество различных моделей решения задач, каждая из которых позволяет решать задачи определенного класса. Расширение областей применения интеллектуальных систем требует от таких систем возможности решать так называемые комплексные задачи, решение каждой из которых требует комбинирования нескольких моделей решения задач, при этом априори неизвестно, в каком порядке и сколько раз будет применяться так или иная модель. Решатели задач, в рамках которых комбинируются несколько моделей решения задач, получили название \textit{гибридных решателей задач}, а интеллектуальные системы, в рамках которых комбинируются различные виды знаний и различные модели решения задач -- название \textit{гибридных интеллектуальных систем}.

В рамках Технологии OSTIS решатель задач ostis-системы определяется как совокупность всех \textit{навыков}, которыми обладает ostis-система на текущий момент времени (более подробно о понятии навыка смотрите в \textit{Главе \ref{chapter_actions} \nameref{chapter_actions}}).

Предлагаемый в рамках \textit{Технологии OSTIS} подход к построению решателей задач позволяет обеспечить их модифицируемость, что, в свою очередь, позволяет \textit{ostis-системе} при необходимости легко приобретать новые \textit{навыки}, модифицировать (совершенствовать) уже имеющиеся, и даже избавляться от некоторых навыков с целью повышения производительности системы. Таким образом, имеет смысл говорить не о жестко фиксированном решателе задач, который разрабатывается один раз при создании первой версии системы и далее не меняется, а о совокупности навыков, фиксированной в каждый текущий момент времени, но постоянно эволюционирующей.

%TODO суть предлагаемого подхода и его обоснование

\begin{SCn}
\scnheader{решатель задач ostis-системы}
\scnrelto{семейство подмножеств}{навык}
\scnsuperset{гибридный решатель задач ostis-системы}
\begin{scnindent}
	\scnidtf{решатель задач ostis-системы, реализующий две и более модели решения задач}
\end{scnindent}
\scnsuperset{объединенный решатель задач ostis-системы}
\begin{scnindent}
\scnidtf{полный решатель задач ostis-системы}
\scnidtf{интегрированный решатель задач ostis-системы}
\scnidtf{решатель задач ostis-системы, реализующий все ее функциональные возможности, как основные, так и вспомогательные}
\end{scnindent}
\end{SCn}

В общем случае \textit{объединенный решатель задач ostis-системы}, решает задачи, связанные с:
	\begin{itemize}
		\item обеспечением основных функциональных возможностей системы (например, решение явно сформулированных задач по требованию пользователя);
		\item обеспечением корректности и оптимизацией работы самой ostis-системы (перманентно на протяжении всего жизненного цикла ostis-системы);
		\item обеспечением повышения квалификации конечных пользователей и разработчиков ostis-системы;
		\item обеспечением автоматизации развития и управления развитием ostis-системы.
\end{itemize}

Под \textit{машиной обработки знаний} будем понимать совокупность интерпретаторов всех \textit{навыков}, составляющих некоторый \textit{решатель задач}. С учетом многоагентного подхода к обработке информации, используемого в рамках Технологии OSTIS, \textit{машина обработки знаний} представляет собой \textit{sc-агент} (чаще всего -- \textit{неатомарный sc-агент}), в состав которого входят более простые sc-агенты, обеспечивающие интерпретацию соответствующего множества \textit{методов}. Таким образом, \textit{машина обработки знаний} в общем случае представляет собой иерархическую систему \textit{sc-агентов}.

\begin{SCn}
\scnheader{машина обработки знаний}
\scnsubset{sc-агент}
\end{SCn}

Рассмотрим классификацию решателей задач ostis-систем по различным признакам.

Классификация решателей задач ostis-систем по типу соответствующей ostis-системы:

\begin{SCn}
\scnheader{решатель задач ostis-системы}
\scnhaselement{Решатель задач Метасистемы IMS.ostis}
\scnsuperset{решатель задач вспомогательной ostis-системы}
\begin{scnindent}
\scnsuperset{решатель задач интерфейса компьютерной системы}
\begin{scnindent}
\begin{scnrelfromset}{разбиение}
	\scnitem{решатель задач пользовательского интерфейса компьютерной системы}
	\scnitem{решатель задач интерфейса компьютерной системы с другими компьютерными системами}
	\scnitem{решатель задач интерфейса компьютерной системы с окружающей средой}
\end{scnrelfromset}
\end{scnindent}
\scnsuperset{решатель задач ostis-подсистемы поддержки проектирования компонентов определенного класса}
\begin{scnindent}
\scnsuperset{решатель задач ostis-подсистемы поддержки проектирования баз знаний}
\begin{scnindent}
\scnsuperset{решатель задач повышения качества базы знаний}
\begin{scnindent}
\scnsuperset{решатель задач верификации базы знаний}
\begin{scnindent}
\scnsuperset{решатель задач поиска и устранения некорректностей в базе знаний}
\scnsuperset{решатель задач поиска и устранения неполноты}
\end{scnindent}
\scnsuperset{решатель задач оптимизации структуры базы знаний}
\scnsuperset{решатель задач выявления и устранения информационного мусора}
\end{scnindent}
\end{scnindent}
\scnsuperset{решатель задач ostis-подсистемы поддержки проектирования решателей задач ostis-систем}
\begin{scnindent}
\begin{scnrelfromset}{разбиение}
	\scnitem{решатель задач ostis-подсистемы поддержки проектирования программ обработки знаний}
	\scnitem{решатель задач ostis-подсистемы поддержки проектирования агентов обработки знаний}
\end{scnrelfromset}
\end{scnindent}
\end{scnindent}
\scnsuperset{решатель задач подсистемы управления проектирования компьютерных систем и их компонентов}
\end{scnindent}
\scnsuperset{решатель задач самостоятельной ostis-системы}
\end{SCn}

Классификация решателей задач ostis-систем по типу интерпретируемой модели решения задач:

\begin{SCn}
\scnheader{решатель задач ostis-системы}
\scnsuperset{решатель задач с использованием хранимых методов}
\begin{scnindent}
\scnidtf{решатель, способный решать задачи тех классов, для которых на данный момент времени известен соответствующий метод решения}
\scnsuperset{решатель задач на основе нейросетевых моделей}
\scnsuperset{решатель задач на основе генетических алгоритмов}
\scnsuperset{решатель задач на основе императивных программ}
\begin{scnindent}
\scnsuperset{решатель задач на основе процедурных программ}
\scnsuperset{решатель задач на основе объектно-ориентированных программ}
\end{scnindent}
\scnsuperset{решатель задач на основе декларативных программ}
\begin{scnindent}
\scnsuperset{решатель задач на основе логических программ}
\scnsuperset{решатель задач на основе функциональных программ}
\end{scnindent}
\end{scnindent}
\scnsuperset{решатель задач в условиях, когда метод решения задач данного класса в текущий момент времени не известен}
\begin{scnindent}
\scnidtf{решатель, реализующий стратегии решения задач, позволяющие породить метод решения задачи, который в текущий момент времени не известен ostis-системе}
\scnidtf{решатель, использующий для решения задач метаметоды, соответствующие более общим классам задач по отношению к заданной}
\scnidtf{решатель задач, позволяющий породить метод, который является частным по отношению какому-либо известному ostis-системе методу и интерпретируется соответствующей машиной обработки знаний}
\scnsuperset{решатель, реализующий стратегию поиска путей решения задачи в глубину}
\scnsuperset{решатель, реализующий стратегию поиска путей решения задачи в ширину}
\scnsuperset{решатель, реализующий стратегию проб и ошибок}
\scnsuperset{решатель, реализующий стратегию разбиения задачи на подзадачи}
\scnsuperset{решатель, реализующий стратегию решения задач по аналогии}
\scnsuperset{решатель, реализующий концепцию интеллектуального пакета программ}
\end{scnindent}
\end{SCn}

Отдельно выделим классификацию машин обработки знаний, которые в общем случае могут соответствовать одним и тем же фрагментам базы знаний, но при этом в совокупности с ними образовывать разные навыки и соответственно разные решатели задач:

\begin{SCn}
\scnheader{машина обработки знаний}
\scnsuperset{машина логического вывода}
\begin{scnindent}
\scnsuperset{машина дедуктивного вывода}
\begin{scnindent}
\scnsuperset{машина прямого дедуктивного вывода}
\scnsuperset{машина обратного дедуктивного вывода}
\end{scnindent}
\scnsuperset{машина индуктивного вывода}
\scnsuperset{машина абдуктивного вывода}
\scnsuperset{машина нечеткого вывода}
\scnsuperset{машина вывода на основе логики умолчаний}
\scnsuperset{машина логического вывода с учетом фактора времени}
\end{scnindent}
\end{SCn}

Классификация решателей задач ostis-систем по типу решаемой задачи (цели решения задачи):

\begin{SCn}
\scnheader{решатель задач ostis-системы}
\scnsuperset{решатель задач информационного поиска}
\begin{scnindent}
\begin{scnrelfromset}{разбиение}
	\scnitem{решатель задач поиска информации, удовлетворяющей заданным критериям}
	\scnitem{решатель задач поиска информации, не удовлетворяющей заданным критериям}
\end{scnrelfromset}
\end{scnindent}
\scnsuperset{решатель явно сформулированных задач}
\begin{scnindent}
\scnidtf{решатель задач, для которых явно сформулирована цель}
\scnsuperset{решатель задач поиска или вычисления значений заданного множества величин}
\scnsuperset{решатель задач установления истинности заданного логического высказывания в рамках заданной формальной теории}
\scnsuperset{решатель задач формирования доказательства заданного высказывания в рамках заданной формальной теории}
\scnsuperset{машина верификации ответа на указанную задачу}
\scnsuperset{машина верификации решения указанной задачи}
\begin{scnindent}
\scnsuperset{машина верификации доказательства заданного высказывания в рамках заданной формальной теории}
\end{scnindent}
\end{scnindent}
\scnsuperset{решатель задач классификации сущностей}
\begin{scnindent}
\scnsuperset{машина соотнесения сущности с одним из заданного множества классов}
\scnsuperset{машина разделения множества сущностей на классы по заданному множеству признаков}
\end{scnindent}
\scnsuperset{решатель задач синтеза информационных конструкций}
\begin{scnindent}
\scnsuperset{решатель задач синтеза естественно-языковых текстов}
\scnsuperset{решатель задач синтеза изображений}
\scnsuperset{решатель задач синтеза сигналов}
\begin{scnindent}
\scnsuperset{решатель задач синтеза речи}
\end{scnindent}
\end{scnindent}
\scnsuperset{решатель задач анализа информационных конструкций}
\begin{scnindent}
\scnsuperset{решатель задач анализа естественно-языковых текстов}
\begin{scnindent}
\scnsuperset{решатель задач понимания естественно-языковых текстов}
\scnsuperset{решатель задач верификации естественно-языковых текстов}
\end{scnindent}
\scnsuperset{решатель задач анализа изображений}
\begin{scnindent}
\scnsuperset{решатель задач сегментации изображений}
\scnsuperset{решатель задач понимания изображений}
\end{scnindent}
\scnsuperset{решатель задач анализа сигналов}
\begin{scnindent}
\scnsuperset{решатель задач анализа речи}
\begin{scnindent}
\scnsuperset{решатель задач понимания речи}
\end{scnindent}
\end{scnindent}
\end{scnindent}
\end{SCn}

\section{Действия, задачи, планы, протоколы и методы, реализуемые ostis-системой, а также внутренние агенты, выполняющие эти действия}

\section{Базовый язык программирования ostis-систем}
\subsection{Синтаксис Базового языка программирования ostis-систем}
\subsection{Денотационная семантика Базового языка программирования ostis-систем}
\subsection{Операционная семантика Базового языка программирования ostis-систем}

%\input{author/references}