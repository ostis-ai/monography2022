\chapauthor{Василевская А.П.\\Орлов М.К.\\Шункевич Д.В.}
\chapter{Логические, продукционные и функциональные модели решения задач в интеллектуальных компьютерных системах нового поколения}
\chapauthortoc{Василевская А.П.\\Орлов М.К.\\Шункевич Д.В.}
\label{chapter_logic_productions}

\abstract{Аннотация к главе.}

\section{Операционная семантика логических языков, используемых ostis-системами}

Логический язык -- искусственный язык логики, предназначенный для воспроизведения логических форм контекстов естественного языка, а также выражения логических законов и способов правильных рассуждений в логических теориях, строящихся в данном языке.

SCL — подъязык SC-кода для записи логических утверждений, логический язык графового типа. Для формализации логических высказываний используется предметная область и онтология логических формул и высказываний. Алфавит языка SCL отдельно не выделяется, так как используется алфавит SC-кода, на котором можно описать любые факты, явления, закономерности, программы и любые другие знания. SC-код является метаязыком как для языка SCL, так и для самого себя, то есть он позволяет описывать смысл формул, записанных на SCL. Многие формальные языки, в отличие от SC, недостаточно богаты, что быть метаязыком для самих себя. Специфика выделения языка SCL в том, что этот тексты этого языка могут обрабатываться особым образом. Рассмотрим операционную семантику языка SCL. 

Язык SCL [2] является аналогом классического логического языка, но в отличие от последнего тексты языка SCL представляют собой однородные семантические сети, являющиеся текстами языка SC. Атомарная формула языка SCL трактуется как множество всех символов некоторого sc-текста. Каждая неатомарная формула языка SCL трактуется как связка, принадлежащая отношению, соответствующему типу неатомарной формулы (конъюнкция, дизъюнкция, импликация, эквиваленция, существование, всеобщность) и связывающая знаки формул, входящих в состав указанной неатомарной формулы.

Абстрактная scl-машина [9, 10, 11] является машиной логического вывода и относится к классу абстрактных sc-машин. Внутренним языком scl-машины является указанный выше графовый логический язык SCL, ее операции соответствуют правилам логического вывода [10,11]. Семейство специализированных абстрактных графодинамических машин обработки знаний является формальным уточнением операционной семантики указанных выше специализированных графовых языков представления знаний, каждому из которых соответствует одна или несколько абстрактных машин. Эти абстрактные машины соответствуют различным моделям решения задач, различным логикам, различным моделям правдоподобных рассуждений [12].

Различные логические подходы позволяют проектировать решатели задач для интеллектуальных систем в разных предметных областях, учитывая их специфику.

Машина обработки знаний каждой конкретной системы во многом зависит от назначения данной системы, множества решаемых задач, предметной областью и другими факторами. Некоторые операции, необходимые в одной предметной области будут избыточными в другой. Например, в системе, решающей задачи по геометрии, химии и другим естественным наукам обоснованным будет использование дедуктивных методов вывода, поскольку решение задач в таких предметных областях основывается только на достоверных правилах. В системах же медицинской диагностики, к примеру, постоянно возникает ситуация, когда диагноз может быть поставлен только с некоторой долей уверенности и абсолютно достоверным ответ на поставленный вопрос быть не может. В связи с этим возникает необходимость использования различных машин обработки знаний в различных системах, при этом состав и возможности машины обработки знаний в конкретной системе определяется не только непосредственно разработчиком, а требует консультаций с экспертами в данной предметной области.

% Начиная отсюда [Коротков, Степанов - Основы формальных логических языков, 2003]
Правильность умозаключений вводится и проверяется совершенно формально, без какой-либо связи с истинностью входящих в него посылок, т.е. исключительно с точки зрения структуры рассуждения. С практической точки зрения самое важное свойство такой формальной правильности рассуждений заключается в следующем: если нам удалось доказать, пользуясь методами формальной логики, правильность рассуждения, и нам известно из опыта, что все используемые посылки истинны, то мы можем быть уверены в истинности заключения. 

Современная логика изучает формальные языки, служащие для выражения логических рассуждений. Используемые для этой цели математические методы пригодны для изучения и значительно более широкого класса формальных языков.
% И до сюда

Агенты логического вывода. К данному классу относятся агенты, предназначенные для генерации новых знаний на основе некоторых логических утверждений. Количество и разнообразие таких агентов зависит от типологии
логических утверждений, которые предполагается использовать в прикладной интеллектуальной системе.

Классификация методов решения задач в интеллектуальных системах.

Проблема автоматического решения задач достаточно давно рассматривается в работах по искусственному интеллекту. Приведем краткую классификацию существующих методов решения задач, рассмотренных в литературе:
\begin{itemize}
	\item{\textbf{Классический дедуктивный вывод.} Классический дедуктивный вывод является наиболее популярным при построении автоматических решателей задач, так как всегда дает достоверный результат. Дедуктивный вывод включает в себя прямой и обратный и логический вывод (принцип резолюции, процедуру Эрбрана и др.) [Вагин и др., 2008], все виды силлогизмов [Малыхина, 2002] и т.д. Основной проблемой
	дедуктивного вывода является невозможность его использования в ряде случаев, когда отсутствуют
	достоверные правила вывода.}
	\item{\textbf{Индуктивный вывод.} Индуктивный вывод предоставляет возможность в процессе решения использовать различные предположения, что делает его удобным для использования в слабоформализованных и
	трудноформализуемых предметных областях, например при построении систем медицинской диагностики. Подробно принципы индуктивного вывода рассмотрены в [Кулик, 2001], [Пойа, 1975].}
	\item{\textbf{Абдуктивный вывод.} Под абдуктивным выводом в искусственном интеллекте, как правило, понимается вывод наилучшего абдуктивного объяснения, т.е. объяснения некоторого события, ставшего
	неожиданным для системы. Причем «наилучшим»	считается такое объяснение, которое удовлетворяет специальным критериям, определяемым в зависимости от решаемой задачи и используемой	формализации. Абдуктивный вывод подробно рассматривается в [Вагин и др., 2008].}
	\item{\textbf{Нечеткие логики.} Теория нечетких множеств и, соответственно, нечетких логик, также применяется в системах, связанных с трудноформализуемыми предметными областями. Подробнее теория нечетких логик рассматривается в [Поспелов, 1989], [Батыршин, 2001], [Деменков, 2001] и других изданиях.}
	\item{\textbf{Логика умолчаний.} Логика умолчаний применяется, в том числе, для того, чтобы оптимизировать процесс рассуждений,	дополняя процесс достоверного вывода вероятностными  редположениями в тех случаях, когда вероятность ошибки крайне мала. Подробнее логика умолчаний рассмотрена в статье [RRIAI, 2012].}
	\item{\textbf{Темпоральная логика.} Применение темпоральной логики является очень актуальным для нестатичных предметных областей, в которых истинность того или иного утверждения меняется со временем, что существенно влияет на ход решения какой-либо задачи. Следует отметить, что используемый в 		данной работе язык представления знаний предоставляет все необходимые возможности для описания таких динамических предметных областей. Более подробно темпоральная логика рассмотрена в работе [Еремеев, 1997]}
\end{itemize}

Данная работа не ставит своей целью разработку нового метода решения задач, нового класса логик или отрицание существующих достижений в данной области. Целью работы является разработка технологии, позволяющей интегрировать любые модели решения задач и принципы логического вывода для решения задач в интеллектуальных системах на основе общей формальной модели. Для того, чтобы использовать какую-либо новую или существующую модель, необходимо привести ее предлагаемому в данной работе формализму, что позволить интегрировать и синхронизировать ее с уже имеющимися в соответствующей библиотеке совместимых компонентов.

Операции логического вывода:
\begin{itemize}
	\item{генерация связки отношения на основании логического утверждения;}
	\item{генерация знаний на основании определения (эквиваленции);}
	\item{генерация определения на основании двух импликаций;}
	\item{получение значения некоторой продукции;}
	\item{вывод обобщенного высказывания;}
	\item{и другие.}
\end{itemize}

Стратегии решения задач путём упрощения задачи (переход от формулировки в терминах предметной области к формулировке на логическом языке):
\begin{itemize}
	\item{операция обобщения;}
	\item{вывод обобщенного логического высказывания;}
	\item{фаззификация;}
	\item{дефаззификация;}
	\item{и другие.}
\end{itemize}

Применение аналогий:
\begin{itemize}
	\item{генерация логического утверждения по аналогии;}
	\item{восстановление решения после применения аналогии;}
	\item{генерация фактов по аналогии;}
	\item{и другие.}
\end{itemize}

Генерация всех возможных следствий (прямой логический вывод):
\begin{itemize}
	\item{поиск всех возможных правил для применения;}
	\item{применение найденных правил;}
	\item{проверка новых фактов на то, что они отсутствуют в базе знаний;}
	\item{дополнение базы знаний сгенерированными фактами;}
	\item{восстановление решения;}
	\item{и другие.}
\end{itemize}

Под операцией логического вывода понимается некоторый sc-агент, который получает на вход теоретико-множественную пару {S,O}, где S - логическое утверждение произвольной конфигурации O - совокупность объектов, в семантической окрестности которых необходимо применить утверждение S. Целью такого агента является генерация в памяти новых знаний на основании уже имеющихся, т.е. по сути, применение утверждения S. Указанный процесс поиска ответа можно разделить на следующие этапы:

\begin{itemize}
	\item{\textbf{Этап работы поисковых операций.} Вне зависимости от типа поставленного вопроса всегда имеется вероятность того, что данная задача уже была решена системой ранее или системе уже 	откуда-либо известен ответ на поставленный вопрос. На данном этапе работу осуществляет коллектив поисковых операций, каждая из которых, как правило, соответствует некоторому классу	решаемых задач. Если ответ найден, подсистема обработки знаний прекращает свою работу. В противном случае происходит переход на следующий этап решения.}
	\item{\textbf{Этап применения стратегий решения задач.}	На данном этапе осуществляется выбор между
	различными стратегиями решения задач, и, при необходимости, параллельный запуск различных стратегий. Целью работы каждой из стратегий является получение набора пар, связывающих некоторое множество объектов и логическое утверждение из базы знаний, которое справедливо для классов, которым принадлежат эти объекты в рамках некоторой теории. Впоследствии при	рассмотрении каждого утверждения осуществляется	попытка применить его в рамках некоторой семантической окрестности рассматриваемых объектов, для чего осуществляется переход на следующий этап решения.}
	\item{\textbf{Этап применения правил логического вывода.} На данном этапе происходит попытка применения утверждения, полученного на предыдущем шаге, с целью генерации в системе	новых знаний. Если такое применение справедливо (например, посылка истинна) и имеет смысл (в результате применения будут сгенерированы новые знания), то осуществляется генерация новых знаний на основе одного из правил логического вывода. При этом применение происходит в контексте объекта, рассматриваемого на предыдущем этапе (в общем случае – ряда объектов). Если в данном контексте вывод на основе данного утверждения	невозможен или нецелесообразен, решение возвращается на предыдущий этап. В случае успешного применения утверждения происходит переход к следующему этапу решения.}
	\item{\textbf{Этап верификации и оптимизации сгенерированных знаний и сборки мусора.} На данном этапе происходит интерпретация арифметических отношений, сгенерированных в процессе решения на предыдущем этапе, то есть попытка вычисления недостающих значений компонентов связок арифметических отношений
	(например, сложение величин и произведение величин) на основе имеющихся значений. Если вычислить все недостающие значения не представляется возможным, то все знания, сгенерированные на предыдущем этапе,
	уничтожаются и решение переходит на этап применения стратегий. В таком случае применение логического вывода для рассматриваемого на предыдущем шаге утверждения считается не	целесообразным. Также на данном этапе происходит устранение синонимии, если таковая появилась на предыдущем этапе решения,
	например, сгенерирована связка отношения совпадения между некоторыми объектами. В конечном итоге происходит удаление конструкций, ставших ненужными и по каким-либо причинам не удаленных на предыдущих этапах решения. Если все этапы решения выполнены успешно, то решение возвращается к первому этапу, и в случае, если ответ не получен, процесс повторяется еще раз. Стоит отметить, что в процессе решения один и тот же объект или одно и тоже высказывание могут быть использованы многократно, если это целесообразно. Однако, очевидно, что применение одного и того же утверждения для одного объекта несколько раз не имеет смысла, при условии, что нужные знания из памяти не удаляются в процессе решения какими-либо сторонними операциями. Следует учитывать тот факт, что агенты сборки мусора, устранения синонимии и верификации знаний могут оказаться полезными и необходимыми не только на завершающем этапе работы интеллектуального решателя задач. В этом смысле 4-ый этап является несколько размытым и может быть частично интегрирован с какими-либо
	из предыдущих.}
\end{itemize}

Таким образом, в структуре описываемой модели можно выделить 4 логических уровня, на каждом из которых возможно использование методов параллельной обработки информации. 

\begin{figure}[H]
	\includegraphics[scale=0.8]{author/part3/figures/Modus_ponens.png}
	\caption{Формализация правила вывода Modus ponens}
	\label{fig:modus_ponens}
\end{figure}

\begin{figure}[H]
	\includegraphics[scale=0.8]{author/part3/figures/resolution.png}
	\caption{Формализация правила резолюции}
	\label{fig:modus_ponens}
\end{figure}

\section{Языки продукционного программирования, используемые ostis-системами}
\subsection{Синтаксис языков продукционного программирования, используемых ostis-системами}
\subsection{Денотационная семантика языков продукционного программирования, используемых ostis-системами}
\subsection{Операционная семантика языков продукционного программирования, используемых ostis-системами}

%\input{author/references}