\chapauthor{Самодумкин С.А.\\Шункевич Д.В.}
\chapter{Язык вопросов для ostis-систем}
\chapauthortoc{Самодумкин С.~А.\\Шункевич Д.~В.\\ Ивашенко В.~П.}
\label{chapter_requests}

\abstract{В главе уточнена формальная трактовка понятия вопроса, что позволило задать язык вопросов.}

В процессе эксплуатации интеллектуальной системы одним из ключевых моментов является возможность формулировать информационную потребность пользователями. Одним из способов выражения такой потребности является вопрос. В процессе диалогового общения всегда существует контекст, который определяет дополнительную информацию, способствующую правильному пониманию смысла сообщения. Особенность представления информации в базах знаний ostis-систем упрощает формирование информационной потребности пользователя, так как представленная инфрормация в базах знаний уже структурирована и известны отношения, заданные на определенном понятии, в отношении которого собственно и разрешается вопросно-проблемная ситуация. В работе Аверьянова показно, что вопросно-проблемная ситуация не может быть решена в рамках формальной логики и природа вопроса может быть понятна в системе субъектно-объектных отношений. В связи с тем, что при формировании баз знаний ostis-систем происходит формирование субъектно-объектных отношений в рамках заданной предметной области, тем самым упрощается выражение информационной потребности пользователем средствами SC-кода.   

Целью разработки языка вопросов и последующее его развитие является реализация возможности понимания действий, осуществляемых ostis-системой, при формировании ответа на поставленный вопрос. В процессе формирования вывода на поставленный вопрос возможны следующие варианты:
1) ответ на поставленный вопрос существует в базе знаний и происходит локализация фрагмента базы знаний в контексте выраженной срествами SC-кода информационной потребности пользователя;
2) ответ связан с разрешением некоторой задачной ситуации, которая содержится в контексте вопроса и формирование ответа возлагается на решатель задач.

\section{Синтаксис языка вопросов для ostis-систем}

Язык вопросов относится к семейству совместимых семантических языков – sc-языков и предназначен для формального описания поискового предписания ostis-систем с целью удовлетворения информационной потребности пользователей.

\section{Денотационная семантика языка вопросов для ostis-систем}

Объектами анализа языка вопросов являются типы вопросов и соответствующие классы ответов в соответствии с семантической классификацией (типологией) вопросов.
С целью семантической классификации вопросов будем исходить из того, что любая предметная область (ПО) содержательно состоит из конкретных понятий и отношений. В работе Сулейманов предложено множество конкретных понятий и отношений по определенным признакам разбить на конечное число типов понятий и типов отношений, которые названы семантическими единицами.
В работе Сулейманов выделено множество различных типов понятий, необходимых для формирования вопроса:
SS - \textit{множество главных понятий} - это понятия, относительно которых задан вопрос (в частном случае, может быть только одно главное понятие);
SO - понятие, состоящие в некотором определенном отношении с главным понятием;
Sоп - \textit{обобщенное понятие} (ОП), то есть такое понятие, находящееся по отношению к главному на более высоком уровне иерархии понятий предметной области
SA - \textit{понятие-аргумент};
SP - \textit{понятие-результат}.

Введём типы отношений, необходимых для формирования вопросов.

\begin{SCn}
	\scnheader{конкретное отношение}
	\scnidtf{определённое отношение между понятиями \textit{предметной области} в контексте вопроса}
\end{SCn}

\begin{SCn}
	\scnheader{типовое отношение}
	\scnidtf{\textit{обобщённое отношение}, объединяющее \textit{конкретное отношение} в семейства отношений, отражающих однотипный смысл и раскрывающих определённый признак понятий \textit{предметной области}}
	\begin{scnrelfromset}{декомпозиция}
		\scnitem{типовое отношение СОСТОЯНИЕ}
		\scnitem{типовое отношение ДЕЙСТВИЕ}
		\scnitem{типовое отношение СОСТАВ}
		\scnitem{типовое отношение ВКЛЮЧЕНИЕ}
		\scnitem{типовое отношение ВРЕМЕННОЕ ОТНОШЕНИЕ}
		\scnitem{типовое отношение ПРОСТРАНСТВЕННОЕ ОТНОШЕНИЕ}
		\scnitem{типовое отношение КОЛИЧЕСТВЕННОЕ ОТНОШЕНИЕ}
		\scnitem{типовое отношение КАЧЕСТВЕННОЕ ОТНОШЕНИЕ}
	\end{scnrelfromset}
\end{SCn}

Например, \textit{конкретные отношения} такие, как "играет", "спит", "плавает", объдиняются в семейство \textit{типовое отношение СОСТОЯНИЕ} по признаку выражать состояние понятия (раскрывает признак понятия предметной области - находиться в некотором состоянии).

\begin{SCn}
	\scnheader{составное отношение}
	\scnidtf{устойчивая комбинация двух типовых отшений ДЕЙСТВИЕ: действия, направленного на \textit{понятие-аргумент}, и действия, направленного на \textit{понятие-результат}}
	\scnsuperset{составное отношение ФУНКЦИЯ заданного понятия}
\end{SCn}

Например, составное отношение ФУНКЦИЯ понятия S1: "S1 переводит S2 в S3".

Смысловая типизация вопросов дает возможность противопоставить каждому типу вопроса ограниченный набор допустимых, то есть логически корректных конструкций, передающий правильный смысл вопроса в зависимости от типа вопроса. При этом семантическая типизация вопросов позваляет разбить множество вопросов на семантичекие классы, в каждом из которых требуется раскрытие некоторого однотипного смысла, определенного типом вопроса. 

\subsection{Семантическая классификация вопросов и ответов}
\label{chapter_questions_sec_sem_classification}

\begin{SCn}
	\scnheader{вопрос к ostis-системе}
	\begin{scnrelfromset}{декомпозиция}
		\scnitem{вопрос, требующий явного задания в ответе \textit{ключевого параметра}}
		\begin{scnindent}
			\begin{scnhaselementrolelist}{пример}
				\scnitem{"Назовите состав компилятора?"{}}
			\end{scnhaselementrolelist}
		\end{scnindent}
		\scnitem{вопрос, требующий раскрытия в ответе \textit{типового отношения} одного \textit{главного понятия}}
		\begin{scnindent}
			\begin{scnhaselementrolelist}{пример}
				\scnitem{"Что легче: железо или дерево?"{}}
			\end{scnhaselementrolelist}
		\end{scnindent}
		\scnitem{вопрос, требующий раскрытия в ответе \textit{специального отношения} одного \textit{главного понятия}}
		\begin{scnindent}
			\scntext{пояснение}{Такому классу вопросов соответствуют классы ответов, в которых\textit{ главное понятие} раскрывается через \textit{специальное отношение}.}
			\begin{scnhaselementrolelist}{пример}
				\scnitem{"Какую функцию выполняет компилятор?"{}}
			\end{scnhaselementrolelist}
		\end{scnindent}
		\scnitem{вопрос, требующий раскрытия в ответе произвольной комбинации \textit{типового отношения} и/или \textit{составного отношения} одного \textit{главного понятия}}
		\begin{scnindent}
			\scntext{пояснение}{Такому классу вопросов соответствуют классы ответов, в которых\textit{ главное понятие} раскрывается через \textit{специальное отношение}.}
			\begin{scnhaselementrolelist}{пример}
				\scnitem{"Какую функцию выполняет компилятор?"{}}
			\end{scnhaselementrolelist}
		\end{scnindent}
		\scnitem{вопрос, требующий раскрытия в ответе более чем одного \textit{главного понятия}}
		\begin{scnindent}
			\begin{scnhaselementrolelist}{пример}
				\scnitem{"Докажите теорему"{}}
			\end{scnhaselementrolelist}
		\end{scnindent}
	\end{scnrelfromset}
\end{SCn}

\begin{SCn}
	\scnheader{ответ на вопрос, требующий раскрытия в ответе \textit{типового отношения} одного \textit{главного понятия}}
	\begin{scnrelfromset}{декомпозиция}
		\scnitem{СОСТАВ}
		\scnidtf{класс ответов, в которых понятие S раскрывается через его типовое отношение СОСТАВ с составляющими понятиями P и Q}
		\begin{scnindent}
			\begin{scnhaselementrolelist}{пример}
				\scnitem{НАПИСАТЬ ПРИМЕР}
			\end{scnhaselementrolelist}
		\end{scnindent}
		\scnitem{ВКЛЮЧЕНИЕ}
		\scnidtf{класс ответов, в которых понятие S раскрывается через его типовое отношение ВКЛЮЧЕНИЕ к другому понятию P, содержащего S как часть}
		\begin{scnindent}
			\begin{scnhaselementrolelist}{пример}
				\scnitem{НАПИСАТЬ ПРИМЕР}
			\end{scnhaselementrolelist}
		\end{scnindent}
		\scnitem{СОСТОЯНИЕ}
		\scnidtf{класс ответов, в которых понятие S раскрывается через его типовое отношение СОСТОЯНИЕ}
		\begin{scnindent}
			\begin{scnhaselementrolelist}{пример}
				\scnitem{"S играет"{}}
			\end{scnhaselementrolelist}
		\end{scnindent}
		\scnitem{ДЕЙСТВИЕ}
		\scnidtf{класс ответов, в которых понятие S раскрывается через его типовое отношение ДЕЙСТВИЕ к другому понятию P}
		\begin{scnindent}
			\begin{scnhaselementrolelist}{пример}
				\scnitem{"S перемещает P"{}}
			\end{scnhaselementrolelist}
		\end{scnindent}
		\scnitem{ВРЕМЕННОЕ ОТНОШЕНИЕ}
		\scnidtf{класс ответов, в которых понятие S раскрывается через его типовое отношение ВРЕМЕННОЕ ОТНОШЕНИЕ к другому понятию P по некоторой временной шкале}
		\begin{scnindent}
			\begin{scnhaselementrolelist}{пример}
				\scnitem{"S выполняется раньше P"{}}
			\end{scnhaselementrolelist}
		\end{scnindent}
		\scnitem{ПРОСТРАНСТВЕННОЕ ОТНОШЕНИЕ}
		\scnidtf{класс ответов, в которых понятие S раскрывается через его типовое отношение ПРОСТРАНСТВЕННОЕ ОТНОШЕНИЕ, отражающее его положение в пространстве относительно другого понятия P}
		\begin{scnindent}
			\begin{scnhaselementrolelist}{пример}
				\scnitem{"S выполняется раньше P"{}}
			\end{scnhaselementrolelist}
		\end{scnindent}
		\scnitem{КОЛИЧЕСТВЕННОЕ ОТНОШЕНИЕ}
		\scnidtf{класс ответов, в которых раскрывается типовое отношение КОЛИЧЕСТВЕННОЕ ОТНОШЕНИЕ понятия S к другому понятию P}
		\begin{scnindent}
			\begin{scnhaselementrolelist}{пример}
				\scnitem{"S больше чем P"{}}
			\end{scnhaselementrolelist}
		\end{scnindent}
		\scnitem{КАЧЕСТВЕННОЕ ОТНОШЕНИЕ}
		\scnidtf{класс ответов, в которых раскрывается типовое отношение КАЧЕСТВЕННОЕ ОТНОШЕНИЕ понятия S к другому понятию P}
		\begin{scnindent}
			\begin{scnhaselementrolelist}{пример}
				\scnitem{"S легче чем P"{}}
			\end{scnhaselementrolelist}
		\end{scnindent}
	\end{scnrelfromset}
\end{SCn}

\begin{SCn}
	\scnheader{ответ на вопрос, требующий раскрытия в ответе произвольной комбинации \textit{типового отношения} и/или \textit{составного отношения} одного \textit{главного понятия}}
	\begin{scnrelfromset}{декомпозиция}
		\scnitem{ОПИСАНИЕ}
		\scnidtf{класс ответов, в которых раскрываются произвольные комбинации \textit{типового отношения} и/или \textit{составного отношения} одного \textit{главного понятия} S с другими понятиями P}
		\begin{scnindent}
			\begin{scnhaselementrolelist}{пример}
				\scnitem{"S состоит из P, Q, W. S переводит X и Y и выполняется раньше Z"{}}
			\end{scnhaselementrolelist}
		\end{scnindent}
		\scnitem{ОПРЕДЕЛЕНИЕ}
		\scnidtf{класс ответов, в которых понятие S раскрывается через \textit{обобщающее понятие} и \textit{класс ОПИСАНИЕ}}
		\begin{scnindent}
			\begin{scnhaselementrolelist}{пример}
				\scnitem{"Студент - это человек, который обучается в ВУЗе"{}}
				\scnitem{"Минск - это столица, которая находится в РБ"{}}
			\end{scnhaselementrolelist}
		\end{scnindent}
		\scnitem{ПРИЧИНА}
		\scnidtf{класс ответов, в которых раскрывается условие существования некоторых отношений S с другими понятими P}
		\begin{scnindent}
			\begin{scnhaselementrolelist}{пример}
				\scnitem{"Почему дерево не тонет в воде?"{}}
			\end{scnhaselementrolelist}
		\end{scnindent}
		\scnitem{СЛЕДСТВИЕ}
		\scnidtf{класс ответов, в которых раскрывается следствие от существования некоторых отношений S с другими понятими P}
		\begin{scnindent}
			\begin{scnhaselementrolelist}{пример}
				\scnitem{"Что следует из того, что удельный вес дерева меньше удельного веса воды?"{}}
			\end{scnhaselementrolelist}
		\end{scnindent}
	\end{scnrelfromset}
\end{SCn}

\begin{SCn}
	\scnheader{ответ на вопрос, требующий раскрытия в ответе более чем одного \textit{главного понятия}}
	\scnsuperset{ДЕТАЛИЗАЦИЯ}
	\begin{scnindent}
		\scnidtf{класс ответов, в которых происходит детализация понятий, стоящих в некоторых отношениях с \textit{главным понятием} P}
		\begin{scnhaselementrolelist}{пример}
			\scnitem{"Каким связь существует между институтом и заводом?"{}}
		\end{scnhaselementrolelist}
	\end{scnindent}
\end{SCn}


\section{Операционная семантика языка вопросов для ostis-систем}

Рассмотренные в \textit{\ref{chapter_questions_sec_sem_classification}~\nameref{chapter_questions_sec_sem_classification}} классы вопросов и ответов обуславливают необходимость определения операционной семантики Языка вопросов. Каждому классу вопросов должен соответствовать определённый коллектив sc-агентов, реализующих вывод из базы знаний ostis-системы соответствующих ответов. Следует отметить, что в зависимости от степени наполненности базы знаний ответы могут содержаться в базе знаний либо в текущей версии базы знаний отсутствовать. В случае наличия ответа в базе знаний информационная потребность пользователя реализуется информационно-поисковыми sc-агентами, в противном случае -- в зависимости от классов вопросов реализация вывода ответов осуществляется специализированные sc-агентами, которые в процессе работы дополнительно выполняют вычислительные задачи либо осуществляют синтез на основе логического вывода.

\begin{SCn}
	\scnheader{интерпретатор Языка вопросов}
	\begin{scnrelfromset}{декомпозиция}
		\scnitem{абстрактный sc-агент, решающий задачу поиска ответа на заданный вопрос}
		\begin{scnindent}
			\begin{scnrelfromset}{декомпозиция}
				\scnitem{абстрактный sc-агент поиска семантической окрестности заданного понятия}
				\scnitem{абстрактный sc-агент поиска ответа СОСТАВА для заданного понятия}
				\scnitem{абстрактный sc-агент поиска ответа ВКЛЮЧЕНИЕ заданного понятия как части другого понятия}
				\scnitem{абстрактный sc-агент поиска ответа СОСТОЯНИЕ заданного понятия}
				\scnitem{абстрактный sc-агент поиска ответа ДЕЙСТВИЕ заданного понятия к другому понятию}
				\scnitem{абстрактный sc-агент поиска ответа ВРЕМЕННОЕ ОТНОШЕНИЕ заданных понятий}
				\scnitem{абстрактный sc-агент поиска ответа ПРОСТРАНСТВЕННОЕ ОТНОШЕНИЕ заданных понятий}
				\scnitem{абстрактный sc-агент поиска ответа КОЛИЧЕСТВЕННОГО ОТНОШЕНИЕ заданных понятий}
				\scnitem{абстрактный sc-агент поиска ответа КАЧЕСТВЕННОЕ ОТНОШЕНИЕ заданных понятий}
				\scnitem{абстрактный sc-агент поиска ответа ОПИСАНИЕ заданного понятия с другими понятиями}
				\scnitem{абстрактный sc-агент поиска ответа ОПРЕДЕЛЕНИЕ заданного понятия с другими понятиями}
				\scnitem{абстрактный sc-агент поиска ответа ПРИЧИНА заданного понятия с другими понятиями}
				\scnitem{абстрактный sc-агент поиска ответа СЛЕДСТВИЕ заданного понятия с другими понятиями}
				\scnitem{абстрактный sc-агент поиска ответа ДЕТАЛИЗАЦИЯ понятий, состоящих в отношении с заданным понятием}
			\end{scnrelfromset}
		\end{scnindent}
		\scnitem{абстрактный sc-агент, решающий задачу синтеза ответа на заданный вопрос}
	\end{scnrelfromset}
\end{SCn}

%\input{author/references}