\chapter{Язык вопросов для ostis-систем}
\chapauthortoc{Самодумкин С.~А.\\Зотов Н.~В.\\Шункевич Д.~В.\\ Ивашенко В.~П.}
\label{chapter_requests}

\vspace{-7\baselineskip}

\begin{SCn}
\begin{scnrelfromlist}{авторы}
	\scnitem{Самодумкин С.~А.}
	\scnitem{Шункевич Д.~В.}
	\scnitem{Ивашенко В.~П.}
\end{scnrelfromlist}

\bigskip

\scntext{аннотация}{В главе уточнена формальная трактовка понятия вопроса, что позволило задать \textit{Язык вопросов}.}

\bigskip

\begin{scnrelfromlist}{подраздел}
	\scnitem{\nameref{sec_requests_syntax}~\ref{sec_requests_syntax}}
	\scnitem{\nameref{sec_requests_den_semantics}~\ref{sec_requests_den_semantics}}
	\scnitem{\nameref{sec_requests_op_semantics}~\ref{sec_requests_op_semantics}}
\end{scnrelfromlist}

\bigskip

\begin{scnrelfromlist}{ключевой знак}
	\scnitem{Язык вопросов}
\end{scnrelfromlist}

\bigskip

\begin{scnrelfromlist}{библиографическая ссылка}
	\scnitem{\scncite{Suleimanov2001}}
	\scnitem{\scncite{Suleimanov2014}}
	\scnitem{\scncite{Bukharev1990}}
	\scnitem{\scncite{Kwok2001}}
	\scnitem{\scncite{Emelyanov2007}}
	\scnitem{\scncite{Averyanov1993}}
	\scnitem{\scncite{Finn1976}}
	\scnitem{\scncite{Finn1981}}
	\scnitem{\scncite{Belnap1981}}
	\scnitem{\scncite{Sosnin2007}}
	\scnitem{\scncite{Zaharov2002}}
	\scnitem{\scncite{Hant1978}}
	\scnitem{\scncite{Lyubarsky1990}}
	\scnitem{\scncite{Samodumkin2009}}
	\scnitem{\scncite{Samodumkin2009a}}
\end{scnrelfromlist}
	
\end{SCn}

Одна из ключевых особенностей \textit{интеллектуальной системы} состоит в том, что \textit{пользователь} имеет возможность формулировать свою информационную потребность. Одним из способов выражения такой потребности является \textit{вопрос}. В процессе диалогового общения всегда существует контекст, который определяет дополнительную информацию, способствующую правильному пониманию \textit{смысла} сообщения. Особенность представления информации в \textit{базах знаний} \textit{ostis-систем} упрощает формирование информационной потребности пользователя, так как представленная информация в \textit{базах знаний} уже структурирована и известны отношения, заданные на определенном понятии, в отношении которого разрешается вопросно-проблемная ситуация. В работе \scncite{Averyanov1993} показно, что вопросно-проблемная ситуация не может быть решена в рамках формальной логики и природа вопроса может быть понятна в системе субъектно-объектных отношений. В связи с тем, что при формировании \textit{баз знаний} \textit{ostis-систем} происходит формирование субъектно-объектных отношений в рамках заданной \textit{предметной области}, тем самым упрощается выражение информационной потребности пользователем средствами \textit{SC-кода}.   

Целью разработки \textit{Языка вопросов} и последующее его развитие является реализация возможности понимания действий, осуществляемых \textit{ostis-системой}, при формировании ответа на поставленный \textit{вопрос}. В процессе формирования вывода на поставленный \textit{вопрос} возможны следующие варианты:
1) ответ на поставленный вопрос существует в \textit{базе знаний} и происходит локализация \textit{фрагмента базы знаний} в контексте выраженной средствами \textit{SC-кода} информационной потребности \textit{пользователя};
2) ответ связан с разрешением некоторой задачной ситуации, которая содержится в контексте \textit{вопроса} и формирование \textit{ответа на вопрос} возлагается на \textit{решатель задач}.

\section{Синтаксис языка вопросов для ostis-систем}
\label{sec_requests_syntax}

\textit{Язык вопросов} относится к семейству семантических совместимых языков – \textit{sc-языков} и предназначен для формального описания поискового предписания \textit{ostis-систем} с целью удовлетворения информационной потребности \textit{пользователей}.

\section{Денотационная семантика языка вопросов для ostis-систем}
\label{sec_requests_den_semantics}

\textit{Денотационная семантика Языка вопросов} включает классы \textit{вопросов} и соответствующие классы \textit{ответов}, необходимые для спецификации формулировок \textit{вопросов} и \textit{ответов} на них. В \textit{Семантическую классификацию вопросов} и \textit{Семантическую классификацию ответов} \textit{Языка вопросов} заложена идея, представленная в работе \scncite{Suleimanov2001}.

Базовыми понятиями в \textit{Языке вопросов} являются:
\begin{textitemize}
	\item \textit{основной знак в рамках заданного вопроса} - \textit{знак}, относительно которого задан вопрос;
	\item \textit{неосновной знак в рамках заданного вопроса} - \textit{знак}, стоящий в некотором отношении с \textit{основным знаком в рамках заданного вопроса};
    \item \textit{параметр вопроса\scnrolesign};
	\item \textit{ответ на вопрос*}.
\end{textitemize}

Введём классы отношений, необходимых для формирования вопросов.

\begin{SCn}
\scnheader{отношение в рамках заданного вопроса\scnsupergroupsign}
\scnidtf{определённое отношение между знаками \textit{предметной области} в контексте \textit{вопроса}}
\end{SCn}

\begin{SCn}
\scnheader{базовое отношение в рамках заданного вопроса}
\scnidtf{\textit{класс отношений}, объединяющий \textit{отношения в заданном вопросе}, отражающие однотипный \textit{смысл} и раскрывающие определённый признак \textit{знаков} \textit{предметной области}}
\begin{scnrelfromset}{декомпозиция}
	\scnitem{отношение состояния}
	\scnitem{отношение действия}
	\scnitem{отношение состава}
	\scnitem{теоретико-множественное отношение}
	\scnitem{темпоральное отношение}
	\scnitem{пространственное отношение}
	\scnitem{количественное отношение}
	\scnitem{качественное отношение}
\end{scnrelfromset}
\end{SCn}

Например, \textit{отношения в рамках заданного вопроса} такие, как \scnqqi{играет*}, \scnqqi{спит*}, \scnqqi{плавает*}, объединяются в класс \textit{отношений состояния} по признаку выражать состояние знака (раскрывает признак знака \textit{предметной области} --- \scnqqi{находиться в некотором состоянии}).

\begin{SCn}
\scnheader{составное отношение в рамках заданного вопроса}
\scnidtf{устойчивая комбинация двух \textit{отношений действия}: действия, направленного на \textit{параметр вопроса\scnrolesign}, и действия, направленного на \textit{ответ на вопрос*}}
\scnsuperset{составное отношение функции}
\end{SCn}

Например, \textit{составное отношение функции} знака \textit{S1} --- \scnqqi{Нефтеперерабатывающий завод перерабатывает нефть в нефтепродукты}.

\subsection{Семантическая классификация вопросов и ответов}
\label{chapter_questions_sec_sem_classification}

Смысловая типизация \textit{вопросов} дает возможность противопоставить каждому типу вопроса ограниченный набор допустимых, то есть \textit{семантически корректных информационных конструкций}, передающий правильный \textit{смысл} \textit{вопроса} в зависимости от класса \textit{вопроса}. При этом \textit{Семантическая классификация вопросов} позволяет разбить множество \textit{вопросов} на классы, в каждом из которых требуется раскрытие некоторого однотипного \textit{смысла}, заданного классом этого \textit{вопроса}. 

\begin{SCn}
\scnheader{вопрос}
\scnidtf{запрос}
\scnidtf{непроцедурная формулировка задачи на поиск (в текущем состоянии базы знаний) или на генерацию знания, удовлетворяющего заданным требованиям}
\scnidtf{каким способом}
\scnidtf{запрос метода (способа) решения заданного (указываемого) \textit{класса задач} либо \textit{плана решения} конкретной указываемой \textit{задачи}}
\scnsubset{задача}

\scnheader{ответ на вопрос}
\scnidtf{ответ на запрос}
\scnidtf{результат запроса}
\scnidtf{результат решения задачи на поиск или генерацию знания, удовлетворяющий заданным требованиям}
\scnidtf{семантическая окрестность \textit{основного знака}, состав которой удовлетворяет информационную потребность пользователя}
\scnidtf{знание в базе знаний ostis-системы, которое удовлетворяет информационную потребность пользователя}
\scnsubset{знание}
\end{SCn}

\begin{SCn}
\scnidtf{задача, направленная на удовлетворение информационной потребности некоторого субъекта-заказчика}
\begin{scnrelfromset}{декомпозиция}
	\scnitem{вопрос, требующий вывода семантической окрестности \textit{основного знака}}
	\begin{scnindent}
		\begin{scnhaselementrolelist}{пример}
			\scnitem{Вопрос. Что такое \textit{Город Минск}}
		\end{scnhaselementrolelist}
	\end{scnindent}
	\scnitem{вопрос, требующий раскрытия в ответе \textit{базового отношения} \textit{основного знака}}
%	\begin{scnindent}
%		\begin{scnhaselementrolelist}{пример}
%			\scnitem{Вопрос. Что легче: железо или дерево}
%		\end{scnhaselementrolelist}
%	\end{scnindent}
	\scnitem{вопрос, требующий раскрытия в ответе \textit{составного отношения} \textit{основного знака}}
	\begin{scnindent}
		\scntext{пояснение}{Такому классу \textit{вопросов} соответствуют классы \textit{ответов}, в которых \textit{главный знак} раскрывается через \textit{составное отношение}.}
		\begin{scnhaselementrolelist}{пример}
			\scnitem{Вопрос. Какие принципы компонентного проектирования в интеллектуальных компьютерных системах нового поколения}
		\end{scnhaselementrolelist}
	\end{scnindent}
	\scnitem{вопрос, требующий раскрытия в ответе произвольной комбинации \textit{базового отношения} и/или \textit{составного отношения} \textit{основного знака}}
	\begin{scnindent}
		\begin{scnhaselementrolelist}{пример}
			\scnitem{Вопрос. Как определяется уровень интеллекта кибернетической системы}
		\end{scnhaselementrolelist}
	\end{scnindent}
	\scnitem{вопрос, требующий раскрытия в ответе более чем одного \textit{основного знака}}
	\begin{scnindent}
		\begin{scnhaselementrolelist}{пример}
			\scnitem{Вопрос. Докажите теорему Пифагора}
		\end{scnhaselementrolelist}
	\end{scnindent}
\end{scnrelfromset}
\end{SCn}

\begin{SCn}
\scnheader{вопрос, требующий раскрытия в ответе \textit{базового отношения} \textit{основного знака}}
\begin{scnrelfromset}{декомпозиция}
	\scnitem{вопрос, требующий раскрытия в ответе \textit{отношения состава} \textit{основного знака}}
	\begin{scnindent}
		\scnidtf{класс вопросов, в ответах на которые \textit{основной знак} \textit{S} раскрывается через его \textit{отношение состава} в связке с его составляющими знаками \textit{P} и \textit{Q}}
		\begin{scnhaselementrolelist}{пример}
			\scnitem{Вопрос. Какие административные районы входят в состав Города Витебск}
			\begin{scnindent}
				\scneqimage[30em]{author/part3/figures/question\_about\_vitebsk\_regions.png}
				\scnrelfrom{ответ на вопрос}{\{Железнодорожный район Города Витебск, Октябрьский район Города Витебск, Первомайский район Города Витебск\}}
				\begin{scnindent}
					\scneqimage[30em]{author/part3/figures/question\_about\_vitebsk\_regions\_answer.png}
				\end{scnindent}
			\end{scnindent}
		\end{scnhaselementrolelist}
	\end{scnindent}
	\scnitem{вопрос, требующий раскрытия в ответе \textit{теоретико-множественного отношения} \textit{основного знака}}
	\begin{scnindent}
		\scnidtf{класс вопросов, в ответах на которые \textit{основной знак} \textit{S} раскрывается через его \textit{теоретико-множественное отношение} в связке с другим знаком \textit{P}, содержащего \textit{S} как часть}
		\begin{scnhaselementrolelist}{пример}
			\scnitem{Вопрос. Частью какой области является Смолевичский район}
			\begin{scnindent}
				\scneqimage[30em]{author/part3/figures/question\_about\_smolevichi\_inclusion.png}
				\scnrelfrom{ответ на вопрос}{\{Смолевичский район является частью Минской области\}}
			\end{scnindent}
		\end{scnhaselementrolelist}
	\end{scnindent}
	\scnitem{вопрос, требующий раскрытия в ответе \textit{отношения состояния} \textit{основного знака}}
	\begin{scnindent}
		\scnidtf{класс вопросов, в ответах на которые \textit{основной знак} \textit{S} раскрывается через его \textit{отношение состояния}}
		\begin{scnhaselementrolelist}{пример}
			\scnitem{Вопрос. Какие города современной территории Республики Беларусь имели Магдебургское право}
			\begin{scnindent}
				\scneqimage[30em]{author/part3/figures/question\_about\_minsk\_district\_town\_with\_mag\_act.png}
				\scnrelfrom{ответ на вопрос}{\{Волковыск, Гродно, Мозырь и другие имели Магдебургское право\}}
			\end{scnindent}
		\end{scnhaselementrolelist}
	\end{scnindent}
	\scnitem{вопрос, требующий раскрытия в ответе \textit{отношения действия} \textit{основного знака}}
	\begin{scnindent}
		\scnidtf{класс вопросов, в ответах на которые \textit{основной знак} \textit{S} раскрывается через его \textit{отношение действия} в связке с другим знаком \textit{P}}
	\end{scnindent}
	\scnitem{вопрос, требующий раскрытия в ответе \textit{темпорального отношения} \textit{основного знака}}
	\begin{scnindent}
		\scnidtf{класс вопросов, в ответах на которые \textit{основной знак} \textit{S} раскрывается через его \textit{темпоральное отношение} в связке с другим знаком \textit{P} по некоторой временной шкале}
		\begin{scnhaselementrolelist}{пример}
			\scnitem{Вопрос. Какое событие произошло раньше: Первый раздел Речи Посполитой или Бородинское сражение}
			\begin{scnindent}
				\scneqimage[30em]{author/part3/figures/question\_about\_events.png}
				\scnrelfrom{ответ на вопрос}{\{Первый раздел Речи Посполитой был раньше Бородинского сражения\}}
				\begin{scnindent}
					\scneqimage[30em]{author/part3/figures/question\_about\_event\_answer.png}
				\end{scnindent}
			\end{scnindent}
		\end{scnhaselementrolelist}
	\end{scnindent}
	\scnitem{вопрос, требующий раскрытия в ответе \textit{пространственного отношения} \textit{основного знака}}
	\begin{scnindent}
		\scnidtf{класс вопросов, в ответах на которые \textit{основной знак} \textit{S} раскрывается через \textit{пространственное отношение}, отражающее его положение в пространстве относительно другого знака \textit{P}}
	\end{scnindent}
	\scnitem{вопрос, требующий раскрытия в ответе \textit{количественного отношения} \textit{основного знака}}
	\begin{scnindent}
		\scnidtf{класс вопросов, в ответах на которые раскрывается \textit{количественное отношение} \textit{основного знака}}
		\begin{scnhaselementrolelist}{пример}
			\scnitem{Вопрос. Какова высота Горы Дзержинская}
			\begin{scnindent}
				\scneqimage[30em]{author/part3/figures/question\_about\_mountain\_length.png}
				\scnrelfrom{ответ на вопрос}{\{Высота Горы Дзержинская - 345 м\}}
			\end{scnindent}
		\end{scnhaselementrolelist}
	\end{scnindent}
	\scnitem{вопрос, требующий раскрытия в ответе \textit{качественного отношения} \textit{основного знака}}
	\begin{scnindent}
		\scnidtf{класс вопросов, в ответах на которые раскрывается \textit{качественное отношение} \textit{основного знака} \textit{S} в связке с другим знаком \textit{P}}
		\begin{scnhaselementrolelist}{пример}
			\scnitem{Вопрос. Территория какой административной области больше: Минской или Брестской}
			\begin{scnindent}
				\scneqimage[30em]{author/part3/figures/question\_about\_district\_squares.png}
				\scnrelfrom{ответ на вопрос}{\{Территория Минской области больше Брестской\}}
				\begin{scnindent}
					\scneqimage[30em]{author/part3/figures/question\_about\_district\_squares\_answer.png}
				\end{scnindent}
			\end{scnindent}
		\end{scnhaselementrolelist}
	\end{scnindent}
е\end{scnrelfromset}
\end{SCn}

\begin{SCn}
\scnheader{вопрос, требующий раскрытия в ответе произвольной комбинации \textit{базового отношения} и/или \textit{составного отношения} \textit{основного знака}}
\begin{scnrelfromset}{декомпозиция}
	\scnitem{вопрос, требующий раскрытия в ответе произвольной комбинации \textit{составного отношения описания} \textit{основного знака}}
	\begin{scnindent}
		\scnidtf{класс вопросов, в ответах на которые раскрываются произвольные комбинации \textit{базового отношения} и/или \textit{составного отношения} \textit{основного знака} \textit{S} в связке с другими знаками}
		\begin{scnhaselementrolelist}{пример}
			\scnitem{\{S состоит из P, Q, W. S переводит X и Y и выполняется раньше Z\}}
			\begin{scnindent}
				\scnrelto{ответ на вопрос}{Вопрос. Что такое S}
			\end{scnindent}
		\end{scnhaselementrolelist}
	\end{scnindent}
	\scnitem{вопрос, требующий раскрытия в ответе произвольной комбинации \textit{составного отношения определения} \textit{основного знака}}
	\begin{scnindent}
		\scnidtf{класс ответов, в которых \textit{основной знак} \textit{S} раскрывается через \textit{первостепенное понятие} и его \textit{описание}}
		\begin{scnhaselementrolelist}{пример}
			\scnitem{\{Минск - это столица, которая находится в РБ\}}
			\begin{scnindent}
				\scnrelto{ответ на вопросы}{Вопрос. Как определяется город Минск}
			\end{scnindent}
		\end{scnhaselementrolelist}
	\end{scnindent}
	\scnitem{вопрос, требующий раскрытия в ответе произвольной комбинации \textit{составного отношения причины} \textit{основного знака}}
	\begin{scnindent}
		\scnidtf{класс вопросов, в ответах на которые раскрывается условие существования некоторых отношений \textit{основного знака} \textit{S} в связке с другими знаками}
		\begin{scnhaselementrolelist}{пример}
			\scnitem{Вопрос. Почему время в пути от города Минска до города Борисова меньше чем время в пути от города Минска до города Орша}
			\begin{scnindent}
				\scnrelfrom{ответ на вопрос}{\{Время в пути от города Минска до города Борисова меньше чем время в пути от города Минска до города Орша, потому что расстояние от города Минска меньше до города Борисова, чем до города Орша\}}
			\end{scnindent}
		\end{scnhaselementrolelist}
	\end{scnindent}
	\scnitem{вопрос, требующий раскрытия в ответе произвольной комбинации \textit{составного отношения следствия} \textit{основного знака}}
	\scnidtf{класс вопросов, в ответах на которые раскрывается следствие от существования некоторых отношений \textit{основного знака} \textit{S} в связке с другими знаками}
	\begin{scnindent}
		\begin{scnhaselementrolelist}{пример}
			\scnitem{Вопрос. Что следует из того, что расстояние от города Минска до города Борисова меньше расстояния от города Минска до города Орша}
			\begin{scnindent}
				\scnrelfrom{ответ на вопрос}{\{Расстояние от города Минска до города Борисова меньше расстояния от города Минска до города Орша, поэтому от города Минска до города Борисова время в пути меньше чем до города Орша\}}
			\end{scnindent}
		\end{scnhaselementrolelist}
	\end{scnindent}
\end{scnrelfromset}
\end{SCn}

\begin{SCn}
\scnheader{вопрос, требующий раскрытия в ответе более чем одного \textit{основного знака}}
\scnsuperset{вопрос, требующий раскрытия в ответе \textit{отношение детализации} знаков, стоящих в некоторых отношениях с \textit{основным знаком}}
\begin{scnindent}
	\scnidtf{класс вопросов, в ответах на которые происходит детализация знаков, стоящих в некоторых отношениях с \textit{основным знаком} \textit{S}}
	\begin{scnhaselementrolelist}{пример}
		\scnitem{Вопрос. Какая связь водной сети существует  между городом Минск и городом Свтлогорск}
		\begin{scnindent}
			\scnrelto{ответ на вопрос}{\{Город Минск расположен на реке Свислочь, которая впадает в реку Березина, протекающую через город Светлогорск\}}
		\end{scnindent}
	\end{scnhaselementrolelist}
\end{scnindent}
\end{SCn}

\section{Операционная семантика языка вопросов для ostis-систем}
\label{sec_requests_op_semantics}

Рассмотренные в \textit{\ref{chapter_questions_sec_sem_classification}~\nameref{chapter_questions_sec_sem_classification}} классы вопросов и ответов обуславливают необходимость определения операционной семантики Языка вопросов. Каждому классу вопросов должен соответствовать определённый коллектив sc-агентов, реализующих вывод из базы знаний ostis-системы соответствующих ответов. Следует отметить, что в зависимости от степени наполненности базы знаний ответы могут содержаться в базе знаний либо в текущей версии базы знаний отсутствовать. В случае наличия ответа в базе знаний информационная потребность пользователя реализуется информационно-поисковыми sc-агентами, в противном случае -- в зависимости от классов вопросов реализация вывода ответов осуществляется специализированные sc-агентами, которые в процессе работы дополнительно выполняют вычислительные задачи либо осуществляют синтез на основе логического вывода.

\begin{SCn}
	\scnheader{интерпретатор Языка вопросов}
	\begin{scnrelfromset}{декомпозиция}
		\scnitem{абстрактный sc-агент, решающий задачу поиска ответа на заданный вопрос}
		\begin{scnindent}
			\begin{scnrelfromset}{декомпозиция}
				\scnitem{абстрактный sc-агент поиска семантической окрестности \textit{основного знака}}
				\scnitem{абстрактный sc-агент поиска ответа на вопрос, требующий раскрытия в ответе \textit{отношения состава} для \textit{основного знака}}
				\scnitem{абстрактный sc-агент поиска ответа на вопрос, требующий раскрытия в ответе \textit{теоретико-множественного отношения} для \textit{основного знака}}
				\scnitem{абстрактный sc-агент поиска ответа на вопрос, требующий раскрытия в ответе \textit{отношения состояния} для \textit{основного знака}}	
				\scnitem{абстрактный sc-агент поиска ответа на вопрос, требующий раскрытия в ответе \textit{отношения действия} для \textit{основного знака}}	
				\scnitem{абстрактный sc-агент поиска ответа на вопрос, требующий раскрытия в ответе \textit{темпорального отношения} для \textit{основного знака}}
				\scnitem{абстрактный sc-агент поиска ответа на вопрос, требующий раскрытия в ответе \textit{пространственного отношения} для \textit{основного знака}}
				\scnitem{абстрактный sc-агент поиска ответа на вопрос, требующий раскрытия в ответе \textit{количественного отношения} для \textit{основного знака}}
				\scnitem{абстрактный sc-агент поиска ответа на вопрос, требующий раскрытия в ответе \textit{качественного отношения} для \textit{основного знака}}
				\scnitem{абстрактный sc-агент поиска ответа на вопрос, требующий раскрытия в ответе \textit{отношения описания} для \textit{основного знака}}
				\scnitem{абстрактный sc-агент поиска ответа на вопрос, требующий раскрытия в ответе \textit{отношения определения} для \textit{основного знака}}
				\scnitem{абстрактный sc-агент поиска ответа на вопрос, требующий раскрытия в ответе \textit{отношения причины} для \textit{основного знака}}
				\scnitem{абстрактный sc-агент поиска ответа на вопрос, требующий раскрытия в ответе \textit{отношения следствия} для \textit{основного знака}}
				\scnitem{абстрактный sc-агент поиска ответа на вопрос, требующий раскрытия в ответе \textit{отношения детализации} для \textit{основного знака}}
			\end{scnrelfromset}
		\end{scnindent}
		\scnitem{абстрактный sc-агент, решающий задачу синтеза ответа на заданный вопрос}
	\end{scnrelfromset}
\end{SCn}

%\input{author/references}