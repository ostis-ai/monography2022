\chapter{Язык вопросов для ostis-систем}
\chapauthortoc{Самодумкин С.~А.\\Зотов Н.~В.\\Шункевич Д.~В.\\ Ивашенко В.~П.}
\label{chapter_requests}

\vspace{-7\baselineskip}

\begin{SCn}
\begin{scnrelfromlist}{авторы}
	\scnitem{Самодумкин С.~А.}
	\scnitem{Зотов Н.~В.}
	\scnitem{Шункевич Д.~В.}
	\scnitem{Ивашенко В.~П.}
\end{scnrelfromlist}

\bigskip

\scntext{аннотация}{Возможности \textit{баз знаний} \textit{ostis-систем} позволяют не только представлять и структурировать знания об окружающем мире, но и быстро получать и формировать эти знания о нём, тем самым удовлетворяя информационную потребность пользователя. В данной главе уточнена формальная спецификация \textit{Языка вопросов для ostis-систем}, позволяющая описывать и интерпретировать любые классы \textit{вопросов} \textit{пользователей ostis-систем}.}

\bigskip

\begin{scnrelfromlist}{подраздел}
	\scnitem{\ref{sec_requests_syntax}~\nameref{sec_requests_syntax}}
	\scnitem{\ref{sec_requests_den_semantics}~\nameref{sec_requests_den_semantics}}
	\scnitem{\ref{sec_requests_op_semantics}~\nameref{sec_requests_op_semantics}}
\end{scnrelfromlist}

\bigskip

\begin{scnrelfromlist}{ключевое понятие}
	\scnitem{вопрос}
	\scnitem{ответ на вопрос}
	\scnitem{знак в рамках заданного вопроса}
	\scnitem{основной знак в рамках заданного вопроса}
	\scnitem{неосновной знак в рамках заданного вопроса}
	\scnitem{отношение в рамках заданного вопроса}
	\scnitem{базовое отношение в рамках заданного вопроса}
	\scnitem{интерпретатор Языка вопросов для ostis-систем}
\end{scnrelfromlist}

\bigskip

\begin{scnrelfromlist}{ключевой знак}
	\scnitem{Язык вопросов для ostis-систем}
	\scnitem{Синтаксис Языка вопросов для ostis-систем}
	\scnitem{Денотационная семантика Языка вопросов для ostis-систем}
	\scnitem{Семантическая классификация вопросов}
	\scnitem{Операционная семантика Языка вопросов для ostis-систем}
\end{scnrelfromlist}

\bigskip

\begin{scnrelfromlist}{библиографическая ссылка}
	\scnitem{\scncite{Suleimanov2001}}
	\scnitem{\scncite{Suleimanov2014}}
	\scnitem{\scncite{Bukharev1990}}
	\scnitem{\scncite{Kwok2001}}
	\scnitem{\scncite{Emelyanov2007}}
	\scnitem{\scncite{Averyanov1993}}
	\scnitem{\scncite{Finn1976}}
	\scnitem{\scncite{Finn1981}}
	\scnitem{\scncite{Belnap1981}}
	\scnitem{\scncite{Sosnin2007}}
	\scnitem{\scncite{Zaharov2002}}
	\scnitem{\scncite{Hant1978}}
	\scnitem{\scncite{Lyubarsky1990}}
	\scnitem{\scncite{Samodumkin2009}}
	\scnitem{\scncite{Samodumkin2009a}}
\end{scnrelfromlist}
	
\end{SCn}

\section*{Введение в Главу \ref{chapter_requests}}

Одна из ключевых особенностей \textit{интеллектуальной системы} состоит в том, что \textit{пользователь} имеет возможность формулировать свою информационную потребность. Cпособом выражения такой потребности является \textit{вопрос}. В процессе общения всегда существует контекст, который определяет дополнительную информацию, способствующую правильному пониманию \textit{смысла} сообщения. Особенность представления информации в \textit{базах знаний} \textit{ostis-систем} упрощает формирование информационной потребности пользователя, так как представленная информация в \textit{базах знаний} уже структурирована и известны отношения, заданные на определенном понятии, в отношении которого разрешается вопросно-проблемная ситуация. В работе \scncite{Averyanov1993} показано, что вопросно-проблемная ситуация не может быть решена в рамках формальной логики и природа вопроса может быть понятна в системе субъектно-объектных отношений. В связи с тем, что при формировании \textit{баз знаний} \textit{ostis-систем} происходит формирование субъектно-объектных отношений в рамках заданной \textit{предметной области}, тем самым упрощается выражение информационной потребности пользователем средствами \textit{SC-кода}.   

Целью разработки \textbf{\textit{Языка вопросов для ostis-систем}} и последующего его развития является реализация возможности понимания действий, осуществляемых \textit{ostis-системой}, при формировании ответа на поставленный \textit{вопрос}. В процессе формирования ответа на поставленный \textit{вопрос} возможны следующие варианты:
\begin{textitemize}
	\item \textit{ответ на} поставленный \textit{вопрос} существует в \textit{базе знаний} и происходит локализация \textit{фрагмента базы знаний} в контексте выраженной средствами \textit{SC-кода} информационной потребности \textit{пользователя};
	\item ответ связан с разрешением некоторой задачной ситуации, которая содержится в контексте \textit{вопроса} и формирование \textit{ответа на вопрос} возлагается на \textit{решатель задач} (см. \ref{chapter_situation_management}~\nameref{chapter_situation_management}).
\end{textitemize}

\begin{SCn}
\scnheader{Язык вопросов для ostis-систем}
\scnidtf{Предлагаемый нами вариант языка для описания вопросов и ответов на них в ostis-системах}
\scniselement{sc-язык}
\scnrelfrom{синтаксис языка}{Синтаксис Языка вопросов для ostis-систем}
\begin{scnindent}
	\scnsubset{Синтаксис SC-кода}
\end{scnindent}
\scnrelfrom{денотационная семантика языка}{Денотационная семантика Языка вопросов для ostis-систем}
\begin{scnindent}
	\scnidtf{Онтология классов знаков и отношений для описания формулировок вопросов на SC-коде}
	\scnsuperset{Семантическая классификация вопросов}
\end{scnindent}
\scnrelfrom{операционная семантика языка}{Операционная семантика Языка вопросов для ostis-систем}
\begin{scnindent}
	\scnidtf{Коллектив sc-агентов вывода ответов на заданные вопросы пользователя ostis-системы}
\end{scnindent}
\end{SCn}

\section{Синтаксис Языка вопросов для ostis-систем}
\label{sec_requests_syntax}

\textit{Язык вопросов для ostis-систем} относится к семейству семантических совместимых языков --- \textit{sc-языков} и предназначен для формального описания поискового предписания \textit{ostis-систем} с целью удовлетворения информационной потребности \textit{пользователя}. Поэтому \textbf{\textit{Синтаксис Языка вопросов для ostis-систем}}, как и \textit{синтаксис} любого другого \textit{sc-языка}, является \textit{Синтаксисом SC-кода}. Такой подход позволяет:
\begin{textitemize}
	\item унифицировать форму представления \textit{вопросов} и \textit{знаний}, с помощью которых строятся ответы на поставленные \textit{вопросы};
	\item использовать минимум средств для интерпретации заданных \textit{вопросов пользователей};
	\item сводить формирование ответов на большую часть заданных \textit{вопросов} к поиску информации в текущем состоянии \textit{базы знаний ostis-системы}.
\end{textitemize}

\section{Денотационная семантика Языка вопросов для ostis-систем}
\label{sec_requests_den_semantics}

\begin{SCn}
\begin{scnrelfromlist}{ключевое понятие}
	\scnitem{вопрос}
	\scnitem{ответ на вопрос}
	\scnitem{знак в рамках заданного вопроса}
	\scnitem{основной знак в рамках заданного вопроса}
	\scnitem{неосновной знак в рамках заданного вопроса}
	\scnitem{отношение в рамках заданного вопроса}
	\scnitem{базовое отношение в рамках заданного вопроса}
\end{scnrelfromlist}
\end{SCn}

\textbf{\textit{Денотационная семантика Языка вопросов для ostis-систем}} включает \textit{классы вопросов} и соответствующие \textit{классы ответов}, необходимые для спецификации формулировок \textit{вопросов} и \textit{ответов} на них, а также \textit{классы знаков} и \textit{отношений}, входящих в структуру любого \textit{вопроса}. В \textit{Семантической классификации вопросов} \textit{Языка вопросов для ostis-систем} заложена идея, описанная в работе \scncite{Suleimanov2001}.

Любой \textbf{\textit{вопрос}} на \textit{Языке вопросов для ostis-систем} представляет собой \textit{спецификацию действия} на поиск или синтез \textit{знания}, удовлетворяющего информационную потребность \textit{пользователя}, инициирующего этот \textit{вопрос}. То есть \textit{вопрос} --- это ничто иное как \textit{задача}, с помощью которой выражается потребность пользователя в некоторой информации, возможно хранимой или выводимой в \textit{базе знаний} \textit{ostis-системы} (см. \ref{chapter_actions}~\nameref{chapter_actions}). 

Каждому \textit{вопросу} можно взаимно однозначно сопоставить некоторое множество \textit{ответов на} этот \textit{вопрос}. Каждый \textit{ответ на вопрос} представляет собой некоторую \textit{sс-структуру} \textit{семантической окрестности основного знака}, раскрываемого в этом \textit{ответе на} заданный \textit{вопрос}.

\begin{SCn}
\scnheader{вопрос}
\scnidtf{запрос}
\scnidtf{непроцедурная формулировка задачи на поиск (в текущем состоянии базы знаний) или на синтез знания, удовлетворяющего заданным требованиям}
\scnidtf{запрос метода (способа) решения заданного (указываемого) \textit{класса задач} либо \textit{плана решения} конкретной указываемой \textit{задачи}}
\scnidtf{задача, направленная на удовлетворение информационной потребности некоторого субъекта-заказчика}
\scnsubset{задача}

\scnheader{ответ на вопрос}
\scnidtf{ответ на запрос}
\scnidtf{результат запроса}
\scnidtf{результат решения задачи на поиск или синтез знания, удовлетворяющий заданным требованиям}
\scnidtf{семантическая окрестность \textit{основного знака}, знание в которой удовлетворяет информационную потребность пользователя}
\scnidtf{знание в базе знаний ostis-системы, которое удовлетворяет информационную потребность пользователя}
\scnsubset{знание}
\end{SCn}

Среди всех классов \textit{знаков в рамках заданного вопроса} \textit{Языка вопросов для ostis-систем} можно выделить наиболее общие по иерархии классы \textit{знаков}:

\begin{SCn}
\scnheader{знак в рамках заданного вопроса}
\scnsubset{знак}
\begin{scnrelfromset}{разбиение}
	\scnitem{основной знак в рамках заданного вопроса}
	\begin{scnindent}
		\scnidtf{ключевой sc-элемент в рамках заданного вопроса}
		\scnidtf{\textit{знак}, относительно которого задан вопрос}
	\end{scnindent}
	\scnitem{неосновной знак в рамках заданного вопроса}
	\begin{scnindent}
		\scnidtf{\textit{знак}, стоящий в некотором отношении с \textit{основным знаком в рамках заданного вопроса}}
	\end{scnindent}
\end{scnrelfromset}
\end{SCn}

\textbf{\textit{знаком в рамках заданного вопроса}} является любой \textit{знак понятия} или \textit{сущности}, принадлежащий этому \textit{вопросу}. Между \textit{знаками в рамках заданного вопроса} задаётся множество связей \textit{отношений}, входящих в состав различных \textit{предметных областей}. Кроме того, любое \textbf{\textit{отношение в рамках заданного вопроса}} представляет собой \textit{отношение} между \textit{знаками} \textit{предметной области}, принадлежащих заданному \textit{вопросу}. Среди всех классов \textit{отношений в рамках заданного вопроса} можно выделить класс \textbf{\textit{базовых отношений в рамках заданного вопроса}} и класс \textbf{\textit{составных отношений в рамках заданного вопроса}}.

\begin{SCn}
\scnheader{отношение в рамках заданного вопроса}
\scnidtf{определённое отношение между знаками \textit{предметной области} в контексте \textit{вопроса}}
\scnsubset{отношение}

\scnheader{базовое отношение в рамках заданного вопроса}
\scnidtf{\textit{класс отношений}, объединяющий \textit{отношения в заданном вопросе}, отражающие однотипный \textit{смысл} и раскрывающие определённый признак \textit{знаков} \textit{предметной области}}
\scnsubset{отношение в рамках заданного вопроса}
\begin{scnrelfromset}{декомпозиция}
	\scnitem{отношение состояния}
	\scnitem{отношение действия}
	\scnitem{отношение состава}
	\scnitem{теоретико-множественное отношение}
	\scnitem{темпоральное отношение}
	\scnitem{пространственное отношение}
	\scnitem{количественное отношение}
	\scnitem{качественное отношение}
\end{scnrelfromset}
\end{SCn}

Например, \textit{отношения в рамках заданного вопроса} такие, как \scnqqi{играет*}, \scnqqi{спит*}, \scnqqi{плавает*}, объединяются в класс \textit{отношений состояния} по признаку выражать состояние знака (то есть данные отношения раскрывают признак \textit{знака} \textit{предметной области} --- \scnqqi{находиться в некотором состоянии}).

\begin{SCn}
\scnheader{составное отношение в рамках заданного вопроса}
\scnidtf{устойчивая комбинация двух \textit{отношений действия}: действия, направленного на \textit{параметр вопроса\scnrolesign}, и действия, направленного на \textit{ответ на вопрос*}}
\end{SCn}

Например, элемент \textit{составного отношения в рамках заданного вопроса} между \textit{знаками}: \scnqqi{\textit{Нефтеперерабатывающий завод}}, \scnqqi{\textit{нефть}} и \scnqqi{\textit{нефтепродукты}} --- может быть представлен как \scnqqi{Нефтеперерабатывающий завод перерабатывает нефть в нефтепродукты}.

Смысловая классификация \textit{вопросов} дает возможность противопоставить каждому типу вопроса ограниченный набор допустимых, то есть \textit{семантически корректных информационных конструкций}, передающий правильный \textit{смысл} \textit{вопроса} в зависимости от класса \textit{вопроса}. При этом \textbf{\textit{Семантическая классификация вопросов}} позволяет разбить множество \textit{вопросов} на классы, в каждом из которых требуется раскрытие некоторого однотипного \textit{смысла}, заданного классом этого \textit{вопроса}. 

\begin{SCn}
\scnheader{вопрос}
\begin{scnrelfromset}{декомпозиция}
	\scnitem{вопрос, требующий вывода семантической окрестности \textit{основного знака}}
	\begin{scnindent}
		\begin{scnhaselementrolelist}{пример}
			\scnitem{Вопрос. Что такое \textit{Город Минск}}
		\end{scnhaselementrolelist}
	\end{scnindent}
	\scnitem{вопрос, требующий раскрытия в ответе \textit{базового отношения} \textit{основного знака}}
	\begin{scnindent}
		\begin{scnhaselementrolelist}{пример}
			\scnitem{Вопрос. Что легче: железо или дерево}
		\end{scnhaselementrolelist}
	\end{scnindent}
	\scnitem{вопрос, требующий раскрытия в ответе \textit{составного отношения} \textit{основного знака}}
	\begin{scnindent}
		\scntext{пояснение}{Такому классу \textit{вопросов} соответствуют классы \textit{ответов}, в которых \textit{главный знак} раскрывается через \textit{составное отношение}.}
		\begin{scnhaselementrolelist}{пример}
			\scnitem{Вопрос. Какие Принципы компонентного проектирования в интеллектуальных компьютерных системах нового поколения}
		\end{scnhaselementrolelist}
	\end{scnindent}
	\scnitem{вопрос, требующий раскрытия в ответе произвольной комбинации \textit{базового отношения} и/или \textit{составного отношения} \textit{основного знака}}
	\begin{scnindent}
		\begin{scnhaselementrolelist}{пример}
			\scnitem{Вопрос. Как определяется уровень интеллекта кибернетической системы}
		\end{scnhaselementrolelist}
	\end{scnindent}
	\scnitem{вопрос, требующий раскрытия в ответе более чем одного \textit{основного знака}}
	\begin{scnindent}
		\begin{scnhaselementrolelist}{пример}
			\scnitem{Вопрос. Докажите теорему Пифагора}
		\end{scnhaselementrolelist}
	\end{scnindent}
\end{scnrelfromset}
\end{SCn}

\begin{SCn}
\scnheader{вопрос, требующий раскрытия в ответе \textit{базового отношения} \textit{основного знака}}
\begin{scnrelfromset}{декомпозиция}
	\scnitem{вопрос, требующий раскрытия в ответе \textit{отношения состава} \textit{основного знака}}
	\begin{scnindent}
		\scnidtf{класс вопросов, в ответах на которые \textit{основной знак} \textit{S} раскрывается через его \textit{отношение состава} в связке с его составляющими знаками \textit{P} и \textit{Q}}
		\begin{scnhaselementrolelist}{пример}
			\scnitem{Вопрос. Какие административные районы входят в состав Города Витебск}
			\begin{scnindent}
				\scneqimage[30em]{author/part3/figures/question\_about\_vitebsk\_regions.png}
				\scnrelfrom{ответ на вопрос}{\{Железнодорожный район Города Витебск, Октябрьский район Города Витебск, Первомайский район Города Витебск\}}
				\begin{scnindent}
					\scneqimage[30em]{author/part3/figures/question\_about\_vitebsk\_regions\_answer.png}
				\end{scnindent}
			\end{scnindent}
		\end{scnhaselementrolelist}
	\end{scnindent}
	\scnitem{вопрос, требующий раскрытия в ответе \textit{теоретико-множественного отношения} \textit{основного знака}}
	\begin{scnindent}
		\scnidtf{класс вопросов, в ответах на которые \textit{основной знак} \textit{S} раскрывается через его \textit{теоретико-множественное отношение} в связке с другим знаком \textit{P}, содержащего \textit{S} как часть}
		\begin{scnhaselementrolelist}{пример}
			\scnitem{Вопрос. Частью какой области является Смолевичский район}
			\begin{scnindent}
				\scneqimage[30em]{author/part3/figures/question\_about\_smolevichi\_inclusion.png}
				\scnrelfrom{ответ на вопрос}{\{Смолевичский район является частью Минской области\}}
			\end{scnindent}
		\end{scnhaselementrolelist}
	\end{scnindent}
	\scnitem{вопрос, требующий раскрытия в ответе \textit{отношения состояния} \textit{основного знака}}
	\begin{scnindent}
		\scnidtf{класс вопросов, в ответах на которые \textit{основной знак} \textit{S} раскрывается через его \textit{отношение состояния}}
		\begin{scnhaselementrolelist}{пример}
			\scnitem{Вопрос. Какие города современной территории Республики Беларусь имели Магдебургское право}
			\begin{scnindent}
				\scneqimage[30em]{author/part3/figures/question\_about\_minsk\_district\_town\_with\_mag\_act.png}
				\scnrelfrom{ответ на вопрос}{\{Волковыск, Гродно, Мозырь и другие имели Магдебургское право\}}
			\end{scnindent}
		\end{scnhaselementrolelist}
	\end{scnindent}
	\scnitem{вопрос, требующий раскрытия в ответе \textit{отношения действия} \textit{основного знака}}
	\begin{scnindent}
		\scnidtf{класс вопросов, в ответах на которые \textit{основной знак} \textit{S} раскрывается через его \textit{отношение действия} в связке с другим знаком \textit{P}}
	\end{scnindent}
	\scnitem{вопрос, требующий раскрытия в ответе \textit{темпорального отношения} \textit{основного знака}}
	\begin{scnindent}
		\scnidtf{класс вопросов, в ответах на которые \textit{основной знак} \textit{S} раскрывается через его \textit{темпоральное отношение} в связке с другим знаком \textit{P} по некоторой временной шкале}
		\begin{scnhaselementrolelist}{пример}
			\scnitem{Вопрос. Какое событие произошло раньше: Первый раздел Речи Посполитой или Бородинское сражение}
			\begin{scnindent}
				\scneqimage[30em]{author/part3/figures/question\_about\_events.png}
				\scnrelfrom{ответ на вопрос}{\{Первый раздел Речи Посполитой был раньше Бородинского сражения\}}
				\begin{scnindent}
					\scneqimage[30em]{author/part3/figures/question\_about\_event\_answer.png}
				\end{scnindent}
			\end{scnindent}
		\end{scnhaselementrolelist}
	\end{scnindent}
	\scnitem{вопрос, требующий раскрытия в ответе \textit{пространственного отношения} \textit{основного знака}}
	\begin{scnindent}
		\scnidtf{класс вопросов, в ответах на которые \textit{основной знак} \textit{S} раскрывается через \textit{пространственное отношение}, отражающее его положение в пространстве относительно другого знака \textit{P}}
	\end{scnindent}
	\scnitem{вопрос, требующий раскрытия в ответе \textit{количественного отношения} \textit{основного знака}}
	\begin{scnindent}
		\scnidtf{класс вопросов, в ответах на которые раскрывается \textit{количественное отношение} \textit{основного знака}}
		\begin{scnhaselementrolelist}{пример}
			\scnitem{Вопрос. Какова высота Горы Дзержинская}
			\begin{scnindent}
				\scneqimage[30em]{author/part3/figures/question\_about\_mountain\_length.png}
				\scnrelfrom{ответ на вопрос}{\{Высота Горы Дзержинская --- 345 м\}}
			\end{scnindent}
		\end{scnhaselementrolelist}
	\end{scnindent}
	\scnitem{вопрос, требующий раскрытия в ответе \textit{качественного отношения} \textit{основного знака}}
	\begin{scnindent}
		\scnidtf{класс вопросов, в ответах на которые раскрывается \textit{качественное отношение} \textit{основного знака} \textit{S} в связке с другим знаком \textit{P}}
		\begin{scnhaselementrolelist}{пример}
			\scnitem{Вопрос. Территория какой административной области больше: Минской или Брестской}
			\begin{scnindent}
				\scneqimage[30em]{author/part3/figures/question\_about\_district\_squares.png}
				\scnrelfrom{ответ на вопрос}{\{Территория Минской области больше Брестской\}}
				\begin{scnindent}
					\scneqimage[30em]{author/part3/figures/question\_about\_district\_squares\_answer.png}
				\end{scnindent}
			\end{scnindent}
		\end{scnhaselementrolelist}
	\end{scnindent}
\end{scnrelfromset}
\end{SCn}

\begin{SCn}
\scnheader{вопрос, требующий раскрытия в ответе произвольной комбинации \textit{базового отношения} и/или \textit{составного отношения} \textit{основного знака}}
\begin{scnrelfromset}{декомпозиция}
	\scnitem{вопрос, требующий раскрытия в ответе произвольной комбинации \textit{составного отношения описания} \textit{основного знака}}
	\begin{scnindent}
		\scnidtf{класс вопросов, в ответах на которые раскрываются произвольные комбинации \textit{базового отношения} и/или \textit{составного отношения} \textit{основного знака} \textit{S} в связке с другими знаками}
		\begin{scnhaselementrolelist}{пример}
			\scnitem{\{S состоит из P, Q, W. S переводит X и Y и выполняется раньше Z\}}
			\begin{scnindent}
				\scnrelto{ответ на вопрос}{Вопрос. Что такое S}
			\end{scnindent}
		\end{scnhaselementrolelist}
	\end{scnindent}
	\scnitem{вопрос, требующий раскрытия в ответе произвольной комбинации \textit{составного отношения определения} \textit{основного знака}}
	\begin{scnindent}
		\scnidtf{класс ответов, в которых \textit{основной знак} \textit{S} раскрывается через \textit{первостепенное понятие} и его \textit{описание}}
		\begin{scnhaselementrolelist}{пример}
			\scnitem{\{Минск --- это столица, которая находится в РБ\}}
			\begin{scnindent}
				\scnrelto{ответ на вопросы}{Вопрос. Как определяется город Минск}
			\end{scnindent}
		\end{scnhaselementrolelist}
	\end{scnindent}
	\scnitem{вопрос, требующий раскрытия в ответе произвольной комбинации \textit{составного отношения причины} \textit{основного знака}}
	\begin{scnindent}
		\scnidtf{класс вопросов, в ответах на которые раскрывается условие существования некоторых отношений \textit{основного знака} \textit{S} в связке с другими знаками}
		\begin{scnhaselementrolelist}{пример}
			\scnitem{Вопрос. Почему время в пути от города Минска до города Борисова меньше чем время в пути от города Минска до города Орша}
			\begin{scnindent}
				\scnrelfrom{ответ на вопрос}{\{Время в пути от города Минска до города Борисова меньше чем время в пути от города Минска до города Орша, потому что расстояние от города Минска меньше до города Борисова, чем до города Орша\}}
			\end{scnindent}
		\end{scnhaselementrolelist}
	\end{scnindent}
	\scnitem{вопрос, требующий раскрытия в ответе произвольной комбинации \textit{составного отношения следствия} \textit{основного знака}}
	\scnidtf{класс вопросов, в ответах на которые раскрывается следствие от существования некоторых отношений \textit{основного знака} \textit{S} в связке с другими знаками}
	\begin{scnindent}
		\begin{scnhaselementrolelist}{пример}
			\scnitem{Вопрос. Что следует из того, что расстояние от города Минска до города Борисова меньше расстояния от города Минска до города Орша}
			\begin{scnindent}
				\scnrelfrom{ответ на вопрос}{\{Расстояние от города Минска до города Борисова меньше расстояния от города Минска до города Орша, поэтому от города Минска до города Борисова время в пути меньше чем до города Орша\}}
			\end{scnindent}
		\end{scnhaselementrolelist}
	\end{scnindent}
\end{scnrelfromset}
\end{SCn}

\begin{SCn}
\scnheader{вопрос, требующий раскрытия в ответе более чем одного \textit{основного знака}}
\scnsuperset{вопрос, требующий раскрытия в ответе \textit{отношение детализации} знаков, стоящих в некоторых отношениях с \textit{основным знаком}}
\begin{scnindent}
	\scnidtf{класс вопросов, в ответах на которые происходит детализация знаков, стоящих в некоторых отношениях с \textit{основным знаком} \textit{S}}
	\begin{scnhaselementrolelist}{пример}
		\scnitem{Вопрос. Какая связь водной сети существует между городом Минск и городом Светлогорск}
		\begin{scnindent}
			\scnrelfrom{ответ на вопрос}{\{Город Минск расположен на реке Свислочь, которая впадает в реку Березина, протекающую через город Светлогорск\}}
		\end{scnindent}
	\end{scnhaselementrolelist}
\end{scnindent}
\end{SCn}

Таким образом, для каждого \textit{вопроса} \textit{пользователя ostis-системы} можно найти класс \textit{вопросов}, на котором можно реализовывать \textit{вывод ответов} на этот \textit{вопрос}. Описанная \textit{Семантическая классификация вопросов} позволяет:
\begin{textitemize}
	\item автоматически структурировать \textit{вопросы} \textit{пользователей} по описанию этих \textit{вопросов};
	\item а также формировать \textit{ответы на} эти \textit{вопросы} с учётом \textit{непроцедурных формулировок} этих \textit{вопросов}.
\end{textitemize}

\section{Операционная семантика языка вопросов для ostis-систем}
\label{sec_requests_op_semantics}

\begin{SCn}
\begin{scnrelfromlist}{ключевое понятие}
	\scnitem{вопрос}
	\scnitem{ответ на вопрос}
	\scnitem{знак в рамках заданного вопроса}
	\scnitem{основной знак в рамках заданного вопроса}
	\scnitem{неосновной знак в рамках заданного вопроса}
	\scnitem{отношение в рамках заданного вопроса}
	\scnitem{базовое отношение в рамках заданного вопроса}
\end{scnrelfromlist}
\end{SCn}

Каждому классу \textit{вопросов} должен соответствовать определённый \textit{коллектив sc-агентов}, реализующих поиск или синтез из \textit{базы знаний} \textit{ostis-системы} соответствующих ответов на поставленные \textit{вопросы}. Следует отметить, что в зависимости от степени наполненности \textit{базы знаний} \textit{ответы} могут содержаться в \textit{базе знаний} либо отсутствовать в текущей версии \textit{базы знаний}. В случае наличия в \textit{базе знаний} \textit{ответа на} поставленный \textit{вопрос} информационная потребность пользователя реализуется \textit{информационно-поисковыми sc-агентами}, в противном случае --- в зависимости от \textit{классов вопросов} формирование ответов осуществляется специализированными \textit{sc-агентами}, которые в процессе работы дополнительно выполняют вычислительные задачи либо осуществляют синтез на основе \textit{логического вывода} (см. \ref{chapter_logic}~\nameref{chapter_logic}) или других \textit{моделей решения задач}. 

\begin{SCn}
\scnheader{интерпретатор Языка вопросов для ostis-систем}
\scniselement{неатомарный sc-агент}
\begin{scnrelfromset}{декомпозиция абстрактного sc-агента*}
	\scnitem{Абстрактный sc-агент поиска ответа на заданный вопрос}
	\begin{scnindent}
		\begin{scnrelfromset}{декомпозиция абстрактного sc-агента*}
			\scnitem{Абстрактный sc-агент поиска семантической окрестности \textit{основного знака}}
			\scnitem{Абстрактный sc-агент поиска ответа на вопрос, требующий раскрытия в ответе \textit{отношения состава} для \textit{основного знака}}
			\scnitem{Абстрактный sc-агент поиска ответа на вопрос, требующий раскрытия в ответе \textit{теоретико-множественного отношения} для \textit{основного знака}}
			\scnitem{Абстрактный sc-агент поиска ответа на вопрос, требующий раскрытия в ответе \textit{отношения состояния} для \textit{основного знака}}	
			\scnitem{Абстрактный sc-агент поиска ответа на вопрос, требующий раскрытия в ответе \textit{отношения действия} для \textit{основного знака}}	
			\scnitem{Абстрактный sc-агент поиска ответа на вопрос, требующий раскрытия в ответе \textit{темпорального отношения} для \textit{основного знака}}
			\scnitem{Абстрактный sc-агент поиска ответа на вопрос, требующий раскрытия в ответе \textit{пространственного отношения} для \textit{основного знака}}
			\scnitem{Абстрактный sc-агент поиска ответа на вопрос, требующий раскрытия в ответе \textit{количественного отношения} для \textit{основного знака}}
			\scnitem{Абстрактный sc-агент поиска ответа на вопрос, требующий раскрытия в ответе \textit{качественного отношения} для \textit{основного знака}}
			\scnitem{Абстрактный sc-агент поиска ответа на вопрос, требующий раскрытия в ответе \textit{отношения описания} для \textit{основного знака}}
			\scnitem{Абстрактный sc-агент поиска ответа на вопрос, требующий раскрытия в ответе \textit{отношения определения} для \textit{основного знака}}
			\scnitem{Абстрактный sc-агент поиска ответа на вопрос, требующий раскрытия в ответе \textit{отношения причины} для \textit{основного знака}}
			\scnitem{Абстрактный sc-агент поиска ответа на вопрос, требующий раскрытия в ответе \textit{отношения следствия} для \textit{основного знака}}
			\scnitem{Абстрактный sc-агент поиска ответа на вопрос, требующий раскрытия в ответе \textit{отношения детализации} для \textit{основного знака}}
		\end{scnrelfromset}
	\end{scnindent}
	\scnitem{Абстрактный sc-агент синтеза ответа на заданный вопрос}
\end{scnrelfromset}
\end{SCn}

Все \textit{sc-агенты}, выводящие \textit{ответы на} поставленные \textit{вопросы}, формируют \textit{коллектив sc-агентов} --- \textbf{\textit{интерпретатор Языка вопросов для ostis-систем}}, с помощью которого можно интерпретировать любые классы \textit{вопросов}. \textit{интерпретатор Языка вопросов для ostis-систем} может быть реализован по-разному: в виде \textit{коллектива scp-агентов} или \textit{платформенно-зависимых sc-агентов}.

\section*{Заключение к Главе \ref{chapter_requests}}

Перечислим основные положения данной главы:
\begin{textitemize}
	\item информационная потребность \textit{пользователей ostis-системы} может быть выражена в виде \textit{вопросов}, а удовлетворение этой информационной потребности --- в виде \textit{ответов на} заданные \textit{вопросы};
	\item вывод \textit{ответов на} заданные \textit{вопросы} \textit{пользователем ostis-системы} может быть осуществлён путём поиска \textit{знаний} в текущем состоянии \textit{базы знаний} этой \textit{ostis-системы}, либо синтеза новых знаний, отсутствующих в \textit{базе знаний} этой \textit{ostis-системы};
	\item каждый \textit{вопрос} может быть представлен в виде некоторой \textit{спецификации задачи}, инициированной \textit{пользователем ostis-системы} для удовлетворения своей информационной потребности, а \textit{ответ на} этот \textit{вопрос} --- в виде \textit{семантической окрестности} \textit{основного знака в рамках заданного вопроса};
	\item каждому \textit{вопросу} может быть сопоставлен соответствующий класс \textit{вопросов} в \textit{Семантической классификации вопросов};
	\item для синтеза отсутствующих \textit{ответов на} поставленные \textit{вопросы} могут быть использованы различные \textit{модели решения задач}, в том числе \textit{логические модели решения задач};
	\item \textit{ответы на} поставленные \textit{вопросы} могут быть транслированы в \textit{естественно-языковой текст} и визуализированы при помощи соответствующих \textit{естественно-языковых интерфейсов} для удобства выдачи информации любому пользователю.
\end{textitemize}

%\input{author/references}