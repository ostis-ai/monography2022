\chapauthor{Сердюков Р.Е.\\Зотов Н.В.\\Шункевич Д.В.}
\chapter{Семантическая теория программ в интеллектуальных компьютерных системах нового поколения}
\chapauthortoc{Сердюков Р.Е.\\Зотов Н.В.\\Шункевич Д.В.}
\label{chapter_programs}

\abstract{}

Современный персональный компьютер теперь имеет производительность большой электронной вычислительной машины 80-х годов
прошлого века. За долгий период развития компьютерных систем практически сняты аппаратные ограничения на решение ими
задач. Оставшиеся ограничения отводятся на долю программного обеспечения.
Прежде всего эти ограничения связаны с текущими проблемами развития программного обеспечения:
\begin{scnitemize}
    \item аппаратная сложность опережает умение человечества строить программные системы, использующее потенциальные
    возможности аппаратуры;
    \item навыки и технологии разработки программ отстают от требований, предъявлемых к разработке программ нового
    поколения;
    \item возможностям эксплуатировать существующие программы угрожает низкое качество их разработки.
\end{scnitemize}
Ключом к решению этих проблем является глубокое понимание и грамотное использование существующих языков программирования 
как основного инструмента для массового создания программных систем нового поколения, в том числе разумная организация 
процесса разработки и реинжиниринга таких систем.

Авторы данной главы стремятся достичь следующих результатов:
\begin{scnitemize}
    \item изложить классические основы, отражающие накопленный мировой опыт в области языков программирования;
    \item показать научные и практические достижения, характеризующие динамику развития языков программирования;
    \item систематизировать все основные результаты в этой области и представить их в виде единой универсальной
    семантической теории программ.
\end{scnitemize}

\section{Программы и языки программирования для ostis-систем}
\section{Принципы интерпретации современных языков программирования в ostis-системах}

%\input{author/references}
