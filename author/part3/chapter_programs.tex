\chapauthor{Сердюков Р.Е.\\Зотов Н.В.\\Шункевич Д.В.}
\chapter{Семантическая теория программ в интеллектуальных компьютерных системах нового поколения}
\chapauthortoc{Сердюков Р.Е.\\Зотов Н.В.\\Шункевич Д.В.}
\label{chapter_programs}

\abstract{}

Современный персональный компьютер теперь имеет производительность большой электронной вычислительной машины 80-х годов
прошлого века. За долгий период развития компьютерных систем практически сняты аппаратные ограничения на решение ими
задач. Оставшиеся ограничения отводятся на долю программного обеспечения.
Прежде всего эти ограничения связаны с текущими проблемами развития программного обеспечения:
\begin{scnitemize}
    \item аппаратная сложность опережает умение человечества строить программные системы, использующее потенциальные
    возможности аппаратуры;
    \item навыки и технологии разработки программ отстают от требований, предъявлемых к разработке программ нового
    поколения;
    \item возможностям эксплуатировать существующие программы угрожает низкое качество их разработки.
\end{scnitemize}
Ключом к решению этих проблем является глубокое понимание и грамотное использование существующих языков программирования 
как основного инструмента для массового создания программных систем нового поколения, в том числе разумная организация 
процесса разработки и реинжиниринга таких систем.

Авторы данной главы стремятся достичь следующих результатов:
\begin{scnitemize}
    \item изложить классические основы, отражающие накопленный мировой опыт в области языков программирования;
    \item показать научные и практические достижения, характеризующие динамику развития языков программирования;
    \item систематизировать все основные результаты в этой области и представить их в виде единой универсальной
    семантической теории программ.
\end{scnitemize}

В первом параграфе данной главы подробно описывается текущее состояние в области программ и языков программирования,
которые могут и должны быть использования для разработки интеллектуальных компьютерных систем нового поколения. Он
посвящен базовым понятиям языков программирования, дается обзорная характеристика пяти типовых областей применения
языков программирования, достаточно востребованных современным человеческим обществом, подробно описываются формы и
содержание критериев для оценки эффективности языков и рассматриваются способы построения этих критериев.

% добавить параграф для компьютерных языков

\section{Программы и языки программирования для ostis-систем}

В современную эру развития информационных технологий существует огромное количество языков программирования, каждый из
который имеет своё важное назначение в области проектирования программных систем. Каждый язык демонстрирует не только
свою специфику, но имеет свои достоинства и недостатки. Многообразие языков программирования и решений, созданных
на них, настолько велико, что очень легко потеряться в море информации о всех аспектах применения и проектирования
языках программирования.

Зачем необходима общая теория программ и языков программирования?
\begin{scnnumerize}
    \item Ключом к легкому и глубокому освоению конкретного языка как основного профессионального инструмента
    программиста является понимание общих принципов построения и применения языков программирования, описываемых их
    общей теорией.
    \item Достижение максимума услуг и средств при минимуме затрат возможно только путём глубокого понимания принципов
    построения языков программирования.
\end{scnnumerize}

Каждая теория должна быть согласована понятийна. Не смотря на то, что в литературе сложилась разное трактование понятия
языка программирования, должно быть одно универсальное. Под \textit{языком программирования} будем подразумевать
формальный язык, (1) знаковыми конструкциями которого являются соответствующие программы, для которых существуют
общие правила построения и (2) общие правила соотнесения с теми сущностями и связями между ними, которые описываются
этими программами. Понятие синтакиса, денотационной и операционной семантики языков программирования сводятся к
понятию синтаксиса, денотационной и операционной семантики вообще любого языка.

С помощью языка программирования формируются сообщения (программы) для компьютера. Эти сообщения должны быть понятны
(семантически корректны и целостны) компьютеру.

\begin{SCn}
\scnheader{язык программирования}
\scnsubset{компьютерный язык общего назначения}
\begin{scnindent}
    \scnsubset{компьютерный язык}
    \begin{scnindent}
        \scnsubset{формальный язык}
    \end{scnindent}
\end{scnindent}
\scnidtf{формальный язык, (1) знаковыми конструкциями которого являются соответствующие программы, для которых
существуют общие правила построения и (2) общие правила соотнесения с теми сущностями и связями между ними, которые
описываются этими программами}
\scnidtf{средство общения между человеком (пользователем) и компьютером (исполнителем)}
\scnidtf{инструмент для производства программных услуг}
\end{SCn}

\begin{SCn}
\scnheader{отношение, заданное на множестве языков программирования\scnsupergroupsign}
\scnidtf{отношение, область определения которого включает в себя множество всевозможных языков программирования}
\scnhaselement{программа заданного языка программирования*}
\begin{scnindent}
    \scnidtf{программа, принадлежащая заданному языку программирования*}
    \scnsubset{текст заданного языка*}
    \scnrelfrom{второй домен}{программа}
    \begin{scnreltoset}{объединение}
        \scnitem{\scnnonamednode}
        \begin{scnindent}
            \begin{scnreltoset}{объединение}
                \scnitem{синтаксически корректная программа для заданного языка программирования*}
                \scnitem{синтаксически целостная программа для заданного языка программирования*}
            \end{scnreltoset}
        \end{scnindent}
        \scnitem{\scnnonamednode}
        \begin{scnindent}
            \begin{scnreltoset}{объединение}
                \scnitem{семантически корректная программа для заданного языка программирования*}
                \scnitem{синтаксически целостная программа для заданного языка программирования*}
            \end{scnreltoset}
        \end{scnindent}
    \end{scnreltoset}
\end{scnindent}
\scnhaselement{синтаксически корректная программа для заданного языка программирования*}
\begin{scnindent}
    \scnidtf{программа, не содержащая синтаксических ошибок для заданного языка программирования*}
    \scnsubset{синтаксически корректная знаковая конструкция для заданного языка*}
\end{scnindent}
\scnhaselement{синтаксически целостная программа для заданного языка программирования*}
\begin{scnindent}
    \scnsubset{синтаксически целостная знаковая конструкция для заданного языка*}
\end{scnindent}
\scnhaselement{семантически корректная программа для заданного языка программирования*}
\begin{scnindent}
    \scnidtf{программа, не содержащая семантических ошибок для заданного языка программирования*}
    \scnsubset{семантически корректная знаковая конструкция для заданного языка*}
\end{scnindent}
\scnhaselement{семантически целостная программа для заданного языка программирования*}
\begin{scnindent}
    \scnsubset{семантически целостная знаковая конструкция для заданного языка*}
    \scnidtf{программа заданного языка программирования, содержащий достаточную информацию для установления его
    истинности*}
\end{scnindent}
\end{SCn}

\begin{SCn}
\scnheader{язык программирования}
\scnrelfrom{разбиение}{парадигма языка программирования\scnsupergroupsign}
\begin{scnindent}
    \begin{scneqtoset}
        \scnitem{процедурный язык программирования}
        \begin{scnindent}
            \scnidtf{императивный язык программирования}
            \begin{scnhaselementrolelist}{пример}
                \scnitem{Fortran}
                \scnitem{Си}
            \end{scnhaselementrolelist}
        \end{scnindent}
        \scnitem{функциональный язык программирования}
        \begin{scnindent}
            \scnidtf{аппликативный язык программирования}
            \begin{scnhaselementrolelist}{пример}
                \scnitem{LISP}
            \end{scnhaselementrolelist}
        \end{scnindent}
        \scnitem{логический язык программирования}
        \begin{scnindent}
            \scnidtf{декларативный язык программирования}
            \begin{scnhaselementrolelist}{пример}
                \scnitem{Prolog}
            \end{scnhaselementrolelist}
        \end{scnindent}
        \scnitem{объектно-ориентированный язык программирования}
        \begin{scnindent}
            \begin{scnhaselementrolelist}{пример}
                \scnitem{Smalltalk}
                \scnitem{HTML}
            \end{scnhaselementrolelist}
        \end{scnindent}
    \end{scneqtoset}
\end{scnindent}
\scnrelfrom{разбиение}{типология языков программирования по цели использования\scnsupergroupsign}
\begin{scnindent}
    \begin{scneqtoset}
        \scnitem{традиционный язык программирования}
        \begin{scnindent}
            \begin{scnhaselementrolelist}{пример}
                \scnitem{C++}
                \scnitem{Fortran}
                \scnitem{Java}
            \end{scnhaselementrolelist}
        \end{scnindent}
        \scnitem{скриптовой язык программирования}
        \begin{scnindent}
            \scnidtf{склеивающий язык программирования}
            \begin{scnhaselementrolelist}{пример}
                \scnitem{Python}
                \scnitem{Perl}
                \scnitem{JavaScript}
                \scnitem{PHP}
                \scnitem{Ruby}
                \scnitem{Lua}
            \end{scnhaselementrolelist}
        \end{scnindent}
        \scnitem{гибридный язык программирования}
        \begin{scnindent}
            \begin{scnhaselementrolelist}{пример}
                \scnitem{XSLT}
                \scnitem{JSP}
            \end{scnhaselementrolelist}
        \end{scnindent}
    \end{scneqtoset}
\end{scnindent}
\end{SCn}

\begin{SCn}
\scnheader{программа}
\end{SCn}

\begin{SCn}
\scnheader{эффективность языка программирования\scnsupergroupsign}
\end{SCn}

\section{Принципы интерпретации современных языков программирования в ostis-системах}

%\input{author/references}
