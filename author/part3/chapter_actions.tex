\chapauthor{Шункевич Д.В.\\Ковалёв М.В.\\Никифоров С.А.}
\chapter{Формализация понятий действия, задачи, метода, средства, навыка и технологии}
\chapauthortoc{Шункевич Д.В.\\Ковалёв М.В.\\Никифоров С.А.}
\label{chapter_actions}

\abstract{Аннотация к главе.}

\section{Глобальная предметная область и онтология воздействий, действий, методов, средств и технологий}

\subsection{Понятие метода (программы)}
\label{sec_actions_method_concept}

Формально, \textit{метод} -- это спецификация решения задачи какого-то класса \cite{Standard2021}, \cite{Tuzov1986}. В состав спецификации каждого класса задач входит описание способа \scnqq{привязки} метода к исходным данным конкретной задачи, решаемой с помощью этого метода.

\begin{SCn}
	\scnheader{метод}
	\scnidtf{программа}
	\scnidtf{описание того, как может быть выполнено любое или почти любое действие, принадлежащее соответствующему классу действий}
	\scnidtf{метод решения соответствующего класса задач, обеспечивающий решение любой или большинства задач
		указанного класса}
	\scnidtf{обобщенная спецификация решения задач соответствующего класса}
	\scnidtf{программа решения задач соответствующего класса, которая может быть как процедурной, так и декларативной (непроцедурной)}
	\scnidtf{знание о том, как можно решать задачи соответствующего класса}
	\scnsubset{знание}
	\scniselement{вид знаний}
	\scnidtf{способ}
	\scnsuperset{модель решения задач}
\end{SCn}

\subsection{Понятие класса методов. Общая классификация методов}
\label{sec_actions_method_class_concept}

Иногда целесообразным может считаться выделение некоторого подмножества методов (например, множества методов, с помощью которых решается определённая задача), тогда в таком случае для этих методов можно описать требования, которые они должны выполнять. Такие множества методов являются \textit{классами методов} некоторого я.п.м., которым ставится в соответствие конкретная \textit{модель решения задач}. Методы могут быть как \textit{процедурные}, так и \textit{непроцедурные} \cite{Turner2007}.

\begin{SCn}
	\scnheader{класс методов}
	\scnrelto{семейство подклассов}{метод}
	\scnidtf{множество методов, для которых можно унифицировать представление (спецификацию) этих методов}
	\scnidtf{множество всевозможных методов решения задач, имеющих общий язык представления этих методов}
	\scnidtf{множество методов, для которых задан язык представления этих методов}
	\scnhaselement{процедурный метод решения задач}
	\begin{scnindent}
		\scnsuperset{алгоритмический метод решения задач}
	\end{scnindent}
	\scnhaselement{непроцедурный метод решения задач}
	\begin{scnindent}
		\scnsuperset{логический метод решения задач}
		\scnsuperset{продукционный метод решения задач}
		\scnsuperset{функциональный метод решения задач}
		\begin{scnindent}
			\scnsuperset{искусственная нейронная сеть}
			\scnsuperset{генетический \scnqq{алгоритм}}
		\end{scnindent}
	\end{scnindent}
	\scnidtf{множество методов, которому ставится в соответствие отдельная модель решения задач}
\end{SCn}

Поскольку каждому методу соответствует обобщенная формулировка задач, решаемых с помощью этого метода, то каждому классу методов должен соответствовать не только определенный я.п.м, принадлежащих указанному \textit{классу методов}, но и определенный язык представления обобщенных формулировок задач для различных классов задач, решаемых с помощью методов, принадлежащих указанному классу методов.

Для процедурных и непроцедурных методов хоть и можно задать \textit{входные} и \textit{выходные параметры}, но нельзя задать общую денотационную семантику их логических элементов: для процедурных методов -- это операторы, для непроцедурных -- математические объекты предметной области.

\subsection{Понятие языка представления методов (языка программирования)}
\label{sec_actions_method_representation_language_concept}

Каждому конкретному классу методов взаимно однозначно соответствует я.п.м., принадлежащих этому (специфицируемому) классу методов. Таким образом, спецификация каждого класса методов сводится к спецификации соответствующего я.п.м., т.е. к описанию его синтаксической, денотационной семантики и операционной семантики. Примерами я.п.м. являются все языки программирования, которые в основном относятся к подклассу я.п.м. Но сейчас все большую актуальность приобретает необходимость создания эффективных формальных я.п.м. выполнения действий во внешней среде кибернетических систем. Без этого комплексная автоматизация \cite{Pospelov2021}, в частности, в промышленной сфере невозможна.

Под \textit{языком представления методов} будем подразумевать формальный язык, (1) знаковыми конструкциями которого являются соответствующие методы, для которых существуют общие правила построения и (2) общие правила соотнесения с теми сущностями и связями между ними, которые описываются этими методами.

С помощью я.п.м. формируются \textit{сообщения} (методы) для компьютера. Эти сообщения должны быть понятны (семантически корректны и целостны) компьютеру.

\begin{SCn}
	\scnheader{язык представления методов}
	\scnidtf{язык программирования}
	\scnsubset{язык представления знаний}
	\begin{scnindent}
		\scnsubset{формальный язык}
	\end{scnindent}
	\scnidtf{компьютерный язык}
	\scnidtf{формальный язык, (1) знаковыми конструкциями которого являются соответствующие методы, для которых существуют общие правила построения и (2) общие правила соотнесения с теми сущностями и связями между ними, которые описываются этими методами}
	\scnidtf{средство общения между человеком (пользователем) и компьютером (исполнителем)}
	\scnidtf{инструмент для производства программных услуг}
\end{SCn}

Метод принадлежит языку представления методов, если он является синтаксически корректным, синтаксически целостным, семантически корректным и семантически целостным методом заданного я.п.м. (!).

\begin{SCn}
	\scnheader{отношение, заданное на множестве языков представления методов\scnsupergroupsign}
	\scnidtf{отношение, область определения которого включает в себя множество всевозможных языков представления методов}
	\scnhaselement{метод заданного языка представления методов*}
	\scnhaselement{синтаксически корректный метод для заданного языка представления методов*}
	\begin{scnindent}
		\scnidtf{метод, не содержащий синтаксических ошибок для заданного языка представления методов*}
		\scnsubset{синтаксически корректная знаковая конструкция для заданного языка*}
	\end{scnindent}
	\scnhaselement{синтаксически целостный метод для заданного языка представления методов*}
	\begin{scnindent}
		\scnsubset{синтаксически целостная знаковая конструкция для заданного языка*}
	\end{scnindent}
	\scnhaselement{семантически корректный метод для заданного языка представления методов*}
	\begin{scnindent}
		\scnidtf{метод, не содержащий семантических ошибок для заданного языка представления методов*}
		\scnsubset{семантически корректная знаковая конструкция для заданного языка*}
	\end{scnindent}
	\scnhaselement{семантически целостный метод для заданного языка представления методов*}
	\begin{scnindent}
		\scnsubset{семантически целостная знаковая конструкция для заданного языка*}
		\scnidtf{метод заданного языка представления методов, содержащий достаточную информацию для установления его
			истинности*}
	\end{scnindent}
\end{SCn}

\begin{SCn}
	\scnheader{метод заданного языка представления методов*}
	\scnidtf{метод, принадлежащий заданному языку программирования*}
	\scnsubset{текст заданного языка*}
	\scnrelfrom{второй домен}{метод}
	\begin{scnreltoset}{объединение}
		\scnitem{\scnnonamednode}
		\begin{scnindent}
			\begin{scnreltoset}{объединение}
				\scnitem{синтаксически корректный метод для заданного языка представления методов*}
				\scnitem{синтаксически целостный метод для заданного языка представления методов*}
			\end{scnreltoset}
		\end{scnindent}
		\scnitem{\scnnonamednode}
		\begin{scnindent}
			\begin{scnreltoset}{объединение}
				\scnitem{семантически корректный метод для заданного языка представления методов*}
				\scnitem{синтаксически целостный метод для заданного языка представления методов*}
			\end{scnreltoset}
		\end{scnindent}
	\end{scnreltoset}
\end{SCn}

\subsection{Общая классификация языков представления методов}
\label{sec_actions_method_representation_language_classification}

Языки представления методов в современном информационном обществе различают по их парадигмам: \textit{процедурные}, \textit{функциональные}, \textit{логические}, \textit{объектно-ориентированные} и т. д. Таким, например, в методах процедурного я.п.м. решение задачи компьютером формируется в виде последовательности операторов, в методах функционального я.п.м. — указанием других методов. В логическом я.п.м. применяются высказывания, а в объектно-ориентированном — объекты.

\begin{SCn}
	\scnheader{язык представления методов}
	\scnsuperset{язык представления методов общего назначения}
	\begin{scnindent}
		\scnidtf{язык программирования общего назначения}
	\end{scnindent}
	\scnsuperset{предметно-ориентированный язык представления методов}
	\begin{scnindent}
		\scnidtf{предметно-ориентированный язык программирования}
	\end{scnindent}
	\scnrelfrom{разбиение}{парадигма языка представления методов\scnsupergroupsign}
	\begin{scnindent}
		\begin{scneqtoset}
			\scnitem{процедурный язык представления методов}
			\scnitem{непроцедурный язык представления методов}
		\end{scneqtoset}
	\end{scnindent}
\end{SCn}

\textit{Процедурные языки представления методов} задают вычисления как последовательность операторов (команд).
Они ориентированы на компьютеры с архитектурой фон Неймана. Основные понятия процедурных я.п.м. тесно связаны с компонентами компьютера:
\begin{textitemize}
	\item переменными различных типов, которые моделируют ячейки памяти компьютера;
	\item операторами присваивания, которые моделируют пересылки данных между участками памяти;
	\item повторений действий в форме итерации, которые моделируют хранение информации в смежных ячейках памяти;
	\item и другое.
\end{textitemize}

\begin{SCn}
	\scnheader{процедурный язык представления методов}
	\scnidtf{императивный язык представления методов}
	\scnsuperset{структурный язык представления методов}
	\begin{scnindent}
		\begin{scnhaselementrolelist}{пример}
			\scnitem{Fortran}
			\scnitem{Си}
			\scnitem{Pascal}
		\end{scnhaselementrolelist}
	\end{scnindent}
	\scnsuperset{объектно-ориентированный язык представления методов}
	\begin{scnindent}
		\begin{scnhaselementrolelist}{пример}
			\scnitem{Java}
			\scnitem{Smalltalk}
			\scnitem{HTML}
		\end{scnhaselementrolelist}
		\scnsuperset{аспектно-ориентированный язык представления методов}
	\end{scnindent}
	\scnsuperset{скриптовый язык представления методов}
	\begin{scnindent}
		\scnidtf{склеивающий язык представления методов}
	\end{scnindent}
\end{SCn}

\textit{Непроцедурные языки представления методов}, в отличие от процедурных, задают вычисления как последовательность связанных между собой объектов. Основные понятия непроцедурных я.п.м. обычно не связаны с компонентами компьютера.

\begin{SCn}
	\scnheader{непроцедурный язык представления методов}
	\scnidtf{декларативный язык представления методов}
	\scnsuperset{логический язык представления методов}
	\begin{scnindent}
		\begin{scnhaselementrolelist}{пример}
			\scnitem{Prolog}
		\end{scnhaselementrolelist}
	\end{scnindent}
	\scnsuperset{продукционный язык представления методов}
	\scnsuperset{функциональный язык представления методов}
	\begin{scnindent}
		\scnidtf{аппликативный язык представления методов}
		\begin{scnhaselementrolelist}{пример}
			\scnitem{LISP}
		\end{scnhaselementrolelist}
	\end{scnindent}
\end{SCn}

\section{Локальные предметные области и онтологии действий}

%\input{author/references}