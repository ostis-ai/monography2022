\begin{SCn}

\bigskip

\begin{scnrelfromlist}{ключевое понятие}
    \scnitem{...}
\end{scnrelfromlist}

\bigskip

\begin{scnrelfromlist}{ключевое знание}
    \scnitem{...}
\end{scnrelfromlist}

\bigskip

\begin{scnrelfromlist}{библиографическая ссылка}
    \scnitem{\scncite{...}}
\end{scnrelfromlist}

\end{SCn}


\subsection{Формальная модель Экосистемы интеллектуальных компьютерных систем нового поколения}
{\label{sec_ecosystem_formal_model}} 

Экосистема OSTIS - Социотехническая экосистема, представляющая собой коллектив взаимодействующих семантических компьютерных систем и осуществляющая перманентную поддержку эволюции и семантической совместимости всех входящих в нее систем, на протяжении всего их жизненного цикла. 

Интеллектуальная компьютерная система, которая построена в соответствии с требованиями и стандартами Технологии OSTIS, определяется как ostis-система. 
Это обеспечивает существенное развитие целого ряда свойств этой компьютерной системы, позволяющих значительно повысить уровень интеллекта этой системы (и, прежде всего, ее уровень обучаемости и уровень социализации). 

Экосистема OSTIS представляет собой коллектив взаимодействующих:
\begin{itemize}
    \item{самих ostis-систем;}
    \item{пользователей указанных ostis-систем (как конечных пользователей, так и разработчиков);}
    \item{некоторых компьютерных систем, не являющихся ostis-системами, но рассматриваемых ими в качестве дополнительных информационных ресурсов или сервисов.}
\end{itemize}

Участники коллектива Экосистемы OSTIS характеризуются как:
\begin{itemize}
    \item{семантически совместимые;}
    \item{постоянно эволюционирующие индивидуально;}
    \item{постоянно поддерживающие свою совместимость с другими участниками в ходе своей индивидуальной эволюции;}
    \item{способные децентрализованно координировать свою деятельность.}
\end{itemize}

Цель Экосистемы OSTIS - обеспечить постоянную поддержку совместимости компьютерных систем, входящих в Экосистему как на этапе их разработки, так и в ходе их эксплуатации. 
Проблема заключается в том, что в ходе эксплуатации систем, входящих в Экосистему OSTIS, они могут изменяться из-за чего совместимость может нарушаться. 
Задачами Экосистемы OSTIS являются:
\begin{itemize}
    \item{оперативное внедрение всех согласованных изменений стандарта ostis-систем (в том числе, и изменений систем используемых понятий и соответствующих им терминов);}
    \item{перманентная поддержка высокого уровня взаимопонимания всех систем, входящих в Экосистему OSTIS, и всех их пользователей;}
    \item{корпоративное решение различных сложных задач, требующих координации деятельности нескольких ostis-систем, а также, возможно, некоторых пользователей.}
\end{itemize}

Экосистема OSTIS – это переход от самостоятельных ostis-систем к коллективам самостоятельных ostis-систем, т.е. к распределенным ostis-системам.

\begin{figure}[htbp]
\begin{SCn}
\scnheader{ostis-система}
\begin{scnrelfromset}{разбиение}
    \scnitem{самостоятельная ostis-система}
    \scnitem{встроенная ostis-система}
    \scnitem{коллектив ostis-систем}
\end{scnrelfromset}
\end{SCn}
\end{figure}


\subsection{Поддержка совместимости между ostis-системами, входящими в состав Экосистемы OSTIS}
{\label{sec_ecosystem_interoperability_support}} 

Каждая система, входящая в состав Экосистемы OSTIS, должна:
\begin{itemize}
    \item{интенсивно, активно и целенаправленно обучаться, как с помощью учителей-разработчиков, так и самостоятельно;}
    \item{сообщать всем другим системам о предлагаемых или окончательно утвержденных изменениях в онтологиях и, в частности, в наборе используемых понятий;}
    \item{принимать от других ostis-систем предложения об изменениях в онтологиях, в том числе в наборе используемых понятий, для согласования или утверждения этих предложений;}
    \item{реализовывать утвержденные изменения в онтологиях, хранимых в ее базе знаний;}
    \item{способствовать поддержанию высокого уровня семантической совместимости не только с другими ostis-системами, входящими в Экосистему OSTIS, но и со своими пользователями (обучать их, информировать их об изменениях в онтологиях).}
\end{itemize}

К самостоятельным ostis-системам, входящим в состав Экосистемы OSTIS, предъявляются особые требования:
\begin{itemize}
    \item{они должны обладать всеми необходимыми знаниями и навыками для обмена сообщениями и целенаправленной организации взаимодействия с другими ostis-системами, входящими в Экосистему OSTIS;}
    \item{в условиях постоянного изменения и эволюции ostis-систем, входящих в Экосистему OSTIS, каждая из них должна сама следить за состоянием своей совместимости (согласованности) со всеми остальными ostis-системами, т.е. должна самостоятельно поддерживать эту совместимость, согласовывая с другими ostis-системами все требующие согласования изменения, происходящие у себя и в других системах.}
\end{itemize}

По назначению ostis-системы, входящие в Экосистему OSTIS, могут быть:
\begin{itemize}
    \item ассистентами конкретных пользователей или конкретных пользовательских коллективов;
    \item типовыми встраиваемыми подсистемами ostis-систем;
    \item системами информационной и инструментальной поддержки проектирования различных компонентов и различных классов ostis-систем;
    \item системами информационной и инструментальной поддержки проектирования или производства различных классов технических и других искусственно создаваемых систем;
    \item порталами знаний по самым различным научным дисциплинам;
    \item системами автоматизации управления различными сложными объектами (производственными предприятиями, учебными заведениями, кафедрами вузов, конкретными обучаемыми);
    \item интеллектуальными справочными и help-системами;
    \item интеллектуальными робототехническими системами.
\end{itemize}

Для обеспечения высокой эффективности эксплуатации и высоких темпов эволюции Экосистемы OSTIS необходимо постоянно повышать уровень информационной совместимости (уровень взаимопонимания) не только между компьютерными системами, входящими в состав Экосистемы OSTIS, но также между этими системами и их пользователями. 

Одним из направлений обеспечения такой совместимости является стремление к тому, чтобы база знаний (картина мира) каждого пользователя стала частью (фрагментом) Объединенной базы знаний Экосистемы OSTIS. 
Это значит, что каждый пользователь должен знать, как устроена структура каждой научно-технической дисциплины (объекты исследования, предметы исследования, определения, закономерности и т.д.), как могут быть связаны между собой различные дисциплины.

Поддержка совместимости Экосистемы OSTIS с ее пользователями осуществляется следующим образом:
\begin{itemize}
    \item в каждую ostis-систему включаются встроенные ostis-системы, ориентированные
    \begin{itemize}
        \item на перманентный мониторинг деятельности конечных пользователей и разработчиков этой ostis-системы,
        \item на анализ качества и, в первую очередь, корректности этой деятельности,
        \item на повышение квалификации пользователей (персонифицированное обучение);
    \end{itemize}
    \item в состав Экосистемы OSTIS включаются ostis-системы, специально предназначенные для обучения пользователей Экосистемы OSTIS базовым общепризнанным знаниям и навыкам решения соответствующих классов задач.
\end{itemize}

Экосистеме OSTIS ставится в соответствие ее объединенная база знаний, которая представляет собой виртуальное объединение баз знаний всех ostis-систем, входящих в состав Экосистемы OSTIS. 
Качество этой базы знаний (полнота, непротиворечивость, чистота) является постоянной заботой всех самостоятельных ostis-систем, входящих в состав Экосистемы OSTIS.


\subsection{Описание структуры Экосистемы OSTIS}
{\label{sec_ecosystem_structure_description}} 

Субъект, входящий в состав Экосистемы OSTIS, является агентом Экосистемы OSTIS.

\begin{figure}[htbp]
\begin{SCn}
\scnheader{агент Экосистемы OSTIS}
\begin{scnrelfromset}{разбиение}
    \scnitem{индивидуальная ostis-система Экосистемы OSTIS}
	\begin{scnindent}
    \begin{scnrelfromset}{разбиение}
        \scnitem{самостоятельная ostis-система Экосистемы OSTIS}
        \scnitem{встроенная ostis-система Экосистемы OSTIS}
    \end{scnrelfromset}
	\end{scnindent}
    \scnitem{ostis-сообщество}
	\begin{scnindent}
    \begin{scnrelfromset}{разбиение}
        \scnitem{простое ostis-сообщество}
        \scnitem{иерархическое ostis-сообщество}
    \end{scnrelfromset}
	\end{scnindent}
    \scnitem{пользователь Экосистемы OSTIS}
\end{scnrelfromset}
\end{SCn}
\end{figure}

Понятие ostis-сообщества представляет собой не только коллектив ostis-систем, но также определенный коллектив людей (пользователей и разработчиков соответствующих ostis-систем). 
ostis-сообщество является устойчивым фрагментом Экосистемы OSTIS, обеспечивающим комплексную автоматизацию определенной части коллективной человеческой деятельности и перманентное повышение ее эффективности. 
Иерархическим ostis-сообществом называется такое ostis-сообщество, по крайней мере одним из членов которого является некоторое другое ostis-сообщество.

Правила поведения агентов Экосистемы OSTIS:
\begin{itemize}
    \item Согласовывать денотационную семантику всех используемых знаков (в первую очередь понятий);
    \item Согласовывать терминологию, устранять противоречия и информационные дыры;
    \item Постоянно бороться с синонимией и омонимией как на уровне sc-элементов (внутренних знаков), так и на уровне соответствующих им терминов и прочих внешних идентификаторов (внешних обозначений);
    \item Каждый агент Экосистемы OSTIS по своей инициативе может стать членом любого ostis-сообщества Экосистемы OSTIS после соответствующей регистрации;
\end{itemize}

Все правила поведения агентов Экосистемы OSTIS должны соблюдаться не только ostis-системами, которые являются агентами Экосистемы OSTIS, но и людьми, являющиеся ими. 
Корректное поведение ostis-систем как агентов Экосистемы OSTIS значительно проще обеспечить, чем корректное поведение людей в качестве таких агентов. 
Поведение пользователей (естественных агентов) Экосистемы OSTIS необходимо внимательно мониторить и контролировать, постоянно способствуя повышению уровня их квалификации как агентов Экосистемы OSTIS, а также повышению уровня их мотивации, целенаправленности и самореализации.

Экосистема OSTIS является максимальным иерархическим ostis-сообществом, обеспечивающим комплексную автоматизацию всех видов человеческой деятельности. 
Оно не может входить в состав какого-либо другого ostis-сообщества. 
Принципы, лежащие в основе Экосистемы OSTIS:
\begin{itemize}
    \item Экосистема OSTIS представляет собой сеть ostis-сообществ;
    \item Каждому ostis-сообществу взаимно однозначно соответствует корпоративная ostis-система этого ostis-сообщества;
    \item Каждое ostis-сообщество может входить в состав любого другого ostis-сообщества по своей инициативе. Формально это означает, что корпоративная ostis-система первого ostis-сообщества является членом другого ostis-сообщества;
    \item Каждому специалисту, входящему в состав Экосистемы OSTIS ставится во взаимнооднозначное соответствие его персональный ostis-ассистент, который трактуется как корпоративная ostis-система вырожденного ostis-сообщества, состоящего из одного человека.
\end{itemize}

В Экосистеме OSTIS можно выделить следующие уровни иерархии:
\begin{itemize}
    \item индивидуальные компьютерные системы (индивидуальные ostis-системы и люди, являющиеся конечными пользователями ostis-систем);
    \item иерархическая система ostis-сообществ, членами каждого из которых могут быть индивидуальные ostis-системы, люди, а также другие ostis-сообщества;
    \item Максимальное ostis-сообщество Экосистемы OSTIS, не являющееся членом никакого другого ostis-сообщества, входящего в состав Экосистемы OSTIS.
\end{itemize}

Качество Экосистемы OSTIS во многом определяется эффективностью взаимодействия каждой ostis-системы (в том числе каждого ostis-сообщества), каждого человека со своей внешней средой, а также качеством и чистотой самой внешней среды. 
Потому основной целью Экосистемы OSTIS является повышение качества информационной внешней среды для всех субъектов, входящих в состав Экосистемы OSTIS.
Иными словами, Экосистема OSTIS обеспечивает поддержку Информационной экологии человеческого общества.

\begin{figure}[htbp]
\begin{SCn}
\scnheader{ostis-сообщество}
\begin{scnrelfromset}{разбиение}
    \scnitem{минимальное ostis-сообщество}
    \scnitem{коллектив ostis-систем}
\end{scnrelfromset}
\end{SCn}
\end{figure}

Каждой персоне, входящей в состав Экосистемы OSTIS взаимно однозначно соответствует его личный (персональный) ассистент в виде персонального ostis-ассистента. 
Таким образом, количество персональных ostis-ассистентов, входящих в состав Экосистемы OSTIS, совпадает с числом персон, входящих в состав Экосистемы OSTIS.
Пример персон и соответствующих им персональных ostis-ассистентов приведён на рисунке \ref{fig:ostis_assistant}.

% \begin{figure}[htbp]
%   \center
%   \includegraphics[scale=0.8]{figures/personal_ostis_assistant_example1.png}
%   \caption{Jack, Tom, and Sam as humans and their corresponding personal ostis-assistants}
%     \label{fig:ostis_assistant}
% \end{figure}

Коллектив, состоящий из персоны и соответствующего ей персонального ostis-ассистента, фактически является минимальным ostis-сообществом.
Пример минимальных ostis-сообществ приведён на рисунке \ref{fig:ostis_assistant}.

% \begin{figure}[htbp]
%   \center
%   \includegraphics[scale=0.8]{figures/corporate_ostis_system_example.png}
%   \caption{Sam\'s and Jack\'s communities as objects of the \textit{minimal ostis-community} class}
%     \label{fig:ostis_corporate}
% \end{figure}

Поскольку формально в не минимальные ostis-сообщества входят не персоны, а соответствующие им персональные ostis-ассистенты, все ostis-сообщества, кроме минимальных ostis-сообществ, являются коллективами ostis-систем.

Корпоративная ostis-система есть центральная ostis-система, осуществляющая координацию, организацию, а также поддержку эволюции деятельности членов соответствующего ostis-сообщества. 
Корпоративная ostis-система является представителем соответствующего ostis-сообщества в других ostis-сообществах, членом которых оно является.
Пример корпоративной ostis-системы сообщества представлен на рисунке \ref{fig:ostis_collectivity}

% \begin{figure}[htbp]
%   \center
%   \includegraphics[scale=0.8]{figures/ostis-system collectivity_example.png}
%   \caption{The chess club community with a corporate ostis-system}
%     \label{fig:ostis_community}
% \end{figure}

Основным назначением Корпоративной системы Экосистемы OSTIS является организация общего взаимодействия при выполнении самых различных видов и областей человеческой деятельности, которые могут быть либо полностью автоматизированными, либо частично автоматизированными, либо вообще неавтоматизированными. 
Из этого следует, что база знаний Корпоративной системы Экосистемы OSTIS должна содержать Общую формальную теорию человеческой деятельности, включающей в себя типологию видов и областей человеческой деятельности, а также общую методологию этой деятельности.

\begin{figure}[htbp]
\begin{SCn}
\scnheader{Деятельность в области Искусственного интеллекта, осуществляемая на основе Технологии OSTIS}
\scnrelfrom{основной продукт}{Экосистема OSTIS}
\begin{scnrelfromlist}{подпроект}
    \scnitem{Проект Метасистемы OSTIS}
    \scnitem{Проект программной реализации абстрактной sc-машины}
    \scnitem{Проект разработки универсального sc-компьютера}
\end{scnrelfromlist}
\end{SCn}
\end{figure}

Продуктом человеческой Деятельности в области Искусственного интеллекта, осуществляемой на основе Технологии OSTIS, является не просто множество ostis-систем различного назначения, а Экосистема, состоящая из взаимодействующих ostis-систем и их пользователей. 
Типология ostis-систем, являющихся агентом Экосистемы OSTIS, представлена ниже.

\begin{figure}[htbp]
\begin{SCn}
\scnheader{ostis-система, являющаяся агентом Экосистемы OSTIS}
\scnsuperset{персональный ostis-ассистент}
\scnsuperset{корпоративная ostis-система}
\scnsuperset{ostis-портал научно-технических знаний}
\scnsuperset{ostis-система автоматизации проектирования}
\scnsuperset{ostis-система автоматизации производства}
\scnsuperset{ostis-система автоматизации образовательной деятельности}
\begin{scnindent}
	\scnsuperset{обучающаяся ostis-система}
	\scnsuperset{корпоративная ostis-система виртуальной кафедры}
\end{scnindent}
\scnsuperset{ostis-система автоматизации бизнес-деятельности}
\scnsuperset{ostis-система автоматизации управления}
\begin{scnindent}
	\scnsuperset{ostis-система управления проектами соответствующего вида}
	\scnsuperset{ostis-система сенсомоторной координации при выполнении определенного вида сложных действий во внешней среде}
    \begin{scnindent}
        \scnsuperset{ostis-система управления самостоятельным перемещением} 
		\scnsuperset{робота по пересеченной местности}
    \end{scnindent}
\end{scnindent}
\end{SCn}
\end{figure}
