\section{Персональные ostis-ассистенты пользователей}
{\label{sec_ostis_assistant}} 

\begin{SCn}

\bigskip

\begin{scnrelfromlist}{ключевое понятие}
    \scnitem{...}
\end{scnrelfromlist}

\bigskip

\begin{scnrelfromlist}{ключевое знание}
    \scnitem{...}
\end{scnrelfromlist}

\bigskip

\begin{scnrelfromlist}{библиографическая ссылка}
    \scnitem{\scncite{...}}
\end{scnrelfromlist}

\end{SCn}

Общество должно повернуться "лицом"{} к каждому человеку, нести ответственность и действительно способствовать каждому человеку персонально, адаптруясь к его особенностям, обеспечить:
\begin{textitemize}
    \item максимальный уровень физического здоровья, активности и максимальное долголетие;
    \item максимальный уровень физического комфорта, личного пространства, материального потребления;
    \item максимальный уровень социального, административно-правового комфорта.
\end{textitemize}

Для этого должен осуществляться:
\begin{textitemize}
    \item персональный мониторинг каждой личности по всем направлениям;
    \item диагностика и устранение нежелательных отклонений;
    \item оказание своевременной персональной помощи в уточнении направлений дальнейшей эволюции каждой личности.
\end{textitemize}

Необходимо перейти от оказания услуг в решении различных проблем по инициативе самих лиц, столкнувшихся с этими проблемами, к своевременному обнаружению возможности возникновения этих проблем и к соответствующей профилактике. 
Это возможно только при наличии четкой системной организации персонального мониторинга. 

Клиент не обязан знать множество сервисов, из которых он должен выбирать подходящий ему функционал. 
Для клиента комплекс семантически совместимых сервисов должен располагаться "за кадром"{}. 
Следовательно, все используемые клиентом информационные ресурсы и сервисы должны быть семантически совместимы. 
Выбор подходящего для пользователя ресурса или сервиса должен производить его персональный ассистент.

Персональный ассистент должен учитывать, что роли клиентов могут меняться, расширяться, как и его интересы и цели. 
При этом, все персональные ассистенты должны быть семантически совместимыми с целью понимания друг друга, а также обладать способностью самостоятельно взаимодействовать в рамках различных корпоративных систем, представляя интересы своих клиентов.

Персональный ostis-ассистент есть ostis-система, являющаяся персональным ассистентом пользователя в рамках Экосистемы OSTIS.
Такая система предоставляет возможности:
\begin{textitemize}
    \item анализа деятельности пользователя и формирования рекомендаций по ее оптимизации;
    \item адаптации под настроение пользователя, его личностные качества, общую окружающую обстановку, задачи, которые чаще всего решает пользователь;
    \item перманентного обучения самого ассистента в процессе решения новых задач, при этом обучаемость потенциально не ограничена;
    \item вести диалог с пользователем на естественном языке, в том числе в речевой форме;
    \item отвечать на вопросы различных классов, при этом если системе что-то не понятно, то она сама может задавать встречные вопросы;
    \item автономного получения информации от всей окружающей среды, а не только от пользователя (в текстовой или речевой форме).
\end{textitemize}

При этом система может как анализировать доступные информационные источники (например, в интернете), так и анализировать окружающий ее физический мир, например, окружающие предметы или внешний вид пользователя.

Достоинства персонального ostis-ассистента:
\begin{textitemize}
    \item пользователю нет необходимости хранить разную информацию в разной форме в разных местах, вся информация хранится в единой базе знаний компактно и без дублирований;
    \item благодаря неограниченной обучаемости ассистенты могут потенциально автоматизировать практически любую деятельность, а не только самую рутинную;
    \item благодаря базе знаний, ее структуризации и средствам поиска информации в базе знаний пользователь может получить более точную информацию более быстро.
\end{textitemize}

Персональные ассистенты имеют самое различное назначение и могут быть использованы для самых различных категорий пользователей (пациент, юридическое обслуживание, административное обслуживание, покупатель, потребитель различных услуг). 
