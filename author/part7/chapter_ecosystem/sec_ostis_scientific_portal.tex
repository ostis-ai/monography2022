\section{Семантически совместимые интеллектуальные ostis-порталы знаний}
{\label{sec_ostis_scientific_portal}} 

\begin{SCn}
\begin{scnrelfromlist}{ключевое понятие}
    \scnitem{портал знаний}
    \scnitem{ostis-портал знаний}
\end{scnrelfromlist}

\bigskip

\begin{scnrelfromlist}{библиографическая ссылка}
    \scnitem{\scncite{Van2005}}
    \scnitem{\scncite{Mack2001}}
\end{scnrelfromlist}

\end{SCn}

Понятие \textit{портала знаний} представляет собой один из способов создания централизованного доступа к информации, которая может быть необходима для решения задач, связанных с работой в организации. Такие порталы могут содержать информацию, связанную с процессами, документацией, процедурами, обучающими материалами, а также ответы на часто задаваемые вопросы (см. \scncite{Van2005}, \scncite{Mack2001}).

Одним из ключевых преимуществ \textit{порталов знаний} является их способность к сбору и хранению информации из различных источников, таких как базы данных, системы управления документами, системы управления проектами и так далее. Это позволяет пользователям получать полную и актуальную информацию в одном месте.

На основе \textit{портала знаний} обеспечивается возможность взаимодействия между пользователями путем создания форумов, обсуждений и коллективного редактирования документов. Это может способствовать обмену знаниями и опытом между сотрудниками организации и повысить эффективность их работы.

При создании \textit{порталов знаний} возникают проблемы, связанные с организацией и управлением информацией. Например, необходимо обеспечить корректное и структурированное хранение информации, ее поиск и обновление. Также необходимо учитывать потребности пользователей и обеспечить удобный и интуитивно понятный интерфейс.

Целями интеллектуального портала знаний являются:
\begin{textitemize}
    \item ускорение погружения каждого человека в новые для него области при постоянном сохранении общей целостной картины Мира (образовательная цель);
    \item фиксация в систематизированном виде новых результатов так, чтобы все основные связи новых результатов с известными были четко обозначены;
    \item автоматизация координации работ по рецензированию новых результатов;
    \item автоматизация анализа текущего состояния базы знаний.
\end{textitemize}

Создание интеллектуальных \textit{порталов знаний}, обеспечивающих повышение темпов интеграции и согласования различных точек зрения, – это способ существенного повышения темпов эволюции научно-технической деятельности.
Совместимые \textit{порталы знаний}, реализованные в виде \textit{ostis-систем}, входящих в \textit{Экосистему OSTIS}, являются основой новых принципов организации научной деятельности, в которой результатами этой деятельности являются не статьи, монографии, отчеты и другие научно-технические документы, а фрагменты глобальной \textit{базы знаний}, разработчиками которых являются свободно формируемые научные коллективы, состоящие из специалистов в соответствующих научных дисциплинах. 
С помощью \textit{ostis-порталов знаний} осуществляется как координация процесса рецензирования новой научно-технической информации, поступающей от научных работников в \textit{базы знаний} этих порталов, так и процесс согласования различных точек зрения ученых (в частности, введению и семантической корректировке понятий, а также введению и корректировке терминов, соответствующих различным сущностям).

Реализация семейства семантически совместимых \textit{ostis-порталов знаний} в виде совместимых \textit{ostis-систем}, входящих в состав \textit{Экосистемы OSTIS}, предполагает разработку иерархической системы семантически согласованных формальных онтологий, соответствующих различным дисциплинам, с четко заданным наследованием свойств описываемых сущностей и с четко заданными междисциплинарными связями, которые описываются связями между соответствующими формальными онтологиями и специфицируемыми ими предметными областями.

Реализация \textit{ostis-порталов знаний} на основе \textit{Технологии OSTIS} предоставляет ряд преимуществ по сравнению с другими подходами. Ниже приведены некоторые из них:
\begin{textitemize}
    \item использование методов семантической обработки информации, что позволяет более точно и эффективно организовывать и искать информацию на портале знаний;
    \item высокий уровень гибкости и расширяемости, что позволяет адаптировать \textit{ostis-порталы знаний} под различные нужды и требования пользователей;
    \item автоматическая интеграция \textit{ostis-порталов знаний} с другими \textit{ostis-системами} в рамках \textit{Экосистемы OSTIS}, что позволяет создать централизованный доступ к информации из различных источников.
    \item возможность создания персонализированного \textit{ostis-портала знаний}, который учитывает интересы и потребности каждого пользователя, что позволяет более эффективно использовать знания \textit{ostis-систем};
    \item возможность производить \textit{ostis-порталы знаний} быстро и с минимальными затратами благодаря использованию существующих компонентов и инструментов.
\end{textitemize}

Примером \textit{портала знаний}, построенного в виде \textit{ostis-системы} является \textit{Метасистема OSTIS}, содержащая все известные на текущий момент знания и навыки, входящие в состав \textit{Технологии OSTIS}.

Таким образом, реализация \textit{порталов знаний} на основе \textit{Технологии OSTIS} позволяет создать более эффективную и гибкую систему для хранения, организации и поиска знаний, которая может быть адаптирована под различные требования пользователей и организаций.
