\chapter{Структура Экосистемы OSTIS}
\chapauthortoc{Загорский А.Г.\\Голенков В.В.\\Шункевич Д.В.}
{\label{chapter_ecosystem}}

\vspace{-7\baselineskip}

\begin{SCn}
\begin{scnrelfromlist}{автор}
	\scnitem{Загорский А.С.}
	\scnitem{Голенков В.В.}
	\scnitem{Шункевич Д.В.}
\end{scnrelfromlist}

\bigskip

\scntext{аннотация}{Рассмотрена архитектура экосистемы интеллектуальных компьютерных систем на основе Технологии OSTIS. Уточнена формальная трактовка таких понятий как ostis-система, ostis-сообщество, выделена типология ostis-систем, что в совокупности позволяет определить структуру Экосистемы OSTIS.}

\bigskip

\begin{scnrelfromlist}{подраздел}
	\scnitem{\ref{sec_ecosystem_structure}~\nameref{sec_ecosystem_structure}}
	\scnitem{\ref{sec_human_activity_automation}~\nameref{sec_human_activity_automation}}
	\scnitem{\ref{sec_ecosystem_automation_principles}~\nameref{sec_ecosystem_automation_principles}}
	\scnitem{\ref{sec_ostis_scientific_portal}~\nameref{sec_ostis_scientific_portal}}
	\scnitem{\ref{sec_corporate_ostis_system}~\nameref{sec_corporate_ostis_system}}
	\scnitem{\ref{sec_ostis_assistant}~\nameref{sec_ostis_assistant}}
\end{scnrelfromlist}

\end{SCn}

% Добавить описание, про что вообще глава, сделать введение более плавным.  
Понятие архетипа сети сводится к объединению множества автономных объектов друг с другом. Эти объекты сильно связны друг с другом, и при этом не имеют какого-либо центра. Таким образом, они образуют децентрализованную сеть, в которой отсутствует единый центр управления.
% \cite{Briscoe_agents_evolving_2012}

Можно выделить следующие характеристики такой системы:
\begin{textitemize}
    \item отсутствие или слабовыраженность централизованного управления;
    \item автономная природа участников, объектов такой сети;
    \item сильная связность участников такой сети друг с другом;
    \item влияние участников такой сети друг на друга нелинейно и достаточно сложно.
\end{textitemize}

У таких распределённых искусственных систем выделяются как преимущества (высокий уровень адаптивности, устойчивости, связности), так и недостатки (неоптимальность, неуправляемость, непредсказуемость поведения). Наиболее подходящим примером реализованной технологии на основе концепции сети является интернет.

Архетип сети удобно использовать для отображения сложных процессов, взаимозависимость компонентов, экономических, социальных, экологических процессов, процессов коммуникации. В таких процессах нет начала или конца, всё является центром. Сеть является единственной топологией, способной к безграничному расширению или самостоятельному обучению, остальные топологии имеют собственные ограничения. "The Atom is the icon of 20th century science. The symbol of science for the next century is the dynamical Net"{}.
% \cite{Kelly_out_of_control_1995}

Концепция экосистемы стала популярным способом описания взаимодействия самостоятельно действующих систем во внешней среде. 
% \cite{Boley_ecosystem_principles_2007}
Экосистемы имеют две отличительные характеристики по сравнению с другими концепциями сотрудничества: взаимодополняемость и взаимозависимость присутствуют одновременно, и система не полностью иерархически контролируется.
% \cite{Masahary_ecosystem_concept_2019}

При традиционных подходах к решению проблемы формирования экосистемы возникают проблемы, связанные с низким уровнем интероперабельности таких систем.
% \cite{li2012problems}
Зачастую каждая из систем будет иметь свой специализированный программный интерфейс и формат данных для общения с ней, что ведёт к дополнительным расходам на устранение недостатков таких проблем. Поддержка жизненного цикла, модификация уже существующих систем может накладывать дополнительные временные и ресурсные затраты. 

Ключевым направлением повышения уровня интеллекта индивидуальных интеллектуальных кибернетических систем --- это переход от абсолютно независимых друг от друга индивидуальных интеллектуальных кибернетических систем к их универсальным сообществам, т.е. к многоагентным системам, самостоятельными агентнами которых являются указанные индивидуальные интеллектуальные кибернетические системы.
В рамках таких систем обеспечивается возможность коммуникации каждого агента с каждым, а также обеспечивается возможность формирования специализированных коллективов для коллективного решения сложных коллективных задач.
Реализация указанного универсального сообщества интероперабельных интеллектуальных кибернетических систем осуществляется в виде Глобальной Экосистемы OSTIS. 

\section{Иерархическая система взаимодействующих ostis-сообществ}
{\label{sec_ecosystem_structure}} 

\begin{SCn}


\begin{scnrelfromlist}{ключевое понятие}
    \scnitem{ostis-система}
    \scnitem{самостоятельная ostis-система}
    \scnitem{встроенная ostis-система}
    \scnitem{коллектив ostis-систем}
\end{scnrelfromlist}

\begin{scnrelfromlist}{подраздел}
	\scnitem{\ref{sec_ecosystem_interoperability_support}~\nameref{sec_ecosystem_interoperability_support}}
	\scnitem{\ref{sec_ecosystem_structure_description}~\nameref{sec_ecosystem_structure_description}}
\end{scnrelfromlist}

\end{SCn}

\subsection*{Введение в \ref{sec_ecosystem_structure}}
Для создания успешной \textit{цифровой экосистемы} необходимо решать множество проблем, связанных с обеспечением высокого уровня интероперабельности между самостоятельно действующими системами. Одним из возможных решений является переход к универсальным сообществам индивидуальных \textit{интеллектуальных кибернетических систем}, которые объединяются в \textit{многоагентные системы}.

Реализация такого универсального сообщества интероперабельных \textit{интеллектуальных кибернетических систем} может осуществляться в виде глобальной \textit{Экосистемы OSTIS}. 
\textit{Экосистема OSTIS} --- социотехническая экосистема, представляющая собой коллектив взаимодействующих \textit{семантических компьютерных систем} и осуществляющая перманентную поддержку эволюции и семантической совместимости всех входящих в нее систем, на протяжении всего их жизненного цикла. 

Система, построенная в соответствии с требованиями и стандартами \textit{Технологии OSTIS}, определяется как \textit{ostis-система}. 

\textit{Экосистема OSTIS} представляет собой коллектив взаимодействующих:
\begin{textitemize}
    \item самих \textit{ostis-систем};
    \item пользователей указанных \textit{ostis-систем} (как конечных пользователей, так и разработчиков);
    \item некоторых \textit{компьютерных систем}, не являющихся \textit{ostis-системами}, но рассматриваемых ими в качестве дополнительных \textit{информационных ресурсов} или \textit{сервисов}.
\end{textitemize}

В рамках \textit{Экосистемы OSTIS} \textit{ostis-системы} способны коммуницировать друг с другом и формировать специализированные коллективы для коллективного решения сложных задач. Такой подход не только повышает уровень интеллекта каждой \textit{индивидуальной кибернетической системы}, но и обеспечивает более эффективное взаимодействие между ними в рамках единой \textit{цифровой экосистемы}. Это обеспечивает существенное развитие целого ряда свойств каждой \textit{компьютерной системы}, позволяющих значительно повысить \textit{уровень интеллекта} (и, прежде всего, их \textit{уровень обучаемости} и \textit{уровень социализации}). 

Участники коллектива \textit{Экосистемы OSTIS} характеризуются как:
\begin{textitemize}
    \item \textit{семантически совместимые};
    \item постоянно эволюционирующие индивидуально;
    \item постоянно поддерживающие свою совместимость с другими участниками в ходе своей индивидуальной эволюции;
    \item способные децентрализованно координировать свою деятельность.
\end{textitemize}

\textit{Экосистема OSTIS} – это переход от \textit{самостоятельных ostis-систем} к \textit{коллективам ostis-систем}, то есть к распределенным \textit{ostis-системам}.

\begin{SCn}
\scnheader{ostis-система}
\begin{scnrelfromset}{разбиение}
    \scnitem{самостоятельная ostis-система}
    \scnitem{встроенная ostis-система}
    \scnitem{коллектив ostis-систем}
\end{scnrelfromset}
\end{SCn}

Цель \textit{Экосистемы OSTIS} --- \myuline{обеспечить постоянную поддержку совместимости компьютерных систем}, входящих в \textit{Экосистему OSTIS} как на этапе их разработки, так и в ходе их эксплуатации. 
Проблема заключается в том, что в ходе эксплуатации систем, входящих в \textit{Экосистему OSTIS}, они могут изменяться из-за чего совместимость может нарушаться. 
\begin{SCn}
    \scnheader{Экосистема OSTIS}
    \begin{scnrelfromlistcustom}{задачи}
        \scnitemcustom{оперативное внедрение всех согласованных изменений стандарта ostis-систем (в том числе, и изменений систем используемых понятий и соответствующих им терминов)}
        \scnitemcustom{перманентная поддержка высокого уровня взаимопонимания всех систем, входящих в \textit{Экосистему OSTIS}, и всех их пользователей}
        \scnitemcustom{корпоративное решение различных комплексных задач, требующих координации деятельности нескольких ostis-систем, а также, возможно, некоторых пользователей}
    \end{scnrelfromlistcustom}
\end{SCn}


\subsection{Поддержка совместимости между ostis-системами, входящими в состав Экосистемы OSTIS}
{\label{sec_ecosystem_interoperability_support}} 

Каждая система, входящая в состав \textit{Экосистемы OSTIS}, должна:
\begin{textitemize}
    \item интенсивно, активно и целенаправленно обучаться, как с помощью учителей-разработчиков, так и самостоятельно;
    \item сообщать всем другим системам о предлагаемых или окончательно утвержденных изменениях в онтологиях и, в частности, в наборе используемых понятий;
    \item принимать от других ostis-систем предложения об изменениях в онтологиях, в том числе в наборе используемых понятий, для согласования или утверждения этих предложений;
    \item реализовывать утвержденные изменения в онтологиях, хранимых в ее базе знаний;
    \item способствовать поддержанию высокого уровня семантической совместимости не только с другими ostis-системами, входящими в \textit{Экосистему OSTIS}, но и со своими пользователями (обучать их, информировать их об изменениях в онтологиях).
\end{textitemize}

К \textit{самостоятельной ostis-системе}, входящей в состав \textit{Экосистемы OSTIS}, предъявляются особые требования:
\begin{textitemize}
    \item она должны обладать всеми необходимыми знаниями и навыками для обмена сообщениями и целенаправленной организации взаимодействия с другими \textit{ostis-системами}, входящими в \textit{Экосистему OSTIS};
    \item в условиях постоянного изменения и эволюции \textit{ostis-систем}, входящих в \textit{Экосистему OSTIS}, она должна сама следить за состоянием своей совместимости (согласованности) со всеми остальными \textit{ostis-системами}, то есть должна самостоятельно поддерживать эту совместимость, согласовывая с другими \textit{ostis-системами} все требующие согласования изменения, происходящие у себя и в других системах.
\end{textitemize}

Для обеспечения высокой эффективности эксплуатации и высоких темпов эволюции \textit{Экосистемы OSTIS} необходимо постоянно повышать уровень информационной совместимости (уровень взаимопонимания) не только между компьютерными системами, входящими в состав \textit{Экосистемы OSTIS}, но также между этими системами и их пользователями. 

Одним из направлений обеспечения такой совместимости является стремление к тому, чтобы база знаний, картина мира каждого пользователя стала частью объединенной базы знаний \textit{Экосистемы OSTIS}. Это значит, что каждый пользователь должен знать, как устроена структура каждой научно-технической дисциплины (объекты исследования, предметы исследования, определения, закономерности и так далее), как могут быть связаны между собой различные дисциплины.

Поддержка совместимости \textit{Экосистемы OSTIS} с ее пользователями осуществляется следующим образом:
\begin{textitemize}
    \item в каждую \textit{ostis-систему} включаются встроенные \textit{ostis-системы}, ориентированные
    \begin{textitemize}
        \item на перманентный мониторинг деятельности конечных пользователей и разработчиков этой \textit{ostis-системы},
        \item на анализ качества и, в первую очередь, корректности этой деятельности,
        \item на повышение квалификации пользователей (персонифицированное обучение);
    \end{textitemize}
    \item в состав \textit{Экосистемы OSTIS} включаются \textit{ostis-системы}, специально предназначенные для обучения пользователей \textit{Экосистемы OSTIS} базовым общепризнанным знаниям и навыкам решения соответствующих классов задач.
\end{textitemize}

\textit{Экосистеме OSTIS} ставится в соответствие ее объединенная база знаний, которая представляет собой виртуальное объединение баз знаний всех ostis-систем, входящих в состав \textit{Экосистемы OSTIS}.
Качество этой базы знаний (полнота, непротиворечивость, чистота) является постоянной заботой всех \textit{самостоятельных ostis-систем}, входящих в состав \textit{Экосистемы OSTIS}.


\subsection{Описание структуры Экосистемы OSTIS}
{\label{sec_ecosystem_structure_description}} 

\begin{SCn}
\begin{scnrelfromlist}{ключевой знак}
    \scnitem{Корпоративная система Экосистемы OSTIS}
\end{scnrelfromlist}

\begin{scnrelfromlist}{ключевое понятие}
    \scnitem{агент Экосистемы OSTIS}
    \scnitem{пользователь Экосистемы OSTIS}
    \scnitem{ostis-сообщество}
\end{scnrelfromlist}
\end{SCn}

Субъект, входящий в состав \textit{Экосистемы OSTIS}, является \textit{агентом Экосистемы OSTIS}.

\begin{SCn}
\scnheader{агент Экосистемы OSTIS}
\begin{scnrelfromset}{разбиение}
    \scnitem{индивидуальная ostis-система Экосистемы OSTIS}
	\begin{scnindent}
    \begin{scnrelfromset}{разбиение}
        \scnitem{самостоятельная ostis-система Экосистемы OSTIS}
        \scnitem{встроенная ostis-система Экосистемы OSTIS}
    \end{scnrelfromset}
	\end{scnindent}
    \scnitem{пользователь Экосистемы OSTIS}
    \scnitem{ostis-сообщество}
	\begin{scnindent}
    \begin{scnrelfromset}{разбиение}
        \scnitem{простое ostis-сообщество}
        \scnitem{иерархическое ostis-сообщество}
    \end{scnrelfromset}
	\end{scnindent}
\end{scnrelfromset}
\end{SCn}

Понятие \textit{ostis-сообщества} представляет собой не только \textit{коллектив ostis-систем}, но также определенный коллектив людей (пользователей и разработчиков соответствующих \textit{ostis-систем}). 
\textit{ostis-сообщество} является устойчивым фрагментом \textit{Экосистемы OSTIS}, обеспечивающим комплексную автоматизацию определенной части коллективной человеческой деятельности и перманентное повышение ее эффективности. 
\textit{иерархическим ostis-сообществом} называется такое \textit{ostis-сообщество}, по крайней мере одним из членов которого является некоторое другое \textit{ostis-сообщество}.

Правила поведения \textit{агентов Экосистемы OSTIS}:
\begin{textitemize}
    \item Согласовывать денотационную семантику всех используемых знаков (в первую очередь понятий);
    \item Согласовывать терминологию, устранять противоречия и информационные дыры;
    \item Постоянно бороться с синонимией и омонимией как на уровне sc-элементов (внутренних знаков), так и на уровне соответствующих им терминов и прочих внешних идентификаторов (внешних обозначений);
    \item Каждый \textit{агент Экосистемы OSTIS} по своей инициативе может стать членом любого ostis-сообщества Экосистемы OSTIS после соответствующей регистрации;
\end{textitemize}

Все правила поведения \textit{агентов Экосистемы OSTIS} должны соблюдаться не только \textit{ostis-системами}, которые являются агентами \textit{Экосистемы OSTIS}, но и людьми, являющиеся ими. 
Корректное поведение \textit{ostis-систем} как \textit{агентов Экосистемы OSTIS} значительно проще обеспечить, чем корректное поведение людей в качестве таких агентов. 
Поведение пользователей (естественных агентов) \textit{Экосистемы OSTIS} необходимо внимательно мониторить и контролировать, постоянно способствуя повышению уровня их квалификации как \textit{агентов Экосистемы OSTIS}, а также повышению уровня их мотивации, целенаправленности и самореализации.

\textit{Экосистема OSTIS} является максимальным \textit{иерархическим ostis-сообществом}, обеспечивающим комплексную автоматизацию всех видов человеческой деятельности. 
Оно не может входить в состав какого-либо другого \textit{ostis-сообщества}. 
Принципы, лежащие в основе \textit{Экосистемы OSTIS}:
\begin{textitemize}
    \item \textit{Экосистема OSTIS} представляет собой сеть \textit{ostis-сообществ};
    \item Каждому \textit{ostis-сообществу} взаимно однозначно соответствует \textit{корпоративная ostis-система} этого \textit{ostis-сообщества};
    \item Каждое \textit{ostis-сообщество} может входить в состав любого другого \textit{ostis-сообщества} по своей инициативе. Формально это означает, что \textit{корпоративная ostis-система} первого \textit{ostis-сообщества} является членом другого \textit{ostis-сообщества};
    \item Каждому специалисту, входящему в состав \textit{Экосистемы OSTIS} ставится во взаимнооднозначное соответствие его \textit{персональный ostis-ассистент}, который трактуется как \textit{корпоративная ostis-система} вырожденного \textit{ostis-сообщества}, состоящего из одного человека.
\end{textitemize}

В \textit{Экосистеме OSTIS} можно выделить следующие уровни иерархии:
\begin{textitemize}
    \item индивидуальные компьютерные системы (индивидуальные \textit{ostis-системы} и люди, являющиеся конечными пользователями \textit{ostis-систем});
    \item иерархическая система \textit{ostis-сообществ}, членами каждого из которых могут быть \textit{индивидуальные ostis-системы}, люди, а также другие \textit{ostis-сообщества};
    \item \textit{Максимальное ostis-сообщество} \textit{Экосистемы OSTIS}, не являющееся членом никакого другого \textit{ostis-сообщества}, входящего в состав \textit{Экосистемы OSTIS}.
\end{textitemize}

Качество \textit{Экосистемы OSTIS} во многом определяется эффективностью взаимодействия каждой \textit{ostis-системы} (в том числе каждого \textit{ostis-сообщества}), каждого человека со своей внешней средой, а также качеством и чистотой самой внешней среды. 
Потому основной целью \textit{Экосистемы OSTIS} является повышение качества информационной внешней среды для всех субъектов, входящих в состав \textit{Экосистемы OSTIS}.
Иными словами, \textit{Экосистема OSTIS} обеспечивает поддержку информационной экологии человеческого общества.

\begin{SCn}
\scnheader{ostis-сообщество}
\begin{scnrelfromset}{разбиение}
    \scnitem{минимальное ostis-сообщество}
    \scnitem{коллектив ostis-систем}
\end{scnrelfromset}
\end{SCn}

Каждой персоне, входящей в состав \textit{Экосистемы OSTIS} взаимно однозначно соответствует его личный (персональный) ассистент в виде \textit{персонального ostis-ассистента}.
Таким образом, количество \textit{персональных ostis-ассистентов}, входящих в состав \textit{Экосистемы OSTIS}, совпадает с числом персон, входящих в состав \textit{Экосистемы OSTIS}.
% Пример персон и соответствующих им \textit{персональных ostis-ассистентов} приведён на рисунке \ref{fig:ostis_assistant}.

% \begin{figure}[htbp]
%   \center
%   \includegraphics[scale=0.8]{figures/personal_ostis_assistant_example1.png}
%   \caption{Jack, Tom, and Sam as humans and their corresponding personal ostis-assistants}
%     \label{fig:ostis_assistant}
% \end{figure}

Коллектив, состоящий из персоны и соответствующего ей \textit{персонального ostis-ассистента}, фактически является \textit{минимальным ostis-сообществом}.
% Пример минимальных ostis-сообществ приведён на рисунке \ref{fig:ostis_assistant}.

% \begin{figure}[htbp]
%   \center
%   \includegraphics[scale=0.8]{figures/corporate_ostis_system_example.png}
%   \caption{Sam\'s and Jack\'s communities as objects of the \textit{minimal ostis-community} class}
%     \label{fig:ostis_corporate}
% \end{figure}

Поскольку формально в не \textit{минимальные ostis-сообщества} входят не персоны, а соответствующие им \textit{персональные ostis-ассистенты}, все \textit{ostis-сообщества}, кроме \textit{минимальных ostis-сообществ}, являются \textit{коллективами ostis-систем}.

\textit{корпоративная ostis-система} есть центральная \textit{ostis-система}, осуществляющая координацию, организацию, а также поддержку эволюции деятельности членов соответствующего \textit{ostis-сообщества}. 
\textit{корпоративная ostis-система} является представителем соответствующего \textit{ostis-сообщества} в других \textit{ostis-сообществах}, членом которых оно является.
% Пример корпоративной ostis-системы сообщества представлен на рисунке \ref{fig:ostis_collectivity}

% \begin{figure}[htbp]
%   \center
%   \includegraphics[scale=0.8]{figures/ostis-system collectivity_example.png}
%   \caption{The chess club community with a corporate ostis-system}
%     \label{fig:ostis_community}
% \end{figure}

Основным назначением \textit{Корпоративной системы Экосистемы OSTIS} является организация общего взаимодействия при выполнении самых различных видов и областей человеческой деятельности, которые могут быть либо полностью автоматизированными, либо частично автоматизированными, либо вообще неавтоматизированными. 
Из этого следует, что база знаний \textit{Корпоративной системы Экосистемы OSTIS} должна содержать \textit{Общую формальную теорию человеческой деятельности}, включающей в себя типологию видов и областей человеческой деятельности, а также общую методологию этой деятельности.

\begin{SCn}
\scnheader{Деятельность в области Искусственного интеллекта, осуществляемая на основе Технологии OSTIS}
\scnrelfrom{основной продукт}{Экосистема OSTIS}
\begin{scnrelfromlist}{подпроект}
    \scnitem{Проект Метасистемы OSTIS}
    \scnitem{Проект программной реализации абстрактной sc-машины}
    \scnitem{Проект разработки универсального sc-компьютера}
\end{scnrelfromlist}
\end{SCn}

Продуктом человеческой деятельности в области Искусственного интеллекта, осуществляемой на основе \textit{Технологии OSTIS}, является не просто множество \textit{ostis-систем} различного назначения, а Экосистема, состоящая из взаимодействующих \textit{ostis-систем} и их пользователей. 

По назначению \textit{ostis-системы}, входящие в \textit{Экосистему OSTIS}, могут быть:
\begin{textitemize}
    \item ассистентами конкретных пользователей или конкретных пользовательских коллективов;
    \item типовыми встраиваемыми подсистемами \textit{ostis-систем};
    \item системами информационной и инструментальной поддержки проектирования различных компонентов и различных классов \textit{ostis-систем};
    \item системами информационной и инструментальной поддержки проектирования или производства различных классов технических и других искусственно создаваемых систем;
    \item порталами знаний по самым различным научным дисциплинам;
    \item системами автоматизации управления различными сложными объектами (производственными предприятиями, учебными заведениями, кафедрами вузов, конкретными обучаемыми);
    \item интеллектуальными справочными и help-системами;
    \item интеллектуальными робототехническими системами.
\end{textitemize}
Типология \textit{ostis-систем}, являющихся агентом Экосистемы OSTIS, представлена ниже.

\begin{SCn}
\scnheader{ostis-система, являющаяся агентом Экосистемы OSTIS}
\scnsuperset{персональный ostis-ассистент}
\scnsuperset{корпоративная ostis-система}
\scnsuperset{ostis-портал знаний}
\scnsuperset{ostis-система автоматизации проектирования}
\scnsuperset{ostis-система автоматизации производства}
\scnsuperset{ostis-система автоматизации образовательной деятельности}
\begin{scnindent}
	\scnsuperset{обучающаяся ostis-система}
	\scnsuperset{корпоративная ostis-система виртуальной кафедры}
\end{scnindent}
\scnsuperset{ostis-система автоматизации бизнес-деятельности}
\scnsuperset{ostis-система автоматизации управления}
\begin{scnindent}
	\scnsuperset{ostis-система управления проектами соответствующего вида}
	\scnsuperset{ostis-система сенсомоторной координации при выполнении определенного вида сложных действий во внешней среде}
    \begin{scnindent}
        \scnsuperset{ostis-система управления самостоятельным перемещением} 
		\scnsuperset{робота по пересеченной местности}
    \end{scnindent}
\end{scnindent}
\end{SCn}

\section{Автоматизация человеческой деятельности в области Искусственного интеллекта, осуществляемая в рамках Экосистемы OSTIS}
{\label{sec_human_activity_automation}} 

\begin{SCn}

\bigskip

\begin{scnrelfromlist}{ключевое понятие}
    \scnitem{...}
\end{scnrelfromlist}

\bigskip

\begin{scnrelfromlist}{ключевое знание}
    \scnitem{...}
\end{scnrelfromlist}

\bigskip

\begin{scnrelfromlist}{библиографическая ссылка}
    \scnitem{\scncite{...}}
\end{scnrelfromlist}

\end{SCn}

Экосистема OSTIS представляет собой саморазвивающуюся сеть ostis-систем, которая обеспечивает комплексную автоматизацию всевозможных видов и областей человеческой деятельности. 
Особое место среди ostis-систем, входящих в состав Экосистемы OSTIS, занимают корпоративные ostis-системы, через которое осуществляется координация и эволюция деятельности некоторых групп ostis-систем и их пользователей. 
Основная цель корпоративных ostis-систем – локализовать базы знаний указанных групп ostis-систем, перевести их из статуса виртуальных в статус реальных и автоматизировать их эволюцию.

Экосистема OSTIS является следующим этапом развития человеческого общества, обеспечивающий существенное повышение уровня общественного (коллективного) интеллекта путем преобразования человеческого общества в экосистему, состоящую из людей и семантически совместимых интеллектуальных систем. 
Экосистема OSTIS --- предлагаемый подход к реализации smart-общества или Общества 5.0, построенного на основе Технологии OSTIS.

Сверхзадачей Экосистемы OSTIS является не просто комплексная автоматизация всех видов человеческой деятельности (разумеется, только тех видов деятельности, автоматизация которых целесообразна), но и существенное повышение уровня интеллекта различных человеческих (точнее человеко-машинных) сообществ и всего человеческого общества в целом.
%\section{Принципы автоматизации различных видов и областей человеческой деятельности в рамках Экосистемы OSTIS}
{\label{sec_ecosystem_automation_principles}} 

\begin{SCn}

\bigskip

\begin{scnrelfromlist}{ключевое понятие}
    \scnitem{...}
\end{scnrelfromlist}

\bigskip

\begin{scnrelfromlist}{ключевое знание}
    \scnitem{...}
\end{scnrelfromlist}

\bigskip

\begin{scnrelfromlist}{библиографическая ссылка}
    \scnitem{\scncite{...}}
\end{scnrelfromlist}

\end{SCn}    

\section{Семантически совместимые интеллектуальные ostis-порталы знаний}
{\label{sec_ostis_scientific_portal}} 

\begin{SCn}

\bigskip

\begin{scnrelfromlist}{ключевое понятие}
    \scnitem{портал знаний}
    \scnitem{ostis-портал знаний}
\end{scnrelfromlist}


\begin{scnrelfromlist}{библиографическая ссылка}
    \scnitem{\scncite{...}}
\end{scnrelfromlist}

\end{SCn}

\textit{порталы знаний} представляют собой один из способов создания централизованного доступа к информации, которая может быть необходима для решения задач, связанных с работой в организации. Такие порталы могут содержать информацию, связанную с процессами, документацией, процедурами, обучающими материалами, а также ответы на часто задаваемые вопросы.

Одним из ключевых преимуществ \textit{порталов знаний} является их способность к сбору и хранению информации из различных источников, таких как базы данных, системы управления документами, системы управления проектами и т.д. Это позволяет пользователям получать полную и актуальную информацию в одном месте.

\textit{порталы знаний} могут обеспечивать возможность взаимодействия между пользователями, например, путем создания форумов, обсуждений и коллективного редактирования документов. Это может способствовать обмену знаниями и опытом между сотрудниками организации и повысить эффективность их работы.

При создании \textit{порталов знаний} возникают проблемы, связанные с организацией и управлением информацией. Например, необходимо обеспечить корректное и структурированное хранение информации, ее поиск и обновление. Также необходимо учитывать потребности пользователей и обеспечить удобный и интуитивно понятный интерфейс.

Целями интеллектуального портала знаний являются:
\begin{textitemize}
    \item ускорение погружения каждого человека в новые для него области при постоянном сохранении общей целостной картины Мира (образовательная цель);
    \item фиксация в систематизированном виде новых результатов так, чтобы все основные связи новых результатов с известными были четко обозначены;
    \item автоматизация координации работ по рецензированию новых результатов;
    \item автоматизация анализа текущего состояния базы знаний.
\end{textitemize}

Создание интеллектуальных \textit{порталов знаний}, обеспечивающих повышение темпов интеграции и согласования различных точек зрения, – это способ существенного повышения темпов эволюции научно-технической деятельности.
Совместимые \textit{порталы знаний}, реализованные в виде \textit{ostis-систем}, входящих в \textit{Экосистему OSTIS}, являются основой новых принципов организации научной деятельности, в которой результатами этой деятельности являются не статьи, монографии, отчеты и другие научно-технические документы, а фрагменты глобальной \textit{базы знаний}, разработчиками которых являются свободно формируемые научные коллективы, состоящие из специалистов в соответствующих научных дисциплинах. 
С помощью \textit{ostis-порталов знаний} осуществляется как координация процесса рецензирования новой научно-технической информации, поступающей от научных работников в \textit{базы знаний} этих порталов, так и процесс согласования различных точек зрения ученых (в частности, введению и семантической корректировке понятий, а также введению и корректировке терминов, соответствующих различным сущностям).

Реализация семейства семантически совместимых \textit{ostis-порталов знаний} в виде совместимых \textit{ostis-систем}, входящих в состав \textit{Экосистемы OSTIS}, предполагает разработку иерархической системы семантически согласованных формальных онтологий, соответствующих различным дисциплинам, с четко заданным наследованием свойств описываемых сущностей и с четко заданными междисциплинарными связями, которые описываются связями между соответствующими формальными онтологиями и специфицируемыми ими предметными областями.

Реализация \textit{ostis-порталов знаний} на основе \textit{Технологии OSTIS} предоставляет ряд преимуществ по сравнению с другими подходами. Ниже приведены некоторые из них:
\begin{textitemize}
    \item использование методов семантической обработки информации, что позволяет более точно и эффективно организовывать и искать информацию на портале знаний;
    \item высокий уровень гибкости и расширяемости, что позволяет адаптировать \textit{ostis-порталы знаний} под различные нужды и требования пользователей;
    \item автоматическая интеграция \textit{ostis-порталов знаний} с другими \textit{ostis-системами} в рамках \textit{Экосистемы OSTIS}, что позволяет создать централизованный доступ к информации из различных источников.
    \item возможность создания персонализированного \textit{ostis-портала знаний}, который учитывает интересы и потребности каждого пользователя, что позволяет более эффективно использовать знания \textit{ostis-систем};
    \item возможность производить \textit{ostis-порталы знаний} быстро и с минимальными затратами благодаря использованию существующих компонентов и инструментов.
\end{textitemize}

Примером \textit{портала знаний}, построенного в виде \textit{ostis-системы} является \textit{Метасистема OSTIS}, содержащая все известные на текущий момент знания и навыки, входящие в состав \textit{Технологии OSTIS}.

Таким образом, реализация \textit{порталов знаний} на основе \textit{Технологии OSTIS} позволяет создать более эффективную и гибкую систему для хранения, организации и поиска знаний, которая может быть адаптирована под различные требования пользователей и организаций.

\section{Семантически совместимые интеллектуальные корпоративные ostis-системы различного назначения}
{\label{sec_corporate_ostis_system}} 

\begin{SCn}

\begin{scnrelfromlist}{ключевое понятие}
    \scnitem{корпоративная система}
    \scnitem{корпоративная ostis-систем}
\end{scnrelfromlist}

\bigskip

\begin{scnrelfromlist}{библиографическая ссылка}
    \scnitem{\scncite{...}}
\end{scnrelfromlist}

\end{SCn}

\textit{корпоративные системы} представляют собой программные решения, предназначенные для автоматизации бизнес-процессов и управления ресурсами и данными внутри организации. Они могут включать в себя различные подсистемы, такие как управление отношениями с клиентами, управление контентом, управление проектами, управление ресурсами предприятия, управление документами и многое другое.

Роль \textit{корпоративных систем} в современных организациях заключается в обеспечении эффективного управления бизнес-процессами и ресурсами, повышении производительности и качества работы, а также обеспечении прозрачности и оперативности принятия решений на основе актуальных данных.

\textit{корпоративные системы} могут использоваться для следующих целей:

\begin{textitemize}
    \item автоматизация многих рутинных задач, таких как обработка заказов, управление складом, учет финансовых операций и так далее. Это позволяет сократить время на выполнение задач и уменьшить количество ошибок.
    \item сбор, хранение и обработка данных о бизнес-процессах и ресурсах организации. Это позволяет увеличить точность и оперативность принятия решений, а также обеспечить прозрачность в управлении организацией.
    \item эффективное управление ресурсами организации, такими как финансы, трудовые ресурсы, материальные и технические ресурсы и так далее. Это позволяет сократить затраты на управление ресурсами и повысить эффективность их использования.
    \item управление отношениями с клиентами, автоматизация процессов продаж и обслуживания, а также анализ данных о клиентах. Это позволяет повысить удовлетворенность клиентов и увеличить объемы продаж.
    \item управление проектами, планирование и отслеживание выполнения работ, управление ресурсами и расписание проектов. Это позволяет повысить эффективность выполнения проектов, уменьшить сроки выполнения работ и снизить затраты на проекты.
    \item управление документами, контроль версиями, автоматизация процессов редактирования и утверждения документов. Это позволяет повысить эффективность работы с документами и обеспечить безопасность их хранения и передачи.
\end{textitemize}

Хотя \textit{корпоративные системы} могут принести значительные выгоды для организаций, они также могут столкнуться с рядом проблем, связанных с их внедрением и эксплуатацией. Ниже перечислены некоторые из этих проблем:

\begin{textitemize}
    \item Внедрение корпоративных систем может быть дорогостоящим и трудоемким процессом, который требует значительных ресурсов и экспертизы. Кроме того, многие системы могут потребовать изменения бизнес-процессов и требовать адаптации культуры организации.
    \item Корпоративные системы могут столкнуться с проблемами совместимости с другими системами, используемыми в организации. Это может привести к проблемам с обменом данными и снижению эффективности работы.
    \item Корпоративные системы могут стать мишенью для кибератак, поэтому важно обеспечить безопасность хранения и передачи данных, используемых в системах.
    \item Корпоративные системы могут потребовать значительных затрат на обслуживание и поддержку, включая установку обновлений, устранение ошибок и техническую поддержку.
    \item Внедрение новых корпоративных систем может потребовать обучения персонала, что может быть трудоемким и затратным процессом.
    \item Внедрение корпоративных систем может потребовать изменения бизнес-процессов, что может быть сложным и вызвать сопротивление со стороны сотрудников.
\end{textitemize}

Для создания семантически совместимых интеллектуальных \textit{корпоративных систем} необходимо обеспечить высокую степень гибкости, масштабируемости, автоматизации и интеграции. Это позволит организациям более эффективно управлять ресурсами и данными и повысить их конкурентоспособность на рынке. Для достижения этих целей необходимо использовать современные технологии, такие как Аналитика данных, Машинное обучение, Искусственный интеллект и технологии распределенных вычислений. Кроме того, необходимо учитывать особенности организации и ее бизнес-процессов, чтобы обеспечить максимальную эффективность использования системы.

Для решения задачи формирования корпоративной системы целесообразно применение \textit{Технологии OSTIS}. \textit{корпоративная ostis-система} позволяет отслеживать, анализировать и постепенно автоматизировать все процессы обработки данных в рамках ostis-сообщества. 
Такая система действует по следующим принципам:
\begin{textitemize}
    \item интеллектуальные подсистемы (агенты) упорядочивают структуру данных таким образом, что актуальная информация всегда доступна, а устаревшая информация автоматически архивируется или удаляется в соответствии с законами о хранении и защите данных в режиме реального времени;
    \item запросы к системе выполняются в виде простых инструкций, система помогает менеджерам вводить необходимую информацию, осуществляет частичную или полную автоматизацию обновления информации из баз данных, доступных через Интернет;
    \item интеллектуальные подсистемы выполняют структуризацию и классификацию документов и информации для принятия быстрых и правильных решений, автоматически обрабатывает документы и доступные базы данных для отбора ключевой информации, необходимой в данный момент и в будущем;
    \item существующее системное окружение на предприятии может быть легко подключено к системе через открытые интерфейсы, вся информация остается доступной;
    \item все ключевые системы данных синхронизируются с основной системой, данные постоянно сравниваются друг с другом, чтобы избежать потерь;
    \item вся информация доступна в базе знаний, которая является источником данных для рабочих процессов, отчетности и комплексных проверок;
\end{textitemize}

Достоинствами внедрения предложенной системы являются:
\begin{textitemize}
    \item помощь сбора и оценки информации без преднамеренных искажений или ошибок, связанных с человеческим фактором;
    \item предоставление возможности полного контроля своих данных;
    \item система предоставляет только высококачественные, достоверные и актуальные данные;
    \item цифровое представление всех процессов сообщества обеспечивает интегрированную обработку информации внутри сообщества, что дает полную прозрачность управления, облегчает доступ ко всей информации и ее анализ;
    \item благодаря поддержке интеллектуальных подсистем все необходимые данные из документов, процессов и внешних источников могут быть извлечены, структурированы и грамотно оценены.
\end{textitemize}

С точки зрения структуры \textit{Экосистемы OSTIS}, \textit{корпоративная ostis-система} осуществляет координацию и эволюцию деятельности некоторых групп \textit{ostis-систем} и их пользователей. 
Основная цель \textit{корпоративных ostis-систем} – локализовать \textit{базы знаний} указанных групп \textit{ostis-систем}, перевести их из статуса виртуальных в статус реальных и автоматизировать их эволюцию.


\textit{корпоративные ostis-системы} могут быть применены в различных областях: медицина и здравоохранение, образовательная деятельность широкого профиля, страховой бизнес, промышленная деятельность, административная деятельность, недвижимость, транспорт и так далее.

\section{Персональные ostis-ассистенты пользователей}
{\label{sec_ostis_assistant}} 

\begin{SCn}

\begin{scnrelfromlist}{ключевое понятие}
    \scnitem{персональный ассистент}
    \scnitem{персональный ostis-ассистент}
\end{scnrelfromlist}

\bigskip

\begin{scnrelfromlist}{библиографическая ссылка}
    \scnitem{\scncite{Meurisch2017}}
    \scnitem{\scncite{Meurisch2020}}
    \scnitem{\scncite{Jeni2022}}
    \scnitem{\scncite{Akbar2022}}
\end{scnrelfromlist}

\end{SCn}

Общество должно обеспечивать персональную поддержку каждому человеку, учитывая его индивидуальные особенности, с целью достижения следующих целей:
\begin{textitemize}
    \item максимального уровня физического здоровья, активности и долголетия;
    \item максимального уровня физического комфорта, личного пространства и материального благосостояния;
    \item максимального уровня социального комфорта и защиты прав и свобод.
\end{textitemize}

Для этого должен осуществляться:
\begin{textitemize}
    \item персональный мониторинг каждой личности по всем направлениям;
    \item диагностика и устранение нежелательных отклонений;
    \item оказание своевременной персональной помощи в уточнении направлений дальнейшей эволюции каждой личности.
\end{textitemize}

Необходимо перейти от оказания услуг в решении различных проблем по инициативе самих лиц, столкнувшихся с этими проблемами, к своевременному обнаружению возможности возникновения этих проблем и к соответствующей профилактике. 
Это возможно только при наличии четкой системной организации персонального мониторинга. 

Цифровые \textit{персональные ассистенты} --- это программы, основанные на технологиях искусственного интеллекта и машинного обучения, которые помогают пользователям в выполнении повседневных задач, таких как составление расписания, управление контактами, поиск информации, напоминание о важных событиях и так далее (см. \scncite{Meurisch2017}, \scncite{Meurisch2020}, \scncite{Jeni2022}, \scncite{Akbar2022}).

\textit{Персональный ассистент} должен учитывать, что роли пользователя в обществе могут меняться, расширяться, также как и его интересы и цели. 
При этом, все \textit{персональные ассистенты} должны быть семантически совместимыми с целью понимания друг друга, а также обладать способностью самостоятельно взаимодействовать в рамках различных \textit{корпоративных систем}, представляя интересы своих пользователей.

Одной из основных проблем, связанных с реализацией цифровых \textit{персональных ассистентов}, является необходимость точного понимания запросов и задач, поступающих от пользователя. Это может быть вызвано различными факторами, такими как нечеткость и неоднозначность формулировок, использование аббревиатур и сленга, а также многозначность некоторых слов.

Пользователь не обязан знать множество сервисов, из которых он должен выбирать подходящий ему функционал. Комплекс семантически совместимых сервисов должен располагаться "за кадром"{}. Следовательно, все используемые информационные ресурсы и сервисы должны быть семантически совместимы. Выбор подходящего для пользователя ресурса или сервиса должен производить его \textit{персональный ассистент}.

Таким образом, при реализации цифровых \textit{персональных ассистентов} необходимо обеспечить их масштабируемость и адаптивность к потребностям пользователей. Это означает, что система должна быть способна автоматически адаптироваться к изменениям в поведении пользователя, учитывая его предпочтения, особенности работы и другие факторы.

\textit{Технология OSTIS} позволяет создавать семантически совместимые системы, которые способны обрабатывать запросы и задачи пользователей, учитывая их контекст и смысл. Это достигается за счет использования семантических сетей, которые позволяют описывать знания и связи между ними. Кроме того, \textit{технология OSTIS} обеспечивает масштабируемость и гибкость системы, что позволяет ей адаптироваться к изменениям в поведении пользователей и изменениям в их потребностях.

\textit{Персональный ostis-ассистент} есть \textit{ostis-система}, являющаяся \textit{персональным ассистентом} пользователя в рамках \textit{Экосистемы OSTIS}.
Такая система предоставляет возможности:
\begin{textitemize}
    \item анализа деятельности пользователя и формирования рекомендаций по ее оптимизации;
    \item адаптации под настроение пользователя, его личностные качества, общую окружающую обстановку, задачи, которые чаще всего решает пользователь;
    \item перманентного обучения самого ассистента в процессе решения новых задач, при этом обучаемость потенциально не ограничена;
    \item вести диалог с пользователем на естественном языке, в том числе в речевой форме;
    \item отвечать на вопросы различных классов, при этом если системе что-то не понятно, то она сама может задавать встречные вопросы;
    \item автономного получения информации от всей окружающей среды, а не только от пользователя (в текстовой или речевой форме).
\end{textitemize}

При этом система может как анализировать доступные информационные источники (например, в интернете), так и анализировать окружающий ее физический мир, например, окружающие предметы или внешний вид пользователя.

Достоинства \textit{персонального ostis-ассистента}:
\begin{textitemize}
    \item пользователю нет необходимости хранить разную информацию в разной форме в разных местах, вся информация хранится в единой базе знаний компактно и без дублирований;
    \item благодаря неограниченной обучаемости ассистенты могут потенциально автоматизировать практически любую деятельность, а не только самую рутинную;
    \item благодаря базе знаний, ее структуризации и средствам поиска информации в базе знаний пользователь может получить более точную информацию более быстро.
\end{textitemize}

\textit{Персональные ассистенты} имеют самое различное назначение и могут быть использованы для самых различных категорий пользователей (пациент, юридическое обслуживание, административное обслуживание, покупатель, потребитель различных услуг). \textit{Персональный ostis-ассистент} может использовать знания и данные, хранящиеся в других \textit{ostis-системах}, таких как \textit{корпоративные ostis-системы}, чтобы предоставлять пользователю более полную и актуальную информацию. Это может быть особенно полезно для пользователей, которые работают с большим количеством данных и информации. \textit{Персональный ostis-ассистент} автоматически интегрируется с другими \textit{ostis-системами}, что позволяет ему более эффективно работать с данными и информацией. Он может использовать технологии машинного обучения и искусственного интеллекта для адаптации к поведению пользователя и улучшения его производительности и эффективности. \textit{Персональный ostis-ассистент} может быть создан и настроен с учетом конкретных потребностей организации и ее процессов, что может привести к значительным экономическим и производственным преимуществам.

Таким образом, \textit{персональные ostis-ассистенты} обладают рядом преимуществ по сравнению с другими реализациями цифровых \textit{персональных ассистентов}, таких как более точное понимание запросов и задач пользователей, доступ к актуальным данным и информации, автоматическая интеграция с другими \textit{ostis-системами} в рамках \textit{Экосистемы OSTIS} и адаптация к потребностям организации и ее процессов.


%\input{author/references}