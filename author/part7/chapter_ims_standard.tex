\chapter{Метасистема OSTIS}
\chapauthortoc{Голенков В.~В.\\Шункевич Д.~В.\\Банцевич К.~А.\\Загорский А.~Г.}
\label{chapter_ims_standard}

\vspace{-7\baselineskip}

\begin{SCn}
	\begin{scnrelfromlist}{авторы}
		\scnitem{Голенков В.~В.}
		\scnitem{Шункевич Д.~В.}
		\scnitem{Банцевич К.~А.}
		\scnitem{Загорский А.~Г}
	\end{scnrelfromlist}
		
	\bigskip
	
	\scntext{аннотация}{Данная глава посвящена рассмотрению подхода к автоматизации процессов создания, развития и применения стандартов на основе \textit{Технологии OSTIS}. Также в главе сформулированы основные принципы стандартизации \textit{интеллектуальных компьютерных систем}, методов и средств их проектирования в рамках предлагаемого подхода.}
	
	\bigskip
	
	\begin{scnrelfromlist}{подраздел}
		\scnitem{\ref{sec_metasystem}~\nameref{sec_metasystem}}
		\scnitem{\ref{sec_standard}~\nameref{sec_standard}}
	\end{scnrelfromlist}
	
	\bigskip
	
	\begin{scnrelfromlist}{ключевое понятие}
		\scnitem{Метасистема OSTIS}
		\scnitem{Стандарт Технологии OSTIS}
	\end{scnrelfromlist}
	
	\bigskip
	
	\begin{scnrelfromlist}{библиографическая ссылка}
		\scnitem{\scncite{Serenkov2004}}
		\scnitem{\scncite{Golenkov2018a}}
		\scnitem{\scncite{Uglev2012}}
		\scnitem{\scncite{Golenkov2019}}
		\scnitem{\scncite{Bantsevich2022}}
		\scnitem{\scncite{IMS}}		
	\end{scnrelfromlist}
	
\end{SCn}

\section*{Введение в Главу~\ref{chapter_ims_standard}}

В основе каждой развитой сферы человеческой деятельности лежит ряд стандартов, формально описывающих различные ее аспекты --- систему понятий (включая терминологию), типологию и последовательность действий, выполняемых в процессе применения соответствующих методов и средств (см. \scncite{Golenkov2019}).

Стандарты в самых различных областях являются важнейшим видом знаний, главной целью которых является обеспечение совместимости различных видов деятельности.  Несмотря на развитие информационных технологий, в настоящее время подавляющее большинство стандартов представлено либо в виде традиционных линейных документов, либо в виде web-ресурсов содержащих набор статических страниц, связанных гиперссылками. Для того чтобы стандарты выполняли свою главную функцию, они должны постоянно совершенствоваться. 

Текущее оформление стандартов имеет ряд недостатков, которые мешают эффективному и грамотному использованию стандартов в различных областях (см. \scncite{Serenkov2004} и \\ \scncite{Uglev2012}):
\begin{textitemize}
	\item дублирование информации в рамках документа, описывающего стандарт;
	\item трудоемкость сопровождения самого стандарта, обусловленная в том числе дублированием информации, в частности, трудоемкость изменения терминологии;
	\item проблема интернационализации стандарта -- фактически перевод стандарта на несколько языков приводит к необходимости поддержки и согласования независимых версий стандарта на разных языках;
	\item неудобство применения стандарта, в частности, трудоемкость поиска необходимой информации. Как следствие --- трудоемкость изучения стандарта;
	\item несогласованность формы различных стандартов между собой, как следствие --- трудоемкость автоматизации процессов развития и применения стандартов;
	\item трудоемкость автоматизации проверки соответствия объектов или процессов требования того или иного стандарта;
	\item  и другие.
\end{textitemize}

Перечисленные проблемы связаны в основном с формой представления стандартов. 

Задачей любого стандарта в общем случае является описание согласованной системы понятий (и соответствующих терминов), бизнес-процессов, правил и других закономерностей, способов решения определенных классов задач и так далее. Для формального описания информации такого рода с успехом применяются онтологии. Более того, в настоящее время в ряде областей вместо разработки стандарта в виде традиционного документа разрабатывается соответствующая онтология. Такой подход дает очевидные преимущества в плане автоматизации процессов согласования и использования стандартов.

Однако, актуальной остается проблема, связанная не с формой, а с сутью (семантикой) стандартов --- проблема несогласованности системы понятий и терминов между различными стандартами, которая актуальна даже для стандартов в рамках одной и той же сферы деятельности.

В настоящее время \textit{Информатика} преодолевает важнейший этап своего развития --- переход от информатики данных (data science) к информатике знаний (knowledge science), где акцентируется внимание на \myuline{семантических} аспектах представления и обработки \textit{знаний}.

Без фундаментального анализа такого перехода невозможно решить многие проблемы, связанные с управлением \textit{знаниями}, экономикой \textit{знаний}, с \textit{семантической совместимостью} \textit{интеллектуальных компьютерных систем}.

С семантической точки зрения каждый стандарт есть иерархическая \textit{онтология}, уточняющих структуру и систем понятий соответствующих им \textit{предметных областей}, которая описывает структуру и функционирование либо некоторого класса технических или иных искусственных систем, либо некоторого класса организаций, либо некоторого вида деятельности. 

Наиболее перспективным подходом к решению перечисленных проблем является преобразование каждого конкретного стандарта в \textit{базу знаний}, в основе которой лежит набор \textit{онтологий}, соответствующих данному стандарту. Такой подход позволяет в значительной мере автоматизировать процессы развития стандарта и его применения.

В рамках \textit{Технологии OSTIS} данный подход используется при построении \textit{Стандарта OSTIS}.
 
Предлагаемый \textit{Стандарт OSTIS} оформлен в виде \textit{семейства разделов базы знаний} специальной интеллектуальной компьютерной \textit{Метасистемы OSTIS} (Intelligent MetaSystem for ostis-systems) (см. \scncite{IMS}), которая построена по \textit{Технологии OSTIS} и представляет собой постоянно совершенствуемый интеллектуальный \textit{портал научно-технических знаний}, который поддерживает перманентную эволюцию \textit{Стандарта OSTIS}, а также разработку различных \textit{ostis-систем} (интеллектуальных компьютерных систем, построенных по \textit{Технологии OSTIS}).


\section{Структура, назначение, особенности и достоинства Метасистемы OSTIS}
\label{sec_metasystem}

\begin{scnrelfromlist}{подраздел}
	\scnitem{\ref{ims_structure}~\nameref{ims_structure}}
	\scnitem{\ref{ims_purpose}~\nameref{ims_purpose}}
	\scnitem{\ref{ims_peculiarities}~\nameref{ims_peculiarities}}
	\scnitem{\ref{ims_advantages}~\nameref{ims_advantages}}
\end{scnrelfromlist}

\bigskip

\begin{SCn}
\begin{scnrelfromlist}{ключевое понятие}
	\scnitem{Метасистема OSTIS}
\end{scnrelfromlist}
\end{SCn}

\bigskip

\begin{SCn}
	\begin{scnrelfromlist}{библиографическая ссылка}
	\scnitem{\scncite{IMS}}
\end{scnrelfromlist}
\end{SCn}

Эффективность любой технологии, в том числе и \textbf{\textit{Технологии OSTIS}} определяется не только сроками создания искусственных систем соответствующего класса, но и темпами совершенствования самой технологии (темпами совершенствования средств автоматизации и темпами совершенствования системы стандартов, лежащих в основе технологии).

Для фиксации текущего состояния \textit{Технологии OSTIS}, а также для организации ее эффективного использования и ее перманентного совершенствования с участием ученых, работающих в области искусственного интеллекта, и инженеров, разрабатывающих семантические компьютерные системы различного назначения, в состав \textit{Экосистемы OSTIS} вводится \textit{Метасистема OSTIS}, назначение которой делает ее ключевой \textit{ostis-системой} в рамках \textit{Экосистемы OSTIS}.

\begin{SCn}
	\scnheader{Метасистема OSTIS}
	\scnidtf{Intelligent MetaSystem for ostis-systems design}
	\scnidtf{ostis-система автоматизации проектирования ostis-систем}
	\scnnote{При разработке \textit{Teхнологии OSTIS} средством автоматизации этой деятельности является не вся \textit{Метасистема OSTIS}, а только ее часть – входящая в состав \textit{Метасистемы OSTIS}, \textit{Встраиваемая ostis-система поддержки реижиниринга ostis-систем}, которая поддерживает деятельность разработчиков базы знаний \textit{Метасистемы OSTIS}. Это обусловлено тем, что вся деятельность по разработке  \textit{Teхнологии OSTIS} сводится к разработке (инженирингу) и обновлению (совершенствованию, реинжинирингу) \textit{Базы знаний Метасистемы OSTIS}).}
\end{SCn}	

\textit{Метасистема OSTIS} --- интеллектуальная компьютерная система, обеспечивающая 
\begin{textitemize}
	\item комплексную информационную поддержку всех этапов \textit{жизненного цикла} \textit{интеллектуальных компьютерных систем нового поколения};
	\item автоматизацию проектирования всех компонентов \textit{интеллектуальных компьютерных систем нового поколения};
	\item комплексную автоматизацию всех этапов жизненного цикла \textit{интеллектуальных компьютерных систем нового поколения}.
\end{textitemize}

\textit{Метасистема OSTIS} --- метасистема, являющаяся:
\begin{textitemize}
\item \textit{корпоративной ostis-системой}, обеспечивающей организацию (координацию) деятельности \textit{Консорциума OSTIS};
\item формой представления реализации и фиксации текущего состояния \textit{Ядра Технологии OSTIS};
\item корпоративной \textit{ostis-системой}, взаимодействующей со всеми корпоративными ostis-системами, каждая из которых координирует развитие соответствующей \textit{специализированной ostis-технологии}.
\end{textitemize}

Формой реализации представленной \textit{Метасистемы OSTIS} является \textit{Технология OSTIS}.

\begin{SCn}
\scnheader{Метасистема OSTIS}
\scnidtf{Универсальная базовая (предметно-независимая) ostis-система автоматизации проектирования ostis-систем (любых ostis-систем)}
\scniselement{ostis-система}
\scnidtf{Интеллектуальная метасистема, построенная по стандартам \textit{Технологии OSTIS} и предназначенная (1) для инженеров \textit{ostis-систем} -- для поддержки проектирования. Реализации и обновления (реинжиниринга) \textit{ostis-систем} и для разработчиков \textit{Технологии OSTIS} -- для поддержки коллективной деятельности по развитию стандартов и библиотек \textit{Технологии OSTIS.}}
\scnrelto{форма реализации}{Технология OSTIS}
\scnidtf{Интеллектуальная Метасистема, являющаяся формой (вариантом) реализации (представления, оформления) \textit{Технологии OSTIS} в виде \textit{ostis-системы}}
\scntext{примечание}{Тот факт, что \textit{Технология OSTIS }реализуется в виде \textit{ostis-системы}, является весьма важным для эволюции \textit{Технологии OSTIS}, поскольку методы и средства эволюции (перманентного совершенствования) \textit{Технологии OSTIS} становятся фактически совпадающими с методами и средствами разработки любой (!) \textit{ostis-системы} на всех этапах их жизненного цикла.
Другими словами, эволюция \textit{Технологии OSTIS} осуществляется методами и средствами самой этой технологии.}
\scnidtf{Система комплексной автоматизации (информационной и инструментальной поддержки) проектирования и реализации ostis-систем, которая сама реализована также в виде ostis-системы.}
\scnidtf{\textit{Портал знаний по Технологии OSTIS}, интегрированный с с.а.п.р. ostis-систем и реализованный в виде ostis-системы.}
\scniselement{портал научно-технических знаний}	
\end{SCn}	

\subsection{Структура Метасистемы OSTIS}
\label{ims_structure}

Принципы технической реализации \textit{Метасистема OSTIS} полностью совпадают с принципами технической реализации прикладных интеллектуальных систем, разрабатываемых с помощью этой метасистемы.

\begin{SCn}
\scnheader{Метасистема OSTIS}
	\begin{scnrelfromset}{декомпозиция}
	\scnitem{полное описание самой Технологии OSTIS}
	\scnitem{история эволюции Технологии OSTIS}
	\scnitem{описание правил использования Технологии OSTIS}
	\scnitem{описание организационной инфраструктуры, направленной на развитие Технологии OSTIS}
	\scnitem{библиотека многократно используемых компонентов ostis-систем}
	\scnitem{методы и инструментальные средства проектирования различного вида компонентов ostis-систем}
	\scnitem{технические средства координации деятельности участников проекта, направленные на постоянное совершенствование Технологии OSTIS}
\end{scnrelfromset}
\end{SCn}

\vspace{-\baselineskip}

\begin{SCn}
	\scnheader{Метасистема OSTIS}
	\begin{scnrelfromset}{декомпозиция ostis-системы}
		\scnitem{SC-модель Метасистемы OSTIS}
		\begin{scnindent}
			\begin{scnrelfromset}{декомпозиция sc-модели ostis-системы}
				\scnitem{База знаний Метасистемы OSTIS}
				\scnitem{Решатель задач Метасистемы OSTIS}
				\scnitem{Пользовательский интерфейс Метасистемы OSTIS}
			\end{scnrelfromset}
			\end{scnindent}
		\scnitem{Программный вариант реализации ostis-платформы}
\end{scnrelfromset}
\begin{scnrelfromlist}{подсистема}
	\scnitem{Библиотека Метасистемы OSTIS}
	\scnitem{Средства разработки компонентов ostis-систем}
	\begin{scnindent}
		\begin{scnrelfromset}{декомпозиция}
			\scnitem{Средства поддержки проектирования баз знаний ostis-систем}
			\scnitem{Средства поддержки проектирования решателей задач ostis-систем}
			\scnitem{Средства поддержки проектирования пользовательских интерфейсов ostis-систем}
		\end{scnrelfromset}
	\end{scnindent}
\end{scnrelfromlist}

\scnheader{База знаний Метасистемы OSTIS}
\begin{scnrelfromset}{декомпозиция}
	\scnitem{Стандарт OSTIS}
	\scnitem{Раздел Проект OSTIS. История, текущее состояние и перспективы эволюции и применения Технологии OSTIS}
	\scnitem{Документация Метасистемы OSTIS}
	\scnitem{История и текущие процессы эксплуатации Метасистемы OSTIS}
	\scnitem{Раздел Проект OSTIS. История, текущие процессы и план развития Метасистемы OSTIS}
\end{scnrelfromset}

\scnheader{Решатель задач Метасистемы OSTIS}
\scnnote{Решатель задач собственно \textit{Метасистемы OSTIS}, без учета подсистем, включает в себя набор sc-агентов информационного поиска, реализующих базовые механизмы навигации по базе знаний.}
\begin{scnrelfromset}{декомпозиция}
	\scnitem{Абстрактный sc-агент поиска всех входящих константных позитивных стационарных sc-дуг принадлежности}
	\scnitem{Абстрактный sc-агент поиска всех идентификаторов заданного sc-элемента}
	\scnitem{Абстрактный sc-агент поиска полной семантической окрестности заданного элемента}
	\scnitem{Абстрактный sc-агент поиска связок декомпозиции для заданного sc-элемента}
	\scnitem{Абстрактный sc-агент поиска всех известных сущностей, являющихся общими по отношению к заданной}
	\scnitem{Абстрактный sc-агент поиска определения или пояснения для заданного объекта}
	\scnitem{Абстрактный sc-агент поиска всех известных сущностей, являющихся частными по отношению к заданной}
	\scnitem{Абстрактный sc-агент поиска всех выходящих константных позитивных стационарных sc-дуг принадлежности с их ролевыми отношениями}
	\scnitem{Абстрактный sc-агент поиска всех выходящих константных позитивных стационарных sc-дуг принадлежности}
	\scnitem{Абстрактный sc-агент поиска всех входящих константных позитивных стационарных sc-дуг принадлежности с их ролевыми отношениями}
\end{scnrelfromset}
\end{SCn}

\subsection{Назначение Метасистемы OSTIS}
\label{ims_purpose}

\textit{Метасистема OSTIS} является в \textit{Экосистеме OSTIS} ключевой интеллектуальной системой, которая поддерживает не только проектирование новых интеллектуальных систем и не только замену устаревших компонентов в интеллектуальных системах, входящих в состав \textit{Экосистемы OSTIS}, но и включение (интеграцию) в состав \textit{Экосистемы OSTIS} новых создаваемых интеллектуальных систем.

\textit{Метасистема OSTIS} ориентирована на разработку и практическое внедрение методов и средств компонентного проектирования семантически совместимых интеллектуальных систем, которая предоставляет возможность быстрого создания интеллектуальных приложений различного назначения. Подчеркнем при этом, что сферы практического применения методики компонентного проектирования семантически совместимых интеллектуальных систем ничем не ограничены.

Описываемая \textit{Метасистема OSTIS} является:

\begin{textitemize}
	\item системой информационной и инструментальной поддержки всех этапов жизненного цикла и.к.с. нового поколения (\textit{ostis-систем}) самого различного назначения;
	\item порталом знаний по \textit{Технологии OSTIS}, обеспечивающим:
		\begin{textitemize}
			\item координацию работ по развитию \textit{Технологии OSTIS};
			\item автоматизацию анализа качества \textit{Стандарта OSTIS}.
		\end{textitemize}
	То есть \textit{Метасистема OSTIS} является системой управления \textit{Проектом создания и развития Стандарта OSTIS}.	
\end{textitemize}

Важнейшим направлением \textit{Метасистемы OSTIS} и, соответственно, важнейшим направлением применения \textit{Стандарта OSTIS}, подробно описанного в~\textit{\ref{sec_standard}~\nameref{sec_standard}}, является использование их в качестве комплексного интегрированного компьютерного учебного пособия по специальности \scnqqi{Искуственный интеллект}. Для этого устанавливается связь между разделами \textit{Стандарта OSTIS} и программами различных \textit{учебных дисциплин} указанной специальности. Важно подчеркнуть при этом: \textit{Стандарт OSTIS} содержит достаточно полный сравнительный анализ с различными альтернативными подходами, то есть ни в коем случае не ограничивается рассмотрением \uline{только} \textit{Технологией OSTIS}.

\subsection{Особенности Метасистемы OSTIS}
\label{ims_peculiarities}

Особенность \textit{Метасистемы OSTIS} заключается в унификации представления различного вида информации в памяти компьютерных систем на основе смыслового (семантического) представления этой информации, что обеспечивает:
	\begin{textitemize}
		\item устранение дублирования одной и той же информации в разных интеллектуальных системах и в разных компонентах одной и той же системы;
		\item семантическую совместимость различных компонентов интеллектуальных систем и различных \textit{интеллектуальных систем} в целом;
		\item существенное расширение \textit{библиотек многократно используемых компонентов ostis-систем} за счет "крупных"\ компонентов и, в частности, \textit{встраиваемых ostis-систем}.
	\end{textitemize}

\subsection{Достоинства Метасистемы OSTIS}
\label{ims_advantages}

\textit{Метасистема OSTIS} взаимодействует не только со своими разработчиками и конечными пользователями, но и с другими ostis-системами, которые созданы с помощью \textit{Технологии OSTIS} и представляют собой ее дочерние системы*. 

\textit{Метасистема OSTIS} для своих дочерних систем может:
\begin{textitemize}
	\item осуществлять автоматическую сборку \textit{дочерних ostis-систем} стартовых версий по инструкциям. Таким образом, генерировать новые \textit{дочерние ostis-системы};
	\item включать в \textit{дочерние ostis-системы} новые многократно используемые компоненты
	из постоянно пополняемой \textit{библиотеки многократно используемых семантически совместимых компонентов ostis-систем};
	\item заменять в \textit{дочерних ostis-системах} устаревшие версии многократно используемых компонентов на новые версии из \textit{Библиотеки Метасистемы OSTIS};
	\item включать в \textit{дочерние ostis-системы} подсистему совершенствования своей расширенной базы знаний и, при необходимости, подсистему улучшения ее интегрированной машины обработки знаний и пользовательского интерфейса.
\end{textitemize}

Таким образом, после появления \textit{дочерней ostis-системы} ее связь с \textit{Метасистемой OSTIS} не прерывается и она становится постоянным участником процесса совершенствования всех \textit{дочерних ostis-систем}.

\textit{Метасистема OSTIS} является одновременно и системой автоматизации проектирования \textit{ostis-систем}, и интеллектуальной системой, обучающей методам и средствам проектирования \textit{ostis-систем}. Этот факт существенно повышает качество проектирования прикладных \textit{ostis-систем}, расширяет контингент разработчиков \textit{ostis-систем} и интегрирует проектную (инженерную) деятельность в области искусственного интеллекта с образовательной деятельностью в этой области.

Стоит отметить также, что все опубликованные материалы о \textit{Технологии OSTIS} в формализованном виде входят в базу знаний \textit{Метасистемы OSTIS}.

\section{Структура, назначение, особенности и достоинства Стандарта OSTIS}
\label{sec_standard}

\begin{scnrelfromlist}{подраздел}
	\scnitem{\ref{standard_contents}~\nameref{standard_contents}}
	\scnitem{\ref{standard_key_signs}~\nameref{standard_key_signs}}
	\scnitem{\ref{standard_purpose}~\nameref{standard_purpose}}
	\scnitem{\ref{standard_analogs}~\nameref{standard_analogs}}
	\scnitem{\ref{standard_peculiarities}~\nameref{standard_peculiarities}}
	\scnitem{\ref{standard_users}~\nameref{standard_users}}
	\scnitem{\ref{standard_authors}~\nameref{standard_authors}}
	\scnitem{\ref{standard_requirements}~\nameref{standard_requirements}}
	\scnitem{\ref{standard_rules}~\nameref{standard_rules}}
	\scnitem{\ref{standard_development_directions}~\nameref{standard_development_directions}}
	\scnitem{\ref{standard_advantages}~\nameref{standard_advantages}}
\end{scnrelfromlist}

\bigskip

\begin{SCn}
\begin{scnrelfromlist}{ключевое понятие}
	\scnitem{Стандарт Технологии OSTIS}
\end{scnrelfromlist}

\bigskip

\begin{scnrelfromlist}{библиографическая ссылка}
	\scnitem{\scncite{Standart2021}}
	\scnitem{\scncite{Savushkin2020}}
\end{scnrelfromlist}
\end{SCn}

\textit{Стандарт Технологии OSTIS} представляет собой Документацию \textit{Технологии OSTIS}, которая представлена в виде \textit{основной части базы знаний} специальной \textit{интеллектуальной компьютерной системы}, предназначенной для комплексной поддержки жизненного цикла семантически совместимых \textit{интеллектуальных компьютерных систем нового поколения} (\textit{Метасистемы OSTIS}). 

\begin{SCn}
	\scnheader{Стандарт OSTIS}
	\scnidtf{Документация \textit{Технологии OSTIS}}
	\scnidtf{Документация Открытой технологии онтологического проектирования, производства и эксплуатации семантически совместимых гибридных \textit{интеллектуальных компьютерных систем}}
	\scnidtf{Описание \textit{Технологии OSTIS} (Open Semantic Technology for Intelligent Systems), представленная в виде семейства разделов \textit{базы знаний специальной ostis-системы} (системы, построенной по \textit{Технологии OSTIS}) на внутреннем языке \textit{ostis-систем} и обладающее достаточной полнотой для использования этой \textit{технологии} разработчиками \textit{интеллектуальных компьютерных систем}}
	\scnidtf{Полное описание текущего состояния \textit{Технологии OSTIS}, представленное в виде семейства разделов \textit{базы знаний}, построенной по \textit{Технологии OSTIS}}
	\scnidtf{Семейство разделов \textit{базы знаний} \textit{Метасистемы OSTIS}, которое предназначено для комплексной поддержки онтологического проектирования семантически совместимых \textit{гибридных интеллектуальных компьютерных систем}}
	\scniselement{семейство разделов базы знаний}
	\begin{scnindent}
		\scnidtf{семейство разделов внутреннего представления \textit{базы знаний ostis-системы} -- \textit{интеллектуальной компьютерной системы}, построенной по \textit{Технологии OSTIS}}
	\end{scnindent}
	\scnidtf{Достаточно полная формальная Документация текущей версии \textit{Технологии OSTIS}, представленная либо в виде основной части базы знаний \textit{Метасистемы OSTIS}, либо в виде внешнего формального представления этой \textit{базы знаний}}
	\scnidtf{Основная часть базы знаний \textit{Метасистемы OSTIS}, описывающая текущую версию \textit{Технологии OSTIS}}
	\scnidtf{\uline{Формальный} текст, объектом описания которого является \textit{Технология OSTIS}, то есть текст, являющий достаточно полным описанием текущего состояния \textit{Технологии OSTIS}}
	\scnidtf{Документация \textit{Технологии OSTIS}, полностью отражающая \uline{текущее} состояние \textit{Технологии OSTIS} и представленная соответствующим  \textit{семейством разделов базы знаний} специальной \textit{ostis-системы}, которая ориентирована на поддержку проектирования, производства, эксплуатации и эволюции (реинжиринга) \textit{ostis-систем}, а также на поддержку эволюции самой \textit{Технологии OSTIS} и которая названа нами \textit{Метасистемой OSTIS}}
	\scnidtf{Семейство разделов, в состав которого входят все \textit{разделы Стандарта OSTIS}}
	\scntext{основной sc-идентификатор}{Стандарт OSTIS}
	\begin{scnindent}
		\scnrelto{сокращение}{\scnfilelong{Стандарт \textit{Технологии OSTIS}}}
		\begin{scnindent}
			\scnrelto{сокращение}{\scnfilelong{Стандарт Открытой \textit{технологии} комплексной поддержки \textit{жизненного цикла} семантически совместимых \textit{интеллектуальных компьютерных систем нового поколения}}}
		\end{scnindent}
	\end{scnindent}
\end{SCn}

Следует подчеркнуть, что \textit{Стандарт OSTIS} --- это не описание некоторого состояния \textit{Технологии OSTIS}, а \uline{динамическая} информационная модель процесса эволюции этой \textit{технологии} (см. \textit{Главу~\textit{\ref{chapter_ostis_tech}~\nameref{chapter_ostis_tech}}}).

В рамках \textit{Стандарта OSTIS} вводится оглавление, система ключевых знаков. Также строго регламентированы: 
\begin{textitemize}
	\item требования, предъявляемые к \textit{Стандарту OSTIS};
	\item правила построения \textit{Стандарта OSTIS};
	\item направления развития \textit{Стандарта OSTIS}.
\end{textitemize}

Это позволяет воспринимать пользователю \textit{Стандарт OSTIS}, как целостный понятный текст. Также избежать возможных противоречий.


\subsection{Оглавление Стандарта OSTIS}
\label{standard_contents}

Одним из составляющих \textit{Стандарта OSTIS} является \textit{Оглавление Стандарта OSTIS}.

\begin{SCn}
	\scnheader{Оглавление Стандарта OSTIS}
	\scnidtfexp{Иерархический перечень разделов, входящих в состав \textit{Стандарта OSTIS}, с дополнительной спецификацией некоторых разделов}
	\begin{scnindent}
		\scnnote{Существенно подчеркнуть, что иерархия разделов \textit{Стандарта OSTIS} не означает то, что \textit{разделы} более низкого уровня иерархии входят в состав (являются частями) соответствующих разделов более высокого уровня. Связь между \textit{разделами} разных уровней иерархии означает то, что \textit{раздел} более низкого уровня иерархии является \textit{дочерним} разделом по отношению к соответствующему \textit{разделу} более высокого уровня, то есть \textit{разделом}, который наследует свойства указанного \textit{раздела} более высокого уровня.\\
			В отличие от этого каждая \textit{часть Стандарта OSTIS}, а также сам \textit{Стандарт OSTIS} является \textit{семейством разделов} (совокупность разделов), входящих в ее состав.
		}
	\end{scnindent}
	\scnnote{Описание логико-семантических связей каждого раздела \textit{Стандарта OSTIS} с другими разделами \textit{Стандарта OSTIS} приводится в рамках \textit{титульной спецификации} каждого \textit{раздела}.}
\end{SCn}	

Основной текст \textit{Стандарта OSTIS} состоит из следующих частей:

\begin{SCn}
	\scnheader{Стандарт OSTIS}
	\scnrelfrom{декомпозиция}{Структура Стандарта OSTIS верхнего уровня}
	\begin{scnindent}
		\begin{scneqtovector}
			\scnitem{Часть 1 Стандарта OSTIS.\\
				Введение в интеллектуальные компьютерные системы нового поколения}
			\begin{scnindent}
				\scnidtf{Анализ текущего состояния \textit{технологий Искусственного интеллекта} и постановка задачи на создание \uline{комплекса} совместимых \textit{технологий искусственного интеллекта}, обеспечивающего поддержку всего \textit{жизненного цикла интеллектуальных компьютерных систем нового поколения} и названного нами \textit{Технологией OSTIS}}
			\end{scnindent}
			\scnitem{Собственно документация Технологии OSTIS}
			\begin{scnindent}
				\begin{scnrelfromvector}{декомпозиция}
					\scnitem{Стандарт ostis-систем}
					\begin{scnindent}
						\scnidtf{Стандарт интеллектуальных компьютерных систем нового поколения, построенных по Технологии OSTIS}
						\scnidtf{Формальная теория ostis-систем}
						\scnidtf{Формальные структурно-функциональные и логико-семантические модели ostis-систем}
						\begin{scnrelfromvector}{декомпозиция}
							\scnitem{Часть 2 Стандарта OSTIS.\\
								Смысловое представление и онтологическая систематизация знаний в и.к.с. нового поколения.}
							\begin{scnindent}
								\scnidtf{Стандарт представления информации в ostis-системах}
								\scnidtf{Модели представления знаний и баз знаний в ostis-системах}
							\end{scnindent}
							\scnitem{Часть 3 Стандарта OSTIS.\\
								Многоагентные решатели задач и.к.с. нового поколения}
							\begin{scnindent}
								\scnidtf{Стандарт процессов и методов обработки информации в ostis-системах}
								\scnidtf{Модели обработки знаний в ostis-системах (логические, продукционные, функциональные, нейросетевые, процедурные и непроцедурные, четкие и нечеткие)}
							\end{scnindent}
							\scnitem{Часть 4 Стандарта OSTIS.\\
								Онтологические модели интерфейсов и.к.с. нового поколения}
							\begin{scnindent}
								\scnidtf{Стандарт информационных ресурсов и моделей решения интерфейсных задач в ostis-системах}
							\end{scnindent}
						\end{scnrelfromvector}							
					\end{scnindent}
					\scnitem{Стандарт методов и средств поддержки жизненного цикла ostis-систем}
					\begin{scnindent}
						\scnidtf{Стандарт бизнес-процессов и методик, автоматически реализуемых процессов и методов, информационных средств и инструментальных средств, используемых для поддержки жизненного цикла ostis-систем}
						\begin{scnrelfromvector}{декомпозиция}
							\scnitem{Часть 5 Стандарта OSTIS.\\
								Методы и средства проектирования и.к.с. нового поколения}
							\begin{scnindent}
								\scnidtf{Методики, методы и средства проектирования баз знаний, решателей задач и интерфейсов ostis-систем}
							\end{scnindent}
							\scnitem{Часть 6 Стандарта OSTIS.\\
								Платформы реализации и.к.с. нового поколения}
							\begin{scnindent}
								\scnidtf{Методы и средства реализации ostis-систем (на основе программных ostis-платформ и специально созданных для этого компьютеров)}
							\end{scnindent}
							\scnitem{Часть 7 Стандарта OSTIS.\\
								Методы и средства реинжиниринга и эксплуатации и.к.с. нового поколения}
							\begin{scnindent}
								\scnidtf{Методы и средства эксплуатации ostis-систем конечными пользователями, а также их сопровождения (поддержки работоспособности) и реинжиниринга (обновления, модернизации)}
							\end{scnindent}
						\end{scnrelfromvector}
					\end{scnindent}
				\end{scnrelfromvector}
			\end{scnindent}
			\scnitem{Часть 8 Стандарта OSTIS.\\
				Экосистема и.к.с. нового поколения и их пользователей}
			\begin{scnindent}
				\scnidtf{Описание продуктов, создаваемых с помощью Технологии OSTIS, основными их которых является \textit{Экосистема OSTIS}, семантически совместимых и активно взаимодействующих ostis-систем и их пользователей}
				\scnidtf{Теория Экосистемы OSTIS и ее эволюции}
			\end{scnindent}
			\scnitem{Библиография OSTIS}
			\begin{scnindent}
				\scnidtf{Спецификация \textit{библиографических источников}, семантически близких \textit{Технологии OSTIS}, в контексте их сравнительного анализа со \textit{Стандартом OSTIS}}
			\end{scnindent}
		\end{scneqtovector}
	\end{scnindent}
\end{SCn}	


\subsection{Ключевые знаки Стандарта OSTIS}
\label{standard_key_signs}

Система ключевых знаков \textit{Стандарта OSTIS} упорядочена в точном соответствии с \textit{Оглавлением Стандарта OSTIS} и является уточнением указанного Оглавления путем перечисления и пояснения ключевых сущностей, описываемых в разделах Стандарта, и, в первую очередь тех сущностей, которые указываются в идентификаторах (названиях) разделов \textit{Стандарта OSTIS}.

\textit{Система ключевых знаков Стандарта OSTIS} является целостным дополнением к \textit{Оглавлению Стандарта OSTIS}, поскольку:
\begin{textitemize}
	\item иерархия и последовательность ключевых знаков четко соответствуют иерархии и последовательности разделов стандарта;
	\item система ключевых знаков \textit{Стандарта OSTIS}, как и его Оглавление, воспринимается (читаться) как целостный понятный текст. 
\end{textitemize}


\subsection{Назначение Стандарта OSTIS}
\label{standard_purpose}

Поскольку \textit{Стандарт OSTIS} является неотъемлемой частью \textit{Метасистемы OSTIS} (основной частью ее \textit{базы знаний}), основным назначением \textit{Стандарта OSTIS} является обеспечение максимально эффективной реализации того, для чего предназначена \textit{Метасистема OSTIS}.

\textit{Стандарт OSTIS} рассматривается как результат конвергенции и интеграции всевозможных направлений \textit{Искусственного интеллекта}, что позволяет студентам и магистрантам сформировать целостное представление о тематике \textit{Искусственного интеллекта}, а не мозаичное представление в виде множества дисциплин (направлений), связи между которыми подробно и тем более формально не рассматриваются.

\textit{Стандарт OSTIS} перманентно и достаточно быстро эволюционирует. За время обучения студентов и магистрантов происходит весьма существенные изменения текущей версии \textit{Стандарта OSTIS}.

Студенты и магистранты активно вовлекаются в процесс эволюции \textit{Стандарта OSTIS}, это обеспечивает:

\begin{textitemize}
	\item формирование необходимого уровня их квалификации в условиях быстрого морального старения того, чему их уже научили;
	\item формирование необходимых навыков, позволяющих им в процессе реальной профессиональной деятельности быстро адаптироваться к новым условиям этой деятельности и, в частности, к новым версиям соответствующих технологий.
\end{textitemize}


\subsection{Аналоги Стандарта OSTIS}
\label{standard_analogs}

Аналогами \textit{Стандарта OSTIS} можно считать:

\begin{textitemize}
	\item любую серьезную попытку систематизации результатов, полученных в области Искусственного интеллекта к текущему моменту:
	\begin{textitemize}
		\item учебник, достаточно полно отражающий текущее состояние \textit{Искусственного интеллекта};
		\item справочник, содержащий достаточно полную информацию о текущем состоянии \textit{Искусственного интеллекта}.
	\end{textitemize}
	\item любую попытку перехода от частных формальных моделей различных многократно используемых компонентов ostis-систем к общей (объединенной, интегрированной) формальной модели и.к.с в целом --- теории и.к.с.
	\item любую унификацию технических решений, устранения многообразия форм технических решений при разработке и.к.с.
	\item первые попытки разработки стандартов \textit{интеллектуальных компьютерных систем}, а также \textit{технологий Искусственного интеллекта}, которые, чаще, всего, ограничиваются построением систем соответствующих понятий.
\end{textitemize}


\subsection{Особенности Стандарта OSTIS}
\label{standard_peculiarities}

\textit{Стандарт OSTIS} --- это не просто систематизация современного состояния результатов в области \textit{Искусственного интеллекта}, это систематизация, представленная в виде общей комплексной \uline{формальной} модели \textit{интеллектуальных компьютерных систем} и комплексной \uline{формальной} модели поддержки их жизненного цикла. Более того, текст \textit{Стандарта OSTIS} представляет собой \textit{основную часть базы знаний} специальной \textit{интеллектуальной метасистемы}, которая ориентирована:

\begin{textitemize}
	\item на поддержку разработки \textit{интеллектуальных компьютерных систем} различного назначения;
	\item на поддержку эволюции \textit{Стандарта OSTIS};
	\item на поддержку \textit{подготовки специалистов в области Искусственного интеллекта}.
\end{textitemize}

\textit{Стандарт OSTIS} --- это \uline{динамический} текст, перманентно отражающий новые научно-технические результаты, получаемые в области \textit{Искусственного интеллекта} в рамках \textit{Общей теории интеллектуальных компьютерных систем} и \textit{Общей комплексной технологии разработки интеллектуальных компьютерных систем}. Здесь важной является оперативность фиксации новых научно-технических результатов, то есть минимизация отрезка времени между моментом получения новых результатов и моментом интеграции описания этих результатов в состав \textit{Стандарта OSTIS}. В перспективе авторы новых научно-технических результатов в области \textit{Искусственного интеллекта} будут заинтересованы лично публиковать (интегрировать) свои результаты в состав \textit{Стандарта OSTIS}, то есть становиться соавторами \textit{Стандарта OSTIS}, чтобы обеспечить необходимую оперативность такой публикации и отсутствие искажений своих результатов. Динамичность \textit{Стандарта OSTIS} и достаточная оперативность интеграции в его состав новых научно-технических результатов в области \textit{Искусственного интеллекта} делает \textit{Стандарт OSTIS} всегда актуальным и никогда морально устаревшим.

В рамках \textit{Стандарта OSTIS} нет противопоставления между научно-технической информацией, добываемой в области \textit{Искусственного интеллекта}, и учебно-методической информацией, используемой для подготовки и самоподготовки специалистов в области Искусственного интеллекта. информация о том, чему учить, должна быть "переплетена"{}, интегрирована с информацией о том, как учить.

Важно заметить, что \textit{Стандарт OSTIS} в отличие от остальных стандартов является структурированным \myuline{формальным} текстом, который может быть непосредственно использован не только разработчиками \textit{интеллектуальных компьютерных систем}, но также и интеллектуальными компьютерными системами, осуществляющими автоматизацию проектирования разрабатываемых \textit{интеллектуальных компьютерных систем} и поддержку последующих этапов их жизненного цикла. Таким образом, разработка \textit{Стандарта OSTIS} является неотъемлемой частью разработки комплекса средств информационной и инструментальной поддержки всего жизненного цикла \textit{интеллектуальных компьютерных систем}, а указанные средства поддержки \textit{жизненного цикла интеллектуальных компьютерных систем} становятся равноправными партнерами в процессе создания, эксплуатации и сопровождения \textit{интеллектуальных компьютерных систем} благодаря своему осознанию (пониманию) того, что такое \textit{интеллектуальные компьютерные системы} и их \textit{жизненный цикл}.

Содержательно \textit{Стандарт OSTIS} охватывает не только описание моделей разрабатываемых \textit{интеллектуальных компьютерных систем}, но также и описание методик, автоматизируемых методов и инструментальных средств поддержки (автоматизации) всех этапов жизненного цикла разрабатываемых интеллектуальных компьютерных систем.

Проект развития Стандарта OSTIS ориентирован на \uline{высокие темпы} эволюции \textit{Стандарта OSTIS} благодаря автоматизации управления этим проектом с помощью \textit{Метасистемы OSTIS}, которая является полноправным участником этого проекта.

Построение и структуризация текста \textit{Стандарта OSTIS} ориентированы на максимально возможное снижение языкового и когнитивного барьера для начинающих его пользователей. Для этой цели используются (1) различного рода естественно-языковые примечания и комментарии, имеющие соответствующие семантические связи с поясняемыми сущностями, а также (2) различного рода дидактические знания, указывающие на различные аналогии, различия, примеры, принципы, лежащие в основе описываемых сущностей и тому подобные.


\subsection{Пользователи Стандарта OSTIS}
\label{standard_users}

Рассмотрим целевую аудиторию \textit{Стандарта OSTIS}.
\begin{SCn}
	\scnheader{Стандарт OSTIS}
	\scnrelfrom{класс пользователей}{\textbf{пользователь Стандарта OSTIS}}
	\begin{scnindent}
		\scnidtf{целевая аудитория Стандарта OSTIS}
		\begin{scnrelfromset}{разбиение}
			\scnitem{разработчик ostis-системы}
			\begin{scnindent}
				\scnsuperset{разработчик Метасистемы OSTIS}
				\begin{scnindent}
					\begin{scnrelfromset}{разбиение}
						\scnitem{разработчик Стандарта OSTIS}
						\scnitem{разработчик решателя задач Метасистемы OSTIS}
						\scnitem{разработчик пользовательского интерфейса Метасистемы OSTIS}
					\end{scnrelfromset}
				\end{scnindent}
				\scnsuperset{разработчик базы знаний ostis-системы}
				\begin{scnindent}
					\scnsuperset{разработчик Стандарта OSTIS}
				\end{scnindent}
				\scnsuperset{разработчик решателя задач ostis-системы}
				\begin{scnindent}
					\scnsuperset{разработчик решателя задач Метасистемы OSTIS}
				\end{scnindent}
				\scnsuperset{разработчик интерфейса ostis-системы} 
				\begin{scnindent}
					\scnsuperset{разработчик пользовательского интерфейса Метасистемы OSTIS}
				\end{scnindent}
				\scnsuperset{разработчик ostis-платформы}
			\end{scnindent}
			\scnitem{потенциальный разработчик ostis-системы}
			\scnitem{специалист в области Искусственного интеллекта, желающий интегрировать свои результаты в состав общей теории и.к.с. нового поколения и соответствующей комплексной технологии}
			\scnitem{студент или магистрант специальности \scnqqi{Искусственный интеллект} либо другой смежной специальности, желающий приобрести практический опыт в разработке прикладных и.к.с. нового поколения или в разработке соответствующей комплексной технологии}
		\end{scnrelfromset}
	\end{scnindent}	
\end{SCn}


\subsection{Авторский коллектив Стандарта OSTIS}
\label{standard_authors}

Для обеспечения перманентной эволюции существует ряд требований, ориентированный на авторов \textit{Стандарта OSTIS}. 

Авторы \textit{Стандарта OSTIS} должны:

\begin{textitemize}
	\item Отслеживать и изучать новые публикации по тематике, рассматриваемой в Стандарте OSTIS. Близкими источниками для этого являются:
	\begin{textitemize}
		\item выпуски журналов;
		\item материалы конференций:
		\begin{textitemize}
			\item организуемых Консорциумом 3WC;
			\item по интеграции различных направлений \textit{Искусственного интеллекта};
		\end{textitemize}
		\item стандарты в области \textit{Искусственного интеллекта};
		\item публикации, рассматривающие:
		\begin{textitemize}
			\item формальные онтологии;
			\item онтологии верхнего уровня;
			\item семантические сети;
			\item графы знаний;
			\item графовые базы данных и графовые с.у.б.д.;
			\item смысловое представление знаний;
			\item конвергенцию различных направлений \textit{Искусственного интеллекта}.
		\end{textitemize}
	\end{textitemize}
	\item Фиксировать результаты изучения новых публикаций по тематике, близкой \textit{Стандарту OSTIS}, в \textit{Библиографии OSTIS}, а также в основном тексте \textit{Стандарта OSTIS} в виде соответствующих ссылок, цитат, сравнительного анализа. 
	\item Отслеживать текущее состояние \uline{всего} текста \textit{Стандарта OSTIS}, формировать предложения, направленные на развитие \textit{Стандарта OSTIS} и на повышение темпов этого развития. Активно участвовать в обсуждении проблем развития \textit{Технологии OSTIS}.
	\item Максимально возможным образом увязывать персональную работу над \textit{Стандартом OSTIS} с другими формами деятельности -- научной, учебной, прикладной.
	\item Указывать авторство своих предложений по дополнению и/или корректировке текущего текста \textit{Стандарта OSTIS}.
	\item Участвовать в рецензировании и согласовании предложений, представленных другими авторами \textit{Стандарта OSTIS}.
\end{textitemize}

Большой объем работ по созданию и развитию \textit{Стандарта OSTIS} и, соответственно, \textit{Технологии OSTIS}, комплексный характер этих работ, в которых необходима глубокая \textit{конвергенции} и \textit{интеграции} различных направлений \textit{Искусственного интеллекта} предъявляют к \textit{Авторскому коллективу Стандарта OSTIS}  высокие требования по уровню мотивации, по уровню качества творческой атмосферы, по уровню \textit{интероперабельности} всех членов коллектива, то есть по уровню способности быстро и качественно согласовывать персональные точки зрения.

Поскольку \textit{Проект создания и развития Стандарта OSTIS} является открытым, членом \textit{Авторского коллектива Стандарта OSTIS} может стать \uline{любой} желающий, соблюдающий Правила организации взаимодействия членов \textit{Авторского коллектива Стандарта OSTIS}, разделяющий цели и задачи разработки такого стандарта.

Выделяются следующие ключевые пункты \textit{Правил организации взаимодействия членов Авторского коллектива Стандарта OSTIS}:
\begin{textitemize}
	\item Коллегиально формировать тактические и стратегические направления развития Стандарта OSTIS и, соответственно, Технологии OSTIS;
	\item Коллегиально распределять задачи по реализации утвержденных направлений развития Стандарта OSTIS с учетом (1) научных интересов, квалификации и возможности каждого члена Авторского коллектива, (2) приоритета задач и при достаточно полном охвате \uline{всех} приоритетных задач.
\end{textitemize}

В рамках \textit{Авторского коллектива Стандарта OSTIS} выделяется также \textit{\textbf{Редакционная коллегия Стандарта OSTIS}}. 

\textit{Редакционная коллегия Стандарта OSTIS} --- часть \textit{Авторского коллектива Стандарта OSTIS}, являющаяся центром коллегиального принятия решений по основным направлениям развития Стандарта OSTIS и, соответственно, Технологии OSTIS, по уточнению соответствующих приоритетов и сроков. \textit{Редакционная коллегия Стандарта OSTIS} также несет ответственность за формирование и реализацию стратегических направлений развития Стандарта OSTIS и, в частности, за подбор и назначение \textit{ответственных исполнителей разделов Стандарта OSTIS}.

Основными направлениями деятельности \textit{Редакционной коллегии Стандарта OSTIS} являются:
\begin{textitemize}
	\item Обеспечение целостности и повышения качества постоянно развиваемой (совершенствуемой) Технологии OSTIS, а также достаточно точное описание (документирование) каждой текущей версии этой технологии.
	\item Обеспечение четкого контроля совместимости версий Технологии OSTIS в целом, а также версий различных компонентов этой технологии.
	\item Постоянное уточнение степени важности различных направлений развития Технологии OSTIS для каждого текущего момента.
	\item Формирование и постоянное уточнение плана тактического и стратегического развития самой \textit{Технологии OSTIS}, а также полной документации этой Технологии в виде \textit{Стандарта OSTIS}. Подчеркнем при этом, что указанная документация является неотъемлемой частью \textit{Технологии OSTIS}.
\end{textitemize}

\begin{SCn}
	\scnheader{Стандарт OSTIS}
	\scnrelfrom{редакционная коллегия}{Редакционная коллегия Стандарта OSTIS}
	\begin{scnindent}
		\scnidtf{Редколлегия Стандарта OSTIS}
		\scnidtf{Рабочий орган, обеспечивающий организацию коллективного творческого процесса по развитию \textit{Стандарта OSTIS}, совмещенного с учебно-методическим обеспечением подготовки соответствующих специалистов.}
		\scnidtf{Редакционная коллегия, несущая ответственность за качество перманентно эволюционируемого \textit{Стандарта OSTIS}.}
		\begin{scnrelfromlist}{обязанности}
			\scnitem{\scnfilelong{Обеспечение развития \textit{Стандарта OSTIS}, совмещенного (интегрированного) с комплексным учебно-методическим обеспечением \textit{подготовки специалистов в области Искусственного интеллекта}}}
			\scnitem{\scnfilelong{Обеспечение корректности (непротиворечивости), системности, целостности, полноты всех разрабатываемых материалов \textit{Стандарта OSTIS} и, в том числе, \textit{учебно-методического обеспечения подготовки специалистов в области Искусственного интеллекта}.}}
			\scnitem{\scnfilelong{Кординация деятельности авторов разработки очередной (следующей) версии \textit{Стандарта OSTIS}.}}
			\scnitem{\scnfilelong{В частности, \textit{Редколлегия Стандарта OSTIS} может осуществлять распределение работ по построению следующей версии \textit{Стандарта OSTIS} с четкой привязкой ответственных лиц \textit{Стандарта OSTIS} к соответствующим разделам \textit{Стандарта OSTIS}.}}
		\end{scnrelfromlist}
		\scnnote{Очень важная структура, определяющая научно-технический уровень, авторитет и репутацию всей \textit{Технологии OSTIS} в глазах международной научно-технической общественности.}
		\scnnote{Каждый член \textit{Редакционной коллегии Стандарта OSTIS} должен быть активным членом \textit{Авторского коллектива Стандарта OSTIS}.}
	\end{scnindent}
\end{SCn}

\subsection{Требования, предъявляемые к Стандарту OSTIS}
\label{standard_requirements}

\textit{Стандарт OSTIS} должен соответствовать следующим требованиям:
\begin{textitemize}
	\item вводимые понятия (в т.ч. дидактические отношения) должны быть четко пояснены и/или определены в соответствующем разделе Стандарта OSTIS;
	\item доступность понимания текста \textit{Стандарта OSTIS} для читателей;
	\item обеспечение возможности поэтапной формализации информации, начиная от ея-текстов, которые могут быть в последствии записаны на формальном языке;
	\item независимость от естественных языков (только на уровне внешних идентификаторов (имен) sc-элементов);
	\item четкая логико-семантическая спецификация каждой предметной области, рассматриваемой в Стандарте OSTIS. Названная спецификация должна отражать как внутреннюю структуру предметной области (роли ее ключевых элементов), так и связи с другими предметными областями;
	\item \uline{конвергенция}, ("бесшовная"{}) \uline{интеграция} различных видов \textit{знаний}, описывающих самые различные сущности, к числу которых, в частности относятся и сами знания всевозможного вида;
	\item целостность, полнота, связность:
	\begin{textitemize}
		\item отсутствие информационных дыр;
		\item достаточно полная спецификация всех сущностей;
		\item согласованность основных идентификаторов (терминов), отсутствие синонимов и омонимов.
	\end{textitemize}
	\item отсутствие информационных излишеств и информационного мусора;
	\item четкая семантическая \uline{стратификация} -- каждый фрагмент базы знаний должен иметь свою семантическую "полочку"{} (никакого дублирования);
	\item строгая логическая последовательность текста (все используемые сущности должны быть введены либо в заданной предметной области, либо в предметной области более высокого уровня);
	\item унификация стилистики -- текст не должен вызывать трудностей для его понимания;
	\item богатая библиография и сравнительный анализ;
	\item четкое соблюдение и совершенствование правил идентификации и спецификации описываемых сущностей;
	\item достаточно подробная спецификация каждого вводимого понятия в соответствующей предметной области.
\end{textitemize}


\subsection{Правила построения Стандарта OSTIS}
\label{standard_rules}

В рамках разработки \textit{Стандарта OSTIS} выделяются \textit{Общие правила построения Стандарта OSTIS} и \textit{Частные правила построения Стандарта OSTIS}.

\begin{SCn}
	\scnheader{Общие правила построения Стандарта OSTIS}
	\scnidtf{принципы, лежащие в основе структуризации и оформления Стандарта OSTIS}
\end{SCn}

Рассмотрим основные положения:
\begin{textitemize}
	\item Основной формой представления \textit{Стандарта OSTIS} как полной документации текущего состояния \textit{Технологии OSTIS} является \textit{внутреннее представление} основной части \textit{базы знаний} специальной интеллектуальной компьютерной \textit{Метасистемы OSTIS}, обеспечивающей использование и эволюцию (перманентное совершенствование) \textit{Технологии OSTIS}. Такое представление \textit{Стандарта OSTIS} обеспечивает эффективную семантическую навигацию по содержанию \textit{Стандарта OSTIS} и возможность задавать \textit{Метасистеме OSTIS} широкий спектр нетривиальных вопросов о самых различных деталях и тонкостях \textit{Технологии OSTIS};
	\item Кроме представления \textit{Стандарта OSTIS} на внутреннем \textit{языке представления знаний} используется также внешняя форма представления \textit{Стандарта OSTIS} на \textit{внешнем языке представления знаний}. При этом указанное внешнее представление \textit{Стандарта OSTIS} должно быть структурировано и оформлено так,чтобы читатель мог достаточно легко "вручную"{} найти в этом тексте практически любую интересующую его \textit{информацию}. В качестве \textit{формального языка} внешнего представления \textit{Стандарта OSTIS} используется \textit{SCn-код};
	\item \textit{Стандарт OSTIS} имеет онтологическую структуризацию, то есть представляет собой иерархическую систему связанных между собой \textit{формальных предметных областей} и соответствующих им \textit{формальных онтологий}.	Благодаря этому обеспечивается высокий уровень стратифицированности \textit{Стандарта OSTIS};
	\item Каждому \textit{понятию}, используемому в \textit{Стандарте OSTIS}, соответствует свое место в рамках этого Стандарта, своя \textit{предметная область} и соответствующая ей \textit{онтология}, где это \textit{понятие} подробно рассматривается (исследуется), где концентрируется вся основная информация об этом \textit{понятии}, о различных его свойствах.
	\item В состав \textit{Стандарта OSTIS} входят также файлы информационных конструкций, не являющихся конструкциями \textit{SC-кода} (в том числе и sc-текстов, принадлежащих различным естественным языкам). Такие файлы позволяют формально описывать в базе знаний синтаксис и семантику различных внешних языков, а также позволяют включать в состав базы знаний различного рода пояснения, примечания, адресуемые непосредственно пользователям и помогающие им в понимании формального текста базы знаний;
	\item С семантической точки зрения \textit{Стандарт OSTIS} представляет собой иерархическую систему формальных моделей \textit{предметных областей} и соответствующих им \textit{формальных онтологий};
	\item С семантической точки зрения \textit{Стандарт OSTIS} представляет собой большую \textit{рафинированную семантическую сеть}, которая, соответственно, имеет нелинейный характер и которая включает в себя знаки любых видов описываемых сущностей(материальных сущностей, абстрактных сущностей, понятий, связей, структур) и, соответственно этому, содержит связи между всеми этими видами сущностей(в частности, связи между связями, связи между структурами);
	\item \textit{Стандарт OSTIS} представляет собой иерархическую систему \textit{предметных областей} и соответствующих им \textit{онтологий}, специфицирующих эти \textit{предметные области}. Каждая из \textit{предметных областей} описывает соответствующие \textit{классы объектов исследования} с максимально возможной степенью детализации, определяемой набором \textit{отношений} и \textit{параметров}, заданных на \textit{классах объектов исследования}. На множестве \textit{предметных областей}, задано отношение \textit{дочерняя предметная область*}, которое указывает направление наследования свойств объектов исследования, рассматриваемых в разных \textit{предметных областях};
	\item Каждый \textit{раздел Стандарта OSTIS} может содержать те \textit{знания}, которые входят в состав той \textit{предметной области и онтологии}, которая либо полностью представлена указанным \textit{разделом}, либо представлена частично в виде спецификации одного или нескольких конкретных объектов исследования;
	\item недопустима синонимия и омонимия основных sc-идентификаторов в рамках каждого семейства;
	\item Спецификация каждой предметной области и каждого раздела должна иметь достаточную степень полноты. Как минимум, для каждой предметной области должна быть указана роль \uline{каждого} используемого в ней понятия;
	\item В состав \textit{Стандарта OSTIS} входят также файлы информационных конструкций, не являющихся конструкциями \textit{SC-кода} (в том числе и sc-текстов, принадлежащих различным естественным языкам). Такие файлы позволяют формально описывать в базе знаний синтаксис и семантику различных внешних языков, а также позволяют включать в состав базы знаний различного рода пояснения, примечания, адресуемые непосредственно пользователям и помогающие им в понимании формального текста базы знаний;
	\item Непосредственно сам \textit{Стандарт OSTIS} представляет собой внутреннее \textit{смысловое представление} основной части базы знаний \textit{Метасистемы OSTIS} на внутреннем смысловом языке \textit{ostis-систем} (этот язык назван нами \textit{SC-кодом} - Semantic Computer Code).
\end{textitemize}

Кроме \textit{Общих правил построения Стандарта OSTIS} в \textit{Стандарте OSTIS} приводятся описания различных частных (специализированных) правил построения (оформления) различных видов фрагментов \textit{Стандарта OSTIS}.

К таким видам фрагментов относятся следующие:
\begin{textitemize}
	\item \textit{sc-идентификатор}
	\begin{SCn}
		\scnidtf{внешний идентификатор внутреннего знака (\textit{sc-элемента}) входящего в состав \textit{базы знаний ostis-системы}}	
		\scnidtf{\textit{информационная конструкция} (чаще всего это строка символов), обеспечивающая однозначную идентификацию соответствующей сущности, описываемой в \textit{базах знаний ostis-систем}, и являющаяся, чаще всего, именем (термином), соответствующим описываемой сущности, именем, обозначающим эту сущность во внешних текстах \textit{ostis-систем}}
	\end{SCn}
	\item \textit{sc-спецификация}
	\begin{SCn}
		\scnidtf{семантическая окрестность}
		\scnidtf{семантическая окрестность соответствующего \textit{sc-элемента} (внутреннего знака, хранимого в памяти \textit{ostis-системы} в составе ее \textit{базы знаний}, представленной на внутреннем языке \textit{ostis-систем.})}
		\scnidtf{семантическая окрестность некоторого \textit{sc-элемента}, хранимого в \textit{sc-памяти}, в рамках текущего состояния этой \textit{sc-памяти}}
	\end{SCn}
	\item \textit{sc-конструкция} $\setminus$ \textit{sc-спецификация}
	\begin{SCn}
		\scnidtf{\textit{sc-конструкция} (конструкция \textit{SC-кода} -- внутреннего языка \textit{ostis-систем}), не являющаяся \textit{sc-спецификацией}}
	\end{SCn}
	\item \textit{файл ostis-системы $\setminus$ sc-идентификатор}
	\begin{SCn}
		\scnidtf{\textit{файл ostis-системы}, не являющийся sc-идентификатором}
	\end{SCn}
\end{textitemize}

Важно также отметить, что к числу частных правил построения \textit{sc-конструкций} относятся \textit{Правила построения баз знаний ostis-систем}. Эти правила направлены на обеспечение целостности баз знаний ostis-систем, на обеспечение (1) \uline{востребованности} (нужности) знаний, входящих в состав каждой базы знаний, и (2) целостности самой базы знаний, то есть достаточности знаний, входящих в состав каждой базы знаний для эффективного функционирования соответствующей ostis-системы.


\subsection{Направления развития Стандарта OSTIS}
\label{standard_development_directions}

\begin{SCn}
	\scnheader{Стандарт OSTIS}
	\begin{scnrelfromset}{общие направления развития}
		\scnfileitem{Включить в Стандарт OSTIS достаточно подробные правила построения (оформления) sc-идентификаторов и sc-спецификаций различного вида сущностей, а также различного вида файлов ostis-систем}
		\scnfileitem{На каждом этапе четко распределять работу по развитию различных разделов Стандарта OSTIS}
		\scnfileitem{Все инструментальные средства, входящие в состав Технологии OSTIS, должны быть достаточно детально описаны (специфицированы) в виде соответствующих онтологических моделей, имеющих четкую семантическую связь с соответствующими онтологиями и смежными предметными областями, входящими в состав Стандарта OSTIS}
		\scnfileitem{Доработать структуру и содержание титульной спецификации Стандарта OSTIS}
		\scnfileitem{Доработать титульные спецификации вспх раделов Стандарта OSTIS}
		\scnfileitem{Обеспечить достаточную \uline{полноту} sc-спецификации \uline{всех} рассматриваемых (описываемых, обозначаемых sc-элементами) сущностей}
		\scnfileitem{Существенно расширить Библиографию OSTIS}
		\scnfileitem{Постоянно отслеживать синонимию/омонимию sc-идентификаторов}
		\scnfileitem{Постоянно совершенствовать контроль качества выполнения работ по развитию Стандарта OSTIS}
		\scnfileitem{Необходимо постоянно проводить анализ публикаций других авторов по вопросам, близким тематике Стандарта OSTIS, и фиксировать в Стандарте OSTIS результаты сравнительного анализа точки зрения, представленной в Стандарте OSTIS с точками зрения иных авторов путем включения в Стандарт OSTIS спецификаций соответствующих библиографических источников с полезными цитатами, используемых в них терминов по сравнению с терминологией Стандарта OSTIS и так далее.}
		\scnfileitem{Все sc-идентификаторы, входящие в состав sc.g-текстов, sc.n-текстов и различных иллюстраций должны иметь одинаковй шрифт и размер. За исключением, возможно размера sc-идентификаторов в sc.g-текстах}
	\end{scnrelfromset}
\end{SCn}


\subsection{Достоинства Стандарта OSTIS}
\label{standard_advantages}

\textit{Стандарт OSTIS} является примером перехода к принципиально новой форме  представления и публикации \textit{научно-технической информации}, результатов исследовательской деятельности --- не просто к форме электронного документа, а к форме семантически структурированного электронного документа, являющегося частью \textit{базы знаний} по соответствующей \textit{научно-технической дисциплине}. Это существенно повышает эффективность использования накапливаемой человеком \textit{научно-технической информации}, поскольку пользователь этой информации может не только ее просматривать (читать), но и взаимодействовать с \textit{и.к.с.}, которая становится \textit{партнером} в использовании нужной ему информации.


\textit{Проект создания и развития Стандарта OSTIS} является прообразом принципиально нового подхода к организации \textit{научно-технической деятельности} в рамках каждой \textit{научной дисциплины}. Эта деятельность реализуется в форме открытого проекта, направленного на развитие \textit{базы знаний} интеллектуального портала знаний по соответствующей научно-технической дисциплине. Такой уровень формализации \textit{научно-технической информации}, который понятен не только специалистам, но и \textit{интеллектуальным компьютерным систем} существенно повышает эффективность и расширяет области использования этой информации в \textit{интеллектуальных компьютерных системах}. Так, например, интеллектуальный портал знаний по какой-либо технической дисциплине естественным образом становится \textit{интеллектуальной системой автоматизации проектирования} технических систем соответствующего класса.

Также важным достоинством является факт того, что \textit{Стандарт OSTIS} является прототипом учебных пособий нового поколения, имеющих четкую логико-семантическую структуризацию и стратификацию учебного материала, а также теоретико-множественную и логическую классификацию всех используемых понятий. Следовательно, использование \textit{Стандарта OSTIS} не только как основы интеллектуальной системы автоматизации комплексной поддержки \textit{жизненного цикла и.к.с. нового поколения}, но и в качестве комплексного учебного пособия по подготовке молодых специалистов в области \textit{Искусственного интеллекта}, является прообразом широкого использования различных интеллектуальных порталов научно-технических знаний в качестве комплексных учебных пособий для подготовки молодых специалистов по соответствующим специальностям. Это существенно повысит качество образования, которое не должно отставать от развития соответствующих научно-технических направлений, а должно становиться неотъемлемой составляющей этого развития.

\textit{Стандарт OSTIS} является основной частью базы знаний \textit{Метасистемы OSTIS}, описывающей текущую версию \textit{Технологии OSTIS} (см. рисунок~\textit{\nameref{fig:main_page}}).

\begin{figure}[H]
	\caption{Cтартовая страница базы знаний Метасистемы OSTIS}
	\includegraphics[scale=0.8, width=1.0\textwidth]{images/part7/chapter_ims_standard/ims_main_page.png}
	\label{fig:main_page}
\end{figure}

Предлагаемое представление \textit{Стандарта OSTIS} обеспечивает эффективную семантическую навигацию по содержанию \textit{Стандарта OSTIS}, а именно, обеспечивает возможность перейти к любой интересующей главе, позволяет задавать \textit{Метасистеме OSTIS} широкий спектр нетривиальных вопросов о самых различных деталях и тонкостях \textit{Технологии OSTIS}, и получать ответы на заданные вопросы (см. рисунок~\textit{\nameref{fig:table_of_contents}}). 

\begin{figure}[H]
	\caption{Оглавление Стандарта OSTIS}
	\includegraphics[scale=0.8, width=1.0\textwidth]{images/part7/chapter_ims_standard/table_of_contents.png}
	\label{fig:table_of_contents}
\end{figure}

По умолчанию ответы системы пользователю отображаются в \textit{SCn-коде}, который является гипертекстовым вариантом внешнего отображения текстов \textit{SC-кода} и может читаться как линейный текст.

\section*{Заключение к Главе~\ref{chapter_ims_standard}}

Для реализации кооперативного целенаправленного и адаптивного взаимодействия \textit{интеллектуальных компьютерных систем} в рамках автоматически формируемых коллективов \textit{интеллектуальных компьютерных систем} необходима их семантическая совместимость, а это, в свою очередь, требует унификации интеллектуальных компьютерных систем. Унификация \textit{интеллектуальной компьютерной системы} возможна только на основе общей формальной теории \textit{интеллектуальных компьютерных систем} и соответствующего ей \textbf{стандарта интеллектуальных компьютерных систем}, а для этого необходима глубокая конвергенция различных направлений исследований в области искусственного интеллекта.

Поскольку результатом развития искусственного интеллекта как научной дисциплины является перманентная эволюция общей теории интеллектуальных компьютерных систем и соответствующего стандарта интеллектуальных компьютерных систем, для повышения темпов развития искусственного интеллекта и, соответственно, технологии разработки интеллектуальных компьютерных систем необходимо создание портала научно-технических знаний по искусственному интеллекту, обеспечивающего координацию деятельности специалистов, а также согласование и интеграцию результатов этой деятельности.

В главе рассмотрен подход к автоматизации процессов создания, развития и применения стандартов на основе \textit{Технологии OSTIS}. На основе \textit{Стандарта Технологии OSTIS} рассмотрены основные принципы, лежащие в основе предлагаемого подхода к стандартизации.

Предложенный в данной главе подход позволяет обеспечить не только возможность автоматизации процессов создания, согласования и развития стандартов, но и позволяет значительно повысить эффективность процессов применения стандарта, как в ручном, так и в автоматическом режиме.

%\input{author/references}
