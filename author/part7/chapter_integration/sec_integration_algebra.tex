\newpage

\section{Интеграция инструментов компьютерной алгебры в приложения OSTIS}
{\label{sec_integration_algebra}} 

\begin{SCn}

    \bigskip
    
    \begin{scnrelfromlist}{ключевое понятие}
        \scnitem{...}
    \end{scnrelfromlist}
    
    \bigskip
    
    \begin{scnrelfromlist}{ключевое знание}
        \scnitem{...}
    \end{scnrelfromlist}
    
    \bigskip
    
    \begin{scnrelfromlist}{библиографическая ссылка}
        \scnitem{\scncite{...}}
    \end{scnrelfromlist}
    
    \end{SCn}

Системы компьютерной алгебры на настоящий момент представляют собой мощные инструментальные комплексы, возможности которых давно вышли за рамки алгебраических вычислений и даже классической математики в целом. К такого рода системам можно отнести такие активно развиваемые в настоящее продукты, как Wolfram Mathematica [], Maple [], MathCAD [] и ряд других. Так, например, официальная документация проекта Wolfram Mathematica [] выделяет 18 категорий функций данного проекта, среди которых выделяются как фундаментально математические, так и прикладные:
\begin{textitemize}
	\item Машинное обучение;
	\item Символьные и числовые вычисления;
	\item Обработка строк и текстов;
	\item Геометрия;
	\item Графы и сети;
	\item Обработка изображений;
	\item Обработка звуков и видео;	
	\item Представление знаний и обработка естественного языка;	
	\item Инженерные вычисления;
	\item Финансовые вычисления;
	\item Обработка геоданных;	
	\item И другие.
\end{textitemize}

При этом общее количество функций системы Wolfram Mathematica к настоящему моменту насчитывает более 6000.

Отдельно стоит отметить, что компания Wolfram занимается разработкой целого комплекса проектов, в число которых кроме системы Wolfram Mathematica входят также вопросно-ответная система Wolfram Alpha \cite{WolframAlpha}, содержащая обширную базу знаний и набор вычислительных алгоритмов. В основе представления фактографических, логических и процедурных знаний для систем семейства Wolfram лежит мультипарадигмальный язык программирования Wolfram Language. Наличие такого внутреннего языка описания функций систем Wolfram и в целом высокий уровень документированности этих функций выгодно отличает системы Wolfram от других сервисов, позволяющих решать некоторые частные задачи из задач, решаемых системами Wolfram.  Отличие заключается в том, что во многих случаях система семейства Wolfram может не просто решить задачу, но и \myuline{объяснить} ход решения, а также помочь пользователю в выборе той или иной функции, подходящей для решения его задачи, или предложить набор функций, которые можно применить к данным, полученным в результате решения исходной задачи. Другим достоинством систем семейства Wolfram является им комплексность, позволяющая решать достаточно сложные задачи в рамках одного приложения и исключая необходимость интеграции разнородных сервисов.

Несмотря на перечисленные достоинства, современные системы компьютерной алгебры обладают и существенным минусом -- работа пользователя с ними осуществляется в \myuline{процедурном ключе}, то есть пользователь должен сам инициировать все действия системы, и, соответственно, знать, какими функциями она обладает и как их правильно использовать. Если задача не может быть решена при помощи одной функции, пользователь сам должен скомбинировать нужные функции, то есть, по сути, составить план решения задачи.

Учитывая сказанное, можно сделать вывод о целесообразности интеграции систем компьютерной алгебры с различными ostis-системами, входящими в состав Экосистемы OSTIS. Учитывая широкий спектр возможностей современных систем компьютерной алгебры, практически для любого класса ostis-систем можно найти такие функции системы компьютерной алгебры, которые были бы полезны для ostis-систем данного класса. 

Подход к решению задач, основанный на интеграции ostis-систем и систем компьютерной алгебры, обладает рядом преимуществ:
\begin{textitemize}
	\item При разработке ostis-систем исключается необходимость реализовывать многие функции, которые уже реализованы, оттестированы и апробированы в рамках систем компьютерной алгебры. Учитывая тот факт, что в разработке систем компьютерной алгебры принимают участие высококвалифицированные специалисты в соответствующих областях, реализация аналогичных функций в ostis-системах с нуля может потребовать значительных финансовых и временных затрат;
	\item Конкретная ostis-система, использующая некоторые из функций систем компьютерной алгебры, благодаря подходу к разработке гибридных решателей задач в Технологии OSTIS (см. \textit{Главу \ref{chapter_situation_management}~\nameref{chapter_situation_management}}) получает возможность \myuline{самостоятельно} планировать ход решения задач при условии, что некоторые его этапы будут реализованы при помощи функций системы компьютерной алгебры. С точки зрения подхода, предлагаемого в рамках Технологии OSTIS, каждая функция системы компьютерной алгебры становится \textit{методом} решения задач некоторого класса. Этот класс задач описывается в базе знаний ostis-системы и позволяет ей самостоятельно делать вывод о целесообразности применения той или иной функции системы компьютерной алгебры при решении конкретной задачи. Таким образом, интеграция с ostis-системами позволит устранить сформулированный ранее ключевой недостаток систем компьютерной алгебры.
\end{textitemize}

Можно говорить об ostis-надстройках

По аналогии с тем, как осуществляется интеграция в ostis-системы различных современных методов решения задач, например, искусственных нейронных сетей (см. \textit{Главу \ref{chapter_ann}~\nameref{chapter_ann}}), интеграция ostis-систем и систем компьютерной алгебры в общем случае может осуществляться двумя способами:
\begin{textitemize}
	\item Интеграция по принципу "черного ящика"{}, когда в базе знаний ostis-системы присутствует спецификация некоторой функции системы компьютерной алгебры, а также спецификация способа вызова данной функции (например, указание того, через какой программный интерфейс осуществляется взаимодействие с данной внешней системой). Такой вариант интеграции является наиболее простым в плане реализации и в целом обладает всеми перечисленными выше достоинствами. В то же время, данный вариант обладает и недостатком, связанным с тем, что ostis-система не обладает возможностью анализа и объяснения того, как был выполнен конкретный шаг решения задачи, реализуемый данной функцией системы компьютерной алгебры.
	\item Более тесная интеграция, при которой конкретная функция по-прежнему остается частью сторонней системы компьютерной алгебры, но в базу знаний ostis-системы погружается не просто результат ее выполнения, но и всевозможная его спецификация, например, объяснение хода решения задачи, указание конкретных алгоритмов и формул, которые были задействованы в решении, описание возможных альтернативных вариантов решения, оценка эффективности решения и т.д. В данном варианте интеграции ostis-система получает больше возможностей по анализу и объяснению хода решения задачи.
	\item Полная интеграция, при которой осуществляется трансляция используемых функций системы компьютерной алгебры с внутреннего языка этой системы в ostis-систему. Данный вариант является наиболее трудоемким и сложным с точки зрения актуализации реализации возможностей систем компьютерной алгебры в соответствующих ostis-системах с учетом их постоянного развития. В то же время данный вариант интеграции по сравнению в двумя предыдущими обладает важным достоинством -- он обеспечивает платформенную независимость полученного решения и позволяет использовать при решении задачи все достоинства предлагаемых в рамках Технологии OSTIS подходов, в частности, возможность параллельной обработки знаний и возможность оптимизации плана решения задачи или его фрагментов непосредственно в ходе решения задачи.
\end{textitemize}

Важно отметить, что предложенные варианты интеграции не исключают друг друга и могут комбинироваться. Кроме того, углубление интеграции может осуществляться поэтапно с учетом перечисленных достоинств и недостатков, а также с учетом актуальности использования тех или иных функций систем компьютерной алгебры при решении конкретных задач в рамках Экосистемы OSTIS и соответствующих ostis-систем.

Таким образом, поэтапная интеграция систем компьютерной алгебры в Экосистему OSTIS предполагает, как минимум, описание спецификации основных функций выбранной системы компьютерной алгебры средствами Технологии OSTIS, другими словами -- разработку онтологии функций выбранной системы компьютерной алгебры. В случае с системами семейства Wolfram Mathematica процесс разработки такой онтологии может быть автоматизирован благодаря наличию формального языка Wolfram Language и хорошей документированности функций системы.

На рисунке \ref{} показан пример спецификации функции такой-то на языке SCg. В примере есть то-то и то-то.

На данном этапе развития Технологии OSTIS и ее применения целесообразным видится интеграция возможностей систем компьютерной алгебры в комплекс интеллектуальных обучающих систем, построенных по Технологии OSTIS (см. \textit{Главу \ref{chapter_learning_systems}~\nameref{chapter_learning_systems}}). Это связано с тем, что системы компьютерной алгебры обладают несомненным преимуществом и широкими возможностями при решении задач, актуальных для обучающих систем практически по всем естественно-научным и техническим дисциплинам, предполагающим использование достаточно сложного математического аппарата. С другой стороны, несмотря на популярность тематики, связанной с автоматизацией и интеллектуализацией образовательной деятельности по естественно-научным дисциплинам и разработкой соответствующих компьютерных систем, на данный момент на рынке практически отсутствуют широкоизвестные обучающие системы, способные самостоятельно генерировать и решать различные задачи, а также проверять корректность решения задачи пользователем. В качестве исключения можно привести отдельные системы, способные решать нетривиальные задачи, например, по геометрии [https://geometry.allenai.org/, https://mathpix.com/handwriting-recognition ] и теории графов [https://graphonline.ru/].

Интеграция обучающих систем, разрабатываемых на базе Технологии OSTIS и систем компьютерной алгебры позволит разработать системы, обладающие указанными свойствами, в более сжатые сроки, поскольку значительная часть функций может быть заимствована в системах компьютерной алгебры в готовом виде.

Рассмотрим несколько примеров решения задач в обучающих ostis-системах с применением функций системы Wolfram Mathematica. 

Для интеграции с системой Wolfram Mathematica используется соответствующий программный интерфейс [https://reference.wolfram.com/language/workflow/CallAWolframAPIFromAnExternalProgram.html, https://reference.wolfram.com/language/ref/APIFunction.html], дающий возможность выполнить размещенный в облаке Wolfram код на Wolfram Language в рамках пользовательской программы на Python или C++.

Примеры.

Сценарий 1 (генерация графа).
\begin{textitemize}
	\item Используем функцию генерации графа Wolfram, показываем код, граф в Wolfram, тот же граф в OSTIS
\end{textitemize}	

Сценарий 2 (решение задачи).
\begin{textitemize}
	\item Рисуем граф в OSTIS
	\item Показываем, как описана в OSTIS соответствующая функция Wolfram
	\item Показываем, как он транспортируется и погружается в Wolfram
	\item Показываем код для решения задачи в Wolfram
	\item Показываем результат решения задачи в Wolfram
	\item Показываем результат решения задачи в OSTIS
\end{textitemize}	

Выводы.