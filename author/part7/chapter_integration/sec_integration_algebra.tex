\newpage

\section{Интеграция инструментов компьютерной алгебры в приложения OSTIS}
{\label{sec_integration_algebra}} 

\begin{SCn}

 \bigskip
 
 \begin{scnrelfromlist}{ключевое понятие}
 \scnitem{...}
 \end{scnrelfromlist}
 
 \bigskip
 
 \begin{scnrelfromlist}{ключевое знание}
 \scnitem{...}
 \end{scnrelfromlist}
 
 \bigskip
 
 \begin{scnrelfromlist}{библиографическая ссылка}
 \scnitem{\scncite{...}}
 \end{scnrelfromlist}
 
 \end{SCn}

В середине XX века на стыке математики и информатики возникло и интенсивно развивается фундаментальное научное направление -- компьютерная алгебра, наука об эффективных алгоритмах вычислений математических объектов. Синонимами термина ``компьютерная алгебра'' являются: ``символьные вычисления'', ``аналитические вычисления'', ``аналитические преобразования'', а иногда и ``формальные вычисления''. Направление ``компьютерная алгебра'' представлено теорией, технологиями, программными средствами. К прикладным результатам относят разработанные алгоритмы и программное обеспечение для решения с помощью компьютера задач, в которых исходные данные и результаты имеют вид математических выражений, формул. Основным продуктом компьютерной алгебры стали программные системы компьютерной алгебры -- СКА (англ. Computer Algebra System, CAS). 

Круг математических задач, которые можно решить с помощью СКА, непрерывно расширяется. Значительные усилия исследователей направлены на разработку алгоритмов вычисления топологических инвариантов многообразий, узлов, алгебраических кривых, когомологий различных математических объектов, арифметических инвариантов колец целых элементов в полях алгебраических чисел. Другое направление современных исследований -- квантовые алгоритмы, имеющие иногда полиномиальную сложность, тогда как существующие классические алгоритмы имеют экспоненциальную.

Исследования и разработки теоретических основ и технологий реализации методов и программных реализаций инструментов компьютерной алгебры продолжаются. Термины, определения, названия в описаниях функций и инструментов этих систем также претерпевают изменения, некоторые формулировки, ранее приводимые в отдельных руководствах, обзорах возможностей инструментария не только уточняются, но и изменяются. Это нормально для новых научных направлений и технологий. Читатель не должен удивляться, если в других источниках встретит иные формулировки, термины.

Не следует отождествлять системы компьютерной алгебры и компьютерной математики (СКМ), которые в ряде изданий условно делятся на две категории: системы компьютерной алгебры и системы для численных вычислений. Обычно к СКМ относят:
\begin{textitemize}
	\item табличные процессоры, например, Microsoft Excel, Lotus Symphony Spreadsheets, Gnumeric, OpenOffice.org Calc;
	\item системы для статистических расчётов, например, STATISTICA, PASW Statistics (первоначальное название SPSS Statistics);
	\item системы для моделирования, анализа и принятия решений, например, GPSS, AnyLogic, DSS;
	\item системы компьютерной алгебры;
	\item универсальные математические системы.
\end{textitemize}

Если с выделением в отдельные группы первых трёх из перечисленных СКМ есть согласие большинства авторов, то отнесение в отдельную группу универсальных математических систем прослеживается относительно редко. Основные СКА в контексте численных и аналитических вычислений большинством авторов считаются универсальными системами.

Одним из вариантов взаимодействия Экосистемы OSTIS и СКА могут быть подходы, аналогичные осуществляемой интеграцией в ostis-системы искусственных нейронных сетей (см. \textit{Главу \ref{chapter_ann}~\nameref{chapter_ann}}). Можно рассматривать следующие методические и технические решения:
\begin{textitemize}
	\item Интеграция по принципу "черного ящика"{}, когда в базе знаний ostis-системы присутствует спецификация используемой функции ядра системы компьютерной алгебры, а также спецификация способа вызова данной функции (например, указание через какой программный интерфейс осуществляется взаимодействие с данной внешней системой). Такой вариант интеграции является наиболее простым в плане реализации и в целом обладает перечисленными выше достоинствами. В то же время, данный вариант обладает и недостатком, связанным с тем, что ostis-система не содержит средств анализа и объяснения того, как был выполнен конкретный шаг решения задачи, реализуемый используемой функцией СКА.
	\item Более тесная интеграция, при которой конкретная функция по-прежнему остается частью сторонней СКА, когда в базу знаний ostis-системы погружается не просто результат ее выполнения, а и всевозможная его спецификация, например, объяснение хода решения задачи, указание конкретных алгоритмов и формул, которые могут быть задействованы в решении, описание возможных альтернативных вариантов решения, оценка эффективности решения и т.д. В данном варианте интеграции ostis-система получает больше возможностей по анализу и объяснению хода решения задачи. (Следует отметить, что конкретно к СКА Wolfram Mathematica сказанное не относится, потому что в ней всегда по всем решениям есть подробные пояснения и допустим режим пошагового выполнения).
	\item Полная интеграция, при которой осуществляется трансляция используемых функций системы компьютерной алгебры с внутреннего языка этой системы в ostis-систему. Данный вариант является наиболее трудоемким и сложным с точки зрения актуализации реализации возможностей систем компьютерной алгебры в соответствующих ostis-системах с учетом их постоянного развития. В то же время такой вариант интеграции по сравнению в двумя предыдущими обладает важным достоинством -- он обеспечивает платформенную независимость полученного решения и позволяет использовать при решении конкретной задачи все достоинства предлагаемых в рамках Технологии OSTIS подходов, в частности, возможность многопользовательской, параллельной обработки знаний и возможность оптимизации плана решения задачи или его фрагментов непосредственно в ходе решения.
\end{textitemize}

С прикладной точки зрения на данном этапе развития и применения Экосистемы OSTIS представляется целесообразной интеграция в едином комплексе возможностей систем компьютерной алгебры (СКА) и построенных в рамках Экосистемы OSTIS интеллектуальных обучающих систем (см. \textit{Главу \ref{chapter_learning_systems}~\nameref{chapter_learning_systems}}), что обусловлено содержанием систем компьютерной алгебры, которые обладают несомненным преимуществом и широкими возможностями при решении актуальных для обучающих систем задач практически по всем естественно-научным и техническим дисциплинам, предполагающим использование сложного математического аппарата.

С другой стороны, несмотря на популярность тематики, связанной с автоматизацией и интеллектуализацией образовательной деятельности по естественно-научным дисциплинам и разработкой соответствующих компьютерных систем, на данный момент на рынке практически отсутствуют апробированные \underline{интеллектуальные} обучающие системы, способные \underline{самостоятельно генерировать} и решать различные задачи, а также \underline{проверять корректность решения} задачи пользователем. В качестве прототипов можно привести отдельные системы, в которых решаются нетривиальные задачи, например, по геометрии [https://geometry.allenai.org/, https://mathpix.com/handwriting-recognition ] и теории графов [https://graphonline.ru/]. Но, справедливости ради, следует отметить, что в упомянутых нет интеллектуальности (по факту заложен только конкретный набор действий, в самих приложениях задачи не генерируются), нет средств проверки решений, имеющих даже незначительные отклонения правил оформления.

Подход к решению задач интеллектуализации образовательной деятельности, основанный на интеграции ostis-систем и систем компьютерной алгебры, обладает рядом преимуществ:
\begin{textitemize}
	\item При разработке ostis-систем исключается необходимость программировать многие функции, которые уже реализованы, оттестированы и апробированы в СКА. Это принципиально, так как системы компьютерной алгебры разрабатываются высококвалифицированными специалистами в соответствующих областях, реализация аналогичных функций в ostis-системах может потребовать значительных финансовых и временных затрат.
	\item Конкретная ostis-система, использующая отдельные функции СКА, благодаря подходу к разработке гибридных решателей задач в Технологии OSTIS (см. \textit{Главу \ref{chapter_situation_management}~\nameref{chapter_situation_management}}) получает возможность \myuline{самостоятельно} планировать ход решения задач при условии, что некоторые его этапы будут реализованы при помощи присоединяемых функций. С точки зрения подхода, предлагаемого в рамках Технологии OSTIS, каждая функция системы компьютерной алгебры становится \textit{методом} решения задач некоторого класса. Этот класс задач описывается в базе знаний ostis-системы и позволяет ей при решении конкретной задачи самостоятельно делать вывод о целесообразности применения той или иной функции СКА. Такая интеграция с ostis-системами позволит устранить сформулированный ранее возможный недостаток систем компьютерной алгебры (определяется тем, какие СКА используются -- отдельно поясняется ниже в обзоре систем компьютерной математики, условий их применения и доступа к отдельным компонентам).
\end{textitemize}

Особо отметим, что обозначенные варианты интеграции не исключают друг друга и могут комбинироваться. Кроме того, углубление интеграции может осуществляться поэтапно с учетом перечисленных достоинств и недостатков, а также с учетом актуальности использования тех или иных функций систем компьютерной алгебры при решении конкретных задач в рамках Экосистемы OSTIS и соответствующих ostis-систем.

Поэтапная интеграция СКА c Экосистемой OSTIS предполагает, как минимум, описание спецификации основных функций выбранной системы компьютерной алгебры средствами Технологии OSTIS, другими словами -- разработку онтологии внешних функций. В случае с системами семейства Wolfram Mathematica процесс разработки такой онтологии может быть автоматизирован благодаря наличию формального языка Wolfram Language и хорошей документированности функций системы.

Обобщая изложенное, констатируем -- интеграция обучающих систем, разрабатываемых на базе Технологии OSTIS, и систем компьютерной алгебры позволит создавать системы, обладающие указанными свойствами интеллектуальности, в более сжатые сроки, причем, с использованием тщательно отработанных (математически, алгоритмически) и многократно апробированных инструментов.

\textbf{Назначение, как работают системы компьютерной алгебры.}

Основное назначение СКА -- работа с математическими выражениями в символьной форме. Базовые типы данных СКА: числа и математические выражения. Числа: короткие и длинные целые (одинарной и кратной точности), рациональные, комплексные, алгебраические числа. Алгебраическое число задается своим минимальным многочленом, а иногда для его задания требуется указать интервал на прямой или область в комплексной плоскости, где содержится единственный корень данного многочлена. Математические выражения: арифметика, функции, уравнения, производные, интегралы, векторы, матрицы, тензоры. Кроме того, в компьютерной алгебре рассматриваются такие объекты, как: функциональные, дифференциальные поля, допускающие показательные, логарифмические, тригонометрические функции; матричные кольца и другие. Перечислим основные отличительные признаки систем компьютерной алгебры \textcolor{red}{([2 - 5])}.
% 2.	Дьяконов В.П. Энциклопедия компьютерной алгебры. – М.: ДМК Пресс, 2009. –1264 с.
% 3.	Аладьев В.З., Шишаков М.Л. Введение в среду пакета Mathematica 2.2. – М: Информационно-издательский дом "Филинъ", 1997. –368 с.
% 4.	Аладьев В.З., Ваганов В.А. Модульное программирование: Maple vs Mathematica, and vice versa. –CA: Palo Alto, Fultus Corp., 2011. –417 с.
% 5.	List of computer algebra systems. [Электронный ресурс]. Режим доступа: http://en.wikipedia.org/wiki/List_of_computer_algebra_systems

\textit{СКА работают следующим образом}:
\begin{textitemize}
	\item математические объекты (алгебраические выражения, ряды, уравнения, векторы, матрицы и др.) и указания, что с ними делать, задаются пользователем на входном языке системы в виде символьных выражений;
	\item интерпретатор анализирует и переводит символьные выражения во внутреннее представление;;
	\item символьный процессор системы выполняет требуемые преобразования или вычисления и выдает ответ в математической нотации.
\end{textitemize}
Алгоритмы внутренних преобразований имеют алгебраическую природу, что и отражено в названии систем -- системы компьютерной алгебры.
\textit{Содержание техники символьных вычислений}:
\begin{textitemize}
	\item внутреннее представление математического выражения в системе символьных вычислений -- синтаксическое дерево (список списков);
	\item суть символьных вычислений (аналитических преобразований) -- переписывание терма с помощью последовательного применения правил из определённого пользователем или системой списка;
	\item преобразование из внешнего представления во внутреннее и обратно обеспечивается дополнительными инструментальными средствами.
\end{textitemize}

Далеко не каждая математическая задача имеет определяемое существующими математическими формализмами аналитическое решение; есть алгоритмически неразрешимые задачи; в исследованиях проблемы оценки трудоёмкости алгоритмов алгоритмически разрешимых задач при наличии принципиальных достижений остаётся значительное число вопросов. Специалисты в областях прикладной и компьютерной математики единодушны во мнении, что много практически важных задач и не могут быть формализованы настолько, чтобы решаться аналитически, в лучшем случае они могут решаться только численными методами.

\textbf{Относительно классификации СКА.} 
Достаточно полный перечень с указанием функциональности систем символьных вычислений и платформ, на которых эксплуатируются, можно найти в \textcolor{red}{[5]}. Классификационными признаками СКА являются: функциональное назначение, тип архитектуры, средства реализации, области применения, интегральные оценки качества. Отметим несколько наиболее часто упоминаемых классов.

\textit{СКА общего назначения и специализированные}. Наиболее известные системы из первой группы (обеспечивают решение задач для 
большинства основных разделов символьной математики): Derive, Mathematica, Maple, Macsyma и её потомок Maxima, Scratchpad и её потомок Axiom, Reduce, MuPAD, Sage, SMath Studio, Yacas, Scientific WorkPlace, Kalamaris. Системы для решения задач одного или нескольких смежных разделов символьной математики -- это специализированные СКА. Примерами таковых являются: GAP (теория групп), Cadabra (тензорная алгебра), KANT (алгебра и теория чисел), Singular (полиномиальные вычисления с акцентом на нужды коммутативной алгебры, алгебраическая геометрия), Calc3D (для работы с 3D матрицами, векторами, комплексными числами), GRTensorII (дифференциальная геометрия).

\textit{Классы СКА по типу архитектуры}. В \textcolor{red}{[CAS\_((Malyshev.pdf]} предлагается следующее разделение: 
% CAS\_((Malyshev\_Computer\_algebra.Full\_lecture\_notes.pdf
% https://math-it.petrsu.ru/users/semenova/CAS/Prezentation/CAS_1.pdf
\begin{textitemize}
	\item СКА классической 
	архитектуры: системное ядро + прикладные расширения, примеры:
	Axiom, Maple, Mathematica;
	\item Программный пакет для расширения базовой прикладной математической системы, примеры: ядро Maple для MATLAB и MathCAD;
	\item Встраиваемое расширение (плагин) для языка и / или системы программирования, примеры: MathEclipse / Symja 
	-- Java-библиотека;
	\item Open Source, GNU GPL, мультиплатформные СКА, примеры: Maxima (Lisp), PARI/GP (C).
\end{textitemize}

\textbf{Типовая структура СКА.}

Составляющие СКА:
\begin{textitemize}
	\item ядро системы -- содержит машинные коды реализаций операторов и встроенных функций СКА, обеспечивающих выполнение аналитических (символьных) преобразований математических выражений на основе системы определённых правил;
	\item интерфейсная оболочка -- обеспечивают поддержку всех функций, 
	необходимых для информационных и управляющих взаимодействий между 
	СКА и пользователями (людьми, программами, аппаратными средствами);
	\item библиотеки специализированных программных модулей и функций -- содержат каталогизированные (по типам 
	обрабатываемых абстрактных объектов – числа, функции, алгебры и т.п. и/или методам вычислений -- аналитические, численные, смешанные) реализации алгоритмов решения типовых математических задач; они функционально расширяют ядро СКА;
	\item пакеты расширения -- обеспечивают различные формы адаптации СКА к классам математических задач, внешнему ПО (операционным системам, графическим пакетам и т.п.) и целям пользователей;
	\item справочная система -- содержит описание функциональных возможностей и примеров работы в СКА, информационные сообщения о текущем состоянии системы, а также сведения о математических основах алгоритмов СКА.
\end{textitemize}
Функции ядра всегда тщательно отлажены, как правило, реализуются на машинно-ориентированном языке, т.к. требуется высокая производительность их выполнения. У некоторых СКА оптимизация машинного кода обеспечивается, в том числе, с помощью частичной реализации функциональности на языке ассемблера или аппаратно. Ядро содержит реализации операторов и встроенных функций, обеспечивающих выполнение аналитических преобразований математических выражений на основе системы определённых правил. Объем ядра обычно ограничивают, но к нему добавляют библиотеки дополнительных процедур и функций. Распределение состава поддерживаемых системой алгоритмов символьных вычислений между ядром и библиотеками осуществляется по принципу баланса производительности и функциональности с учётом текущего состояния наиболее распространённого аппаратного обеспечения. У большинства коммерческих СКА алгоритмы вычислений и программные модули ядра являются ноу-хау разработчиков и относятся к разряду тщательно скрываемых данных.

Интерфейсные оболочки обеспечивают поддержку всех функций, необходимых для информационных и управляющих взаимодействий между системой и пользователями, в том числе ввод, редактирование, сохранение, обмен программами, использование разных аппаратных средств. У большинства СКА интерфейсные оболочки разные для разных ОС, при этом ведущие системы компьютерной алгебры работают без перекомпилирования исходного кода, как на различных аппаратных платформах, так и под управлением разных операционных систем; пользовательские интерфейсы обеспечивают похожие визуальные сценарии работы в СКА на разных компьютерах, в разных ОС и обычно реализуются в видах: текстовые (поле ввода символьных строк, поле вывода символьных строк), графические (ячейки/секции ввода данных / вывода результатов, окна отображения графиков), командные (меню и кнопки управления СКА, панели библиотек функций, индикаторы состояний СКА).

Библиотеки специализированных программных модулей и функций, пакеты расширения содержат систематизированные по назначению реализации алгоритмов обработки абстрактных объектов, решения типовых математических задач. Библиотеки и пакеты функционально расширяют ядро, а также обеспечивают возможности программирования алгоритмов не только на языке самой системы, но и на языке её реализации, а у многих СКА и на основных языках программирования высокого уровня.

Справочная система всех СКА содержит и обеспечивает пользователей описаниями функциональных возможностей и демонстрационными примерами работы, информационными сообщениями о текущем состоянии системы, а также сведениями о математических основах алгоритмов. Справедливо утверждение, что многие СКА, по сути, являются не только инструментами для получения и анализа решений, но и математическими энциклопедиями. Для СКА типичны организация и обеспечение диалога получения справок пошагово с вложенными уровнями абстракции и/или конкретизации информации. Обычно пользователю доступны: краткая контекстная справка о функциональном назначении выбранного элемента, информация о синтаксисе и семантике операторов и функций языка с поясняющими примерами, описание реализованных вариантов решения. Информативность справочной системы обеспечивается обязательным описанием всех функций ядра, инструментами поиска сведений об объекте СКА по имени, тематическому разделу, ключевым словам. У многих в системе помощи содержатся обучающие материалы с разделением по категориям пользователей, интерактивные учебные курсы решения математических задач в среде системы, некоторые также имеют консультант-репетитора, выполняющего пошаговое решение примеров с поясняющими комментариями.

\textbf{Основные функциональные возможности СКА.}

СКА позволяют реализовывать с использованием компьютера аналитические и численные методы решения задач, представляя результаты в математической нотации, обеспечивают графическую визуализацию, оформление результатов и подготовку к изданию. 
Используя СКА и компьютер, можно выполнять в аналитической форме:

\begin{textitemize}
	\item упрощение выражений или приведение к стандартному виду, 
	\item подстановки символьных и численных значений в выражения, 
	\item выделение общих множителей и делителей;
	\item раскрытие произведений и степеней, факторизацию, 
	\item разложение на простые дроби, 
	\item нахождение пределов функций и последовательностей, 
	\item операции с рядами, 
	\item дифференцирование в полных и частных производных, 
	\item нахождение неопределённых и определённых интегралов, 
	\item анализ функций на непрерывность, , 
	\item поиск экстремумов функций и их асимптот, 
	\item операции с векторами, 
	\item матричные операции, 
	\item нахождение решений линейных и нелинейных уравнений, 
	\item символьное решение задач оптимизации, 
	\item алгебраическое решение дифференциальных уравнений, 
	\item интегральные преобразования, 
	\item прямое и обратное быстрое преобразование Фурье, 
	\item интерполяция, экстраполяция и аппроксимация, 
	\item статистические вычисления, 
	\item машинное доказательство теорем.
\end{textitemize}
Если задача имеет точное аналитическое решение, пользователь СКА может получить это решение в явном виде (разумеется, речь идет о задачах, для которых известен алгоритм построения решения).

Также большинство СКА обеспечивают:
\begin{textitemize}
	\item числовые операции произвольной точности, 
	\item целочисленную арифметику для больших чисел, 
	\item вычисление фундаментальных констант с произвольной точностью, 
	\item поддержку функций теории чисел, 
	\item редактирование математических выражений в двумерной форме, 
	\item построение графиков аналитически заданных функций, 
	\item построение графиков функций по табличным значениям, 
	\item построение графиков функций в двух или трёх измерениях, 
	\item анимацию формируемых графиков разных типов, 
	\item использование пакетов расширения специального назначения, 
	\item программирование на встроенном языке, 
	\item автоматическую формальную верификацию, 
	\item синтез программ.
\end{textitemize}

В СКА можно производить вычисления в арифметике с плавающей точкой и указывать точность, реализована точная рациональная арифметика, т.е. можно производить численные расчеты без потери точности.
К особенностям СКА относят преимущественно интерактивный характер работы – пользователь не знает заранее ни размера, ни формы результатов и поэтому должен иметь возможность корректировать ход вычислений на всех этапах, задавать режим пошагового выполнения с выводом промежуточных результатов.

Большинство СКА в современной реализации не только применимы для исследования различных математических и научно-технических задач с использованием встроенных и дополнительных функций, но и содержат все составляющие языков программирования -- де факто являются проблемно ориентированными языками программирования высокого уровня. 

Лидерами СКА являются Mathematica и Maple -- мощные системы с собственными ядрами, оснащенные развитым пользовательским интерфейсом и обладающие разнообразными графическими и редакторскими возможностями. Широкое распространение в настоящее время имеют и СКА: Derive, Maxima, Axiom, Reduce, MuPAD. Особое место занимают системы компьютерной математики MATLAB, MathCad.

\textbf{Некоммерческие универсальные СКА}

Отличительной чертой современного состояния ИТ является то, что коммерческие программные продукты во многих случаях полностью или частично можно заменить некоммерческим программным обеспечением, аналогами с открытым исходным кодом – свободными программами. К таковым относят программные продукты, которые с изменениями или без них не имеют ограничений применения, копирования и передачи другим пользователям, за плату или безвозмездно. Ниже упоминаются программные средства, публикуемые под лицензией GPL. Поясним, что это предполагает. GPL предоставляет получателям компьютерных программ следующие права («свободы»): свободу запуска программы с любой целью; свободу изучения того, как работает программа, а также её модификации (предварительным условием для этого является доступ к исходному коду); свободу распространения копий исходного и исполняемого кода; свободу улучшения программы и выпуска улучшений в публичный доступ. В общем случае распространитель программы, полученной на условиях GPL, либо программы, основанной на таковой, обязан предоставить получателю возможность работать с соответствующим исходным кодом.

\underline{Maxima}.

Maxima – свободная полнофункциональная система компьютерной алгебры, потомок системы Macsyma, разрабатывавшейся в рамках проекта создания искусственного интеллекта в Массачусетском Технологическом Институте с 1968 по 1982 годы (этапы разработки и руководители групп разработчиков основных разделов перечислены в \textcolor{red}{[25]}). Macsyma (от MAC Symbolic MAnipulation), будучи первой системой аналитических вычислений, произвела в свое время переворот в компьютерной алгебре и оказала влияние на многие другие системы, в числе которых Mathematica и Maple. Изначально Macsyma была закрытым коммерческим проектом, доступность которого OpenSource-сообществу стала возможной благодаря профессору Техасского университета В. Шелтеру (William Schelter), который добился от Энергетического Управления США (Department of Energy, DOE) получения кода Macsyma и его публикации под лицензией GPL с именем Maxima. Профессор В. Шелтер долгое время разрабатывал как саму систему, так и один из диалектов лиспа – GCL (GNU Common Lisp), на котором разрабатывалась Maxima.
Кроме того, что Maxima является свободной полнофункциональной СКА, она имеет и другие преимущества, основным из которых является сравнительно небольшой объём – размер дистрибутива составляет около 23 Мб, а в установленном виде системе со всеми расширениями потребуется менее 85 Мб. Maxima состоит из интерпретатора макроязыка, написанного на Lisp, и нескольких поколений пакетов расширений, написанных на макроязыке системы или непосредственно на Lisp.
% 25.	Система компьютерной алгебры Maxima. [Электронный ресурс]. Режим доступа: http://maxima.sourceforge.net/ru/

Maxima является полноценной системой компьютерной алгебры, в которой можно выполнять:
\begin{textitemize}
	\item операции с многочленами, списками, векторами, матрицами и тензорами, множествами, рациональными функциями, обобщенными функциями Дирака и Хэвисайда;
	\item синтаксические, алгебраические и подстановки по шаблону;
	\item преобразования тригонометрических и выражений со степенями и логарифмами, выносить за скобки, а также раскрывать скобки, упрощение выражений;
	\item нахождение пределов в конечных точках (в том числе – поиск односторонних пределов), на бесконечности;
	\item вычисление сумм ряда;
	\item дифференцирование, интегрирование;
	\item нахождение разложений в ряд, вычетов;
	\item преобразование Лапласа, 
	\item вычисление длины кривых, площади и объема двух-, трех- и многомерных фигур.
\end{textitemize}
Используя ядро и дополнительные пакеты, в Maxima можно решать:
\begin{textitemize}
	\item уравнения, системы линейных алгебраических уравнений (алгоритмы численного решения задач линейной алгебры почти соответствуют популярной системе компьютерной математики MATLAB);
	\item аналитическими методами обыкновенные дифференциальные уравнения первого и второго порядка, в частности, линейные и нелинейные дифференциальные уравнения первого порядка, линейные дифференциальные уравнения второго порядка и системы линейных дифференциальных уравнений первого порядка;
	\item приближенными методами широкий класс обыкновенных дифференциальных уравнений (разложение в ряд Тейлора и три метода возмущений для решения, классические алгоритмы Рунге-Кутта а также алгоритмы решения жестких дифференциальных уравнений);
	\item интегральные уравнения с фиксированными и переменными пределами интегрирования;
	\item задачи теории вероятностей, математической статистики и статистической обработки данных.
\end{textitemize}

Эксперты отмечают, что Maxima, в отличие от Mathematica и Maple, в основном ориентирована на прикладные математические расчеты. В связи с этим в системе отсутствуют или сокращены разделы, связанные с теоретическими методами, как, например, теория чисел, теория групп, алгебраические поля, математическая логика. В то же время, числа в математических выражениях в системе по умолчанию предполагаются действительными. Это позволяет получать аналитические решения для многих вычислений, встречающихся в прикладных задачах (например, таких как алгебраические преобразования и упрощения, интегрирование, решение дифференциальных уравнений), для которых в комплексной области решения не существуют.

\underline{Расчеты}. Maxima производит численные расчеты высокой точности, используя точные дроби, целые числа и числа с плавающей точкой произвольной точности.

\underline{Графика}. Maxima позволяет иллюстрировать функции и статистические данные в двух и трех измерениях. В системе реализованы возможности получения качественных иллюстраций, включая параметрические графики кривых и поверхностей, а также графики векторных полей и анимацию. Число настраиваемых атрибутов в системе большое. Например, типичный трехмерный график имеет около 200 атрибутов, которые можно менять по предпочтениям пользователя. Настройки и управление сгруппированы в простых интерфейсных диалогах, при работе с графическими объектами возможны: вращение, преобразование, увеличение, включение/выключение перспективы и осей. Оформление включает вывод и установки вида заголовка иллюстрации, других текстовых комментариев, задание цвета поверхности, толщины сетки и линий осей, наименований осей, числа точек шкалы осей и шрифта чисел; возможны уточнения внешнего вида поверхностных узлов, линий и точек; можно задать цвет наружного освещения, положение и цвета осветителей. В рабочем документе можно производить анимацию положения камеры, цветов, освещения, планов и других атрибутов. Графику можно экспортировать в основные векторные и растровые форматы.

\underline{Программирование}. Как и другие СКА, Maxima имеет средства процедурного программирования и программирования по заданному правилу. Система имеет открытую архитектуру, большинство команд, хранящихся в командных файлах (с расширением .mac) могут быть прочитаны и изменены пользователем. Пользователь может программировать свои команды, пополняя библиотеку. Система генерирует коды языков Fortran и C, включая управляющие операторы (циклы, ветвления), определения subroutine и function, описания типов переменных, включая матрицы, сегментацию выражений и возможность задания оптимизации общих частей выражений.
Переносимость. Maxima успешно работает на всех современных операционных системах: Windows (готовые сборки доступны на сайте проекта), Linux и UNIX, Mac OS и даже на КПК под управлением Windows CE/Mobile. Главную роль в переносимости Maxima играет язык Lisp, на котором она написана. Исторически Lisp имеет большое количество несовместимых друг с другом диалектов, но сейчас эпоха разнообразия закончилась, поскольку появился официальный стандарт ANSI Common Lisp. Maxima была модифицирована в соответствии с этим стандартом, и в результате может работать под управлением разных реализаций Common Lisp, как свободных, так и проприетарных.

\underline{Взаимодействие с системой, интерфейс}. Сама по себе Maxima -- консольная программа, все математические формулы она “отрисовывает” обычными текстовыми символами. В этом есть плюсы. Например, саму систему можно использовать как ядро, надстраивая поверх нее разные графические специализированные интерфейсы. Соответствующих примеров на сегодняшний день существует несколько.

Традиционно все СКА имеют интерфейсные пользовательские оболочки, способные представить данные в математической нотации и облегчающие взаимодействие с пользователем. Одной из таких оболочек для Maxima является TeXmacs -- самостоятельная программа, которую классифицируют как научный WYSIWYG-редактор. TeXmacs разработан и развивается для визуального редактирования текстов со значительным объёмом математической нотации, в котором пользователь видит на экране редактируемый текст практически в том же виде, в котором он будет распечатан. В частности, доступен так называемый математический режим ввода, удобный для работы с самыми разными формулами. TeXmacs поддерживает также импорт/экспорт содержания в LaTeX и XML/HTML. Именно возможностями по работе с формулами пользуется Maxima, вызванная из TeXmacs’а. Фактически, формулы отображаются в привычной математической нотации, но при этом их можно редактировать и копировать в другие документы.

Также есть несколько других оболочек, лучшей из которых считается wxMaxima, которая как и сама Maxima, помимо Linux/*BSD существует еще и в версии для MS Windows, причём реализована и в версии русского языка, но пока без встроенной справки на русском. Нужно отметить, что по функциональности графические оболочки свободных систем компьютерной алгебры пока уступают коммерческим аналогам.

Maxima хорошо документирована -- имеет справочное руководство с описанием практически всех встроенных функций, оно интегрировано в систему в виде онлайнового справочника, оснащенного средствами поиска. Руководство уже переведено на несколько языков, и в настоящее время переводится на русский. Система имеет отладчик, не имеет утечек памяти, для проверки работы с ней поставляются большое число тестов.

Maxima -- результат коллективного труда сотен людей. Несмотря на свой солидный возраст, система продолжают активно развиваться. Последний релиз Maxima 5.28.0 выпущен 27 августа 2012 г. Для первичного знакомства с СКА Maxima можно рекомендовать доступное в электронном виде учебное пособие \textcolor{red}{[26]}.
% 26.	Стахин Н.А. Основы работы с системой аналитических (символьных) вычислений Maxima. (ПО для решения задач аналитических (символьных) вычислений): учебное пособие. –М.: 2008. –86 с.

\textbf{Axiom}.

Axiom -- свободная система компьютерной алгебры (\textcolor{red}{[27]}). Она состоит из среды интерпретатора, компилятора и библиотеки, описывающей строгую, математически правильную иерархию типов. Разработка системы была начата в 1971 году группой исследователей IBM под руководством Ричарда Дженкса (Richard Dimick Jenks). Изначально система называлась Scratchpad. Первоначально проект рассматривался как исследовательская платформа для разработки новых идей в вычислительной математике. В 90-х система была продана компании Numerical Algorithms Group (NAG), получила название Axiom и стала коммерческим продуктом, но не получила коммерческого успеха и была отозвана с рынка в октябре 2001 г. NAG сделала Axiom свободным программным обеспечением и открыла исходные коды под модифицированной лицензией BSD. Разработка системы продолжается, выходят новые версии (\textcolor{red}{[27]}). В 2007 г. у Axiom появились две ветки (форка) с открытым исходным кодом: OpenAxiom и FriCAS.
% 27.	AXIOM. The Scientific Computation System. [Электронный ресурс]. Режим доступа: http://axiom-developer.org/index.html

\underline{OpenAxiom} (http://open-axiom.sourceforge.net) в апреле 2013 г. выпустила версию 1.4.2. Основные изменения, реализованные в этой версии, относятся к работе компилятора. Упомянутая выше система подготовки и редактирования документов с математической нотацией GNU TeXmacs может использоваться как интерфейс OpenAxiom.

Другой активно развиваемой веткой Axiom является \underline{FriCAS} (http://fricas.sourceforge.net), сейчас используется версия 1.3.8 (версия 22/06/2022). FriCAS выгодно отличается от других СКА общего назначения развитой иерархией типов, соответствующей реальным математическим структурам.
Axiom и названные ветки на данном этапе в темпе развития уступают Maxima. Начинающим лучше ориентироваться на Maxima.


\textbf{Система компьютерной математики MATLAB}.

Интерактивная система программирования MATLAB (сокращение от Matrix Laboratory) разработана компанией The MathWorks, Inc. Это одна из старейших, тщательно проработанных и проверенных временем систем автоматизации математических расчетов, построенная на расширенном представлении и применении матричных операций. В настоящее время система вышла далеко за пределы специализированной матричной и стала одной из наиболее мощных универсальных интегрированных СКМ. MATLAB включает инструменты разработки сложных программ с развитым графическим интерфейсом, является эффективной средой для проведения исследований, создания моделей, решения естественнонаучных и инженерных задач \textcolor{red}{[28, 29]}. Система де-факто стала одним из мировых стандартов в области современного математического и научно-технического программного обеспечения. В первую очередь СКМ ориентирована на численные расчеты, особо выделяется матричная алгебра. Эффективность системы обусловлена, прежде всего, ориентацией на работу с многомерными массивами, большими и разреженными матрицами с программной эмуляцией параллельных вычислений и упрощенными средствами задания циклов. Последние версии системы поддерживают 64-разрядные микропроцессоры и многоядерные микропроцессоры, например Intel Core 2 Duo и Quad. Функциональные возможности системы обеспечены богатой библиотекой команд и своим языком программирования. Из-за большого числа поставляемых с MATLAB пакетов расширения она является и самой большой из СКМ, ориентированных на персональные компьютеры. Объем ее файлов превышает 3 Гб. 
MATLAB работает на большинстве современных операционных систем, включая Linux, macOS, Solaris (начиная с версии R2010b поддержка Solaris прекращена) и Windows. 
Есть много изданий с описанием системы, её составляющих -- из русскоязычных можно отметить \textcolor{red}{[30]}. 
% 30. Дьяконов В.П. MATLAB. Полный самоучитель. – М.\: ДМК Пресс, 2012. – 768 с.
% 30.	Дьяконов В.П. MATLAB R2007/2008/2009 для радиоинженеров. – М.: ДМК Пресс, 2010. –976 с.
% https://elprivod.nmu.org.ua/files/mathapps/Дьяконов_MATLAB_полный%20самоучитель.pdf

Историю версий MATLAB можно проследить по ресурсу \textcolor{red}{[MATLAB Версии]}.
% https://ru.education-wiki.com/2303364-MATLAB-version -- MATLAB Версия -- Особенности и преимущества версий MATLAB.
% MATLAB Информация о релизах https://docs.exponenta.ru/matlab/release-notes.html?s_tid=doc_ftr
% https://www.mathworks.com/help/matlab/release-notes.html = R2020a - R2023a

Отметим только несколько знаковых позиций в части ИИ, машинного обучения, интеллектуального анализа данных по версиям после 2012 г. (код R2012* означает -- версия 2012 г.):
\begin{textitemize}
	\item MATLAB 7.14 R2012a -- была последняя версия для поддержки 32-битного Linux;
	\item MATLAB 8.2 R2013b -- добавлен тип данных таблицы, среда выполнения Java обновлена до версии 7;
	\item MATLAB 8.4 R2014b -- добавлены улучшенная пользовательская панель инструментов, новые функции и пакеты, такие как py (для использования Python), счетчик веб-страниц, гистограммы, TCP-клиент и другие;
	\item MATLAB 8.6 R2015b -- для работы с графиками добавлен новый механизм исполнения (LXE) и новые классы, такие как графики и орграфы;
	\item MATLAB 9.1 R2016b -- официальный движок MATLAB для JAVA, новые функции кодирования и декодирования для JSON, добавлен новый «строковый» тип данных; алгоритмы для обработки данных, не помещающихся в оперативной памяти, включая алгоритмы понижения размерности, описательной статистики, кластеризации методом к-средних, линейной регрессии, логистической регрессии и дискриминантного анализа; Байесова оптимизация для автоматической настройки параметров алгоритмов машинного обучения, анализ окрестности компонента (NCA) для выбора функций модели машинного обучения;
	\item MATLAB 9.5 R2018b -- реализовано взаимодействие осей графиков, что обеспечивает эффективный анализ данных с панорамированием, изменением масштаба; добавлены функции: удаление выбросов в массиве, таблице или расписании; задание локального окружения о каждом элементе во входных данных; 
	\item MATLAB 9.6 R2019a -- добавлены Функци задания местоположения отсутствующего значения, обнаружения выбросов с помощью процентилей; реализованы улучшения для искусственного интеллекта и аналитики; 
	\item MATLAB 9.7 R2019b -- включает обновления по искусственному интеллекту (новые возможности позволяют пользователям обучать продвинутые сетевые архитектуры с использованием пользовательских циклов обучения, автоматического дифференцирования, общих весов и пользовательских функций потерь; пользователи могут создавать генеративные состязательные сети GAN, сиамские сети, вариационные автокодеры и сети внимания; Deep Learning Toolbox также теперь может экспортировать в сети формата ONNX, которые объединяют слои CNN и LSTM, и сети, которые включают 3D-слои CNN); 
	\item MATLAB 9.11 R2021b -- добавлены: набор инструментов для статистики и машинного обучения (анализ сигналов и изображений, предварительная обработка и извлечение параметров с помощью вейвлет-методов и интерактивных приложений для моделей искусственного интеллекта), кластеризация k-средних в реальной задачах; 
	\item MATLAB 9.11 R2021b (2021 г.);
	% MathWorks MATLAB R2022a 9.12.0.1884302
	\item MATLAB 9.13 R2022b включает в себя обновления по искусственному интеллекту, набор инструментов идентификации системы -- создавайте нелинейные модели пространства состояний на основе глубокого обучения, используя нейронные обыкновенные дифференциальные уравнения (ОДУ); методы машинного обучения и глубокого обучения также могут представлять нелинейную динамику в нелинейных моделях ARX и Хаммерштейна-Винера.
\end{textitemize}

MATLAB -- коммерческая система; существуют некоммерческие варианты программных продуктов её типа, совместимые по базовым конструкциям языка, но не совместимые по библиотечным функциям. Например, Scilab, Maxima, Euler Math Toolbox и Octave.

В состав MATLAB входят интерпретатор команд, графическая оболочка, редактор-отладчик, профилировщик (profiler), компилятор, символьное ядро СКА Maple для проведения аналитических вычислений, математические библиотеки и библиотеки Toolboxes, предназначенные для работы со специальными классами задач.

\underline{Язык MATLAB}.
Система MATLAB -- это одновременно и операционная среда, и язык программирования. Язык MATLAB является высокоуровневым интерпретируемым языком программирования, включающим основанные на матрицах структуры данных, широкий спектр функций, интегрированную среду разработки, объектно-ориентированные возможности и интерфейсы к программам, написанным на других языках программирования. Среда разработки позволяет создавать графические интерфейсы пользователя с различными элементами управления; имеется возможность создавать специальные наборы инструментов (toolbox), расширяющих функциональность и представляющих коллекции функций, написанных для решения определённого класса задач.

Программы, написанные на MATLAB, бывают двух типов: функции и скрипты. Функции имеют входные и выходные аргументы, собственное рабочее пространство для хранения промежуточных результатов вычислений и переменных. Скрипты используют общее рабочее пространство. Скрипты и функции не компилируются в машинный код, они сохраняются в виде текстовых файлов. Существует возможность сохранять так называемые pre-parsed программы -- функции и скрипты, обработанные в удобный для машинного исполнения вид. В общем случае такие модули выполняются быстрее запрограммированных в других средах, особенно если функция содержит команды графики.
Программы MATLAB, как консольные, так и с графическим интерфейсом пользователя, могут быть собраны с помощью компоненты MATLAB Compiler в независимые исполняемые приложения или динамические библиотеки. Программы-компоновщики MATLAB Builder расширяют возможности MATLAB Compiler и позволяют создавать самостоятельные компоненты Java, .NET или Excel.

\underline{Основные пакеты расширения}.
Особенностью СКМ MATLAB является возможность создавать специальные наборы инструментов (toolbox). Компания MathWorks поставляет более 80 наборов, которые используются во многих областях. В последних релизах компании они классифицируются по трём семействам – MATLAB, SIMULINK и Polyspace (\textcolor{red}{[28]}), а также партнёрские продукты.
% 28.	MathWorks. MATLAB. The Language of Technical Computing. [Электронный ресурс]. Режим доступа: http://www.mathworks.com/products/matlab/

\textbf{СКА Maple}

Mathematica и Maple являются лидерами СКА, часто их сравнивают. Представляется, что это непродуктивно. Каждая из названных систем имеет свои особенности, у них есть свои достоинства и недостатки; постоянно конкурируя друг с другом, они развиваются и совершенствуются. Большинство пользователей СКА, прежде чем выбрать для себя основную систему, испытывали несколько других. Обмен мнениями, анализ публикаций, выступлений на конференциях и семинарах позволяют утверждать, что у каждой системы есть свои приверженцы, а специалистов, использующих СКА достаточно продолжительное время, бесполезно убеждать, что иная, нежели предпочитаемая ими, система в чем-то лучше других. В большинстве случаев основным фактором использования конкретной СКА является привычка пользователя. При этом многие отмечают, что, освоив любую из систем, легко работать с другими.

По полноте функционала и интерфейсным решениям инструментарий реализации символьных и численных вычислений в системах Mathematica и Maple совершенен, вопрос не в отсутствии каких-то функций или инструментов, а в навыках пользователей.
Дать полный обзор возможностей системы Maple, как и Mathematica, невозможно. Вряд ли, кто-то из авторов даже специализированных изданий с ориентацией книги на конкретный класс задач может изложить все инструменты названных СКА из рассматриваемого ими спектра. Данный материал можно рассматривать лишь как введение в возможности системы с упоминанием классов задач по интересам студентов, магистрантов, аспирантов, исследователей, программистов. Повторим, что функционал систем Mathematica и Maple почти во всём касающемся математики, прикладной математики, информатики не только достаточен, но и избыточен. Т.к. основной перечень возможностей СКА уже приведен выше, а в Maple они реализованы, здесь отметим то, что в ряде источников или опущено, или названо другими терминами.

Системам Maple во всем мире посвящены много книг, список которых можно найти на сайте компании разработчика в соответствующем разделе \textcolor{red}{[21]}. 
% 21.	Maplesoft. Books. [Электронный ресурс]. Режим доступа: http://www.maplesoft.com/books/index.aspx
Издания на русском можно проследить по \textcolor{red}{[4, 22]}. 

% 4.	Аладьев В.З., Ваганов В.А. Модульное программирование: Maple vs Mathematica, and vice versa. –CA: Palo Alto, Fultus Corp., 2011. –417 с.
% 22.	Дьяконов В. Maple 10/11/12/13/14 в математических расчетах. –М.: ДМК Пресс, 2011. –800 с.
Несмотря на фундаментальность и направленность на самые серьезные математические вычисления, системы класса Maple необходимы довольно широкой категории пользователей: студентам и преподавателям вузов, инженерам, аспирантам, научным работникам и даже учащимся математических классов общеобразовательных и специальных школ.
Maple –- типичная интегрированная программная система. Она объединяет в себе следующие составляющие (\textcolor{red}{[22 - 24]}):
% 22.	Дьяконов В. Maple 10/11/12/13/14 в математических расчетах. –М.: ДМК Пресс, 2011. –800 с.
% 23.	Maple Product History. [Электронный ресурс]. Режим доступа: http://www.maplesoft.com/products/maple/history/
% 24.	Аладьев В.З. Системы компьютерной алгебры: Maple: Искусство программирования. M.: Лаборатория Базовых Знаний, 2006, –792 с.
\begin{textitemize}
	\item редактор для подготовки и изменения документов и программных модулей, 
	\item ядро алгоритмов и правил преобразования математических выражений, 
	\item численный и символьный процессоры, 
	\item язык программирования, 
	\item многооконный пользовательский интерфейс с возможностью работы в диалоговом режиме, 
	\item справочную систему с пояснениями всех функций и опций, 
	\item систему диагностики, 
	\item библиотеки встроенных и дополнительных функций, 
	\item пакеты функций сторонних производителей, 
	\item поддержку нескольких других языков программирования.
\end{textitemize}

Основной документ Maple -- Worksheet, работа с которым подобна обычному редактированию в текстовом редакторе. Текст можно форматировать на уровне абзацев, оформляя их различными стилями, или символов. Содержимое документа можно структурировать по секциям, подсекциям и т.д. вплоть до ячеек. Секция может состоять из различных объектов: текстовых комментариев, строк ввода, строк вывода, графиков и других секций (подсекций). Отличием является наличие активной строки ввода, воспринимающей команды пользователя. Введенные команды (операторы) передаются ядру системы и возвращаются, как правило, в виде текста или графического изображения. 

Как и в большинстве СКА, в интерфейсе Maple соединены функции текстового и командного процессоров. Начиная с версии 8, в систему добавлены средства Maplets (маплеты) для поддержки визуально-ориентированного диалога. Маплеты позволяют вводить диалоговые окна, индивидуальные палитры, средства интерфейса, привычного пользователям Windows-приложений. В версиях, начиная с 11, реализована концепция «умных» документов, доступны инструменты, которые обеспечивают создание самодокументируемых контекстных меню. Например, контекстное меню для матриц позволяет вызвать матричный редактор (Matrix Browser), выполнить основные операции линейной алгебры, конвертировать матрицу в различные внешние форматы. Причём не нужно знать синтаксис команд, достаточно выбрать нужный пункт из меню, появляющегося при нажатии правой кнопкой на выбранном объекте. Эксперты отмечают, что интерфейс Maple организован в формате интуитивно понятного и дружественного диалога, что облегчает и ускоряет освоение системы.

Система Maple (как и Mathematica), интегрирует в себе три языка: входной или язык общения с системой, реализации, программирования.
Входной язык является интерпретирующим языком сверхвысокого уровня, ориентированным на решение математических задач практически любой сложности в диалоговом режиме. Он служит для задания системе вопросов или, говоря иначе, задания входных данных для последующей их обработки. Язык имеет большое число заранее определенных математических и графических функций, а также обширную библиотеку дополнительных функций, подключаемую по мере необходимости. Встроенный Maple-язык программирования считается одним из самых лучших и мощных языков программирования математических задач. Его классифицируют, как язык процедурного программирования.

Ядро системы Maple и все её составляющие улучшаются от версии к версии. Многие встроенные в систему функции, как и функции ядра, могут использоваться без какого-либо объявления, другие нуждаются в объявлении. Имеется ряд подключаемых проблемно-ориентированных пакетов (packages), тематика которых охватывает множество разделов классической и современной математики. Общее число функций в системе Maple, с учетом встроенных в ядро и размещенных в пакетах, превышает 5500. Ядро (не в полном составе) используют MATLAB и MathCad (начиная с 14 версии, используется символьное ядро MuPAD).

Основные этапы разработки, дополнения в версиях Maple можно проследить по \textcolor{red}{[Основные ..., ]}, но следует констатировать, что вопросы машинного обучения, искусственного интелекта, интеллектуального анализа данных для системы пока приоритетными не являются.
% https://www.maplesoft.com/products/maple/history/\textbf{СКА Wolfram Mathematica}.

\textbf{Mathematica}.

Mathematica -- система компьютерной алгебры компании Wolfram Research является одним из наиболее мощных и широко применяемых интегрированных программных комплексов мультимедиа-технологии (\textcolor{red}{[2, 6-9]}). Mathematica признана фундаментальным достижением в области компьютерного проектирования. По объёму программных модулей она является одним из самых больших программных комплексов; содержит много новых алгоритмов, при её реализации применено много уникальных технических решений. В системе реализованы и доступны пользователям практически все возможности аналитических преобразований и численных расчётов, она поддерживает работу с базами данных, графикой и звуком. Mathematica даёт пользователю возможности работать, анализировать, манипулировать, иллюстрировать графиками практически все функции чистой и прикладной математики. Система обеспечивает расчеты с любой заданной точностью; построение двух- и трёхмерных графиков, их анимацию, рисование геометрических фигур; импорт, обработку, экспорт изображений и звука.

Система Mathematica прошла путь от программы, используемой преимущественно для математических и технических расчетов, до инструмента, широко применяемого в различных других областях (\textcolor{red}{[6, 7]}). Среди специалистов она отмечается, как платформа для разработки, полностью интегрирующая вычисления в рабочий процесс от начала до конца, плавно проводя пользователя от первоначальных идей до развернутых индивидуальных и промышленных решений.

Mathematica имеет встроенный язык программирования Wolfram Language, включающий средства создания программ и пользовательских интерфейсов, подключения внешних dll, параллельных вычислений. Язык программирования системы является типичным интерпретатором, он не предназначен для создания исполняемых файлов, но вобрал в себя лучшее из таких языков программирования, как Бейсик, Фортран, Паскаль и C. Язык программирования Mathematica поддерживает все известные парадигмы: функциональное, структурное, объектно-ориентированное, математическое, логическое, рекурсивное и т.д. В него включены и средства визуально-ориентированного программирования на основе применения шаблонов математических символов, таких как знаки интеграла, суммирования, произведения и т.д.; по своим возможностям в выполнении математических и научно-технических вычислений этот язык превосходит обычные универсальные языки программирования.

Как и всякая система компьютерной алгебры, Mathematica представляет собой тип программного средства, предназначенного для манипулирования математическими формулами. Основная её цель состоит в автоматизации зачастую утомительных и в целом ряде случаев трудных алгебраических преобразований. Пользователь работает в системе с блокнотами -- NB документами, каждый из которых содержит как минимум одну секцию (cell). В русскоязычной литературе можно встретить термин не секция, а ячейка. Пояснением принятого здесь предпочтения является сопоставление с MS Excel, где устойчиво и всеми применяется термин ячейка. Имеющие опыт работы с Excel и Mathematica, понимают разницу и то, что в MS Excel именно ячейки, а в блокнотах Mathematica более общие объекты. 

NB документы можно открывать, просматривать, редактировать, сохранять, выполнять полностью или отдельные секции. Интерфейс блокнота содержит много палитр (меню) и графических инструментов для создания, редактирования, просмотра документов, отправки и получения данных к ядру и от ядра. Блокнот включает одну или набор секций, которые при необходимости можно объединять в группы. Каждая секция содержит, по крайней мере, одну строку текста или формул, цифровой объект аудио либо видео. Блокноты можно редактировать как текст в любом редакторе или в интерфейсной оболочке Mathematica. Ядро выполняет вычисления, может быть запущено на том же самом компьютере, на котором выполняется интерфейс, или на другом, подключенном посредством сети. Как правило, ядро запускается в момент, начала выполнения вычислений.
Секции в Mathematica можно условно разделить на секции ввода и результата (вывода). В секциях ввода пользователь вводит или размещает команды, комментарии, объекты мультимедиа, они могут быть исполняемыми и другими; исполняемые секции обрабатываются -- система возвращает результаты.

Все версии Mathematica содержат мощную справочную базу данных, встроенный в систему Центр Документации (Help, Documentation Center) сам по себе является примером NB документа. Не прерывая работу с модулями, можно уточнить назначение любой функции, опции, директивы или служебного слова системы; изучить, выполняя «живые» примеры, возможности получения и оформления результатов; вставить примеры целиком или фрагменты кода из примеров в собственный код.

\underline{Фрагменты хронологии версий Mathematica}.
Первый выпуск Mathematica – июнь 1988 г., основной концепцией стало однажды и навсегда создать одну систему для разных вычислений в последовательном и объединенном виде. Основой этому стало создание нового символьного компьютерного языка для управления при минимальном числе исходных большого числа объектов, вовлеченных в технические вычисления. С момента появления все разработки Wolfram Research Inc. регулярно занимают первые места среди достижений ИТ, отмечаются средствами массовой информации. 

Даты выпуска и дополнения, обновления версий Mathematica полно отражены в ряде изданий и на сайтах, например \textcolor{red}{[6, 7]}. 
% 6.	Mathematica. История версий. [Электронный ресурс]. Режим доступа: http://ru.wikibooks.org/wiki/Mathematica/История_версий 
% 7.	Mathematica Quick Revision History. [Электронный ресурс]. Режим доступа: http://www.wolfram.com/mathematica/quick-revision-history.html
Здесь отметим только знаковые моменты, которые оказали принципиальное влияние и на другие СКА, а в расширенном понимании и на развитие ИТ.
\begin{textitemize}
	\item Mathematica 1.0, 23 июня 1988 -- первый выпуск Mathematica.
	\item Версия 1.2, август 1989 г. -- интерфейс под Macintosh, поддержка удалённых ядер, добавлены стандартные пакеты Statistics и Graphics.
	\item Версия 2.0, январь 1991 г. -- Notebook интерфейс, протокол MathLink межпроцессорного и сетевого взаимодействия, поддержка звука, поддержка наборов букв не только латинского алфавита, добавлен ParametricPlot3D.
	\item Версия 2.1, июнь 1992 г. -- MathLink под Macintosh, поддержка Windows 3.1, синтез звука.
	\item Версия 2.2, июнь 1993 г. -- реализация для Linux, трёхмерное построение контурных графиков, вариационное исчисление, музыка, онлайновые руководства для Windows и браузер функций для Macintosh и NeXT.
	\item Версия 3.0, сентябрь 1996 г. -- интерактивная система математического набора, интервальная арифметика.
	\item Версия 4.0, май 1999 г. –- публикация документов в ряд форматов, прямой импорт и экспорт в более чем 20 форматов графических, звуковых файлов и файлов стандартных данных.
	\item Версия 4.1, ноябрь 2000 г. –- интеграция с Java посредством J/Link 1.1, поддержка управления в реальном времени трёхмерной графики под Linux и Unix.
	\item Версия 4.2, ноябрь 2002 г. –- XML-расширения, которые позволяют сохранять файлы и выражения Mathematica как XML.
	\item Версия 5.0, июнь 2003 г. –- включена .NET/Link, обеспечивающая полную интеграцию с Microsoft’s .NET Framework.
	\item Версия 5.1, октябрь 2004 г. –- встроенная подключаемость к универсальной базе данных, интегрированная поддержка веб-сервисов, интерфейс GUIKit и встроенный разработчик программ.
	\item Версия 5.2, июнь 2005 г. –- поддержка многоядерности на основных платформах, MathematicaMark 5.2 –- обеспечивает работу с grid и кластерами.
	\item Версия 6, май 2007 г. –- язык для интеграции данных, включая автоматическую интеграцию сотен стандартных форматов данных, объединение активных графиков и элементов управления с поточным текстом и вводом.
	\item Версия 7, ноябрь 2008 г. -- встроенная поддержка параллельных высокопроизводительных вычислений, визуализация векторных полей, полная поддержка сплайнов, включая неоднородный рациональный В-сплайн, интегрированные геодезические и GIS данные.
	\item Версия 8, ноябрь 2010 г. -- интеграция с Wolfram|Alpha, вейвлет-анализ, встроенная поддержка CUDA и OpenCL, автоматическое генерирование кода С, расширенная двух- и трёхмерная графика, включающая отображение текстур и аппаратное ускорение 3D рендеринга, интерактивный мастер создания CDF-документов.
	\item Версия 9, ноябрь 2012 г. -- набор функций анализа социальных сетей, включая выявление сообществ; встроенная связь с Facebook, LinkedIn, Twitter; расширенная поддержка случайных процессов; универсальная платформа для моделирования систем, которые случайным образом изменяются во времени, включая поддержку построения реализаций, оценивание параметров (калибровку), нахождение распределений временных срезов; значительно расширенный набор функций для задач теории вероятностей и статистики, включая критерии статистической независимости, новые тесты для проверки статистических гипотез, поддержка взвешенных данных, параметрические и вторичные распределения вероятностей; поддержка графов и сетей, новые и оптимизированные распределения вероятностей на графах, функции расчёта транспортных сетей; встроенная интеграция с языком R, обеспечение использования кода на языке R в процессе работы в системе Mathematica, обмена данными между системой Mathematica и средой R путем выполнения R кода непосредственно из блокнота Mathematica; поддержка объемных операций с 3D изображениями; поддержка полного спектра интернет доступа -- доступ к интернету со стороны клиента для обмена информацией с удалёнными серверами, и для работы с программными интерфейсами веб-приложений, асинхронное соединение для программирования в стиле AJAX.
	\item Версии 10.0 (2014/12/10), 10.1, 10.2, 10.3, 10.4 (2016/03/02), в указанных и следующих версиях можно пользоваться, как настольным, так и облачным вариантом Mathematica Online. Избранные составляющие: существенные обновления в разделах Математические структуры, Решение дифференциальных уравнений, Структурные и семантические данные, Расширения базового языка программирования, Вычисления, связанные со временем, Анализ случайных процессов, Визуализации и графика, Обработка изображений, Инженерные вычисления, Работа с внешними объектами; новое: Геометрические вычисления (Символьная геометрия, Именные и формульные геометрические области, Сеточные геометрические области), Машинное обучение (функции Classify и Predict, Машинное обучение с высокой степенью автоматизации, Встроенный комплект классификаторов, Автоматический анализ временных рядов), Географические вычисления (Географические визуализации, Свойства, связанные с геоположением, Георасчёты с использованием логических объектов).
	\item Версии 11.0 (2016/10/01), 11.1 (2017/03/18), 11.2 (2017/09/14), 11.3 (2018/03/09). Избранные составляющие, существенные обновления в разделах: Работа в интернете, Облачные данные, Географические вычисления и визуализации, Подключение к внешним сервисам (Facebook, Twitter, Instagram, ArXiv, Reddit и многим другим). 3D печать. В части машинного обучения -- новые функции позволяют пользователям извлекать признаки моделируемых объектов, уменьшать их размер, группировать данные, оптимизировать гиперпараметры и строить интерпретируемые модели; функциональные возможности извлечения признаков могут быть использованы для визуализации данных или для создания семантических расстояний для поисковых систем; доступны: вычисление нейронных сетей, аудио интеграция и вычислительная обработка лингвистических данных. Значительно усилено машинное зрение (ImageIdentify может определить более 10000 объектов), реализованы возможности встроенного распознавания объектов на изображениях, обучение собственного идентификатора изображений. Сформированы базы знаний Образование, сведения о Вселенной, ряд других. 
	\item Версия 12.0 (2019/04/16), 12.1 (2020/03/18), 12.2 (2020/12/16), 12.3 (2021/05/20). Существенно модифицированы функции, расширены возможности работы в разделах: Символьные и числовые вычисления, Визуализация и графика, Геометрия и география, Наука о данных и вычисления, Изображение и аудио (новые функции и возможности: Вычисление изображений, Аудио вычисления, Вычисление изображений для микроскопии, Машинное обучение (новые функции и возможности: Суперфункции машинного обучения, Фреймворк нейронной сети, Машинное обучение для изображений, Машинное обучение для аудио, Обработка естественного языка). Отдельно следует отметить: 25 новых типов сетей, включая популярную языковую модель рендеринга BERT и генератор преформированного преобразования 2, используемый для систем генерации текста; импорт новых реализаций нейронных сетей становится немного проще, так как версия 12.1 поддерживает ONNX, открытый формат для представления моделей машинного обучения; работающие с обработкой изображений, получают дополнительную помощь с такими дополнениями, как FindImageText, который обнаруживает текст на изображении и маркирует его, аудиофилы могут воспользоваться преимуществами SpeechInterpreter и SpeechCases. Разработаны и доступны функции работы с базами знания, в частности: открыто, наполнено хранилище Wolfram Knowledgebase -- в открытом доступе в облаке, включает: Язык запросов к базе знаний, Объекты астрономии и науки о космосе, Биологические и медицинские объекты, Математические объекты, Географические объекты, Объекты питания и нутрициологии, Физические и химические объекты, Финансовые и социально-экономические объекты, Культурные и исторические объекты.
	\item Версия 13.0 (2021/12/13), 13.1 (2022/06/29), 13.2 (2022/12/14). Существенно модифицированы функции, расширены возможности работы в разделах: Символьные и числовые вычисления (Непрерывное и дискретное исчисление, Асимптотика), Визуализация и графика (Векторная и комплексная визуализация, Многопанельная и многоосевая визуализация, Графическое освещение, наполнители и шейдеры), Графы, деревья и геометрия (Графы и сети, Деревья, Геометрическое вычисление), Оптимизация, дифференциальное уравнение в частных производных(PDE) и системное моделирование (Математическая оптимизация, PDE моделирование, Системное моделирование и системы управления), Данные и наука о данных (Машинное обучение и нейронные сети, База знаний, Дата и время, Пространственная статистика), Видео, карты и молекулы (Видео, изображения и аудио, География, Молекулы и биомолекулярные последовательности). 
\end{textitemize}

Приведенный выше перечень отражает много совершенно новых достижений, которые нашли применение, развитие и в других системах, информационных технологиях. Для разработок Wolfram Research Inc. в основном характерны преемственность интерфейса и возможность применения исходных кодов из предыдущих версий, хотя, к сожалению, это не всегда так. 

\textbf{Основные возможности Mathematica}.

Перечисление полного перечня возможностей системы потребовало бы в несколько раз большего объёма, чем разрешённый для данного издания. Например, руководство \textcolor{red}{[8]} имеет объём содержательной части более 600 страниц, но фактически в нём изложены только основные функции СКА. 
% 8.	Дьяконов В. Mathematica 5/6/7. Полное руководство. – М.: ДМК Пресс, 2009. –624 с.
Первоначально вышедшая в 1988 г. и в 5-ом издании актуализированная по версии Mathematica 5 книга “The Mathematica Book by Stephen Wolfram”, Fifth Edition, 2003 года издания имеет объём 1488 страниц. Перечень книг S. Wolfram можно посмотреть на сайте \textcolor{red}{[9]}.
% 9.	Stephen Wolfram. Books. [Электронный ресурс]. Режим доступа: http://www.stephenwolfram.com/publications/books/

\textbf{Помощь и справка в Mathematica}.

Центр документации (Documentation Center), Навигатор по функциям (Function Navigator), Виртуальный учебник (Virtual Book) обеспечивают помощь пользователям в изучении программного языка и функциональных возможностей системы Mathematica.
Встроенная документация системы содержит свыше 150 тысяч представительных и наглядных примеров кодов на языке Wolfram. Все документы полностью интерактивны, это документы Mathematica, в которых можно пробовать свои коды, модифицировать готовые примеры непосредственно в справке.

\underline{Использование встроенного Центра документации}. Центр документации -- панель с выпадающими меню, гипертекстовый список основных разделов справки. Открыть Documentation Center можно через меню Help. После открытия пользователь видит документ, в котором структурированы основные разделы справки (Wolfram Mathematica Documentation Center). 
Гиперссылки в справке не имеют привычного вида подчеркнутых снизу надписей. Они представлены обычными надписями, но (как и обычные гиперссылки) активизируются при наведении курсора мыши на них и щелчке левой клавишей мыши. 
Все справки реализованы, как набор блокнотов, написанных на языке программирования системы Mathematica. Это означает, что все примеры в справке могут модифицироваться пользователем, и можно немедленно запускать измененные примеры, наблюдая результаты их работы. В принципе, копирование примеров в собственные блокноты вполне возможно, но по сути для изучения примеров оно не требуется.

\underline{Использование встроенного Навигатора по функциям}. Возможность иерархического просмотра справочных материалов по категориям реализована в Навигаторе по функциям (Function Navigator) -- панели с открывающимися окнами, ссылками на справочные страницы. Навигатор по функциям открывается в отдельном окне, обеспечивая просмотр функций, перечисленных в списках на справочных страницах.

\underline{Использование встроенного Виртуального учебника}. 
Виртуальный учебник (Virtual Book Overview) является упорядоченной коллекцией учебных материалов по Mathematica, объединенных по функциональным группам. Это энциклопедический источник информации для пользователей всех уровней подготовки, желающих получить практические навыки и более детальную информацию, знания по аспектам взаимодействия с системой и выполнения функций Mathematica.
Учебные материалы содержат подробные пояснения, примеры и ссылки на документацию по наиболее часто используемым функциям. В нижней части страницы учебных материалов расположены ссылки на документацию, родственную или имеющую отношение к данной функции. Кроме того, веб-ссылки на часто используемые функции и связанные с ними учебные материалы можно найти в правом верхнем углу окна каждого учебника.

Изложенные выше тезисы представляются важными с позиций понимания разработчиками систем ИИ окружения в близких областях, в частности, потому что системы компьютерной алгебры, реализующие с помощью компьютера интеллектуальные вычисления, также являются одним из (и достаточно успешно развиваемым) направлений внедрения в жизнь искусственного интеллекта.

\textbf{О возможностях интеграции инструментов Wolfram Mathematica в приложения OSTIS}.

Следуя приведенной в \textcolor{red}{[1]} оценке текущего состояния работ в области Искусственного интеллекта (ИИ), можно отметить активное локальное развитие самых различных направлений (неклассические логики, формальные онтологии, искусственные нейронные сети, машинное обучение, мягкие вычисления, многоагентные системы и др.), но комплексного повышения уровня интеллекта современных интеллектуальных компьютерных систем не происходит. 
% [1] V. Golenkov, N. Guliakina, and D. Shunkevich, Otkrytaja tehnologija ontologicheskogo proektirovanija, proizvodstva i jekspluatacii semanticheski sovmestimyh gibridnyh intellektual’nyh komp’juternyh sistem [Open technology of ontological design, production and operation of semantically compatible hybrid intelligent computer systems], V.Golenkov, Ed. Minsk: Bestprint [Bestprint], 2021.
Что особенно актуально на текущий момент? 
Требуется сближение и интеграция всех направлений Искусственного интеллекта и соответствующее построение общей формальной теории интеллектуальных компьютерных систем (ИКС), 
необходимы конвергенция и унификация интеллектуальных компьютерных систем нового поколения. При развитии технологиий на первом месте в разработке и модернизации ИКС должны быть решения, реализации интеграции таких систем и средств систем компьютерной алгебры. 
Важно превращение современного многообразия инструментальных средств (frameworks) разработки различных компонентов ИКС в единую технологию комплексного проектирования и поддержки полного жизненного цикла этих систем, гарантирующую совместимость всех разрабатываемых компонентов, а также совместимость самих ИКС как самостоятельных субъектов, взаимодействующих между собой в рамках комплексных систем автоматизации сложных видов коллективной человеческой деятельности. 
Необходима конвергенция и унификация интеллектуальных компьютерных систем нового поколения и их компонентов. 
При этом, под конвергентными решениями, в основном, подразумеваются оптимизированные комплексы, содержащие в себе все необходимое для решения задач ИИ, организованные или сконфигурированные для эффективного использования информационных ресурсов, для упрощения процессов внедрения; в том числе, должны обеспечиваться возможности решения определенных задач с требованиями оптимизации и достижения максимальной производительности, и во всех реализациях – оптимизированные для простоты использования. 
Перечисленное в полной мере относится к Экосистеме OSTIS. 

Ключевые причины методологических проблем современного состояния Искусственного интеллекта, а также ряд требуемых действий для их решения обозначены в \textcolor{red}{[1]}. 
% [1] V. Golenkov, N. Guliakina, and D. Shunkevich, Otkrytaja tehnologija ontologicheskogo proektirovanija, proizvodstva i jekspluatacii semanticheski sovmestimyh gibridnyh intellektual’nyh komp’juternyh sistem [Open technology of ontological design, production and operation of semantically compatible hybrid intelligent computer systems], V.Golenkov, Ed. Minsk: Bestprint [Bestprint], 2021.
Поддерживая обозначенные концепции, отметим, что подобные проблемы решаются при разработке, совершенствовании, систематическом обновлении содержания и расширения возможностей систем компьютерной алгебры. 
Ниже приведены, иллюстрируются примерами несколько методологических и технических решений конвергенции и интеграции различных видов знаний, реализованных в системе Wolfram Mathematica, Wolfram Language.


\textbf{Wolfram Mathematica. Семантическое представление чистой математики}.

\underline{Текущее состояние}. 
Основываясь на более чем тридцатилетних исследованиях, разработках и использовании во всем мире, Mathematica и язык Wolfram нацелены на долгосрочную перспективу и особенно успешны в вычислительной математике
Порядка 6000 символов, встроенных в язык Wolfram, позволяют представлять и манипулировать огромным разнообразием вычислительных объектов – от специальных функций до графики и геометрических областей.
Кроме того, база знаний Wolfram \textcolor{red}{[2]} и связанная с ней структура сущностей \textcolor{red}{[3]} позволяют пояснять, интерпретировать, формализовывать сотни конкретных "вещей (фактов, ситуаций, предметов)". Например: люди, города, продукты питания, конструкции, планеты, ... представляются объектами, которыми можно манипулировать, их можно обсчитывать. 
% 2.	Wolfram Mathematica. Наиболее полная система для современных технических вычислений в мире [Электронный ресурс] / Wolfram Computation Meets Knowledge. – Режим доступа : http://www.wolfram. com/mathematica. – Дата доступа : 9.03.2019.
% 3.	Таранчук, В.Б. Интеллектуальные вычисления, анализ, визуализация больших данных / В.Б. Таранчук // BIG DATA and Advanced Analytics = BIG DATA и анализ высокого уровня : сб. материалов V Междунар. науч.-практ. конф. (Республика Беларусь, Минск, 13–14 марта 2019 года). В 2 ч. Ч. 1. – Минск : БГУИР, 2019. – С. 337–346.

\underline{WM. Ближайшие планы}. Несмотря на быстрое и постоянно растущее число областей, известных языку Wolfram, многие области знаний еще ожидают вычислительного представления. В своем блоге "Вычислительные знания и будущее чистой математики" Стивен Вольфрам изложил видение представления абстрактной математики, известное по-разному как Вычислимый архив математики или проект Mathematics Heritage Project (MHP). Конечная цель этого проекта -- перевести все примерно 100 миллионов страниц рецензируемых научных работ по математике, опубликованных за последние несколько столетий, в машиночитаемую форму.

\underline{WM. О семантическом представлении абстрактной математики}. 
В блоге \textcolor{red}{[4]} ведущие специалисты Wolfram излагают свое видение будущего семантического представления абстрактной математики на двух примерах: абстрактных математических понятий функциональных, топологических пространств; концепций и теорем общей топологии. 
Представляется, что такие концепции, подходы должны использоваться при решении методологических проблем современного состояния Искусственного интеллекта. 
%[4] The Semantic Representation of Pure Mathematics. Available at: https://blog.wolfram.com/2016/12/22/the-semantic-representation-of-pure-mathematics/ (accessed 2022, Oct).

\textbf{
	База знаний Wolfram. Примеры}.

Постоянно растущая база знаний Wolfram (Wolfram
Data Repository -- WDR), основанная на Wolfram|Alpha и языке Wolfram, на сегодняшний день является крупнейшим в мире хранилищем вычисляемых знаний. 
WDR, охватывающая тысячи областей, содержит тщательно отобранные экспертные знания, полученные непосредственно из первоисточников. 
Она включает в себя не только триллионы элементов данных, но и огромное количество алгоритмов, инкапсулирующих методы и модели практически из каждой области.
База знаний Wolfram основана на трех десятилетиях накопления вычисляемых знаний Wolfram. Все данные в БЗ Wolfram могут быть немедленно использованы для вычислений на языке Wolfram. Каждую миллисекунду каждого дня база знаний Wolfram обновляется последними данными. 

Основные предметные поля WDR (\textcolor{red}{[2]}) иллюстрирует рисунок \textit{\nameref{fig:integr_alg1}} 
% 2.	Wolfram Mathematica. Наиболее полная система для современных технических вычислений в мире [Электронный ресурс] / Wolfram Computation Meets Knowledge. – Режим доступа : http://www.wolfram. com/mathematica. – Дата доступа : 9.03.2019.
\begin{figure}[h]
	\centering
	\includegraphics[scale=0.9]{images/part7/chapter_integr_algebra/integr_alg1.png}
	\caption{Предметные поля WDR}
	\label{fig:integr_alg1}
\end{figure}


Отметим несколько примеров. Располагая обширной статистикой по сотням тысяч учебных заведений по всему миру, Wolfram|Alpha может вычислять ответы на сложные вопросы об образовании.
Можно запросить, какие академические степени получают студенты престижных университетов. 
Также можно рассчитать среднюю зарплату учителей в конкретном школьном округе, узнать больше о баллах обучаемых, сравнить соотношение учащихся и преподавателей в разных странах и многое другое.

Рисунок \textit{\nameref{fig:integr_alg2}} иллюстрирует ответ на запрос о числе студентов в РБ. 
\begin{figure}[h]
	\centering
	\includegraphics[scale=0.7]{images/part7/chapter_integr_algebra/integr_alg2.png}
	\caption{Число студентов в РБ}
	\label{fig:integr_alg2}
\end{figure}

Сопоставление для университетов БГУ и БГУИР иллюстрирует рисунок \textit{\nameref{fig:integr_alg3}}
\begin{figure}[h]
	\centering
	\includegraphics[scale=0.86]{images/part7/chapter_integr_algebra/integr_alg3.png}
	\caption{Сравнение БГУ и БГУИР}
	\label{fig:integr_alg3}
\end{figure}

\textbf{База знаний Wolfram. Представление знаний и доступ к ним.}

Доступ к базе знаний Wolfram глубоко интегрирован в Wolfram Language. Лингвистика свободной формы позволяет легко идентифицировать многие миллионы сущностей и многие тысячи свойств и автоматически генерировать точные представления Wolfram Language, подходящие для обширных дальнейших вычислений. WL также поддерживает пользовательские хранилища сущностей, которые позволяют выполнять те же вычисления, что и встроенная база знаний, и могут быть связаны с внешними реляционными базами данных. 

Отметим основные группы функций WM для работы с WDR: 
Entity, EntityClass, EntityValue, Transformations, Computations on Entity Classes, Standard Properties, Specific Domains, Setting Up Custom Entity Stores, Wolfram Data Repository, Wolfram Data Drop, Setting Up Custom Entity Stores, External Knowledgebases, External Database Connectivity, Web Content, Textual Question Answering, System Configuration. 
В каждой из перечисленных групп более трех подгрупп. Например, в группу Textual Question Answering включены: 
\begin{textitemize}
	\item FindTextualAnswer attempt to find answers to questions from text;
	\item SemanticInterpretation convert free-form linguistics to Wolfram Language for; SemanticInterpretation["string"] attempts to give the best semantic interpretation of the specified free-form string as a Wolfram Language expression; 
	\item SemanticImport import data, converting entities etc. to Wolfram Language form, 
	\item Interpreter interpret input of various types (e.g. "City", "Date", etc.); Interpreter attempt to interpret strings of a wide variety of types; Interpreter[form] represents an interpreter object that can be applied to an input to try to interpret it as an object of the specified form.
\end{textitemize}

\textbf{Semantic Analysis}. 

Люди взаимодействуют друг с другом с помощью речи и текста, и это называется естественным языком. Компьютеры понимают естественный язык людей с помощью обработки естественного языка (Natural Language Processing -- NLP).
NLP -- это процесс манипулирования речью текста людьми с помощью искусственного интеллекта, чтобы компьютеры могли их понимать. 
Человеческий язык имеет много значений, выходящих за рамки буквального значения слов. Есть много слов, которые имеют разные значения, или любое предложение может иметь разные оттенки, такие как эмоциональный или саркастический. Компьютерам очень трудно интерпретировать значение этих предложений. 

\underline{NLP. Основные приложения, инструменты, реализованные в системе Wolfram Mathematica NLP}.
Распознавание речи. Голосовые помощники и чат-боты. Автозамена и автоматическое предсказание. Фильтрация электронной почты. Анализ настроений. Дивертисменты для целевой аудитории. Перевод. Аналитика в социальных сетях. Набор персонала (комплектование). Краткое изложение текста (реферирование). 
Ниже упомянуты, приведены несколько представительных примеров с пояснениями функций WL групп Structural Text Manipulation, Text Analysis, Natural Language Processing. 
В некотором смысле приведенные категории условны, функций и реализуемых ими возможностей много. Например, к подгруппе Structural Text Manipulation можно отнести следующие: 
TextCases – extract symbolically specified elements (TextCases[text, form] gives a list of all cases of text identified as being of type form that appear in text); TextSentences – extract a list of sentences (TextSentences["string"] gives a list of the runs of characters identified as sentences in string); TextWords – extract a list of words (TextWords["string"] gives a list of the runs of characters identified as words in string); SequenceAlignment – find matching sequences in text; TextStructure["text"] generates a nested collection of TextElement objects representing the grammatical structure of natural language text.

Пример и результат выполнения функции TextStructure к тексту «Open Semantic Technologies for Intelligent Systems» с опцией «PartsOfSpeech» показан на рисунок \textit{\nameref{fig:integr_alg4}}
\begin{figure}[h]
	\centering
	\includegraphics[scale=0.8]{images/part7/chapter_integr_algebra/integr_alg4.png}
	\caption{Составляющие фрагмента текста}
	\label{fig:integr_alg4}
\end{figure}

Вариант вывода с опцией «DependencyGraphs» показан на рисунке \textit{\nameref{fig:integr_alg5}}
\begin{figure}[h]
	\centering
	\includegraphics[scale=0.71]{images/part7/chapter_integr_algebra/integr_alg5.png}
	\caption{Составляющие фрагмента текста в формате DependencyGraphs}
	\label{fig:integr_alg5}
\end{figure}

\underline{NLP. Примеры использования функции FindTextualAnswer}.
Ответы на вопросы на естественном языке из текста. Найти текстовый ответ NLP. 
В приведенных ниже двух примерах объектом обработки является текст 

\textit{«International scientific and technical conference proceedings Open Semantic Technologies for Intelligent Systems (OSTIS). Established: 2011. Scientific areas of the conference: 05.13.11, 05.13.15, 05.13.17»}.

В варианте запроса с опцией "Date of establishment of the conference?" ответом является
\begin{center}
	\fbox{``2011''}
\end{center}
В варианте запроса с опциями "Date of establishment of the conference?", "Scientific areas of the conference?" ответом является список 
\begin{center}
	\fbox{``2011'', ``05.13.11, 05.13.15, 05.13.17''}
\end{center}

\underline{Примеры извлечения знаний, сущности или темы из статей Википедии}.
Данные Википедии используют API MediaWiki для извлечения содержимого статей и категорий, а также метаданных из Википедии. Статья может быть указана, как строка или объект языка Wolfram. 
Извлечение статей, ассоциированных с сущностями языка, обеспечивает функция WM TextSentences, в частности, можно работать с ресурсами Википедии. 
Ниже на рисунке \textit{\nameref{fig:integr_alg6}} представлен результат выполнения функции TextSentences, с параметрами WikipediaData, Entity, «Person», «AlexeiLeonov», 
\begin{figure}[h]
	\centering
	\includegraphics[scale=0.55]{images/part7/chapter_integr_algebra/integr_alg6.png}
	\caption{Сведения из Wikipedia о космонавте Leonov}
	\label{fig:integr_alg6}
\end{figure}

Рисунок \textit{\nameref{fig:integr_alg7}} иллюстрирует ответ системы на выполнение функции WikipediaData с параметрами «Voskhod 2», «ImageList»
\begin{figure}[h]
	\centering
	\includegraphics[scale=0.56]{images/part7/chapter_integr_algebra/integr_alg7.png}
	\caption{Выборка в Wikipedia иллюстраций по Voskhod 2}
	\label{fig:integr_alg7}
\end{figure}

Приведенные примеры работы с базами знаний средствами Wolfram Mathematica, так как функции ядра системы можно использовать в разработанных на других платформах программах, можно трактовать, как предложения инновационного совершенствования имеющихся инструментальных средств, компонентов любых интеллектуальных компьютерных систем, и конечно же в Экосистему OSTIS. 
%\textcolor{red}{!!!! Примеры про «AlexeiLeonov» можно заменить на Лучшие белорусские разработки в IT-сфере}
% https://itmouse.by/blog/luchshie-razrabotki-belorusskih-aytishnikov

% \textbf{Инструменты создания и сопровождения базы знаний системы Wolfram Mathematica}.
\textbf{Пример интеграции прототипа обучающей ostis-системы по дискретной математике и Wolfram Mathematica}.
% \textcolor{red}{Для интеграции с системой Wolfram Mathematica используется соответствующий программный интерфейс https://reference.wolfram.com/language/ref/APIFunction.html], дающий возможность выполнить размещенный в облаке Wolfram код на Wolfram Language в рамках пользовательской программы, например, на Python или C++.}

Приведем иллюстрации совместного использования при работе с графами прототипа обучающей ostis-системы по дискретной математике, входящей в состав Экосистемы OSTIS, и Wolfram Mathematica. Отмеченные ниже результаты показывают возможности использования в ostis-системе выполняемых в WM результатов расчетов и визуализации. Причем, реализации доступны с использованием соответствующего программного интерфейса (можно выполнить размещенный в облаке Wolfram код на Wolfram Language в рамках пользовательской программы, например, на Python или C++ -- https://reference.wolfram.com/language/ref/APIFunction.html]) или средств импорта, экспорта. 

Отдельно отметим, какие форматы поддерживаются системой WM. Согласно системной функции Mathematica \underline{\$ImportFormats}, имеем следующий перечень: 

3DS, ACO, Affymetrix, AgilentMicroarray, AIFF, ApacheLog, ArcGRID, AU, AVI, Base64, BDF, Binary, Bit, BMP, BSON, Byte, BYU, BZIP2, CDED, CDF, Character16, Character8, CIF, Complex128, Complex256, Complex64, CSV, CUR, DAE, DBF, DICOM, DIF, DIMACS, Directory, DOT, DXF, EDF, EML, EPS, ExpressionJSON, ExpressionML, FASTA, FASTQ, FCS, FITS, FLAC, GenBank, GeoJSON, GeoTIFF, GIF, GPX, Graph6, Graphlet, GraphML, GRIB, GTOPO30, GXL, GZIP, HarwellBoeing, HDF, HDF5, HIN, HTML, HTTPRequest, HTTPResponse, ICC, ICNS, ICO, ICS, Ini, Integer128, Integer16, Integer24, Integer32, Integer64, Integer8, JavaProperties, JavaScriptExpression, JCAMP-DX, JPEG, JPEG2000, JSON, JVX, KML, LaTeX, LEDA, List, LWO, M4A, MAT, MathML, MBOX, MCTT, MDB, MESH, MGF, MIDI, MMCIF, MO, MOL, MOL2, MP3, MPS, MTP, MTX, MX, MXNet, NASACDF, NB, NDK, NetCDF, NEXUS, NOFF, OBJ, ODS, OFF, OGG, OpenEXR, Package, Pajek, PBM, PCAP, PCX, PDB, PDF, PGM, PHPIni, PLY, PNG, PNM, PPM, PXR, PythonExpression, QuickTime, Raw, RawBitmap, RawJSON, Real128, Real32, Real64, RIB, RLE, RSS, RTF, SCT, SDF, SDTS, SDTSDEM, SFF, SHP, SMA, SME, SMILES, SND, SP3, Sparse6, STL, String, SurferGrid, SXC, Table, TAR, TerminatedString, TeX, Text, TGA, TGF, TIFF, TIGER, TLE, TSV, UBJSON, UnsignedInteger128, UnsignedInteger16, UnsignedInteger24, UnsignedInteger32, UnsignedInteger64, UnsignedInteger8, USGSDEM, UUE, VCF, VCS, VTK, WARC, WAV, Wave64, WDX, WebP, WLNet, WMLF, WXF, XBM, XHTML, XHTMLMathML, XLS, XLSX, XML, XPORT, XYZ, ZIP.

В рассматриваемом ниже примере исходные данные (некоторый конкретный граф) поступают (выполняется импорт) из обучающей ostis-системы по дискретной математике, визуализируются средствами графики Wolfram Mathematica, затем осуществляется решение типичной задачи и предпочтительные итоговые результаты экспортируются обратно в обучающую ostis-систему по дискретной математике. Исходные данные к задаче, используемый далее конкретный граф показан на рисунке \textit{\nameref{fig:integr_alg31}}:

\begin{figure}[h]
	\centering
	\includegraphics[scale=0.9]{images/part7/chapter_integr_algebra/integr_alg31.png}
	\caption{Используемый далее граф, вершины и дуги (в варианте ostis-системы)}
	\label{fig:integr_alg31}
\end{figure}


Следующие иллюстрации получены в Wolfram Mathematica. 

Для импортированного графа в WM можно получить общую информацию, например: число вершин, дуг, список ребер (напомним, что граф выше определен/сформирован в приложении ??ostis), визуализировать уже в Mathematica. На рисунке \textit{\nameref{fig:integr_alg31d}}
показаны результаты вывода в Wolfram Mathematica списка вершин (VertexList), числа дуг (EdgeCount), дуги (EdgeList):
\begin{figure}[h]
	\centering
	\includegraphics[scale=0.65]{images/part7/chapter_integr_algebra/integr_alg31d.png}
	\caption{Используемый граф, общая информация (вывод в Wolfram Mathematica)}
	\label{fig:integr_alg31d}
\end{figure}

Для примера визуализации ниже показаны 3 варианта вывода. На рисунке \textit{\nameref{fig:integr_alg32}} приведены связи с указанием направлений:
\begin{figure}[h]
	\centering
	\includegraphics[scale=0.68]{images/part7/chapter_integr_algebra/integr_alg32.png}
	\caption{Используемый граф, дуги (вывод в Wolfram Mathematica)}
	\label{fig:integr_alg32}
\end{figure}

Вывод на рисунке \textit{\nameref{fig:integr_alg33}} реализован с указанием стилей вершин и их номеров, задано, что все дуги от узла с большим номером к узлу с меньшим выводятся пунктирными красными линиями, а остальные сплошными зелеными:

\begin{figure}[h]
	\centering
	\includegraphics[scale=0.88]{images/part7/chapter_integr_algebra/integr_alg33.png}
	\caption{Используемый граф с оформлением по правилам (вывод в Wolfram Mathematica)}
	\label{fig:integr_alg33}
\end{figure}

Пример вывода с заданием формата укладки графа DiscreteSpiralEmbedding, применения опций оформления вершин реализован при выводе на рисунке \textit{\nameref{fig:integr_alg34}} 

\begin{figure}[h]
	\centering
	\includegraphics[scale=0.87]{images/part7/chapter_integr_algebra/integr_alg34.png}
	\caption{Используемый граф с оформлением DiscreteSpiralEmbedding (вывод в Wolfram Mathematica)}
	\label{fig:integr_alg34}
\end{figure}

Пример решения задачи нахождения кратчайшего пути между двумя вершинами иллюстрирует рисунок \textit{\nameref{fig:integr_alg34}}. (При решении использованы функции Wolfram Mathematica: GraphDistance, NeighborhoodGraph, Sow, DirectedEdge, Placed, Union, Flatten).

\begin{figure}[h]
	\centering
	\includegraphics[scale=0.55]{images/part7/chapter_integr_algebra/integr_alg35.png}
	\caption{Решение задачи нахождения кратчайшего пути между двумя вершинами (вывод в Wolfram Mathematica)}
	\label{fig:integr_alg35}
\end{figure}

Полученные и рассмотренные результаты включают трудоемкие для реализации на языках программирования задачи графики, а также математически и алгоритмически сложные задачи предметной области. Представленные варианты визуализации, нахождения решения требуют только внимательного изучения примеров системы помощи Wolfram Mathematica, определенных навыков программирования, т.е. доступны большинству инженеров-программистов. Перенести результаты в другие программные приложения также не сложно, потому что WM предоставляет возможности экспорта в любые типовые форматы.

Согласно системной функции Mathematica \underline{\$ExportFormats}, имеем следующий перечень: 

{3DS, ACO, AIFF, AU, AVI, Base64, Binary, Bit, BMP, BSON, Byte, BYU, BZIP2, C, CDF, Character16, Character8, Complex128, Complex256, Complex64, CSV, CUR, DAE, DICOM, DIF, DIMACS, DOT, DXF, EMF, EPS, ExpressionJSON, ExpressionML, FASTA, FASTQ, FCS, FITS, FLAC, FLV, FMU, GeoJSON, GIF, Graph6, Graphlet, GraphML, GXL, GZIP, HarwellBoeing, HDF, HDF5, HTML, HTMLFragment, HTTPRequest, HTTPResponse, ICNS, ICO, Ini, Integer128, Integer16, Integer24, Integer32, Integer64, Integer8, JavaProperties, JavaScriptExpression, JPEG, JPEG2000, JSON, JVX, KML, LEDA, List, LWO, M4A, MAT, MathML, Maya, MCTT, MGF, MIDI, MO, MOL, MOL2, MP3, MTX, MX, MXNet, NASACDF, NB, NetCDF, NEXUS, NOFF, OBJ, OFF, OGG, Package, Pajek, PBM, PCX, PDB, PDF, PGM, PHPIni, PLY, PNG, PNM, POV, PPM, PXR, PythonExpression, QuickTime, RawBitmap, RawJSON, Real128, Real32, Real64, RIB, RTF, SCT, SDF, SMA, SND, Sparse6, STL, String, SurferGrid, SVG, SWF, Table, TAR, TerminatedString, TeX, TeXFragment, Text, TGA, TGF, TIFF, TSV, UBJSON, UnsignedInteger128, UnsignedInteger16, UnsignedInteger24, UnsignedInteger32, UnsignedInteger64, UnsignedInteger8, UUE, VideoFrames, VRML, VTK, WAV, Wave64, WDX, WebP, WLNet, WMF, WMLF, WXF, X3D, XBM, XHTML, XHTMLMathML, XLS, XLSX, XML, XYZ, ZIP, ZPR}

\textbf{Выводы.}

Системы компьютерной алгебры на настоящий момент представляют собой мощные инструментальные комплексы, возможности которых давно вышли за рамки алгебраических вычислений и даже классической математики в целом. Лидеры СКА предоставляют множество возможностей вычислений, алгоритмов обработки, анализа, визуализации. Одним из лидеров является система Wolfram Mathematica, ядро которой содержит более 6000 функций. Компанией Wolfram также разработаны много уникальных проектов, в число которых кроме системы Wolfram Mathematica входит вопросно-ответная система Wolfram Alpha \cite{WolframAlpha}, содержащая обширную базу знаний и набор вычислительных алгоритмов. 

В основе представления фактографических, логических и процедурных знаний для систем семейства Wolfram лежит мультипарадигмальный язык программирования Wolfram Language. Наличие такого внутреннего языка описания функций систем Wolfram и в целом высокий уровень документированности этих функций выгодно отличает системы Wolfram от других сервисов, позволяющих решать общие и частные задачи. Отличие заключается в том, что во многих случаях система семейства Wolfram может не просто решить задачу, но и \myuline{объяснить} ход решения, а также помочь пользователю в выборе той или иной функции, подходящей для решения его задачи, или предложить набор функций, которые можно применить к данным, полученным в результате решения исходной задачи. Другим достоинством систем семейства Wolfram является им комплексность, позволяющая решать достаточно сложные задачи в рамках одного приложения и без необходимости интеграции разнородных сервисов.

Учитывая перечисленное, можно сделать вывод о целесообразности интеграции системы компьютерной алгебры Wolfram Mathematica с ostis-системами, входящими в состав Экосистемы OSTIS. 