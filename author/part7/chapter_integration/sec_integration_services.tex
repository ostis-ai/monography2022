\section{Принципы интеграции Экосистемы OSTIS с разнородными сервисами}
{\label{sec_integration_services}} 

\begin{SCn}

    \bigskip
    
    \begin{scnrelfromlist}{ключевое понятие}
        \scnitem{...}
    \end{scnrelfromlist}
    
    \bigskip
    
    \begin{scnrelfromlist}{ключевое знание}
        \scnitem{...}
    \end{scnrelfromlist}
    
    \bigskip
    
    \begin{scnrelfromlist}{библиографическая ссылка}
        \scnitem{\scncite{...}}
    \end{scnrelfromlist}
    
\end{SCn}

Под сервисом понимается программно реализованный механизм преобразования данных в соответствии с заданной функцией. Функциональность приложения может быть абсолютно любой: вывести прогноз погоды в заданном городе, увеличить размер изображения, синтезировать речь текстового сообщения, и т.д. Зачастую такое приложение может предоставить программный интерфейс (API -- Application Programming Interface), который можно использовать с определённым форматом входов, которым будут соответствовать определённые форматы выходов.

Основной проблемой интеграции функциональных сервисов при формировании цифровой экосистемы из множества взаимодействующих сервисов является различие форматов данных, с которыми работают участници этой экосистемы. Два сервиса, предполагающих обработку данных из одной предметной области (ПрО), с высокой вероятностью будут иметь различный формат данных. Проблема согласования формата данных различных сервисов значительно усложняет разработку самих сервисов и ведёт к увеличению временных затрат.

Под интеграцией Экосистемы OSTIS с сервисом следует понимать возможность использовать функционал сервиса для изменения внутреннего состояния базы знаний системы. 

В рамках Экосистемы OSTIS выделены несколько уровней интеграции. 

Под полной интеграцией будет подразумеваться исполнение функции сервиса на платформонезависимом уровне и использованием языка SCp. То есть, задача интеграции такого сервиса сводится к выделению алогиртма обработки графовой конструкции и его реализации в рамках БЗ системы. В результате такой интеграции отпадает надобность вовсе использовать сторонний сервис. 

Частичная интеграция означает изменение состояния БЗ системы на этапах исполнения функции сервиса. Степень глубины интеграции может различаться. В некоторых случаях сервис может обращаться в БЗ для получения дополнительной информации или для записи промежуточных результатов. В простейшем случае изменение БЗ может происходить единожды, после получения результата отработки функции сервиса. 

Таким образом, миниммальными требованиями для интеграции сервиса являются:
\begin{textitemize}
    \item сформировать спецификацию входной конструкции в БЗ системы;
    \item сформировать спецификацию выходной конструкции в БЗ системы;
    \item реализовать агент обработки знаний, который преобразует конструкцию БЗ в необходимые данные для запуска сервиса, а также погрузит результаты работы сервиса в БЗ системы в соответствии со спецификацией.
\end{textitemize}

В рамках задачи формирования спецификаций необходимо выделить ключевые понятия, необходимых для формирования запросов к функциональному сервису, а также для погружения в БЗ результатов.

Обобщённый алгоритм агента обработки знаний с использованием стороннего сервиса выглядит следующим образом:
\begin{textitemize}
    \item извлечение из БЗ необходимых структур знаний;
    \item преобразование извлечённых знаний в формат, необходимый для подачи на вход сервиса;
    \item отправка запроса и ожидание ответа сервиса;
    \item формирование конструкции знаний по полученным данным;
    \item погружение новых знаний в БЗ ИС.
\end{textitemize}

    % ### Выбор архитектуры систем

    % - Сервисы как агенты основной ostis-системы. Такие сервисы не являются самостоятельными, они не способны выполнять свой функционал одновременно на нескольких системах. Должны быть установлены рядом с самой системой. Аналог монолитной архитектуры.
    %     - Преимущество — прост в реализации, скорее всего будет отдельным скриптом подниматься ещё один сервер, который будет держать агенты в активе, не даст им завершиться раньше времени.
    %     - Недостаток — формирование монолитной структуры. В случае использования достаточно тяжёлых с точки зрения памяти моделей решения задач, например нейросетевые, может возникнуть проблема с нехваткой мощностей системы
    %     - Когда использовать: решение создаётся под конкретную систему и не будет переиспользоваться в других системах.
    % - Можно подключить агенты к отдельной системе. Переход к распределённым ostis-системам. Аналог микросервисной архитектуры.
    %     - Преимущество — система распределена и децентрализована, агенты одной системы являются агентами другой, общая виртуальная БЗ. Общение может быть налажено как многое-ко-многим. Альтернатива микросервисному подходу на основе Технологии OSTIS.
    %     - Недостаток — Сложное в разработке. Необходимо продумывать логику сопоставления адресов различных систем, виртуализацию БЗ, регистрации и вызова агентов другой системы. Замедление за счёт сетевых обработок. Мы такого ещё не делали.
    %     - Необходимо для перехода:
    %         - Клиент рядом с sc-сервером для способности общаться с другой системой
    %         - Логика виртуализации БЗ, сопоставление адресов всех ostis-систем

