\section{Принципы интеграции Экосистемы OSTIS с разнородными сервисами}
{\label{sec_integration_services}} 

\begin{SCn}

    \bigskip
    
    \begin{scnrelfromlist}{ключевое понятие}
        \scnitem{...}
    \end{scnrelfromlist}
    
    \bigskip
    
    \begin{scnrelfromlist}{ключевое знание}
        \scnitem{...}
    \end{scnrelfromlist}
    
    \bigskip
    
    \begin{scnrelfromlist}{библиографическая ссылка}
        \scnitem{\scncite{...}}
    \end{scnrelfromlist}
    
\end{SCn}

Под сервисом понимается программно реализованный механизм преобразования данных в соответствии с заданной функцией. Функциональность приложения может быть абсолютно любой: вывести прогноз погоды в заданном городе, увеличить размер изображения, синтезировать речь текстового сообщения, и т.д. Зачастую такое приложение может предоставить программный интерфейс (API -- Application Programming Interface), который можно использовать с определённым форматом входов, которым будут соответствовать определённые форматы выходов.

Основной проблемой интеграции функциональных сервисов при формировании цифровой экосистемы из множества взаимодействующих сервисов является различие форматов данных, с которыми работают участници этой экосистемы. Два сервиса, предполагающих обработку данных из одной предметной области (ПрО), с высокой вероятностью будут иметь различный формат данных. Проблема согласования формата данных различных сервисов значительно усложняет разработку самих сервисов и ведёт к увеличению временных затрат.

Под интеграцией Экосистемы OSTIS с сервисом следует понимать возможность использовать функционал сервиса для изменения внутреннего состояния базы знаний системы. 

В рамках Экосистемы OSTIS выделены несколько уровней интеграции. 

Под полной интеграцией будет подразумеваться исполнение функции сервиса на платформонезависимом уровне и использованием языка SCP. То есть, задача интеграции такого сервиса сводится к выделению алогиртма обработки графовой конструкции и его реализации в рамках базы знаний системы. В результате такой интеграции отпадает надобность вовсе использовать сторонний сервис. 

Частичная интеграция означает изменение состояния базы знаний системы на этапах исполнения функции сервиса. Степень глубины интеграции может различаться. В некоторых случаях сервис может обращаться к базе знаний для получения дополнительной информации или для записи промежуточных результатов. В простейшем случае изменение базы знаний может происходить единожды, после получения результата отработки функции сервиса. 

Таким образом, миниммальными требованиями для интеграции сервиса являются:
\begin{textitemize}
    \item сформировать спецификацию входной конструкции в базе знаний системы;
    \item сформировать спецификацию выходной конструкции в базе знаний системы;
    \item реализовать агент обработки знаний, который преобразует конструкцию базы знаний в необходимые данные для запуска сервиса, а также погрузит результаты работы сервиса в базу знаний системы в соответствии со спецификацией.
\end{textitemize}

В рамках задачи формирования спецификаций необходимо выделить ключевые понятия, необходимых для формирования запросов к функциональному сервису, а также для погружения в базу знаний результатов.

Обобщённый алгоритм агента обработки знаний с использованием стороннего сервиса выглядит следующим образом:
\begin{textitemize}
    \item извлечение из базы знаний необходимых структур знаний;
    \item преобразование извлечённых знаний в формат, необходимый для подачи на вход сервиса;
    \item отправка запроса и ожидание ответа сервиса;
    \item формирование конструкции знаний по полученным данным;
    \item погружение новых знаний в базу знаний интеллектуальной системы.
\end{textitemize}

Рассмотрим процесс внедрения такого агента с точки зрения архитектуры Экосистемы. 

Одним из вариантов внедрение агента является его подключение в рамках уже существующей, основной ostis-системы. В таком случае все зависимости для установки сервиса будут также являться и зависимостями основной ostis-системы. Архитектурным паттерном такой системы будет являться монолитная архитектура. 

Преимуществом такого варианта является простота реализации и внедрения. Хорошим вариантом использования такой интеграции является отсутсвия возможности и необходимости изменить процесс выполнения функции сервиса, или же когда обращение к сервису производится по сети. Так могут быть интегрированы и сервисы получения знаний из внешних источников (получение прогноза погоды, обработка статистической информации, и т.д.), и функциональные сервисы (обработка аудиоинформации и погружение результатов обработки в базу знаний, получение синтаксического анализа предложения). 

Альтернативным вариантом внедрения является реализация отдельной ostis-системы, в рамках которой будет интегрирована функция сервиса, что характеризует переход к распределённым взаимодействующим ostis-системам. Архитектурным паттерном такой системы будет являться микросервисная архитектура.

Преимуществом такого варианта интеграции является распределённость, децентрализованность и доступность нового функционала. Система выходитза рамки технических ограничений, функционал может быть распределён на различном аппаратном обеспечении. Полученный функционал может быть использован различными ostis-системами в рамках Экосистемы для достижения своих целей. Минусами такой системы является сложность разработки, а также увеличение временных затрат на общение систем друг с другом по сетевым протоколам. 

Одним из вариантов использования является интеграция сервиса, которому для успешного выполнения своей функции необходимо взаимодействовать с базой знаний системы. Примерами такой интеграции могут служить сервисы считывания эмоционального состояния пользователя, или же помощи принятия решения на основе актуальных знаний.
