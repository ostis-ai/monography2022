\section{Принципы интеграции Экосистемы OSTIS со структурированными информационными ресурсами}
{\label{sec_integration_resources}} 

\begin{SCn}

    \bigskip
    
    \begin{scnrelfromlist}{библиографическая ссылка}
        \scnitem{\scncite{RDF}}
        \scnitem{\scncite{R2RML}}
        \scnitem{\scncite{R2RMLIO}}
        \scnitem{\scncite{INFOLAKE}}
    \end{scnrelfromlist}
    
    \end{SCn}

Существует несколько причин, по которым следует интегрировать интеллектуальную систему с информационными источниками:
\begin{textitemize}
    \item Обеспечение полноты и точности данных: Интеллектуальная система создается для обработки больших объемов данных и принятия решений на их основе. Информационные источники являются основой для этих данных, и интеграция системы с ними гарантирует полноту и точность данных.
    \item Уменьшение времени и усиление эффективности работы: Интеграция информационных источников в интеллектуальную систему обеспечивает быстрый и удобный доступ к нужной информации. Это уменьшает время на поиск и обработку данных, что повышает эффективность работы системы и уменьшает количество ошибок.
    \item Расширение возможностей системы: Информационные источники обладают большим количеством данных, которые могут быть полезны для работы интеллектуальной системы. Интеграция системы с различными источниками расширяет возможности системы и позволяет ей повысить качество работы.
    \item Повышение надежности: Разнообразные источники данных обеспечивают резервирование и возможность сравнения, что позволяет интеллектуальной системе работать более надежно и безопасно в случае сбоя одного или нескольких источников.
    \item Улучшение качества прогнозирования: Интеграция информационных источников с интеллектуальной системой способствует улучшению качества прогнозирования, так как позволяет объединять данные из различных источников и анализировать их вместе для получения более точных результатов.
\end{textitemize}

Существует большое количество информационных источников, из которых можно получать информацию. Основными можно назвать следующие:
\begin{textitemize}
    \item Интернет --- сайты, блоги, форумы, социальные сети, новостные порталы и другие ресурсы в сети.
    \item Книги и учебники --- доступные в библиотеках, книжных магазинах или в электронном формате.
    \item СМИ --- телевизионные программы, радио, газеты, журналы и другие источники новостей.
    \item Официальные документы и отчеты --- включая законы, правительственные статистические данные, отчеты об исследованиях и другие официальные документы.
\end{textitemize}

Принципы интеграции \textit{Экосистемы OSTIS} со структурированными информационными ресурсами основаны на пополнении базы знаний системы новыми знаниями. Один из востребованных подходов в этом направлении --- интеграция с ресурсами на основе RDF (Resource Description Framework, см. \scncite{RDF}), который является моделью данных, предложенной консорциумом W3C.

Для успешной интеграции структурированных информационных ресурсов в Экосистему OSTIS, важно уделить должное внимание пониманию и применению принципов RDF-модели, поскольку они играют ключевую роль в организации связей между различными ресурсами. RDF используется для описания ресурсов в сети Интернет, и является основой для построения семантических веб-приложений, таких как Linked Open Data.

Основная структура абстрактного синтаксиса RDF --- это тройка, состоящая из субъекта, предиката и объекта. Набор таких троек называется графом RDF. Граф RDF может быть визуализирован как диаграмма узла и направленной дуги, в которой каждая тройка представлена как связь ``узел --- дуга --- узел''.

Графы RDF атемпоральны, т.е. представляют собой статические снимки информации. Однако графы RDF могут выражать информацию о событиях и временных аспектах других сущностей, учитывая соответствующие термины из словаря. Поскольку графы RDF определены как математические наборы, добавление или удаление троек из графа RDF дает другой граф RDF.

Узел может иметь следующий тип:
\begin{textitemize}
    \item IRI. Представляет собой короткую последовательность символов, идентифицирующую абстрактный или физический ресурс на любом языке мира. IRI представляе собой обобщение URI;
    \item Литерал. Представляя собой структуру, состоящую из лексической формы (UNICODE-строка) и типа данных;
    \item Пустой узел. Представляет собой локальный идентификатор, который используются в некоторых конкретных синтаксисах RDF или реализациях хранилища RDF.
\end{textitemize}

RDF поддерживает основные типы данных, такие как строковый (string), логический (boolean), числовые (integer, double, float и др.), временные и некоторые другие.

В RDF существует такое понятие, как словарь RDF. Он представляет собой совокупность IRI, ссылающихся на другие графы с классами, литералами и др. Часто группа IRI может начинаться с одинакового префикса.

RDF нашел широкое применение. Так, например, RDF используется в оформлении \textit{баз знаний} в рамках различных проектов во множестве институтов, университетов и иных организаций. Поисковые системы предлагают веб-мастерам использовать RDF и аналогичные языки разметки страниц для повышения информативности ссылок на их сайты в результатах поиска. Социальные сети, с подачи Facebook, предлагают веб-мастерам использовать RDF для описания свойств страниц, так же позволяющих красиво оформить ссылку на неё в записи пользователя социальной сети.

В ходе анализа были выявлены следующие подходы к интеграции информационных ресурсов на основе RDF с другими системами:
\begin{textitemize}
    \item R2RML (см. \scncite{R2RML}) --- это стандарт W3C для выражения настраиваемых отображений из реляционных БД в RDF. Такие отображения предоставляют возможность просматривать существующие реляционные данные в модели данных RDF, выраженные в структуре и целевом словаре по выбору автора сопоставления;
    \item R2RML.io (см. \scncite{R2RMLIO}) --- это open-source проект, разрабатываемый с 2013 года. Данная технология предназначена для генерации базы знаний на основе данных из полуструктурированных источников;
    \item ``Озеро данных'' (см. \scncite{INFOLAKE}) --- это централизованное хранилище, которое позволяет хранить все структурированные и неструктурированные данные в любом масштабе. ``Семантическое озеро данных'' --- это особая форма озер данных, в которых верхний семантический слой обогащает и связывает данные семантически. Семантический уровень преодолевает разрозненность данных и обеспечивает семантический поиск по всем данным.
\end{textitemize}

Интеграция ostis-систем с внешними информационными ресурсами удобна по многим причинам. Технология OSTIS изначально предлагает инструменты для описания синтаксиса и семантики внешних языков (см. \textit{Главу \ref{chapter_ext_lang} \nameref{chapter_ext_lang}}). Данный инструментарий позволяет сократить время разработки в несколько раз. Также из достоинств можно выделить:
\begin{textitemize}
    \item способность осуществлять интеграцию знаний в своей памяти на высоком уровне;
    \item возможность интегрировать различные виды знаний;
    \item возможность интегрировать различные модели решения задач.
\end{textitemize}

Для интеграции информационного ресурса на основе RDF в Экосистему OSTIS был реализован соответствующий \textit{абстрактный sc-агент}. Его работу можно разбить на следующие этапы:
\begin{textitemize}
    \item интеграция с использованием готовых правил;
    \item интеграция с сохранением исходной схемы;
    \item дополнительные преобразования.
\end{textitemize}

\textbf{Интеграция с использованием готовых правил}

На этом этапе ко всем сгенерированным тройкам применяются готовые правила интеграции, хранящиеся в \textit{базе знаний}. Создание и применение подобных правил необходимо в ситуациях, когда способ представления конкретного знания во внешнем информационном ресурсе по какой-то причине не соответствует представлению аналогичного знания в ostis-системе.

\textbf{Интеграция с сохранением исходной схемы}

На данном этапе оставшиеся тройки будут преобразованы с сохранением той структуры отношения, в которой находились участвовавшие в нем сущности. Это значит, что порядок элементов в итоговой конструкции будет аналогичен порядку сущностей в исходной.

\textbf{Дополнительные преобразования}

На данном этапе проходят оставшиеся интеграционные преобразования, которым не нашлось места в предыдущих пунктах, но которые необходимы для завершения процесса интеграции.

Для выгрузки информации из \textit{базы знаний} в какой-либо внешний формат можно использовать те же правила, что и для загрузки, так как в основном они представляют собой утверждения об \textit{эквиваленции}. То есть изначально производится поиск необходимых конструкций, затем они к ним применяется соответствующее правило, и, в результате получается множество троек. В дальнейшем данные тройки преобразуются в необходимый формат.