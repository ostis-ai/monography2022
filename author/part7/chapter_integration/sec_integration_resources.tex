\section{Принципы интеграции Экосистемы OSTIS со структурированными информационными ресурсами}
{\label{sec_integration_resources}} 

\begin{SCn}

    \bigskip
    
    \begin{scnrelfromlist}{ключевое понятие}
        \scnitem{...}
    \end{scnrelfromlist}
    
    \bigskip
    
    \begin{scnrelfromlist}{ключевое знание}
        \scnitem{...}
    \end{scnrelfromlist}
    
    \bigskip
    
    \begin{scnrelfromlist}{библиографическая ссылка}
        \scnitem{\scncite{...}}
    \end{scnrelfromlist}
    
    \end{SCn}

Проблема автоматизированного пополнения базы знаний не нова. Попытки ее решить предпринимались и ранее. Далее рассмотрены несколько подходов к автоматизированному пополнению БЗ.

Принципы интеграции Экосистемы OSTIS со структурированными информационными ресурсами основаны на пополнении базы знаний системы новыми знаниями. Данная проблема актуальна уже длительное время, и для ее решения были предложены различные подходы. Один из таких подходов - использование RDF (Resource Description Framework), который является моделью данных, предложенной консорциумом W3C.

Для успешной интеграции структурированных информационных ресурсов в Экосистему OSTIS, важно уделить должное внимание пониманию и применению принципов RDF-модели, поскольку они играют ключевую роль в организации связей между различными ресурсами. RDF используется для описания ресурсов в сети Интернет, и является основой для построения семантических веб-приложений, таких как Linked Open Data.

Основная структура абстрактного синтаксиса RDF – это тройка, состоящая из субъекта, предиката и объекта. Набор таких троек называется графом RDF. Граф RDF может быть визуализирован как диаграмма узла и направленной дуги, в которой каждая тройка представлена как связь «узел - дуга - узел».

Графы RDF атемпоральны, т.е. представляют собой статические снимки информации. Однако графы RDF могут выражать информацию о событиях и временных аспектах других сущностей, учитывая соответствующие термины из словаря. Поскольку графы RDF определены как математические наборы, добавление или удаление троек из графа RDF дает другой граф RDF.

Узел может иметь следующий тип:

1. IRI. Представляет собой короткую последовательность символов, идентифицирующую абстрактный или физический ресурс на любом языке мира. IRI представляе собой обобщение URI.

2. Литерал. Представляя собой структуру, состоящую из лексической формы (UNICODE-строка) и типа данных.

3. Пустой узел. Представляет собой локальный идентификатор, который используются в некоторых конкретных синтаксисах RDF или реализациях хранилища RDF.

RDF поддерживает основные типы данных, такие как строковый (string), логический (boolean), числовые (integer, double, float и др.), временные и некоторые другие.

В RDF существует такое понятие, как словарь RDF. Он представляет собой совокупность IRI, ссылающихся на другие графы с классами, литералами и др. Часто группа IRI может начинаться с одинакового префикса.

RDF нашел широкое применение. Так, например, RDF используется в оформлении БЗ в рамках различных проектов во множестве институтов, университетов и иных организаций. Поисковые системы предлагают веб-мастерам использовать RDF и аналогичные языки разметки страниц для повышения информативности ссылок на их сайты в результатах поиска. Социальные сети, с подачи Facebook, предлагают веб-мастерам использовать RDF для описания свойств страниц, так же позволяющих красиво оформить ссылку на неё в записи пользователя социальной сети;

Существующие подходы:
\begin{textitemize}
    \item R2RML – это стандарт W3C для выражения настраиваемых отображений из реляционных БД в RDF. Такие отображения предоставляют возможность просматривать существующие реляционные данные в модели данных RDF, выраженные в структуре и целевом словаре по выбору автора сопоставления;
    \item RML.io – это open-source проект, разрабатываемый с 2013 года. Данная технология предназначена для генерации БЗ на основе данных из полуструктурированных источников;
    \item «Озеро данных» – это централизованное хранилище, которое позволяет хранить все структурированные и неструктурированные данные в любом масштабе. «Семантическое озеро данных» – это особая форма озер данных, в которых верхний семантический слой обогащает и связывает данные семантически. Семантический уровень преодолевает разрозненность данных и обеспечивает семантический поиск по всем данным.
\end{textitemize}

Система наполнения БЗ будет реализована с использованием OSTIS-технологии.

Из достоинств OSTIS-систем можно выделить:
\begin{textitemize}
    \item способность осуществлять семантическую интеграцию знаний в своей;
    \item возможность интегрировать различные виды знаний;
    \item возможность интегрировать различные модели решения задач.
\end{textitemize}

Для решения задачи интеграции, использование OSTIS-системы целесообразно по многим причинам. Технология изначально предлагает инструменты для описания синтаксиса и семантики внешних языков. Данный инструментарий позволяет сократить время разработки в несколько раз.

Память OSTIS-системы представляет собой семантическую сеть, в основе которой лежит графовая структура. Все элементы семантической сети являются знаками различных сущностей. Такими сущностями могут быть всевозможные внешние описываемые объекты, а также различные множества, состоящие их элементов (атомарных фрагментов) этой же семантической сети.

Для интеграции системы нужно реализовать агент. Его работу можно разбить на следующие этапы:
1. интеграция с использованием готовых правил;
2. интеграция с сохранением исходной схемы;
3. дополнительные преобразования.

\textbf{Интеграция с использованием готовых правил}

На этом этапе ко всем сгенерированным тройкам применяются готовые правила интеграции, хранящиеся в БЗ. Создание и применение подобных правил необходимо в ситуациях, когда способ представления конкретного знания по какой-то причине не соответвествует представлению аналогичного знания в OSTIS-системе.

\textbf{Интеграция с сохранением исходной схемы}

На данном этапе оставшиеся тройки будут преобразованы с сохранением той структуры отношения, в которой находились участвовавшие в нем сущности. Это значит, что порядок элементов в итоговой конструкции будет аналогичен порядку сущностей в исходной.

\textbf{Дополнительные преобразования}

На данном этапе проходят оставшиеся интеграционные преобразования, которым не нашлось места в предыдущих пунктах, но которые необходимы для завершения процесса интеграции.

Для выгрузки информации из БЗ можно использовать те же правила, что и для загрузки, так как в основном они содержат эквиваленцию. То есть изначально производится поиск необходимых кострукций, затем они прогоняются через правило, и на выходе мы получаем тройки. В дальнейшем данные тройки преобразуются в интересующий нас формат.

Таким образом, мы получаем полностью интегрированную систему, которая способна отвечать всем запросам пользователя: загружать информацию, производить над знаниями полезную работу, и затем выдать эти знания в заданном формате.
