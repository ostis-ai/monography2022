\section{Автоматизация высшего технического образования с помощью ostis-систем}
\label{sec_automation_higher_technical_education}

\begin{SCn}
	
	\bigskip
	
	\begin{scnrelfromlist}{ключевое понятие}
		\scnitem{...}
	\end{scnrelfromlist}
	
	\bigskip
	
	\begin{scnrelfromlist}{ключевое знание}
		\scnitem{...}
	\end{scnrelfromlist}
	
	\bigskip
	
	\begin{scnrelfromlist}{библиографическая ссылка}
		\scnitem{\scncite{...}}
	\end{scnrelfromlist}
	
\end{SCn}

Можно выделить три основных направления интеллектуализации учебного процесса, соответствующих трем уровням учебной деятельности.

Во-первых, это -- самообучение на уровне одной дисциплины. В предположении, что обучаемый положительно мотивирован, процесс обучения строится таким образом, чтобы предоставить ему максимальную свободу, помогая быстро ориентироваться в незнакомой предметной области. В связи с этим учебный материал должен быть так структурирован, чтобы его изучение было максимально удобным и, следовательно, эффективным. Здесь требуется совместная кропотливая работа эксперта-предметника и эксперта-педагога. В настоящее время актуальной является проблема повышения степени наглядности, когнитивности учебной информации электронного учебника с целью повышения самостоятельной познавательной деятельности обучаемого. Для решения этой задачи предлагается семантический электронный учебник, который представляет собой интерактивный интеллектуальный самоучитель по некоторой предметной области, содержащий подробные методические рекомендации по ее изучению и предназначенный для мотивированного, самостоятельного и активного пользователя, желающего овладеть знаниями по соответствующей дисциплине.

Во-вторых, это -- управление обучением на уровне отдельной дисциплины. В связи с повышением сложности и информационной насыщенности компьютерных средств обучения возникает необходимость в осуществлении управления обучением и процессом взаимодействия с пользователем. Поскольку обучающая система становится более сложной и многофункциональной и предназначена для различных категорий пользователей, то требуется адаптация к индивидуальным особенностям и обстоятельствам для каждого конкретного пользователя. Способность обучающей системы адаптироваться к пользователю является одним из показателей ее эффективности и, как следствие, интеллектуальности. Интеллектуальные обучающие системы представляет собой сложную иерархическую систему, состоящую из совокупности взаимодействующих между собой подсистем, каждая из которых решает определенный класс задач. В качестве базового компонента интеллектуальных обучающих систем используется семантический электронный учебник.

В-третьих, это -- управление учебной деятельностью на уровне специальности. Учебная организация и процесс обучения -- это не просто совокупность автоматизированных и интеллектуальных обучающих систем по определенным дисциплинам, обладающих средствами мультимедиа, гибкими стратегиями обучения, подсистемами адаптации к пользователю и т.д. Для эффективного использования всех этих средств необходима инфраструктура, в которой осуществляется обработка информации, взаимодействие пользователей и подсистем, совместное решение задач, в которое вовлекаются как пользователи, так и подсистемы.