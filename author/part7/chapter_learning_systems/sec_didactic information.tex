\section{Представление дидактической информации в базах знаний ostis-систем}
\label{sec_didactic information}

\begin{SCn}
	
	\bigskip
	
	\begin{scnrelfromlist}{ключевое понятие}
		\scnitem{...}
	\end{scnrelfromlist}
	
	\bigskip
	
	\begin{scnrelfromlist}{ключевое знание}
		\scnitem{...}
	\end{scnrelfromlist}
	
	\bigskip
	
	\begin{scnrelfromlist}{библиографическая ссылка}
		\scnitem{\scncite{Golenkov2001b}}
		\scnitem{Давыденко, И. Т. Разработка интеллектуальных обучающих систем на основе технологии OSTIS / И. Т. Давыденко // Дистанционное обучение – образовательная среда XXI века: материалы VIII международной научно-методической конференции. (Минск, 5–6 декабря 2013 года). – Минск: БГУИР, 2013. – С. 196 - 197. (https://libeldoc.bsuir.by/handle/123456789/26175)}
		\scnitem{Колб, Д. Г. Проблемы обеспечения мультимедийного поиска на основе семантических сетей в системах дистанционного обучения / Д. Г. Колб // Дистанционное обучение – образовательная среда XXI века: материалы VIII международной научно-методической конференции. (Минск, 5–6 декабря 2013 года). – Минск: БГУИР, 2013. – С. 188 - 189. (https://libeldoc.bsuir.by/handle/123456789/26172)}
		\scnitem{Гракова, Н. В. Классификация участников системы управления пректированием интеллектуальных систем / Н. В. Гракова, И. И. Жуков // Дистанционное обучение – образовательная среда XXI века: материалы VIII международной научно-методической конференции. (Минск, 5–6 декабря 2013 года). – Минск: БГУИР, 2013. – С. 203 - 204. (https://libeldoc.bsuir.by/handle/123456789/26136)}
		\scnitem{Русецкий, К. В. Лингвистическая база знаний интеллектуальной обучающей системы по естественному языку / К. В. Русецкий // Дистанционное обучение – образовательная среда XXI века: материалы VIII международной научно-методической конференции. (Минск, 5–6 декабря 2013 года). – Минск: БГУИР, 2013. – С. 198 - 199. (https://libeldoc.bsuir.by/handle/123456789/26134)}
		\scnitem{Шункевич, Д. В. Базовые понятия технологии компонентного проектирования машин обработки знаний систем дистанционного  обучения Дистанционное обучение – образовательная среда XXI века: материалы VIII международной научно-методической конференции. (Минск, 5–6 декабря 2013 года). – Минск: БГУИР, 2013. – С. 186 - 187. (https://libeldoc.bsuir.by/handle/123456789/26123)}
		\scnitem{Мошенко, С. Г. Семантически структурированные сайты, ориентированные на поддержку подготовки и проведения научно-учебных мероприятий. // Дистанционное обучение - образовательная среда ХХІ века : материалы VII Международной научно-методической конференции. - Минск: БГУИР, 2011. - С. 264-266. (https://libeldoc.bsuir.by/handle/123456789/26944)}
		\scnitem{Заливако, С. С. Применение семантической технологии компонентного проектирования интеллектуальных решателей задач в дистанционном обучении. // Дистанционное обучение - образовательная среда ХХІ века : материалы VII Международной научно-методической конференции. - Минск: БГУИР, 2011. - С. 247-249. (https://libeldoc.bsuir.by/handle/123456789/26941))}
		\scnitem{Колб, Д. Г. Web-ориентированная реализация семпнтических моделей интеллектуальных систем для систем дистанционного обучения. // Дистанционное обучение - образовательная среда ХХІ века : материалы VII Международной научно-методической конференции. - Минск: БГУИР, 2011. - С. 258-260. (https://libeldoc.bsuir.by/handle/123456789/26913)}
		\scnitem{Адерихо, И. А. Модель структуры описания библиографического источника, представленная на языке семантических сетей / И. А. Адерихо, Н. В. Граков, И. А. Черников // Дистанционное обучение – образовательная среда XXI века : материалы IX международной научно-методической конференции (Минск, 3-4 декабря 2015 года). – Минск : БГУИР, 2015. – C. 148 – 149. (https://libeldoc.bsuir.by/handle/123456789/5539)}
		\scnitem{Гракова, Н. В. Модель структуры описания курсового и дипломного проектирования, представленная на языке семантических сетей / Н. В. Гракова, Д. И. Коновал, В. С. Семенов // Дистанционное обучение – образовательная среда XXI века : материалы IX международной научно-методической конференции (Минск, 3-4 декабря 2015 года). – Минск : БГУИР, 2015. – C. 155 – 156. (https://libeldoc.bsuir.by/handle/123456789/5540)}
		\scnitem{Давыденко, И. Т. Семантическая структуризация базы знаний интеллектуальной справочной системы по геометрии / И. Т. Давыденко, Е. А. Дюбина // Дистанционное обучение – образовательная среда XXI века : материалы IX международной научно-методической конференции (Минск, 3-4 декабря 2015 года). – Минск : БГУИР, 2015. – C. 153 – 154. (https://libeldoc.bsuir.by/handle/123456789/5544)}
		\scnitem{Таранчук, В. Б. Методические и технические решения, примеры создания интеллектуальных образовательных ресурсов / Таранчук В. Б. // Дистанционное обучение – образовательная среда XXI века : материалы XI Международной научно-методической конференции, Минск, 12-13 декабря 2019 г. / редкол.: В. А. Прытков [и др.]. – Минск : БГУИР, 2019. – C. 312–313. (https://libeldoc.bsuir.by/handle/123456789/38468)}
		\scnitem{Ли, Вэньцзу. Онтологическая модель генерации вопросов для интеллектуальных обучающих систем / Ли Вэньцзу, Цянь Лунвэй // Дистанционное обучение – образовательная среда XXI века : материалы XI Международной научно-методической конференции, Минск, 12-13 декабря 2019 г. / редкол.: В. А. Прытков [и др.]. – Минск : БГУИР, 2019. – C. 184–185. (https://libeldoc.bsuir.by/handle/123456789/38287)}
		\scnitem{}
		\scnitem{}
	\end{scnrelfromlist}
	
\end{SCn}

Создание интеллектуальной творческой среды, обеспечивающей формирование интереса и мотивации
\begin{textitemize}
	\item на уровне индивидуальной деятельности обучаемого;
	\item на уровне коллективных форм обучения;
	\item на уровне общей организации учебного процесса
	\begin{textitemize}
		\item в школе,
		\item на кафедре,
		\item в вузе.
	\end{textitemize}
\end{textitemize}

Проблемы образовательной деятельности:
\begin{textitemize}
	\item Отсутствует семантическая совместимость между разными учебными дисциплинами и разными специальностями (мозаика) (отсутствует фиксация текущей и постоянно эволюционирующий интегрированной картины мира). Это существенно усложняет миграцию между разными специальностями и восприятия комплекса учебных дисциплин.
	\item Отсутствует автоматизация индивидуального подхода к обучению и к профессиональной ориентации. Целью должно быть максимально возможное раскрытие творческого потенциала \myuline{каждого} человека.
	\item Отсутствует глубокая конвергенция между деятельностью по подготовке кадров и той деятельностью, для которой эти кадры готовятся. Образование должно происходить путём поэтапного погружения обучаемых в \myuline{реальную} профессиональную деятельность. Для этого интеллектуальные обучающие системы должны быть \myuline{частью} комплекса семантически совместимых интеллектуальных компьютерных систем, обеспечивающего автоматизацию соответствующего вида и области человеческой деятельности.
\end{textitemize}

%что конкретно необходимо описать
Семейство семантически совместимых интеллектуальных обучающих систем, обеспечивающих комплексную подготовку специалистов в области \textit{Искусственного интеллекта} (в частности, по специальности <<Искусственный интеллект>>) и интегрированных в комплекс интеллектуальных компьютерных систем нового поколения, обеспечивающих \myuline{комплексную} автоматизацию \myuline{всех} видов и направлений человеческой деятельности в области \textit{Искусственного интеллекта} (общая постановка данной задачи рассмотрена в \ref{sec_activity_ai}~\nameref{sec_activity_ai}).
%система ostis-сообществ, обеспечивающих образовательную деятельность в области ИИ в рамках Экосистемы OSTIS

%что конкретно необходимо описать
Семантическая спецификация (модель) обучаемого, обеспечивающая персонификацию обучения (адаптацию) к обучаемому
%конкретные онтологии

%что конкретно необходимо описать
Семантические модели различных педагогических методик

%что конкретно необходимо описать
Семантическая модель общего управления процессом обучения (в том числе \myuline{индивидуального})
\begin{textitemize}
	\item лекции, семинары, индивидуальные упражнения;
	\item ненавязчивое тестирование
\end{textitemize}

%что конкретно необходимо описать
Система ostis-сообществ, обеспечивающих образовательную деятельность во всех областях (не только в области \textit{Искусственного интеллекта})
\begin{textitemize}
	\item образовательная деятельность по техническим специальностям интеллектуальным компьютерным ostis-системам поддержки проектирования и всего жизненного цикла различных искусственных систем
	\begin{textitemize}
		\item строительство,
		\item машиностроение,
		\item пищевое производство;
	\end{textitemize}
	\item медицина;
	\item физика;
	\item математика;
	\item языкознания;
	\item история.
\end{textitemize}

\begin{SCn}
	\scnheader{интеллектуальные обучающие системы нового поколения}
	\scnidtf{интеллектуальная обучающая система нового поколения}
	\scnsuperset{обучающая ostis-система}
	\begin{scnindent}
		\scnsuperset{встроенная обучающая ostis-система}
		\begin{scnindent}
			\scnidtf{встроенная ostis-система, осуществляющая перманентное обучение своих пользователей эффективному взаимодействию с собой}
		\end{scnindent} 
	\end{scnindent} 
	\scnidtf{интероперабельная интеллектуальная обучающая система, семантически совместимая и взаимодействующая с другими интеллектуальными обучающими системам --- формирующая вместе с ними систематизированную комплексную картину мира}
\end{SCn}