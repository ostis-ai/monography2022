\newpage
\section{Представление дидактической информации в базах знаний ostis-систем}
\label{sec_didactic information}

\begin{SCn}
	
	\bigskip
	
	\begin{scnrelfromlist}{ключевое понятие}
		\scnitem{...}
	\end{scnrelfromlist}
	
	\bigskip
	
	\begin{scnrelfromlist}{ключевое знание}
		\scnitem{...}
	\end{scnrelfromlist}
	
	\bigskip
	
	\begin{scnrelfromlist}{библиографическая ссылка}
		\scnitem{\scncite{Golenkov2001b}}
		\scnitem{\scncite{Grakova2015}}
		\scnitem{\scncite{Davydenko2013b}}
		\scnitem{\scncite{Davydenko2015}}
		\scnitem{\scncite{Zalivako2011}}
		\scnitem{\scncite{Kolb2011}}
		\scnitem{\scncite{Kolb2013}}
		\scnitem{\scncite{Li2019}}
		\scnitem{\scncite{Rusetski2013}}
		\scnitem{\scncite{Taranchuk2019}}
		\scnitem{\scncite{Shunkevich2013b}}
	\end{scnrelfromlist}

	\bigskip
	
	\begin{scnrelfromlist}{подраздел}
		\scnitem{\ref{subsec_means_specification_described_objects}~\nameref{subsec_means_specification_described_objects}}
		\scnitem{\ref{subsec_means_logical-semantic_systematization_used_information_structures}~\nameref{subsec_means_logical-semantic_systematization_used_information_structures}}
		\scnitem{\ref{subsec_means_indicating_similarities_differences_specimens}~\nameref{subsec_means_indicating_similarities_differences_specimens}}
		
		\scnitem{\ref{subsec_rules_constructing_entities_various_classes}~\nameref{subsec_rules_constructing_entities_various_classes}}
		\scnitem{\ref{subsec_methodological_specification_trends_evolution_knowledge}~\nameref{subsec_methodological_specification_trends_evolution_knowledge}}
		\scnitem{\ref{subsec_means_presentation_educational_tasks_educational_methodical_structuring_kb}~\nameref{subsec_means_presentation_educational_tasks_educational_methodical_structuring_kb}}
	\end{scnrelfromlist}
\end{SCn}

\section*{Введение в \ref{sec_didactic information}}
Важнейшим критерием качества создаваемых \textit{ostis-систем} любого назначения является создание условий для того, чтобы недостаточно квалифицированные пользователи каждой \textit{ostis-системы} (как конечные пользователи, так и ответственные за её эффективную эксплуатацию и модернизацию) могли достаточно быстро \myuline{с помощью этой же} \myuline{\textit{ostis-системы}} приобрести требуемую квалификацию. Это означает, что каждая \textit{ostis-система} независимо от её прямого назначения (автоматизации конкретных видов человеческой деятельности в конкретной области) должна быть также обучающей системой, то есть должна уметь обучать своих пользователей в направлении повышения их квалификации. Квалифицированный пользователь любой категории должен понимать возможности \textit{ostis-системы}, с которой он взаимодействует, должен понимать то, что знает и что умеет система, а также то, как можно управлять её деятельностью. Отсутствие взаимопонимания между \textit{ostis-системами} и их пользователями --- это нарушение требования \textit{интероперабельности}, предъявляемого как к \textit{ostis-системам}, так и к их пользователям.

К сожалению, современные традиции представления различного вида документации, стандартов, отчетов, научно-технических статей и монографий не только не ориентированы на их адекватное понимание \textit{интеллектуальными компьютерными системами}, но также и не способствуют быстрому их пониманию со стороны тех людей, которым эти тексты адресованы. Последнее обстоятельство требует разработки (написания) учебников и учебных пособий, \myuline{специально} предназначенных для тех людей, которые \myuline{начинают} усваивать соответствующую область знаний, которые ещё не приобрели необходимую им квалификацию в этой области. Но очевидно, что это предполагает существенное дублирование представляемой информации.

В идеале \textit{база знаний} каждой \textit{ostis-системы} должна содержать достаточно полную собственную документацию (руководство конечного пользователя, руководство по эксплуатации, руководство по модернизации). Кроме этого, \textit{ostis-система} должна уметь:
\begin{textitemize}
	\item отвечать на самые разные (в том числе, глупые) вопросы о себе;
	\item анализировать действия своих пользователей и давать им полезные советы, помогающие им быстрее приобрести требуемую квалификацию.
\end{textitemize}

Для того, чтобы \textit{база знаний ostis-системы} содержала всю информацию, помогающую ей выполнить свою обучающую функцию по отношению к своим пользователям и, соответственно, помогающую пользователям \myuline{быстрее} приобрести необходимую пользовательскую квалификацию, \textit{база знаний} должна содержать дополнительную информацию дидактического и, в частности, учебно-методического характера. Рассмотрим подробнее, что можно считать такой дидактической информацией и каковы языковые средства её представления в \textit{базе знаний ostis-системы}.

\begin{SCn}
	\scnheader{дидактическая информация}
	\scnidtf{информация, хранимая в \textit{базе знания ostis-системы}, предназначенная для её пользователей и помогающая им быстрее и адекватнее усвоить денотационную семантику знаний, хранимых в памяти этой системы}
	\scnidtf{\textit{знание ostis-системы}, предназначенное для её пользователей и помогающее им быстрее и адекватнее усвоить денотационную семантику знаний хранимых в памяти этой системы}
	\scnidtf{\textit{знание}, входящее в состав \textit{базы знаний} и помогающее пользователю быстрее и качественнее (адекватнее) понять смысл (\textit{денотационную семантику}) \textit{базы знаний}}
	\scnidtf{информация, способствующая более глубокому пониманию и усвоению смысла различного рода сущностей (в том числе, и различных знаний)}
	\scnidtfexp{информация, которая при её представлении в памяти \textit{ostis-систем} способствует установлению взаимопонимания между ostis-системами и их пользователями, которая ускоряет процесс формирования у пользователей требуемой квалификации в различных областях}
	\scntext{примечание}{"дидактический"{} эффект дидактической информации обеспечивается:
		\begin{textitemize}
			\item достаточной детализацией изучаемой сущности (полнотой семантической окрестности, описывающей связи этой сущности с другими сущностями)
			\begin{textitemize}
				\item декомпозицией рассматриваемых сущностей;
				\item указанием аналогов (сходных сущностей в различных смыслах);
				\item указанием метафор (эпиграфов);
				\item указанием антиподов (сущностей, отличающихся в различных смыслах);
			\end{textitemize}
			\item упражнениями --- решением различных задач с использованием изучаемых сущностей;
			\item ссылками на знания, хранимые в рамках той же \textit{базы знаний};
			\item ссылками на библиографические источники.
		\end{textitemize}}
		\scnnote{Типология описываемых (рассматриваемых, исследуемых, изучаемых) сущностей (объектов) ничем не ограничивается и включает в себя как конкретные материальные и абстрактные объекты и процессы, конкретные связи и структуры, конкретные информационные конструкции и, в частности, различного вида знания, так и различные классы указанных объектов}
\end{SCn}

\subsection{Средства спецификации описываемых объектов}
\label{subsec_means_specification_described_objects}
\begin{scnrelfromlist}{библиографическая ссылка}
	\scnitem{\textit{\ref{sec_sem_neighborhood}~\nameref{sec_sem_neighborhood}}}
\end{scnrelfromlist}

Для представления дидактической информации данного вида используются рассматриваемые ниже понятия:

\begin{SCn}
	\scnheader{спецификация}
	\scnidtftext{часто используемый sc-идентификатор}{\textit{семантическая окрестность}}
	\scnsuperset{однозначная спецификация}
	\begin{scnindent}
		\scnidtf{спецификация, которая \myuline{однозначно} соответствует специфицируемой сущности}
		\scnsuperset{высказывание необходимости и достаточности}
		\begin{scnindent}
			\scnidtf{высказывание о необходимых и достаточных условиях принадлежности сущностей соответствующему специфицируемому \textit{множеству}}
		\end{scnindent}
		\scnsuperset{определение}
		\begin{scnindent}
			\scnnote{Большинство определений являются высказываниями о необходимости и достаточности}
			\scnsuperset{формальное определение}
			\begin{scnindent}
				\scnidtf{определение, представленное на формальном языке (в частности, в SC-коде)}
			\end{scnindent}
			\scnsuperset{строгое естественно-языковое определение}
			\begin{scnindent}
				\scnidtf{естественно-языковое определение, которое достаточно легко и адекватно перевести на формальный язык (в том числе, на SC-код)}
			\end{scnindent}
			\scnsuperset{нестрогое естественно-языковое определение}
		\end{scnindent} 
	\end{scnindent}
	\scnsuperset{пояснение}
	\begin{scnindent}
		\scnidtf{нестрогая спецификация, которая например, используется для спецификаций \myuline{неопределяемых} понятий и обычно представляется на естественном языке}
	\end{scnindent}
	\scnsuperset{представление множества}
	\begin{scnindent}
		\scnidtf{перечисление всех или многих (желательно разнообразных) элементов специфицируемого множества}
	\end{scnindent}
	\scnsuperset{представление иерархии множеств}
	\scnsuperset{представление разбиения}
	\scnsuperset{представление иерархии разбиений}
	\begin{scnindent}
		\scnidtf{представление \myuline{иерархического} разбиения специфицируемого множества по различным признакам разбиения}
		\scnidtf{текст разбиения множества}
		\scnsuperset{классификация}
		\begin{scnindent}
			\scnidtf{представление иерархической классификации специфицируемого класса по различным признакам}
			\scnidtf{текст классификации}
		\end{scnindent}
	\end{scnindent}
	\scnsuperset{представление декомпозиции}
	\begin{scnindent}
		\scnsuperset{представление пространственной декомпозиции}
		\scnsuperset{представление темпоральной декомпозиции}
	\end{scnindent} 
	\scnsuperset{представление иерархии декомпозиции}
	\begin{scnindent}
		\scnidtf{представление иерархической декомпозиции специфицируемого объекта на компоненты (части) по различным критериям}
		\scnidtf{иерархическое описание структуры специфицируемого объекта}
	\end{scnindent}
	\scnsuperset{представление логической формулы}
	\begin{scnindent}
		\scnidtf{полное представление логической формулы до уровня входящих в неё атомарных логических формул}
	\end{scnindent}
	\scnsuperset{логическая структура понятия}
	\scnsuperset{иерархия логической структуры понятия}
	\begin{scnindent}
		\scnidtf{логическая структура специфицируемого понятия, соответствующая его определению и иерархии определений \myuline{всех} используемых понятий (вплоть до неопределяемых понятий)}
	\end{scnindent}
	\scnsuperset{иерархическая структура доказательства}
	\begin{scnindent}
		\scnidtf{иерархическая структура доказательства специфицируемого высказывания (теоремы), в каждом шаге которого указывается (1) используемое правило вывода, (2) высказывания, на основе которых осуществляется шаг логического вывода, (3) высказывание, являющееся результатом (следствием) этого шага вывода}
	\end{scnindent}
\end{SCn}

Перечисленные виды спецификаций (1) широко используются в "дидактических"{} целях, (2) соответствуют различным классам специфицируемых сущностей (любым сущностям, множествам, понятиям, высказываниям, сущностям, которые множествами не являются и другим классам), (3) предполагают использование целого ряда других понятий --- прежде всего, отношений, связывающих спецификации со специфицируемыми сущностями. К числу таких дополнительно используемых понятий относятся следующие понятия.

\begin{SCn}
	\scnheader{однозначная спецификация*}
	\scnidtf{быть однозначной спецификацией заданной сущности*}
\end{SCn}

\begin{SCn}
	\scnheader{высказывание о необходимости и достаточности*}
	\scnidtf{быть высказыванием о необходимости и достаточности принадлежности заданному множеству*}
\end{SCn}

\begin{SCn}
	\scnheader{определение*}
	\scnidtf{быть определением заданного класса*}
	\scnidtf{быть определением заданного понятия*}
\end{SCn}

\begin{SCn}
	\scnheader{пояснение*}
	\scnidtf{быть пояснением заданной сущности*}
\end{SCn}

\begin{SCn}
	\scnheader{принадлежность*}
	\scnidtf{быть элементом заданного множества*}
	\scnidtf{элемент*}
\end{SCn}

\begin{SCn}
	\scnheader{подмножество*}
	\scnidtf{\textit{включение*}}
\end{SCn}

\begin{SCn}
	\scnheader{разбиение*}
	\scnidtf{разбиение множества*}
\end{SCn}

\begin{SCn}
	\scnheader{покрытие*}
	\scnidtf{покрытие множества*}
\end{SCn}

\begin{SCn}
	\scnheader{представление множества*}
\end{SCn}

\begin{SCn}
	\scnheader{представление иерархии множеств*}
\end{SCn}

\begin{SCn}
	\scnheader{представление разбиения*}
\end{SCn}

\begin{SCn}
	\scnheader{представление иерархии разбиения*}
	\scnidtf{быть представлением разбиения заданного множества*}
	\scnsuperset{классификация*}
\end{SCn}

\begin{SCn}
	\scnheader{часть*}
	\scnsuperset{пространственная часть*}
	\scnsuperset{темпоральная часть*}
\end{SCn}

\begin{SCn}
	\scnheader{декомпозиция*}
	\scnsuperset{пространственная декомпозиция*}
	\scnsuperset{темпоральная декомпозиция*}
\end{SCn}

\begin{SCn}
	\scnheader{представление декомпозиции*}
	\scnidtf{быть представлением иерархической декомпозиции заданной сущности*}
\end{SCn}

\begin{SCn}
	\scnheader{представление иерархии декомпозиций*}
\end{SCn}

\begin{SCn}
	\scnheader{определение*}
	\scnidtf{быть определением заданного понятия*}
\end{SCn}

\begin{SCn}
	\scnheader{логическая структура понятия*}
	\scnidtf{быть логической структурой заданного понятия*}
	\scnidtf{быть семейством понятий, используемых в определении заданного понятия*}
\end{SCn}

\begin{SCn}
	\scnheader{иерархия логической структуры понятия*}
\end{SCn}

\begin{SCn}
	\scnheader{шаг логического вывода*}
	\scnidtf{быть семейством высказываний, из которых логически следует заданное высказывание*}
\end{SCn}

\begin{SCn}
	\scnheader{иерархическая структура доказательства*}
	\scnidtf{быть иерархической структурой доказательства заданного высказывания*}
\end{SCn}

Подчеркнём, что структуризация и систематизация как самих описываемых сущностей (объектов, не являющихся множествами., понятий различного вида, определений, высказываний, не являющихся определениями --- аксиом и теорем), так и знаний, входящих в состав баз знаний ostis-систем, является важнейшим видом дидактической информации. Ключевым видом структуризации баз знаний ostis-систем является их онтологическая стратификация путём выделения различных предметных областей и соответствующих им онтологий, специфицирующих эти предметные области. Подробнее об этом см. в \ref{sec_sd}~\nameref{sec_sd}, а также в \ref{sec_ontology}~\nameref{sec_ontology}.

Важным видом дидактической информации являются спецификации не только таких видов знаний, как высказывания и предметные области, но также и различных используемых библиографических источников и информационных ресурсов (в том числе и различных разделов этих источников и информационных ресурсов). Для представления таких спецификаций в базах знаний ostis-систем используются следующие понятия.

\begin{SCn}
	\scnheader{официальный библиографический идентификатор*}
	\scnidtf{библиографическая запись, соответствующая заданному информационному ресурсу*}
\end{SCn}

\begin{SCn}
	\scnheader{автор*}
\end{SCn}

\begin{SCn}
	\scnheader{редактор*}
\end{SCn}

\begin{SCn}
	\scnheader{подраздел*}
\end{SCn}

\begin{SCn}
	\scnheader{цитата*}
\end{SCn}

\begin{SCn}
	\scnheader{библиографическая ссылка*}
	\scnidtf{семантически близкий библиографический источник*}
\end{SCn}

\begin{SCn}
	\scnheader{рассматриваемый вопрос*}
\end{SCn}

\begin{SCn}
	\scnheader{аннотация*}
\end{SCn}

\begin{SCn}
	\scnheader{эпиграф*}
\end{SCn}

\begin{SCn}
	\scnheader{ключевой знак*}
	\scnsuperset{ключевая сущность, не являющаяся понятием*}
	\scnsuperset{ключевое понятие*}
	\begin{scnindent}
		\scnsuperset{ключевое понятие, не являющееся ни отношением, ни параметром*}
		\scnsuperset{ключевое отношение*}
		\scnsuperset{ключевой параметр*}
	\end{scnindent}
	\scnsuperset{\textbf{ключевое знание*}}
	\begin{scnindent}
		\scnidtf{основной тезис*}
		\scnsuperset{основное положение*}
		\begin{scnindent}
			\scnidtf{основной вывод (результат)*}
		\end{scnindent}
	\end{scnindent}
\end{SCn}

\subsection{Средства логико-семантической систематизации используемых информационных конструкций}
\label{subsec_means_logical-semantic_systematization_used_information_structures}

В состав данных средств входят:
\begin{textitemize}
	\item Средства явного описания синонимии и омонимии идентификаторов (имён, терминов), соответствующих рассматриваемым объектам, которые описаны в \textit{\ref{sec_identifiers}~\nameref{sec_identifiers}};
	\item Средства явного описания семантической эквивалентности, семантического включения, семантической смежности и других семантических связей между используемыми \textit{информационными конструкциями}. При этом описываемыми \textit{информационными конструкциями} могут быть как \textit{sc-конструкции}, хранимые в составе базы знаний \textit{ostis-системы}, так и различные внешние для \textit{sc-памяти} ostis-системы информационные конструкции (различные файлы, различные "неоцифрованные"{} документы);
	\item Средства систематизации информационных ресурсов, представленных в дидактически эффективных (наглядных) мультимедийных формах (графики, таблицы, диаграммы, фотографии, рисунки, 3D--изображения, аудио-записи, видео-записи лекций, семинаров, конференций, бесед, виртуальная реальность).
\end{textitemize}

\newpage
\subsection{Средства указания и описания сходств, отличий, типовых экземпляров}
\label{subsec_means_indicating_similarities_differences_specimens}

\begin{SCn}
	\begin{scnrelfromlist}{эпиграф}
		\scnfileitem{Всё познаётся в сравнении}
		\scnfileitem{Важнейшим проявлением интеллекта является умение "видеть"{} сходства в различных объектах и различия в сходных}
	\end{scnrelfromlist}
\end{SCn}

Для представления дидактической информации данного вида используются следующие понятия:

\begin{SCn}
	\scnheader{аналог*}
	\scnidtf{быть аналогичной сущностью*}
	\scnidtf{быть похожей сущностью*}
	\scnidtf{быть аналогом*}
	\scnidtf{быть аналогом заданной сущности*}
	\scnidtf{Бинарное неориентированное отношение, каждая пара которого связывает знаки двух аналогичных (в том или ином смысле) сущностей*}
	\scnidtf{пара аналогичных сущностей*}
\end{SCn}

\begin{SCn}
	\scnheader{аналоги*}
	\scnidtf{множество аналогичных сущностей*}
	\scnsuperset{аналог*}
	\begin{scnindent}
		\scnidtf{быть аналогом заданной сущности*}
	\end{scnindent}
\end{SCn}

\begin{SCn}
	\scnheader{близкий аналог*}
	\scnidtf{сущность-близнец*}
	\scnidtf{быть очень похожей сущностью*}
\end{SCn}

\begin{SCn}
	\scnheader{следует отличать*}
	\scnidtf{быть семейством похожих, но отличающихся сущностей, которые не следует путать*}
	\scnsubset{аналоги*}
\end{SCn}

\begin{SCn}
	\scnheader{аналогичность*}
	\scniselement{параметр}
	\scnnote{степень аналогичности может быть разной}
	\scnidtf{степень аналогичности*}
	\scnidtf{Параметр, заданный на множестве пар Отношения "быть аналогом"*}
	\begin{scnrelbothlist}{следует отличать}
		\scnitemcustom{\textit{сходство}*}
		\begin{scnindent}
			\scnidtf{описание того, в чём конкретно заключается аналогичность (сходства)}
		\end{scnindent} 
		\scnitemcustom{\textit{следует отличать}*}
		\begin{scnindent}
			\scnidtf{указание факта наличие отличий между перечисленными сущностями*}
		\end{scnindent} 
	\end{scnrelbothlist}
\end{SCn}

\begin{SCn}
	\scnheader{антипод*}
	\scnidtf{пара принципиально отличающихся сущностей*}
\end{SCn}

\begin{SCn}
	\scnheader{сходство*}
	\scnidtf{сходство сущностей, принадлежащих заданному семейству*}
	\scnidtf{уточнение того, в чём заключается аналогичность заданного семейства сущностей*}
	\scnidtf{аналогия*}
\end{SCn}

\begin{SCn}
	\scnheader{отличие*}
	\scnidtf{описание отличий двух заданных сущностей*}
\end{SCn}

\begin{SCn}
	\scnheader{сравнение*}
	\scnidtf{текст, описывающий сходства и отличия сущностей, принадлежащих заданному семейству}
\end{SCn}

\begin{SCn}
	\scnheader{сравнительный анализ*}
	\scnidtf{сравнительный анализ заданной сущности*}
\end{SCn}

\begin{SCn}
	\scnheader{пример\scnrolesign}
	\scnsubset{(включение* $\bigcup$ принадлежность*)}
\end{SCn}

\begin{SCn}
	\scnheader{типичный экземпляр\scnrolesign}
	\scnidtf{экземпляр (элемент) заданного класса, аналогичный большинству экземпляров этого класса \scnrolesign}
	\scnidtf{типичный представитель заданного класса\scnrolesign}
	\scnidtf{типичный пример\scnrolesign}
	\scnsubset{пример\scnrolesign}
\end{SCn}

\begin{SCn}
	\scnheader{примечание*}
	\scnidtf{неформальное описание отличительной особенности заданной сущности*}
\end{SCn}

В качестве примера приведём указание факта аналогичности групп понятий, используемых для представления дидактической информации вида структуризации и систематизации описываемых объектов (см. \textit{\ref{subsec_means_detecting_described_objects}~\nameref{subsec_means_detecting_described_objects}}).

В качестве примера приведём указание факта аналогичности групп понятий, используемых для представления дидактической информации вида структуризации и систематизации описываемых объектов (см. \textit{\ref{subsec_means_detecting_described_objects}~\nameref{subsec_means_detecting_described_objects}}).

\begin{scnhaselementset}
	\scnitem{
		\begin{scnvector}
			\scnitem{представление множества*}
			\scnitem{представление множества}
			\scnitem{принадлежность\scnrolesign}
			\scnitem{множество}		
	\end{scnvector}}
	\scnitem{
		\begin{scnvector}
			\scnitem{представление разбиения*}
			\scnitem{представление разбиения}
			\scnitem{разбиение*}
			\scnitem{множество}		
	\end{scnvector}}
	\scnitem{
		\begin{scnvector}
			\scnitem{классификация*}
			\scnitem{классификация}
			\begin{scnindent}
				\scnidtf{представление классификации}
				\scnidtf{текст классификации}
			\end{scnindent} 
			\scnitem{разбиение*}
			\scnitem{класс}
			\begin{scnindent}
				\scnsubset{множество}
			\end{scnindent} 
	\end{scnvector}}
	\scnitem{
		\begin{scnvector}
			\scnitem{представление декомпозиции*}
			\scnitem{представление декомпозиции}
			\scnitem{декомпозиция*}
			\scnitem{сущность}		
	\end{scnvector}}
	\scnitem{
		\begin{scnvector}
			\scnitem{представление логической формулы*}
			\scnitem{представление логической формулы}
			\scnitem{принадлежность\scnrolesign}
			\scnitem{логическая формула}
			\begin{scnindent}
				\scnsubset{множество}
			\end{scnindent} 
	\end{scnvector}}
	\scnitem{
		\begin{scnvector}
			\scnitem{иерархическая структура определения*}
			\scnitem{иерархическая структура определения}
			\scnitem{определяющее понятие*}
			\begin{scnindent}
				\scnidtf{понятие, используемое в определении заданного понятия*}
			\end{scnindent} 
			\scnitem{определение}
	\end{scnvector}}
	\scnitem{
		\begin{scnvector}
			\scnitem{иерархическая структура доказательства*}
			\scnitem{иерархическая структура доказательства}
			\scnitem{посылка шага вывода*}
			\scnitem{шаг вывода}		
	\end{scnvector}}
\end{scnhaselementset}

\subsection{Правила построения сущностей различных классов}
\label{subsec_rules_constructing_entities_various_classes}
Для представления дидактической информации данного вида используются следующие понятия:

\begin{SCn}
	\scnheader{правила построения*}
	\scnidtf{принципы, лежащие в основе заданного продукта некоторой деятельности или продуктов, принадлежащих заданному классу*}
	\scnidtf{требования, предъявляемые к заданному продукту или к продуктам заданного класса*}
	\scnidtf{свойства и характеристики правильно построенного (хорошо построенного) заданного продукта или правильно построенных продуктов заданного класса*}
	\scnidtf{быть правилом построения сущностей заданного класса*}
	\scnsuperset{правила оформления*}
	\begin{scnindent}
		\scnidtf{правила придания окончательной формы (вида), правила "упаковки"{} заданного создаваемого продукта или создаваемых продуктов заданного класса*}
	\end{scnindent}
	\scnexplanation{С формальной точки зрения правила построения сущностей заданного класса --- это ядро (аксиоматическая система) логической онтологии, которая специфицирует \textit{предметную область}, классом объектов исследования которой является указанный выше (заданный) класс сущностей}
	\scntext{примечание}
	{О правилах построения можно говорить по отношению к таким классам сущностей, как:
		\begin{textitemize}
			\item различного вида \textit{sc-знания} (см. \textit{Предметная область и онтология знаний и баз знаний});
			\item различного вида \textit{файлы ostis-систем}, например, \textit{ея-файлы} ostis-систем (см. \textit{Предметная область и онтология файлов, внешних информационных конструкций и внешних языков ostis-систем});
			\item \textit{sc-идентификаторы} различных классов \textit{sc-элементов}.
		\end{textitemize}
	}
	\scnsuperset{правила идентификации*}
	\begin{scnindent}
		\scnidtf{правила построения \textit{sc-идентификаторов} для заданного класса \textit{sc-элементов}}
		\scnrelto{ключевое отношение}{\textit{\ref{sec_identifiers}~\nameref{sec_identifiers}}}
	\end{scnindent}
	\scnsuperset{правила спецификации*}
	\begin{scnindent}
		\scnidtf{правила построения \textit{sc-спецификаций} для заданного класса \textit{sc-элементов}}
		\scnrelto{ключевое отношение}{\textit{\ref{sec_sem_neighborhood}~\nameref{sec_sem_neighborhood}}}
		\scnnote{Специфицируемыми \textit{sc-элементами} и, соответственно, обозначаемыми ими сущностями могут быть \textit{персоны}, \textit{библиографические источники}, разделы и сегменты \textit{баз знаний}, \textit{файлы} ostis-систем и многое другое}
	\end{scnindent}
\end{SCn}

\newpage
\subsection{Методологическая спецификация тенденций эволюции знаний}
\label{subsec_methodological_specification_trends_evolution_knowledge}
Для представления дидактической информации данного вида используются следующие понятия:

\begin{SCn}
	\scnheader{результат анализа*}
	\scnidtf{результат анализа заданного объекта или класса объектов*}
	\scnsuperset{оценка качества*}
	\begin{scnindent}
		\scnidtf{анализ качества заданного объекта}
	\end{scnindent}
\end{SCn}

\begin{SCn}
	\scnheader{принципы, лежащие в основе*}
	\scnidtf{принципы, лежащие в основе заданной сущности или всех сущностей, принадлежащих заданному классу*}
	\scnidtf{основные свойства и характеристики заданной сущности или сущностей заданного класса*}
	\scnidtf{основные положения (свойства, закономерности), присущие заданной (описываемой) сущности*}
	\scnidtf{принципы, лежащие в основе заданной сущности (объекта системы, информационной конструкции) или сущностей заданного класса*}
	\scnidtf{ключевые особенности*}
\end{SCn}

\begin{SCn}
	\scnheader{достоинства*}
	\scnidtf{преимущества*}
\end{SCn}

\begin{SCn}
	\scnheader{недостатки*}
\end{SCn}

\begin{SCn}
	\scnheader{назначение*}
	\scnidtf{предъявляемые требования*}
	\scnidtf{требования, которым должны удовлетворять сущности заданного класса}
	\scnidtf{требования, предъявляемые к заданным сущностям*}
\end{SCn}

\begin{SCn}
	\scnheader{проблемы*}
	\scnidtf{проблемы заданного объекта или класса объектов*}
	\scnidtf{в чём заключается проблема*}
	\scnidtf{проблема, ассоциируемая с заданной сущностью*}
\end{SCn}

\begin{SCn}
	\scnheader{актуальные задачи*}
	\scnidtf{актуальные задачи совершенствования заданного объекта или класса объектов*}
\end{SCn}

\begin{SCn}
	\scnheader{известные варианты решения заданных проблем*}
\end{SCn}

\begin{SCn}
	\scnheader{предлагаемый подход к решению заданных проблем*}
	\scnidtf{принципы, лежащие в основе предлагаемого подхода к решению заданных проблем*}
\end{SCn}

\begin{SCn}
	\scnheader{новизна предлагаемого подхода*}
	\scnidtfexp{Бинарное ориентированное отношение, каждая пара которого связывает некоторую сущность \myuline{с описанием} того, чем она \myuline{принципиально} отличается от предшествующих ей аналогов. Такой сущностью может быть новая техническая система, новый метод решения некоторого класса задач и другое}
	\scnsuperset{научная новизна предлагаемого подхода*}
	\begin{scnindent}
		\scnidtf{"изюминка"{} предполагаемой системы, метода, принципов, подхода к решению}
	\end{scnindent} 
\end{SCn}

\begin{SCn}
	\scnheader{реализация предлагаемого подхода*}
\end{SCn}

\begin{SCn}
	\scnheader{оценка качества реализации предлагаемого подхода*}
\end{SCn}

\begin{SCn}
	\scnheader{направления дальнейшего развития*}
	\scnidtf{направления дальнейшего развития заданного объекта*}
	\begin{scnsubdividing}
		\scnitem{перманентные направления развития*}
		\scnitem{тактические направления развития текущего состояния}
		\begin{scnindent}
			\scnidtf{план ближайших работ по развитию данного объекта*}
		\end{scnindent} 
		\scnitem{стратегические направления развития текущего состояния*}
		\begin{scnindent}
			\scnidtf{перспективный план развития данного объекта*}
		\end{scnindent} 
	\end{scnsubdividing}
\end{SCn}

\subsection{Средства представления учебных заданий и учебно-методической структуризации баз знаний ostis-систем}
\label{subsec_means_presentation_educational_tasks_educational_methodical_structuring_kb}
Для представления дидактической информации данного вида используются следующие понятия:

\begin{SCn}
	\scnheader{учебный вопрос}
	\scnidtf{вопрос для самопроверки или проверки качества усвоения знаний}
	\scnsuperset{аттестационный вопрос}
\end{SCn}

\begin{SCn}
	\scnheader{учебный вопрос*}
	\scnidtf{учебный вопрос для заданного раздела (контекста) учебного материала*}
	\scnsuperset{аттестационный вопрос*}
\end{SCn}

\begin{SCn}
	\scnheader{учебная задача}
	\scnidtf{упражнение}
	\begin{scnsubdividing}
		\scnitem{учебная задача не требующая дополнительных средств}
		\scnitem{учебная лабораторная работа}
	\end{scnsubdividing}
	\scnsuperset{аттестационная задача}
\end{SCn}

\begin{SCn}
	\scnheader{учебная задача*}
	\scnidtf{бинарно ориентированное отношение, любая пара которого связывает предметную область и онтологию с соответствующими упражнениями}
	\scnsuperset{аттестационная задача*}
	\scnnote{Необходимо разбиение множества упражнений по уровню сложности, по тематике, по используемым методам, а также задание порядка упражнений.}
\end{SCn}

\begin{SCn}
	\scnheader{аттестационный вопрос}
	\scnidtf{тестовый вопрос контроля знаний}
	\scnidtf{вопрос, который рекомендуется задавать при проверке усвоения знаний}
\end{SCn}

\begin{SCn}
	\scnheader{аттестационный вопрос*}
	\scnidtf{быть аттестационным вопросом для заданного контекста (например, для заданной предметной области и онтологии)}
\end{SCn}

\begin{SCn}
	\scnheader{аттестационная задача}
\end{SCn}

\begin{SCn}
	\scnheader{аттестационная задача*}
\end{SCn}

Учебно-методическая структуризация \textit{баз знаний ostis-систем} требует не только установления соответствия учебных вопросов и учебных задач с соответствующими разделами баз знаний, но также и включения в состав баз знаний рекомендаций о последовательности изучения разделов учебного материала и о последовательности решения учебных задач.

%\section*{Заключение \ref{sec_didactic information}}