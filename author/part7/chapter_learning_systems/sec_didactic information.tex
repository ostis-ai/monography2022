\section{Представление дидактической информации в базах знаний ostis-систем}
\label{sec_didactic information}

\begin{SCn}
	
	\bigskip
	
	\begin{scnrelfromlist}{ключевое понятие}
		\scnitem{...}
	\end{scnrelfromlist}
	
	\bigskip
	
	\begin{scnrelfromlist}{ключевое знание}
		\scnitem{...}
	\end{scnrelfromlist}
	
	\bigskip
	
	\begin{scnrelfromlist}{библиографическая ссылка}
		\scnitem{\scncite{Golenkov2001b}}
		\scnitem{Давыденко, И. Т. Разработка интеллектуальных обучающих систем на основе технологии OSTIS / И. Т. Давыденко // Дистанционное обучение – образовательная среда XXI века: материалы VIII международной научно-методической конференции. (Минск, 5–6 декабря 2013 года). – Минск: БГУИР, 2013. – С. 196 - 197. (https://libeldoc.bsuir.by/handle/123456789/26175)}
		\scnitem{Колб, Д. Г. Проблемы обеспечения мультимедийного поиска на основе семантических сетей в системах дистанционного обучения / Д. Г. Колб // Дистанционное обучение – образовательная среда XXI века: материалы VIII международной научно-методической конференции. (Минск, 5–6 декабря 2013 года). – Минск: БГУИР, 2013. – С. 188 - 189. (https://libeldoc.bsuir.by/handle/123456789/26172)}
		\scnitem{Гракова, Н. В. Классификация участников системы управления пректированием интеллектуальных систем / Н. В. Гракова, И. И. Жуков // Дистанционное обучение – образовательная среда XXI века: материалы VIII международной научно-методической конференции. (Минск, 5–6 декабря 2013 года). – Минск: БГУИР, 2013. – С. 203 - 204. (https://libeldoc.bsuir.by/handle/123456789/26136)}
		\scnitem{Русецкий, К. В. Лингвистическая база знаний интеллектуальной обучающей системы по естественному языку / К. В. Русецкий // Дистанционное обучение – образовательная среда XXI века: материалы VIII международной научно-методической конференции. (Минск, 5–6 декабря 2013 года). – Минск: БГУИР, 2013. – С. 198 - 199. (https://libeldoc.bsuir.by/handle/123456789/26134)}
		\scnitem{Шункевич, Д. В. Базовые понятия технологии компонентного проектирования машин обработки знаний систем дистанционного  обучения Дистанционное обучение – образовательная среда XXI века: материалы VIII международной научно-методической конференции. (Минск, 5–6 декабря 2013 года). – Минск: БГУИР, 2013. – С. 186 - 187. (https://libeldoc.bsuir.by/handle/123456789/26123)}
		\scnitem{Мошенко, С. Г. Семантически структурированные сайты, ориентированные на поддержку подготовки и проведения научно-учебных мероприятий. // Дистанционное обучение - образовательная среда ХХІ века : материалы VII Международной научно-методической конференции. - Минск: БГУИР, 2011. - С. 264-266. (https://libeldoc.bsuir.by/handle/123456789/26944)}
		\scnitem{Заливако, С. С. Применение семантической технологии компонентного проектирования интеллектуальных решателей задач в дистанционном обучении. // Дистанционное обучение - образовательная среда ХХІ века : материалы VII Международной научно-методической конференции. - Минск: БГУИР, 2011. - С. 247-249. (https://libeldoc.bsuir.by/handle/123456789/26941))}
		\scnitem{Колб, Д. Г. Web-ориентированная реализация семпнтических моделей интеллектуальных систем для систем дистанционного обучения. // Дистанционное обучение - образовательная среда ХХІ века : материалы VII Международной научно-методической конференции. - Минск: БГУИР, 2011. - С. 258-260. (https://libeldoc.bsuir.by/handle/123456789/26913)}
		\scnitem{Адерихо, И. А. Модель структуры описания библиографического источника, представленная на языке семантических сетей / И. А. Адерихо, Н. В. Граков, И. А. Черников // Дистанционное обучение – образовательная среда XXI века : материалы IX международной научно-методической конференции (Минск, 3-4 декабря 2015 года). – Минск : БГУИР, 2015. – C. 148 – 149. (https://libeldoc.bsuir.by/handle/123456789/5539)}
		\scnitem{Гракова, Н. В. Модель структуры описания курсового и дипломного проектирования, представленная на языке семантических сетей / Н. В. Гракова, Д. И. Коновал, В. С. Семенов // Дистанционное обучение – образовательная среда XXI века : материалы IX международной научно-методической конференции (Минск, 3-4 декабря 2015 года). – Минск : БГУИР, 2015. – C. 155 – 156. (https://libeldoc.bsuir.by/handle/123456789/5540)}
		\scnitem{Давыденко, И. Т. Семантическая структуризация базы знаний интеллектуальной справочной системы по геометрии / И. Т. Давыденко, Е. А. Дюбина // Дистанционное обучение – образовательная среда XXI века : материалы IX международной научно-методической конференции (Минск, 3-4 декабря 2015 года). – Минск : БГУИР, 2015. – C. 153 – 154. (https://libeldoc.bsuir.by/handle/123456789/5544)}
		\scnitem{Таранчук, В. Б. Методические и технические решения, примеры создания интеллектуальных образовательных ресурсов / Таранчук В. Б. // Дистанционное обучение – образовательная среда XXI века : материалы XI Международной научно-методической конференции, Минск, 12-13 декабря 2019 г. / редкол.: В. А. Прытков [и др.]. – Минск : БГУИР, 2019. – C. 312–313. (https://libeldoc.bsuir.by/handle/123456789/38468)}
		\scnitem{Ли, Вэньцзу. Онтологическая модель генерации вопросов для интеллектуальных обучающих систем / Ли Вэньцзу, Цянь Лунвэй // Дистанционное обучение – образовательная среда XXI века : материалы XI Международной научно-методической конференции, Минск, 12-13 декабря 2019 г. / редкол.: В. А. Прытков [и др.]. – Минск : БГУИР, 2019. – C. 184–185. (https://libeldoc.bsuir.by/handle/123456789/38287)}
		\scnitem{}
		\scnitem{}
	\end{scnrelfromlist}

	\bigskip
	
	\begin{scnrelfromlist}{подраздел}
		\scnitem{\ref{subsec_means_structuring_systematizing_described_entities}~\nameref{subsec_means_structuring_systematizing_described_entities}}
		\scnitem{\ref{subsec_means_specification__described_entities}~\nameref{subsec_means_specification__described_entities}}
		\scnitem{\ref{subsec_means_explicit_description_synonymy_homonymy_sc-ident}~\nameref{subsec_means_explicit_description_synonymy_homonymy_sc-ident}}
		\scnitem{\ref{subsec_means_explicitly_describing_semantic_equivalence_other_semantic_relationships_between_information_constructs}~\nameref{subsec_means_explicitly_describing_semantic_equivalence_other_semantic_relationships_between_information_constructs}}
		\scnitem{\ref{subsec_means_explicit_description_logical_equivalence_between_sc-constructions}~\nameref{subsec_means_explicit_description_logical_equivalence_between_sc-constructions}}
		\scnitem{\ref{subsec_means_systematization_various_forms_knowledge_representation}~\nameref{subsec_means_systematization_various_forms_knowledge_representation}}
		\scnitem{\ref{subsec_means_indicating_similarities_differences_specimens}~\nameref{subsec_means_indicating_similarities_differences_specimens}}
		\scnitem{\ref{subsec_rules_constructing_entities_various_classes}~\nameref{subsec_rules_constructing_entities_various_classes}}
		\scnitem{\ref{subsec_methodological_specification_causes_trends_evolution_knowledge}~\nameref{subsec_methodological_specification_causes_trends_evolution_knowledge}}
		\scnitem{\ref{subsec_means_presentation_educational_tasks_educational_methodical_structuring_kb}~\nameref{subsec_means_presentation_educational_tasks_educational_methodical_structuring_kb}}
	\end{scnrelfromlist}
\end{SCn}

\section*{Введение в \ref{sec_didactic information}}
Важнейшим критерием качества создаваемых \textit{ostis-систем} любого назначения является создание условий для того, чтобы недостаточно квалифицированные пользователи каждой \textit{ostis-системы} (как конечные пользователи, так и ответственные за её эффективную эксплуатацию и модернизацию) могли достаточно быстро с помощью этой же \textit{ostis-системы} приобрести требуемую квалификацию. Это означает, что каждая \textit{ostis-система} независимо от её прямого назначения (автоматизации конкретных видов человеческой деятельности в конкретной области) должна быть также обучающей системой, то есть должна уметь обучать своих пользователей в направлении повышения их квалификации. Квалифицированный пользователь любой категории должен понимать возможности \textit{ostis-системы}, с которой он взаимодействует, должен понимать то, что знает и что умеет система, а также то, как можно управлять её деятельностью. Отсутствие взаимопонимания между ostis-системами и их пользователями --- это нарушение требования интероперабельности, предъявляемого как к ostis-системам, так и к их пользователям.

К сожалению, современные традиции представления различного вида документации, стандартов, отчетов, научно-технических статей и монографий не только не ориентированы на их адекватное понимание \textit{интеллектуальными компьютерными системами}, но также и не способствуют быстрому их пониманию со стороны тех людей, которым эти тексты адресованы. Последнее обстоятельство требует разработки (написания) учебников и учебных пособий, специально предназначенных для тех людей, которые \myuline{начинают} усваивать соответствующую область знаний, которые ещё не приобрели необходимую им квалификацию в этой области.

Таким образом, база знаний каждой \textit{ostis-системы} должна содержать достаточно полную собственную документацию (руководство конечного пользователя, руководство по эксплуатации, руководство по модернизации). Кроме этого, \textit{ostis-система} должна уметь:
\begin{textitemize}
	\item отвечать на самые разные (в том числе глупые) вопросы о себе;
	\item анализировать действия своих пользователей и давать им полезные советы, помогающие им быстрее приобрести требуемую квалификацию.
\end{textitemize}

Для того, чтобы база знаний ostis-системы содержада всю информацию, помогающую ей выполнить свою обучающую функцию по отношению к своим пользователям и, соответственно, помогающую пользователям \myuline{быстрее} приобрести необходимую пользовательскую квалификацию, база знаний должна содержать дополнительную информацию дидактического и, в частности, учебно-методического характера. Рассмотрим подробнее, что можно считать такой дидактической информацией и каковы языковые средства её представления в базе знаний ostis-системы.

\begin{SCn}
	\scnheader{дидактическая информация}
	\scnidtf{\textit{знание} \textit{ostis-системы}, предназначенное для её пользователей и помогающее им быстрее и адекватнее усвоить денотационную семантику знаний, хранимых в памяти этой системы}
	\scnidtf{\textit{знание}, входящее в состав \textit{базы знаний} и помогающее пользователю быстрее и качественнее (адекватнее) понять смысл (\textit{денотационную семантику}) \textit{базы знаний}}
	\scnidtf{информация, способствующая более глубокому пониманию и усвоению смысла различного рода сущностей (в том числе, и различных знаний)}
	\scnidtfexp{информация, которая при её представлении в памяти ostis-систем способствует установлению взаимопонимания между ostis-системами и их пользователями, которая ускоряет процесс формирования у пользователей требуемой квалификации в различных областях}
	\scntext{пояснение}{Дидактический эффект дидактической информации обеспечивается:
		\begin{textitemize}
			\item достаточной детализацией изучаемой сущности (полнотой семантической окрестности, описывающей связи этой сущности с другими сущностями)
			\begin{textitemize}
				\item декомпозицией,
				\item аналогами (сходные сущности в заданном смысле),
				\item метафорами (эпиграфы),
				\item антиподами (сущности, отличающиеся в заданном смысле);
			\end{textitemize}
			\item упражнениями --- решением различных задач с использованием изучаемых сущностей;
			\item ссылками на знания, хранимые в рамках той же базы знаний;
			\item ссылками на библиографические источники.
		\end{textitemize}
	}
\end{SCn}

\newpage
\subsection{Средства структуризации и систематизации описываемых сущностей}
\label{subsec_means_structuring_systematizing_described_entities}
Для представления дидактической информации данного вида используются следующие понятия:
\begin{SCn}
	\begin{scnitemize}
		\item \scnheader{структура разбиения}
			  \scnnote{Данная структура может быть иерархической, а также может описывать разбиение множеств пр нескольким признакам}
			  \scnsuperset{семантическая окрестность}
			  \scnsuperset{классификация понятий}
		\item \scnheader{структура разбиения*}
			  \scnidtf{быть структурой, описывающей разбиение заданного множества*}
		\item \scnheader{определение}
		\item \scnheader{определение*}
			  \scnidtf{быть определением заданного понятия*}
		\item \scnheader{семейство понятий, используемых в определении*}
			  \scnidtf{быть семейством понятий, используемых в определении заданного понятия*}
		\item \scnheader{логическая иерархия понятий}
		\item \scnheader{логическая иерархия понятий*}
	\end{scnitemize}
\end{SCn}

Структуризация (самой базы знаний и описываемых сущностей) заданного объекта:
\begin{textitemize}
	\item онтологическая структуризация (стратификация) и связи между предметными областями, иерархической системой наследованиями свойств;
	\item классификация (иерархия по различным признакам);
	\item иерархическая декомпозиция: пространственная, темпоральная;
	\item основное доказательство;
	\item логическая иерархия понятий (какое понятие на основании которого определяется);
	\item  иерархия определений;
	\item логическая иерархия высказываний (какое на основании каких доказывается).
\end{textitemize}

Структуризация и систематизация описываемых структур

Иерархическая структура
\begin{textitemize}
	\item классификация
	\item декомпозиция
\end{textitemize}

текст доказательства
текст обоснования

\begin{textitemize}
	\item из семейства каких высказываний \myuline{логически следует} данное (специфицируемое) высказывание
	\item что является \myuline{причиной данной} ситуации или события
\end{textitemize}

\newpage
\subsection{Средства спецификации описываемых сущностей}
\label{subsec_means_specification__described_entities}
Для представления дидактической информации данного вида используются следующие понятия:

\begin{SCn}
	\begin{scnitemize}
		\item \scnheader{спецификация (некоторой сущности \ = семантическая окрестность)}
		\scnsuperset{однозначная спецификация}
		\begin{scnindent}
			\scnsuperset{определение}
			\scnsuperset{высказывание о необходимости и достаточности}
		\end{scnindent} 
		\item \scnheader{Спецификация сущности, обозначаемой sc-элементом (тип сущности, ключевые связи обозначаемой сущности с другими сущностями)}
		\begin{scnitemize}
			\item однозначная спецификация
			\item определение
			\item пояснение (нестрогая спецификация, обычно на естественном языке)
		\end{scnitemize}
		\begin{scnrelfromlistcustom}{используемое отношение}
			\scnitemcustom{\scnheader{определение*}}
			\begin{scnindent}
				\scnsuperset{формальное определение*}
				\scnsuperset{строгое естественное-языковое определение*}
				\begin{scnindent}
					\scnidtf{точное ея-определение, которое легко перевести на формальный язык (в том числе в SC-код)}
				\end{scnindent} 
				\scnsuperset{ея-определение*}
			\end{scnindent} 
			\scnitemcustom{\scnheader{пояснение*}}
			\begin{scnindent}
				\scnsuperset{спецификация понятия}
				\scnsuperset{спецификация сущности, не являющейся ни понятием, ни информационной конструкцией}
				\begin{scnindent}
					\scnsuperset{спецификация физического лица}
				\end{scnindent} 
			\end{scnindent} 
		\end{scnrelfromlistcustom}
	\end{scnitemize}
\end{SCn}

Средства спецификации информационных конструкций
\begin{textitemize}
	\item спецификация sc-конструкции
	\item спецификация файла
	\item спецификация информационной конструкцией не являющейся ни sc-конструкцией, ни файлом
\end{textitemize}

\newpage
\subsection{Средства явного описания синонимии и омонимии sc-идентификаторов}
\label{subsec_means_explicit_description_synonymy_homonymy_sc-ident}
Явная фиксация синонимии sc-элементов с различными их sc-идентификаторами (именами)

Явное выделение омонимичных имён

Систематизация терминов и специальных обозначений (пиктограмм, условных знаков)

См. \ref{sec_identifiers}~\nameref{sec_identifiers}


\newpage
\subsection{Средства явного описания семантической эквивалентности и других семантических связей между информационными конструкциями}
\label{subsec_means_explicitly_describing_semantic_equivalence_other_semantic_relationships_between_information_constructs}
Явная фиксация \myuline{семантической эквивалентности} между sc-конструкциями и различными иными представлениями этих sc-конструкций и иных \myuline{семантических} связей между информационными конструкциями

\newpage
\subsection{Средства явного описания логической эквивалентности между sc-конструкциями и другими логическими связями}
\label{subsec_means_explicit_description_logical_equivalence_between_sc-constructions}
Для представления дидактической информации данного вида используются следующие понятия:

\begin{SCn}
	\scnheader{логическое следствие*}
\end{SCn}

Явная фиксация \myuline{логической} эквивалентности между sc-конструкциями


\newpage
\subsection{Средства систематизации различных (в том числе наглядных) форм представления знаний}
\label{subsec_means_systematization_various_forms_knowledge_representation}
Внешнее (в том числе наглядное)

Наглядные формы представления различных знаний
\begin{textitemize}
	\item графики,
	\item таблицы (таблица Менделеева),
	\item графы, диаграммы,
	\item изображения (фото, видео),
	\item аудио,
	\item виртуальная реальность (сенсорно-моторное погружение в учебную среду),
	\item тексты (доклады).
\end{textitemize}

\newpage
\subsection{Средства указания сходств, отличий, типовых экземпляров}
\label{subsec_means_indicating_similarities_differences_specimens}

\begin{SCn}
	\begin{scnrelfromlist}{эпиграф}
		\scnfileitem{Всё познаётся в сравнении}
		\scnfileitem{Важнейшим проявлением интеллекта является умение "видеть"{} сходства в различных объектах и различия в сходных}
	\end{scnrelfromlist}
\end{SCn}


\newpage
\subsection{Правила построения сущностей различных классов}
\label{subsec_rules_constructing_entities_various_classes}

\newpage
\subsection{Методологическая спецификация причин и тенденций эволюции знаний}
\label{subsec_methodological_specification_causes_trends_evolution_knowledge}

\newpage
\subsection{Средства представления учебных задач и учебно-методическая структуризация баз знаний ostis-систем}
\label{subsec_means_presentation_educational_tasks_educational_methodical_structuring_kb}

%Создание интеллектуальной творческой среды, обеспечивающей формирование интереса и мотивации
%\begin{textitemize}
%	\item на уровне индивидуальной деятельности обучаемого;
%	\item на уровне коллективных форм обучения;
%	\item на уровне общей организации учебного процесса
%	\begin{textitemize}
%		\item в школе,
%		\item на кафедре,
%		\item в вузе.
%	\end{textitemize}
%\end{textitemize}
%
%Проблемы образовательной деятельности:
%\begin{textitemize}
%	\item Отсутствует семантическая совместимость между разными учебными дисциплинами и разными специальностями (мозаика) (отсутствует фиксация текущей и постоянно эволюционирующий интегрированной картины мира). Это существенно усложняет миграцию между разными специальностями и восприятия комплекса учебных дисциплин.
%	\item Отсутствует автоматизация индивидуального подхода к обучению и к профессиональной ориентации. Целью должно быть максимально возможное раскрытие творческого потенциала \myuline{каждого} человека.
%	\item Отсутствует глубокая конвергенция между деятельностью по подготовке кадров и той деятельностью, для которой эти кадры готовятся. Образование должно происходить путём поэтапного погружения обучаемых в \myuline{реальную} профессиональную деятельность. Для этого интеллектуальные обучающие системы должны быть \myuline{частью} комплекса семантически совместимых интеллектуальных компьютерных систем, обеспечивающего автоматизацию соответствующего вида и области человеческой деятельности.
%\end{textitemize}
%
%%что конкретно необходимо описать
%Семейство семантически совместимых интеллектуальных обучающих систем, обеспечивающих комплексную подготовку специалистов в области \textit{Искусственного интеллекта} (в частности, по специальности <<Искусственный интеллект>>) и интегрированных в комплекс интеллектуальных компьютерных систем нового поколения, обеспечивающих \myuline{комплексную} автоматизацию \myuline{всех} видов и направлений человеческой деятельности в области \textit{Искусственного интеллекта} (общая постановка данной задачи рассмотрена в \ref{sec_activity_ai}~\nameref{sec_activity_ai}).
%%система ostis-сообществ, обеспечивающих образовательную деятельность в области ИИ в рамках Экосистемы OSTIS
%
%%что конкретно необходимо описать
%Семантическая спецификация (модель) обучаемого, обеспечивающая персонификацию обучения (адаптацию) к обучаемому
%%конкретные онтологии
%
%%что конкретно необходимо описать
%Семантические модели различных педагогических методик
%
%%что конкретно необходимо описать
%Семантическая модель общего управления процессом обучения (в том числе \myuline{индивидуального})
%\begin{textitemize}
%	\item лекции, семинары, индивидуальные упражнения;
%	\item ненавязчивое тестирование
%\end{textitemize}
%
%%что конкретно необходимо описать
%Система ostis-сообществ, обеспечивающих образовательную деятельность во всех областях (не только в области \textit{Искусственного интеллекта})
%\begin{textitemize}
%	\item образовательная деятельность по техническим специальностям интеллектуальным компьютерным ostis-системам поддержки проектирования и всего жизненного цикла различных искусственных систем
%	\begin{textitemize}
%		\item строительство,
%		\item машиностроение,
%		\item пищевое производство;
%	\end{textitemize}
%	\item медицина;
%	\item физика;
%	\item математика;
%	\item языкознания;
%	\item история.
%\end{textitemize}
%
%\begin{SCn}
%	\scnheader{интеллектуальные обучающие системы нового поколения}
%	\scnidtf{интеллектуальная обучающая система нового поколения}
%	\scnsuperset{обучающая ostis-система}
%	\begin{scnindent}
%		\scnsuperset{встроенная обучающая ostis-система}
%		\begin{scnindent}
%			\scnidtf{встроенная ostis-система, осуществляющая перманентное обучение своих пользователей эффективному взаимодействию с собой}
%		\end{scnindent} 
%	\end{scnindent} 
%	\scnidtf{интероперабельная интеллектуальная обучающая система, семантически совместимая и взаимодействующая с другими интеллектуальными обучающими системам --- формирующая вместе с ними систематизированную комплексную картину мира}
%\end{SCn}

\newpage
\section*{Заключение \ref{sec_didactic information}}