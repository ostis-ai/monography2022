\chapauthor{Иванюк Д.С.\\Таберко В.В.\\Прохоренко В.А.\\Смородин В.С.}
\chapter{Автоматизация производственной деятельности в рамках Экосистемы OSTIS}
\chapauthortoc{Иванюк Д.С.\\Таберко В.В.\\Прохоренко В.А.\\Смородин В.С.}
\label{chapter_enterprise}

\abstract{Аннотация к главе.}

\section{Адаптивное управление технологическим циклом производства на основе Технологии OSTIS}

текст

\subsection{Онтология предметной области "технологические процессы с вероятностными характеристиками"}

текст

\subsubsection{Качественные характеристики технологических процессов}

текст

\subsubsection{Надежностные характеристики функционирования оборудования}

текст

\subsubsection{Параметры (переменные) управления технологическим циклом}

текст

\subsubsection{Математическая модель микротехнологической операции}

текст

\subsubsection{Модель функционирования устройств оборудования}

текст

\subsubsection{Математическая модель вероятностного технологического процесса}

текст

\subsubsection{Математическая модель системы управления вероятностным технологическим процессом}

текст

\subsubsection{Имитационное моделирование технологических процессов}

текст

\subsubsection{Формализация систем управления на основе агрегатного способа имитации}

текст

\subsubsection{Пример построения технологического процесса производства в рамках Industry 4.0}

текст

\subsection{Решатели задач ostis-системы адаптивного управления вероятностным технологическим процессом производства}

текст

\subsubsection{Проблемы адаптивного управления производственной деятельностью при разработке интеллектуальных компьютерных систем нового поколения}

текст

\subsubsection{Создание математической модели производственной системы в рамках концепции Industry 4.0}

текст

\subsubsection{Разработка моделей нейрорегуляторов для решения задач адаптивного управления в условиях Экосистемы OSTIS}

текст

\subsubsection{Формализация управления технологическим циклом производства на основе применения ostis-систем}

текст

\subsubsection{Принципы оптимизации процессов управления технологическим циклом с использованием нейросетевого моделирования}

текст

\subsubsection{Построение фазовой плоскости (фазового пространства) состояний системы адаптации управления технологическим циклом производства}

текст

\subsubsection{Алгоритмы адаптации управления технологическим циклом при использовании нейросетевого моделирования}

текст

\subsubsection{Решатели задач ostis-системы адаптивного управления технологическим циклом производства}

текст

\subsubsection{Решатели задач ostis-системы адаптивного управления технологическим циклом производства}

текст

\subsubsection{Система адаптации управления}

текст

\subsubsection{Пример реализации системы адаптации управления}

текст

\section{Построение умных предприятий рецептурного производства с помощью ostis-систем}

текст

\subsection{Проблемы проектирования умных предприятий}

текст

\subsubsection{Проблемы разработки систем комплексной автоматизации}

текст

\subsubsection{Требования, предъявляемые к стандартизации предприятий}

текст

\subsubsection{Проблемы стандартизации в области производственной деятельности}

Анализ работы позволил сформулировать наиболее важные и общие проблемы, связанные с разработкой и применением современных стандартов в различных областях \cite{Серенков-2004,Углев-2012}:

\begin{itemize}
\item Прежде всего сложность ведения самих стандартов из-за дублирования информации, особенно сложность изменения терминологии.
\item Дублирующаяся информация в документации, описывающей стандарт.
\item Проблемы интернационализации стандартов -- перевод стандарта на несколько языков фактически требует поддержки и координации независимых версий стандарта на разных языках.
\item В результате несоответствия форматов разных стандартов. В результате автоматизация процесса разработки и применения стандартов усложняется.
\item Неудобство использования стандарта, особенно сложность поиска нужной информации. Как следствие, сложность изучения стандартов.
\item Сложность автоматизации проверки соответствия объекта или процесса требованиям определенного стандарта.
\item etc.
\end{itemize}

Эти проблемы в основном связаны с представлением стандартов. Наиболее перспективным подходом к решению этих задач является преобразование каждого конкретного стандарта в базу знаний, в основе которой лежит набор соответствующих этому стандарту онтологий \cite{Golenkov2019,Серенков-2004,Углев-2012,Совершенствование-2017,ISA-88-формализация}. Такой подход позволяет значительно автоматизировать процессы разработки стандарта и его применения.

В качестве примера рассмотрим стандарт \textbf{\textit{ISA-88}} \cite{ISA88} (базовый стандарт для рецептурного производства). Хотя этот стандарт широко используется американскими и европейскими компаниями и активно внедряется на территории Республики Беларусь, он имеет ряд недостатков, перечисленных ниже. Опыт автора со стандартами ISA-88 и ISA-95 выявил следующие проблемы, связанные с версиями стандарта:
\начать{элемент}
\item Американская версия стандарта -- \textit{ANSI/ISA-88.00.01-2010} -- была обновлена и находится в третьем издании 2010 года;
\item \textit{ISA-88.00.02-2001} – содержит структуры данных и рекомендации для языков;
\item \textit{ANSI/ISA-TR88.00.02-2015} -- описывает пример реализации ANSI/ISA-88.00.01;
\item \textit{ISA-88.00.03-2003} -- действие, описывающее использование общих рецептов сайта внутри и между компаниями;
\item \textit{ISA-TR88.0.03-1996} -- показывает возможные форматы представления процедур рецепта;
\item \textit{ANSI/ISA-88.00.04-2006} -- структура записей серийного производства;
\item \textit{ISA-TR88.95.01-2008} -- объясняет совместное использование ISA-88 и ISA-95;
\item В то же время европейская версия, утвержденная в 1997 году -- \textit{IEC 61512-1} -- основана на более старой версии \textit{ISA-88.01-1995};
\item Русская версия стандарта -- \textit{ГОСТ Р МЭК 61512-1-2016} -- идентична \textit{МЭК 61512-1}, то есть также устарела. Также вызывает ряд вопросов, связанных с не очень удачным переводом оригинальных английских терминов на русский язык.
\end{itemize}

Другим стандартом, часто используемым в контексте Индустрии 4.0, является \textit{\textbf{ISA-95}} \cite{ISA95}. \textit{\textbf{ISA-95}} – это отраслевой стандарт для описания систем управления высокого уровня. Его основная цель — упростить разработку таких систем, абстрагироваться от аппаратной реализации и предоставить единый интерфейс для взаимодействия со слоями ERP и MES. Состоит из следующих частей:
\начать{элемент}
\item \textit {ANSI/ISA-95.00.01-2000}, Интеграция системы управления предприятием, часть 1:
«Модели и терминология» — состоит из стандартной терминологии и объектных моделей, которые можно использовать для определения того, какая информация подлежит обмену;
\item \textit {ANSI/ISA-95.00.02-2001}, Интеграция системы управления предприятием, часть 2:
«Атрибуты объектной модели» — он состоит из атрибутов для каждого объекта, определенного в части 1. Объекты и атрибуты могут использоваться для обмена информацией между различными системами, а также могут использоваться в качестве основы для реляционных баз данных;
\item \textit {ANSI/ISA-95.00.03-2005}, Интеграция системы управления предприятием, часть 3:
«Модели управления производственными операциями» — основное внимание уделяется функциям и действиям Уровня 3 (Производство/MES);
\item \textit {ISA-95.00.04} Объектные модели \& Атрибуты Часть 4:
«Объектные модели и атрибуты для управления производственными операциями». Комитет SP95 все еще разрабатывает эту часть ISA-95. Эта техническая спецификация определяет объектную модель, которая определяет обмен информацией между действиями MES (определено в Части 3 стандарта ISA-95). Модель и атрибуты Части 4 составляют основу для разработки и внедрения стандартов интерфейса, обеспечивая гибкий поток сотрудничества и обмена информацией между различными видами деятельности MES;
\item \textit {ISA-95.00.05} Операции B2M, часть 5:
"Бизнес на производство Создание транзакций». Часть 5 стандарта ISA-95 все еще находится в разработке. Эта техническая спецификация определяет операции между рабочими местами и структурами автоматизации производства, которые могут использоваться вместе с моделями элементов частей 1 и 2. Операции объединяют и упорядочивают описанные производственные элементы и действия. в рамках предыдущей части стандарта.Такие операции возникают в любом отношении внутри организации, однако внимание данной технической спецификации уделяется взаимодействию между организацией и системой управления.
\end{itemize}

Модели помогают определить границы между бизнес-системами и системами управления. Они помогают ответить на вопросы о том, какие функции могут выполнять какие задачи и какой информацией необходимо обмениваться между приложениями.

Первый этап построения модели цифрового двойника требует встраивания данных на более низких уровнях производства, таких как производственные процессы и оборудование. Схема производства P\&ID служит источником этих данных. Поэтому стандарт ISA 5.1 \cite{ISA_5_1} должен работать со схемой P\&ID и широко используется в системах управления наряду со стандартом ISA 88 для полного описания нижних уровней производства. Этот стандарт полезен, когда требуется ссылка на оборудование в химической, нефтяной, энергетической, кондиционирующей, металлообрабатывающей и многих других отраслях промышленности. Стандарт позволяет любому человеку с достаточным уровнем знаний о предприятии читать блок-схемы, чтобы понять, как измерять и контролировать процесс, не вдаваясь в детали приборов или знаний эксперта по приборам. Он предназначен для предоставления достаточной информации, чтобы SA5.1 Целью настоящего стандарта было установить последовательный метод наименования приборов и контрольно-измерительных систем, используемых для измерения и контроля. Для этого представлена система обозначений, включающая символы и идентификационные коды. Последним выпуском подкомитета ISA5.1 является обновленный \textit {ISA-5.1-2022}, «Символы приборов и идентификация».

Обучение — это простой способ достичь этих стандартов. Международное общество автоматизации (\textit {ISA}) — некоммерческая профессиональная ассоциация и признанный лидер в области обучения автоматизации и управлению, занимающийся подготовкой рабочей силы к технологическим изменениям и отраслевым вызовам. Однако цена относительно высока, около 1000 долларов на человека в день. Для 2 человек это 10000\$ за обычный курс на 5 дней. Для одних стран это доступно, для других нет.

Принимаются во внимание различные установленные процедурные требования различных организаций, но это делается путем предоставления альтернативных методов символики, если это не противоречит целям стандарта. Существует множество вариантов добавления информации или упрощения символа при желании.

Эти и другие стандарты сейчас распространяются в виде документов, неудобных для автоматизированной обработки и, как отмечалось выше, сильно зависят от языка, на котором написан каждый документ.


\subsection{Подходы к проектированию и стандартизации умных предприятий рецептурного производства с помощью ostis-систем}

текст

\subsubsection{Подходы к автоматизации предприятий}

текст

\subsubsection{Предлагаемый подход к автоматизации предприятий}

текст

\subsubsection{Принципы формализации стандартов рецептурного производства}

текст

\subsection{Примеры реализации проектирования умных предприятий на основе онтологического подхода в рамках концепции Industry 4.0}

текст

\subsubsection{Процесс проектирования умных предприятий рецептурного производства с помощью ostis-систем на примере ОАО "Савушкин продукт"}

текст

\subsubsection{Пример построения системы автоматизации деятельности инженера-технолога на основе онтологического подхода в рамках концепции Industry 4.0}

текст


%\input{author/references}
