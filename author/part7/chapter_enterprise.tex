\chapauthor{Иванюк Д.С.\\Таберко В.В.\\Прохоренко В.А.\\Смородин В.С.}
\chapter{Автоматизация производственной деятельности в рамках Экосистемы OSTIS}
\chapauthortoc{Иванюк Д.С.\\Таберко В.В.\\Прохоренко В.А.\\Смородин В.С.}
\label{chapter_enterprise}

\abstract{Аннотация к главе.}

\section{Адаптивное управление технологическим циклом производства на основе Технологии OSTIS}

текст

\subsection{Онтология предметной области "технологические процессы с вероятностными характеристиками"}

текст

\subsubsection{Качественные характеристики технологических процессов}

текст

\subsubsection{Надежностные характеристики функционирования оборудования}

текст

\subsubsection{Параметры (переменные) управления технологическим циклом}

текст

\subsubsection{Математическая модель микротехнологической операции}

текст

\subsubsection{Модель функционирования устройств оборудования}

текст

\subsubsection{Математическая модель вероятностного технологического процесса}

текст

\subsubsection{Математическая модель системы управления вероятностным технологическим процессом}

текст

\subsubsection{Имитационное моделирование технологических процессов}

текст

\subsubsection{Формализация систем управления на основе агрегатного способа имитации}

текст

\subsubsection{Пример построения технологического процесса производства в рамках Industry 4.0}

текст

\subsection{Решатели задач ostis-системы адаптивного управления вероятностным технологическим процессом производства}

текст

\subsubsection{Проблемы адаптивного управления производственной деятельностью при разработке интеллектуальных компьютерных систем нового поколения}

текст

\subsubsection{Создание математической модели производственной системы в рамках концепции Industry 4.0}

текст

\subsubsection{Разработка моделей нейрорегуляторов для решения задач адаптивного управления в условиях Экосистемы OSTIS}

текст

\subsubsection{Формализация управления технологическим циклом производства на основе применения ostis-систем}

текст

\subsubsection{Принципы оптимизации процессов управления технологическим циклом с использованием нейросетевого моделирования}

текст

\subsubsection{Построение фазовой плоскости (фазового пространства) состояний системы адаптации управления технологическим циклом производства}

текст

\subsubsection{Алгоритмы адаптации управления технологическим циклом при использовании нейросетевого моделирования}

текст

\subsubsection{Решатели задач ostis-системы адаптивного управления технологическим циклом производства}

текст

\subsubsection{Решатели задач ostis-системы адаптивного управления технологическим циклом производства}

текст

\subsubsection{Система адаптации управления}

текст

\subsubsection{Пример реализации системы адаптации управления}

текст

\section{Построение умных предприятий рецептурного производства с помощью ostis-систем}

текст

\subsection{Проблемы проектирования умных предприятий}

текст

\subsubsection{Проблемы разработки систем комплексной автоматизации}

Существующие средства автоматизации деятельности предприятия имеют высокую стоимость, трудны в освоении и адаптации к конкретному производству. Как правило, такие
средства, с одной стороны, жѐстко ориентированы на решение некоторого ограниченного
класса задач, с другой стороны, разработчики стремятся сделать такого рода средства как
можно более универсальными, наращивая их частными решениями, что приводит к сложности и громоздкости таких систем. Вследствие подобного подхода к наращиванию функционала существующие средства автоматизации деятельности предприятия имеют низкий уровень гибкости (возможности внесения изменений), что приводит к существенным накладным
расходам при адаптации таких средств к новым требованиям. Как правило, внесение изменений в указанные средства требует вмешательства разработчиков (часто сторонних с точки
зрения предприятия), что влечѐт значительные временные и финансовые затраты. Как следствие указанных проблем, далеко не всякое предприятие может обеспечить высокий уровень
автоматизации своей деятельности, даже в случае наличия на рынке подходящих решений.
Отсутствие общих унифицированных моделей и средств построения систем автоматизации деятельности предприятия приводит к большому количеству дублирований аналогичных
решений как в рамках различных предприятий, так и в рамках разных подразделений одного
предприятия. При этом часто возникает ситуация, когда некоторые частные системы, решающие различные задачи в рамках одного предприятия, оказываются несовместимыми между
собой, что приводит к дополнительным расходам на реализацию механизмов согласования,
например, преобразование форматов данных. Отсутствие такого рода моделей препятствует
дальнейшему повышению уровня автоматизации предприятия, в частности, в области автоматизации принятия решений в нештатных ситуациях, прогнозирования дальнейшего развития событий.

\subsubsection{Требования, предъявляемые к стандартизации предприятий}

Средства автоматизации предприятия должны оперативно и с минимальными затратами
времени сотрудников адаптироваться к любым изменениям самого производства – к расширению или сокращению объѐмов производства, изменениям номенклатуры производства,
изменению используемого оборудования, изменению общей структуры производства, изменению взаимодействия с поставщиками и потребителями, к изменению нормативноправовых актов (включая стандарты), к различного рода непредвиденным обстоятельствам.
Адаптация средств автоматизации предприятия ко всем видам изменений самого предприятия и всем аспектам его взаимодействия с внешней средой требует внесения изменений в
модель предприятия, полностью отражающую текущее состояние его деятельности.
Средства автоматизации предприятия должны быть гибкими не только для оперативной
адаптации к реконфигурации производства, но и для оперативного внесения изменений в сами средства автоматизации в направлении их постоянного совершенствования. Здесь существенным является не только снижение трудоѐмкости повышения уровня автоматизации, но
и поддержка высоких темпов повышения уровня автоматизации, а также чѐтко продуманный
переходный процесс от одного уровня автоматизации к следующему, в ходе которого одновременно используется и устаревший вариант, и новый.
Эксплуатация системы автоматизации предприятия текущего уровня и перманентный
процесс повышения этого уровня требуют согласованного и квалифицированного взаимодействия сотрудников предприятия. Основой такого взаимодействия является хорошо структурированная, достаточно полная и оперативно актуализируемая модель предприятия, отражающая все аспекты текущего состояния структуры и деятельности предприятия, а такжеОнтология проектирования, том 7, №2(24)/2017 125
В.В. Голенков, В.В. Таберко, Д.С. Иванюк, К.В. Русецкий, Д.В. Шункевич и др.
планы его развития. Такого рода комплексная модель называется знаниями предприятия, которыми надо управлять (добывать, хранить, модернизировать, распространять и т.д.) [1, 2].
Повышение уровня автоматизации предприятия предполагает существенное расширение
числа автоматически или автоматизированно решаемых задач, а это, в свою очередь, приводит к автоматизации решения интеллектуальных задач, т.е. к использованию технологий искусственного интеллекта. К числу интеллектуальных задач, решаемых на предприятии, можно отнести:
 анализ производственных ситуаций (в том числе нештатных);
 принятие решений на различных уровнях;
 планирование поведения в сложных обстоятельствах;
 генерация, актуализация документации;
 обучение новых и повышение квалификации действующих сотрудников; и т.д.
Для того, чтобы обеспечить широкое применение технологий искусственного интеллекта
в автоматизации предприятия, все корпоративные знания предприятия должны быть записаны на формальном языке представления знаний. При этом указанный язык должен быть удобен не только для использования в интеллектуальных компьютерных системах, но и для использования всеми сотрудниками предприятия.
3

\subsubsection{Проблемы стандартизации в области производственной деятельности}

Анализ работы позволил сформулировать наиболее важные и общие проблемы, связанные с разработкой и применением современных стандартов в различных областях \cite{Серенков-2004,Углев-2012}:

\begin{itemize}
\item Прежде всего сложность ведения самих стандартов из-за дублирования информации, особенно сложность изменения терминологии.
\item Дублирующаяся информация в документации, описывающей стандарт.
\item Проблемы интернационализации стандартов -- перевод стандарта на несколько языков фактически требует поддержки и координации независимых версий стандарта на разных языках.
\item В результате несоответствия форматов разных стандартов. В результате автоматизация процесса разработки и применения стандартов усложняется.
\item Неудобство использования стандарта, особенно сложность поиска нужной информации. Как следствие, сложность изучения стандартов.
\item Сложность автоматизации проверки соответствия объекта или процесса требованиям определенного стандарта.
\item etc.
\end{itemize}

Эти проблемы в основном связаны с представлением стандартов. Наиболее перспективным подходом к решению этих задач является преобразование каждого конкретного стандарта в базу знаний, в основе которой лежит набор соответствующих этому стандарту онтологий \cite{Golenkov2019,Серенков-2004,Углев-2012,Совершенствование-2017,ISA-88-формализация}. Такой подход позволяет значительно автоматизировать процессы разработки стандарта и его применения.

В качестве примера рассмотрим стандарт \textbf{\textit{ISA-88}} \cite{ISA88} (базовый стандарт для рецептурного производства). Хотя этот стандарт широко используется американскими и европейскими компаниями и активно внедряется на территории Республики Беларусь, он имеет ряд недостатков, перечисленных ниже. Опыт автора со стандартами ISA-88 и ISA-95 выявил следующие проблемы, связанные с версиями стандарта:
\begin{itemize}
\item Американская версия стандарта -- \textit{ANSI/ISA-88.00.01-2010} -- была обновлена и находится в третьем издании 2010 года;
\item \textit{ISA-88.00.02-2001} – содержит структуры данных и рекомендации для языков;
\item \textit{ANSI/ISA-TR88.00.02-2015} -- описывает пример реализации ANSI/ISA-88.00.01;
\item \textit{ISA-88.00.03-2003} -- действие, описывающее использование общих рецептов сайта внутри и между компаниями;
\item \textit{ISA-TR88.0.03-1996} -- показывает возможные форматы представления процедур рецепта;
\item \textit{ANSI/ISA-88.00.04-2006} -- структура записей серийного производства;
\item \textit{ISA-TR88.95.01-2008} -- объясняет совместное использование ISA-88 и ISA-95;
\item В то же время европейская версия, утвержденная в 1997 году -- \textit{IEC 61512-1} -- основана на более старой версии \textit{ISA-88.01-1995};
\item Русская версия стандарта -- \textit{ГОСТ Р МЭК 61512-1-2016} -- идентична \textit{МЭК 61512-1}, то есть также устарела. Также вызывает ряд вопросов, связанных с не очень удачным переводом оригинальных английских терминов на русский язык.
\end{itemize}

Другим стандартом, часто используемым в контексте Индустрии 4.0, является \textit{\textbf{ISA-95}} \cite{ISA95}. \textit{\textbf{ISA-95}} – это отраслевой стандарт для описания систем управления высокого уровня. Его основная цель — упростить разработку таких систем, абстрагироваться от аппаратной реализации и предоставить единый интерфейс для взаимодействия со слоями ERP и MES. Состоит из следующих частей:
\begin{itemize}
\item \textit {ANSI/ISA-95.00.01-2000}, Интеграция системы управления предприятием, часть 1:
«Модели и терминология» — состоит из стандартной терминологии и объектных моделей, которые можно использовать для определения того, какая информация подлежит обмену;
\item \textit {ANSI/ISA-95.00.02-2001}, Интеграция системы управления предприятием, часть 2:
«Атрибуты объектной модели» — он состоит из атрибутов для каждого объекта, определенного в части 1. Объекты и атрибуты могут использоваться для обмена информацией между различными системами, а также могут использоваться в качестве основы для реляционных баз данных;
\item \textit {ANSI/ISA-95.00.03-2005}, Интеграция системы управления предприятием, часть 3:
«Модели управления производственными операциями» — основное внимание уделяется функциям и действиям Уровня 3 (Производство/MES);
\item \textit {ISA-95.00.04} Объектные модели \& Атрибуты Часть 4:
«Объектные модели и атрибуты для управления производственными операциями». Комитет SP95 все еще разрабатывает эту часть ISA-95. Эта техническая спецификация определяет объектную модель, которая определяет обмен информацией между действиями MES (определено в Части 3 стандарта ISA-95). Модель и атрибуты Части 4 составляют основу для разработки и внедрения стандартов интерфейса, обеспечивая гибкий поток сотрудничества и обмена информацией между различными видами деятельности MES;
\item \textit {ISA-95.00.05} Операции B2M, часть 5:
"Бизнес на производство Создание транзакций». Часть 5 стандарта ISA-95 все еще находится в разработке. Эта техническая спецификация определяет операции между рабочими местами и структурами автоматизации производства, которые могут использоваться вместе с моделями элементов частей 1 и 2. Операции объединяют и упорядочивают описанные производственные элементы и действия. в рамках предыдущей части стандарта.Такие операции возникают в любом отношении внутри организации, однако внимание данной технической спецификации уделяется взаимодействию между организацией и системой управления.
\end{itemize}

Модели помогают определить границы между бизнес-системами и системами управления. Они помогают ответить на вопросы о том, какие функции могут выполнять какие задачи и какой информацией необходимо обмениваться между приложениями.

Первый этап построения модели цифрового двойника требует встраивания данных на более низких уровнях производства, таких как производственные процессы и оборудование. Схема производства P\&ID служит источником этих данных. Поэтому стандарт ISA 5.1 \cite{ISA_5_1} должен работать со схемой P\&ID и широко используется в системах управления наряду со стандартом ISA 88 для полного описания нижних уровней производства. Этот стандарт полезен, когда требуется ссылка на оборудование в химической, нефтяной, энергетической, кондиционирующей, металлообрабатывающей и многих других отраслях промышленности. Стандарт позволяет любому человеку с достаточным уровнем знаний о предприятии читать блок-схемы, чтобы понять, как измерять и контролировать процесс, не вдаваясь в детали приборов или знаний эксперта по приборам. Он предназначен для предоставления достаточной информации, чтобы SA5.1 Целью настоящего стандарта было установить последовательный метод наименования приборов и контрольно-измерительных систем, используемых для измерения и контроля. Для этого представлена система обозначений, включающая символы и идентификационные коды. Последним выпуском подкомитета ISA5.1 является обновленный \textit {ISA-5.1-2022}, «Символы приборов и идентификация».

Обучение — это простой способ достичь этих стандартов. Международное общество автоматизации (\textit {ISA}) — некоммерческая профессиональная ассоциация и признанный лидер в области обучения автоматизации и управлению, занимающийся подготовкой рабочей силы к технологическим изменениям и отраслевым вызовам. Однако цена относительно высока, около 1000 долларов на человека в день. Для 2 человек это 10000\$ за обычный курс на 5 дней. Для одних стран это доступно, для других нет.

Принимаются во внимание различные установленные процедурные требования различных организаций, но это делается путем предоставления альтернативных методов символики, если это не противоречит целям стандарта. Существует множество вариантов добавления информации или упрощения символа при желании.

Эти и другие стандарты сейчас распространяются в виде документов, неудобных для автоматизированной обработки и, как отмечалось выше, сильно зависят от языка, на котором написан каждый документ.


\subsection{Подходы к проектированию и стандартизации умных предприятий рецептурного производства с помощью ostis-систем}

текст

\subsubsection{Подходы к автоматизации предприятий}

В настоящее время существует ряд подходов, ориентированных на повышение уровня
автоматизации и гибкости предприятий различного рода. Рассмотрим те из них, которые оказали влияние на развитие подхода, предлагаемого в данной работе.
Онтологические модели предприятий. Подход к проектированию различного рода систем на основе онтологических моделей широко используется в настоящее время [3], при
этом в особую область исследований выделяют «онтологии предприятия» [2]. Суть предлагаемых подходов состоит в построении онтологий, описывающих деятельность того или
иного предприятия или его подразделений. Недостатками данных моделей являются отсутствие унификации представления различных видов знаний предприятия, отсутствие единого
подхода к выделению и формированию онтологий, отсутствие единого подхода к построению иерархии онтологий, что ограничивает возможность построения комплексной взаимосвязанной системы онтологий.
Модели управления знаниями предприятий. Управление знаниями в организации – это
систематический процесс идентификации, использования и передачи информации, знаний,
которые люди могут создавать, совершенствовать и применять. Это процесс, в ходе которого
организация генерирует знания, накапливает их и использует в интересах получения конкурентных преимуществ. В настоящее время управление знаниями предприятия реализуется в
виде систем управления знаниями [4]. Наиболее актуальным направлением в формализации
процесса накопления и управления знаниями предприятия является применение онтологического подхода к построению моделей такого рода процессов.
Модели ситуационного управления. Термин «ситуационное управление» впервые появился в работах Д.А. Поспелова [5]. Это направление получило дальнейшее развитие [6, 7],
а в ряде новых работ показано применение в реализации методов ситуационного управления
онтологического подхода [8, 9]. Таким образом, модели и методы ситуационного управления
могут быть использованы при построении онтологической модели предприятия с целью повышения эффективности разрабатываемых решений в конкретных производственных ситуациях.
Многоагентные модели предприятий. В настоящее время многоагентная модель широко применяется при проектировании систем автоматизации производства на различных
уровнях. Удобство такого подхода и широта его использования обусловлены схожестью
многоагентной модели с реальными процессами, происходящими на предприятии. Действительно, в классической многоагентной системе под агентом понимается некий субъект, как
правило, активный и способный взаимодействовать с окружающей средой [10, 11]. Будучи
объединѐнными в коллективы, такие агенты способны решать задачи гораздо более сложные,
чем мог бы решить один агент. К достоинствам многоагентного подхода можно отнести возможность построения на его основе распределѐнных многоуровневых систем.
Наиболее очевидной интерпретацией такого рода модели в применении к конкретному
предприятию является рассмотрение его работников как агентов, каждый из которых способен решать определѐнный класс задач, вынужденных координировать свои действия для достижения общей коллективной цели. С учѐтом иерархии структурных подразделений конкретной организации могут быть выделены и уровни иерархии агентов, соответствующие
отделам или цехам.
Модели реинжиниринга бизнес-процессов предприятий. Реинжиниринг бизнеспроцессов – это фундаментальное переосмысление и радикальное перепроектирование бизнес-процессов предприятий для достижения резких, скачкообразных улучшений в основных
актуальных показателях их деятельности: стоимость, качество, услуги и темпы [12]. Он базируется на понятиях будущего образа фирмы и модели бизнеса, раскрываемых в [13]. Для
того, чтобы повысить эффективность реинжиниринга, необходимо обеспечить возможность
построения формальных моделей, описывающих предприятие на разных уровнях детализации, и обеспечить унификацию таких моделей, их интегрируемость и иерархичность.
Основной недостаток всех приведѐнных выше моделей заключается в том, что ни одна из
них не обладает достаточной полнотой, и для наиболее адекватного соответствия реальному
предприятию его модель должна быть результатом интеграции всех этих моделей.

\subsubsection{Предлагаемый подход к автоматизации предприятий}

В основе предлагаемого подхода к решению указанных проблем лежат следующие принципы.
 Предприятие рассматривается как распределѐнная, интеллектуальная социотехническая
система, в основе которой лежит хорошо структурированная общая база знаний предприятия.
 В рамках базы знаний предприятия интегрируются все вышеуказанные модели.
 Предприятие рассматривается как иерархическая многоагентная система. В качестве
агентов выступают как сотрудники предприятия, так и программные (программноаппаратные) агенты. Иерархичность многоагентной системы означает то, что агенты могут быть неатомарными, т.е. коллективами взаимодействующих между собой агентов,
причѐм такая структура может быть многократно вложенной.
 Весь комплекс средств (как информационных, так и материальных), обеспечивающих
деятельность предприятия, оформляется в виде интегрированной распределѐнной интеллектуальной системы, которую будем называть интеллектуальной корпоративной системой. Основными пользователями этой системы являются сотрудники предприятия.
 Проектирование онтологической модели предприятия сводится к проектированию онтологической модели его интеллектуальной корпоративной системы, которая далее может
интерпретироваться имеющимся набором материальных ресурсов. При этом онтологическая модель предприятия является и объектом, и результатом проектирования.
Для реализации корпоративной системы предприятия предлагается использовать технологию OSTIS [14, 15]. Из этого следует:
 в качестве основы для представления знаний используется унифицированный, универсальный язык представления – SC-код;
 разработка системы сводится к разработке еѐ модели, описанной средствами SC-кода
(sc-модели), которая затем интерпретируется одной из платформ интерпретации;
 база знаний имеет иерархическую структуру, позволяющую рассматривать хранимые
знания на различных уровнях детализации (прежде всего это иерархия предметных областей (ПрО) и соответствующих им онтологий [16]);
 частью технологии являются средства коллективного проектирования баз знаний, средства проектирования машин обработки знаний и их компонентов;
 модель обработки знаний основана на многоагентном подходе, позволяющем строить
параллельные асинхронные машины обработки знаний, интегрировать различные
частные модели обработки в рамках одной системы;
 все агенты взаимодействуют исключительно посредством общей памяти, хранящей конструкции SC-кода (sc-памяти); такой подход позволяет обеспечить гибкость системы и
возможность параллельного решения различных задач [17];
 для разработки программ агентов используется внутренний параллельный язык SCP,
тексты которого также представлены в SC-коде, что позволяет обеспечить платформенную независимость таких агентов.
На данном этапе работы основное внимание уделено решению задачи разработки онтологической модели базы знаний, в частности – построению набора онтологий ПрО, описывающих содержание основных стандартов. Формальное представление стандартов является основой для согласования всех ключевых аспектов деятельности предприятия и построения
общей онтологической модели всего предприятия в целом и отдельных его компонентов.
Чтобы убедиться в актуальности решаемых задач, рассмотрим текущее состояние и историю развития автоматизации на конкретном предприятии.

\subsubsection{Принципы формализации стандартов рецептурного производства}

текст

\subsection{Примеры реализации проектирования умных предприятий на основе онтологического подхода в рамках концепции Industry 4.0}

текст

\subsubsection{Процесс проектирования умных предприятий рецептурного производства с помощью ostis-систем на примере ОАО "Савушкин продукт"}

текст

\subsubsection{Пример построения системы автоматизации деятельности инженера-технолога на основе онтологического подхода в рамках концепции Industry 4.0}

текст


%\input{author/references}
