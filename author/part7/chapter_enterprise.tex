\chapauthor{Иванюк Д.С.\\Таберко В.В.\\Прохоренко В.А.\\Смородин В.С.}
\chapter{Автоматизация производственной деятельности в рамках Экосистемы OSTIS}
\chapauthortoc{Иванюк Д.С.\\Таберко В.В.\\Прохоренко В.А.\\Смородин В.С.}
\label{chapter_enterprise}

\abstract{Аннотация к главе.}

\section{Адаптивное управление технологическим циклом производства на основе Технологии OSTIS}

текст

\subsection{Онтология предметной области "технологические процессы с вероятностными характеристиками"}

текст

\subsubsection{Качественные характеристики технологических процессов}

текст

\subsubsection{Надежностные характеристики функционирования оборудования}

текст

\subsubsection{Параметры (переменные) управления технологическим циклом}

текст

\subsubsection{Математическая модель микротехнологической операции}

текст

\subsubsection{Модель функционирования устройств оборудования}

текст

\subsubsection{Математическая модель вероятностного технологического процесса}

текст

\subsubsection{Математическая модель системы управления вероятностным технологическим процессом}

текст

\subsubsection{Имитационное моделирование технологических процессов}

текст

\subsubsection{Формализация систем управления на основе агрегатного способа имитации}

текст

\subsubsection{Пример построения технологического процесса производства в рамках Industry 4.0}

текст

\subsection{Решатели задач ostis-системы адаптивного управления вероятностным технологическим процессом производства}

текст

\subsubsection{Проблемы адаптивного управления производственной деятельностью при разработке интеллектуальных компьютерных систем нового поколения}

текст

\subsubsection{Создание математической модели производственной системы в рамках концепции Industry 4.0}

текст

\subsubsection{Разработка моделей нейрорегуляторов для решения задач адаптивного управления в условиях Экосистемы OSTIS}

текст

\subsubsection{Формализация управления технологическим циклом производства на основе применения ostis-систем}

текст

\subsubsection{Принципы оптимизации процессов управления технологическим циклом с использованием нейросетевого моделирования}

текст

\subsubsection{Построение фазовой плоскости (фазового пространства) состояний системы адаптации управления технологическим циклом производства}

текст

\subsubsection{Алгоритмы адаптации управления технологическим циклом при использовании нейросетевого моделирования}

текст

\subsubsection{Решатели задач ostis-системы адаптивного управления технологическим циклом производства}

текст

\subsubsection{Решатели задач ostis-системы адаптивного управления технологическим циклом производства}

текст

\subsubsection{Система адаптации управления}

текст

\subsubsection{Пример реализации системы адаптации управления}

текст

\section{Построение умных предприятий рецептурного производства с помощью ostis-систем}

текст

\subsection{Проблемы проектирования умных предприятий}

текст

\subsubsection{Проблемы разработки систем комплексной автоматизации}

текст

\subsubsection{Требования, предъявляемые к стандартизации предприятий}

текст

\subsubsection{Проблемы стандартизации в области производственной деятельности}

текст

\subsection{Подходы к проектированию и стандартизации умных предприятий рецептурного производства с помощью ostis-систем}

текст

\subsubsection{Подходы к автоматизации предприятий}

текст

\subsubsection{Предлагаемый подход к автоматизации предприятий}

текст

\subsubsection{Принципы формализации стандартов рецептурного производства}

текст

\subsection{Примеры реализации проектирования умных предприятий на основе онтологического подхода в рамках концепции Industry 4.0}

текст

\subsubsection{Процесс проектирования умных предприятий рецептурного производства с помощью ostis-систем на примере ОАО "Савушкин продукт"}

текст

\subsubsection{Пример построения системы автоматизации деятельности инженера-технолога на основе онтологического подхода в рамках концепции Industry 4.0}

текст


%\input{author/references}