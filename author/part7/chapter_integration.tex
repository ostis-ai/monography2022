\chapter{Интеграция Экосистемы OSTIS с современными сервисами и информационными ресурсами}
\chapauthortoc{Загорский~А.~Г.\\Таранчук~В.Б.\\Шункевич~Д.В.\\Соловьёв~А.~М.\\Коршунов ~Р.~А.\\Савёнок~В.~А.}
{\label{chapter_integration}}

\vspace{-7\baselineskip}

\begin{SCn}
\begin{scnrelfromlist}{автор}
	\scnitem{Загорский~А.~С.}
	\scnitem{Таранчук~В.~Б.}
	\scnitem{Шункевич~Д.~В.}
	\scnitem{Соловьёв~А.~М.}
	\scnitem{Коршунов~Р.~А.}	
	\scnitem{Савёнок~В.~А.}
\end{scnrelfromlist}

\bigskip

\scntext{аннотация}{В главе описываются общие принципы \textit{интеграции}, а также принципы \textit{интеграции} \textit{Экосистемы OSTIS} с разнородными \textit{сервисами} и структурированными \textit{информационными ресурсами}. Также описывается \textit{интеграция} инструментов компьютерной алгебры в \textit{ostis-системы}. Эта глава будет полезна для специалистов в области программной инженерии, искусственного интеллекта и системного анализа, которые заинтересованы в \textit{интеграции} различных \textit{сервисов} и \textit{ресурсов} в \textit{интеллектуальные системы}.}

\begin{SCn}

\bigskip
\begin{scnrelfromlist}{ключевое понятие}
	\scnitem{интеграция}
	\scnitem{информационный ресурс}
	\scnitem{сервис}
\end{scnrelfromlist}

\bigskip
\begin{scnrelfromlist}{ключевой знак}
	\scnitem{Экосистема OSTIS}
\end{scnrelfromlist}

\bigskip
\begin{scnrelfromlist}{библиографическая ссылка}
	\scnitem{\scncite{Li2012a}}
	\scnitem{\scncite{Valdez2019}}
	\scnitem{\scncite{Caldarola2015}}
	\scnitem{\scncite{Bork2019}}
	% \scnitem{\scncite{Burstrom2022}} move to ecosystem
\end{scnrelfromlist}

\end{SCn}

\bigskip

\begin{scnrelfromlist}{подраздел}
	\scnitem{\ref{sec_integration_common_principles}~\nameref{sec_integration_common_principles}}
	\scnitem{\ref{sec_integration_algebra}~\nameref{sec_integration_algebra}}
\end{scnrelfromlist}

\end{SCn}

\section*{Введение в Главу \ref{chapter_integration}}
Важным является не только сама \textit{Экосистема OSTIS} как форма реализации Общества 5.0, но и процесс поэтапного перехода от современной глобальной сети \textit{компьютерных систем} к глобальной сети \textit{ostis-систем}, то есть к \textit{Экосистеме OSTIS}.

\textit{интеграция} современных \textit{сервисов} и \textit{информационных ресурсов} с \textit{Экосистемой OSTIS} является важнейшим элементом для продвижения технологических инноваций в современном мире. Поскольку спрос на эффективные, надежные и доступные системы продолжает расти, очень важно иметь комплексную и интегрированную платформу, способную удовлетворить различные потребности.

\textit{Технология OSTIS} обеспечивает основу для разработки сложных систем и их беспрепятственной \textit{интеграции} c существующими услугами и \textit{ресурсами}. Системы, разработанные в раммках комплексной экосистемы в совокупности предоставляют возможности использовать самый разнообразный функционал: управление информацией, анализ данных, принятие решений, автоматизацию и так далее.

В данной главе рассматривается значение \textit{интеграции} \textit{Экосистемы OSTIS} с современными \textit{сервисами} и \textit{информационными ресурсами}, а также потенциальные преимущества такой \textit{интеграции}. 
Цель данной главы - дать всесторонний анализ и представить аргументы в пользу внедрения современных систем и \textit{сервисов} в \textit{Экосистему OSTIS}.

\section{Общие принципы интеграции Экосистемы OSTIS с современными сервисами и информационными ресурсами}
{\label{sec_integration_common_principles}}

\section{Интеграция инструментов компьютерной алгебры в приложения OSTIS}
{\label{sec_integration_algebra}} 


%\input{author/references}
\section*{Заключение к Главе \ref{chapter_integration}}

При \textit{интеграции} различных \textit{сервисов} и \textit{информационных ресурсов} с \textit{Экосистемой OSTIS} пользователь может получить ряд значительных преимуществ, которые могут повысить эффективность его деятельности:
\begin{textitemize}
	\item повышение точности и качества данных, получение более точной и полной информации, что может повысить качество принимаемых решений (это особенно важно в условиях быстро меняющейся среды, когда точность и качество данных играют решающую роль);
	\item улучшение управления и контроля процессов обмена информацией, что может помочь в принятии быстрых и правильных решений;
	\item снижение затрат на разработку и поддержку приложений, улучшение скорости разработки новых приложений, а также повышение качества их реализации (это связано с тем, что \textit{Экосистема OSTIS} позволяет использовать уже существующие компоненты, что сокращает время и затраты на разработку).
\end{textitemize}
