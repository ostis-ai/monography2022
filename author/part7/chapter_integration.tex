\chapter{Интеграция Экосистемы OSTIS с современными сервисами и информационными ресурсами}
\chapauthortoc{Загорский А.Г.\\Коршунов Р.А.\\Таранчук В.Б.\\Голенков В.В.\\Шункевич Д.В.}
{\label{chapter_integration}}

\vspace{-7\baselineskip}

\begin{SCn}
\begin{scnrelfromlist}{автор}
	\scnitem{Загорский А.С.}
	\scnitem{Коршунов Р.А.}
	\scnitem{Таранчук В.Б.}
	\scnitem{Голенков В.В.}
	\scnitem{Шункевич Д.В.}
\end{scnrelfromlist}

\bigskip

\scntext{аннотация}{***}

\bigskip

\begin{scnrelfromlist}{подраздел}
	\scnitem{\ref{sec_integration_common_principles}~\nameref{sec_integration_common_principles}}
	\scnitem{\ref{sec_integration_services}~\nameref{sec_integration_services}}
	\scnitem{\ref{sec_integration_resources}~\nameref{sec_integration_resources}}
	\scnitem{\ref{sec_integration_algebra}~\nameref{sec_integration_algebra}}
\end{scnrelfromlist}

\end{SCn}

Очень важно проектировать не только саму Экосистему OSTIS как форму реализации Общества 5.0, но и процесс поэтапного перехода от современной глобальной сети компьютерных систем к глобальной сети ostis-систем (т.е. к Экосистеме OSTIS).

В рамках такого переходного периода ostis-системы могут выполнять роль системных интеграторов различных ресурсов и сервисов, реализованных современными компьютерными системами, поскольку уровень интеллекта ostis-систем позволяет им с любой степенью детализации специфицировать интегрируемые компьютерные системы и, следовательно, достаточно адекватно "понимать", что знает и/или умеет каждая из них. 
Также ostis-системы способны достаточно качественно координировать деятельность стороннего ресурса и сервиса и обеспечивать "релевантный" поиск нужного ресурса и сервиса. 
Кроме того системы могут выполнять роль интеллектуальных help-систем – помощников и консультантов по вопросам эффективной эксплуатации различных компьютерных систем со сложными функциональными возможностями, имеющими пользовательский интерфейс с нетривиальной семантикой и использующимися в сложных предметных областях. 
Такие интеллектуальные help-системы можно сделать интеллектуальными посредниками между соответствующими компьютерными системами их пользователями.

На первых этапах перехода к Обществу 5.0 нет необходимости преобразовывать в ostis-системы все современные системы автоматизации некоторых видов и областей человеческой деятельности. 
Однако, ostis-системы должны взять на себя координационно-связующую роль благодаря высокому уровню своей интероперабельности. 
ostis-системы должны научиться либо выполнять миссию активной интероперабельной надстройки над различными современными средствами автоматизации, либо ставить перед современными средствами автоматизации выполнимые для них задачи, обеспечивая их непосредственное участие в решении сложных комплексных задач и организуя управление взаимодействием различных средств автоматизации в процессе коллективного решения сложных комплексных задач.


\section{Общие принципы интеграции Экосистемы OSTIS с современными сервисами и информационными ресурсами}
{\label{sec_integration_common_principles}}

\section{Принципы интеграции Экосистемы OSTIS с разнородными сервисами}
{\label{sec_integration_services}} 

\begin{SCn}

    \bigskip
    
    \begin{scnrelfromlist}{ключевое понятие}
        \scnitem{...}
    \end{scnrelfromlist}
    
    \bigskip
    
    \begin{scnrelfromlist}{ключевое знание}
        \scnitem{...}
    \end{scnrelfromlist}
    
    \bigskip
    
    \begin{scnrelfromlist}{библиографическая ссылка}
        \scnitem{\scncite{...}}
    \end{scnrelfromlist}
    
\end{SCn}

В контексте интеграции Экосистемы OSTIS с разнородными сервисами, под "сервисами" понимаются приложения, программы, веб-сервисы и другие информационные системы, которые предоставляют определенный функционал, механизм преобразования данных в соответствии с заданной функцией. Зачастую такое приложение может предоставить программный интерфейс (API -- Application Programming Interface), который можно использовать с определённым форматом входов, которым будут соответствовать определённые форматы выходов.

Под интеграцией Экосистемы OSTIS с сервисом следует понимать возможность использовать функционал сервиса для изменения внутреннего состояния базы знаний системы. 

При интеграции сервисов в цифровых экосистемах возникают ряд проблем, которые могут затруднять процесс интеграции и уменьшать эффективность экосистемы. Некоторые из этих проблем могут включать в себя:

\begin{textitemize}
    \item Различные форматы данных и протоколы обмена, которые могут привести к ошибкам при обмене данными между сервисами, что затрудняет взаимодействие между сервисами;
    \item Несовместимость версий приложений, что может привести к конфликтам при обмене данными;
    \item Разные уровни безопасности, что может стать причиной утечки конфиденциальной информации;
    \item Отсутствие единой точки управления, что затрудняет мониторинг и управление процессами интеграции;
    \item Отсутствие механизмов для анализа и управления данными, что затрудняет контроль над процессами обмена данными.
\end{textitemize}

Перечисленные проблемы значительно усложняют разработку самих сервисов и ведёт к значительному увеличению временных и материальных затрат. Для решения этих проблем при интеграции цифровых экосистем с различными сервисами и ресурсами используются различные подходы и технологии. Некоторые из них могут включать в себя:

\begin{textitemize}
\item Использование стандартных протоколов и форматов обмена данных, таких как XML, JSON и другие, что позволяет сделать обмен данными более надежным и универсальным;
\item Разработка единой схемы данных и правил доступа, что позволяет сделать интеграцию более простой и управляемой;
\item Реализация механизмов для автоматической обработки ошибок и конфликтов, что позволяет снизить количество ошибок и улучшить надежность экосистемы;
\item Использование инструментов и технологий для анализа и управления данными, таких как системы бизнес-аналитики и управления данными, что позволяет контролировать процессы обмена данными и оптимизировать их работу.
\end{textitemize}


В рамках Экосистемы OSTIS выделены несколько уровней интеграции. 

Под полной интеграцией будет подразумеваться исполнение функции сервиса на платформонезависимом уровне и использованием языка SCP. То есть, задача интеграции такого сервиса сводится к выделению алогиртма обработки графовой конструкции и его реализации в рамках базы знаний системы. В результате такой интеграции отпадает надобность вовсе использовать сторонний сервис. 

Частичная интеграция означает изменение состояния базы знаний системы на этапах исполнения функции сервиса. Степень глубины интеграции может различаться. В некоторых случаях сервис может обращаться к базе знаний для получения дополнительной информации или для записи промежуточных результатов. В простейшем случае изменение базы знаний может происходить единожды, после получения результата отработки функции сервиса. 

Таким образом, миниммальными требованиями для интеграции сервиса являются:
\begin{textitemize}
    \item сформировать спецификацию входной конструкции в базе знаний системы;
    \item сформировать спецификацию выходной конструкции в базе знаний системы;
    \item реализовать агент обработки знаний, который преобразует конструкцию базы знаний в необходимые данные для запуска сервиса, а также погрузит результаты работы сервиса в базу знаний системы в соответствии со спецификацией.
\end{textitemize}

В рамках задачи формирования спецификаций необходимо выделить ключевые понятия, необходимых для формирования запросов к функциональному сервису, а также для погружения в базу знаний результатов.

Обобщённый алгоритм агента обработки знаний с использованием стороннего сервиса выглядит следующим образом:
\begin{textitemize}
    \item извлечение из базы знаний необходимых структур знаний;
    \item преобразование извлечённых знаний в формат, необходимый для подачи на вход сервиса;
    \item отправка запроса и ожидание ответа сервиса;
    \item формирование конструкции знаний по полученным данным;
    \item погружение новых знаний в базу знаний интеллектуальной системы.
\end{textitemize}

Рассмотрим процесс внедрения такого агента с точки зрения архитектуры Экосистемы. 

Одним из вариантов внедрение агента является его подключение в рамках уже существующей, основной ostis-системы. В таком случае все зависимости для установки сервиса будут также являться и зависимостями основной ostis-системы. Архитектурным паттерном такой системы будет являться монолитная архитектура. 

Преимуществом такого варианта является простота реализации и внедрения. Хорошим вариантом использования такой интеграции является отсутсвия возможности и необходимости изменить процесс выполнения функции сервиса, или же когда обращение к сервису производится по сети. Так могут быть интегрированы и сервисы получения знаний из внешних источников (получение прогноза погоды, обработка статистической информации, и т.д.), и функциональные сервисы (обработка аудиоинформации и погружение результатов обработки в базу знаний, получение синтаксического анализа предложения). 

Альтернативным вариантом внедрения является реализация отдельной ostis-системы, в рамках которой будет интегрирована функция сервиса, что характеризует переход к распределённым взаимодействующим ostis-системам. Архитектурным паттерном такой системы будет являться микросервисная архитектура.

Преимуществом такого варианта интеграции является распределённость, децентрализованность и доступность нового функционала. Система выходитза рамки технических ограничений, функционал может быть распределён на различном аппаратном обеспечении. Полученный функционал может быть использован различными ostis-системами в рамках Экосистемы для достижения своих целей. Минусами такой системы является сложность разработки, а также увеличение временных затрат на общение систем друг с другом по сетевым протоколам. 

Одним из вариантов использования является интеграция сервиса, которому для успешного выполнения своей функции необходимо взаимодействовать с базой знаний системы. Примерами такой интеграции могут служить сервисы считывания эмоционального состояния пользователя, или же помощи принятия решения на основе актуальных знаний.

\section{Принципы интеграции Экосистемы OSTIS со структурированными информационными ресурсами}
{\label{sec_integration_resources}} 

\begin{SCn}

    \bigskip
    
    \begin{scnrelfromlist}{ключевое понятие}
        \scnitem{...}
    \end{scnrelfromlist}
    
    \bigskip
    
    \begin{scnrelfromlist}{ключевое знание}
        \scnitem{...}
    \end{scnrelfromlist}
    
    \bigskip
    
    \begin{scnrelfromlist}{библиографическая ссылка}
        \scnitem{\scncite{...}}
    \end{scnrelfromlist}
    
    \end{SCn}

Проблема автоматизированного пополнения базы знаний не нова. Попытки ее решить предпринимались и ранее. Далее рассмотрены несколько подходов к автоматизированному пополнению БЗ.

Принципы интеграции Экосистемы OSTIS со структурированными информационными ресурсами основаны на пополнении базы знаний системы новыми знаниями. Данная проблема актуальна уже длительное время, и для ее решения были предложены различные подходы. Один из таких подходов - использование RDF (Resource Description Framework), который является моделью данных, предложенной консорциумом W3C.

Для успешной интеграции структурированных информационных ресурсов в Экосистему OSTIS, важно уделить должное внимание пониманию и применению принципов RDF-модели, поскольку они играют ключевую роль в организации связей между различными ресурсами. RDF используется для описания ресурсов в сети Интернет, и является основой для построения семантических веб-приложений, таких как Linked Open Data.

Основная структура абстрактного синтаксиса RDF – это тройка, состоящая из субъекта, предиката и объекта. Набор таких троек называется графом RDF. Граф RDF может быть визуализирован как диаграмма узла и направленной дуги, в которой каждая тройка представлена как связь «узел - дуга - узел».

Графы RDF атемпоральны, т.е. представляют собой статические снимки информации. Однако графы RDF могут выражать информацию о событиях и временных аспектах других сущностей, учитывая соответствующие термины из словаря. Поскольку графы RDF определены как математические наборы, добавление или удаление троек из графа RDF дает другой граф RDF.

Узел может иметь следующий тип:

1. IRI. Представляет собой короткую последовательность символов, идентифицирующую абстрактный или физический ресурс на любом языке мира. IRI представляе собой обобщение URI.

2. Литерал. Представляя собой структуру, состоящую из лексической формы (UNICODE-строка) и типа данных.

3. Пустой узел. Представляет собой локальный идентификатор, который используются в некоторых конкретных синтаксисах RDF или реализациях хранилища RDF.

RDF поддерживает основные типы данных, такие как строковый (string), логический (boolean), числовые (integer, double, float и др.), временные и некоторые другие.

В RDF существует такое понятие, как словарь RDF. Он представляет собой совокупность IRI, ссылающихся на другие графы с классами, литералами и др. Часто группа IRI может начинаться с одинакового префикса.

RDF нашел широкое применение. Так, например, RDF используется в оформлении БЗ в рамках различных проектов во множестве институтов, университетов и иных организаций. Поисковые системы предлагают веб-мастерам использовать RDF и аналогичные языки разметки страниц для повышения информативности ссылок на их сайты в результатах поиска. Социальные сети, с подачи Facebook, предлагают веб-мастерам использовать RDF для описания свойств страниц, так же позволяющих красиво оформить ссылку на неё в записи пользователя социальной сети;

Существующие подходы:
\begin{textitemize}
    \item R2RML – это стандарт W3C для выражения настраиваемых отображений из реляционных БД в RDF. Такие отображения предоставляют возможность просматривать существующие реляционные данные в модели данных RDF, выраженные в структуре и целевом словаре по выбору автора сопоставления;
    \item RML.io – это open-source проект, разрабатываемый с 2013 года. Данная технология предназначена для генерации БЗ на основе данных из полуструктурированных источников;
    \item «Озеро данных» – это централизованное хранилище, которое позволяет хранить все структурированные и неструктурированные данные в любом масштабе. «Семантическое озеро данных» – это особая форма озер данных, в которых верхний семантический слой обогащает и связывает данные семантически. Семантический уровень преодолевает разрозненность данных и обеспечивает семантический поиск по всем данным.
\end{textitemize}

Система наполнения БЗ будет реализована с использованием OSTIS-технологии.

Из достоинств OSTIS-систем можно выделить:
\begin{textitemize}
    \item способность осуществлять семантическую интеграцию знаний в своей;
    \item возможность интегрировать различные виды знаний;
    \item возможность интегрировать различные модели решения задач.
\end{textitemize}

Для решения задачи интеграции, использование OSTIS-системы целесообразно по многим причинам. Технология изначально предлагает инструменты для описания синтаксиса и семантики внешних языков. Данный инструментарий позволяет сократить время разработки в несколько раз.

Память OSTIS-системы представляет собой семантическую сеть, в основе которой лежит графовая структура. Все элементы семантической сети являются знаками различных сущностей. Такими сущностями могут быть всевозможные внешние описываемые объекты, а также различные множества, состоящие их элементов (атомарных фрагментов) этой же семантической сети.

Для интеграции системы нужно реализовать агент. Его работу можно разбить на следующие этапы:
1. интеграция с использованием готовых правил;
2. интеграция с сохранением исходной схемы;
3. дополнительные преобразования.

\textbf{Интеграция с использованием готовых правил}

На этом этапе ко всем сгенерированным тройкам применяются готовые правила интеграции, хранящиеся в БЗ. Создание и применение подобных правил необходимо в ситуациях, когда способ представления конкретного знания по какой-то причине не соответвествует представлению аналогичного знания в OSTIS-системе.

\textbf{Интеграция с сохранением исходной схемы}

На данном этапе оставшиеся тройки будут преобразованы с сохранением той структуры отношения, в которой находились участвовавшие в нем сущности. Это значит, что порядок элементов в итоговой конструкции будет аналогичен порядку сущностей в исходной.

\textbf{Дополнительные преобразования}

На данном этапе проходят оставшиеся интеграционные преобразования, которым не нашлось места в предыдущих пунктах, но которые необходимы для завершения процесса интеграции.

Для выгрузки информации из БЗ можно использовать те же правила, что и для загрузки, так как в основном они содержат эквиваленцию. То есть изначально производится поиск необходимых кострукций, затем они прогоняются через правило, и на выходе мы получаем тройки. В дальнейшем данные тройки преобразуются в интересующий нас формат.

Таким образом, мы получаем полностью интегрированную систему, которая способна отвечать всем запросам пользователя: загружать информацию, производить над знаниями полезную работу, и затем выдать эти знания в заданном формате.

\section{Интеграция инструментов компьютерной алгебры в приложения OSTIS}
{\label{sec_integration_algebra}} 


%\input{author/references}