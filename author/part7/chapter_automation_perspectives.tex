\chapter{Структура и проблемы организации человеческой деятельности}
\chapauthortoc{Голенков В.В.\\Гулякина Н.А.}
\label{chapter_automation_perspectives}

\vspace{-7\baselineskip}

\begin{SCn}
\begin{scnrelfromlist}{автор}
	\scnitem{Голенков В.В.}
	\scnitem{Гулякина Н.А.}	
\end{scnrelfromlist}

\bigskip

\scntext{аннотация}{В главе рассмотрены принципы автоматизации различных областей человеческой деятельности с использованием интеллектуальных компьютерных систем нового поколения. Предлагается онтология различных видов деятельности и соответствующих технологий. Детализация указанных принципов осуществляется на примере человеческой деятельности в области Искусственного интеллекта.}

\bigskip

\begin{scnrelfromlist}{подраздел}
	\scnitem{\ref{sec_activity_concepts}~\nameref{sec_activity_concepts}}
	\scnitem{\ref{sec_activity_ai}~\nameref{sec_activity_ai}}
	\scnitem{\ref{sec_activity_types}~\nameref{sec_activity_types}}
	\scnitem{\ref{sec_activity_perspectives}~\nameref{sec_activity_perspectives}}
\end{scnrelfromlist}

\end{SCn}

\section*{Введение в Главу \ref{chapter_automation_perspectives}}

Ключевая проблема современного уровня комплексной автоматизации \textit{человеческой деятельности} заключается в следующем. В настоящее время осуществляется либо полная автоматизация некоторых классов действий, инициируемых соответствующими командами, либо частичная автоматизация некоторых видов \textit{человеческой деятельности}, в рамках которой человек осуществляет управление соответствующими средствами автоматизации. При этом автоматизация решения комплексных задач, которые сводятся к нескольким \underline{частично} автоматизированным подзадачам, требует "ручного"{} (неавтоматизированного) управления одновременно \underline{несколькими} средствами автоматизации.

Принципы, лежащие в основе перехода к более высокому уровню автоматизации \textit{человеческой деятельности} сводятся к следующему. Все средства автоматизации (все сервисы), управляемые в настоящее время людьми, переходят "под управление"{} \textbf{\textit{интероперабельными интеллектуальными компьютерными системами}}, которые способны эффективно взаимодействовать между собой и, соответственно, способны полностью автоматизировать решение комплексных задач, требующих использования нескольких средств автоматизации (сервисов) \scncite{Yaghoobirafi2022,Ouksel1999,Lanzenberger2008,Neiva2016,Pohl2004,Waters2009}.

В рамках данной работы сначала уточним основные понятия, используемые нами для рассмотрения структуры \textit{человеческой деятельности}, потом подробно рассмотрим структуру текущего состояния и проблемы развития \textit{Человеческой деятельности в области Искусственного интеллекта}. Далее обобщим принципы организации и автоматизации \textit{Человеческой в области Искусственного интеллекта} на все многообразие видов и областей \textit{человеческой деятельности}.

\section{Основные понятия, лежащие в основе формального описания структуры человеческой деятельности}
\label{sec_activity_concepts}

\begin{SCn}
	\scnheader{деятельность}
	\scnidtf{процесс ситуативного воздействия на некоторую динамическую систему, направленного либо на создание этой системы, либо на поддержание определённых характеристик этой системы, либо на её разрушение, либо на ее развитие (совершенствование)}
	\scnidtf{система всех действий, выполняемых некоторым индивидуальным или коллективным \underline{субъектом} либо целостная подсистема таких действий, соответствующая назначению (обязанностям) этого субъекта}
	
	
	\scnheader{следует отличать*}
	\begin{scnhaselementset}
		\scnitem{деятельность}
		\scnitem{действие}
	\end{scnhaselementset}
	\scntext{сравнение}{Каждому \textit{действию} обязательно соответствует формулировка задачи, которая решается в результате выполнения этого действия. При этом \textit{действие} может быть сложным (неатомарным), то есть представлять собой иерархическую систему поддействий, обеспечивающих решение \textit{подзадач} исходной \textit{задачи}. 
		
		В отличие от этого каждой \textit{деятельности} может соответствовать несколько исходных задач, не являющихся под задачами других задач, решаемых в рамках этой \textit{деятельности}. 
		
		Пример такой \textit{деятельности}, которая не является действием, -- это деятельность, осуществляемая некоторым субъектом в рамках некоторой \textit{предметной области} (имеется в виду \textit{деятельность} по решению всевозможных задач самого различного вида, которые могут быть сформулированы в рамках указанной \textit{предметной области}).}
	\scntext{примечание}{Каждой деятельности и, соответственно, каждому действию однозначно соответствует субъект, выполняющий эту деятельность.}
	
	\scnheader{человеческая деятельность}
	\scnidtf{система действий, совершаемых людьми или человеческими сообществами либо "вручную"{}, либо с помощью "пассивных"{} инструментов (палок, веревок, лопат, топоров, ...), либо с помощью "активных"{} инструментов (транспортных средств, элеваторов, бензопил, ...)}
	
	\scnheader{действие}
	\scnidtf{процесс изменения состояния некоторой динамической системы (из заданного состояния в требуемое целевое состояние), инициированный и, возможно, непосредственно осуществляемый некоторым субъектом с возможным использованием некоторых инструментов, дополнительных материалов и информационных ресурсов (в частности, методов)}
	\scnsuperset{информационное действие}
	\begin{scnindent}
		\scnidtf{информационный процесс, выполняемый некоторым субъектом (в том числе процессором компьютера)}
		\scnidtf{процесс изменения состояния некоторого информационного ресурса}
	\end{scnindent}
	
	\scnheader{следует отличать*}
	\begin{scnhaselementset}
		\scnitem{действие}
		\scnitem{задача}
		\begin{scnindent}
			\scnidtf{спецификация некоторого действия, содержащая достаточную информацию для выполнения этого действия}
			\scnidtf{формулировка задачи}
			\scnsuperset{декларативная формулировка задачи}
			\scnsuperset{процедурная формулировка задачи}
		\end{scnindent}
	\end{scnhaselementset}
	
	\scnheader{действие}
	\scnidtf{целенаправленный (осознанный) процесс воздействия некоторого субъекта на некоторый объект}
	\scntext{примечание}{Субъект действия может быть как индивидуальным, так и коллективным. Объект воздействия может иметь сколь угодно сложную структуру и состоять из любого количества компонентов, на которые оказывается воздействие. Инициирование и выполнение действия может осуществляться разными субъектами с использованием различных вспомогательных инструментов.}
	
	\scnheader{задача}
	\scnidtf{формулировка задачи}
	\scnidtf{спецификация некоторого действия, позволяющая осуществлять поиск метода выполнения этого действия}
	
	\scnheader{отношение, заданное на множестве действий*}
	\scnhaselement{действие*}
	\begin{scnindent}
		\scnidtf{быть действием, выполняемым для решения заданной задачи}
		\scnrelboth{обратное отношение}{задача*}
		\begin{scnindent}
			\scnidtf{быть задачей, решаемой в результате выполнения данного действия*}
		\end{scnindent}
	\end{scnindent}
	
	\scnheader{отношение, заданное на множестве деятельностей}
	\scnhaselement{объект деятельности*}
	\begin{scnindent}
		\scnidtf{быть объектом, над которым выполняется заданная деятельность*}
	\end{scnindent}
	\scnhaselement{продукт деятельности*}
	\begin{scnindent}
		\scnidtf{быть продуктом заданной деятельности*}
	\end{scnindent}
	\scnhaselement{отрезок времени существования*}
	\begin{scnindent}
		\scnidtf{быть отрезком времени существования данной темпоральной сущности*}
	\end{scnindent}
	\scnhaselement{деятельности, классы объектов которых совпадают*}
	\scnhaselement{деятельности, выполняемые одновременно*}
	
	\scnheader{вид деятельности}
	\scnidtf{класс деятельностей}
	\scnidtf{класс аналогичных деятельностей, которому можно поставить в соответствие общую технологию, обеспечивающую выполнение всех деятельностей данного класса}
	\scnidtf{класс, экземплярами (элементами) которого являются эквивалентные (похожие) деятельности, выполняемые в общем случае разными кибернетическими системами}
	\scnhaselementrole{пример}{проектирование}
	\begin{scnindent}
		\scnidtf{проектная деятельность}
		\scnidtf{человеческая деятельность, продуктом которой является проектная документация некоторой создаваемой сущности}
		\scnidtf{построение полной спецификации (проектной документации) некоторой создаваемой сущности}
		\scntext{примечание}{Продуктами деятельности для этого вида деятельности являются спецификации любых социально значимых создаваемых сущностей.}
		\begin{scnrelfromlist}{частный вид деятельности над подклассом объектов деятельности}
			\scnitem{проектирование ostis-систем}
			\begin{scnindent}
				\begin{scnrelfromlist}{частный вид деятельности над подклассом объектов деятельности}
					\scnitem{проектирование обучающих ostis-систем}
					\scnitem{проектирование ostis-систем медицинской диагностики}
				\end{scnrelfromlist}
			\end{scnindent}
			\scnitem{проектирование микросхем}
			\scnitem{проектирование зданий}
		\end{scnrelfromlist}
	\end{scnindent}
	\scnhaselementrole{пример}{поддержка жизненного цикла}
	\begin{scnindent}
		\begin{scnrelfromlist}{частный вид деятельности над подклассом объектов деятельности}
			\scnitem{поддержка жизненного цикла микросхем}
			\scnitem{поддержка жизненного цикла зданий}
			\scnitem{поддержка жизненного цикла ostis-систем}
			\begin{scnindent}
				\begin{scnrelfromlist}{частный вид деятельности над подклассом объектов деятельности}
					\scnitem{поддержка жизненного цикла баз знаний ostis-систем}
					\scnitem{поддержка жизненного цикла решателей задач ostis-систем}
					\scnitem{поддержка жизненного цикла интерфейсов ostis-систем}
				\end{scnrelfromlist}
			\end{scnindent}
		\end{scnrelfromlist}
	\end{scnindent}
	\scnhaselement{подготовка кадровых ресурсов}
	\begin{scnindent}
		\scnidtf{образовательная деятельность}
	\end{scnindent}
	\scnhaselement{производство}
	\begin{scnindent}
		\scnidtf{производственная деятельность}
	\end{scnindent}
	\scnhaselement{природоохранная деятельность}
	\scnhaselement{строительная деятельность}
	\scnhaselement{здравоохранительная деятельность}
	\scnhaselement{административная деятельность}
	\scnhaselement{научно-исследовательская деятельность}
	
	\scnheader{следует отличать*}
	\begin{scnhaselementset}
		\scnitem{вид деятельности}
		\scnitem{класс действий}
	\end{scnhaselementset}
	
	\scnheader{следует отличать*}
	\begin{scnhaselementset}
		\scnitem{класс действий}
		\scnitem{класс задач}
		\begin{scnindent}
			\scnidtf{множество аналогичных (похожих) формулировок конкретных задач, которые легко обобщить, заменяя некоторые константы, ходящие в эти формулировки, на переменные}
		\end{scnindent}
		\scnitem{формулировка класса задач}
		\begin{scnindent}
			\scnidtf{обобщенная формулировка (спецификация) класса задач, "превращающаяся"{} в формулировку конкретной задачи при присваивании конкретных значений всем свободным переменным, входящим в эту обобщённую формулировку класса задач}
		\end{scnindent}
	\end{scnhaselementset}
	
	\scnheader{отношение, заданное на множестве действий}
	\scnhaselement{метод*}
	\begin{scnindent}
		\scnidtf{быть методом, обеспечивающим выполнение всех действий заданного класса действий, или решение всех задач заданного класса задач, или выполнение конкретного заданного действия, или решение конкретной заданной задачи*}
	\end{scnindent}
	\scnhaselement{максимальный класс действий*}
	\begin{scnindent}
		\scnidtf{быть максимальным классом действий, выполняемых с помощью заданного метода*}
	\end{scnindent}
	
	\scnheader{метод}
	\scnidtf{информационная конструкция, интерпретация которой любым субъектом, принадлежащим соответствующему классу субъектов, обеспечивает выполнение любого действия, принадлежащего соответствующему классу действий}
	\scnsuperset{методика}
	\begin{scnindent}
		\scnidtf{метод, реализуемый человеком или коллективом людей}
	\end{scnindent}
	
	
	\scnheader{отношение, заданное на множестве видов деятельности}
	\scnhaselement{технология*}
	\begin{scnindent}
		\scnidtf{быть технологией, обеспечивающей выполнение каждой деятельности, принадлежащей заданному виду деятельности*}
	\end{scnindent}
	\scnhaselement{класс объектов деятельности*}
	\begin{scnindent}
		\scnidtf{быть классом объектов деятельности для заданного вида деятельности*}
	\end{scnindent}
	\scnhaselement{класс продуктов деятельности*}
	\begin{scnindent}
		\scnidtf{быть классом продуктов деятельности для заданного вида деятельности*}
	\end{scnindent}
	\scnhaselement{частный вид деятельности над подклассом объектов деятельности*}
	\begin{scnindent}
		\scnidtf{быть технологией, обеспечивающей выполнение каждой деятельности, принадлежащей заданному виду деятельности*}
	\end{scnindent}
	\scnhaselement{частный вид деятельности над классом компонентов объектов деятельности*}
	\begin{scnindent}
		\scnidtf{частный вид деятельности, классом объектов которого является является класс компонентов объектов деятельности заданного вида деятельности*}
	\end{scnindent}
	\scnhaselement{частный вид одновременно выполняемой деятельности*}
	\begin{scnindent}
		\scnidtf{частный вид деятельности, каждая из которых выполняется одновременно (параллельно) с деятельностями, принадлежащими заданному виду деятельности*}
	\end{scnindent}
	\scnhaselement{частный вид деятельности, выполняемой на некотором этапе*}
	\begin{scnindent}
		\scnidtf{частный вид деятельности, каждая из которых выполняется в качестве одного из этапов выполнения деятельности, принадлежащих заданному виду деятельности*}
	\end{scnindent}
	\scnhaselement{виды деятельности, классы объектов которых совпадают*}
	\scnhaselement{виды деятельности, выполняемые одними и теми же субъектами*}
	
	\scnheader{технология}
	\scnidtf{система \underline{знаний} и \underline{навыков}, обеспечивающих выполнение деятельности соответствующего вида соответствующими субъектами}
	\scntext{примечание}{В основе любой \textit{технологии} лежит \underline{многократность} -- либо многократность создания \underline{похожих} (аналогичных) сущностей, либо многократность выполнения \underline{похожих} действий, многократность использования похожих методов (способов, методик). Очевидно, что, чем выше степень сходства (близости, конвергентности) многократно создаваемых сущностей и многократно используемых методов, чем выше их \underline{унификация}, тем проще будет соответствующая \textit{технология}.}
	
	\scnheader{область выполнения*}
	\scnsuperset{область выполнения действия*}
	\scnsuperset{область выполнения класса действий*}
	\scnsuperset{область выполнения деятельности*}
	\scnsuperset{область выполнения вида деятельности*}
	\scnsuperset{быть предметной областью и соответствующей онтологией возможно с некоторыми дочерними предметными областями и соответствующими онтологиями, содержащей знания, достаточные для выполнения заданного действия, или заданного класса действий, или заданной деятельности, или заданного вида деятельности*}
	
	\scnheader{субъект*}
	\scnidtf{быть субъектом, выполняющим заданное действие или выполняющим заданную деятельность*}
	\scntext{примечание}{Инициирование действия или деятельности, выполняемой субъектом, может осуществляться либо самостоятельно (по собственной инициативе), либо по инициативе (команде, заданию) другого субъекта.}
	
	\scnheader{информационный ресурс}
	\scnidtf{социально-значимая информационная конструкция, являющаяся \underline{продуктом} соответствующей человеческой деятельности, который требует не только создания, но и сопровождения (обновления, актуализации)}
	\scnsuperset{проектная документация}
	\scnsuperset{научная теория}
	\begin{scnindent}
		\scnidtf{продукт научно-исследовательской деятельности, являющийся строгим описанием свойств и закономерностей некоторого класса сущностей}
	\end{scnindent}
	\scnsuperset{стандарт}
	\scnsuperset{спецификация комплекса методик соответствующей технологии}
	\scnsuperset{спецификация инструментальных средств соответствующей технологии}
	
	\scnheader{проектная документация}
	\scnidtf{полная спецификация создаваемой сущности}
	\scnidtf{продукт проектной деятельности}
	\scnidtf{"цифровая"{} копия сущности}
	\scnidtf{информационная модель (описание) сущности, обладающая достаточной полнотой (детализацией) для воспроизведения (воспроизводства) этой сущности}
\end{SCn}

\section{Структура и проблемы организации человеческой деятельности в области Искусственного интеллекта}
\label{sec_activity_ai}

В предыдущих главах мы уточнили:

\begin{textitemize}
	\item
	архитектуру \textit{интеллектуальных компьютерных систем нового поколения}
	\item
	то, как осуществляется деятельность (функционирование) \textit{интеллектуальных компьютерных систем нового поколения}
	\item
	то, как автоматизируется \textit{прикладная инженерная человеческая деятельность по проектированию интеллектуальных компьютерных систем нового поколения} и поддержке всех последующих этапов их жизненного цикла.
\end{textitemize}

Уточним то, как осуществляется и автоматизируется \underline{весь комплекс} \textit{человеческой деятельности в области Искусственного интеллекта}.

\subsection{Общая оценка современного состояния человеческой деятельности в области Искусственного интеллекта}

Рассмотрим необходимость перехода организации \textbf{человеческой деятельности} \textbf{\textit{Искусственного интеллекта}} на принципиально новый уровень, обеспечивающий формирование рынка \textbf{семантически совместимых} \textbf{\textit{интеллектуальных компьютерных систем}} \textbf{\textit{нового поколения}}, разрабатываемых на основе принципиально нового комплекса \textit{семантически совместимых} \textbf{\textit{технологий Искусственного интеллекта}}.

Сейчас актуально исследовать не только \textit{модели решения интеллектуальных задач} в интеллектуальных компьютерных системах различного вида, но также методологические проблемы текущего состояния \textit{Искусственного интеллекта} в целом и пути решения этих проблем.

Анализ современного состояния работ в области \textit{Искусственного интеллекта} показывает то, что указанная научно-техническая дисциплина находится в серьезном методологическом кризисе. Поэтому необходимо:

\begin{textitemize}
	\item
	Выявление основных причин возникновения указанного кризиса;
	\item
	Уточнение основных мер, направленных на его устранение.
\end{textitemize}

Решение рассматриваемых кризисных проблем требует:

\begin{textitemize}
	\item
	Существенного фундаментального общесистемного переосмысления всего того, \textit{что} мы делаем в области \textit{Искусственного интеллекта} и \textit{как} мы это делаем, т.е. требует уточнения характеристик \textit{интеллектуальных компьютерных систем}, уточнения понятия сообщества, состоящего из \textit{интеллектуальных компьютерных систем} и взаимодействующих с ними пользователей, уточнения требований, предъявляемых к \textit{интеллектуальным компьютерным системам}, а также уточнения методик и средств их создания и использования.
	\item
	Осознания того, что \textit{Кибернетика, Информатика} и \textit{Искусственный интеллект} -- это общая фундаментальная наука, требующая комплексного подхода к построению общих формальных моделей систем, основанных на обработке информации (\textit{кибернетических систем}), путём \textit{конвергенции} и \textit{интеграции} формальных моделей различных компонентов этих систем \scncite{Yankovskaya_2017,Palagin2013}. Таким образом, современный этап развития \textit{Искусственного интеллекта} -- это переход от накопленного к текущему моменту многообразия моделей решения различного вида задач к преобразованию этого многообразия в стройную систему \textit{семантически совместимых} моделей;
	\item
	Осознания того, что сейчас требуется не расширять многообразие точек зрения, а учиться их согласовывать, обеспечивать их \textit{семантическую совместимость}, совершенствуя соответствующие методы.
\end{textitemize}

Обсуждая современную проблематику \textit{конвергенции} различных моделей в области \textit{Искусственного интеллекта} и построения интегрированных \textit{гибридных моделей}, уместно вспомнить <<фантастический рассказ Д.А. Поспелова ``Соприкосновение'', посвященный контакту различных миров. В нем главный герой популярно излагает свою теорию \textit{концептуальных разломов} <...>. Эта теория напоминает историю долгого периода \underline{дифференциации} наук, когда различные научные дисциплины развивались \underline{независимо}, словно параллельные миры, лишь изредка соприкасаясь друг с другом, а отдельные ученые, получая все более узкую специализацию, мало что знали о достижениях даже своих "близких собратьев". К счастью, в последние годы все чаще и чаще возникают новые области контакта между отдельными дисциплинами, происходит взаимопроникновение идей, установление \underline{аналогий} между полученными результатами и тенденциями развития. Во многом это объясняется появлением и широким внедрением во все сферы жизни общества передовых информационных и коммуникационных технологий <...>. Современные технологии опираются на достижения многих научно-технических дисциплин, среди которых на первый план выходят \underline{синтетические науки нового поколения} -- науки об искусственном>>.
\begin{SCn}
	\scnrelto{цитата}{\scncite{Tarasov2002}/с.13}
\end{SCn}

Анализируя современное состояние работ в области \textit{Искусственного интеллекта (ИИ)}, следует констатировать то, что \textit{концептуальный разлом} между различными направлениями \textit{Искусственного интеллекта} является очевидным фактом. Это подтверждается следующей цитатой из книги В.Б. Тарасова \scncite{Tarasov2002} <<вновь, как и на заре ИИ, актуальными становятся формирование единых методологических основ ИИ, разработка теоретических проблем создания интеллектуальных систем новых поколений, развитие нетрадиционных аппаратно-программных средств. Здесь большие перспективы связаны с использованием идей и принципов синергетики в ИИ. Сам термин "синергетика"{} происходит от слова "синергия"{}, означающего совместное действие, сотрудничество. По мнению "отца синергетики"{} Г.Хакена, такое название вполне подходит для современной теории сложных самоорганизующихся систем по двум причинам: а) исследуются совместные действия многих элементов развивающейся системы; б) для отыскания общих принципов самоорганизации требуется объединение усилий представителей различных дисциплин>>.
\begin{SCn}
	\scnrelto{цитата}{\scncite{Tarasov2002}/с.14}
\end{SCn}

Для того, чтобы убедиться в наличии \textit{концептуального разлома} между различными направлениями \textit{Искусственного интеллекта}, достаточно просто перечислить основные направления работы конференций по тематике \textit{Искусственного интеллекта}, обращая внимание на то, что многие из них развиваются независимо от других:

\begin{textitemize}
	\item
	синергетические модели самоорганизации интеллектуальных компьютерных систем;
	\item
	гибридные интеллектуальные компьютерные системы;
	\item
	коллаборативные интеллектуальные компьютерные системы;
	\item
	мягкие вычисления, интеллектуальные вычисления;
	\item
	моделирование не-факторов;
	\item
	неклассические, многозначные, модальные, псевдофизические, индуктивные, нечеткие логики и приближенные рассуждения, логические программы;
	\item
	нечеткие множества, отношения, графы, алгоритмы;
	\item
	функциональные программы, нечеткие алгоритмы, генетические алгоритмы, продукционные модели;
	\item
	нейросетевые модели;
	\item
	параллельные асинхронные модели децентрализованного решения задач;
	\item
	обработка сигналов;
	\item
	мультисенсорная конвергенция, сенсо-моторная координация;
	\item
	модели ситуационного управления.
\end{textitemize}

Преодоление \textit{концептуального разлома} между различными направлениями исследований в области \textit{Искусственного интеллекта} -- это, своего рода "прыжок"{} через "концептуальную пропасть"{}, который требует особой концентрации усилий. Через пропасть нельзя перепрыгнуть двумя прыжками.

Если кратко охарактеризовать \textbf{текущее состояние} всего комплекса работ в области \textbf{\textit{Искусственного интеллекта}}, то это -- \textbf{иллюзия благополучия}. Происходит активное \underline{локальное} развитие самых различных направлений \textit{Искусственного интеллекта} (\textit{неклассические логики}, \textit{формальные онтологии}, \textit{искусственные нейронные сети}, \textit{машинное обучение}, \textit{мягкие вычисления}, \textit{многоагентные системы} и др.), но \underline{комплексного} повышения уровня \textit{интеллекта} современных \textit{интеллектуальных компьютерных систем} не происходит. Для этого прежде всего требуется сближение и \textit{интеграция} \underline{всех} направлений \textit{Искусственного интеллекта} и соответствующее построение \textbf{\textit{Общей формальной теории интеллектуальных компьютерных систем}}, а также превращение современного многообразия \textit{инструментальных средств} (frameworks) разработки различных компонентов \textit{интеллектуальных компьютерных систем} в единую \textbf{\textit{Технологию комплексного проектирования и поддержки всего жизненного цикла интеллектуальных компьютерных систем}}, гарантирующую \underline{совместимость} всех разрабатываемых компонентов \textit{интеллектуальных компьютерных систем}, а также совместимость самих \textit{интеллектуальных компьютерных систем} как \underline{самостоятельных} субъектов (агентов, акторов), взаимодействующих между собой в рамках комплексных систем автоматизации сложных видов коллективной \textit{человеческой деятельности} (умных домов, умных больниц, умных школ, умных производственных предприятий, умных городов и т.д.). Таким образом, эпиграфом текущего состояния работ в области \textit{Искусственного интеллекта} является известное высказывание из Экклезиаста: ``Время разбрасывать камни и время собирать камни -- всему своё время''.

<<К сожалению, в современных дискуссиях по теме ИИ (Искусственного интеллекта) научные споры часто подменяются завышенными ожиданиями от скорого внедрения ИИ и значительным сужением темы ИИ, которая оказалась сведена лишь к \textit{машинному обучению} на основе \textit{искусственных нейронных сетей}. <...> При этом за бортом Национальной стратегии пока остались \textit{онтологии}, \textit{базы знаний}, \textit{методы рассуждений} и \textit{принятия решений}, \textit{методы синтеза и} \textit{анализа сложных конструкций}, умные кибер-физические системы, \textit{цифровые двойники}, \textit{автономные системы}, системы анализа как "больших"{}, так и "малых"{} данных. <...>

Признавая всю важность \textit{машинного обучения} на базе \textit{искусственных} \textit{нейронных сетей}, научные и практические результаты мирового уровня следует искать на стыке разных дисциплин в \textbf{\textit{конвергенции}} различных технологий ИИ и \textbf{\textit{интеграции}} полипредметных \textit{знаний}. В этой связи формализация \textit{знаний} в виде \textit{онтологий} и \textit{баз знаний} в рамках \textit{Semantic Web} рассматривается как одно из фундаментальных направлений для создания ИИ. Действительно, какой же может быть \textit{интеллект} без использования \textit{знаний} современных учебников, на основе чего ИИ будет понимать \textit{контекст ситуации}, делать \textit{выводы} и \textit{принимать решения}? <...>

Еще одной ключевой сферой ИИ, не нашедшей отражения в Российской стратегии по ИИ, является \textit{распределенное принятие решений}, которое все больше становится коллективным для стремительно развивающихся систем умного Интернета вещей и автономных систем управления, начиная с беспилотных автомобилей, самолетов, кораблей и т.д.

Компанией Гартнер 2020 год был объявлен годом "автономных вещей"{}, которые по мнению компании уже прошли большую эволюцию от "цифровых"{} к "умным"{}. Ожидается, что на следующем этапе автономные вещи, обладающие собственным ИИ, "заговорят"{} друг с другом и в научную повестку войдут вопросы \textbf{\textit{семантической интероперабельности}} систем ИИ, которые будут не только обмениваться данными, но и вести переговоры для согласования решений. Дорожная карта научных исследований по ИИ США в качестве ключевых выделяет такие направления, как \textit{связность} систем \textit{Искусственного интеллекта} (Integrated Intelligence) и их \textit{осмысленное взаимодействие} (Meaningful Interaction), наряду с разными видами \textit{самообучения} в системах (Self-Aware Learning).>>
\begin{SCn}
	\scnrelto{цитата}{\scncite{Barinov2021}/с. 264-265}
\end{SCn}

\textbf{Ключевой причиной} \textbf{методологических проблем} современного состояния \textit{Искусственного интеллекта} и серьёзным вызовом для специалистов в этой области является проклятие \textbf{\textit{Вавилонского столпотворения}} \scncite{Illiadis2019}, которое преследует нас на всех уровнях:

\begin{textitemize}
	\item
	
	на уровне внутренней организации \textit{решения задач} в \textit{интеллектуальных компьютерных системах;}
	
	\item
	
	на уровне взаимодействия \textit{интеллектуальных компьютерных систем} как между собой, так и с пользователями;
	
	\item
	
	на уровне взаимодействия учёных, работающих в области \textit{Искусственного интеллекта}, что препятствует созданию \textit{Общей формальной теории и стандарта интеллектуальных компьютерных систем}, а также \textit{Технологии комплексного проектирования и поддержки всего жизненного цикла интеллектуальных компьютерных систем}
	
	\item
	
	на уровне взаимодействия между учёными, инженерами, разрабатывающими прикладные \textit{интеллектуальные компьютерные системы}, преподавателями ВУЗов, которые готовят специалистов в области \textit{Искусственного интеллекта}, а также студентами, магистрантами и аспирантами.
	
\end{textitemize}

Сложность разрабатываемых в настоящее время \textit{интеллектуальных компьютерных систем} и технологий Искусственного интеллекта достигла такого уровня, что для их разработки требуются не просто большие творческие коллективы, но и существенное повышение квалификации и качества этих коллективов. Как известно, квалификация коллектива разработчиков определяется не только квалификацией его членов, но также эффективностью и атмосферой их взаимодействия. Известно также, что качество любой технической системы является отражением качества того коллектива, который эту систему разработал. Может ли коллектив достаточно квалифицированных специалистов, многие из которых не обладают высоким уровнем интероперабельности, разработать интеллектуальную компьютерную систему с высоким уровнем интероперабельности, а тем более технологию комплексной поддержки всего жизненного цикла интеллектуальных компьютерных систем такого уровня? Очевидный ответ на этот вопрос и очевидная сложность создания работоспособных творческих коллективов указывают на основной вызов, адресованный специалистам в области Искусственного интеллекта в настоящее время. Таким образом, требования, предъявляемые к интеллектуальным компьютерным системам нового поколения и определяющие их способность к индивидуальному и коллективному решению комплексных сложных задач, должны предъявляться и к разработчикам этих систем, а также к разработчикам любых других сложных объектов, поскольку все сложные виды и области человеческой деятельности являются коллективными и творческими.

Создание быстро развивающегося рынка семантически совместимых \textit{интеллектуальных компьютерных систем} -- это основная цель, адресованная специалистам в области Искусственного интеллекта, требующая преодоления \textbf{\textit{Вавилонского столпотворения}} во всех его проявлениях, формирования высокой культуры договороспособности и унифицированной, согласованной формы представления коллективно накапливаемых, совершенствуемых и используемых знаний. Ученые, работающие в области \textit{Искусственного интеллекта}, должны обеспечить \textbf{\textit{конвергенцию}} результатов различных направлений \textit{Искусственного интеллекта} и построить \textit{Общую формальную теорию интеллектуальных компьютерных систем}, а также \textit{Комплексную технологию проектирования семантически совместимых интеллектуальных компьютерных систем,} включающую соответствующие стандарты \textit{интеллектуальных компьютерных систем} и их компонентов. Инженеры, \textit{разрабатывающие прикладные интеллектуальные компьютерные системы}, должны сотрудничать с учеными и участвовать в развитии \textit{Комплексной технологии проектирования семантически совместимых интеллектуальных компьютерных систем}, и поддержки всех последующих этапов жизненного цикла этих систем.

Разобщенность различных направлений исследований в области \textit{Искусственного интеллекта} является главным препятствием создания \textit{Комплексной технологии проектирования семантически совместимых интеллектуальных компьютерных систем}, а также \textit{Технологии комплексной поддержки} всех последующих этапов жизненного цикла \textit{интеллектуальных компьютерных систем}.

\subsection{Структура деятельности в области \textit{Искусственного интеллекта}}

Для того, чтобы рассмотреть проблемы дальнейшего развития \textit{деятельности} в области \textit{Искусственного интеллекта} и, в частности, проблемы комплексной автоматизации этой \textit{деятельности}, необходимо уточнить структуру указанной \textit{деятельности}.

Человеческая деятельность в области \textit{Искусственного интеллекта} направлена на исследование и создание \textit{интеллектуальных компьютерных систем} различного вида и различного назначения. Объектами исследования в области \textit{Искусственного интеллекта} являются:

\begin{textitemize}
	\item
	\textit{индивидуальные интеллектуальные компьютерные системы} (в частности, когнитивные агенты);
	\item
	\textit{многоагентные интеллектуальные компьютерные системы} (в частности, сообщества, состоящие из \textit{индивидуальных интеллектуальных компьютерных систем});
	\item
	человеко-машинные сообщества, состоящие из \textit{интеллектуальных компьютерных систем} и их пользователей.
\end{textitemize}

Основными целями человеческой деятельности в области \textit{Искусственного интеллекта} являются:

\begin{textitemize}
	\item
	Построение формальной теории \textit{интеллектуальных компьютерных систем} (искусственных \textit{интеллектуальных систем});
	\item
	Создание технологий (методик и средств), обеспечивающих \textit{проектирование, реализацию, сопровождение и эксплуатацию интеллектуальных компьютерных систем};
	\item
	Переход на принципиально новый уровень комплексной автоматизации всевозможных \textit{видов человеческой деятельности}, который основан на массовом применение \textit{интеллектуальных компьютерных систем} и который предполагает:
	\begin{textitemize}
		\item
		не только наличие \textit{интеллектуальных компьютерных систем,} способных понимать друг друга и согласовывать свою деятельность,
		\item
		но и учёт общей структуры \textit{человеческой деятельности,} осуществляемой в условиях нового уровня её автоматизации (деятельности smart-общества), которая должна быть "понятна" используемым \textit{интеллектуальным компьютерным системам} и которая потребует существенного переосмысления современной организации \textit{человеческой деятельности}.
	\end{textitemize}
\end{textitemize}

\textbf{\textit{Искусственный интеллект}} как \textit{область} \textit{человеческой} \textit{деятельности} включает в себя следующие \textit{направления деятельности}:

\begin{textitemize}
	\item
	\textbf{\textit{Научно-исследовательскую деятельность в области Искусственного интеллекта}},\\
	в процессе которой осуществляется конкуренция различных точек зрения и подходов к построению формальных моделей различных компонентов \textit{интеллектуальных компьютерных систем}. Конечной целью такой деятельности является постоянно развиваемая \textit{Общая теория} \textit{интеллектуальных} \textit{компьютерных систем}, объектами исследования которой являются \textit{интеллектуальные компьютерные системы} и их формальные \textit{логико-семантические модели}, включающие в себя формальные модели различного вида \textit{знаний}, входящих в состав \textit{баз знаний} интеллектуальных компьютерных систем, а также различные \textit{модели решения задач} (логические модели различного вида, нейросетевые, генетические, продукционные, функциональные и др.).
	\item
	\textbf{\textit{Разработку Стандарта интеллектуальных компьютерных систем}}, включающую в себя перманентную эволюцию этого стандарта и поддержку целостности каждой его версии. Текущая версия \textit{Стандарта интеллектуальных компьютерных систем} -- это \underline{согласованная} (общепризнанная) \underline{на текущий момент} часть \textit{Общей теории интеллектуальных компьютерных систем.}
	\item
	\textbf{\textit{Разработку технологии проектирования интеллектуальных компьютерных систем}}, которая включает в себя семейство методик проектирования, а также методов и средств автоматизации \textit{проектирования} различных \textit{компонентов} \textit{интеллектуальных компьютерных систем} и \textit{интеллектуальных компьютерных систем} в целом. Результатом проектирования \textit{интеллектуальных компьютерных систем} является полная формальная логико-семантическая модель этой системы.
	\item
	\textbf{\textit{Разработку технологии реализации}} \textbf{\textit{спроектированных интеллектуальных компьютерных систем}}, а также технологий эксплуатации и сопровождения \textit{интеллектуальных компьютерных систем}. В основе технологии реализации (производства) спроектированных \textit{интеллектуальных компьютерных систем} лежит \textit{универсальный интерпретатор формальных логико-семантических моделей интеллектуальных компьютерных систем}, являющихся результатом проектирования указанных систем. Указанный универсальный интерпретатор может быть реализован либо в виде \textit{программной системы} на современных компьютерах, либо в виде \textit{универсального компьютера нового поколения}, ориентированного на интерпретацию формальных \textit{логико-семантических моделей интеллектуальных компьютерных систем}.
	\item
	\textbf{\textit{Прикладную инженерную деятельность в области Искусственного интеллекта}}, т.е. непосредственное проектирование, реализацию и сопровождение, включающее в себя обновление (реинжиниринг), осуществляемое в ходе эксплуатации, конкретных \textit{интеллектуальных компьютерных систем.}
	\item
	\textbf{\textit{Учебную деятельность в области Искусственного интеллекта}}, направленную на подготовку специалистов области \textit{Искусственного интеллекта} и на перманентное повышение квалификации действующих специалистов в этой области. Без эффективной организации учебной деятельности в области \textit{Искусственного интеллекта} быстрый прогресс в этой области невозможен. Непосредственное включение учебной деятельности в общую структуру человеческой деятельности в области \textit{Искусственного интеллекта} обусловлено следующими обстоятельствами:
	\begin{textitemize}
		\item
		необходимостью глубокой \textit{конвергенции} между различными направлениями и видами деятельности в области \textit{Искусственного интеллекта} и соответствующей спецификой требований, предъявляемых к специалистам в этой области -- каждый такой специалист должен быть достаточно компетентен и в научно-исследовательской деятельности в области \textit{Искусственного интеллекта}, и в разработке технологий (методик и средств) \textit{проектирования интеллектуальных компьютерных систем}, и в разработке технологий \textit{воспроизводства} (реализации) спроектированных \textit{интеллектуальных компьютерных систем}, а также технологий их \textit{эксплуатации} и \textit{сопровождения}, и в прикладной \textit{инженерной деятельности в области} \textit{Искусственного интеллекта};
		\item
		высокими темпами эволюции результатов в области \textit{Искусственного интеллекта}, что делает необходимой организацию обучения соответствующих специалистов путем их непосредственного подключения не к учебным (упрощенным) проектам, а к реальным проектам, реализуемым в текущий момент. Иначе подготовленные специалисты будут иметь квалификацию "вчерашнего дня"{};
		\item
		существенным расширением объёмов работ в области \textit{Искусственного интеллекта} и острой необходимостью массовой подготовки соответствующих специалистов.
	\end{textitemize}
	
	Сложность \textbf{\textit{Подготовки молодых специалистов в области Искусственного интеллекта}} заключается не только в высокой степени наукоемкости этой области, но и в том, что формирование у них соответствующих знаний и навыков осуществляется в условиях быстрого морального старения текущего состояния технологий \textit{Искусственного интеллекта}, существенные изменения в которых происходят за время обучения студентов и магистрантов. Поэтому надо учить не текущему уровню развития \textit{Искусственного интеллекта}, а тому уровню развития, который будет достигнут через пять и более лет.
	
	При подготовке молодых специалистов в области \textit{Искусственного интеллекта} необходимо формировать у них:
	\begin{textitemize}
		\item культуру формализации (математическую культуру);
		\item системную культуру (в частности, умение осуществлять качественную стратификацию сложных динамических систем);
		\item технологическую культуру (в частности, умение отличать то, что следует унифицировать и то, унификация чего ограничивает направление эволюции заданного класса сложных систем);
		\item технологическую дисциплину;
		\item культуру коллективного творчества (в частности, первоначальную \textit{интероперабельность});
		\item высокую \textit{познавательную активность} и мотивацию;
		\item умение сочетать индивидуальную творческую свободу и самостоятельность с обеспечением совместимости своих результатов с результатами коллег, то есть сочетать свободу в создании (порождении) новых смыслов при согласованности (совместимости) форм их представления -- о понятиях, терминах и синтаксисе не спорят, а договариваются.
	\end{textitemize}
	
	\item
	\textbf{\textit{Организационную деятельность в области Искусственного интеллекта}}, направленную на создание инфраструктуры для качественного выполнения всех остальных видов деятельности в области \textit{Искусственного интеллекта}, а именно:
	\begin{textitemize}
		\item
		для обеспечения глубокой \textit{конвергенции} между различными направлениями и видами деятельности в области \textit{Искусственного интеллекта} и, в частности, между теорией, технологиями и инженерной практикой в этой области;
		\item
		для обеспечения баланса между тактикой и стратегией в развитии деятельности в области \textit{Искусственного интеллекта} как ключевой основы существенного повышения уровня автоматизации всевозможных видов \textit{человеческой деятельности} и перехода к \textit{smart-обществу}.
	\end{textitemize}
\end{textitemize}

Рассмотренная декомпозиция \textit{человеческой} \textit{деятельности} в области \textit{Искусственного интеллекта} по \textit{видам} \textit{деятельности} не является традиционным признаком декомпозиции \textit{научно-технических дисциплин}. Обычно декомпозиция \textit{научно-технических дисциплин} осуществляется по содержательным направлениям, которые соответствуют декомпозиции \textit{технических систем}, исследуемых и разрабатываемых в рамках этих \textit{научно-технических дисциплин}, т.е. соответствуют выделению в этих \textit{технических системах} различного вида компонентов. Для \textit{Искусственного интеллекта} такими направлениями являются:

\begin{textitemize}
	\item
	исследование и разработка формальных моделей и языков представления знаний;
	\item
	исследование и разработка баз знаний;
	\item
	исследование и разработка логических моделей обработки знаний;
	\item
	исследование и разработка искусственных нейронных сетей;
	\item
	исследование и разработка подсистем компьютерного зрения;
	\item
	исследование и разработка подсистем обработки естественно-языковых текстов (синтаксический анализ, понимание, синтез);
	\item
	и многие другие.
\end{textitemize}

Важность декомпозиции \textit{Искусственного интеллекта} по \textit{видам} \textit{деятельности} определяется тем, что выделение различных \textit{видов деятельности} позволяет четко ставить задачу на разработку средств автоматизации этих \textit{видов деятельности}.

Приведем общую структуру \textit{Человеческой деятельности в области Искусственного интеллекта}.

\begin{SCn}
	\scnheader{Человеческая деятельность в области Искусственного интеллекта}
	\scnidtf{Искусственный интеллект (как научно-техническая дисциплина)}
	\scniselement{научно-техническая дисциплина}
	\scnidtf{Искусственный интеллект (как научно-техническая дисциплина)}
	\scnidtf{Человеческая деятельность в Предметной области интеллектуальных компьютерных систем}
	\scniselement{деятельность}
	\begin{scnrelfromset}{декомпозиция}
		\scnitem{Интегральная деятельность по поддержке жизненного цикла всевозможных интеллектуальных компьютерных систем}
		\begin{scnindent}
			\scnrelfrom{декомпозиция}{поддержка жизненного цикла интеллектуальных компьютерных систем}
			\begin{scnindent}
				\scniselement{вид деятельности}
				\scnsuperset{поддержка жизненного цикла ostis-систем}
				\begin{scnrelfromset}{обобщенная декомпозиция}
					\scnitem{проектирование интеллектуальных компьютерных систем}
					\scnitem{производство интеллектуальных компьютерных систем}
					\scnitem{начальное обучение интеллектуальных компьютерных систем}
					\scnitem{мониторинг качества интеллектуальных компьютерных систем}
					\scnitem{восстановление требуемого уровня качества интеллектуальных компьютерных систем}
					\scnitem{реинжиниринг интеллектуальных компьютерных систем}
					\scnitem{обеспечение безопасности интеллектуальных компьютерных систем}
					\scnitem{эксплуатация интеллектуальных компьютерных систем конечными пользователями}
				\end{scnrelfromset}
			\end{scnindent}
		\end{scnindent}
		\scnitem{Поддержка жизненного цикла Общей теории интеллектуальных компьютерных систем}
		\begin{scnindent}
			\scniselement{научно-исследовательская деятельность}
		\end{scnindent}        
		\scnitem{Поддержка жизненного цикла Стандарта интеллектуальных компьютерных систем}
		\begin{scnindent}
			\scniselement{стандартизация}
			\scnrelfrom{часть}{Поддержка жизненного цикла Стандарта ostis-систем}
		\end{scnindent}        
		\scnitem{Поддержка жизненного цикла Технологии комплексной поддержки жизненного цикла интеллектуальных компьютерных систем}
		\begin{scnindent}
			\scniselement{поддержка жизненного цикла технологий}
			\begin{scnindent}
				\scnidtf{создание и сопровождение технологий}
			\end{scnindent}
			\scnrelfrom{часть}{Поддержка жизненного цикла Технологии OSTIS}
		\end{scnindent}
		\scnitem{Поддержка жизненного цикла кадровых ресурсов для Человеческой деятельности в области Искусственного интеллекта}
		\scnitem{Поддержка жизненного цикла системы комплексной организации взаимодействия между всеми направлениями Человеческой деятельности в области Искусственного интеллекта}
		\begin{scnindent}
			\scniselement{поддержка жизненного цикла метасистем комплексного управления поддержкой и обеспечением поддержки жизненного цикла сущностей соответствующего класса}
		\end{scnindent}        
	\end{scnrelfromset}
	
	\scnheader{Технология поддержки жизненного цикла интеллектуальных компьютерных систем}
	\scnrelfrom{вид деятельности}{поддержка жизненного цикла интеллектуальных компьютерных систем}
	\begin{scnrelfromset}{декомпозиция}
		\scnitem{Технология проектирования интеллектуальных компьютерных систем}
		\begin{scnindent}
			\scnrelfrom{вид деятельности}{проектирование интеллектуальных компьютерных систем}
		\end{scnindent}
		\scnitem{Технология производства интеллектуальных компьютерных систем}
		\begin{scnindent}
			\scnrelfrom{вид деятельности}{производство интеллектуальных компьютерных систем}
		\end{scnindent}
		\scnitem{Технология начального обучения интеллектуальных компьютерных систем (адаптации к конкретной деятельности)}
		\begin{scnindent}
			\scnrelfrom{вид деятельности}{начальное обучение интеллектуальных компьютерных систем}
		\end{scnindent}
		\scnitem{Технология мониторинга качества интеллектуальных компьютерных систем}
		\begin{scnindent}
			\scnrelfrom{вид деятельности}{мониторинг качества интеллектуальных компьютерных систем}
			\begin{scnindent}
				\scnidtf{плановое тестирование и диагностика интеллектуальных компьютерных систем}
			\end{scnindent}
		\end{scnindent}
		\scnitem{Технология восстановления требуемого уровня качества интеллектуальных компьютерных систем в ходе их эксплуатации}
		\begin{scnindent}
			\scnidtf{Технология выявления и исправления потенциально опасных ситуаций и событий в деятельности интеллектуальных компьютерных систем (ошибок, противоречий ...)}
			\scnrelfrom{вид деятельности}{восстановление требуемого уровня качества интеллектуальных компьютерных систем}
		\end{scnindent}
		\scnitem{Технология реинжиниринга  интеллектуальных компьютерных систем}
		\begin{scnindent}
			\scnidtf{Технология совершенствования, модернизации, обновления интеллектуальных компьютерных систем}
			\scnrelfrom{вид деятельности}{реинжиниринг интеллектуальных компьютерных систем}
		\end{scnindent}
		\scnitem{Технология обеспечения безопасности интеллектуальных компьютерных систем}
		\begin{scnindent}
			\scnrelfrom{вид деятельности}{обеспечение безопасности интеллектуальных компьютерных систем}
		\end{scnindent}
		\scnitem{Технология эксплуатации интеллектуальных компьютерных систем конечными пользователями}
		\begin{scnindent}
			\scnrelfrom{вид деятельности}{эксплуатация интеллектуальных компьютерных систем конечными пользователями}
		\end{scnindent}
	\end{scnrelfromset}
\end{SCn}

\subsection{Текущее состояние и современные проблемы \textit{Искусственного интеллекта}}

Рассмотрим в каких направлениях должна происходить эволюция (повышение качества) деятельности в области \textit{Искусственного интеллекта}, а также эволюция продуктов этой деятельности.

\subsubsection{Текущее состояние и современные проблемы \textbf{\textit{научно-исследовательской деятельности в области}} \textbf{\textit{Искусственного интеллекта}}}

В настоящее время научные исследования в области \textit{Искусственного интеллекта} активно развиваются по широкому спектру различных направлений (\textit{модели представления знаний}, различного вида \textit{логики} -- дедуктивные, индуктивные, абдуктивные, четкие, нечеткие, различного вида \textit{искусственные нейронные сети}, машинное обучение, принятие решений, целеполагание, планирование поведения, ситуационное поведение, многоагентные системы, компьютерное зрение, распознавание, интеллектуальный анализ данных, мягкие вычисления и многое другое).

Однако:

\begin{textitemize}
	\item
	Отсутствует согласованность систем \textit{понятий} в разных направлениях \textit{Искусственного интеллекта} и, как следствие, отсутствует \textit{семантическая совместимость} и \textit{конвергенция} этих направлений, в результате чего существенно затруднена работа в направлении построения \textit{Общей теории интеллектуальных систем} с высоким уровнем формализации. Существование и продолжающееся увеличение "высоты барьеров" между различными направлениями исследований в области \textit{Искусственного интеллекта} проявляется в том, что специалист, работающий в рамках какого-либо направления \textit{Искусственного интеллекта}, посещая заседания "не своей" секции на конференции по \textit{Искусственному интеллекту}, мало что там может понять и, соответственно, извлечь полезного для себя;
	\item
	Отсутствует мотивация и осознание острой необходимости в указанной \textit{конвергенции} между различными направлениями \textit{Искусственного интеллекта};
	\item
	Отсутствует реальное движение в направлении построения \textit{Общей теории интеллектуальных систем}, поскольку отсутствует соответствующая мотивация и осознание острой практической необходимости в этом;
	\item
	Отсутствует строгое и согласованное уточнение понятия \textit{интеллектуальной компьютерные системы}. До сих пор для этого используется Тест Тьюринга. Поверхностная трактовка Теста Тьюринга породила различные имитации интеллекта в стиле "светского"{} разговора или разговора на "завалинке"{}. На самом деле должна учитываться содержательная, целевая установка диалога, в рамках которого интеллект \textit{интеллектуальной компьютерной системы} определяется как её нетривиальный вклад в коллективное решение некоторой интеллектуальной (творческой) задачи.
\end{textitemize}

\subsubsection{Текущее состояние \textbf{\textit{Стандарта}} \textbf{\textit{интеллектуальных компьютерных систем}}}

В настоящее время необходимость унификации и стандартизации \textit{интеллектуальных компьютерных систем} не осознана, что существенно тормозит создание \textit{комплексной технологии} \textit{Искусственного интеллекта}.

\subsubsection{Текущее состояние и современные проблемы \textbf{\textit{развития}} \textbf{\textit{технологии проектирования интеллектуальных компьютерных систем}}}

Современная \textit{технология} \textit{Искусственного интеллекта} представляет собой целое семейство всевозможных частных технологий, ориентированных на разработку различного вида компонентов \textit{интеллектуальных компьютерных систем,} реализующих самые различные модели представления и обработки информации, а также ориентированных на разработку различных классов \textit{интеллектуальных компьютерных систем}.

Однако:

\begin{textitemize}
	\item
	Высока трудоемкость разработки \textit{интеллектуальных компьютерных систем};
	\item
	Необходима высокая квалификация разработчиков;
	\item
	Современные \textit{технологии} \textit{Искусственного интеллекта} принципиально не обеспечивают разработки таких \textit{интеллектуальных компьютерных систем,} в которых устраняются недостатки современных \textit{интеллектуальных компьютерных систем} и, в частности, обеспечивается достаточно высокий уровень интероперабельности;
	\item
	Совместимость технологий \textit{проектирования различных классов интеллектуальных компьютерных систем} \textit{Искусственного интеллекта} практически отсутствует и, как следствие, не обеспечивается \textit{семантическая совместимость} и взаимодействие разрабатываемых \textit{интеллектуальных компьютерных систем}, поэтому системная интеграция \textit{интеллектуальных компьютерных систем} осуществляется вручную;
	\item
	Отсутствует \textit{комплексная технология проектирования} \textit{интеллектуальных компьютерных систем};
	\item
	Нет совместимости между существующими \textit{частными технологиями проектирования различных компонентов интеллектуальных компьютерных систем} (баз знаний, решателей задач, интеллектуальных интерфейсов). Есть инструментальные средства по разработке компонентов, но "склеивать" (соединять, интегрировать) разработанные компоненты надо вручную, поскольку нет комплексных инструментальных средств, позволяющих разрабатывать \textit{интеллектуальные компьютерные системы} в целом.
\end{textitemize}

\subsubsection{Текущее состояние и современные проблемы \textbf{\textit{развития}} \textbf{\textit{технологии реализации спроектированных интеллектуальных компьютерных систем}}, а также их эксплуатации и сопровождения}

Был сделан целый ряд попыток разработки \textit{компьютеров} \textit{нового поколения}, ориентированных на использование \textit{в интеллектуальных компьютерных системах.} Но все они оказались неудачными, так как не были ориентированы на всё многообразие \textit{моделей решения задач} в \textit{интеллектуальных компьютерных системах}. В этом смысле они не были \textit{универсальными компьютерами} для \textit{интеллектуальных компьютерных систем.}

Разрабатываемые \textit{интеллектуальные компьютерные системы} могут использовать самые различные комбинации \textit{моделей решения интеллектуальных задач} (логических моделей, соответствующих различного вида логикам, нейросетевых моделей различного вида, моделей целеполагания, синтеза планов, моделей управления сложными объектами, моделей понимания и синтеза текстов естественного языка и т.д.). Однако, современные традиционные (фон-неймановские) \textit{компьютеры} не в состоянии достаточно производительно интерпретировать всё многообразие указанных \textit{моделей решения задач}. При этом разработка специализированных \textit{компьютеров}, ориентированных на интерпретацию какой-либо одной \textit{модели решения задач} (нейросетевой модели или какой-либо логической модели) проблему не решает, так как в \textit{интеллектуальной компьютерной системе} необходимо использовать сразу несколько разных \textit{моделей решения задач}, причём в различных сочетаниях.

В настоящее время отсутствует комплексный подход к технологическому обеспечению всех этапов жизненного цикла \textit{интеллектуальных компьютерных систем} -- не только к поддержке проектирования и реализации (сборки, производства) \textit{интеллектуальных компьютерных систем}, но также и к технологической поддержке сопровождения, реинжиниринга и эксплуатации \textit{интеллектуальных компьютерных систем}.

Семантическая недружественность \textit{пользовательского интерфейса} и отсутствие встроенных интеллектуальных справочных систем, позволяющих запрашивать информацию об элементах интерфейса и возможностях системы, приводят к низкой эффективности эксплуатации всех возможностей \textit{интеллектуальной компьютерной системы}.

\subsubsection{Текущее состояние и современные проблемы \textbf{\textit{прикладной}} \textbf{\textit{инженерной деятельности в области Искусственного интеллекта}}}

Накоплен достаточно большой опыт разработки \textit{интеллектуальных компьютерных систем} самого различного назначения -- систем автоматизации медицинской диагностики, а также диагностики сложных технических систем, интеллектуальных обучающих, информационно-справочных и help-систем, систем естественно-языкового общения, интеллектуальных компьютерных персональных ассистентов, интеллектуальных корпоративных систем, интеллектуальных систем ситуационного управления различного рода сложными объектами, систем интеллектуального анализа больших данных, систем технического зрения и анализа сцен, интеллектуальных порталов знаний, интеллектуальных систем автоматизации проектирования различного вида сложных объектов, интеллектуальных систем автоматизации подготовки к производству спроектированной продукции различного вида, интеллектуальных автоматизированных систем управления производства различного вида продуктов, а также многих других \textit{интеллектуальных компьютерных систем.}

Однако:

\begin{textitemize}
	\item
	Уровень эффективности практического использования научных результатов в области \textit{Искусственного интеллекта} явно не соответствует современному уровню развития самих этих научных результатов. Для того чтобы повысить уровень эффективности практического использования указанных научных результатов, \underline{необходимы} \underline{совместные усилия} и ученых, создающих новые модели решения интеллектуальных задач, и разработчиков технологий проектирования и реализации \textit{интеллектуальных компьютерных систем}, и разработчиков прикладных \textit{интеллектуальных компьютерных систем.}
\end{textitemize}

\begin{textitemize}
	\item
	Отсутствует четкая систематизация многообразия \textit{интеллектуальных компьютерных систем}, соответствующая систематизации автоматизируемых \textit{видов человеческой деятельности};
	\item
	Отсутствует \textit{конвергенция \textbf{интеллектуальных компьютерных систем}}\textbf{,} обеспечивающих автоматизацию \textit{областей человеческой деятельности}, принадлежащих одному и тому же \textit{виду человеческой деятельности};
	\item
	Отсутствует \textit{семантическая совместимость} (семантическая унификация, взаимопонимание) между \textit{интеллектуальными компьютерными системами}, основной причиной чего является отсутствие согласованной системы общих используемых \textit{понятий};
	\item
	Анализ проблем комплексной автоматизации всех \textit{видов человеческой деятельности} убеждает в том, что дальнейшая \textit{автоматизация человеческой деятельности} требует не только повышения уровня \textit{интеллекта} соответствующих \textit{интеллектуальных компьютерных} систем, но и к существенному повышению уровня их способности:
	\begin{textitemize}
		\item
		устанавливать свою \textit{семантическую совместимость} (взаимопонимание) как с другими \textit{компьютерными системами}, так и со своими пользователями;
		\item
		поддерживать эту \textit{семантическую совместимость} в процессе собственной эволюции, а также эволюции пользователей и других \textit{компьютерных систем};
		\item
		координировать свою деятельность с пользователями и другими \textit{компьютерными системами} при коллективном решении различных задач;
		\item
		участвовать в распределении работ (подзадач) при коллективном решении различных задач.
	\end{textitemize}
\end{textitemize}


Важно подчеркнуть то, что реализация вышеперечисленных способностей создаст возможность для существенной и даже полной автоматизации \textit{системной интеграции компьютерных систем} в комплексы взаимодействующих \textit{интеллектуальных компьютерных систем} и автоматизации реинжиниринга таких комплексов. Такая автоматизация системной интеграции и её реинжиниринга:

\begin{textitemize}
	\item
	даст возможность комплексам компьютерных систем самостоятельно адаптироваться к решению новых задач;
	\item
	существенно повысит эффективность эксплуатации таких комплексов компьютерных систем, так как реинжиниринг системной интеграции компьютерных систем, входящих в такой комплекс, часто востребован (например, при реконструкции предприятий);
	\item
	существенно сокращает число ошибок по сравнению с "ручным"{} (неавтоматизированным) выполнением \textit{системной} \textit{интеграции} и её \textit{реинжиниринга}, которые, к тому же, требует высокой квалификации.
\end{textitemize}

Таким образом следующий этап повышения уровня автоматизации \textit{человеческой деятельности} настоятельно требует создания таких \textit{интеллектуальных компьютерных систем}, которые могли бы \underline{сами} (без системного интегратора) объединяться для совместного решения сложных задач.

\subsubsection{Текущее состояние и современные проблемы \textbf{\textit{учебной деятельности в области Искусственного интеллекта}}}

Многие ведущие университеты осуществляют подготовку специалистов в области \textit{Искусственного интеллекта}. При этом необходимо отметить следующие особенности и проблемы текущего состояния этой деятельности:

\begin{textitemize}
	\item
	Поскольку \textit{деятельность} \textit{в области Искусственного интеллекта} сочетает в себе и высокую степень наукоемкости и высокую степень сложности инженерных работ, подготовка специалистов в этой области требует одновременного формирования у них как научно-исследовательских навыков и знаний, так и инженерно-практических навыков и знаний, а также системной и технологической культуры и стиля мышления. С точки зрения методики и психологии обучения сочетание фундаментальной научной и инженерно-практической подготовки специалистов является достаточно сложной педагогической задачей;
	\item
	Отсутствует \textit{семантическая совместимость} между различными учебными дисциплинами, что приводит к "мозаичности"{} восприятия информации;
	\item
	Отсутствует системный подход к подготовке молодых специалистов в области \textit{Искусственного интеллекта};
	\item
	Нет персонификации обучения, а также установки на выявление, раскрытие и развитие индивидуальных способностей;
	\item
	Отсутствует целенаправленное формирование мотивации к творчеству;
	\item
	Нет формирования навыков работы в реальных коллективах разработчиков. Отсутствует адаптация к реальной практической деятельности;
	\item
	Любая современная технология (в том числе и технология \textit{Искусственного интеллекта}) должна иметь высокие темпы своего развития, поскольку без этого невозможно поддерживать высокий уровень её конкурентоспособности. Но для быстро развиваемой технологии требуется:
	\begin{textitemize}
		\item
		не просто высокая квалификация кадров, использующих и развивающих технологию;
		\item
		но и высокие темпы повышения уровня этой квалификации, так как без этого невозможно эффективно использовать и развивать быстро меняющуюся технологию.
	\end{textitemize}
\end{textitemize}

Из этого следует, что \textit{учебная деятельность в области} \textit{Искусственного интеллекта} и соответствующая ей технология должна быть не просто важной частью \textit{деятельности} в области \textit{Искусственного интеллекта}, а частью, глубоко интегрированной во все остальные \textit{виды} \textit{деятельности} в области \textit{Искусственного интеллекта}. Так, например, каждая \textit{интеллектуальная компьютерная система} должная быть ориентирована не только на обслуживание своих конечных пользователей, не только на организацию целенаправленного взаимодействия со своими разработчиками, которые постоянно совершенствуют эту систему, и не только на обеспечение минимального "порога вхождения"{} для новых конечных пользователей и разработчиков, но и на организацию постоянного и \underline{персонифицированного} повышения квалификации каждого своего конечного пользователя и разработчика в условиях постоянных изменений, вносимых в указанную \textit{интеллектуальную компьютерную систему.} Для этого эксплуатируемая \textit{интеллектуальная компьютерная система} должна "знать"{}, что в ней изменилось, на что она способна и как эти способности инициировать (содержание и форма, соответствующих пользовательских команд).

Когда мы говорим о \textit{конвергенции} и \textit{интеграции} в области \textit{Искусственного интеллекта}, речь идет не только о конвергенции между \textit{интеллектуальными компьютерными системами}, но также и между различными \underline{видами} и областями \textit{человеческой деятельности}. Таким образом, \textit{учебная деятельность}, направленная на подготовку специалистов в области \textit{Искусственного интеллекта}, органически входит в состав \textit{деятельности в области} \textit{Искусственного интеллекта}, а важнейшим направлением повышения эффективности этой деятельности является её \textit{конвергенция} и \textit{интеграция} с другими видами \textit{деятельности в области Искусственного интеллекта}.

\subsubsection{Текущее состояние и современные проблемы \textbf{\textit{организационной деятельности в области Искусственного интеллекта}}}

Острая потребность в существенном повышении уровня автоматизации в самых различных областях \textit{человеческой деятельности} (в промышленности, медицине, транспорте, образовании, строительстве и во многих других), а также современные результаты в развитии \textit{технологий Искусственного интеллекта} привели к существенному расширению работ по созданию \textit{прикладных интеллектуальных компьютерных систем} и к появлению большого количества коммерческих организаций, ориентированных на разработку таких приложений.

Однако:
\begin{textitemize}
	\item
	Не так просто обеспечить баланс тактических и стратегических направлений развития всех видов деятельности в области \textit{Искусственного интеллекта} (научно-исследовательской деятельности, развития технологии проектирования и производства интеллектуальных компьютерных систем, разработки прикладных систем, образовательной деятельности), а также баланс между всеми перечисленными \textit{видами деятельности};
	\item
	В настоящее время отсутствует глубокая \textit{конвергенция} различных \textit{видов деятельности} в области \textit{Искусственного интеллекта} (в первую очередь, конвергенция развития технологий \textit{Искусственного интеллекта} и разработки различных прикладных \textit{интеллектуальных компьютерных систем}), что существенно затрудняет развитие каждого из этих видов деятельности и, в частности, существенно затрудняет при разработке интеллектуальных решателей задач интеграцию различных моделей решения задач (например, логических моделей, нейросетевых моделей, моделей обработки текстов естественных языков, моделей обработки сигналов -- аудиосигналов, изображений);
	\item
	Высокий уровень наукоемкости работ в области \textit{Искусственного интеллекта} предъявляет особые требования к квалификации сотрудников и к их способности работать в составе \textit{творческих коллективов}.
\end{textitemize}

\subsection{Ключевые задачи и методологические проблемы современного этапа развития \textbf{\textit{Искусственного интеллекта}}}

К числу \textbf{ключевых задач} современного этапа развития \textit{Искусственного интеллек}та следует отнести:

\begin{textitemize}
	\item
	Построение \textbf{\textit{Общей формальной теории интеллектуальных компьютерных систем}}, в рамках которой была бы обеспечена совместимость всех направлений \textit{Искусственного интеллекта}, всех моделей представления знаний, всех моделей решения задач, всех компонентов \textit{интеллектуальных компьютерных систем}. Это предполагает:
	
	\begin{textitemize}
		\item
		Уточнение требований, предъявляемых к \textit{интеллектуальным компьютерным системам нового поколения} -- уточнение свойств \textit{интеллектуальных компьютерных систем}, определяющих высокий уровень их \textit{интеллекта};
		\item
		\textbf{\textit{Конвергенцию}} и \textit{интеграцию} всевозможных видов \textit{знаний} и всевозможных \textit{моделей решения задач} в рамках каждой \textit{интеллектуальной компьютерной системы}.
	\end{textitemize}
	
	\item
	Создание \textbf{\textit{инфраструктуры}}, обеспечивающей интенсивное перманентное развитие \textit{Общей формальной теории интеллектуальных компьютерных систем} в самых различных направлениях, гарантирующее сохранение логико-семантической целостности этой \textit{теории} и совместимости всех направлений ее развития;
	\item
	На основе \textit{Общей формальной теории интеллектуальных компьютерных систем} построение \textbf{\textit{Технологии комплексной поддержки жизненного цикла интеллектуальных компьютерных систем нового поколения}}, обладающих высоким уровнем интероперабельности и совместимости;
	\item
	Создание \textbf{\textit{инфраструктуры}}, обеспечивающей интенсивное перманентное развитие \textit{Комплексной технологии разработки и эксплуатации интеллектуальных компьютерных систем нового поколения} в самых различных направлениях, гарантирующее сохранение целостности этой \textit{технологии} и совместимости всех направлений ее развития;
	\item
	Разработку \textbf{\textit{компьютеров нового поколения}}, ориентированных на высокопроизводительную интерпретацию \textit{логико-семантических моделей интеллектуальных компьютерных систем нового поколения;}
	\item
	Создание \textbf{\textit{Глобальной экосистемы интеллектуальных компьютерных систем нового поколения}}, ориентированной на комплексную автоматизацию различных видов человеческой деятельности.
\end{textitemize}

Эпицентром современных методологических проблем развития \textit{человеческой} \textit{деятельности} в области \textit{Искусственного интеллекта} является \textbf{\textit{конвергенция}} и \textit{глубокая интеграция} всех видов, направлений и результатов этой \textit{деятельности}. Уровень взаимосвязи, взаимодействия и \textbf{\textit{конвергенции}} между различными видами и направлениями деятельности в области \textit{Искусственного интеллекта} в настоящее время явно недостаточен. Это приводит к тому, что каждая из них развивается обособленно, независимо от других. Речь идет о \textbf{\textit{конвергенции}} между такими направлениями \textit{Искусственного интеллекта}, как представление знаний, решение интеллектуальных задач, интеллектуальное поведение, понимание и др., а также между такими \textit{видами человеческой деятельности в области Искусственного интеллекта}, как научные исследования, разработка технологий, разработка приложений, образование, бизнес. Почему на фоне уже достаточно длительного интенсивного развития научных исследований в области \textit{Искусственного интеллекта} до сих пор не создан рынок \textit{интеллектуальных компьютерных систем} и комплексная технология \textit{Искусственного интеллекта}, обеспечивающая разработку широкого спектра \textit{интеллектуальных компьютерных систем} самого различного назначения и доступной широкому контингенту инженеров. Потому что сочетание высокого уровня наукоемкости и прагматизма этой проблемы требует для ее решения принципиально нового подхода к организации взаимодействия ученых, работающих в области \textit{Искусственного интеллекта}, разработчиков средств автоматизации проектирования \textit{интеллектуальных компьютерных систем,} разработчиков средств реализации \textit{интеллектуальных компьютерных систем}, включая средства аппаратной поддержки \textit{интеллектуальных компьютерных систем}, разработчиков прикладных \textit{интеллектуальных компьютерных систем}. Такое целенаправленное взаимодействие должно осуществляться как в рамках каждой из этих форм деятельности в области \textit{Искусственного интеллекта}, так и между ними. Таким образом, основной тенденцией дальнейшего развития теоретических и практических работ в области \textit{Искусственного интеллекта} является \textbf{\textit{конвергенция}} как самых разных видов (форм и направлений) \textit{человеческой деятельности} в области \textit{Искусственного интеллекта}, так и самых разных продуктов (результатов) этой деятельности. Необходимо ликвидировать барьеры между различными видами и продуктами деятельности в области \textit{Искусственного интеллекта} в целях обеспечения их совместимости и интегрируемости.

\textbf{\textit{Конвергенция}} разрабатываемых \textit{интеллектуальных компьютерных систем} преобразует набор индивидуальных (автономных) \textit{интеллектуальных компьютерных систем} различного назначения в коллектив активно взаимодействущих \textit{интеллектуальных компьютерных систем} для совместного (коллективного) решения сложных (комплексных) задач и для перманентной поддержки совместимости между всеми \textit{интеллектуальными компьютерными системами}, входящими в коллектив, в процессе индивидуальной эволюции каждой из этих систем.

\textbf{\textit{Конвергенция}} конкретных искусственных сущностей (например, технических систем) есть стремление к их унификации (в частности, к стандартизации), т.е. стремление к минимизации многообразия форм решения аналогичных практических задач -- стремление к тому, чтобы все, что можно сделать одинаково, делалось одинаково, но без ущерба требуемого качества. Последнее очень важно, так как безграмотная стандартизация может привести к существенному торможению прогресса. Ограничение многообразия форм не должно приводить к ограничению содержания, возможностей. Образно говоря, "словам должно быть тесно, а мыслям--свободно"{}.

Методологически \textbf{\textit{конвергенция}} искусственно создаваемых сущностей (артефактов) сводится (1) к выявлению (обнаружению) принципиальных сходств между этими сущностями, которые часто весьма закамуфлированы и их трудно "увидеть"{} и (2) к реализации обнаруженных сходств одинаковым образом (в одинаковой форме, в одинаковом "синтаксисе"{}). Образно говоря, от "семантической"{} (смысловой) эквивалентности требуется перейти и к "синтаксической"{} эквивалентности. Кстати, в этом как раз и заключается суть \textit{смыслового представления информации}, целью которого является создание такой языковой среды (смыслового пространства), в рамках которого (1) семантически эквивалентные информационные конструкции полностью совпадали бы, а (2) \textbf{\textit{конвергенция}} информационных конструкций сводилась бы к выявлению изоморфных фрагментов этих конструкций.

К числу общих \underline{методологических проблем} современного этапа развития \textit{Искусственного интеллекта} можно отнести:

\begin{textitemize}
	\item
	Отсутствие массового осознания того, что создание рынка \textit{интеллектуальных компьютерных систем нового поколения}, обладающих \textit{семантической совместимостью} и высоким уровнем \textit{интероперабельности}, а также создание комплексов (экосистем), состоящих из таких \textit{интеллектуальных компьютерных систем} и обеспечивающих автоматизацию различных \textit{видов человеческой деятельности}, \underline{невозможно}, если коллективы разработчиков таких систем и комплексов существенно не повысят уровень \textit{социализации} \textbf{всех} своих сотрудников. Уровень качества коллектива разработчиков, т.е. уровень квалификации сотрудников и уровень согласованности их деятельности, должен превышать уровень качества систем, разрабатываемых этим коллективом. Особое значение рассматриваемая проблема согласованности деятельности специалистов в области \textit{Искусственного интеллекта} имеет для построения \textit{Общей формальной теории интеллектуальных компьютерных систем нового поколения}, а также \textit{Комплексной технологии разработки и эксплуатации интеллектуальных компьютерных систем нового поколения};
	\item
	Далеко не всеми учеными, работающими в области \textit{Искусственного интеллекта}, принимается прагматичность, практическая направленность \textit{Искусственного интеллекта};
	\item
	Не всеми принимается необходимость \textbf{\textit{конвергенции}} различных направлений \textit{Искусственного интеллекта} и необходимость их интеграции в целях построения \textit{Общей формальной теории интеллектуальных компьютерных систем};
	\item
	Не всеми принимается необходимость \textbf{\textit{конвергенции}} различных видов деятельности в области \textit{Искусственного интеллекта};
	\item
	Важным препятствием для \textbf{\textit{конвергенции}} результатов научно-технической деятельности является сформировавшийся в науке и технике акцент на выявлении не сходств, а отличий. Чтобы убедиться в этом достаточно обратить внимание на то, что уровень научных результатов оценивается научной \underline{новизной}, которая может имитироваться новизной не по существу, а по форме представления (например, с помощью новых понятий или даже новых терминов). Результаты в технике, например, в патентах также оцениваются \underline{отличиями} от предшествующих технических решений. Но для \textbf{\textit{конвергенции}} нужны другие акценты -- не поиск отличий, а выявление неочевидных сходств и превращение их в очевидные сходства, представленные в одинаковой \underline{форме};
	\item
	Нет движения к построению \textit{Комплексной технологии проектирования, реализации, сопровождения, реинжиниринга и эксплуатации интеллектуальных компьютерных систем}. Речь идет о комплексном подходе к технологическому обеспечению \underline{всех этапов} \underline{жизненного цикла} \textit{интеллектуальных компьютерных систем};
	\item
	Нет активного развития работ по созданию \textit{Глобальной} \textit{экосистемы интеллектуальных компьютерных систем нового поколения};
	\item
	В основе современной организации и автоматизации \textit{человеческой деятельности} лежит "\textit{Вавилонское столпотворение}"{} постоянно расширяемого многообразия \textit{языков.} Имеются в виду не только \textit{естественные} \textit{языки}, но и \textit{формальные} \textit{язык}и, направленные на точное представление \textit{знаний} различного вида. Многообразие различных \textit{специализированных} \textit{языков} пронизывает всю \textit{человеческую деятельность} -- во многих областях \textit{человеческой деятельности} для решения различных видов \textit{задач}, для разработки различных \textit{моделей решения задач} создаются \textit{специализированные языки}. Примером этого является многообразие \textit{языков программирования}. \textit{Специализированные языки} могут и должны появляться, но только как \textit{\textit{подъязыки}} более общих \textit{языков}, синтаксис каждого из которых совпадает с \textit{синтаксисом} всех соответствующих ему \textit{подъязыков}. При этом в рамках \textit{Общей формальной теории интеллектуальных компьютерных систем} должен быть выделен один \textit{универсальный формальный} \textit{язык} -- язык-ядро, по отношению к которому все остальные используемые \textit{формальные языки} являются \textit{подъязыками}. \textit{Денотационная семантика} указанного \textit{универсального формального языка} должна задаваться соответствующей \textit{формальной онтологией} максимально высокого уровня. Иначе о какой \textbf{\textit{конвергенции}} и \textit{интеграции} \textit{знаний}, о какой \textit{семантической совместимости} компьютерных систем можно вести речь.
\end{textitemize}

В основе предлагаемой организации \textit{человеческой деятельности} в области \textit{Искусственного интеллекта} лежат следующие положения:

\begin{textitemize}
	\item
	\textbf{\textit{комплексная конвергенция}} -- как "вертикальная"{} \textit{конвергенция} между различными \textit{видами деятельности} в области \textit{Искусственного интеллекта}, так и "горизонтальная"{} \textit{конвергенция} в рамках каждого из этих \textit{видов деятельности}, соответствующая различным компонентам или различным классам \textit{интеллектуальных компьютерных систем} -- базам знаний, решателям задач, различным моделям решения задач, различным видам интерфейсов (зрительным, аудио, естественно-языковым), робототехническим интеллектуальным компьютерным системам, интеллектуальным обучающим системам, интеллектуальным автоматизированным системам управления, интеллектуальным системам автоматизации проектирования и т.д.);
	\item
	\textbf{\textit{"горизонтальная"{} конвергенция}} в рамках каждого вида \textit{человеческой деятельности} в области \textit{Искусственного интеллекта} включает в себя:
	
	\begin{textitemize}
		\item
		\textit{конвергенцию} в рамках \textit{научно-исследовательской деятельности в области Искусственного интеллекта}, означающую переход от независимого развития различных направлений \textit{Искусственного интеллекта} к общей теории \textit{интеллектуальных компьютерных систем};
		\item
		\textit{конвергенцию} в рамках развития \textit{технологий Искусственного интеллекта}, означающую переход от независимого развития частных технологий к созданию единого комплекса семантически совместимых частных технологий;
		\item
		\textit{конвергенцию} в рамках \textit{инженерной деятельности в области Искусственного интеллекта}, означающую переход от практики независимой разработки различных прикладных \textit{интеллектуальных компьютерных систем} к разработке комплекса (экосистемы) интероперабельных \textit{интеллектуальных компьютерных систем};
		\item
		\textit{конвергенцию} в рамках \textit{учебной деятельности в области Искусственного интеллекта}, обозначающую переход от изучения отдельных учебных дисциплин к формированию у молодых специалистов целостной картины текущего состояния \textit{Искусственного интеллекта} и проблемных направлений дальнейшего развития;
		\item
		\textit{конвергенцию} в рамках \textit{общей организационной деятельности в области Искусственного интеллекта}, переход от отдельных вышеперечисленных видов деятельности в области \textit{Искусственного интеллекта} к единому комплексу всех этих видов деятельности и обеспечивающую конвергенцию и интеграцию указанных видов деятельности в области \textit{Искусственного интеллекта}, что существенно повысит их качество, поскольку каждый из этих видов деятельности находится в сильной зависимости от всех остальных;
	\end{textitemize}
	
	\item
	организация разработки и перманентного развития предлагаемой \textit{технологии} в виде \textbf{\textit{открытого международного проекта}}, предоставляющего:
	\begin{textitemize}
		\item
		свободный доступ к использованию текущей версии разрабатываемой \textit{технологии};
		\item
		возможность каждому желающему войти в состав коллектива разработчиков этой \textit{технологии};
	\end{textitemize}
	
	\item
	\textbf{\textit{поэтапность}} процесса формирования рынка \textit{семантически совместимых} и \textit{активно взаимодействующи}х между собой \textit{интеллектуальных компьютерных систем нового} \textit{поколения}, начальными этапами которого являются:
	
	\begin{textitemize}
		\item
		разработка \textit{логико-семантических моделей} (баз знаний) нескольких \textit{прикладных интеллектуальных компьютерных систем нового поколения};
		\item
		программная реализация на современных компьютерах \textit{платформы интерпретации логико-семантических моделей интеллектуальных компьютерных систем нового поколения};
		\item
		установка каждой разработанной \textit{логико-семантической модели прикладной интеллектуальной компьютерной системы} на разработанную программную платформу интерпретации таких моделей с последующим \textit{тестированием} и \textit{реинжинирингом} каждой такой модели;
		\item
		разработка и перманентное совершенствование логико-семантической модели (базы знаний) \textit{интеллектуальной компьютерной метасистемы}, которая содержит (1) описание \textit{стандарта интеллектуальных компьютерных систем нового поколения}, (2) \textit{библиотеку} многократно используемых (в различных \textit{интеллектуальных компьютерных системах}) знаний различного вида и, в частности, различных \textit{методов решения задач}, (3) \textit{методы проектирования} и \textit{средства поддержки проектирования} \textit{различных видов компонентов интеллектуальных компьютерных систем} (компонентов \textit{баз знаний, решателей задач, интерфейсов});
		\item
		разработка \textit{ассоциативного семантического компьютера} в качестве аппаратной реализации \textit{платформы интерпретации логико-семантических моделей интеллектуальных компьютерных систем нового поколения};
		\item
		перенос разработанных \textit{логико-семантических моделей интеллектуальных компьютерных систем нового поколения} на новые, более эффективные варианты реализации платформы интерпретации этих моделей;
		\item
		развитие \textit{рынка интеллектуальных компьютерных систем нового поколения} в виде Глобальной экосистемы, состоящей из активно взаимодействующих таких систем и ориентированной на комплексную автоматизацию всех \textit{видов} \textit{человеческой деятельности};
		\item
		создание \textbf{\textit{рынка знаний}} на основе \textit{Глобальной экосистемы} \textit{интеллектуальных компьютерных систем нового поколения};
		\item
		автоматизация \textit{реинжиниринга} эксплуатируемых \textit{интеллектуальных компьютерных систем нового поколения} в направлении приведения их в соответствие с новыми версиями \textit{стандарта интеллектуальных компьютерны}х \textit{систем} путем автоматической замены устаревших \textit{компонентов} в этих системах на текущие версии этих компонентов.
	\end{textitemize}
\end{textitemize}

Следует особо подчеркнуть, что \textbf{ключевым фактором решения рассматриваемых методологических проблем} в области \textit{Искусственного интеллекта} являются различные направления \textbf{\textit{конвергенции}} и \textbf{\textit{интеграции}}, обеспечивающие переход к \textbf{\textit{интеллектуальным компьютерным системам нового поколения}}, к соответствующей технологии комплексной поддержки их жизненного цикла и к существенному повышению уровня автоматизации всего комплекса человеческой деятельности:

\begin{textitemize}
	\item
	\textit{конвергенция} и \textit{интеграция} различных моделей представления и обработки \textit{информации} в \textit{интеллектуальных компьютерных системах нового поколения}
	
	\begin{textitemize}
		\item \textit{конвергенция} и \textit{интеграция} различных \textit{видов} \textit{знаний} в \textit{базах знаний} \textit{интеллектуальных компьютерных систем нового поколения}
		
		\item \textit{конвергенция} и \textit{интеграция} различных \textit{моделей решения задач}
		
		\item \textit{конвергенция} и \textit{интеграция} различных \textit{видов интерфейсов} \textit{интеллектуальных компьютерных систем нового поколения}
	\end{textitemize}
	
	\item
	\textit{конвергенция} и \textit{интеграция} различных направлений \textit{Искусственного интеллекта} в целях построения \textit{Общей формальной теории интеллектуальных компьютерных систем нового поколения }
	\item
	\textit{конвергенция} и \textit{интеграция} технологий \textit{проектирования} различных \textit{компонентов интеллектуальных компьютерных систем нового поколения} в целях построения комплексной \textit{Технологии проектирования интеллектуальных компьютерных систем нового поколения }
	\item
	\textit{конвергенция} и \textit{интеграция} технологий поддержки различных \textit{этапов жизненного цикла} \textit{интеллектуальных компьютерных систем нового поколения} в целях построения \textit{Технологии комплексной поддержки всех этапов жизненного цикла интеллектуальных компьютерных систем нового поколения}
	\item
	\textit{конвергенция} и \textit{интеграция} различных \textit{видов человеческой деятельности в области Искусственного интеллекта} (\textit{научно-исследовательской деятельности}, \textit{развития технологического комплекса}, \textit{прикладной инженерии}, \textit{образовательной деятельности}) для повышения уровня согласованности и координации этих \textit{видов деятельности}, а также для повышения уровня их комплексной автоматизации с помощью \textbf{\textit{семантически совместимых}} \textit{интеллектуальных компьютерных систем нового поколения}
	\item
	\textit{конвергенция} и \textit{интеграция} самых различных \textit{видов и областей человеческой деятельности}, а также средств комплексной автоматизации этой деятельности с помощью \textit{интеллектуальных компьютерных систем нового поколения}
\end{textitemize}

\textbf{Конечным практическим результатом} человеческой деятельности в области \textit{Искусственного интеллекта} является:

\begin{textitemize}
	\item
	Реорганизация и комплексная автоматизация \textit{человеческой деятельности в области Искусственного интеллекта} с помощью \textit{интеллектуальных компьютерных систем нового поколения};
	\item
	\underline{Поэтапное} создание глобальной сети эффективно взаимодействующих \textbf{\textit{интеллектуальных компьютерных систем нового поколения}}, обеспечивающих \underline{комплексную} автоматизацию всевозможных видов и областей \textit{человеческой деятельности}.
\end{textitemize}

Переход от современных интеллектуальных компьютерных систем к \textit{интеллектуальным компьютерным системам нового поколения} и к соответствующей комплексной технологии не требует от специалистов в области Искусственного интеллекта изменения сферы их научных интересов. От них требуется только преодолеть синдром ``\textit{Вавилонского столпотворения}'', оформляя свои научные результаты как часть общего коллективного продукта.

Проблемы текущего этапа развития \textit{Искусственного интеллекта}, направленного на создание Общей теории и технологии \textit{интеллектуальных компьютерных систем нового поколения}, требуют \underline{фундаментального} комплексного междисциплинарного подхода и принципиально новой организации научно-технической деятельности.

\subsection{Комплексная автоматизация человеческой деятельности в области Искусственного интеллекта с помощью интеллектуальных компьютерных систем нового поколения}

В рамках \textit{Технологии OSTIS} поддержка жизненного цикла интеллектуальных компьютерных систем нового поколения (\textit{ostis-систем}) осуществляется на основе \textit{Метасистемы OSTIS}, которая относится к классу \textit{ostis-систем} и фактически является формой реализации указанной Технологии. Автоматизация поддержки жизненного цикла \textit{ostis-систем} осуществляется как в форме инструментального обслуживания инженерной деятельности (в частности, Метасистема OSTIS является системой автоматизации проектирования ostis-систем), так и в форме информационного обслуживания указанной деятельности. Для этого база знаний \textit{Метасистема OSTIS} содержит: 
\begin{textitemize}
	\item текущее состояние полного текста \textit{Стандарта ostis-систем};
	\item Текущее состояние Библиотеки многократно используемых компонентов ostis-систем;
	\item Используемые и реализуемые инженерами методики поддержки жизненного цикла ostis-систем;
	\item Документацию инструментальных средств, инженерами для поддержки жизненного цикла  ostis-систем.
\end{textitemize}

Кроме всего этого, \textit{Метасистема OSTIS}: 
\begin{textitemize}
	\item Обеспечивает автоматизацию \textit{Поддержки жизненного цикла Стандарта ostis-систем}, то есть обеспечивает организацию взаимодействия между авторами этого Стандарта, направленного на перманентное его развитие;
	\item Обеспечивает автоматизацию \textit{Поддержки жизненного цикла Технологии OSTIS}, которая сводится к поддержке жизненного цикла основной части базы знаний \textit{Метасистемы OSTIS}, которая является полной документацией текущего состояния \textit{Технологии OSTIS}. 
\end{textitemize}

Автоматизация других направлений \textit{Человеческой деятельности в области Искусственного интеллекта} также может осуществляться с помощью \textit{ostis-систем}, семантически совместимых и взаимодействующих с \textit{Метасистемой OSTIS} в рамках \textit{Экосистемы OSTIS}.

\section{Ключевые виды и области человеческой деятельности}
\label{sec_activity_types}

\section{Проблемы и перспективы комплексной автоматизации всевозможных видов и областей человеческой деятельности с помощью интеллектуальных компьютерных систем нового поколения}
\label{sec_activity_perspectives}

Выше было рассмотрено то, как осуществляется и автоматизируется с помощью интеллектуальных компьютерных систем нового поколения \underline{весь комплекс} \textit{Человеческой деятельности в области Искусственного интеллекта}. Сейчас обобщим это и рассмотрим принципы организации и комплексной автоматизации \textit{человеческой деятельности} в целом, т.е. автоматизации самых различных видов и областей человеческой деятельности.

\subsection{Общие принципы систематизации человеческой деятельности и ее комплексной автоматизации с помощью интеллектуальных компьютерных систем нового поколения}

Опыт комплексной организации, структуризации и автоматизации \textit{человеческой} деятельности в области \textit{Искусственного интеллекта} (в области создания и сопровождения интеллектуальных компьютерных систем) можно обобщить и для других областей человеческой деятельности. Это обусловлено следующими причинами:

\begin{textitemize}
	\item
	Во-первых, потому, что человеческая деятельность, направленная на поддержку всего \textit{жизненного цикла интеллектуальных компьютерных систем} нового поколения, является \underline{частной} \underline{областью деятельности} по отношению к виду человеческой деятельности, направленному на (обеспечивающему) поддержку всего жизненного цикла \underline{любой искусственной} (искусственно создаваемой) сущности (любого артефакта). В зависимости от сложности искусственно создаваемой сущности, уровень сложности человеческой деятельности, направленной на поддержку жизненного цикла этой сущности, может быть самым различным, но общая структура этой деятельности, соответствующая различным этапам жизненного цикла искусственно создаваемых сущностей, а также необходимым направлениям \underline{обеспечения} этой инженерной деятельности является одинаковой для искусственных сущностей различных классов. К указанным направлениям обеспечения поддержки жизненного цикла искусственных сущностей относятся:
	\begin{textitemize}
		\item научно-исследовательская деятельность, направленная на изучение искусственных сущностей соответствующего класса;
		\item разработка стандарта искусственных сущностей указанного класса;
		\item разработка технологии поддержки искусственных сущностей указанного класса;
		\item подготовка кадров, способных осуществлять поддержку жизненного цикла искусственных сущностей указанного класса, т.е. способных эффективно использовать указанную выше технологию;
		\item подготовка кадров, способных участвовать в указанной выше научно-исследовательской деятельности;
		\item подготовка кадров, способных участвовать в разработке стандарта искусственных сущностей заданного класса;
		\item подготовка кадров, способных участвовать в разработке и развитии указанной выше технологии;
		\item организационное обеспечение всего комплекса работ по развитию и использованию указанной технологии.
	\end{textitemize}
	
	\item
	Во-вторых, потому, что многие сложные технические системы фактически становятся \textit{интеллектуальными компьютерными системами} (в т.ч. распределенными) с различными наборами сенсорных и эффекторных подсистем -- интеллектуальными автомобилями с автопилотом и автоштурманом, интеллектуальными заводами-автоматами, умными домами, умными городами и т.п.
	\item
	В-третьих, потому, что характер деятельности \textit{интеллектуальных компьютерных систем нового поколения} и характер деятельности каждого \textit{человека} и каждой организации по сути мало чем отличаются, поскольку \textit{интеллектуальные компьютерные системы нового поколения} становятся \underline{равноправными} партнерами (субъектами) \textit{человеческой деятельности}, т.к. уровень их самостоятельности, ответственности, интероперабельности и интеллектуальности приближается к соответствующим качествам \textit{естественных} субъектов человеческой деятельности (физических лиц, юридических лиц, подразделений крупных организаций, неформальных организаций).
\end{textitemize}

Итак, структуризацию \textit{человеческой деятельности} в области \textit{Искусственного интеллекта} на основе понятий \textit{вида деятельности}, \textit{области деятельности}, \textit{продукта деятельности} (объекта деятельности) можно легко обобщить для всех \textit{научно-технических дисциплин}, что дает возможность рассматривать автоматизацию деятельности в рамках всех \textit{научно-технических дисциплин} с общих позиций, т.к. автоматизация различных \textit{видов деятельности} в рамках различных \textit{научно-технических дисциплин} может выглядеть аналогичным образом, а иногда может быть реализована с помощью одной и той же \textit{интеллектуальной компьютерной системы}. Так например, любая \textit{интеллектуальная компьютерная система автоматизации проектирования} технических систем заданного вида может быть построена на основе \textit{интеллектуальной компьютерной системы автоматизации проектирования и реинжиниринга баз знаний}, поскольку результатом проектирования любой \textit{технической системы} является формальная модель (описание, спецификация, документация) этой \textit{технической системы}, обладающая достаточно полнотой для воспроизводства (реализации) этой системы.

На текущем этапе развития \textit{Искусственного интеллекта} необходимо переходить от автоматизации отдельных \textit{видов человеческой деятельности} к интегрированной автоматизации всего комплекса \textit{человеческой деятельности}, к созданию и постоянной эволюции всей \textbf{\textit{Глобальной экосистемы интеллектуальных компьютерных систем}}, самостоятельно взаимодействующих как между собой, так и с людьми, автоматизацию деятельности которых они осуществляют, а также с современными компьютерными системами, не являющимися интеллектуальными системами. При этом надо помнить, что основные "накладные"{} расходы, основные проблемы, возникают на "стыках"{} при интеграции различных технических решений. Разработчик каждой подсистемы должен гарантировать отсутствие указанных "накладных"{} расходов. При этом необходимо подчеркнуть, что следует ориентироваться не столько на создание эффективной \textit{Глобальной экосистемы интеллектуальных компьютерных систем}, сколько на создание эффективных методик и средств, направленных на \textit{перманентную эволюцию} такой \textit{экосистемы}.

Методика комплексной автоматизации \textit{человеческой деятельности} включает в себя следующие этапы:

\begin{textitemize}
	\item
	Построение общей \textbf{\textit{структуры человеческой деятельности}}, в основе которой лежит иерархия \textit{человеческой деятельности} по видам деятельности и продуктам деятельности с четкой фиксацией различного вида связей между различными компонентами этой структуры.
	\item
	Формализация различных \textit{видов человеческой деятельности}.
	\item
	Разработка \textbf{\textit{технологии}}, обеспечивающей максимально возможную автоматизацию этой деятельности с помощью \textit{интеллектуальных компьютерных систем нового поколения}.
	\item
	Обеспечение максимально возможной \textbf{\textit{конвергенции}} различных \textit{видов деятельности}, что позволит сократить многообразие средств автоматизации (т.е. соответствующих \textit{интеллектуальных компьютерных систем нового поколения}).
	\item
	Обеспечение максимально возможной \textbf{\textit{конвергенции технологий}} выполнения одного и того же \textit{вида деятельности} для разных объектов деятельности (конвергенции технологий проектирования объектов различных классов, конвергенции технологий мониторинга, профилактики и диагностики для агентов различных классов и т.д.) и, тем самым, обеспечить \textbf{\textit{конвергенцию}} соответствующих средств автоматизации, построенных на основе \textit{интеллектуальных компьютерных систем нового поколения}.
\end{textitemize}

\subsection{Многообразие видов человеческой деятельности и связей между ними}

Базовым видом человеческой деятельности можно считать \textbf{\textit{поддержку жизненного цикла}} различных сущностей.

Классом объектов деятельности для этого \textit{вида деятельности} является класс всевозможных социально значимых объектов, на которые имеет смысл воздействовать, поддержку жизненного цикла которых целесообразно осуществлять.

\begin{SCn}
	\scnheader{поддержка жизненного цикла}
	\scnidtf{поддержка жизненного цикла социально значимых сущностей}
	\scniselement{вид деятельности}
	\begin{scnrelfromlist}{частный вид деятельности, выполняемой на некотором этапе}
		\scnitem{проектирование}
		\scnitem{производство}    
		\scnitem{начальное обучение}  
		\begin{scnindent}
			\scnidtf{настройка}
		\end{scnindent}
		\scnitem{мониторинг качества}
		\begin{scnindent}
			\scnidtf{плановое обследование и диагностика}
		\end{scnindent}
		\scnitem{восстановление требуемого уровня качества}
		\begin{scnindent}
			\scnidtf{ремонт, лечение}
		\end{scnindent}
		\scnitem{реинжиниринг}
		\begin{scnindent}
			\scnidtf{обновление, совершенствование}
		\end{scnindent}
		\scnitem{обеспечение безопасности}
		\scnitem{использование}
		\begin{scnindent}
			\scnidtf{эксплуатация, употребление}
		\end{scnindent}
	\end{scnrelfromlist}
	\begin{scnrelfromlist}{частный вид деятельности над подклассом объектов деятельности}
		\scnitem{научно-исследовательская деятельность}
		\begin{scnindent}
			\scnidtf{поддержка жизненного цикла научных теорий}
			\scnrelfrom{класс объектов деятельности}{научная теория}
		\end{scnindent}
		\scnitem{стандартизация}
		\begin{scnindent}
			\scnidtf{поддержка жизненного цикла стандартов}
			\scnrelfrom{класс объектов деятельности}{стандарт}
		\end{scnindent}
		\scnitem{поддержка жизненного цикла технологий}
		\begin{scnindent}
			\scnrelfrom{класс объектов деятельности}{технология}
		\end{scnindent}
		\scnitem{образовательная деятельность}
		\begin{scnindent}
			\scnidtf{учебная деятельность}
			\scnidtf{поддержка жизненного цикла кадровых ресурсов}
			\scnrelfrom{класс объектов деятельности}{кадровый ресурс}
		\end{scnindent}
		\scnitem{поддержка жизненного цикла метасистем комплексного управления поддержкой и обеспечение поддержки жизненного цикла сущностей соответствующих классов}
		\begin{scnindent}
			\scnrelfrom{класс объектов деятельности}{метасистема комплексного управления поддержкой и обеспечением поддержки жизненного цикла сущностей соответствующих классов}
		\end{scnindent}
	\end{scnrelfromlist}
\end{SCn}

Когда выше рассматривалась общая структура \textit{человеческой деятельности} путем обобщения структуры \textbf{\textit{Человеческой деятельности в области Искусственного интеллекта}}, мы:

\begin{textitemize}
	\item
	ввели понятие \textit{вида деятельности}
	\item
	в качестве "исходной точки"{} обобщения выбрали такой \textit{вид деятельности}, как \textit{поддержка} \textit{жизненного цикла интеллектуальных компьютерных систем}
	\item
	далее расширяли класс \textit{объектов деятельности} выбранного \textit{вида деятельности},
	
	\begin{textitemize}
		\item  переходя от класса \textit{интеллектуальных компьютерных систем} к классу всевозможных \textit{искусственных материальных сущностей}
		
		\item  объединяя класс \textit{искусственных материальных сущностей} с классом \textit{естественных материальных сущностей} (материальных сущностей естественного происхождения), а также с классом \textit{естественно-искусственных материальных сущностей} (либо \textit{естественных искусственно модифицированных материальных сущностей}, либо \textit{гибридных} \textit{естественно-искусственных материальных сущностей}, имеющих компоненты как естественного так и искусственного происхождения);
		
		\item объединяя класс \textit{материальных сущностей} с классом \textit{информационных ресурсов}, т.е. социально значимых информационных конструкций (документов), являющихся продуктами соответствующих действий или деятельностей (\textit{научными теориями}, \textit{стандартами}, \textit{базами знаний}, \textit{методами}, \textit{проектными документациями} соответствующих создаваемых объектов)
		
		\item объединяя класс \textit{материальных сущностей} \textit{информационных ресурсов} с классом \textit{материально-информационных объектов}, к которым, в частности, относятся различные технологии.
	\end{textitemize}
\end{textitemize}

Таким образом, \textit{поддержка жизненного цикла} различных социально значимых объектов является особым видом \textit{человеческой деятельности}. Во-первых, эффективность \textit{Человеческой деятельности} в целом зависит (1) от длительности социально полезной (активной) фазы жизненного цикла используемых объектов и (2) от объема затрат общества на поддержание необходимых социально полезных свойств используемых объектов. Во-вторых, характер и \textit{технология} поддержки жизненного цикла разных видов социально значимых объектов могут существенно отличаться друг от друга. Так, например, существенно отличается организация поддержки жизненного цикла автомобилей, традиционных компьютерных систем различного назначения, современных интеллектуальных компьютерных систем, интероперабельных интеллектуальных компьютерных систем, людей, предприятий, домов, различных юридических лиц, населенных пунктов и др. При этом типология социально значимых объектов, жизненный цикл которых должен поддерживаться, включает в себя самые разнообразные классы объектов - искусственно создаваемые материальные информационные продукты человеческой деятельности, всех людей, всевозможные социальные сообщества и предприятия. Многообразие типов социально значимых объектов порождает многообразие соответствующих им технологий, что усложняет комплексную автоматизацию человеческой деятельности в целом.

Тем не менее, заметим, что \textit{видов человеческой деятельности} значительно меньше, чем \textit{областей человеческой деятельности}. Это в определенной степени обусловлено тем, что видов связей между сущностями (относительных понятий) значительно меньше, чем классов различных сущностей. Данное обстоятельство указывает на то, что в основе движения в направление глобальной автоматизации деятельности \textit{общества} должна лежать ориентация на грамотную систематизацию \textit{видов человеческой деятельности}, и на их максимально глубокую \textit{конвергенцию} (как внутри каждого вида деятельности, так и между различными видами). Благодаря этому искусственно привносимое многообразие средств автоматизации \textit{человеческой деятельности} может быть сведено к минимуму.

\begin{SCn}
	\scnheader{следует отличать}
	\begin{scnhaselementset}
		\scnitem{научно-исследовательская деятельность}
		\begin{scnindent}
			\scnidtf{поддержка жизненного цикла научных теорий}
		\end{scnindent}
		\scnitem{стандартизация}
		\begin{scnindent}
			\scnidtf{разработка и развитие стандартов}
			\scnidtf{поддержка жизненного цикла стандартов}
		\end{scnindent}
		\scnitem{поддержка жизненного цикла технологий}
	\end{scnhaselementset}
\end{SCn}

\textit{Научно-исследовательская деятельность} направлена на \textbf{\textit{изучение сущностей заданного класса}}, на изучение принципов, лежащих в основе их структуры и функционирования. В рамках этого вида деятельности важна новизна и конкуренция идей и подходов, важно соотношение между структурой (архитектурой) организацией функционирования исследуемых \textit{сущностей} и общими характеристиками (параметрами) качества этих сущностей, общими предъявляемыми к ним требованиями. Продуктом рассматриваемой деятельности является \textit{Общая теория сущностей заданного класса}, которая отражает множественность и даже \underline{противоречивость} разных точек зрения и важнейшим направлением развития (эволюции) которой является сближение (\textit{конвергенция}) различных точек зрения и обеспечение совместимости и непротиворечивости между ними. В основе \textit{научно-исследовательской деятельности} лежит конкуренция точек зрения, принципиальная новизна идей и верифицированных результатов, направленных на выявление и обоснование неочевидных свойств и закономерностей соответствующей \textit{предметной области}, на разработку методов решения различных \textit{классов задач}, решаемых в рамках этой \textit{предметной области}. Цель \textit{научно-исследовательской деятельности} и требуемая детализация вырабатываемых знаний об объектах исследований соответствующих \textit{предметных областей}.

В отличие от \textit{научно-исследовательской деятельности} в основе разработки \textit{стандарта} создаваемых сущностей и разработки соответствующей \textit{технологии} поддержки их жизненного цикла лежит \underline{согласование} различных точек зрения (поиск консенсуса) и максимально возможное их \underline{упрощение} (соблюдение принципа Бритвы Оккама). Необходимость такой методологической установки обусловлена массовым характером \textit{человеческой деятельности} по созданию и \textit{поддержке жизненного цикла} соответствующего класса сущностей и необходимостью вовлечения в эту деятельность людей с \textit{разной} (в т.ч. и достаточно низкой) квалификацией. В процессе \textit{разработки стандарта сущностей заданного класса} важна не конкуренция различных точек зрения, а их \textit{конвергенция}, \textit{семантическая совместимость} и глубокая интеграция. Каждый \textit{стандарт} \textit{искусственных сущностей заданного класса} -- это согласованная \underline{на текущий момент} точка зрения (консенсус) о структуре, функционировании, свойствах и закономерностях искусственных сущностей заданного класса, согласованная (общепризнанная) часть \textit{Общей теория искусственных сущностей заданного класса}, доступная для понимания широкому контингенту практиков (инженеров), которые проектируют, производят и поддерживают весь жизненный цикл конкретных \textit{искусственных сущностей заданного класса}.

При создании и \textit{поддержке жизненного цикла технологий} должны учитываться ряд требований, предъявляемых к \underline{любым} \textit{технологиям}:

\begin{textitemize}
	\item
	комплексность -- максимально возможное покрытие всех задач, которые должны решаться с помощью \textit{технологии} (как минимум всех этапов жизненного цикла)
	\item
	максимально возможная простота в использовании \textit{технологии} (необходимая полнота документации, интеллектуальная help-поддержка, отсутствие лишней информации, которая не является необходимой для использования \textit{технологии}, наличие богатой и систематизированной библиотеки типовых многократно используемых решений)
\end{textitemize}

Общество -- это иерархическая система взаимодействующих индивидуальных и коллективных субъектов, каждый из которых:

\begin{textitemize}
	\item
	Производит либо часть социально значимой продукции, производимой коллективным субъектом, в состав которого входит данный субъект, либо целостный социально-значимый продукт (производимый товар), потребляемый другими внешними субъектами или оказывает некоторую услугу другому субъекту, направленную на обеспечение жизнедеятельности и совершенствование этого другого субъекта.
	\item
	Потребляет продукцию, произведенную другими субъектами, необходимую для производства собственной продукции (сырье и оборудование), а также необходимую для обеспечения своей жизнедеятельности.
	\item
	Потребляет услуги, оказываемые другими субъектами необходимые для производства собственной продукции или услуг, а также необходимые для совершенствования своей деятельности.
\end{textitemize}

Основными направлениями автоматизации всего комплекса \textit{человеческой деятельности} являются:

\begin{textitemize}
	\item
	
	автоматизация социально полезной профессиональной деятельности всех субъектов деятельности (как индивидуальных субъектов -- всех физических лиц, так и всевозможных коллективных -- корпоративных субъектов, в т.ч. юридических лиц)
	
	\item
	
	автоматизация обеспечения (создания) комфортных условий для всех субъектов деятельности общества на основе мониторинга деятельности и конкретного (адаптированного) содействия эволюции каждого субъекта с учетом его непосредственных потребностей и проблем.
	
\end{textitemize}

Организация взаимодействий каждого субъекта с внешней средой должна осуществляться как со стороны этого субъекта, так и со стороны указанной внешней среды (т.е. со стороны общества). Общество должно повернуться "лицом"{} к каждому субъекту и не бросать его на произвол судьбы. В настоящее время создание (обеспечение) условий субъектов деятельности общества отдано на откуп каждого такого субъекта. Общество в лице специально предназначенных для этого других субъектов оказывает услуги и снабжает товарами \underline{по заказу} (по инициативе) нуждающегося в этом субъекта. Таким образом, ответственность за развитие каждого субъекта деятельности ложится исключительно на "плечи"{} этого субъекта. Поддержка общества носит общий характер и никак не учитывает особенности текущего положения каждого субъекта.

Важнейшей причиной, препятствующей дальнейшему повышению общего уровня автоматизации человеческой деятельности является то, что автоматизация различных областей человеческой деятельности осуществляется \underline{локально}.

На современном этапе применения интеллектуальных компьютерных систем основной проблемой является не автоматизация локальных видов и областей человеческой деятельности, а автоматизация комплексных процессов человеческой деятельности, требующая \textit{интеграции} в априори \underline{непредсказуемых} комбинациях самых различных информационных ресурсов и самых различных автоматизированных сервисов, реализуемых в виде специализированных интеллектуальных компьютерных систем.

Локальность автоматизации человеческой деятельности приводит к тому, что вся человеческая деятельность приобретает облик "архипелага"{}, состоящего из хорошо автоматизированных "островов"{}, но соединяемых между собой "вручную"{}. Это "ручное"{} не автоматизируемое соединение указанных "островов"{} полностью зависит от человеческого фактора и квалификации соответствующих исполнителей.

Указанное "ручное"{} соединение некоторого множества семантически близких автоматизированных областей человеческой деятельности можно автоматизировать, но делать это надо очень грамотно на высоком уровне системной культуры и на фундаментальной основе общей теории человеческой деятельности.

Еще одна важная причина, препятствующая дальнейшему повышению общего уровня автоматизации общества заключается в том, что автоматизация различных областей человеческой деятельности осуществляется без выявления и глубокого анализа сходства некоторых видов деятельности в разных областях и соответственно без сближения, \textbf{\textit{конвергенции}} и \textbf{\textit{унификации}} этих \textit{видов деятельности}.

Важнейшим направлением повышения уровня автоматизации человеческой деятельности является переход к автоматизации все более и более \underline{комплексных} (крупных) видов и областей человеческой деятельности например, от автоматизации деятельности различных предприятий, организаций, хозяйственных служб к автоматизации деятельности города в целом).

Автоматизации комплексных видов человеческой деятельности требует создания комплекса активно взаимодействующих компьютерных систем, каждая из которых обеспечивает автоматизацию соответствующего частного вида человеческой деятельности, входящего в состав автоматизируемого комплексного вида деятельности. При этом число уровней иерархии автоматизируемых видов человеческой деятельности ничем не ограничивается. Очевидно, что уровень автоматизации комплексных видов человеческой деятельности определяется:

\begin{textitemize}
	\item
	уровнем конвергенции (сближения, совместимости) соответствующих частных видов деятельности;
	\item
	качеством интеграции этих частных видов деятельности;
	\item
	уровнем конвергенции компьютерных систем, обеспечивающих автоматизацию указанных частных видов деятельности;
	\item
	качеством взаимодействия этих компьютерных систем т.е. уровнем интероперабельности этих систем).
\end{textitemize}

\begin{SCn}
	\scnfilelong{Уровень эволюции общества во многом зависит от уровня автоматизации человеческой деятельности, от уровня развития соответствующих технологий такой автоматизации. Но эта зависимость выглядит значительно сложнее чем, кажется на первый взгляд, особенно, если речь идет об автоматизации не физической, интеллектуальной человеческой деятельности (как индивидуальной, так и коллективной). Безграмотная, а тем более социально безответственная или злонамеренная автоматизация информационной деятельности общества способны нанести огромный ущерб его развитию. Такая безграмотность и безответственность, например, приводит к таким побочным факторам, как компьютерная зависимость, виртуализация окружающей среды, поверхностный характер мышления, снижение познавательной мотивации и активности и многое другое.}
	
	\begin{scnrelfromlist}{следовательно}
		\scnfileitem{необходимо существенно повысить уровень социальной ответственности у разработчиков компьютерных систем и соответствующих технологий.}
		\scnfileitem{Опасность от безграмотного, социально безответственного и тем более злонамеренного внедрения интеллектуальных компьютерных систем нового поколения может иметь для человечества летальный характер.}
	\end{scnrelfromlist}
	
	\scnheader{человеческое общество}
	\scntext{направления развития}{Если рассматривать \textit{общество} как \textit{многоагентную систему}, состоящую из самостоятельных интеллектуальных агентов, то, очевидно, что важнейшими факторами, определяющими повышение качества (уровня развития) \textit{общества} являются:
		\begin{scnitemize}
			\item
			повышение эффективности использования опыта, накопленного \textit{обществом}, эффективности использования человечеством \textit{знаний} и \textit{навыков}
			\item
			повышение темпов приобретения, накопления и систематизации эффективно используемым человечеством \textit{знаний} и \textit{навыков}.
		\end{scnitemize}
		
		Решение указанных проблем становится вполне возможным, если для этого использовать \textit{интеллектуальные компьютерные системы нового поколения}, с помощью которых накапливаемые человечеством \textit{знания} и \textit{навыки} будут организованы как систематизированная распределенная библиотека многократно используемых информационных ресурсов (\textit{знаний} и \textit{навыков}).}
	\begin{scnindent}
		\scntext{следовательно}{Систематизация и автоматизация многократного использования накапливаемых человечеством информационных ресурсов требует их конвергенции, глубокой интеграции и формализации. Особое место в этом процессе занимает математика, как основа систематизации и формализации знаний и навыков на уровне формальных \textit{онтологий верхнего уровня}.}
	\end{scnindent}
	
\end{SCn}

\section*{Заключение к Главе \ref{chapter_automation_perspectives}}

Благодаря тому, что \textit{интеллектуальные компьютерные системы нового поколения} становятся самостоятельными и активными субъектами \textit{человеческой деятельности} в достаточной степени равноправными людям (естественным индивидуальным субъектам человеческой деятельности), характер и, соответственно, уровень автоматизации \textit{человеческой деятельности} существенно меняется -- снимается необходимость \underline{управлять} средствами автоматизации, поскольку такое "ручное"{} управление заменяется распределением обязанностей и ответственности между людьми и \textit{интеллектуальными компьютерными системами нового поколения}.

Если автоматизация \underline{любого} вида в любой области \textit{человеческой деятельности} будет осуществляться с помощью \textit{интеллектуальных компьютерных систем нового поколения} и если \textit{интеллектуальные компьютерные системы нового поколения}, обеспечивающие автоматизацию \underline{разных} видов и областей \textit{человеческой деятельности}, будут содержательно взаимодействовать между собой, то общий уровень автоматизации \textit{человеческой деятельности} существенно возрастет благодаря тому, что отпадет необходимость \underline{вручную} координировать использование различных средств автоматизации.

Эффективность и трудоемкость автоматизации различных видов и областей \textit{человеческой деятельности} будет существенно определяться степенью \textbf{\textit{конвергенции}} между различными видами и областями \textit{человеческой деятельности}. Необходимо построить иерархическую модель \textit{человеческой деятельности,} в рамках которой должна быть проведена грамотная систематизация и стратификация всех видов и областей \textit{человеческой деятельности}, направленная против излишнего эклектического многообразия. Таким образом, прежде, чем осуществлять комплексную автоматизации \textit{человеческой деятельности} с помощью \textit{интеллектуальных компьютерных систем} \textit{нового поколения}, необходимо с позиций общей теории систем переосмыслить организацию этой деятельности. В противном случае автоматизация беспорядка приведет к еще большему беспорядку.

Особо подчеркнем то, что многие из рассмотренных нами проблем текущего состояния и направлений дальнейшего развития \textit{Человеческой деятельности в области Искусственного интеллекта} аналогичны проблемам и тенденциям развития многих других научно-технических дисциплин. Следовательно, подходы к решению этих проблем могут носить междисциплинарный характер.

Время каждого человека является главным невосполнимым ресурсом общества и тратить его надо не на рутинную поддержку жизненного цикла всевозможных социально значимых объектов, а на комплексное развитие соответствующих \textit{технологий}. Автоматизация человеческой деятельности с помощью глобальной системы интероперабельных семантически совместимых и активно взаимодействующих \textit{интеллектуальных компьютерных систем} в самых разных областях \textit{человеческой деятельности} позволит существенно сократить время каждого человека на выполнение рутинной, легко автоматизируемой деятельности. Человеческая деятельность должна стать ориентированной на максимально возможную самореализацию, раскрытие \underline{творческого} потенциала каждого человека, направленного на ускорение темпов повышения уровня интеллекта всего общества.

Создание \textit{Глобальной экосистемы интеллектуальных компьютерных систем нового поколения}, предполагает:

\begin{textitemize}
	\item
	Построение формальной модели \textit{человеческой деятельности};
	\item
	Переход от эклектичного построения сложных \textit{интеллектуальных компьютерных систем}, использующих различные виды \textit{знаний} и различные виды \textit{моделей решения задач}, к их глубокой \textit{интеграции} и \textit{унификации}, когда одинаковые модели представления и модели обработки знаний реализуется в разных системах и подсистемах одинаково;
	\item
	Сокращение дистанции между современным уровнем \textit{теории интеллектуальных компьютерных систем} и практики их разработки;
	\item
	Разработку грамотной тактики и стратегии переходного периода, в рамках которого современные \textit{интеллектуальные компьютерные системы} должны постепенно заменяться на \textit{интеллектуальные компьютерные системы нового поколения}, которые должны эффективно взаимодействовать не только между собой, но и с хорошо зарекомендовавшими себя современными информационными ресурсами и сервисами.
\end{textitemize}

%\input{author/references}