\begin{partbacktext}
\part*{Используемые сокращения и предметный указатель OSTIS}
\markboth{Используемые сокращения и предметный указатель OSTIS}{Используемые сокращения и предметный указатель OSTIS}
\label{part_glossary}
\addcontentsline{toc}{part}{Используемые сокращения и  предметный указатель OSTIS}

Предлагаемый вашему вниманию предметный указатель представляет собой алфавитный перечень всех основных терминов, используемых в данной монографии, которые \myuline{взаимно-однозначно} соответствуют элементам рафинированной семантической сети, представляющей собой базу знаний \textit{Метасистемы OSTIS}, основная часть которой семантически эквивалентна тексту данной монографии.

В рамках внутреннего языка \textit{ostis-систем} (в рамках \textit{SC-кода}) указанные основные термины называются \textit{основными sc-идентификаторами} (основными внешними идентификаторами sc-элементов --- элементов рафинированных семантических сетей).

%В данный предметный указатель включаются как русскоязычные, так и англоязычные термины (если нет аналогичного русскоязычного), непереводимые интернациональные названия (например, названия различных программных систем, такие как \textit{Neo4j}, \textit{MySQL} и т.д.), а также различные используемые сокращения.

В данном предметном указателе в алфавитном порядке перечислены:
\begin{textitemize}
	\item все русскоязычные \myuline{основные} термины описываемых в монографии сущностей с указанием
	\begin{textitemize}
		\item их англоязычных эквивалентов;
		\item тех разделов монографии, в которых эти термины являются ключевыми знаками;
	\end{textitemize}
	\item все \myuline{интернациональные} основные термины (например, названия различных программных систем, такие как \textit{Neo4j}, \textit{MySQL} и так далее), используемые в монографии с указанием
	\begin{textitemize}
		\item соответствующих разделов монографии, где эти термины являются ключевыми знаками.
	\end{textitemize}
\end{textitemize}

При этом для каждого \myuline{основного} русскоязычного термина указывается синонимичный ему \myuline{основной} англоязычный термин. Полужирным курсивом выделяются только \myuline{основные} sc-идентификаторы. Например:

\begin{SCn}
	\scnheader{sc-идентификатор}
	\scnidtf{строка символов или пиктограмма, взаимно однозначно представляющая соответствующий sc-элемент, хранимый в sc-памяти}
	\scnidtf{внешний идентификатор sc-элемента}
	\begin{scnsubdividing}
		\scnitem{основной sc-идентификатор}
		\scnitem{неосновной sc-идентификатор}
		\begin{scnindent}
			\scnsuperset{часто используемый неосновной sc-идентификатор}
		\end{scnindent} 
	\end{scnsubdividing}

	\begin{scnsubdividing}
		\scnitem{русскоязычный sc-идентификатор}
		\scnitem{англоязычный sc-идентификатор}
		\scnitem{интернациональный sc-идентификатор}
	\end{scnsubdividing}

	\begin{scnset}
	\vspace{-\baselineskip}
	\scnheader{sc-элемент}
	\scnidtf{\textit{sc-element}}
	\begin{scnindent}
		\scniselement{основной англоязычный sc-идентификатор}
		\begin{scnindent}
			\scnidtf{основной sc-идентификатор для англоязычного режима}
		\end{scnindent}
	\end{scnindent}
	\scnidtf{\textit{sc-элемент}}
	\begin{scnindent}
		\scniselement{основной русскоязычный sc-идентификатор}
		\begin{scnindent}
			\scnidtf{основной sc-идентификатор для русскоязычного режима}
		\end{scnindent}
	\end{scnindent}
	\end{scnset}
	\scnexplanation{Для каждого ключевого термина в предметном указателе указывается его основной русскоязычный идентификатор и основной англоязычный идентификатор. При этом по умолчанию считается, что русскоязычный термин является \myuline{основным} русскоязычным идентификатором, а соответствующий ему англоязычный --- \myuline{основным} англоязычным идентификатором.}
\end{SCn}

\bigskip

Для каждого неосновного, но часто используемого русскоязычного термина указывается синонимичный ему основной русскоязычный термин. Кроме того, для каждого основного русскоязычного термина указывается ссылка на соответствующую главу, параграф или пункт монографии, где сущность, обозначаемая этим термином, является ключевым знаком.

Данный предметный указатель представляет собой перечень кратких спецификаций всех используемых в монографии терминов. Эти спецификации упорядочены по алфавиту специфицируемых терминов. Используемые в монографии термины с формальной точки зрения представляют собой имена (идентификаторы) соответствующих им (именуемых) sc-элементов (элементов рафинированной семантической сети).

Класс все таких терминов разбивается на следующие подклассы:
\begin{textitemize}
	\item основные русскоязычные термины, используемые \myuline{только} для русскоязычного режима отображения элементов рафинированной семантической сети;
	\item основные англоязычные термины.
\end{textitemize}

\begin{SCn}
	\scnheader{Предметный указатель OSTIS}
	\scnidtf{Терминологический указатель}
	\begin{scnitemize}
		\item Спецификация ключевых знаков (основных sc-идентификаторов)
		\begin{scnitemize}
			\item основной русскоязычный;
			\item часто используемый русскоязычный;
			\item основной англоязычный;
			\item ссылка на главы и/или параграфы и/или пункты;
			\item сравнение.
		\end{scnitemize}
	\end{scnitemize}
\end{SCn}
\end{partbacktext}

\begin{SCn}

\scnheader{абстрактная сущность}
\scntext{трактовка}{трактовка этого термина имеет два аспекта:
	\begin{textitemize}
		\item вымышленная (придуманная, реально несуществующая) сущность в отличие оо реальной (материальной) сущности, например, множество, пространственная точка;
		\item сущность, имеющая неоднозначную спецификацию описывающую только те свойства, которые важны (существенны) только для некоторой точки зрения. Например, абстрактная (виртуальная) машина, абстрактный (виртуальный) пациент.
	\end{textitemize}
}
\begin{scnreltolist}{ключевой знак}
	\scnitem{\ref{sec_ps_scp}~\nameref{sec_ps_scp}}
\end{scnreltolist}

\scnheader{Абстрактная scp-машина}
\scnidtf{\textit{Abstract scp-machine}}
\begin{scnreltolist}{ключевой знак}
	\scnitem{\ref{sec_ps_scp}~\nameref{sec_ps_scp}}
\end{scnreltolist}

\scnheader{абстрактный sc-агент}
\scnidtf{\textit{abstract sc-agent}}
\begin{scnreltolist}{ключевой знак}
	\scnitem{\ref{sec_ps_agents}~\nameref{sec_ps_agents}}
\end{scnreltolist}

\scnheader{абстрактный sc-агент, не реализуемый на Языке SCP}	
\scnidtf{\textit{abstract sc-agent, non-implementable in the SCP Language}}
\begin{scnreltolist}{ключевой знак}
	\scnitem{\ref{sec_ps_agents}~\nameref{sec_ps_agents}}
\end{scnreltolist}

\scnheader{абстрактный sc-агент, реализуемый на Языке SCP}
\scnidtf{\textit{abstract sc-agent, implementable in the SCP Language}}
\begin{scnreltolist}{ключевой знак}
	\scnitem{\ref{sec_ps_agents}~\nameref{sec_ps_agents}}
\end{scnreltolist}

\scnheader{адаптивный интерфейс}
\scnidtf{\textit{adaptive interface}}
\begin{scnreltolist}{ключевое понятие}
    \scnitem{Глава \ref{chapter_interfaces}~\nameref{chapter_interfaces}}
\end{scnreltolist}

\scnheader{ассоциативный семантический компьютер}
\scnidtf{\textit{associative semantic computer}}
\begin{scnreltolist}{ключевой знак}
	\scnitem{Глава \ref{chapter_computers}~\nameref{chapter_computers}}
	\scnitem{Глава \ref{chapter_interpreter}~\nameref{chapter_interpreter}}
\end{scnreltolist}

\scnheader{атомарный абстрактный sc-агент}
\scnidtf{\textit{atomic abstract sc-agent}}
\begin{scnreltolist}{ключевой знак}
	\scnitem{\ref{sec_ps_agents}~\nameref{sec_ps_agents}}
\end{scnreltolist}

\scnheader{база знаний}
\scnidtf{\textit{knowledge base}}
\begin{scnreltolist}{ключевое понятие}
	\scnitem{Глава \ref{chapter_top_ontologies}~\nameref{chapter_top_ontologies}}
	\scnitem{\ref{sec_kb}~\nameref{sec_kb}}
\end{scnreltolist}

\scnheader{базовая ostis-платформа}
\scnidtf{\textit{basic ostis-platform}}
\begin{scnreltolist}{ключевой знак}
	\scnitem{Глава \ref{chapter_interpreter}~\nameref{chapter_interpreter}}
\end{scnreltolist}

\scnheader{базовый класс описываемых сущностей}
\scnidtf{\textit{basic entity class}}
\begin{scnreltolist}{ключевое понятие}
	\scnitem{Глава \ref{chapter_top_ontologies}~\nameref{chapter_top_ontologies}}
\end{scnreltolist}

\scnheader{библиотека многократно используемых компонентов ostis-систем}
\scnidtftext{сокращение основного sc-идентификатора}{библиотека м.и.к. ostis-систем}
\scnidtf{\textit{library of ostis-systems reusable components}}
\begin{scnreltolist}{ключевой знак}
	\scnitem{Глава \ref{chapter_library}~\nameref{chapter_library}}
\end{scnreltolist}

\scnheader{Библиотека OSTIS}
\scnidtf{\textit{OSTIS library}}
\begin{scnreltolist}{ключевой знак}
	\scnitem{Глава \ref{chapter_library}~\nameref{chapter_library}}
\end{scnreltolist}


\scnheader{бинарное отношение}
\scnidtf{\textit{binary relation}}
\begin{scnreltolist}{ключевое понятие}
	\scnitem{Глава \ref{chapter_top_ontologies}~\nameref{chapter_top_ontologies}}
\end{scnreltolist}

\scnheader{блокировка*}
\scnidtf{\textit{lock*}}
\begin{scnreltolist}{ключевой знак}
	\scnitem{\ref{sec_ps_sync}~\nameref{sec_ps_sync}}
\end{scnreltolist}

\scnheader{величина}
\scnidtf{\textit{value}}
\begin{scnreltolist}{ключевое понятие}
	\scnitem{Глава \ref{chapter_top_ontologies}~\nameref{chapter_top_ontologies}}
\end{scnreltolist}

\scnheader{вид знаний}
\scnidtf{\textit{knowledge type}}
\begin{scnreltolist}{ключевое понятие}
	\scnitem{\ref{sec_kb}~\nameref{sec_kb}}
\end{scnreltolist}


\scnheader{включение*}
\scnidtf{\textit{inclusion*}}
\begin{scnreltolist}{ключевое отношение}
	\scnitem{Глава \ref{chapter_top_ontologies}~\nameref{chapter_top_ontologies}}
\end{scnreltolist}

\scnheader{временная связь}
\scnidtf{\textit{temporary connection}}
\begin{scnreltolist}{ключевое понятие}
	\scnitem{Глава \ref{chapter_top_ontologies}~\nameref{chapter_top_ontologies}}
\end{scnreltolist}

\scnheader{временная сущность}
\scnidtf{\textit{temporary entity}}
\begin{scnreltolist}{ключевое понятие}
	\scnitem{Глава \ref{chapter_top_ontologies}~\nameref{chapter_top_ontologies}}
\end{scnreltolist}

\scnheader{ГРЗ}
\scnidtftext{сокращение основного sc-идентификатора}{гибридный решатель задач}
\begin{scnreltolist}{ключевой знак}
	\scnitem{\ref{chapter_situation_management}~\nameref{chapter_situation_management}}
\end{scnreltolist}

\scnheader{действие в sc-памяти}
\scnidtf{\textit{action in sc-memory}}
\begin{scnreltolist}{ключевой знак}
	\scnitem{\ref{sec_ps_actions}~\nameref{sec_ps_actions}}
\end{scnreltolist}

\scnheader{действие в sc-памяти, инициируемое вопросом}
\scnidtf{\textit{action in sc-memory, initiated by a question}}
\begin{scnreltolist}{ключевой знак}
		\scnitem{\ref{sec_ps_actions}~\nameref{sec_ps_actions}}
\end{scnreltolist}

\scnheader{действие редактирования базы знаний}
\scnidtf{\textit{action of knowledge base editing}}
\begin{scnreltolist}{ключевой знак}
	\scnitem{\ref{sec_ps_actions}~\nameref{sec_ps_actions}}
\end{scnreltolist}

\scnheader{денотационная семантика метода*}
\scnidtf{\textit{denotational semantics of method*}}
\begin{scnreltolist}{ключевое отношение}
	\scnitem{Глава \ref{chapter_programs}~\nameref{chapter_programs}}
\end{scnreltolist}

\scnheader{денотационная семантика языка представления методов*}
\scnidtf{\textit{denotational semantics of method representation language*}}
\begin{scnreltolist}{ключевое отношение}
	\scnitem{Глава \ref{chapter_programs}~\nameref{chapter_programs}}
\end{scnreltolist}

\scnheader{жизненный цикл}
\begin{scnreltolist}{ключевой знак}
	\scnitem{Часть \ref{part1}~\nameref{part1}}
\end{scnreltolist}

\scnheader{задача, решаемая в sc-памяти}
\scnidtf{\textit{problem, solved in sc-memory}}
\begin{scnreltolist}{ключевой знак}
	\scnitem{\ref{sec_ps_actions}~\nameref{sec_ps_actions}}
\end{scnreltolist}

\scnheader{знание}
\scnidtf{\textit{knowledge}}
\begin{scnreltolist}{ключевое понятие}
	\scnitem{\ref{chapter_top_ontologies}~\nameref{chapter_top_ontologies}}
	\scnitem{\ref{sec_kb}~\nameref{sec_kb}}
\end{scnreltolist}

\scnheader{измерение*}
\scnidtf{\textit{measurement*}}
\begin{scnreltolist}{ключевое отношение}
	\scnitem{Глава \ref{chapter_top_ontologies}~\nameref{chapter_top_ontologies}}
\end{scnreltolist}

\scnheader{измерение с фиксированной единицей измерения}
\scnidtf{\textit{measurement of fixed measure}}
\begin{scnreltolist}{ключевое понятие}
	\scnitem{Глава \ref{chapter_top_ontologies}~\nameref{chapter_top_ontologies}}
\end{scnreltolist}

\scnheader{индивидуальная кибернетическая система}
\begin{scnreltolist}{ключевой знак}
	\scnitem{Глава \ref{chap_intro}~\nameref{chap_intro}}
\end{scnreltolist}

\scnheader{интеллектуальная кибернетическая система}
\begin{scnreltolist}{ключевой знак}
	\scnitem{Глава \ref{chap_intro}~\nameref{chap_intro}}
\end{scnreltolist}

\scnheader{интеллектуальная компьютерная система нового поколения}
\scnidtftext{сокращение основного sc-идентификатора}{и.к.с. нового поколения}
\scntext{определение}{интеллектуальная компьютерная система, обладающая:
\begin{scnitemize}
	\item высоким уровнем самообучаемости, обеспечивающим высокий уровень автоматизации собственной эволюции и, соответственно, высокие темпы этой эволюции;
	\item высоким уровнем интероперабельности.
\end{scnitemize}}
\scnsubset{самообучаемая интеллектуальная компьютерная система}
\scnsubset{интероперабельная интеллектуальная компьютерная система}
\begin{scnreltolist}{ключевое понятие}
	\scnitem{Глава~\ref{chapter_new_generation_systems}~\nameref{chapter_new_generation_systems}}
\end{scnreltolist}

\scnheader{ИСС}
\scnidtftext{сокращение основного sc-идентификатора}{интеллектуальная справочная система}
\begin{scnreltolist}{ключевой знак}
	\scnitem{\ref{}~\nameref{}}
\end{scnreltolist}

\scnheader{инициируемое пользовательским интерфейсом действие*}
\scnidtf{\textit{action initiated by the user interface}}
\begin{scnreltolist}{ключевое отношение}
    \scnitem{Глава \ref{chapter_interfaces}~\nameref{chapter_interfaces}}
\end{scnreltolist}

\scnheader{интеллект\scnsupergroupsign}
\scnidtf{\textit{intelligence\scnsupergroupsign}}
\scnidtf{уровень интеллекта\scnsupergroupsign}
\scniselement{параметр}
\scnrelfrom{область определения}{кибернетическая система}
\begin{scnreltolist}{ключевой знак}
	\scnitem{Глава \ref{chapter_interfaces}~\nameref{chapter_interfaces}}
\end{scnreltolist}

\scnheader{интеллектуальный интерфейс}
\scnidtf{\textit{intelligent interface}}
\begin{scnreltolist}{ключевое понятие}
    \scnitem{Глава \ref{chapter_interfaces}~\nameref{chapter_interfaces}}
\end{scnreltolist}

\scnheader{интерпретатор пользовательских действий}
\scnidtf{\textit{interpreter of user actions}}
\begin{scnreltolist}{ключевое понятие}
    \scnitem{Глава \ref{chapter_interfaces}~\nameref{chapter_interfaces}}
\end{scnreltolist}

\scnheader{интерпретатор sc-моделей пользовательских интерфейсов}
\scnidtf{\textit{interpreter of the sc-models of the user interfaces}}
\begin{scnreltolist}{ключевое понятие}
    \scnitem{Глава \ref{chapter_interfaces}~\nameref{chapter_interfaces}}
\end{scnreltolist}

\scnheader{интерфейс}
\scnidtf{\textit{interface}}
\begin{scnreltolist}{ключевое понятие}
    \scnitem{Глава \ref{chapter_interfaces}~\nameref{chapter_interfaces}}
\end{scnreltolist}

\scnheader{интерфейсное действие пользователя}
\scnidtf{\textit{user interface action}}
\begin{scnreltolist}{ключевое понятие}
    \scnitem{Глава \ref{chapter_interfaces}~\nameref{chapter_interfaces}}
\end{scnreltolist}

\scnheader{интерфейс ostis-систем}
\scnidtf{\textit{интерфейс интеллектуальных компьютерных систем нового поколения}}
\scnidtf{\textit{ostis-system interface}}
\begin{scnreltolist}{ключевое понятие}
    \scnitem{Глава \ref{chapter_interfaces}~\nameref{chapter_interfaces}}
\end{scnreltolist}

\scnheader{искусственный интеллект}
\scnidtf{\textit{artificial intelligence}}
\begin{scnreltolist}{ключевое понятие}
	\scnitem{Глава \ref{}~\nameref{}}
\end{scnreltolist}

\scnheader{кибернетическая система}
\scnsuperset{компьютерная система}
\begin{scnindent}
	\scnidtf{искусственная кибернетическая система}
	\scnsuperset{интеллектуальная компьютерная система}
	\begin{scnindent}
		\scnsuperset{интеллектуальная компьютерная система нового поколения}
		\begin{scnindent}
			\scneq{\textup{(} самообучаемая компьютерная система $\cap$ интероперабельная компьютерная система \textup{)}}
			\scnsuperset{ostis-система}
			\begin{scnindent}
				\scnidtf{предлагаемое уточнение (вариант реализации) интеллектуальной компьютерной системы нового поколения}
			\end{scnindent}
		\end{scnindent} 
	\end{scnindent} 
\end{scnindent} 
\begin{scnreltolist}{ключевой знак}
	\scnitem{Часть \ref{part1}~\nameref{part1}}
	\scnitem{Глава \ref{chap_intro}~\nameref{chap_intro}}
\end{scnreltolist}

\scnheader{класс}
\scnidtf{\textit{class}}
\begin{scnreltolist}{ключевое понятие}
	\scnitem{Глава \ref{chapter_top_ontologies}~\nameref{chapter_top_ontologies}}
\end{scnreltolist}

\scnheader{класс логически атомарных действий}
\scnidtf{\textit{class of logically atomic actions}}
\begin{scnreltolist}{ключевой знак}
	\scnitem{\ref{sec_ps_actions}~\nameref{sec_ps_actions}}
\end{scnreltolist}

\scnheader{компонент пользовательского интерфейса}
\scnidtf{\textit{user interface component}}
\begin{scnreltolist}{ключевое понятие}
    \scnitem{Глава \ref{chapter_interfaces}~\nameref{chapter_interfaces}}
\end{scnreltolist}

\scnheader{компонентное проектирование}
\scnidtf{\textit{component design}}
\begin{scnreltolist}{ключевой знак}
	\scnitem{Глава \ref{chapter_library}~\nameref{chapter_library}}
\end{scnreltolist}

\scnheader{компонентное проектирование интеллектуальных систем}
\scnidtf{\textit{intelligent systems component design}}
\begin{scnreltolist}{ключевой знак}
	\scnitem{Глава \ref{chapter_library}~\nameref{chapter_library}}
\end{scnreltolist}

\scnheader{компьютерное зрение}	
\scnidtf{\textit{computer vision}}
\begin{scnreltolist}{ключевое понятие}
	\scnitem{\ref{sec_3d_models_computervision}~\nameref{sec_3d_models_computervision}}
\end{scnreltolist}

\scnheader{локальный признак изображения}	
\scnidtf{\textit{local image feature}}
\begin{scnreltolist}{ключевое понятие}
	\scnitem{\ref{sec_3d_models_computervision}~\nameref{sec_3d_models_computervision}}
\end{scnreltolist}

\scnheader{машина обработки знаний}
\scnidtf{\textit{knowledge processing machine}}
\begin{scnreltolist}{ключевой знак}
	\scnitem{Глава \ref{chapter_situation_management}~\nameref{chapter_situation_management}}
\end{scnreltolist}

\scnheader{метод}
\scnidtf{\textit{программа}}
\scnidtf{\textit{method}}
\begin{scnreltolist}{ключевое понятие}
	\scnitem{Глава \ref{chapter_programs}~\nameref{chapter_programs}}
\end{scnreltolist}

\scnheader{метаметод}
\scnidtf{\textit{метапрограмма}}
\scnidtf{\textit{meta-method}}
\begin{scnreltolist}{ключевое понятие}
	\scnitem{Глава \ref{chapter_programs}~\nameref{chapter_programs}}
\end{scnreltolist}

\scnheader{менеджер многократно используемых компонентов ostis-систем}
\scnidtftext{сокращение основного sc-идентификатора}{менеджер компонентов}
\scnidtf{\textit{ostis-systems reusable component manager}}
\scnidtf{\textit{sc-component-manager}}
\begin{scnreltolist}{ключевой знак}
	\scnitem{Глава \ref{chapter_library}~\nameref{chapter_library}}
\end{scnreltolist}

\scnheader{материнская ostis-система}
\scnidtf{\textit{maternal ostis-system}}
\begin{scnreltolist}{ключевой знак}
	\scnitem{\ref{ostis_library_section}~\nameref{ostis_library_section}}
\end{scnreltolist}

\scnheader{Метасистема OSTIS}
\scnidtftext{сокращение основного sc-идентификатора}{Intelligent MetaSystem}
\scnidtf{Интеллектуальная метасистема поддержки проектирования интеллектуальных систем}
\begin{scnreltolist}{ключевой знак}
	\scnitem{\ref{}~\nameref{}}
\end{scnreltolist}

\scnheader{многоагентная система}
\scnidtf{\textit{multiagent system}}
\scnidtf{кибернетическая система, представляющая собой коллектив взаимодействующих кибернетических систем, являющихся агентами (членами) этого коллектива}
\scnidtf{коллективная кибернетическая система}
\begin{scnreltolist}{ключевой знак}
	\scnitem{Глава \ref{chap_intro}~\nameref{chap_intro}}
\end{scnreltolist}

\scnheader{многоагентная модель решения задач}
\scnidtf{\textit{subject area}}
\begin{scnreltolist}{ключевой знак}
	\scnitem{Глава \ref{chapter_new_generation_systems}~\nameref{chapter_new_generation_systems}}
\end{scnreltolist}

\scnheader{многоагентный интерфейс интеллектуальной компьютерной системы нового поколения}
\scnidtf{\textit{subject area}}
\begin{scnreltolist}{ключевой знак}
	\scnitem{Глава \ref{chapter_new_generation_systems}~\nameref{chapter_new_generation_systems}}
\end{scnreltolist}

\scnheader{многоагентный решатель задач}
\scnidtf{\textit{subject area}}
\begin{scnreltolist}{ключевой знак}
	\scnitem{Глава \ref{chapter_new_generation_systems}~\nameref{chapter_new_generation_systems}}
\end{scnreltolist}

\scnheader{многократно используемый компонент ostis-систем}
\scnidtftext{сокращение основного sc-идентификатора}{м.и.к. ostis-систем}
\scnidtf{\textit{ostis-systems reusable component}}
\begin{scnreltolist}{ключевой знак}
	\scnitem{Глава \ref{chapter_library}~\nameref{chapter_library}}
\end{scnreltolist}

\scnheader{многократно используемый компонент решателей задач}
\scnidtf{\textit{reusable problem solver component}}
\begin{scnreltolist}{ключевой знак}
	\scnitem{Глава \ref{chapter_ps_design}~\nameref{chapter_ps_design}}
\end{scnreltolist}

\scnheader{множество}
\scnidtf{\textit{set}}
\begin{scnreltolist}{ключевое понятие}
	\scnitem{Глава \ref{chapter_top_ontologies}~\nameref{chapter_top_ontologies}}
\end{scnreltolist}

\scnheader{минимальная конфигурация ostis-системы}
\scnidtf{\textit{ostis-system minimal configuration}}
\begin{scnreltolist}{ключевой знак}
	\scnitem{Глава \ref{chapter_interpreter}~\nameref{chapter_interpreter}}
\end{scnreltolist}

\scnheader{мультимодальный интерфейс}
\scnidtf{\textit{multimodal interface}}
\begin{scnreltolist}{ключевое понятие}
    \scnitem{Глава \ref{chapter_interfaces}~\nameref{chapter_interfaces}}
\end{scnreltolist}

\scnheader{неатомарный абстрактный sc-агент}
\scnidtf{\textit{non-atomic abstract sc-agent}}
\begin{scnreltolist}{ключевой знак}
	\scnitem{\ref{sec_ps_agents}~\nameref{sec_ps_agents}}
\end{scnreltolist}

\scnheader{некорректность в scp-программе}
\scnidtf{\textit{incorrectness in scp-program}}
\begin{scnreltolist}{ключевой знак}
	\scnitem{\ref{chapter_ps_design}~\nameref{chapter_ps_design}}
\end{scnreltolist}

\scnheader{неролевое отношение}
\scnidtf{\textit{norole relation}}
\begin{scnreltolist}{ключевое понятие}
	\scnitem{Глава \ref{chapter_top_ontologies}~\nameref{chapter_top_ontologies}}
\end{scnreltolist}

\scnheader{объединение*}
\scnidtf{\textit{combination*}}
\begin{scnreltolist}{ключевое отношение}
	\scnitem{Глава \ref{chapter_top_ontologies}~\nameref{chapter_top_ontologies}}
\end{scnreltolist}

\scnheader{онтология}
\scnidtf{\textit{ontology}}
\begin{scnreltolist}{ключевой понятие}
	\scnitem{Глава \ref{chapter_new_generation_systems}~\nameref{chapter_new_generation_systems}}
	\scnitem{\ref{chapter_top_ontologies}~\nameref{chapter_top_ontologies}}
	\scnitem{\ref{sec_ontology}~\nameref{sec_ontology}}
\end{scnreltolist}

\scnheader{онтология верхнего уровня}
\scnidtf{\textit{top-level ontology}}
\begin{scnreltolist}{ключевой понятие}
	\scnitem{\ref{chapter_top_ontologies}~\nameref{chapter_top_ontologies}}
	\scnitem{\ref{sec_top_level_ontologies}~\nameref{sec_top_level_ontologies}}
\end{scnreltolist}


\scnheader{операционная семантика метода*}
\scnidtf{\textit{operational semantics of method*}}
\begin{scnreltolist}{ключевое отношение}
	\scnitem{Глава \ref{chapter_programs}~\nameref{chapter_programs}}
\end{scnreltolist}

\scnheader{операционная семантика языка представления методов*}
\scnidtf{\textit{operational semantics of method representation language*}}
\begin{scnreltolist}{ключевое отношение}
	\scnitem{Глава \ref{chapter_programs}~\nameref{chapter_programs}}
\end{scnreltolist}

\scnheader{оптическая система компьютерного зрения}	
\scnidtf{\textit{optical computer vision system}}
\begin{scnreltolist}{ключевое понятие}
	\scnitem{\ref{sec_3d_models_computervision}~\nameref{sec_3d_models_computervision}}
\end{scnreltolist}

\scnheader{отношение}
\scnidtf{\textit{relation}}
\begin{scnreltolist}{ключевое понятие}
	\scnitem{Глава \ref{chapter_top_ontologies}~\nameref{chapter_top_ontologies}}
\end{scnreltolist}

\scnheader{ошибка в scp-программе}	
\scnidtf{\textit{error in scp-program}}
\begin{scnreltolist}{ключевое понятие}
	\scnitem{\ref{chapter_ps_design}~\nameref{chapter_ps_design}}
\end{scnreltolist}

\scnheader{отношение, заданное на множестве знаний}
\scnidtf{\textit{relation defined on a set of knowledge}}
\begin{scnreltolist}{ключевое понятие}
	\scnitem{\ref{sec_kb}~\nameref{sec_kb}}
\end{scnreltolist}

\scnheader{пакетный менеджер}
\scnidtf{\textit{packet manager}}
\begin{scnreltolist}{ключевой знак}
	\scnitem{\ref{ostis_library_analysis}~\nameref{ostis_library_analysis}}
\end{scnreltolist}

\scnheader{параметр}
\scnidtf{\textit{parameter}}
\begin{scnreltolist}{ключевое понятие}
	\scnitem{Глава \ref{chapter_top_ontologies}~\nameref{chapter_top_ontologies}}
\end{scnreltolist}

\scnheader{параметр, заданный на множестве кибернетических систем}
%\scnidtf{\textit{parameter}}
\scnhaselement{качество физической оболочки кибернетической системы\scnsupergroupsign}
\scnhaselement{качество хранимой информации\scnsupergroupsign}
\begin{scnindent}
	\scnidtf{качество информации, хранимой в памяти кибернетической системы\scnsupergroupsign}
\end{scnindent} 
\scnhaselement{качество решателя задач\scnsupergroupsign}
\scnhaselement{качество интерфейса\scnsupergroupsign}
\scnhaselement{обучаемость\scnsupergroupsign}
\scnhaselement{гибкость кибернетической системы\scnsupergroupsign}
\scnhaselement{стратифицированность кибернетической системы\scnsupergroupsign}
\scnhaselement{рефлексивность кибернетической системы\scnsupergroupsign}
\scnhaselement{уровень эволюционных ограничений\scnsupergroupsign}
\begin{scnreltolist}{ключевой знак}
	\scnitem{Глава \ref{chap_intro}~\nameref{chap_intro}}
\end{scnreltolist}

\scnheader{параметр scp-программы\scnrolesign}
\scnidtf{\textit{scp-program parameter\scnrolesign}}
\begin{scnreltolist}{ключевой знак}
	\scnitem{\ref{sec_ps_scp}~\nameref{sec_ps_scp}}
\end{scnreltolist}

\scnheader{пересечение*}
\scnidtf{\textit{intersection*}}
\begin{scnreltolist}{ключевое отношение}
	\scnitem{Глава \ref{chapter_top_ontologies}~\nameref{chapter_top_ontologies}}
\end{scnreltolist}

\scnheader{планируемая блокировка*}
\scnidtf{\textit{planned lock*}}
\begin{scnreltolist}{ключевой знак}
	\scnitem{\ref{sec_ps_sync}~\nameref{sec_ps_sync}}
\end{scnreltolist}

\scnheader{пользовательский интерфейс}
\scnidtf{\textit{user interface}}
\begin{scnreltolist}{ключевое понятие}
    \scnitem{Глава \ref{chapter_interfaces}~\nameref{chapter_interfaces}}
\end{scnreltolist}

\scnheader{предметная область}
\scnidtf{\textit{subject domain}}
\begin{scnreltolist}{ключевой понятие}
	\scnitem{\ref{chapter_top_ontologies}~\nameref{chapter_top_ontologies}}
	\scnitem{\ref{sec_sd}~\nameref{sec_sd}}
\end{scnreltolist}

\scnheader{приоритет блокировки*}
\scnidtf{\textit{lock priority*}}
\begin{scnreltolist}{ключевой знак}
	\scnitem{\ref{sec_ps_sync}~\nameref{sec_ps_sync}}
\end{scnreltolist}

\scnheader{принадлежность*}
\scnidtf{\textit{belonging*}}
\begin{scnreltolist}{ключевое отношение}
	\scnitem{Глава \ref{chapter_top_ontologies}~\nameref{chapter_top_ontologies}}
\end{scnreltolist}

\scnheader{предметная область}
\scnidtf{\textit{subject area}}
\begin{scnreltolist}{ключевой знак}
	\scnitem{Глава \ref{chapter_kb}~\nameref{chapter_kb}}
\end{scnreltolist}

\scnheader{программный вариант ostis-платформы}
\scnidtf{\textit{software version of ostis-platform}}
\begin{scnreltolist}{ключевой знак}
	\scnitem{Глава \ref{chapter_interpreter}~\nameref{chapter_interpreter}}
\end{scnreltolist}

\scnheader{программный интерфейс}
\scnidtf{\textit{application interface}}
\scnidtf{\textit{API}}
\begin{scnreltolist}{ключевое понятие}
    \scnitem{Глава \ref{chapter_interfaces}~\nameref{chapter_interfaces}}
\end{scnreltolist}

\scnheader{произведение*}
\scnidtf{\textit{multiplication*}}
\begin{scnreltolist}{ключевое отношение}
	\scnitem{Глава \ref{chapter_top_ontologies}~\nameref{chapter_top_ontologies}}
\end{scnreltolist}

\scnheader{процесс}
\scnidtf{\textit{process}}
\begin{scnreltolist}{ключевое понятие}
	\scnitem{Глава \ref{chapter_top_ontologies}~\nameref{chapter_top_ontologies}}
\end{scnreltolist}

\scnheader{расширенная ostis-платформа}
\scnidtf{\textit{extended ostis-platform}}
\begin{scnreltolist}{ключевой знак}
	\scnitem{Глава \ref{chapter_interpreter}~\nameref{chapter_interpreter}}
\end{scnreltolist}

\scnheader{решатель задач пользовательского интерфейса ostis-систем}
\scnidtf{\textit{problem solver of the ostis-system user interface}}
\begin{scnreltolist}{ключевое понятие}
    \scnitem{Глава \ref{chapter_interfaces}~\nameref{chapter_interfaces}}
\end{scnreltolist}

\scnheader{решатель задач ostis-системы}
\scnidtf{\textit{problem solver of ostis-system}}
\begin{scnreltolist}{ключевой знак}
	\scnitem{Глава \ref{chapter_situation_management}~\nameref{chapter_situation_management}}
	\scnitem{Глава \ref{chapter_ps_design}~\nameref{chapter_ps_design}}
\end{scnreltolist}

\scnheader{РЗ}
\scnidtftext{сокращение основного sc-идентификатора}{решатель задач}
\begin{scnreltolist}{ключевой знак}
	\scnitem{\ref{}~\nameref{}}
\end{scnreltolist}

\scnheader{ролевое отношение}
\scnidtf{\textit{role relation}}
\begin{scnreltolist}{ключевое понятие}
	\scnitem{Глава \ref{chapter_top_ontologies}~\nameref{chapter_top_ontologies}}
\end{scnreltolist}

\scnheader{связь}
\scnidtf{\textit{sheaf}}
\begin{scnreltolist}{ключевое понятие}
	\scnitem{Глава \ref{chapter_top_ontologies}~\nameref{chapter_top_ontologies}}
\end{scnreltolist}

\scnheader{ролевое отношение}
\scnidtf{\textit{role relation}}
\begin{scnreltolist}{ключевое понятие}
	\scnitem{Глава \ref{chapter_top_ontologies}~\nameref{chapter_top_ontologies}}
\end{scnreltolist}

\scnheader{связь}
\scnidtf{\textit{sheaf}}
\begin{scnreltolist}{ключевое понятие}
	\scnitem{Глава \ref{chapter_top_ontologies}~\nameref{chapter_top_ontologies}}
\end{scnreltolist}

\scnheader{семантическая окрестность}
\scnidtf{\textit{semantic neighborhood}}
\begin{scnreltolist}{ключевое понятие}
	\scnitem{\ref{sec_sem_neighborhood}~\nameref{sec_sem_neighborhood}}
\end{scnreltolist}

\scnheader{семантическая сеть}
\scnidtf{\textit{semantic web}}
\scnsuperset{рафинированная семантическая сеть}
\scnsuperset{иерархическая семантическая сеть}
\begin{scnindent}
	\scnidtf{семантическая сеть, являющаяся метаграфовой}
\end{scnindent} 
\begin{scnreltolist}{ключевой знак}
	\scnitem{Глава \ref{chapter_new_generation_systems}~\nameref{chapter_new_generation_systems}}
\end{scnreltolist}

\scnheader{Семантическая теория программ для ostis-систем}
\scnidtf{\textit{Semantic program theory for ostis-systems}}
\begin{scnreltolist}{ключевой знак}
	\scnitem{Глава \ref{chapter_programs}~\nameref{chapter_programs}}
\end{scnreltolist}

\scnheader{семейство множеств}
\scnidtf{\textit{set of sets}}
\begin{scnreltolist}{ключевое понятие}
	\scnitem{Глава \ref{chapter_top_ontologies}~\nameref{chapter_top_ontologies}}
\end{scnreltolist}

\scnheader{синтаксис метода*}
\scnidtf{\textit{method syntax*}}
\begin{scnreltolist}{ключевое отношение}
	\scnitem{Глава \ref{chapter_programs}~\nameref{chapter_programs}}
\end{scnreltolist}

\scnheader{система локального позиционирования}	
\scnidtf{\textit{real-time locating system}}
\begin{scnreltolist}{ключевое понятие}
	\scnitem{\ref{sec_3d_models_positioning}~\nameref{sec_3d_models_positioning}}
\end{scnreltolist}

\scnheader{ситуация}	
\scnidtf{\textit{situation}}
\begin{scnreltolist}{ключевое понятие}
	\scnitem{\ref{chapter_top_ontologies}~\nameref{chapter_top_ontologies}}
\end{scnreltolist}

\scnheader{смысловое представление информации}
\begin{scnreltolist}{ключевой знак}
	\scnitem{Глава \ref{chapter_new_generation_systems}~\nameref{chapter_new_generation_systems}}
\end{scnreltolist}

\scnheader{событие в sc-памяти}
\scnidtf{\textit{event in sc-memory}}
\begin{scnreltolist}{ключевое понятие}
	\scnitem{Глава \ref{chapter_top_ontologies}~\nameref{chapter_top_ontologies}}
\end{scnreltolist}

\scnheader{сообщение}
\scnidtf{\textit{message}}
\begin{scnreltolist}{ключевое понятие}
    \scnitem{Глава \ref{chapter_interfaces}~\nameref{chapter_interfaces}}
\end{scnreltolist}

\scnheader{специализированная ostis-платформа}
\scnidtf{\textit{specialized ostis-platform}}
\begin{scnreltolist}{ключевой знак}
	\scnitem{Глава \ref{chapter_interpreter}~\nameref{chapter_interpreter}}
\end{scnreltolist}

\scnheader{спецификация метода*}
\scnidtf{\textit{specification of method*}}
\begin{scnreltolist}{ключевое отношение}
	\scnitem{Глава \ref{chapter_programs}~\nameref{chapter_programs}}
\end{scnreltolist}

\scnheader{спецификация языка представления методов*}
\scnidtf{\textit{specification of method representation language*}}
\begin{scnreltolist}{ключевое отношение}
	\scnitem{Глава \ref{chapter_programs}~\nameref{chapter_programs}}
\end{scnreltolist}

\scnheader{структура}
\scnidtf{\textit{structure}}
\begin{scnreltolist}{ключевое отношение}
	\scnitem{\ref{sec_structure}~\nameref{sec_structure}}
\end{scnreltolist}

\scnheader{сумма*}
\scnidtf{\textit{sum*}}
\begin{scnreltolist}{ключевое отношение}
	\scnitem{Глава \ref{chapter_top_ontologies}~\nameref{chapter_top_ontologies}}
\end{scnreltolist}

\scnheader{сцена в трехмерном представлении}	
\scnidtf{\textit{scene in 3D representation}}
\begin{scnreltolist}{ключевое понятие}
	\scnitem{\ref{sec_3d_models_semantics}~\nameref{sec_3d_models_semantics}}
\end{scnreltolist}

\scnheader{технология}
\scnidtf{комплекс моделей, методик, методов и средств, обеспечивающих выполнение соответствующего вида деятельности}
\scnsuperset{технология поддержки жизненного цикла}
\begin{scnindent}
	\scnsuperset{технология комплексной поддержки жизненного цикла}
	\begin{scnindent}
		\scnsuperset{технология комплексной поддержки жизненного цикла интеллектуальных компьютерных систем нового поколения}
		\begin{scnindent}
			\scnhaselement{Технология OSTIS}
			\begin{scnindent}
				\scnidtf{Технология комплексной поддержки жизненного цикла ostis-систем}
			\end{scnindent}
		\end{scnindent} 
	\end{scnindent} 
\end{scnindent} 
\begin{scnreltolist}{ключевой знак}
	\scnitem{Часть \ref{part1}~\nameref{part1}}
\end{scnreltolist}

\scnheader{Технология OSTIS}
\begin{scnreltolist}{ключевой знак}
	\scnitem{Глава \ref{}~\nameref{}}
\end{scnreltolist}

\scnheader{технология проектирования интеллектуальных систем}
\scnidtf{\textit{intelligent systems design technology}}
\begin{scnreltolist}{ключевой знак}
	\scnitem{Глава \ref{chapter_library}~\nameref{chapter_library}}
\end{scnreltolist}

\scnheader{тип блокировки}
\scnidtf{\textit{lock type}}
\begin{scnreltolist}{ключевой знак}
	\scnitem{\ref{sec_ps_sync}~\nameref{sec_ps_sync}}
\end{scnreltolist}

\scnheader{точка останова}
\scnidtf{\textit{breakpoint}}
\begin{scnreltolist}{ключевой знак}
	\scnitem{\ref{chapter_ps_design}~\nameref{chapter_ps_design}}
\end{scnreltolist}

\scnheader{точность*}
\scnidtf{\textit{accuracy*}}
\begin{scnreltolist}{ключевое отношение}
	\scnitem{Глава \ref{chapter_top_ontologies}~\nameref{chapter_top_ontologies}}
\end{scnreltolist}

\scnheader{транзакция в sc-памяти}	
\scnidtf{\textit{transaction in sc-memory}}
\begin{scnreltolist}{ключевой знак}
	\scnitem{\ref{sec_ps_sync}~\nameref{sec_ps_sync}}
\end{scnreltolist}

\scnheader{трехмерная модель объекта}	
\scnidtf{\textit{object 3D model}}
\begin{scnreltolist}{ключевое понятие}
	\scnitem{\ref{sec_3d_models_representation}~\nameref{sec_3d_models_representation}}
\end{scnreltolist}

\scnheader{трехмерная реконструкция}	
\scnidtf{\textit{3D reconstruction}}
\begin{scnreltolist}{ключевое понятие}
	\scnitem{\ref{sec_3d_models_reconstruction}~\nameref{sec_3d_models_reconstruction}}
\end{scnreltolist}

\scnheader{трехмерное представление объекта}	
\scnidtf{\textit{object 3D representation}}
\begin{scnreltolist}{ключевое понятие}
	\scnitem{\ref{sec_3d_models_representation}~\nameref{sec_3d_models_representation}}
\end{scnreltolist}

\scnheader{удаляемые sc-элементы*}
\scnidtf{\textit{sc-elements to be deleted*}}
\begin{scnreltolist}{ключевой знак}
	\scnitem{\ref{sec_ps_sync}~\nameref{sec_ps_sync}}
\end{scnreltolist}

\scnheader{умная больница}
\scnidtf{smart-больница}
\begin{scnreltolist}{ключевой знак}
	\scnitem{\ref{}~\nameref{}}
\end{scnreltolist}

\scnheader{Умное общество}
\scnidtf{smart-общество}
\scnidtf{Общество 5.0}
\scnidtf{Интеллектуальное общество}
\begin{scnreltolist}{ключевой знак}
	\scnitem{\ref{}~\nameref{}}
\end{scnreltolist}

\scnheader{умное предприятие}
\scnidtf{smart-предприятие}
\scnidtf{предприятие 5.0}
\begin{scnreltolist}{ключевой знак}
	\scnitem{\ref{}~\nameref{}}
\end{scnreltolist}

\scnheader{умный город}
\scnidtf{smart-город}
\begin{scnreltolist}{ключевой знак}
	\scnitem{\ref{}~\nameref{}}
\end{scnreltolist}

\scnheader{умный дом}
\scnidtf{smart-дом}
\begin{scnreltolist}{ключевой знак}
	\scnitem{\ref{}~\nameref{}}
\end{scnreltolist}

\scnheader{УСК}
\scnidtftext{сокращение основного sc-идентификатора}{универсальный семантический код}
\scnidtf{универсальный семантический код, разработанный В.~В. Мартыновым}
\begin{scnreltolist}{ключевой знак}
	\scnitem{\ref{}~\nameref{}}
\end{scnreltolist}

\scnheader{физический интерфейс}
\scnidtf{\textit{physical interface}}
\begin{scnreltolist}{ключевое понятие}
    \scnitem{Глава \ref{chapter_interfaces}~\nameref{chapter_interfaces}}
\end{scnreltolist}

\scnheader{число}
\scnidtf{\textit{number}}
\begin{scnreltolist}{ключевое понятие}
	\scnitem{Глава \ref{chapter_top_ontologies}~\nameref{chapter_top_ontologies}}
\end{scnreltolist}

\scnheader{эффективность метода}
\scnidtf{\textit{method efficiency}}
\begin{scnreltolist}{ключевое понятие}
	\scnitem{Глава \ref{chapter_programs}~\nameref{chapter_programs}}
\end{scnreltolist}

\scnheader{язык представления методов}
\scnidtf{язык программирования}
\scnidtf{\textit{method representation language}}
\begin{scnreltolist}{ключевое понятие}
	\scnitem{Глава \ref{chapter_programs}~\nameref{chapter_programs}}
\end{scnreltolist}

\scnheader{Язык SCP}
\scnidtf{\textit{SCP Language}}
\begin{scnreltolist}{ключевой знак}
	\scnitem{\ref{sec_ps_scp}~\nameref{sec_ps_scp}}
\end{scnreltolist}

\scnheader{ACL}
\scnidtftext{сокращение основного sc-идентификатора}{Agent Communication Language}
\scnidtf{Язык взаимодействия агентов, предложенный FIPA в качестве стандарта}
\begin{scnreltolist}{ключевой знак}
	\scnitem{\ref{}~\nameref{}}
\end{scnreltolist}

\scnheader{FIPA}
\scnidtftext{сокращение основного sc-идентификатора}{Foundation for Intelligent Physical Agents}
\scnidtf{Организация, осуществляющая разработку и продвижение стандартов в области многоагентных систем}
\begin{scnreltolist}{ключевой знак}
	\scnitem{\ref{}~\nameref{}}
\end{scnreltolist}

\scnheader{GPS}
\scnidtftext{сокращение основного sc-идентификатора}{General Problem Solver}
\scnidtf{Компьютерная программа, созданная в 1959 г. и предназначенная для работы в качестве универсальной машины для решения задач, сформулированных на языке хорновских дизъюнкторов}
\begin{scnreltolist}{ключевой знак}
	\scnitem{\ref{}~\nameref{}}
\end{scnreltolist}

\scnheader{help-система}
\begin{scnreltolist}{ключевой знак}
	\scnitem{Глава \ref{}~\nameref{}}
\end{scnreltolist}

\scnheader{IACPaaS}
\scnidtftext{сокращение основного sc-идентификатора}{Intelligent Applications, Control and Platform as a Service}
\scnidtf{Исследовательская облачная платформа, объединяющая различные модели парадигмы облачных вычислений}
\begin{scnreltolist}{ключевой знак}
	\scnitem{\ref{}~\nameref{}}
\end{scnreltolist}

\scnheader{JSON}
\scnidtftext{сокращение основного sc-идентификатора}{JavaScript Object Notation}
\begin{scnreltolist}{ключевой знак}
	\scnitem{Глава \ref{}~\nameref{}}
\end{scnreltolist}

\scnheader{KIF}
\scnidtftext{сокращение основного sc-идентификатора}{Knowledge Interchange Language}
\scnidtf{Компьютерноориентированный язык для обмена знаниями между различными компьютерными программами}
\begin{scnreltolist}{ключевой знак}
	\scnitem{\ref{}~\nameref{}}
\end{scnreltolist}

\scnheader{KQML}
\scnidtftext{сокращение основного sc-идентификатора}{Knowledge Query and Manipulation Language}
\scnidtf{Язык взаимодействия между программными агентами и системами, основанными на знаниях}
\begin{scnreltolist}{ключевой знак}
	\scnitem{\ref{}~\nameref{}}
\end{scnreltolist}

\scnheader{Node-RED}
\begin{scnreltolist}{ключевой знак}
	\scnitem{Глава \ref{}~\nameref{}}
\end{scnreltolist}

\scnheader{OSTIS}
\scnidtftext{сокращение основного sc-идентификатора}{Open Semantic Technology for Intelligent Systems}
\scnidtftext{сокращение основного sc-идентификатора}{Открытая семантическая технология проектирования интеллектуальных систем}
\begin{scnreltolist}{ключевой знак}
	\scnitem{\ref{}~\nameref{}}
\end{scnreltolist}

\scnheader{ostis-}
\scnidtftext{сокращение основного sc-идентификатора}{for open semantic technology for intelligent systems}
\scntext{пример применения}{
	\begin{scnitemize}
		\item ostis-система
		\item ostis-платформа
		\item ostis-сообщество
	\end{scnitemize}
}
\begin{scnreltolist}{ключевой знак}
	\scnitem{Глава \ref{}~\nameref{}}
\end{scnreltolist}

\scnheader{ostis-ассистент}
\begin{scnreltolist}{ключевой знак}
	\scnitem{Глава \ref{}~\nameref{}}
\end{scnreltolist}

\scnheader{ostis-платформа}
\scnidtf{\textit{ostis-platform}}
\begin{scnreltolist}{ключевой знак}
	\scnitem{Глава \ref{chapter_interpreter}~\nameref{chapter_interpreter}}
\end{scnreltolist}

\scnheader{ostis-технология}
\scnidtf{частная технология, входящая в состав комплексной Технологии OSTIS}
\begin{scnreltolist}{ключевой знак}
	\scnitem{Глава \ref{}~\nameref{}}
\end{scnreltolist}

\scnheader{OWL}
\scnidtftext{сокращение основного sc-идентификатора}{Web Ontology Language}
\scnidtf{Язык описания онтологий для семантической паутины}
\begin{scnreltolist}{ключевой знак}
	\scnitem{\ref{}~\nameref{}}
\end{scnreltolist}

\scnheader{QA3}
\scnidtftext{сокращение основного sc-идентификатора}{Question Answer system ver. 3}
\scnidtf{Вопросно-ответная дедуктивная система, созданная в 1969 г. на языке LISP}
\begin{scnreltolist}{ключевой знак}
	\scnitem{\ref{}~\nameref{}}
\end{scnreltolist}

\scnheader{RDF}
\scnidtftext{сокращение основного sc-идентификатора}{Resource Description Framework}
\scnidtf{Разработанная Консорциумом Всемирной паутины модель для пред¬ставления данных}
\begin{scnreltolist}{ключевой знак}
	\scnitem{\ref{}~\nameref{}}
\end{scnreltolist}

\scnheader{SC-}
\scnidtftext{сокращение основного sc-идентификатора}{Semantic Code}
\scnidtftext{сокращение основного sc-идентификатора}{Semantic Computer}
\scntext{пример применения}{
	\begin{scnitemize}
		\item SC-код
		\item sc-элемент
		\item sc-множество
		\item sc-модель
	\end{scnitemize}
}
\begin{scnreltolist}{ключевой знак}
	\scnitem{Глава \ref{}~\nameref{}}
\end{scnreltolist}

\scnheader{sc-агент}
\scnidtf{\textit{sc-agent}}
\begin{scnreltolist}{ключевой знак}
	\scnitem{\ref{sec_ps_agents}~\nameref{sec_ps_agents}}
\end{scnreltolist}

\scnheader{sc-идентификатор}
\scnidtf{\textit{sc-identifier}}
\begin{scnreltolist}{ключевой знак}
	\scnitem{Глава \ref{chapter_ext_lang}~\nameref{chapter_ext_lang}}
\end{scnreltolist}

\scnheader{SCg-код}
\scnidtf{\textit{SCg-code}}
\scnidtftext{сокращение основного sc-идентификатора}{Semantic Code graphic}
\scnidtf{Графический нелинейный вариант визуализации текстов SC-кода}
\begin{scnreltolist}{ключевой знак}
	\scnitem{\nameref{chapter_ext_lang}}
\end{scnreltolist}

\scnheader{SСfin-код}
\begin{scnreltolist}{ключевой знак}
	\scnitem{Глава \ref{}~\nameref{}}
\end{scnreltolist}

\scnheader{SСin-код}
\begin{scnreltolist}{ключевой знак}
	\scnitem{Глава \ref{}~\nameref{}}
\end{scnreltolist}

\scnheader{sc-JSON}
\begin{scnreltolist}{ключевой знак}
	\scnitem{Глава \ref{}~\nameref{}}
\end{scnreltolist}

\scnheader{\scnfilelong{\textbf{\textit{SC-код}}}}
\scnidtf{\textit{SC-code}}
\scnidtftext{сокращение основного sc-идентификатора}{Semantic Code}
\scnidtf{Универсальный базовый способ смыслового представления знаний в виде семантических сетей с базовой теоретикомножественной интерпретацией}
\scnrelto{часто используемый sc-идентификатор}{sc-структура}
\scniselement{часто используемый sc-идентификатор}
\scniselement{имя собственное}
\scnidtf{sc-конструкция}
\begin{scnreltolist}{ключевой знак}
	\scnitem{Глава \ref{chapter_new_generation_systems}~\nameref{chapter_new_generation_systems}}
\end{scnreltolist}
\begin{scnreltolist}{ключевой термин}
	\scnitem{Параграф \ref{sec_external_information_constructs_external_lang}~\nameref{sec_external_information_constructs_external_lang}}
\end{scnreltolist}

\scnheader{SCn-код}
\scnidtf{\textit{SCn-code}}
\scnidtftext{сокращение основного sc-идентификатора}{Semantic Code natural}
\scnidtf{Гипертекстовый вариант визуализации текстов \textit{SC-кода}}
\begin{scnreltolist}{ключевой знак}
	\scnitem{\nameref{chapter_ext_lang}}
\end{scnreltolist}

\scnheader{SCs-код}
\scnidtf{\textit{SCs-code}}
\begin{scnreltolist}{ключевой знак}
	\scnitem{\nameref{chapter_ext_lang}}
\end{scnreltolist}

\scnheader{sc-машина}
\scnidtf{\textit{sc-machine}}
\begin{scnreltolist}{ключевой знак}
	\scnitem{Глава \ref{chapter_interpreter}~\nameref{chapter_interpreter}}
\end{scnreltolist}

\scnheader{sc-память}
\scnidtf{\textit{sc-memory}}
\begin{scnreltolist}{ключевой знак}
	\scnitem{Глава \ref{}~\nameref{}}
\end{scnreltolist}

\scnheader{sc-элемент}
\scnidtf{\textit{sc-element}}
\scnrelto{часто используемый sc-идентификатор}{сущность}
\begin{scnreltolist}{ключевой знак}
	\scnitem{Глава \ref{chapter_new_generation_systems}~\nameref{chapter_new_generation_systems}}
	\scnitem{Глава \ref{chapter_top_ontologies}~\nameref{chapter_top_ontologies}}
	\scnitem{Глава \ref{chapter_sc_code}~\nameref{chapter_sc_code}}
\end{scnreltolist}

\scnheader{sc-язык}
\scnidtf{\textit{sc-language}}
\scnidtf{подъязык \textit{SC-кода}}
\begin{scnreltolist}{ключевой знак}
	\scnitem{Глава \ref{}~\nameref{}}
\end{scnreltolist}

\scnheader{SCP}
\scnidtftext{сокращение основного sc-идентификатора}{Semantic Code Programming}
\scnidtf{Графовый процедурный язык программирования, построенный на базе \textit{SC-кода}}
\begin{scnreltolist}{ключевой знак}
	\scnitem{\nameref{chapter_ext_lang}}
\end{scnreltolist}

\scnheader{scp-операнд\scnrolesign}
\scnidtf{\textit{scp-operand\scnrolesign}}
\begin{scnreltolist}{ключевой знак}
	\scnitem{\ref{sec_ps_scp}~\nameref{sec_ps_scp}}
\end{scnreltolist}

\scnheader{scp-оператор}
\scnidtf{\textit{scp-operator}}
\begin{scnreltolist}{ключевой знак}
	\scnitem{\ref{sec_ps_scp}~\nameref{sec_ps_scp}}
\end{scnreltolist}

\scnheader{SPARQL}
\scnidtftext{сокращение основного sc-идентификатора}{SPARQL Protocol and RDF Query Language}
\scnidtf{Язык запросов к данным (является рекомендацией консорциума \textit{W3C} и одной из технологий семантической паутины), представленным по модели \textit{RDF}, а также протокол для передачи этих запросов и ответов на них}
\begin{scnreltolist}{ключевой знак}
	\scnitem{\ref{}~\nameref{}}
\end{scnreltolist}

\scnheader{SQL}
\scnidtftext{сокращение основного sc-идентификатора}{Structured Query Language}
\scnidtftext{сокращение основного sc-идентификатора}{язык структурированных запросов}
\scnidtf{Универсальный язык запросов, применяемый для создания, модификации и управления данными в реляционных базах данных}
\begin{scnreltolist}{ключевой знак}
	\scnitem{\ref{}~\nameref{}}
\end{scnreltolist}

\scnheader{STRIPS}
\scnidtftext{сокращение основного sc-идентификатора}{Stanford Research Institute Problem Solver}
\scnidtf{Планирующая система, использующая декларативно- процедуральное предс1авление знаний в сочетании с эвристическим поиском, создана в 1971~г.}
\begin{scnreltolist}{ключевой знак}
	\scnitem{\ref{}~\nameref{}}
\end{scnreltolist}

\scnheader{W3C}
\scnidtftext{сокращение основного sc-идентификатора}{World Wide Web Consortium}
\scnidtf{Консорциум Всемирной паутины, организация, разрабатывающая и внедряющая технологические стандарты для Всемирной паутины}
\begin{scnreltolist}{ключевой знак}
	\scnitem{\ref{}~\nameref{}}
\end{scnreltolist}

\scnheader{Yandex Cloud}
\begin{scnreltolist}{ключевой знак}
	\scnitem{Глава \ref{}~\nameref{}}
\end{scnreltolist}

\scnheader{YandexIoTCore}
\begin{scnreltolist}{ключевой знак}
	\scnitem{Глава \ref{}~\nameref{}}
\end{scnreltolist}

\scnheader{WebSocket}
\begin{scnreltolist}{ключевой знак}
	\scnitem{Глава \ref{}~\nameref{}}
\end{scnreltolist}

\end{SCn}	