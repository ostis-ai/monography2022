\chapter{Методика и средства компонентного проектирования интерфейсов ostis-систем}
\chapauthortoc{Садовский М.~Е.\\Жмырко А.~В.}
\label{chapter_ui_design}

\vspace{-7\baselineskip}

\begin{SCn}
\begin{scnrelfromlist}{автор}
	\scnitem{Садовский М.~Е.}
	\scnitem{Жмырко А.~В.}
\end{scnrelfromlist}
\bigskip
	
\begin{scnrelfromlist}{подраздел}
	\scnitem{\ref{sec_analysis_UI_design_methodologies}~\nameref{sec_analysis_UI_design_methodologies}}
	\scnitem{\ref{sec_reusable_UI_components}~\nameref{sec_reusable_UI_components}}
\end{scnrelfromlist}	
	
	
\scntext{аннотация}{Проектирование интерфейса – это один из наиболее важных  этапов разработки любой системы.
Пользователь при обращении с интерфейсом должен представить себе, какая информация о выполняемой задаче у него существует, и в каком состоянии находятся средства, с помощью которых он будет решать данную задачу. Эффективность работы пользователя и его интерес обеспечивает правильно сформулированная методика разработки и проектирования пользовательского интерфейса. \newline
В настоящее время организация взаимодействия пользователя с компьютерной системой лежит парадигма \textbf{грамотного пользователя},	который знает, как управлять системой и несёт полную ответственность за качество взаимодействия с ней.
Многообразие форм и видов интерфейсов приводит к необходимости пользователя  адаптироваться к каждой конкретной системе, обучаться принципам взаимодействия с ней для решения необходимых ему задач. \newline
Проектирование пользовательских интерфейсов включает в себя ряд последовательных этапов.
В рамках главы будут рассмотрены этапы проектирования традиционных пользовательских интерфейсов и этапы проектирования адаптивных интеллектуальных мультимодальных пользовательских интерфейсов.}
\end{SCn}

% ПОСМОТРЕТЬ \scncite{Koronchik2011}

\section{Анализ методик проектирования пользовательских интерфейсов}
\label{sec_analysis_UI_design_methodologies}

Среди существующих методик проектирования адаптивных интеллектуальных мультимодальных пользовательских интерфейсов можно выделить методики,
предложенные в \scncite{Ehlert2003} и  \scncite{Kong2011}.

В рамках работы \scncite{Ehlert2003} выделяется 4 основных этапа проектирования:
\begin{textitemize}
    \item анализ;
    \item разработка интерфейса;
    \item оценка интерфейса;
    \item доработка и усовершенствование.
\end{textitemize}

Этап анализа является, вероятно, самой важной фазой в любом процессе проектирования, но тем более в проектировании интерфейсов ostis-систем. В
процессе проектирования обычного неинтеллектуального интерфейса
необходимо проанализировать, кто является обычным пользователем, какие задачи интерфейс должен поддерживать. 

В пользовательском интерфейсе часто нет среднего пользователя.
В идеале, пользовательский интерфейс должен быть способен адаптироваться к любому пользователю в любой среде. Поэтому используемая техника адаптации должна быть разработана таким образом, чтобы она могла поддерживать все типы пользователей.

Этап \textit{анализа} включает выполнение четырех взаимосвязанных видов анализа:
\begin{textitemize}
    \item функциональный анализ;
    \item анализ данных;
    \item анализ пользователей;
    \item анализ среды.
\end{textitemize}

В рамках \textit{функционального анализа} необходимо дать ответ на вопрос: "каковы \uline{основные функции системы}?".
В рамках \textit{анализа данных} необходимо определить \uline{значение и структуру данных}, используемых в приложении.
В рамках \textit{анализа пользователей} необходимо выделить \uline{типы пользователей и их возможности} в интеллектуальном
и когнитивном плане.
В рамках \textit{анализа среды} необходимо определить \uline{требования, предъявляемые к среде}, в которой будет работать система.

Результатом данного этапа является \uline{cпецификация целей и информационных потребностей пользователя}, а также
\uline{спецификация функций и информации}, которые требуются системе.

\textit{Разработка интерфейса} включает следующие шаги:
\begin{textitemize}
	\item \textit{создание модели интерфейса} в соответствии с этапом анализа;
	\item реализация модели интерфейса.
\end{textitemize}

Результатом данного этапа является \uline{пользовательский интерфейс}, который, по мнению разработчика, удовлетворяет требованиям пользователей и соответствует требованиям, сформулированным на этапе анализа.

\textit{Оценка интерфейса} предполагает, что:
\begin{textitemize}
	\item требования, которые были сформулированы на этапе анализа, должны быть удовлетворены;
	\item эффективность модели интерфейса должна быть исследована.
\end{textitemize}

На этапе \textit{оценки интерфейса} необходимо вернуться к требованиям \textit{этапа анализа}. Требования, которые
были сформулированы на \textit{этапе анализа}, должны быть выполнены, а также должна быть исследована эффективность модели интерфейса.
Чтобы определить эту эффективность, необходимо определить критерии эффективности.

Очень важным, но субъективным критерием является удовлетворенность пользователя. Поскольку пользователь должен работать с интерфейсом, он имеет право голоса в вопросе о том, удобно ли работать с интерфейсом и т.п.

Критериями эффективности могут выступать различные показатели, такие как:
\begin{textitemize}
	\item количество ошибок;
	\item время выполнения задачи;
	\item отношение пользователя к интерфейсу;
	\item т.д.
\end{textitemize}

\textit{Доработка и усовершенствование} осуществляется на основе проблем, выявленых на этапе оценки. В рамках данного этапа вносится ряд улучшений в модель интерфейса. Затем начинается новый цикл проектирования. Этот итеративный процесс будет продолжаться до тех пор, пока результат оценки не будет удовлетворять обозначенным требованиям. 

Методика, предложенная в \scncite{Kong2011} включает 6 этапов:
\begin{textitemize}
	\item моделирование пользовательского интерфейса (описание абстрактного пользовательского интерфейса);
	\item проектирование пользовательского интерфейса по умолчанию (стандартная версия, конкретный пользовательский интерфейс);
	\item разработка пользовательского интерфейса (расширение или замена пользовательского интерфейса по умолчанию) - этот шаг опускается, когда система генерирует пользовательский интерфейс по умолчанию автоматически;
	\item создание контекста использования (идентификация и создание контекста использования - модели пользователя, модель устройства и модель среды/платформы);
	\item адаптация пользовательского интерфейса - автоматически - (адаптация пользовательского интерфейса во время выполнения для соответствия конкретного контекста использования);
	\item кастомизация пользовательского интерфейса - настройка пользовательского интерфейса самим пользователем (адаптируемость).
\end{textitemize}

На основе рассмотренных методик проектирования интерфейсов можно выделить следующие общие этапы:
\begin{textitemize}
\item анализ контекста использования и задач пользователей;
\item проектирование и разработка интерфейса;
\item оценка качества спроектированного интерфейса.
\end{textitemize}

Среди недостатков предложенных подходов можно выделить:
\begin{textitemize}
	\item знания по каждому этапу проектирования находятся у разных специалистов в неформализорованном неунифицированном виде;
	\item отсутствие этапа формализованного документирования этапов проектирования приводит в дальнейшем в необходимости создания отдельных help-систем для пользователей, разработчиков и т.д.
\end{textitemize}

Предлагается использовать онтологический подход на основе семантической модели в процессе проектирования и реализации адаптивного интеллектуального мультимодального пользовательского интерфейса. Такой интерфейс предлагается рассматривать как специ-
ализированную подсистему для решения интерфейсных задач пользователя, состоящую из базы знаний и решателя интерфейсных задач. 

Описание модели базы знаний и решателя предлагается осуществлять на основе универсального унифицированного языка представления знаний, что обеспечит совместимость между этими компонентами.

Архитектура интерфейса такой системы была рассмотрена на рисунке \nameref{fig:archit_intel_system}.

\begin{figure}[h]
	\centering
	\includegraphics[scale=0.15]{images/part5/sc-model-ui.png}
	\caption{Архитектура интеллектуального интерфейса}
	\label{fig:archit_intel_system}
\end{figure}

Таким образом, предлагаемая методика проектирования интерфейсов ostis-систем будет включать:
\begin{textitemize}
\item анализ пользователя, его задач и целей, а также контекста использования;
\item анализ требований к пользовательскому интерфейсу;
\item проектирование пользовательского интерфейса по умолчанию;
\item разработка пользовательского интерфейса;
\item анализ пользовательского интерфейса и его адаптации.
\end{textitemize}

Поскольку знания о конкретном этапе обычно находятся у разных экспертов, особенностью предлагаемого подхода является обязательное формализованное документирование знаний в унифицированном виде и применение на каждом из этапов компонентного подхода.

Для применения компонентного подхода предлагается использовать \textit{библиотеку многократно используемых компонентов} базы знаний, решателя задач и интерфейса.

\textbf{Анализ пользователя, его задач и целей, а также контекста использования}

Результаты первого этапа, такие как: модель конкретного пользователя, его потребности и контекст использования системы (устройство, окружающая среда) должны быть формализованы в рамках соответствующих онтологий базы знаний интерфейса. 
При этом в процессе формализации по необходимости должны быть переиспользованы компоненты базы знаний из библиотеки многократно используемых компонентов, а новые компоненты могут пополнить эту же библиотеку.

\textbf{Анализ требований к пользовательскому интерфейсу}

Результатом второго этапа являются конечные требования к интерфейсу, которые должны быть сформулированы относительно модели пользователя и его цели, а также относительно контекста использования.

Результаты должны быть также формализованы, а в процессе выполнения могут быть использованы существующие компонент базы знаний из библиотеки многократно используемых компонентов.

\textbf{Проектирование пользовательского интерфейса по умолчанию}

В соответствии с требованиями к пользовательскому интерфейсу, строится модель адаптивного интеллектуального мультимодального пользовательского интерфейса, которая является результатом третьего этапа.

Такая модель будет включать в себя формализованную модель базы знаний и решателя задач.

При проектировании могут быть использованы компоненты интерфейса, базы знаний и решателя задач. 
Компоненты будут записаны в унифицированном виде, что позволит обеспечить их автоматическую совместимость.

\textbf{Разработка пользовательского интерфейса}

Результатом четвертого этапа является реализация спроектированного пользовательского интерфейса. В данном случае следует использовать готовые компоненты интерфейса из библиотеки многократно используемых компонентов интерфейса.

\textbf{Анализ пользовательского интерфейса и его адаптации}

На данном этапе используются готовые компоненты решателя задач.

Таким образом будет сформирована база знаний проектируемого интерфейса, которая автоматически может быть использована в качестве help-системы для пользователей, разработчиков и т.д.

\subsection{Анализ методов оценки пользовательских интерфейсов}
\label{sec_UI_analisys}

Как было указано выше, этап оценки пользовательских интерфейсов является одним из важных этапов.
Оценка пользовательского интерфейса необходима для улучшения коммуникации между интеллектуальными системами и их пользователями. Существует множество методов для оценки пользовательских интерфейсов, направленных, в основном, на выявление проблем с использованием системы и минимизацию риска ошибок. Однако до сих пор нет комплексного подхода к оценке пользовательских интерфейсов интеллектуальных систем.

При оценке пользовательских интерфейсов большую роль играет человеческий фактор, а основными участниками являются пользователи и эксперты. Очень сложно оценить правильность решения по адаптации пользовательского интерфейса интеллектуальной системы и оценить его без учета человеческого фактора. Конечно, есть небольшие обобщенные правила построения пользовательских интерфейсов, такие как:

\begin{textitemize}
	\item интерфейс должен быть интуитивно понятен для конечного пользователя;
	\item интерфейс должен быть доступным для пользователей с ограниченными возможностями и пользователей, впервые сталкивающимися с информационными технологиями;
	\item достижение цели пользователем должно осуществляться наименее возможным количеством шагов.
\end{textitemize}


Оценка пользовательских интерфейсов интеллектуальных систем имеет свои особенности и требует специальных методов и инструментов.

Некоторые методы оценки пользовательских интерфейсов интеллектуальных систем включают:
\begin{textitemize}
	\item Оценка точности и полноты - это метод, при котором оценивается точность и полнота ответов системы на запросы пользователей. Это может быть выполнено с помощью тестовых сценариев, которые содержат известные ответы, а также с помощью анализа логов и пользовательских отзывов.
	
	Точность означает, насколько хорошо пользовательский интерфейс отражает действительный опыт пользователя и предлагает ему наиболее подходящий контент в соответствии с его потребностями и предпочтениями. Это может быть оценено путем анализа, насколько легко использовать интерфейс, насколько он отражает действительные возможности интеллектуальной системы и насколько он помогает пользователям достигать своих целей более эффективно.
	
	Полнота, с другой стороны, означает, насколько хорошо пользовательский интерфейс предоставляет всю необходимую информацию и функциональность, которые пользователь может потребовать в процессе использования системы. Это может быть оценено путем анализа того, насколько полное и четкое описание функций, возможностей и ограничений системы предоставляется через пользовательский интерфейс, в каком объеме пользователю доступна необходимая информация для принятия решений и насколько система может эффективно реагировать на запросы и потребности пользователя.
	
	\item Оценка персонализации - это метод, при котором оценивается способность системы адаптироваться к потребностям и предпочтениям каждого пользователя. Это может быть выполнено с помощью анализа пользовательских данных, а также проведения опросов среди пользователей, чтобы определить, насколько хорошо система учитывает их потребности.
	
	Оценка качества персонализации может включать в себя измерение следующих параметров:
	\begin{textitemize}
	\item Работоспособность - действительно ли пользовательский интерфейс адаптивный и соответствует ли он потребностям пользователя?
	\item Уникальность - насколько уникальным и индивидуальным является персонализированный пользовательский интерфейс?
	\item Понятность - насколько просто и легко пользователь может настроить и использовать персонализацию?
	\item Эффективность - насколько хорошо персонализированный пользовательский интерфейс помогает пользователю в выполнении задач?
	\item Удовлетворенность - насколько положительным и удовлетворительным является пользовательский интерфейс для пользователя?
	\end{textitemize}

	Оценка персонализации адаптивных пользовательских интерфейсов может быть выполнена различными методами, включая тестирование с использованием фокус-групп, опросы пользователей, анализ данных и др.
	Важно отметить, что оценка персонализации должна проходить на всех этапах разработки и используется для повышения пользовательского опыта и улучшения качества пользовательских интерфейсов интеллектуальных систем.
	\item Оценка безопасности - это метод, при котором оценивается уровень безопасности системы и ее способность защищать конфиденциальные данные пользователей. Это может быть выполнено с помощью тестирования на проникновение и анализа системных журналов, чтобы выявить потенциальные уязвимости системы.
	
	Важно убедиться, что пользовательский интерфейс обеспечивает адекватный уровень защиты данных, например, путем шифрования данных, контроля доступа и механизмов аутентификации.
	\item Оценка степени автоматизации - это метод, при котором оценивается степень автоматизации системы и ее способность выполнять задачи без участия пользователя. Это может быть выполнено с помощью анализа функциональности системы и проведения опросов среди пользователей, чтобы определить, насколько хорошо система выполняет задачи автоматически.
\end{textitemize}










Существуют различные методы оценки качества пользовательского интерфейса интеллектуальных систем, включая:
\begin{textitemize}
	\item тестирование удобства использования;
	\item экспертная оценка;
	\item отслеживание движения глаз;
	\item "когнитивные прогулки";
	\item A/B тестирование.
\end{textitemize}

\textbf{Тестирование удобства использования пользовательского интерфейса} 

Тестирование удобства использования для пользовательских интерфейсов является важной частью процесса проектирования и разработки. Тестирование удобства использования гарантирует, что эти интерфейсы удобны и полезны для их потенциальных пользователей.

Тестирование удобства использования, как правило, включает несколько этапов:

\begin{textitemize}
	\item Определение целевых пользователей. Это может потребовать проведения исследований пользователей для понимания потребностей и предпочтений потенциальных пользователей.
	\item Разработка тестовых сценариев. Следующим шагом является разработка тестовых сценариев, которые отражают типичные случаи использования интерфейса. Эти сценарии должны быть разработаны для проверки адаптивных возможностей интерфейса, а также его удобства использования.
	\item Проведение тестирования пользователей. Тестирование пользователей включает в себя наблюдение за пользователями во время взаимодействия с интерфейсом интеллектуальной системы и просьбу выполнить задачи, связанные с тестовыми сценариями. Тестирование пользователей может проводиться в контролируемой лабораторной среде или удаленно с использованием технологии совместного просмотра экрана.
	\item Анализ результатов тестирования. Результаты тестирования пользователей анализируются для выявления проблем с удобством использования и областей для улучшения. Это может включать проведение эвристической оценки, где эксперты по удобству использования оценивают интерфейс на основе набора установленных принципов удобства использования.
	\item Итерация дизайна. На основе результатов тестирования пользователей и анализа дизайн интерфейса может быть изменен для устранения проблем с удобством использования и улучшения общего пользовательского опыта..
\end{textitemize}

\textbf{Экспертная оценка} 

Достаточно часто метод экспертной оценки используются в тандеме с тестированием удобства использования: экспертная оценка используется для формирования гипотез о проблемах, а тестирование удобства использования — для их проверки.

Процесс экспертной оценки пользовательских интерфейсов интеллектуальных систем обычно включает следующие этапы:

\begin{textitemize}
\item Выявление экспертов. Первый шаг — найти экспертов в этой области, имеющих опыт работы как с интеллектуальными системами, так и с моделью пользовательского интерфейса. В число этих экспертов могут входить специалисты по взаимодействию человека с компьютером (HCI), специалисты по машинному обучению или эксперты в области, для которой разрабатывается интерфейс.
\item Предоставление интерфейса. Экспертам предоставляется доступ к пользовательскому интерфейсу и предлагается взаимодействовать с ним различными способами. Им может быть предложено выполнить задачи или сценарии, которые отражают типичные варианты использования интерфейса.
\item Проведение оценки. Эксперты оценивают интерфейс на основе набора установленных принципов удобства использования, например, разработанных Nielsen или ISO. Их также могут попросить оценить адаптивность интерфейса и предоставить отзывы.
\item Анализ результатов. Результаты оценки анализируются для выявления проблем с удобством использования и областей для улучшения. Эксперты могут дать рекомендации по улучшению интерфейса на основе своей оценки.
\end{textitemize}

\section{Многократно используемые компоненты интерфейсов ostis-систем}
\label{sec_reusable_UI_components}

%\input{author/references}