\chapter{Методика и средства компонентного проектирования интерфейсов ostis-систем}
\chapauthortoc{Садовский М.~Е.\\Жмырко А.~В.}
\label{chapter_ui_design}

\vspace{-7\baselineskip}

\begin{SCn}
\begin{scnrelfromlist}{автор}
	\scnitem{Садовский М.~Е.}
	\scnitem{Жмырко А.~В.}
\end{scnrelfromlist}
\bigskip

\scntext{аннотация}{Проектирование \textit{интерфейса компьютерных систем} --- это один из наиболее важных этапов разработки любой системы.
	Пользователь при использовании \textit{интерфейса} должен представить себе, какая информация о выполняемой задаче у него существует, и в каком состоянии находятся средства, с помощью которых он будет решать данную задачу. Эффективность работы пользователя и его интерес обеспечивает правильно сформулированная \myuline{методика разработки и проектирования пользовательского интерфейса}. \newline
	В рамках главы рассмотрены этапы проектирования \textit{пользовательских интерфейсов} и этапы проектирования \textit{адаптивных интеллектуальных мультимодальных пользовательских интерфейсов}.}
\bigskip
	
\begin{scnrelfromlist}{подраздел}
	\scnitem{\ref{sec_analysis_UI_design_methodologies}~\nameref{sec_analysis_UI_design_methodologies}}
	\scnitem{\ref{sec_reusable_UI_components}~\nameref{sec_reusable_UI_components}}
\end{scnrelfromlist}	
\bigskip

\begin{scnrelfromlist}{ключевое понятие}
	\scnitem{библиотека многократно используемых компонентов пользовательских интерфейсов ostis-систем}
	\scnitem{метод оценки пользовательских интерфейсов}
	\scnitem{качественный метод оценки пользовательских интерфейсов}
	\scnitem{количественный метод оценки пользовательских интерфейсов}
	\scnitem{тестирование удобства использования пользовательских интерфейсов}
	\scnitem{отслеживание движения глаз}
	\scnitem{экспертная оценка пользовательских интерфейсов}
	\scnitem{A/Б тестирование пользовательских интерфейсов}
	\scnitem{древовидное тестирование пользовательских интерфейсов}
\end{scnrelfromlist}
\bigskip

\begin{scnrelfromlist}{библиография}
	\scnitem{\scncite{Ehlert2003}}
	\scnitem{\scncite{Kong2011}}
	\scnitem{\scncite{Koronchik2011}}
	\scnitem{\scncite{ISO9241-161}}
	\scnitem{\scncite{ISO16982}}
\end{scnrelfromlist}

\end{SCn}

\section{Анализ методик проектирования пользовательских интерфейсов}
\label{sec_analysis_UI_design_methodologies}

\begin{SCn}
	\begin{scnrelfromlist}{подраздел}
		\scnitem{\ref{sec_UI_analisys}~\nameref{sec_UI_analisys}}
	\end{scnrelfromlist}
\end{SCn}

В настоящее время организация взаимодействия пользователя с компьютерной системой основана на парадигме \myuline{грамотного пользователя}, который знает, как управлять системой и несёт полную ответственность за качество взаимодействия с ней.
Многообразие форм и видов \textit{интерфейсов} приводит к необходимости пользователя  адаптироваться к каждой конкретной системе, обучаться принципам взаимодействия с ней для решения необходимых ему задач.

Проектирование \textit{пользовательских интерфейсов} включает в себя ряд последовательных этапов.

Методика проектирования \textit{пользовательских интерфейсов} является важной частью Технологии OSTIS, так как она описывает этапы проектирования \textit{пользовательских интерфейсов ostis-систем}, что позволяет ускорить процесс разработки, обеспечивает создание удобных пользовательских интерфейсов, улучшает опыт использования интеллектуальной системы и повышает эффективность работы пользователей, учитывая их потребности и предпочтения.

Среди существующих методик проектирования \textit{адаптивных интеллектуальных мультимодальных пользовательских интерфейсов} можно выделить методики, предложенные в \scncite{Ehlert2003} и  \scncite{Kong2011}.

В рамках работы \scncite{Ehlert2003} выделяется 4 основных этапа проектирования:
\begin{textitemize}
    \item анализ;
    \item разработка \textit{интерфейса};
    \item оценка \textit{интерфейса};
    \item доработка и усовершенствование.
\end{textitemize}

Этап анализа является, вероятно, самой важной фазой в любом процессе проектирования, в том числе в проектировании \textit{интерфейсов ostis-систем}. В процессе проектирования традиционного \textit{интерфейса} необходимо проанализировать, кто является обычным пользователем, какие задачи \textit{интерфейс} должен поддерживать. 

В \textit{пользовательском интерфейсе} часто нет среднего пользователя.
В идеале, \textit{пользовательский интерфейс} должен быть способен адаптироваться к любому пользователю в любой среде. Поэтому используемая техника адаптации должна быть разработана таким образом, чтобы она могла поддерживать все типы пользователей.

Этап \textit{анализа} включает выполнение четырех взаимосвязанных видов анализа:
\begin{textitemize}
    \item \textit{функциональный анализ};
    \item \textit{анализ данных};
    \item \textit{анализ пользователей};
    \item \textit{анализ среды}.
\end{textitemize}

В рамках \textit{функционального анализа} необходимо дать ответ на вопрос: \scnqqi{каковы основные функции системы?}.
В рамках \textit{анализа данных} необходимо определить \uline{значение и структуру данных}, используемых в приложении.
В рамках \textit{анализа пользователей} необходимо выделить \uline{типы пользователей и их возможности} в интеллектуальном и когнитивном плане.
В рамках \textit{анализа среды} необходимо определить \uline{требования, предъявляемые к среде}, в которой будет работать система.

Результатом данного этапа является \uline{cпецификация целей и информационных потребностей пользователя}, а также \uline{спецификация функций и информации}, которые требуются системе.
\textit{Разработка интерфейса} включает следующие шаги:
\begin{textitemize}
	\item \textit{создание модели интерфейса} в соответствии с этапом анализа;
	\item реализация модели \textit{интерфейса}.
\end{textitemize}

Результатом данного этапа является \textit{пользовательский интерфейс}, который, по мнению разработчика, удовлетворяет требованиям пользователей и соответствует требованиям, сформулированным на этапе анализа.

\textit{Оценка интерфейса} предполагает, что:
\begin{textitemize}
	\item требования, которые были сформулированы на этапе \textit{анализа}, должны быть удовлетворены;
	\item эффективность модели \textit{интерфейса} должна быть исследована.
\end{textitemize}

На этапе \textit{оценки интерфейса} необходимо вернуться к требованиям \textit{этапа анализа}. Требования, которые были сформулированы на \textit{этапе анализа}, должны быть выполнены, а также должна быть исследована эффективность модели \textit{интерфейса}. Чтобы определить эту эффективность, необходимо определить критерии эффективности.

Очень важным, но субъективным критерием является удовлетворенность пользователя. Поскольку пользователь должен работать с \textit{интерфейсом}, он имеет право голоса в вопросе о том, удобно ли работать с \textit{интерфейсом}, насколько привлекательным является интерфейс.

Критериями эффективности могут выступать различные показатели, такие как:
\begin{textitemize}
	\item количество ошибок;
	\item время выполнения задачи;
	\item отношение пользователя к \textit{интерфейсу};
	\item и так далее.
\end{textitemize}

\textit{Доработка и усовершенствование} осуществляется на основе проблем, выявленных на этапе оценки. В рамках данного этапа вносится ряд улучшений в модель \textit{интерфейса}. Затем начинается новый цикл проектирования. Этот итеративный процесс будет продолжаться до тех пор, пока результат оценки не будет удовлетворять обозначенным требованиям. 

Методика, предложенная в \scncite{Kong2011} включает 6 этапов:
\begin{textitemize}
	\item моделирование \textit{пользовательского интерфейса} (описание абстрактного \textit{пользовательского интерфейса});
	\item проектирование \textit{пользовательского интерфейса} по умолчанию (стандартная версия, конкретный \textit{пользовательский интерфейс});
	\item разработка \textit{пользовательского интерфейса} (расширение или замена \textit{пользовательского интерфейса} по умолчанию) --- этот шаг опускается, когда система генерирует \textit{пользовательский интерфейс} по умолчанию автоматически;
	\item создание контекста использования (идентификация и создание контекста использования --- модели пользователя, модель устройства и модель среды/платформы);
	\item адаптация \textit{пользовательского интерфейса} --- автоматически (адаптация пользовательского интерфейса во время выполнения для соответствия конкретного контекста использования);
	\item кастомизация \textit{пользовательского интерфейса} --- настройка \textit{пользовательского интерфейса} самим пользователем (адаптируемость).
\end{textitemize}

На основе рассмотренных методик проектирования \textit{интерфейсов} можно выделить следующие общие этапы:
\begin{textitemize}
\item \textit{анализ контекста использования и задач пользователей};
\item \textit{проектирование и разработка \textit{интерфейса}};
\item \textit{оценка качества спроектированного \textit{интерфейса}}.
\end{textitemize}

Среди недостатков предложенных подходов можно выделить:
\begin{textitemize}
	\item знания по каждому этапу проектирования находятся у разных специалистов в неформализорованном неунифицированном виде;
	\item отсутствие \textit{этапа формализованного документирования} этапов проектирования приводит в дальнейшем к необходимости создания отдельных help-систем для пользователей, разработчиков и так далее.
\end{textitemize}

Таким образом, на основе проведенного анализа предлагаемая методика проектирования \textit{интерфейсов ostis-систем} (структура \textit{пользовательского интерфейса ostis-системы} была рассмотрена в Главе \ref{chapter_interfaces} \nameref{chapter_interfaces}) будет включать (см. \scncite{Sadouski2022a}):
\begin{textitemize}
\item анализ пользователя, его задач и целей, а также контекста использования;
\item анализ требований к \textit{пользовательскому интерфейсу} и спецификация проектируемого \textit{пользовательского интерфейса};
\item задачно-ориентированная декомпозиция \textit{пользовательского интерфейса};
\item проектирование \textit{пользовательского интерфейса} по умолчанию;
\item разработка \textit{пользовательского интерфейса};
\item анализ \textit{пользовательского интерфейса} и его адаптация.
\end{textitemize}

Поскольку знания о конкретном этапе обычно находятся у разных экспертов, особенностью предлагаемого подхода является обязательное формализованное документирование знаний в унифицированном виде и применение на каждом из этапов компонентного подхода.

Для применения компонентного подхода предлагается использовать \textit{библиотеку многократно используемых компонентов ostis-систем}.

Далее будет рассмотрен каждый этап по отдельности.

\textbf{Анализ пользователя, его задач и целей, а также контекста использования}

Результаты первого этапа, такие как: модель конкретного пользователя, его потребности и контекст использования системы (устройство, окружающая среда) должны быть формализованы в рамках соответствующих онтологий \textit{базы знаний} \textit{интерфейса}. 
При этом в процессе формализации по необходимости должны быть переиспользованы \textit{компоненты базы знаний} из \textit{библиотеку многократно используемых компонентов ostis-систем}, а новые компоненты могут пополнить эту же библиотеку.

\textbf{Анализ требований к пользовательскому интерфейсу и спецификация проектируемого пользовательского интерфейса}

Результатом второго этапа являются конечные требования к \textit{интерфейсу}, которые должны быть сформулированы относительно модели пользователя и его цели, а также относительно контекста использования.

Спецификация включает в себя список задач решаемых интерфейсом, описание \textit{внешних языков представления знаний}.

Результаты должны быть также формализованы, а в процессе выполнения могут быть использованы существующие компоненты из \textit{библиотеки многократно используемых компонентов интерфейсов ostis-систем}.

\textbf{Задачно-ориентированная декомпозиция пользовательского интерфейса}

На этапе задачно-ориентированной декомпозиции \textit{пользовательского интерфейса} специфицированный интерфейс разбивается на интерфейсные подсистемы, которые могут разрабатываться параллельно. Это позволяет сократить сроки проектирования \textit{пользовательского интерфейса}. Целесообразно проводить разбиение таким образом, чтобы максимальное количество подсистем уже имелось в \textit{библиотеке многократно используемых компонентов пользовательских интерфейсов ostis-систем}.

\textbf{Проектирование пользовательского интерфейса по умолчанию}

В соответствии с требованиями к \textit{пользовательскому интерфейсу}, строится модель \textit{адаптивного интеллектуального мультимодального пользовательского интерфейса}, которая является результатом третьего этапа.

Такая модель будет включать в себя формализованную модель \textit{базы знаний} и \textit{решателя задач}. При проектировании могут быть использованы компоненты \textit{интерфейса}, \textit{базы знаний} и \textit{решателя задач}. Компоненты будут записаны в унифицированном виде, что позволит обеспечить их автоматическую совместимость.

\textbf{Разработка пользовательского интерфейса}

Результатом четвертого этапа является реализация спроектированного \textit{пользовательского интерфейса}. 

После разработки \textit{пользовательского интерфейса} выделяются типовые фрагменты интерфейса. Специфицируя фрагменты интерфейса необходимым образом следует включать их в \textit{библиотеку многократно используемых компонентов пользовательских интерфейсов ostis-систем}.

При разработке пользовательского интерфейса также можно использовать готовые компоненты интерфейса из \textit{библиотеки многократно используемых компонентов пользовательских интерфейсов ostis-систем}.

\textbf{Анализ пользовательского интерфейса и его адаптация}

На данном этапе используются готовые компоненты \textit{решателя задач}: \textit{sc-агенты анализа пользовательского интерфейса} и \textit{sc-агенты изменения модели пользовательского интерфейса на основе логических правил адаптации}.

Таким образом будет сформирована \textit{база знаний} проектируемого \textit{интерфейса}, которая автоматически может быть использована в качестве help-системы для пользователей, разработчиков и так далее.

\subsection{Анализ методов оценки пользовательских интерфейсов}
\label{sec_UI_analisys}

 Оценка \textit{пользовательского интерфейса} необходима для улучшения коммуникации между \textit{интеллектуальными системами} и их пользователями. Существует множество методов для оценки \textit{пользовательских интерфейсов}, направленных, в основном, на выявление проблем с использованием системы и минимизацию риска ошибок. Однако до сих пор отсутствует комплексный подход к оценке \textit{пользовательских интерфейсов интеллектуальных систем}.

При оценке \textit{пользовательских интерфейсов} большую роль играет человеческий фактор, а основными участниками являются пользователи и эксперты. Очень сложно оценить правильность решения по адаптации \textit{пользовательского интерфейса интеллектуальной системы} и оценить его без участия человека. Конечно, есть небольшие обобщенные правила построения \textit{пользовательских интерфейсов}, такие как:

\begin{textitemize}
	\item \textit{интерфейс} должен быть интуитивно понятен для конечного пользователя;
	\item \textit{интерфейс} должен быть доступным для пользователей с ограниченными возможностями и пользователей, впервые сталкивающимися с информационными технологиями;
	\item достижение цели пользователем должно осуществляться наименее возможным количеством шагов(см.\scncite{ISO9241-161}).
\end{textitemize}

Оценка \textit{пользовательских интерфейсов интеллектуальных систем} имеет свои особенности и требует специальных методов и инструментов.

Некоторые \textit{методы оценки пользовательских интерфейсов} интеллектуальных систем включают:
\begin{textitemize}
	\item Оценку точности и полноты. 
	
	Это метод, при котором оценивается точность и полнота ответов системы на запросы пользователей.
	
	Точность означает, насколько хорошо \textit{пользовательский интерфейс} отражает действительный опыт пользователя и предлагает ему наиболее подходящий контент в соответствии с его потребностями и предпочтениями. Оценить точность возможно путем анализа, насколько легко использовать интерфейс, насколько он отражает действительные возможности интеллектуальной системы и насколько он помогает пользователям достигать своих целей более эффективно.
	
	Полнота, с другой стороны, означает, насколько хорошо \textit{пользовательский интерфейс} предоставляет всю необходимую информацию и функциональность, которые пользователь может потребовать в процессе взаимодействия с системой. Оценить полноту возможно путем анализа того, насколько полное и четкое описание функций, возможностей и ограничений системы предоставляется через \textit{пользовательский интерфейс}, в каком объеме пользователю доступна необходимая информация для принятия решений и насколько система может эффективно реагировать на запросы и потребности пользователя.
	
	\item Оценку персонализации. 
	
	Это метод, при котором оценивается способность системы адаптироваться к потребностям и предпочтениям каждого пользователя. Оценка персонализации выполняется с помощью анализа пользовательских данных, а также проведения опросов среди пользователей, чтобы определить, насколько хорошо система учитывает их потребности.
	
	Оценка качества персонализации может включать в себя измерение следующих параметров:
	\begin{textitemize}
	\item Работоспособность --- действительно ли \textit{пользовательский интерфейс} адаптивный и соответствует ли он потребностям пользователя?
	\item Уникальность --- насколько уникальным и индивидуальным является персонализированный \textit{пользовательский интерфейс}?
	\item Понятность --- насколько просто и легко пользователь может настроить и использовать персонализацию?
	\item Эффективность --- насколько хорошо персонализированный \textit{пользовательский интерфейс} помогает пользователю в выполнении задач?
	\item Удовлетворенность --- насколько положительным и удовлетворительным является \textit{пользовательский интерфейс}?
	\end{textitemize}

	Оценка персонализации \textit{адаптивных пользовательских интерфейсов} может быть выполнена различными методами, включая тестирование с использованием фокус-групп, опросы пользователей, анализ данных и другие. Важно отметить, что оценка персонализации должна проходить на всех этапах разработки для повышения пользовательского опыта и улучшения качества \textit{пользовательских интерфейсов интеллектуальных систем}.
\end{textitemize}


\textbf{\textit{методы оценки пользовательских интерфейсов}} можно разделить на две группы: качественные и количественные.
	
Задача \textit{качественных методов оценки пользовательских интерфейсов} --- помочь понять мотивы поведения, потребности и логику пользователей.
	
\textit{качественные методы} нацелены на сбор данных, описывающих предмет изучения. Они позволяют углубиться в предметную область для получения представления о мотивации, мышлении и взглядах пользователей. 
	
К \textit{качественным методам оценки пользовательских интерфейсов} относятся:
\begin{textitemize}
	\item \textit{тестирование удобства использования};
	\item \textit{отслеживание движения глаз};
	\item \textit{экспертная оценка}.
\end{textitemize}	

\textit{количественные методы оценки пользовательских интерфейсов} позволяют выявить сложности или возможности, с которыми сталкиваются пользователи, и отделить реальные проблемы от предполагаемых. Такие методы измеряют числовые показатели. 

\textit{количественными методы} формируют представление о том, чем занимаются пользователи, и включают:
\begin{textitemize}
	\item \textit{A/Б тестирование};
	\item \textit{древовидное тестирование}.
\end{textitemize}	


\textbf{\textit{тестирование удобства использования пользовательских интерфейсов}} является важной частью процесса проектирования и разработки. \textit{тестирование удобства использования} гарантирует, что эти интерфейсы удобны и полезны для потенциальных пользователей.

\textit{тестирование удобства использования}, как правило, включает несколько этапов:

\begin{textitemize}
	\item Определение целевых пользователей. Это может потребовать проведения исследований пользователей для понимания потребностей и предпочтений потенциальных пользователей.
	\item Разработка тестовых сценариев, которые отражают типичные случаи использования \textit{интерфейса}. Эти сценарии должны быть разработаны для проверки адаптивных возможностей \textit{интерфейса}, а также его удобства использования.
	\item Проведение тестирования пользователей. Тестирование пользователей включает в себя наблюдение за пользователями во время взаимодействия с \textit{интерфейсом интеллектуальной системы}. Пользователи выполняют задачи, связанные с тестовыми сценариями. Тестирование пользователей может проводиться в контролируемой лабораторной среде или удаленно с использованием технологии совместного просмотра экрана.
	\item Анализ результатов тестирования. Результаты тестирования пользователей анализируются для выявления проблем с удобством использования и областей для улучшения.
	\item Итерация дизайна. На основе результатов тестирования пользователей и анализа результатов тестирования, \textit{интерфейс} может быть изменен для устранения проблем с удобством использования и улучшения общего пользовательского опыта.
\end{textitemize}


\textbf{\textit{отслеживание движения глаз}} --- это метод, который используется для анализа того, как пользователи взаимодействуют с \textit{пользовательским интерфейсом}. Отслеживание движения глаз определяет точки фиксации взгляда пользователя при взаимодействии с системой, а также переходы между ними. 

При использовании метода определяется \textit{компонент пользовательского интерфейса}, на который пользователь смотрел дольше всего, сколько времени он уделяет каждому компоненту и как легко ему удается найти нужную информацию.

Метод выявляет \textit{компоненты интерфейса}, которым уделяется больше внимания, позволяет обнаружить области, вызывающие у пользователей затруднения.

Метод позволяет получить реалистичный образ отношения пользователя и \textit{интерфейса}, поскольку он фиксирует естественное движение глаз человека. Кроме того, \textit{отслеживание движения глаз} позволяет быстро найти проблемные места в \textit{пользовательском интерфейсе} и предложить улучшения, которые могут увеличить удобство использования \textit{интерфейса}.


Метод {\textit{экспертной оценки пользовательских интерфейсов} заключается в исследовании \textit{пользовательского интерфейса} на соответствие заранее определенным правилам(см.\scncite{ISO16982}).

Достаточно часто метод \textit{экспертной оценки пользовательских интерфейсов} используется в тандеме с \textit{тестированием удобства использования}: \textit{экспертная оценка} используется для формирования гипотез о проблемах, а \textit{тестирование удобства использования} --- для их проверки.

Процесс \textit{экспертной оценки пользовательских интерфейсов} \textit{интеллектуальных систем} обычно включает следующие этапы:

\begin{textitemize}
\item Выявление экспертов. 

Первый этап --- поиск экспертов в предметной области, имеющих опыт работы как с \textit{интеллектуальными системами}, так и с моделью \textit{пользовательского интерфейса}. В число этих экспертов могут входить специалисты по взаимодействию человека с компьютером, специалисты по машинному обучению или эксперты в области, для которой разрабатывается интерфейс.

Экспертам предоставляется доступ к \textit{пользовательскому интерфейсу} и предлагается взаимодействовать с ним различными способами. Им может быть предложено выполнить задачи или сценарии, которые отражают типичные варианты использования интерфейса.

\item Проведение оценки. 

Эксперты оценивают \textit{пользовательский интерфейс} на основе набора установленных принципов удобства использования. Также может быть произведена оценка адаптивности \textit{интерфейса}.

\item Анализ результатов. 

Результаты оценки анализируются для выявления проблем с удобством использования и областей для улучшения. Эксперты могут давать рекомендации по улучшению интерфейса на основе своей оценки.
\end{textitemize}

\textit{A/Б тестирование пользовательских интерфейсов} --- метод сравнения двух версий \textit{пользовательского интерфейса}. Результатом проведения метода является выявление версии \textit{интерфейса} наиболее подходящей для выполнения конкретной задачи.

Пользователи случайным образом разбиваются на два сегмента, каждый из которых видит только одну версию интерфейса.

Процесс \textit{A/Б тестирования пользовательских интерфейсов} включает следующие этапы:

\begin{textitemize}
	\item Определение гипотезы. 
	
	В первую очередь необходимо определить цель для проверки. В данном случае это могут быть как отдельные \textit{компоненты пользовательского интерфейса}, так и \textit{пользовательский интерфейс} в целом.
	
	На основе анализа данных или личных предположений формулируется гипотеза, например, \scnqqi{если мы изменяем цвет и размер кнопки, пользователи будут чаще нажимать на нее} или \scnqqi{если мы добавим кнопку, которая будет видна на каждой странице, пользователи лучше будут понимать, как оставить отзыв}.
	
	\item Определение размеров выборок. 
	
	Чтобы получить репрезентативные результаты, нужно определить размер контрольной и экспериментальной групп. Например, 50\% пользователей попадут в контрольную группу, которая будет видеть старый интерфейс, а другие 50\% в экспериментальную, которая будет видеть измененный интерфейс.
	
	\item Итерация интерфейса. 
	
	На этом этапе вносятся изменения в интерфейс, которые соответствуют определенной гипотезе. 
	
	\item Наблюдение и сбор данных. 
	
	Во время тестирования фиксируются действия пользователей, например, клики и время, проведенное на странице. Эти данные необходимы для определения изменений интерфейса, наиболее повлиявших на поведение пользователей.
	
	\item Анализ результатов. 
	
	После того, как тестирование завершено, проводится анализ данных для понимания, насколько значимы различия между контрольной и экспериментальной группами. Например, если пользователи, которые видели измененный интерфейс, кликали на кнопку в 2 раза чаще, чем пользователи, которые видели старый интерфейс, это говорит о том, что изменения были успешными и их необходимо внедрять в конечный \textit{пользовательский интерфейс}.
\end{textitemize}

\textit{древовидное тестирование пользовательских интерфейсов} --- это метод оценки качества \textit{пользовательских интерфейсов}, который заключается в тестировании древовидной структуры навигации по системе.

Метод помогает определить, насколько эффективно пользователи могут находить нужную информацию и выполнять различные задачи.

В \textit{древовидном тестировании} объектом оценки является древовидная структура навигации по системе. Эта структура представляет собой дерево, в котором корневой элемент является главной страницей, а дочерние элементы --- подстраницы или разделы. 

Цель метода --- проверить, насколько легко пользователи могут навигировать по этой структуре и находить нужную информацию.

Процесс \textit{древовидного тестирования пользовательских интерфейсов} включает следующие этапы:
\begin{textitemize}
	\item На основе структуры навигации создается тестовая среда, которая представляет собой виртуальный \textit{интерфейс}.
	
	\item Респондентам предлагается выполнить задание, связанное с поиском нужной информации. Это может быть, например, поиск конкретной страницы или поиск информации по определенному тематическому разделу.
	
	\item Респондентам дается доступ к тестовой среде, и они проходят по древовидной структуре навигации, пытаясь найти нужную им информацию.
	
	\item В процессе прохождения теста респонденты записывают свои действия и комментарии о том, как они ориентируются в системе, какой путь выбирают для поиска нужной информации, какие ошибки и препятствия им приходится преодолевать и так далее.
	
	\item По результатам тестирования делается анализ путей, выбранных респондентами, выявляются наиболее эффективные и неэффективные способы навигации, а также выделяются проблемные зоны \textit{интерфейса} системы.
	
	\item На основе анализа результатов тестирования и выявленных проблем делается рекомендация по оптимизации древовидной структуры навигации или изменению \textit{пользовательского интерфейса} для улучшения пользовательского опыта.
\end{textitemize}

Представленные \textit{методы оценки пользовательских интерфейсов} предлагается применять для оценки \textit{пользовательских интерфейсов ostis-систем} в рамках рассмотренного ранее этапа \scnqqi{Анализ пользовательского интерфейса и его адаптация}. При этом особенностью проектирования \textit{пользовательских интерфейсов ostis-систем} является постоянная информационная поддержка пользователя на всех этапах проектирования \textit{интерфейса} за счет наличия в \textit{базе знаний} каждой \textit{ostis-системы} \textit{Предметной области методик проектирования пользовательских интерфейсов ostis-систем}, содержащей методики, рассмотренные в рамках данного параграфа.


\section{Многократно используемые компоненты интерфейсов ostis-систем}
\label{sec_reusable_UI_components}

Большое разнообразие \textit{интерфейсов} влечет за собой разработку большого числа компонентов. В качестве \textit{многократно используемых компонентов интерфейсов ostis-систем} могут выступать как уже спроектированные
\textit{интерфейсы}, так и специфицированные \textit{компоненты интерфейсов}. Большое число \textit{многократно используемых компонентов интерфейсов ostis-систем} создает проблему их хранения и поиска. Чтобы решить эту проблему, в технологию включена \textit{библиотека многократно используемых компонентов пользовательских интерфейсов ostis-систем} и \textit{менеджер многократно используемых компонентов ostis-систем} (см. Главу \ref{chapter_library} \nameref{chapter_library}).

В рамках библиотеки, все компоненты специфицируются и классифицируются. Их классификация производится по различным признакам. К примеру, в зависимости от решаемых задач компоненты делятся на следующие классы:

\begin{textitemize}
	\item просмотрщик содержимого \textit{файлов ostis-системы};
	\item редактор содержимого \textit{файлов ostis-системы};
	\item транслятор содержимого \textit{файла ostis-системы} в \textit{SC-код};
	\item транслятор из \textit{SC-кода} в содержимое \textit{файла ostis-системы};
	\item транслятор \textit{базы знаний} во внешнее представление (см.\scncite{Koronchik2011}).
\end{textitemize}

\textit{Технология OSTIS} позволяет интегрировать в качестве компонентов редакторы и просмотрщики, разработанные с использованием других технологий (далее их будем называть \textit{платформенно-зависимыми многократно используемыми компонентами интерфейса ostis-систем}). В основном они используются для просмотра и редактирования содержимого \textit{файлов ostis-системы}. Это значительно позволяет сэкономить время при их разработке.

\textit{пользовательский интерфейс} \textit{библиотеки многократно используемых компонентов интерфейсов ostis-систем} строится на основе SCg-интерфейса (комплекс информационно-программных средств обеспечивающих общение интеллектуальных систем с пользователями на основе \textit{SCg-кода}, как способа внешнего представления информации). Однако, это не исключает возможность использования других способов диалога пользователя с \textit{библиотекой многократно используемых компонентов интерфейсов ostis-систем}.

В рамках \textit{библиотеки многократно используемых компонентов интерфейсов ostis-систем} могут содержаться различные версии и модификации какого-либо компонента. К примеру, компонент просмотра \textit{sc.g-конструкций} может иметь модификации, для отображения в которых может использоваться двумерная, трехмерная или же многослойная визуализация, при этом каждая из модификаций компонента может иметь различные версии.

Использование \textit{библиотеки многократно используемых компонентов интерфейсов ostis-систем} при проектировании \textit{интерфейса} прикладной системы позволяет значительно сократить сроки проектирования, а также снизить требования, предъявляемые к начальной квалификации разработчика. Это достигается за счет проектирования \textit{интерфейса} из уже заранее подготовленных моделей интерфейса, что также позволяет повысить качество проектируемого \textit{интерфейса}.

%\input{author/references}