\chapauthor{Бутрин С.В.\\Шункевич Д.В.\\Банцевич К.А.}
\chapter{Методика и средства проектирования и анализа качества баз знаний интеллектуальных компьютерных систем нового поколения}
\chapauthortoc{Бутрин С.В.\\Шункевич Д.В.\\Банцевич К.А.}
\label{chapter_kb_design}

\abstract{Аннотация к главе.}

\section{Действия и методики проектирования баз знаний ostis-систем}
\section{Логико-семантическая модель комплекса встраиваемых ostis-систем автоматизации проектирования баз знаний ostis-систем}
\section{Логико-семантическая модель ostis-системы редактирования, сборки и ввода исходных текстов различных компонентов проектируемой базы знаний в память ostis-системы}
\section{Логико-семантическая модель ostis-системы редактирования проектируемой базы знаний ostis-системы на уровне её внутреннего представления}
\section{Логико-семантическая модель ostis-системы обнаружения и анализа ошибок и противоречий в базе знаний ostis-системы}

Задача верификации важная часть процесса разработки любой системы и интеллектуальные системы не стали исключением. Более того эффективная верификация знаний имеет решающее значение, так как определяет качество базы знаний, что определяет качество всей системы. Однако интеллектуальные системы требует новых, специальных методов для эффективной верификации в виду своих особенностей.

Большинство существующих средств верификации представляют собой частные решения, не способные быть переиспользованными в других системах, поэтому возникает необходимость в разработке самого подхода к созданию средств верификации. В свою очередь такой подход должен удовлетворять ряду требований:
\begin{itemize}
\item универсальность; должна быть возможность спроектировать и реали-
зовать средство для устранения любого вида противоречия;
\item гибкость; возможность изменить средства верификации, например, в
случае появления алгоритма для устранения определенного вида противоре-
чий;
\item специфицируемость; наличие спецификации, обеспечивающей взаи-
модействие средств друг с другом.
\end{itemize}

%% Описать что это такое и зачем выделяется

\scnheader{проблемная структура}
\scnidtf{структура, описывающая проблемный фрагмент базы знаний}
\scnidtf{структура, описывающая некачественный фрагмент базы знаний}
\begin{scnreltoset}{объединение}
\scnitem{некорректная структура}
\begin{scnindent}
	\scnidtf{структура, содержащая фрагменты, противоречащие каким либо правилам или закономерностям описанным в базе знаний}
\end{scnindent}
\scnitem{структура, описывающая неполноту в базе знаний}
\begin{scnindent}
	\scnidtf{структура, в которой имеется неполнота (то есть имеется некоторое количество информационных дыр)}
	\scntext{примечание}{Под структурой, описывающей неполноту в базе знаний, понимается структура, содержащая фрагмент базы знаний, в котором отсутствует какая-либо информация, которая необходима (или, по крайней мере, желательна) для однозначного и полного понимания смысла данного фрагмента.}
	%% Переформулировать
\end{scnindent}
\scnitem{информационный мусор}
\begin{scnindent}
	\scnidtf{структура, удаление которой существенно не усложнит деятельность системы}
	\scnidtf{структура, содержащая фрагмент базы знаний, который по каким-либо причинам стал ненужным и требует удаления}
\end{scnindent}
\end{scnreltoset}
	
% Больше про определение противоречия, уточнить что чему противоречит 
	
\scnheader{противоречие*}
\scnidtf{пара противоречащих друг другу фрагментов информации, хранимой в памяти кибернетической системы*}
\scntext{примечание}{Чаще всего противоречащими друг другу информационными фрагментами являются:
	\begin{scnitemize}
		\item явно представленная в памяти некоторая закономерность (некоторое правило)
		\item информационный фрагмент, не соответствующий (противоречащий) указанной закономерности
	\end{scnitemize}
	В этом случае некорректность может присутствовать:
	\begin{scnitemize}
		\item либо в информационном фрагменте, который противоречит указанной закономерности;
		\item либо в самой этой закономерности;
		\item либо и там и там.
	\end{scnitemize}
		%% Переформулировать
}

	
\scnheader{некорректная структура}
\begin{scnreltoset}{включение}
	\scnitem{дублирование системных идентификаторов}
	\scnitem{несоответствие элементов связки доменам отношения}
	\scnitem{цикл по отношению порядка}
	\scnitem{структура, противоречащая свойству единственности}
\end{scnreltoset}
	
\scnheader{структура, описывающая неполноту в базе знаний}
\begin{scnreltoset}{включение}
	\scnitem{не указан максимальный класс объектов исследования предметной области}
	\scnitem{для сущности указан системный, но не указаны основные идентификаторы для всех внешних языков}
	\scnitem{не указаны домены отношения}
	\scnitem{понятие не соотнесено ни с одной предметной областью}
\end{scnreltoset}


%% Больше описать, привести примеры, вообщем пересмотреть

\scnheader{требующее внимание разработчика}
\scnidtf{проблемная структура, для исправления которой требуется участие разработчика}
\scnheader{множество элементов, которые должны быть удалены для исправления структуры*}
\scnidtf{множество элементов, удаление которых из структуры позволяет устранить в ней противоречие}
\scnheader{множество элементов, которые должны быть добавлены для исправления структуры*}
\scnidtf{множество элементов, добавление которых в структуру позволяет устранить в ней противоречия}
\scnheader{структура, которую система не способна исправить сама}
\scnidtf{структура, в которой система не способна автоматически устранить противоречия}
%% Возможно переформулировать на что-нибудь покороче


\scnheader{следует отличать*}
\begin{scnhaselementset}
\scnitem{структура, которая система не может решить сама}
\begin{scnindent}
	\scntext{примечание}{Здесь структура, которую система не может решить сама, не может быть исправлена при взаимодействии с разработчиком и требует полного исправления от самого разработчика}
\end{scnindent}
\scnitem{требующее внимание разработчика}
\begin{scnindent}
	\scntext{примечание}{Здесь структура, требующая внимания разработчика, может быть решена в процессе верификации, но потребуется участие разработчика}
\end{scnindent}
\end{scnhaselementset}

%% Описать идею мезанизма согласования средств верификации

\scnheader{Решатель задач средств выявления и устранения противоречий}
\begin{scnreltoset}{декомпозиция абстрактного sc-агента}
\scnitem{Неатомарный агент выявления противоречий}
	\begin{scnindent}
		\scnidtf{Множество агентов, обеспечивающих поиск и фиксирование противоречий в структуре}
	\end{scnindent}
\scnitem{Неатомарный агент устранения противоречий}
	\begin{scnindent}
	\scnidtf{Множество агентов, создающих предложения по исправлению противоречий}
	\scntext{примечание}{Результатом работы таких агентов будут множества предлагаемых к удалению из структуры или добавлению в структуру элементов}
	\end{scnindent}
\scnitem{Агент слияния структур}
	\begin{scnindent}
	\scnidtf{Агент, создающий структуру содержащую все элементы сливаемых структур}
	\end{scnindent}
\scnitem{Агент применения предложений по устранению противоречий}
\scnitem{Агент внесения исправлений в базу знаний}
	\begin{scnindent}
	\scntext{примечание}{Внесение изменений подразумевает не только исправление в базе знаний изначальной проблемной структуры, но и фиксацию самого факта изменения состояния базы знаний}.
	\end{scnindent}
\scnitem{Неатомарный агент верификации структуры}
	\begin{scnindent}
	\scntext{примечание}{Агент обеспечивающий полный цикл верификации структуры координирую другие агенты}
	\end{scnindent}
\end{scnreltoset}

%% Добавить про алгоритм

%% Не ограничиваться средствами верификации, затронуть механизмы предотвращения проблемных ситуаций 

%% Рассмотреть верификацию в рамках взаимодействия систем


\section{Логико-семантическая модель ostis-системы обнаружения и анализа информационных дыр в базе знаний ostis-системы} 
\section{Логико-семантическая модель ostis-системы автоматизации управления взаимодействием разработчиков различных категорий в процессе проектирования базы знаний ostis-системы}

%% Описать механизм редактирования, сравнение с аналогами, акцент на описание стадий с учетом каждой роли 

\scnheader{пользователь, обладающий правом просмотра sc-структуры*}
\scnidtf{бинарное отношение, связывающее sc-элемент, обозначающий sc-структуру (например, фрагмент sc-модели базы знаний), и sc-элемент, обозначающий пользователя этой ostis-системы, который обладает правом просмотра этой sc-структуры.}
\scniselement{бинарное отношение}
\scniselement{ориентированное отношение}
\scntext{примечание*}{Пользователь, обладающий правом просмотра sc-структуры базы знаний ostis-системы может быть зарегистрирован или не зарегистрирован в sc-модели базы знаний.}

\scnheader{пользователь, обладающий правом редактирования sc-структуры*}
\scnidtf{бинарное отношение, связывающее sc-элемент, обозначающий sc-структуру (например, фрагмент sc-модели базы знаний), и sc-элемент, обозначающий зарегистрированного пользователя ostis-системы, который обладает правом редактирования этой sc-структуры.}
\scniselement{бинарное отношение}
\scniselement{ориентированное отношение}
\scnsuperset{пользователь, обладающий правом просмотра sc-структуры*}
\scntext{пояснение}{Связки отношения пользователя, обладающий правом редактирования sc-структуры ostis-системы* связывают sc-структуру (не обязательно всю sc-модель базы знаний) и пользователя, зарегистрированного в этой sc-модели базы знаний.}
\begin{scnrelfromset}{покрытие}
	\scnitem{пользователь, обладающий правом редактирования sc-структуры посредством формирования предложений по внесению изменений в согласованную часть базы знаний этой ostis-системы*}
	\scnitem{пользователь, обладающий правом редактирования sc-структуры с автоматическим формированием и принятием предложений по внесению изменений в согласованную часть базы знаний этой ostis-системы*}
\end{scnrelfromset}

\scnheader{разработчик*}
\scnsubset{пользователь, обладающий правом редактирования sc-структуры*}
\scnidtf{бинарное отношение, связывающее sc-элемент, обозначающий некоторый раздел базы знаний (в пределе – всю базы знаний), и sc-элемент, обозначающий пользователя ostis-системы, который может быть разработчиком данного раздела базы знаний, т. е. выполнять проектные задачи в рамках данного раздела}

%% Добавить больше про классификацию пользователей


%\input{author/references}