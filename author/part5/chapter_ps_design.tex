\chapauthor{Шункевич Д.В.\\Марковец В.С.}
\chapter{Методика и средства компонентного проектирования решателей задач ostis-систем}
\chapauthortoc{Шункевич Д.В.\\Марковец В.С.}
\label{chapter_ps_design}

\abstract{Аннотация к главе.}

В области разработки \textit{решателей задач} существует большое количество конкретных реализаций, однако вопросы совместимости различных решателей при решении одной задачи практически не рассматриваются. Гипотетически возможно существование универсального решателя задач, объединяющего в себе все известные способы и методы решения задач. Однако использование такого решателя в прикладных целях не является целесообразным. Таким образом, наиболее приемлемым вариантом становится создание библиотеки совместимых между собой компонентов, из которых впоследствии может быть скомпилирован решатель, удовлетворяющий необходимым требованиям.

\section{Действия и методики проектирования решателей задач ostis-систем}
\section{Логико-семантическая модель комплекса ostis-систем автоматизации проектирования решателей задач ostis-систем}
\section{Логико-семантическая модель ostis-системы автоматизации проектирования программ Базового языка программирования ostis-систем}
\section{Логико-семантическая модель ostis-системы автоматизации проектирования внутренних агентов ostis-систем, а также коллективов таких агентов}
\section{Многократно используемые компоненты решателей задач ostis-систем}
\section{Многократно используемые методы, хранимые в памяти ostis-систем и интерпретируемые их внутренними агентами}
\section{Многократно используемые внутренние агенты ostis-систем}

%\input{author/references}