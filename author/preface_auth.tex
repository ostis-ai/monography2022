\begin{partbacktext}
\part*{Авторское предисловие}
\markboth{АВТОРСКОЕ ПРЕДИСЛОВИЕ}{АВТОРСКОЕ ПРЕДИСЛОВИЕ}
\label{chap_preface_auth}
\addcontentsline{toc}{part}{Авторское предисловие}

Ключевой особенностью \textbf{\textit{компьютерных систем нового поколения}} является их \textbf{\textit{интероперабельность}}, т.е. способность к эффективному осознанному взаимодействию, необходимым условием которой является способность к взаимопониманию, т.е. \textbf{\textit{семантическая совместимость}}. Таким образом, каждая \textit{компьютерная система нового поколения} должна:

\begin{textitemize}
	\item знать свои обязанности и возможности;
	\item уметь координировать свои действия с другими компьютерными системами нового поколения в ходе коллективного решения сложных задач в непредусмотренных (нештатных) обстоятельствах.
\end{textitemize}

Переход от современных компьютерных систем к компьютерным системам нового поколения, которые, очевидно, должны обладать достаточно высоким уровнем \textbf{\textit{интеллекта}}, означает переход к принципиально новому технологическому укладу в области автоматизации различных видов человеческой деятельности и предполагает переосмысление и использование всего опыта, накопленного при разработке и эксплуатации различных \textit{компьютерных систем}. Участие в проекте создания комплексной \textbf{\textit{Технологии поддержки жизненного цикла интеллектуальных компьютерных систем нового поколения}} не требует от специалистов радикального изменения области их научных интересов. Требуется просто учет дополнительных требований к формализации своих результатов. Основным из этих требований является \textbf{\textit{семантическая совместимость}} с другими (смежными) результатами.

К настоящему моменту 
\begin{textitemize}
	\item Завершен первый этап разработки \textit{Технологии поддержки жизненного цикла интеллектуальных компьютерных систем нового поколения}, которая названа нами \textbf{\textit{Технологией OSTIS}} (Open Semantic Technology for Intelligent Systems). Началом указанного первого этапа является Первая конференция OSTIS (Минск, 10 -- 12 февраля 2011 года).
	\item Сформировался работоспособный стартовый авторский коллектив в рамках созданного Учебно-научного объединения по Искусственному интеллекту, в состав которого входят ведущие университеты Республики Беларусь (в четырех из которых открыта специальность ``Искусственный интеллект''), и ряд других организаций.
\end{textitemize}


Результаты первого этапа разработки \textit{Технологии OSTIS} отражены в материалах проведенных \textbf{\textit{конференций OSTIS}} и в первой версии \textbf{\textit{Метасистемы OSTIS}}, которая непосредственно и осуществляет автоматизацию проектирования, реализации, реинжиниринга и эксплуатации \textbf{\textit{интеллектуальных компьютерных систем нового поколения}} и гарантирует поддержку их \textbf{\textit{семантической совместимости}} и интероперабельности не только на этапе проектирования, но и в ходе их эксплуатации.

На следующем этапе разработки \textit{Технологии поддержки жизненного цикла интеллектуальных компьютерных систем нового поколения} требуется существенное расширение фронта работ и соответствующего авторского коллектива. Этому, в частности, была посвящена \textit{Конференция OSTIS-2022} (24-26 ноября 2022 года), которая внесла важный вклад в развитие открытого проекта развития технологии комплексной поддержки жизненного цикла интеллектуальных компьютерных систем нового поколения (\textit{Проекта OSTIS}). Предлагаемая вашему вниманию монография является результатом работы указанного авторского коллектива.

Главным адресатом монографии является широкий круг лиц (включая \myuline{студентов}), которые хотят \myuline{за короткое время} приобрести знания и навыки, для разработки \textit{прикладных интеллектуальных компьютерных систем нового поколения}. Для такого контингента читателей необходима четко структурированная документация, отражающая \myuline{текущее} \myuline{состояние} соответствующей технологии.

Вторым контингентом читателей являются специалисты в области современных информационных технологий (и, в том числе, технологий Искусственного интеллекта), желающие стать участниками \myuline{дальнейшей} \myuline{эволюции} комплексной технологии создания и сопровождения интеллектуальных компьютерных систем нового поколения --- соавторами последующих версий указанной технологии, которая должна носить \myuline{открытый} характер. Для такого контингента читателей требуется:
\begin{textitemize}
	\item достаточно подробное обоснование принципов, лежащих в основе текущей версии указанной технологии; 
	\item сравнительный анализ указанных принципов с известными альтернативными подходами; 
	\item четкая формулировка проблем текущего состояния рассматриваемой технологии и направлений ее дальнейшего развития.
\end{textitemize}

Написание данной монографии с ориентацией сразу на две указанные выше категории читателей создает благоприятные условия для последующего перехода инженеров прикладных интеллектуальных компьютерных систем нового поколения в категорию соавторов последующих версий соответствующей технологии, что является очень важным фактором ускорения темпов эволюции этой технологии благодаря глубокой \textbf{\textit{конвергенции}} инженерной деятельности, научно-исследовательской деятельности и деятельности по развитию технологии.

\end{partbacktext}