\chapter*{\LARGE Заключение. Основные направления, проблемы и перспективы развития интеллектуальных компьютерных систем нового поколения и соответствующей им технологии}
\label{chap_conclusion}
\addcontentsline{toc}{part}{Заключение. Основные направления, проблемы и перспективы развития интеллектуальных компьютерных систем нового поколения и соответствующей им технологии}
\chapauthortoc{Голенков В.В.\\Гулякина Н.А.}

\begin{SCn}
	\begin{scnrelfromlist}{автор}
		\scnitem{Голенков В.~В.}
		\scnitem{Гулякина Н.~А.}
	\end{scnrelfromlist}
\end{SCn}

\section*{Особенность текущего состояния работ в области Искусственного интеллекта}

Необходимость перехода от современных \textit{компьютерных систем} (в том числе, и от современных \textit{интеллектуальных компьютерных систем}) к \textit{интеллектуальным компьютерных системам нового поколения} обусловлена необходимостью перехода к автоматизации более сложных (комплексных) видов и областей деятельности требующих создания целых комплексов компьютерных систем, способных эффективно взаимодействовать между собой в процессе \uline{коллективного} решения сложных задач.

Компьютерные системы, обладающие указанными способностями называют \textit{интероперабельными компьютерными системами}. Поскольку указанные компьютерные системы не могут не иметь высокого \textit{уровня интеллекта}, мы также называем их \textit{интеллектуальными компьютерными системами нового поколения}. Высокий \textit{уровень интеллекта} интероперабельным компьютерным системам необходим:

\begin{textitemize}
	\item для адекватной оценки собственной компетенции и компетенции своих партнеров;
	\item для обеспечения взаимопонимания, договороспособности и координации (согласованности) своих действий с действиями партнеров в ходе \textit{коллективного решения сложных задач} в условиях возможного возникновения непредсказуемых (нештатных) обстоятельств.
\end{textitemize}

Очевидно, что для создания и эксплуатации таких \textit{интероперабельных интеллектуальных компьютерных систем} необходимо:

\begin{textitemize}
	\item разработать общую формальную теорию таких систем;
	\item разработать комплексную технологию проектирования и поддержки жизненного цикла этих систем;
	\item разработать общую формальную теорию всего многообразия видом и областей человеческой деятельности, которые целесообразно автоматизировать.
\end{textitemize}

В основе предлагаемого нами подхода к построению \textit{\textit{интероперабельных интеллектуальных компьютерных систем}} лежат следующие принципы:

\begin{textitemize}
	\item смысловое представление знаний, хранимых в памяти интерооперабельных интеллектуальных компьютерных систем;
	\item онтологическая структуризация и систематизация хранимых в памяти знаний;
	\item децентрализованная ситуационная агенто-ориентированная организация процесса решения задач;
	\item конвергенция и глубокая (диффузная) интеграция различных моделей решения задач и, как следствие гибридный характер решателей задач;
	\item смысловая интеграция информации поступающей из вне по разным сенсорным каналам и на разных языках в результате трансляции входной информации на общий язык внутреннего смыслового представления знаний.
\end{textitemize}

\begin{SCn}
	\scnheader{следует отличать*}
	\begin{scnhaselementset}
		\scnitem{индивидуальная интеллектуальная компьютерная система нового поколения}
		\scnitem{коллективная интеллектуальная компьютерная система нового поколения}
		\begin{scnindent}
			\scnsubset{коллектив индивидуальных интеллектуальных компьютерных систем нового поколения}
			\scnsubset{иерархический коллектив интеллектуальных компьютерных систем нового поколения}
			\begin{scnindent}
				\scnidtf{коллектив интеллектуальных компьютерных систем нового поколения, членами которого могут быть как коллективные, так и индивидуальные интеллектуальные компьютерные системы}
			\end{scnindent}
		\end{scnindent} 
	\end{scnhaselementset}\scnheader{}
\end{SCn}

Индивидуальные интеллектуальные компьютерные системы нового поколения имеют следующие особенности:

\begin{textitemize}
	\item Индивидуальные интеллектуальные компьютерные системы нового поколения невозможно декомпозировать на подсистемы, которые можно разрабатывать независимо друг от друга и согласовывать только входам-выходам, реализуя принцип ``черного ящика''
	\item В индивидуальной интеллектуальной компьютерной системы нового поколения необходима конвергенция, совместимость и ``осмысленное'' взаимодействие самых различных видов знаний и моделей решения задач. То есть индивидуальная интеллектуальная компьютерная система нового поколения должна быть гибридной системой
\end{textitemize}

\begin{SCn}
	\scnheader{Технология OSTIS}
	\scnidtf{Комплексная технология Искусственного интеллекта, обеспечивающая
		\begin{scnitemize}
			\item совместимость всех частных технологий Искусственного интеллекта,
			\item совместимость и интероперабельность разрабатываемых интеллектуальных компьютерных систем
			\item поддержку не только проектирования интеллектуальных компьютерных систем, но и всего их жизненного пути
		\end{scnitemize}
	}
	\begin{scnrelfromlist}{предъявляемые требования}
		\scnitem{универсальность --- Технология OSTIS должна быть ориентирована на разработку и сопровождение интероперабельных интеллектуальных компьютерных систем \uline{любого} назначения}
		\scnitem{Технология OSTIS должна обеспечить перманентную эволюцию самой Технологии OSTIS (самой себя)}
	\end{scnrelfromlist}
\end{SCn}

\section*{Задачи текущего этапа работ в области Искусственного интеллекта в направлении создания интероперабельных интеллектуальных компьютерных систем}

Достижение указанной цели требует получения авторов на следующие вопросы:
\begin{textitemize}
	\item Какие требования предъявляются к интеллектуальным компьютерным системам, обеспечивающим указанную \uline{комплексную} автоматизацию человеческой деятельности
	\item Почему современные интеллектуальные компьютерные системы указанным выше требованиям не удовлетворяют и, соответственно, почему необходим переход к принципиально новому поколению интеллектуальных компьютерных систем
	\item Какие функциональные принципы должны лежать в основе интеллектуальных компьютерных систем нового поколения;
	\item Какие принципы должны лежать в основе \uline{максимально автоматизируемой} (!) технология проектирования и поддержки всего жизненного цикла интеллектуальной компьютерных систем нового поколения
	\item Какие принципы должны лежать в основе структуры и организации различных видов и областей человеческой деятельности в условиях её комплексной и максимально возможной автоматизации с помощью интеллектуальных компьютерных систем нового поколения
\end{textitemize}

%в другое место
%Теория человеческой деятельности в рамках Общества 5.0
%\begin{textitemize}
%	\item чёткая иерархичность (от индивидуальной деятельности людей и индивидуальных интеллектуальных компьютерных систем к различным уровням коллективов)
%	\item фрактальность !!!
%	\item стирание грани между деятельностью людей и интеллектуальной компьютерной системой
%	\item максимально возможная унификация
%	\item баланс унификации / многообразия
%	\item децентрализация
%\end{textitemize}

К числу текущих задач по созданию теории и технологии интеллектуальных компьютерных систем нового поколения относятся:
\begin{textitemize}
	\item Разработка теории многоагентных систем, агентами в которых являются индивидуальные или коллективные интерооперабельные компьютерные системы.
	\item Унификация и стандартизация различных моделей представления и обработки знаний. Эффект от данной унификации будет виден не сразу. Но, если этого происходить не будет, мы никогда не придем к эффективной \uline{комплексной} автоматизации человеческой деятельности. Эффективное многообразие методов и средств автоматизации приводит только к необоснованному \uline{дублированию} и повышению сложности использования и сопровождения разрабатываемых систем.
	\item Конвергенция и интеграция различных направлений Искусственного интеллекта.
	\item Различные направления Искусственного интеллекта сейчас имеют достаточно высокий уровень развития (signal processing, natural language processing (NLP), логические модели, онтологические модели, модели интеллектуальных компьютерных систем, многоагентные модели). Интеграция всех этих направлений является пусть и достаточно трудоёмкой задачей, но задачей вполне решаемой, в основе решения которой лежит согласование смежных понятий --- ``точек соприкосновения''. Но почему-то этого не происходит!!
	\item Конвергенция таких видов деятельности в области Искусственного интеллекта, как:
	\begin{textitemize}
		\item подготовка специалистов в области искусственного интеллекта;
		\item инженерная деятельность по разработке прикладных интеллектуальных компьютерных систем нового поколения;
		\item развитие технологии проектирования и поддержки жизненного цикла интеллектуальных компьютерных систем нового поколения;
		\item научно-исследовательская деятельность в области Искусственного интеллекта.
	\end{textitemize}
	\item Конвергенция деятельности в области Искусственного интеллекта с другими областями человеческой деятельности. Для развития Технологии OSTIS необходима также конвергенция Технологии OSTIS \uline{со всеми} видами и областями человеческой деятельности, которые не входят в состав деятельности в области Искусственного интеллекта. То есть дальнейшее развитие Технологии OSTIS носит ярко выраженный \uline{междисциплинарный характер}. Это означает что \uline{все знания}, накапливаемые человеческим обществом в самых различных областях должны быть представлены в составе Глобальной базы знаний Экосистемы OSTIS (с помощью порталов научно-технических, административных и прочих знаний), должны быть чётко \uline{стратифицированы} в виде иерархической системы \uline{семантически} совместимых \uline{онтологий} и тем самым превращены в иерархическую систему семантически совместимых компонентов баз знаний ostis-систем различного прикладного назначения.
	\item Обеспечение семантической совместимости интеллектуальных компьютерных систем нового поколения не только на этапе их проектирования, но и на всех последующих этапах их жизненного цикла.
	\item Разработка Модели \uline{коллективного} поведения интероперабельных интеллектуальных компьютерных систем, то есть модели децентрализованного коллективного решения задач в рамках:
	\begin{textitemize}
		\item многоагентной системы, агенты которой являются внутренними агентами индивидуальной интеллектуальной компьютерной системы, взаимодействующими через общую память (через общую базу знаний, хранимую в \uline{одной} памяти)
		\item многоагентной системы, агенты которой являются интероперабельными интеллектуальными компьютерными системами, взаимодействующими через общую базу знаний, хранимую в памяти корпоративной интеллектуальной компьютерной системы или в памяти координатора деятельности \uline{временного} коллектива интеллектуальной компьютерной системы 
	\end{textitemize}
	\item В рамках теории коллективного решения задач следует выделить два типа задач:
	\begin{textitemize}
		\item задача, соответствующая компетенции того коллектива интеллектуальных компьютерных систем, в рамках которого эта задача инициирована
		\item задача, выходящая за пределы компетенции того коллектива интеллектуальной компьютерной системы, в рамках которого эта задача инициирована. Эта задача требует формирования \uline{временного} коллектива, \uline{координатором} (но не менеджером) которого становится та интеллектуальная компьютерная система, в рамках которой эта задача инициировалась. Для этого необходимо найти те интеллектуальные компьютерные системы, которые в совокупности обеспечат необходимую компетенцию.
	\end{textitemize}
	Отметим при этом, что каждая интероперабельная интеллектуальная компьютерная система (как индивидуальная, так и коллективная) должна \uline{знать} свою компетенцию для того, чтобы каждая интероперабельная интеллектуальная компьютерная система могла быстро определить, сможет или не сможет она решить ту или иную заданную (возникшую) задачу. Это, в частности, необходимо для формирования \uline{временных} коллективов интеллектуальной компьютерной системы
	\item Разработка мощной библиотеки многократно используемых и совместимых компонентов интеллектуальной компьютерной системы нового поколения, которая обеспечивает \uline{полную} автоматизацию интеграции этих компонентов в процессе сборки проектируемых систем
	\item Перманентное расширение библиотеки многократно используемых компонентов интероперабельных интеллектуальных компьютерных систем должно стать важнейшим направлением в самых различных областях человеческой деятельности:
	\begin{textitemize}
		\item Разработчики любой интеллектуальной компьютерной интеллектуальной системы должны \uline{декомпозировать} разработанную интеллектуальную компьютерную систему на множество компонентов, вводимых в состав Библиотеки OSTIS --- так, чтобы разработка любой аналогичной системы свелась к сборке компонентов из Библиотеки OSTIS.
		\item Научно-техническая деятельность должна сводиться к развитию баз знаний различных интеллектуальных порталов научно-технических знаний. При этом база знаний каждого такого портала должна декомпозироваться на фрагменты, включаемые в состав Библиотеки многократно используемых компонентов баз знаний ostis-системы (для этого они соответствующим образом специфицируются), которые могут иерархически входить друг в друга.
		\item Таким образом \uline{все}(!) должны заботиться о \uline{расширении} Библиотеки OSTIS, это приводит к существенному снижению трудоёмкости разработки новых ostis-систем в рамках Экосистемы OSTIS. При этом авторство компонентов Библиотеки OSTIS должно \uline{поощряться}, что является фундаментальной основной развития рынка знаний, экономики знаний.  
	\end{textitemize}
\end{textitemize}

Если грамотно использовать Технологию OSTIS, то разработка любой ostis-системы сводиться в её автоматической сборке из указываемых разработчиком компонентов этой системы. Некоторые компоненты разрабатываемой ostis-системы входят в текущее состояние Библиотеки OSTIS, а некоторые из них дополнительно разработаны. Но при этом каждый такой новый компонент является либо результатом модификации существующего компонента из Библиотеки OSTIS, либо компонентом, который может и \uline{должен быть} специфицирован и включен в Библиотеку OSTIS. Таким образом разработчик прикладной ostis-системы должен разработать не только эту систему, но и внести вклад в развитие Библиотеки OSTIS, в результате которого разрабатываемая им ostis-система может быть собрана без дополнительно разрабатываемых компонентов, а только из компонентов Библиотеки OSTIS. Если все разработчики прикладных систем будут так действовать, то темпы повышения уровня автоматизации человеческой деятельности будут существенно возрастать.


