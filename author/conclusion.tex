\begin{partbacktext}
\part*{Заключение}
\label{chap_conclusion}
\addcontentsline{toc}{part}{Заключение}
\chapauthortoc{Голенков В.~В.\\Гулякина Н.~А.}

\vspace{-3\baselineskip}

\begin{SCn}
	\scnidtf{Заключение к Монографии OSTIS-2023}
	\scnidtf{Основные направления, проблемы и перспективы развития интеллектуальных компьютерных систем нового поколения и соответствующей им компьютерной технологии}
	
	\bigskip
	
	\begin{scnrelfromlist}{автор}
		\scnitem{Голенков В.~В.}
		\scnitem{Гулякина Н.~А.}
	\end{scnrelfromlist}

	\bigskip

	\begin{scnrelfromlist}{подраздел}
		\scnitem{\nameref{concl_feature_current_state_work_field_AI}}
		\scnitem{\nameref{concl_tasks_current_stage_theory_and_technology_development}}
		\scnitem{\nameref{concl_methodological_problems_current_stage_work_field_AI}}
		\scnitem{\nameref{concl_actual_projects_current_stage_work_development_technology}}
	\end{scnrelfromlist}

	\bigskip
	
	\begin{scnrelfromlist}{ключевой знак}
		\scnitem{индивидуальный субъект}
		\scnitem{коллектив субъектов}
		\begin{scnindent}
			\scnsubset{многоагентная система}
		\end{scnindent} 
		\scnitem{иерархический субъект}
		\scnitem{социальная ответственность\scnsupergroupsign}
		\scnitem{интероперабельность\scnsupergroupsign}
		\scnitem{индивидуальная деятельность}
		\scnitem{коллективная деятельность}
		\scnitem{интеллект коллектива субъектов\scnsupergroupsign}
		\begin{scnindent}
			\scnidtf{эффективность (качество) коллектива субъектов}
		\end{scnindent} 
		\scnitem{стратегическая цель субъекта*}
		\scnitem{иерархия целей*}
		\begin{scnindent}
			\scnidtf{иерархия задач*}
			\scnidtf{подзадачи*}
		\end{scnindent} 
		\scnitem{Общество}
		\begin{scnindent}
			\scnidtf{Человеческое сообщество}
		\end{scnindent} 
		\scnitem{Экосистема OSTIS}
		\begin{scnindent}
			\scnidtf{Вариант реализации Общества 5.0}
			\scnidtf{Общество + надстройка в виде Глобальной иерархической системы взаимосвязанных ostis-систем}
		\end{scnindent} 
	\end{scnrelfromlist}
	
	\bigskip
	
	\begin{scnrelfromlist}{библиографическая ссылка}
		\scnitem{}
		\scnitem{}
		\scnitem{}
	\end{scnrelfromlist}
	
\end{SCn}
	
\begin{SCn}
	\scnheader{интеллектуальная компьютерная система нового поколения}
	\scneq{\textup{(} самообучаемая интеллектуальная компьютерная система $\bigcap$ интероперабельная интеллектуальная компьютерная система \textup{)}}
\end{SCn}

\begin{SCn}
	\scnheader{самообучаемая интеллектуальная компьютерная система}
	\begin{scnitemize}
		\item высокие темпы самостоятельно реализуемой эволюции!
		\item существенное снижение тредоёмкости (затрат, накладных расходов) на модернизацию!
	\end{scnitemize}
\end{SCn}
	
\end{partbacktext}


\section*{Особенность текущего состояния работ в области Искусственного интеллекта --- интеллектуальные компьютерные системы нового поколения}
\label{concl_feature_current_state_work_field_AI}

Необходимость перехода от современных \textit{компьютерных систем} (в том числе, и от современных \textit{интеллектуальных компьютерных систем}) к \textit{интеллектуальным компьютерных системам нового поколения} обусловлена необходимостью перехода к автоматизации всё более и более сложных видов и областей человеческой деятельности требующих создания целых комплексов интеллектуальных компьютерных систем, способных быстро эволюционировать и эффективно взаимодействовать между собой в процессе \myuline{коллективного} решения сложных задач.

Компьютерные системы, обладающие указанными способностями являются \textit{компьютерные системы нового поколения}. Поскольку указанные компьютерные системы не могут не иметь высокого \textit{уровня интеллекта}, мы также называем их \textit{интеллектуальными компьютерными системами нового поколения}. Высокий \textit{уровень интеллекта} интероперабельным компьютерным системам необходим:
\begin{textitemize}
	\item для адекватной оценки собственной компетенции и компетенции своих партнеров;
	\item для обеспечения взаимопонимания, договороспособности и координации (согласованности) своих действий с действиями партнеров в ходе \textit{коллективного решения сложных задач} в условиях возможного возникновения непредсказуемых (нештатных) обстоятельств.
\end{textitemize}

Очевидно, что для создания и эксплуатации \textit{интеллектуальных компьютерных систем нового поколения} необходимо:
\begin{textitemize}
	\item разработать общую формальную теорию таких систем;
	\item разработать комплексную технологию проектирования и поддержки жизненного цикла этих систем;
	\item разработать общую формальную теорию всего многообразия видов и областей человеческой деятельности, которые целесообразно автоматизировать.
\end{textitemize}

В основе предлагаемого нами подхода к построению \textit{интеллектуальных компьютерных систем нового поколения} лежат следующие принципы:
\begin{textitemize}
	\item смысловое представление знаний, хранимых в памяти \textit{интерооперабельных интеллектуальных компьютерных систем};
	\item онтологическая структуризация и систематизация хранимых в памяти знаний;
	\item децентрализованная ситуационная агенто-ориентированная организация процесса решения задач;
	\item конвергенция и глубокая (диффузная) интеграция различных моделей решения задач и, как следствие гибридный характер решателей задач;
	\item смысловая интеграция входной информации, поступающей извне по разным сенсорным каналам и на разных языках путём трансляции входной информации на общий язык внутреннего смыслового представления знаний.
\end{textitemize}

\begin{SCn}
	\scnheader{следует отличать*}
	\begin{scnhaselementset}
		\scnitem{индивидуальная интеллектуальная компьютерная система нового поколения}
		\scnitem{коллективная интеллектуальная компьютерная система нового поколения}
		\begin{scnindent}
			\scnidtf{коллектив индивидуальных интеллектуальных компьютерных систем нового поколения}
			\scnsubset{иерархический коллектив интеллектуальных компьютерных систем нового поколения}
			\begin{scnindent}
				\scnidtf{коллектив интеллектуальных компьютерных систем нового поколения, членами которого могут быть как коллективные, так и индивидуальные интеллектуальные компьютерные системы}
			\end{scnindent}
		\end{scnindent} 
	\end{scnhaselementset}\scnheader{}
\end{SCn}

\begin{SCn}
	\scnheader{индивидуальная интеллектуальная компьютерная система нового поколения}
	\begin{scnrelfromset}{особенности}
		\scnitem{Индивидуальные интеллектуальные компьютерные системы нового поколения невозможно декомпозировать на подсистемы, которые можно разрабатывать абсолютно независимо друг от друга и согласовывать только входам-выходам, реализуя принцип "черного ящика"{}}
		\scnitem{В индивидуальной интеллектуальной компьютерной системы нового поколения необходима конвергенция, совместимость и "осмысленное"{} взаимодействие самых различных видов знаний и моделей решения задач. То есть индивидуальная интеллектуальная компьютерная система нового поколения должна быть \textit{гибридной системой}}
	\end{scnrelfromset}
\end{SCn}

\bigskip

\begin{SCn}
	\scnheader{Технология OSTIS}
	\scnidtf{Предложенная нами технология разработки и сопровождения интеллектуальных компьютерных систем нового поколения}
	\scnidtf{Open Semantic Technology for Intelligent Systems}
	\scnidtf{Комплексная технология Искусственного интеллекта, обеспечивающая совместимость всех частных технологий Искусственного интеллекта, совместимость и интероперабельность разрабатываемых интеллектуальных компьютерных систем, поддержку не только проектирования интеллектуальных компьютерных систем, но и всего их жизненного цикла}
	\begin{scnrelfromlist}{предъявляемые требования}
		\scnitem{\textup{универсальность --- Технология OSTIS ориентирована на разработку и сопровождение интероперабельных интеллектуальных компьютерных систем \myuline{любого} назначения}}
		\scnitem{\textup{Технология OSTIS обеспечивает перманентную эволюцию самой Технологии OSTIS (самой себя)}}
	\end{scnrelfromlist}
\end{SCn}

\begin{SCn}
	\scnheader{высокий уровень самообучаемости интеллектуальной компьютерной системы}
	\begin{scnrelfromset}{обеспечивается}
		\scnitem{высоким уровнем гибкости интеллектуальной компьютерной системы}
		\scnitem{высоким уровнем стратифицированности интеллектуальной компьютерной системы}
		\scnitem{высоким уровнем рефлексивности интеллектуальной компьютерной системы}
		\scnitem{высоким уровнем познавательной активности}
 		\scnitem{}
	\end{scnrelfromset}
\end{SCn}

\begin{SCn}
	\scnheader{обучаемость}
	\begin{scnitemize}
		\item способность мониторить состояние и динамику окружающей среды и корректировать свои действия при соответствующих изменениях окружающей среды (адаптивность) 
		\item способность анализировать качество и повышать Качество собственной базы знаний (структуризация противоречий информационной дыры, информационного мусора)
		\item способность извлекать знания из внешних источников информации (big data)
		\item способность анализировать и повышать качество собственной деятельности (в том числе способность учиться на собственных ошибках)
		\item способность анализировать качество деятельности других субъектов и извлекать из этого пользу для себя (учиться на чужих ошибках)
	\end{scnitemize}
\end{SCn}

\section*{Задачи текущего этапа разработки теории и технологии интеллектуальных компьютерных систем нового поколения}
\label{concl_tasks_current_stage_theory_and_technology_development}

Создание интеллектуальных компьютерных систем нового поколения требует получения ответов на следующие вопросы:
\begin{textitemize}
	\item Какие требования предъявляются к интеллектуальным компьютерным системам, обеспечивающим указанную выше \myuline{комплексную} автоматизацию человеческой деятельности
	\item Почему современные интеллектуальные компьютерные системы указанным выше требованиям не удовлетворяют и, соответственно, почему необходим переход к принципиально новому поколению интеллектуальных компьютерных систем
	\item Какие фундаментальные принципы должны лежать в основе интеллектуальных компьютерных систем нового поколения;
	\item Какие принципы должны лежать в основе \myuline{максимально автоматизируемой} технология проектирования и поддержки всего жизненного цикла интеллектуальной компьютерных систем нового поколения
	\item Какие принципы должны лежать в основе структуры и организации различных видов и областей человеческой деятельности для обеспечения её комплексной и максимально возможной автоматизации с помощью интеллектуальных компьютерных систем нового поколения
\end{textitemize}

%в другое место
%Теория человеческой деятельности в рамках Общества 5.0
%\begin{textitemize}
%	\item чёткая иерархичность (от индивидуальной деятельности людей и индивидуальных интеллектуальных компьютерных систем к различным уровням коллективов)
%	\item фрактальность !!!
%	\item стирание грани между деятельностью людей и интеллектуальной компьютерной системой
%	\item максимально возможная унификация
%	\item баланс унификации / многообразия
%	\item децентрализация
%\end{textitemize}

К числу текущих фундаментальных задач по созданию теории и технологии интеллектуальных компьютерных систем нового поколения относятся:
\begin{textitemize}
	\item Разработка теории иерархических многоагентных систем, агентами в которых являются индивидуальные или коллективные интерооперабельные интеллектуальные компьютерные системы.
	\item Унификация и стандартизация различных моделей представления и обработки знаний. Эффект от данной унификации будет виден не сразу. Но, если этого происходить не будет, мы никогда не придем к эффективной \myuline{комплексной} автоматизации человеческой деятельности. Эклектичное многообразие методов и средств автоматизации приводит не только к необоснованному \myuline{дублированию} разрабатываемых систем и повышению сложности их использования и сопровождения.
	\item Конвергенция и интеграция различных направлений Искусственного интеллекта.
	
	Различные направления Искусственного интеллекта сейчас имеют достаточно высокий уровень развития (signal processing, natural language processing, логические модели, онтологические модели, модели интеллектуальных компьютерных систем, многоагентные модели и многое другое). Интеграция всех этих направлений является пусть и достаточно трудоёмкой задачей, но задачей вполне решаемой, в основе решения которой лежит согласование смежных понятий --- "точек соприкосновения"{}.
	\item Конвергенция таких видов деятельности в области Искусственного интеллекта, как:
	\begin{textitemize}
		\item подготовка специалистов в области искусственного интеллекта;
		\item инженерная деятельность по разработке прикладных интеллектуальных компьютерных систем нового поколения;
		\item развитие технологии проектирования и поддержки жизненного цикла интеллектуальных компьютерных систем нового поколения;
		\item научно-исследовательская деятельность в области Искусственного интеллекта.
	\end{textitemize}
	\item Для развития Технологии интеоперабельных компьютерных интеллектуальных систем необходима также конвергенция Технологии интеоперабельных компьютерных интеллектуальных систем \myuline{со всеми} видами и областями человеческой деятельности, которые не входят в состав деятельности в области Искусственного интеллекта. То есть дальнейшее развитие Технологии интеоперабельных компьютерных интеллектуальных систем носит ярко выраженный \myuline{междисциплинарный характер}. Это означает что \myuline{все знания}, накапливаемые человеческим обществом в самых различных областях должны быть представлены в составе Глобальной базы знаний Экосистемы интеоперабельных компьютерных интеллектуальных систем (с помощью порталов научно-технических, административных и прочих знаний), должны быть чётко \myuline{стратифицированы} в виде иерархической системы \myuline{семантически} совместимых \myuline{онтологий} и тем самым превращены в иерархическую систему семантически совместимых компонентов баз знаний интеоперабельных компьютерных интеллектуальных систем различного прикладного назначения. Автоматизаций беспорядка, приводит к ещё большему беспорядку.
	\item Обеспечение семантической совместимости интеллектуальных компьютерных систем нового поколения не только на этапе их проектирования, но и на всех последующих этапах их жизненного цикла.
	\item Разработка модели \myuline{коллективного} поведения интероперабельных интеллектуальных компьютерных систем, то есть модели децентрализованного коллективного решения задач в рамках:
	\begin{textitemize}
		\item многоагентной системы, агенты которой являются внутренними агентами индивидуальной интеллектуальной компьютерной системы, взаимодействующими через общую память (через общую базу знаний, хранимую в \myuline{одной} памяти)
		\item многоагентной системы, агенты которой являются интероперабельными интеллектуальными компьютерными системами, взаимодействующими через общую базу знаний, хранимую в памяти корпоративной интеллектуальной компьютерной системы или в памяти координатора деятельности \myuline{временного} коллектива интеллектуальной компьютерной системы 
	\end{textitemize}
	\vspace{-1\baselineskip}
	В рамках теории коллективного решения задач следует выделить два типа задач:
	\begin{textitemize}
		\item задача, соответствующая компетенции того коллектива интеллектуальных компьютерных систем, в рамках которого эта задача инициирована
		\item задача, выходящая за пределы компетенции того коллектива интеллектуальной компьютерной системы, в рамках которого эта задача инициирована. Эта задача требует формирования \myuline{временного} коллектива, \myuline{координатором} (но не менеджером) которого становится та интеллектуальная компьютерная система, в рамках которой эта задача инициировалась. Для этого необходимо найти те интеллектуальные компьютерные системы, которые в совокупности обеспечат необходимую компетенцию.
	\end{textitemize}
	\vspace{-1\baselineskip}
	Отметим при этом, что каждая интероперабельная интеллектуальная компьютерная система (как индивидуальная, так и коллективная) должна \myuline{знать} свою компетенцию для того, чтобы каждая интероперабельная интеллектуальная компьютерная система могла быстро определить, сможет или не сможет она решить ту или иную заданную (возникшую) задачу. Это, в частности, необходимо для формирования \myuline{временных} коллективов интеллектуальной компьютерной системы
	\item Разработка принципов, лежащих в основе мощной библиотеки многократно используемых и совместимых компонентов интеллектуальной компьютерной системы нового поколения, которая обеспечивает \myuline{полную} автоматизацию интеграции этих компонентов в процессе сборки проектируемых систем\\
%	\vspace{-1\baselineskip}
	Разработка методик и средств перманентного расширения библиотеки многократно используемых компонентов интероперабельных интеллектуальных компьютерных систем в самых различных областях человеческой деятельности:
	\begin{textitemize}
		\item Разработчики любой интеллектуальной компьютерной интеллектуальной системы должны \myuline{декомпозиро-} \myuline{вать} разработанную интеллектуальную компьютерную систему на множество компонентов, включаемых в состав Библиотеки Технологии интероперабельных интеллектуальных компьютерных систем --- так, чтобы разработка любой аналогичной системы свелась к сборке компонентов из Библиотеки Технологии интероперабельных интеллектуальных компьютерных систем.
		\item Научно-техническая деятельность в любой области должна сводиться к развитию баз знаний различных интеллектуальных порталов научно-технических знаний. При этом база знаний каждого такого портала должна декомпозироваться на фрагменты, включаемые в состав Библиотеки многократно используемых компонентов баз знаний ostis-системы (для этого они должны соответствующим образом специфицироваться), которые могут иерархически входить друг в друга.
		\item Таким образом \myuline{все}(!) разработчики должны заботиться о \myuline{расширении} Библиотеки Технологии интероперабельных интеллектуальных компьютерных систем, что приведёт к существенному снижению трудоёмкости разработки новых ostis-систем в рамках Экосистемы интероперабельных интеллектуальных компьютерных систем. При этом авторство компонентов Библиотеки Технологии интероперабельных интеллектуальных компьютерных систем должно \myuline{поощряться}, что является фундаментальной основной развития рынка знаний, экономики знаний.  
	\end{textitemize}
	\vspace{-1\baselineskip}
	Если грамотно развиваться и использовать Технологию интероперабельных интеллектуальных компьютерных систем, то разработка любой новой интероперабельной интеллектуальной компьютерной системы будет в основном сводиться к её автоматической сборке из указываемых разработчиком компонентов этой системы. Некоторые компоненты разрабатываемой интероперабельной интеллектуальной компьютерной системы могут входить в текущее состояние Библиотеки Технологии интероперабельных интеллектуальных компьютерных систем, а некоторые из них будут требовать дополнительной разработки. Но при этом каждый такой новый компонент чаще всего является результатом модификации существующего компонента из Библиотеки Технологии интероперабельных интеллектуальных компьютерных систем и \myuline{должен быть} специфицирован и включен в Библиотеку Технологии интероперабельных интеллектуальных компьютерных систем. Таким образом разработчик прикладной интероперабельной интеллектуальной компьютерной системы должен разработать не только эту систему, но и внести вклад в развитие Библиотеки Технологии интероперабельных интеллектуальных компьютерных систем, в результате которого разрабатываемая им интероперабельная интеллектуальная компьютерная система может быть собрана без дополнительно разрабатываемых компонентов, а только из компонентов Библиотеки Технологии интероперабельных интеллектуальных компьютерных систем. Если все разработчики прикладных систем будут так действовать, то темпы повышения уровня автоматизации человеческой деятельности будут существенно возрастать.
\end{textitemize}

\section*{Методологические проблемы текущего этапа работ в области Искусственного интеллекта}
\label{concl_methodological_problems_current_stage_work_field_AI}

\begin{SCn}
	\scntext{эпиграф}
	{Самые сложные проблемы --- это те, которые мы не осознаем и, особенно, те, причиной которых является наше несовершенство.
	}
	
	\begin{scnrelfromlist}{подраздел}
		\scnitem{\textup{Социальная ответственность специалистов в области Искусственного интеллекта}}
		\scnitem{\textup{Глобальная цель деятельности в области Искусственного интеллекта}}
		\scnitem{\textup{Общие требования, предъявляемые к специалистам в области Искусственного интеллекта}}
		\scnitem{\textup{Требования, предъявляемые к фундаментальной подготовке специалистов в области Искусственного интеллекта}}
		\scnitem{\textup{Проблемы текущего этапа разработки теории и технологии интеллектуальных компьютерных систем нового поколения}}
	\end{scnrelfromlist}
\end{SCn}

\subsection*{Социальная ответственность специалистов в области Искусственного интеллекта}
Современный этап развития теории и практики Искусственного интеллекта обнажает целый спектр проблем, препятствующих этому развитию. Дальнейшее развитие технологий Искусственного интеллекта
\begin{textitemize}
	\item с одной стороны, может и достаточно быстро осуществить переход современного общества на принципиально новый уровень его эволюции, обеспечивающий \myuline{комплексную} автоматизацию всех подлежащих автоматизации видов и областей человеческой деятельности, а также обеспечивающий максимально возможный комфорт и максимально возможное раскрытие творческого потенциала \myuline{каждого} человека;
	\item с другой стороны, может достаточно долго и весьма убедительно для неграмотного обывателя \myuline{имитировать} указанный прогресс автоматизации человеческой деятельности --- любая даже весьма достойная цель может быть загублена имитацией её достижения;
	\item с третьей стороны, может достаточно быстро привести человеческое общество к деградации и самоуничтожению.
\end{textitemize}

Таким образом на современном этапе развития технологий Искусственного интеллекта, \myuline{уровень социальной} \myuline{ответственности} специалистов в области Искусственного интеллекта является определяющим фактором развития человеческого общества.
Опасность для человеческого общества исходит не от интеллектуальных компьютерных систем, а от мотивации специалистов, которые разрабатывают эти системы. Очевидно, что создание интеллектуальных компьютерных систем, предназначенных \myuline{осознанного} нанесения любого ущерба человеческому обществу, и требующих создания соответствующих интеллектуальных средств обеспечения безопасности, является короткой дорогой к самоуничтожению.

Усилия специалистов в области Искусственного интеллекта должны быть направлены на существенное повышение уровня интеллекта человеческого общества в целом, основой чего является \myuline{комплексная} автоматизация \myuline{всех} тех видом и областей человеческой деятельности, которые принципиально имеет смысл автоматизировать.

\subsection*{Глобальная цель деятельности в области Искусственного интеллекта}
Рассмотрим вопрос, Почему современный этап деятельности в области Искусственного интеллекта требует формулировки глобальной цели этой деятельности и перманентного её уточнения.

Современное состояние Искусственного интеллекта можно охарактеризовать как глубокий методологический кризис, обусловленных:
\begin{textitemize}
	\item тем, что научные результаты в этой области вышли из научных лабораторий и стали приносить реальную практическую пользу;
	\item отсутствием понимания того, что получение серьезных научных результатов в той или иной области и создание технологий, обеспечивающих \myuline{эффективное} практическое использование этих результатов --- это соизмеримые по значимости и сложности задачи. Особенно это касается Искусственного интеллекта.
\end{textitemize}

Последнее обстоятельство приводит к неоправданной эйфории, иллюзии благоплучия и к бурно расцветающей эклектике, которая абсолютно игнорирует даже казалось бы очевидные законы общей теории систем.

К сожалению, локальное внедрение результатов научных исследований в области Искусственного интеллекта, локальная автоматизация бизнес-процессов какой-либо организации без учёта системной организации всего комплекса методов и средств автоматизации различных видов и областей человеческой деятельности приводит к неоправданному дублированию результатов. 

Современное состояние работ в области Искусственного интеллекта требует уточнения. Если в ближайшее время не произойдёт осознания \textit{глобальной (стратегической) цели} работ в области Искусственного интеллекта, то деятельность в этой области в целом будет осуществляться в стиле "лебедя, рака и щуки"{}. Трата усилий не приведёт к целостному практически значимому результату. "Вектора"{} конкретных направлений этой деятельности, "вектора"{} наших усилий не будут иметь одинаковую направленность, что существенно снизит общую  производительность всей этой деятельности и качество общего (суммарного) результата.

Какова же должна быть \textit{стратегическая задача} (сверхзадача), которую должны решить специалисты в области Искусственного интеллекта. Очевидно, что такой сверхзадачей является переход всего комплекса человеческой деятельности на принципиально новый уровень максимально возможной его автоматизации, в рамках которого принципиально неавтоматизируемой частью человеческой деятельности остаётся, прежде всего, \textit{творческая} деятельность, в частности, научно-исследовательская деятельность, преподавательская и воспитательная деятельность, перманентное повышение уровня комплексной автоматизации человеческой деятельности. Основная цель \textit{комплексной} автоматизации человеческой деятельности заключается не только в том, чтобы автоматизировать то, что \myuline{можно эффективно} автоматизировать с помощью методов Искусственного интеллекта, а в том, чтобы автоматизировать \myuline{все}(!) "узкие места"{} человеческой деятельности, которые определяют общую её производительность в различных областях.

Таким образом, в настоящее время технологии Искусственного интеллекта находятся на пороге перехода к принципиально новому уровню развития --- на пороге перехода от решения частных (локальных) задач к решению глобальной задачи комплексной автоматизации всех видов и областей человеческой деятельности, что требует автоматизации решения не только частных актуальных и важных задач, но и автоматизации решения задач всё более и более высокого уровня, для которых автоматизируемые сейчас задачи становятся подзадачами. Другими словами, при автоматизации решения комплексных задач (надзадач) автоматизация фокусируется на разработке методов и средств \textit{взаимодействия между средствами решения локальных задач} (частных задач).

Перенос акцента на автоматизацию решения не просто интеллектуальных задач, а на автоматизацию решения \myuline{комплексных} задач, подзадачами которых являются \myuline{разнообразные} интеллектуальные задачи, не только переводит технологии Искусственного интеллекта на принципиально новый уровень, но и окажет существенное влияние \myuline{на все стороны человеческой деятельности}:
\begin{textitemize}
	\item  научно-исследовательские и научно-технические работы должны приобрести конвергентный взаимообогащающий характер;
	\item основой образования должна стать междисциплинарность;
	\item основой глобальной автоматизации человеческой деятельности должна стать общая комплексная формальная и перманентно совершенствуемая теория человеческой деятельности, в основе которой должна лежать междисциплинарная конвергентная методика, направленная на преодоление эклектичного подхода.
\end{textitemize}

Следовательно, основной целью комплексной автоматизации всевозможных видов и областей человеческий деятельности с помощью \textit{интеоперабельных интеллектуальных компьютерных систем} является существенное повышение \textit{уровня интеллекта}  человеческого общества в целом.

Современное человеческое общество --- это сложнейшая распределённая многоагентная кибернетическая система, развитие которой осуществляется, к сожалению, с нарушением многих законов кибернетики и, в частности, с нарушением критериев, определяющих уровень интеллекта иерархических многоагентных систем. Уровень интеллекта таких систем определяется целым рядом казалось бы очевидных факторов:
\begin{textitemize}
	\item тем, каков объём и качество знаний, накопленных многоагентной системой и доступных всем агентам (субъектам), входящим в эту систему
	\begin{textitemize}
		\item насколько этих знаний достаточно для организации управления деятельностью этой системы;
		\item насколько эти знания корректны (противоречивы) и адекватны;
		\item насколько велика конвергентность, компактность и чистота этих знаний (здесь учитывается наличие информационного мусора, информационного дублирования);
		\item насколько хорошо структурированы (систематизированы) накапливаемые знаний;
	\end{textitemize}
	\item тем, как осуществляется доступ каждого агента многоагентной системы к знаниям, хранимым в общей памяти всей многоагентной системы;
	\item тем, как эти знания накапливаются и эволюционируют (как многоагентная система самообучается)
	\begin{textitemize}
		\item как многоагентная система учится на собственных ошибках,
		\item как многоагентная система повышает качество своих знаний;
	\end{textitemize}
	\item тем, как многоагентная система в целом и каждый агент в частности используют накопленные в общей памяти знания для решения различных задач.
\end{textitemize}

Таким образом, если рассматривать современное человеческое общество с позиций теории многоагентных систем, являющихся коллективами интеллектуальных систем (в данном случае, не искусственных, а естественных интеллектуальных систем), то очевидно, что следующий этап его эволюций требует:
\begin{textitemize}
	\item автоматизации накопления, анализа и перманентного повышения качества накапливаемых человеком знаний;
	\item автоматизации эффективного использования накопленных человечеством знаний при решении задач самого различного уровня, требующих формирования различных коллективов, состоящих из людей и интеллектуальных компьютерных систем. Каждый такой коллектив предназначается либо для решения какой-либо одной конкретной задачи, либо для решений некоторого множества задач в некоторой области;
	\item повышения уровня конвергенции знаний, методов, действий, создаваемых технических систем;
	\item повышения уровня интероперабельности (как для интеллектуальных компьютерных систем, так и для людей) 
\end{textitemize}

\subsection*{Общие требования, предъявляемые к специалистам в области Искусственного интеллекта}

\begin{SCn}
	\begin{scnrelfromlist}{эпиграф}
		\scnitem{\textup{Требования, предъявляемые к специалистам в области Искусственного интеллекта на \myuline{новом} этапе развития этой области, являются отражением требований, предъявляемых к интеллектуальным компьютерным системам и соответствующим технологиям \myuline{нового} поколения}}
		\scnitem{\textup{Уровень интеллекта (в том числе коллективного интеллекта) разработчиков интеллектуальной компьютерной системы не может быть ниже уровня интеллекта создаваемых интеллектуальной компьютерной системы}}
		\scnitem{\textup{Уровень интеллекта коллектива агентов далеко не всегда выше уровня интеллекта входящих в него агентов}}
		\scnitem{\textup{Разработка интероперабельных интеллектуальных компьютерных систем может быть только коллективной}}
		\scnitem{\textup{Коллектив неинтероперабельных разработчиков не может создавать интероперабельные интеллектуальные компьютерные системы}}
	\end{scnrelfromlist}
\end{SCn}

Высокий уровень \textit{социальной ответственности}, требуемый от \textit{специалистов в области Искусственного интеллекта}, предъявляет к ним целый ряд очевидных, но, к сожалению, часто не учитываемых \myuline{общих требований}, необходимых для качественного участия в сложных коллективных социально значимых проектах. К таким общим требованиям относятся:

\begin{textitemize}
	\item  высокий уровень \textbf{\textit{мотивации}} к участию в \myuline{перманентной} эволюции целостного технологического комплекса, обеспечивающего разработку эффективных интероперабельных интеллектуальных компьютерных систем. \myuline{Комплексная} и \myuline{высококачественная} технология разработки и сопровождения интероперабельных интеллектуальных компьютерных систем должна рассматриваться как \myuline{ключевой} продукт коллективной деятельности в области Искусственного интеллекта.\\
	Указанная мотивация предполагает соответствующую целеустремлённость, отсутствие эгоизма, высокомерия, индивидуализма, изоляционизма, паразитизма;
	\item высокий уровень \textbf{\textit{созидательной активности}}, пассионарности, смелости;
	\item высокий уровень \textbf{\textit{рефлексии}} --- способности анализировать собственные цели и действия и исправлять собственные ошибки, а также анализировать цели, действия и ошибки, совершаемые коллективом, членом которого специалист является.\\
	Одно дело искренне признавать логичность и целесообразность соблюдения тех или иных правил (принципов, требований) и совсем другое дело уметь видеть и исправлять \textit{собственные} нарушения этих правил. Без такой рефлексии прогресс коллективного творчества невозможен. Знать то, как \myuline{надо} делать и реально следовать этому --- не одно и то же.
	\item высокий уровень \textbf{\textit{собственной интероперабельности}}
	\begin{textitemize}
		\item семантической совместимости, требующей перманентного мониторинга текущего состояния и эволюции технологического комплекса;
		\item договороспособности --- способности оперативно согласовывать свои цели и планы, детонационную семантику понятий и терминов, а также децентрализовано распределять подзадачи коллективно решаемой задачи;
		\item способности координировать и синхронизировать свои действия с коллегами в условиях возможного возникновения непредсказуемых обстоятельств является основной проблемой создания и развития технологического обеспечения интеоперабельных интеллектуальных компьютерных систем.
	\end{textitemize}
	\item отсутствие достаточного уровня интероперабельности у разработчиков, без которого невозможно обеспечить:
	\begin{textitemize}
		\item конвергенцию, унификацию, стандартизацию интероперабельных интеллектуальных компьютерных систем;
		\item формирование мощной библиотеки типовых компонентов интеллектуальной компьютерной системы;
		\item существенное снижение трудоёмкости и повышение уровня автоматизации разработки и сопровождения интероперабельных интеллектуальных компьютерных систем;
		\item построение общей теории Экосистемы интероперабельных интеллектуальных компьютерных систем и, соответственно, общей теории человеческой деятельности.\\
		Таким образом, для создания интероперабельных интеллектуальных компьютерных систем необходимо. чтобы сами их создатели имели высокий уровень интероперабельности. Проблема обеспечить это является основным вызовом, который адресован специалистам в области Искусственного интеллекта на текущем этапе развития этой области.\\
		Основной причиной, препятствующий формированию необходимого уровня интеоперабельности у специалистов в области Искусственного интеллекта, является \textit{конкурентный стиль взаимоотношений} между специалистами. Этот стиль взаимоотношений является широко распространённым способом стимулировать активность сотрудников. Но это не единственный способ стимулировать творческую активность в решении стратегически важных задач, каковой, в частности, является задача эффективной комплексной автоматизации всех видов и областей человеческой деятельности с помощью интеоперабельных интеллектуальных компьютерных систем. Более того, конкуренция провоцирует эгоизм и игнорирование интересов иных субъектов (в том числе, и интересов того коллектива, членом которого субъект является). Таким образом конкуренция явно противоречит принципам интероперабельности и, соответственно, принципам организации интеллектуальных сообществ, интеллектуальных творческих коллективов и организаций.\\
		От конкретного стиля взаимоотношений необходимо переходить к \myuline{взаимовыгодному} взаимодействию между субъектами всех уровней иерархии. В этом заключается основная суть интероперабельности и перехода к интеллектуальным коллективам и интеллектуальному обществу.
	\end{textitemize}
	Заметим, что перечисленные общие требования, предъявляемые к специалистам в области Искусственного интеллекта на современном этапе развития технологий Искусственного интеллекта, должны предъявляться не только к ним, но и ко всем людям, готовым способствовать технологическому прогрессу. Просто на данном этапе основная ответственность за это лежит именно на специалистах в области технологий Искусственного интеллекта.
\end{textitemize}

\subsection*{Требования, предъявляемые к фундаментальной подготовке специалистов в области Искусственного интеллекта}
Необходимость существенного повышения уровня практической значимости и эффективности работ в области Искуссвтеного интеллекта, требующего перехода к интеллектуальным компьютерным системам нового поколения и к принципиально новому технологическому комплексу, предъявляет к специалистам в области Искусственного интеллекта не только общие требования, необходимые для эффективного участия в сложных коллективных социально значимых проектах, но так же и высокие требования к их \myuline{фундаментальной} профессиональной подготовке:
\begin{textitemize}
	\item высокому уровню системной культуры, позволяющей "видеть"{} иерархию сложных систем, связи между различными уровнями и иерархии, разницу между тактическими и стратегическими задачами;
	\item высокому уровню математической культуры, культуры формализации;
	\item высокому уровню технологической культуры и технологической дисциплины;
	\item высокому уровню обучаемости в условиях быстрого изменения технологической инфраструктуры
\end{textitemize}

\subsection*{Проблемы текущего этапа разработки теории и технологии интеллектуальных компьютерных систем нового поколения}
\textbf{Перечислим основные методологические проблемы текущего этапа работ в области Искусственного интеллекта}, которые препятствуют решению рассматриваемых выше фундаментальных задач:
\begin{textitemize}
	\item 
	Недостаточно высокий уровень осознания специалистами в области Искусственного интеллекта своей социальной ответственности
	\item 
	Отсутствие согласованного осознания глобальной цели работ в области Искусственного интеллекта, которая заключается в поэтапном повышении \textbf{\textit{уровня интеллекта}} человеческого общества путем \myuline{комплексной} автоматизации всех аспектов его деятельности с помощью сети взаимодействующих между собой интеллектуальных компьютерных систем
	\item 
	Недостаточно высокий уровень интеоперабельности специалистов в области Искусственного интеллекта и преобладание конкретного стиля взаимоотношений. Следствием этого является недостаточное количество мотивированных специалистов в области Искусственного интеллекта, способных к эффективному \myuline{творческому} взаимодействию. Для того, чтобы они появились в достаточном количестве, хорошей системы их \myuline{профессиональной} подготовки недостаточно. Следует также отметить, что хорошие человеческие отношения, психологическая атмосфера и Team Building в команде разработчиков, к которому серьёзно относятся многие компании, является необходимым, но далеко не достаточным условием результативности коллективной разработки сложных компьютерных систем (особенно это касается интеропреабельных интеллектуальных компьютерных систем)
	\item 
	Недостаточно высокий уровень \myuline{комплексной фундаментальной} подготовки специалистов в области Искусственного интеллекта
	\item 
	Ярко выраженный \myuline{междисциплинарный} характер Искусственного интеллекта как области человеческой деятельности
	\item 
	Отсутствие осознания необходимости глубокой конвергенции между различными направлениями Искусственного интеллекта и формализации всего комплекса знаний в области Искусственного интеллекта для их использования в базах знаний интеллектуальных компьютерных систем (прежде всего, инструментальных интеллектуальных компьютерных систем, входящих в состав технологического комплекса разработки и сопровождения интеллектуальных компьютерных систем)
	\item 
	Высокий уровень сложности комплексной формализации \myuline{всех} накапливаемых человечеством знаний (прежде всего, в области математики и общей теории систем) и их \myuline{конвергенции} с комплексом знаний, накапливаемых и формализуемых в области Искусственного интеллекта. Это необходимо для \myuline{непосредственного} использования накапливаемых человечеством знаний в интеллектуальных компьютерных системах различного назначения
	\item 
	Отсутствие осознания необходимости глубокой конвергенции и согласованности между
	\begin{textitemize}
		\item научно-исследовательской деятельностью в области Искусственного интеллекта
		\item деятельностью, направленной на развитие частных технологий Искусственного интеллекта, а также комплексной технологии проектирования и поддержки жизненного цикла интероперабельных интеллектуальных компьютерных систем
		\item инженерной деятельностью, направленной на разработку конкретных интеллектуальных компьютерных систем различного назначения
		\item образовательной деятельностью, направленной на подготовку специалистов в области Искусственного интеллекта
	\end{textitemize}
	\item 
	Проблема обеспечения \textbf{\textit{семантической совместимости}} интероперабельных интеллектуальных компьютерных систем не только на этапе их проектирования, но и на протяжении всего их жизненного цикла в условиях перманентной эволюции самих интеллектуальных компьютерных систем в ходе их эксплуатации, а также перманентной эволюции комплексной технологии их разработки
\end{textitemize}

Основная часть указанных проблем заключается в необходимости перехода к принципиально новому \myuline{стилю и орга-} \myuline{низации} взаимодействия специалистов в области Искусственного интеллекта, без чего невозможен переход от частных теорий Искусственного интеллекта к \textbf{\textit{Общей теории интеллектуальных компьютерных систем}}, обеспечивающей \myuline{совместимость} всех частных теорий Искусственного интеллекта, а также переход от частных технологий Искусственного интеллекта к \textbf{\textit{Компьютерной технологии Искусственного интеллекта}}, обеспечивающей совместимость всех частных технологий Искусственного интеллекта. В основе перехода к новому стилю взаимодействия специалистов в области Искусственного интеллекта лежит переход от конкуренции к \myuline{синергетическому} взаимовыгодному взаимодействию, направленному на \myuline{конвергенцию} и глубокую интеграцию частных (локальных) результатов, что приведёт к преобразованию современного сообщества специалистов в области Искусственного интеллекта в \textbf{\textit{интеллектуальное сообщество}} (см.~\cite{Tarasov2002})

\section*{Актуальные проекты текущего этапа работ по развитию технологии интеллектуальных компьютерных систем нового поколения}
\label{concl_actual_projects_current_stage_work_development_technology}

Перечислим некоторые актуальные на данном этапе проекты прикладных интеллектуальных компьютерных систем нового поколения и средства их разработки:
\begin{textitemize}
	\item 
	Разработка формализованного \textbf{\textit{стандарта интеллектуальных компьютерных систем}}, представленного в составе базы интеллектуального портала научно-технических знаний по теории интеорперабельных интеллектуальных компьютерных систем и обеспечивающего \textbf{\textit{семантическую совместимость интероперабельных интеллектуальных компьютерных систем}}
	\item 
	Разработка формализованного \textbf{\textit{стандарта методов и средств поддержки жизненного цикла интероперабельных интеллектуальных компьютерных систем}}, представленного в составе базы знаний интеллектуальной \textbf{\textit{метасистемы автоматизации проектирования интероперабельных интеллектуальных компьютерных систем}}, а также поддержки всего их жизненного цикла
	\item 
	Разработка комплексной \textbf{\textit{библиотеки типовых}} (многократно используемых) \textbf{\textit{компонентов интероперабельных интеллектуальных компьютерных систем}}, обеспечивающей совместимость этих компонентов и полную автоматизацию их интеграции (соединения) в процессе сборочного (компонентного) проектирования семантически совместимых интероперабельных интеллектуальных компьютерных систем
	\item 
	В рамках Метасистемы OSTIS обеспечение \myuline{широкого} достпуа к текущему состоянию Стандарта OSTIS и разработка соответствующих средств семантической визуализации и навигации
	\item 
	В рамках Метасистемы OSTIS разработка средств автоматизации и управления процессом коллективного совершенствования (модернизации, реинжиниринга) Стандарта OSTIS
	\item
	Разработка \textbf{\textit{программной платформы}} для реализации интероперабельных интеллектуальных компьютерных систем
	\item 
	Разработка \textbf{\textit{ассоциативного семантического компьютера}} для реализации интероперабельных интеллектуальных компьютерных систем. Это \myuline{универсальный} компьютер, в котором осуществляется аппаратная реализация ассоциативной реконфигурируемой (структурно перестраиваемой) памяти, в которой переработка информациисводится к реконфигурации связей между элементами памяти.
	\item Разработка архитектуры интероперабельной интеллектуальной компьютерной системы, которая является \textbf{\textit{\myuline{персональным} интеллектуальным ассистентом}} (секретарем, референтом) для \myuline{каждого} пользователя, обеспечивающим максимально возможную автоматизацию процесса взаимодействия пользователя со всей Глобально экосистемой интероперабельных интеллектуальных компьютерных систем. База знаний каждого такого персонального интеллектуального ассистента включает в себя:
	\begin{textitemize}
		\item персональную информацию соответствующего пользователя, доступ к которой другим интеллектуальным компьютерным системам предоставляет персональный интеллектуальный ассистент этого пользователя, но обязательно ч разрешения этого пользователя и с сообщением пользователю соответствующих факторов риска. Персональная информация пользователя --- это его медицинские данные, биографические данные, личные фотографии, неопубликованная интеллектуальная собственность, формируемые или отправленные сообщения, адресуемые другим пользователям или различным сообществам.
		\item информацию и различных сообществах \textbf{\textit{Глобальной экосистемы интеллектуальных компьютерных систем нового поколения}}, членом которых является соответствующий (ассистируемый) пользователь, с указанием роли (должности, обязанности), которую выполняет указанный пользователь в рамках каждого такого сообщества. Указанных сообществ может быть много --- профессиональные сообщества, друзья, родственники, сообщества потребителей-производителей, административно-гражданское сообщество, банки, сообщество медицинского обслуживания и др.
		\item информацию о собственных планах и намерениях (как о стратегических, так и о ближайших, включая встречи, переговоры, совещания)
	\end{textitemize}
	Решатель задач персонального интеллектуального ассистента
	\begin{textitemize}
		\item обеспечивает максимально возможную автоматизацию различных видов профессиональной \myuline{индивидуаль-} \myuline{ной} деятельности соответствующего (обслуживаемого) пользователя 
		\item обеспечивает интеллектуальное посредничество (представление интересов) обслуживаемого пользователя в рамках всех сообществ, в состав которых он входит
	\end{textitemize}
	Пользовательский интерфейс персонального интеллектуального ассистента
	\begin{textitemize}
		\item предоставляет пользователю средства управления его \myuline{индивидуальной} деятельностью, осуществляемой \myuline{совместно} с соответствующим ему персональным интеллектуальным ассистентом;
		\item обеспечивает \myuline{унифицированный} характер взаимодействия пользователей в рамках различных сообществ, в которые он входит. Простейшим видом сообществ является разовый диалог двух пользователей.
	\end{textitemize}
	\item
	Разработка для каждого пользователя унифицированного \textbf{\textit{комплекса средств автоматизации индивидуального проектирования базы знаний}} собственного персонального интеллектуального ассистента. Напомним, что в состав персональной базы знаний каждого пользователя входят также и разрабатываемые пользователем фрагменты баз знаний для других интероперабельных интеллектуальных компьютерных систем, входящих в состав Экосистемы интеллектуальных компьютерных систем. В состав указанного комплекса средств автоматизации входят
	\begin{textitemize}
		\item редактор внутреннего представления знаний (редактор sc-текстов)
		\item редакторы различных внешних форм представления знаний (sc.g-текстов, sc.n-текстов)
		\item трансляторы с внутреннего представления знаний на различные внешние формы представления
		\item трансляторы с каждой формы внешнего представления знаний во внутреннее их представление
		\item средства синтаксического и семантического анализа проектируемого фрагмента базы знаний
		\item транслятор, обеспечивающий преобразование внутреннего представления знаний (в SC-кода) в естествен\-но-языковое представление в формате языка разметки LaTex, удостоверяющее требованиям, предъявляемым к оформлению статей в сборниках научно-технических материалов. Данный транслятор позволит сконцентрировать усилия разработчиков различных интеллектуальных компьютерных систем на формализации знаний, используемых в интеллектуальных компьютерных систем, и существенно снизить трудоемкость подготовки и оформления публикаций соответствующих научно-технических результатов.\\
		В перспективе различные научно-технические журналы должны преобразовываться в интеллектуальные порталы коллективно разрабатываемых научно-технических знаний в различных областях.
	\end{textitemize}
	\item
	Разработка в рамках персонального интеллектуального ассистента набора \textbf{\textit{средств индивидуального комплексного перманентного медицинского контроля и мониторинга}} соответствующего (обслуживаемого) пользователя
	\item Разработка в рамках каждого сообщества интероперабельных интеллектуальных компьютерных систем комплекса средств коллективной разработки общей базы знаний этого сообщества (базы знаний корпоративной системы указанного сообщества), в состав которого входят:
	\begin{textitemize}
		\item средства сборки (интеграции) разрабатываемой базы знаний из индивидуально-разрабатываемых её фрагментов
		\item средства согласования индивидуально разработанных фрагментов (персональны точек зрения), эпицентром чего является согласование используемых понятий
		\item средства взаимного рецензирования
		\item средства согласованной корректировки базы знаний
		\item средства формирования и согласования плана совершенствования разрабатываемой базы знаний
		\item средства контроля и управления процессом совершенствования разрабатываемой базы знаний
	\end{textitemize}
	\item Расширение набора средств автоматизации проектирования различных видов компонентов интероперабельных интеллектуальных компьютерных систем (ostis-систем) и различных классов интероперабельных интеллектуальных компьютерных систем
	\item Разработка формальной структуры глобального комплекса автоматизируемой человеческой деятельности и соответствующей этому архитектуры Экосистемы OSTIS. Обеспечить существенное расширение направлений применения Технологии OSTIS (медицина, промышленность, строительство, юриспруденция и так далее)
	\item Разработка на основе Технологии OSTIS комплекса средств и методик подготовки специалистов в области Искусственного интеллекта (на уровне обучения студентов, магистрантов и аспирантов)
	\item Разработка средств комплексной информации среднего образования с помощью семантически совместимых интероперабельных интеллектуальных компьютерных систем
	\item Разработка средств комплексной информатизации высшего технического образования с помощью семантически совместимых интероперабельных интеллектуальных компьютерных систем
	\item Обеспечение существенного расширения числа пользователей и разработчиков Технологии поддержки жизненного цикла интероперабельных интеллектуальных компьютерных систем в рамках соответствующего открытого международного проекта
\end{textitemize}