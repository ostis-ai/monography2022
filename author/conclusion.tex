\begin{partbacktext}
\part*{Заключение. Основные направления, проблемы и перспективы развития интеллектуальных компьютерных систем нового поколения и соответствующей им технологии}
\markboth{ЗАКЛЮЧЕНИЕ}{ЗАКЛЮЧЕНИЕ}
\label{part_conclusion}
\addcontentsline{toc}{part}{Заключение. Основные направления, проблемы и перспективы развития интеллектуальных компьютерных систем нового поколения и соответствующей им технологии}
\chapauthortoc{Голенков В.~В.\\Гулякина Н.~А.}

\vspace{-3\baselineskip}

\begin{SCn}
	\scnidtf{Заключение к Монографии OSTIS-2023}
	
	\bigskip
	
	\begin{scnrelfromlist}{автор}
		\scnitem{Голенков В.~В.}
		\scnitem{Гулякина Н.~А.}
	\end{scnrelfromlist}

	\bigskip

	\begin{scnrelfromlist}{подраздел}
		\scnitem{\nameref{concl_feature_current_state_work_field_AI}}
		\scnitem{\nameref{concl_tasks_current_stage_theory_and_technology_development}}
		\scnitem{\nameref{concl_methodological_problems_current_stage_work_field_AI}}
		\scnitem{\nameref{concl_transition_background}}
		\scnitem{\nameref{concl_ostis_history}}
		\scnitem{\nameref{concl_novelty}}
		\scnitem{\nameref{concl_actual_projects_current_stage_work_development_technology}}
	\end{scnrelfromlist}

	\bigskip
	
	\begin{scnrelfromlist}{ключевой знак}
		\scnitem{интеллектуальная компьютерная система нового поколения}
		\begin{scnindent}
			\scnidtf{компьютерная система нового поколения}
		\end{scnindent} 
		\scnitem{самообучаемая интеллектуальная компьютерная система}
		\scnitem{интероперабельная интеллектуальная компьютерная система}
		\begin{scnindent}
			\scnidtf{интероперабельная компьютерная система}
		\end{scnindent} 
		\scnitem{индивидуальный субъект}
		\scnitem{коллектив субъектов}
		\begin{scnindent}
			\scnidtf{коллекивный субъект}
			\scnsubset{многоагентная система}
		\end{scnindent} 
		\scnitem{иерархический субъект}
		\scnitem{социальная ответственность\scnsupergroupsign}
		\scnitem{интероперабельность\scnsupergroupsign}
		\scnitem{индивидуальная деятельность}
		\scnitem{коллективная деятельность}
		\scnitem{интеллект коллектива субъектов\scnsupergroupsign}
		\begin{scnindent}
			\scnidtf{эффективность (качество) коллектива субъектов}
		\end{scnindent} 
		\scnitem{стратегическая задача субъекта*}
		\begin{scnindent}
			\scnidtf{стратегическая цель данного субъекта*}
		\end{scnindent} 
		\scnitem{подзачада*}
		\scnitem{Общество}
		\begin{scnindent}
			\scnidtf{Человеческое сообщество}
		\end{scnindent} 
		\scnitem{Экосистема OSTIS}
		\begin{scnindent}
			\scnidtf{Вариант реализации Общества 5.0}
			\scnidtf{Общество + надстройка в виде Глобальной иерархической системы взаимосвязанных ostis-систем}
		\end{scnindent} 
	\end{scnrelfromlist}
	
	\bigskip
	
	\begin{scnrelfromlist}{библиографическая ссылка}
		\scnitem{\scncite{Golenkov2022a}}
		\scnitem{\scncite{Golenkov2022b}}
		\scnitem{\scncite{Tarasov2002}}
		\scnitem{\scncite{Golenkov2021}}
		\scnitem{\scncite{Golenkov2020}}
		\scnitem{\scncite{Golenkov2019}}		
	\end{scnrelfromlist}
	
\end{SCn}
	
\end{partbacktext}

\section*{Особенность текущего состояния работ в области Искусственного интеллекта --- переход к интеллектуальным компьютерным системам нового поколения}
\label{concl_feature_current_state_work_field_AI}

Эпицентром современного этапа автоматизации человеческой деятельности является низкий уровень автоматизации и большие накладные расходы
\begin{textitemize}
	\item на системную интеграцию различных компьютерных систем, то есть на создание сложных иерархических компьютерных комплексов;
	\item на модернизацию компьютерных систем в ходе их эксплуатации.
\end{textitemize}

Для автоматизации этих аспектов человеческой деятельности у современных компьютерных систем явно не хватает \textit{интеллекта} и \textit{самостоятельности}.

Необходимость перехода от современных \textit{компьютерных систем} (в том числе, и от современных \textit{интеллектуальных компьютерных систем}) к \textit{интеллектуальным компьютерных системам нового поколения} обусловлена необходимостью перехода к автоматизации всё более и более сложных видов и областей \textit{человеческой деятельности} требующих создания целых комплексов \textit{интеллектуальных компьютерных систем}, способных самостоятельно \myuline{эволюционировать} и эффективно \myuline{взаимодействовать} между собой в процессе \textit{\myuline{коллективного} решения сложных задач}.

Компьютерные системы, обладающие указанными способностями, и представляют собой \textit{компьютерные системы нового поколения}. Поскольку указанные \textit{компьютерные системы} не могут не иметь высокого уровня \textit{интеллекта}, следует их также называть \textit{интеллектуальными компьютерными системами нового поколения}. Высокий уровень \textit{интеллекта} компьютерным системам нового поколения необходим:
\begin{textitemize}
	\item для адекватной оценки собственной компетенции и компетенции своих партнеров;
	\item для обеспечения взаимопонимания, договороспособности и координации (согласованности) своих действий с действиями партнеров в ходе \textit{коллективного решения сложных задач} в условиях возможного возникновения непредсказуемых (нештатных) обстоятельств.
\end{textitemize}

Очевидно, что для создания и эксплуатации \textit{интеллектуальных компьютерных систем нового поколения} необходимо:
\begin{textitemize}
	\item разработать общую формальную теорию таких систем;
	\item разработать комплексную технологию проектирования и поддержки последующих этапов жизненного цикла этих систем;
	\item разработать общую формальную теорию всего многообразия видов и областей \textit{человеческой деятельности}, которые целесообразно автоматизировать.
\end{textitemize}

\begin{SCn}
\scnheader{интеллектуальная компьютерная система нового поколения}
\scneq{\textup{(}самообучаемая интеллектуальная компьютерная система $\cap$ интероперабельная интеллектуальная компьютерная система\textup{)}}

\scnheader{самообучаемая интеллектуальная компьютерная система}
\scnidtf{\textit{интеллектуальная компьютерная система}, имеющая высокие темпы \myuline{самостоятельно} реализуемой эволюции, следствием чего является существеное снижение трудоемкости (затрат, накладных расходов) на ее \myuline{модернизацию}}
\end{SCn}

\begin{SCn}
	\scnheader{самообучаемость интеллектуальной компьютерной системы}
	\begin{scnrelfromlist}{предполагает}
		\scnitem{\textup{способность мониторить состояние и динамику окружающей среды и корректировать свои действия при соответствующих изменениях окружающей среды (адаптивность);}} 
		\scnitem{\textup{способность анализировать и повышать качество собственной базы знаний (структуризация и анализ противоречий, информационных дыр, информационного мусора);}}
		\scnitem{\textup{способность извлекать знания из внешних источников информации;}}
		\scnitem{\textup{способность анализировать и повышать качество собственной деятельности (в том числе способность учиться на собственных ошибках);}}
		\scnitem{\textup{способность анализировать качество деятельности других субъектов и извлекать из этого пользу для себя (учиться на чужих ошибках).}}
	\end{scnrelfromlist}
\end{SCn}

\begin{SCn}
	\scnheader{высокий уровень самообучаемости интеллектуальной компьютерной системы}
	\begin{scnrelfromlist}{обеспечивается}
		\scnitem{\textup{высоким уровнем гибкости интеллектуальной компьютерной системы}}
		\scnitem{\textup{высоким уровнем стратифицированности интеллектуальной компьютерной системы}}
		\scnitem{\textup{высоким уровнем рефлексивности интеллектуальной компьютерной системы}}
		\scnitem{\textup{высоким уровнем познавательной активности}}
	\end{scnrelfromlist}

	\scnheader{интероперабельная интеллектуальная компьютерная система}
	\scnidtf{компьютерная система, способная к \myuline{самостоятельному} эффективному взаимодействию с другими системами}

	\scnheader{интероперабельность интеллектуальной компьютерной системы}
	\begin{scnrelfromlist}{предполагает}
		\scnitem{\textup{способность к \myuline{взаимопониманию} с другими системами и ее пользователями}}
		\begin{scnindent}
			\scnrelfrom{предполагает}{\textup{\myuline{семантическую совместимость} с взаимодействующими системами и пользователями}}
		\end{scnindent}
		\scnitem{\textup{\myuline{договороспособность}}}
		\scnitem{\textup{\myuline{способность к координации} своих действий с действиями партнеров}}
	\end{scnrelfromlist}

\end{SCn}



В основе предлагаемого нами подхода к построению \textit{интеллектуальных компьютерных систем нового поколения} лежат следующие принципы:
\begin{textitemize}
	\item \myuline{смысловое} представление знаний, хранимых в памяти \textit{интеллектуальных компьютерных систем нового поколения};
	\item \myuline{онтологическая} структуризация и систематизация хранимых в памяти знаний;
	\item \myuline{децентрализованная} ситуационная агенто-ориентированная организация \textit{процессов решения задач};
	\item \myuline{конвергенция} и глубокая (диффузная) интеграция различных моделей решения задач и, как следствие, гибридный характер \textit{решателей задач};
	\item \myuline{смысловая интеграция} входной информации, поступающей в индивидуальную интеллектуальную компьютерную систему извне по разным сенсорным каналам и на разных языках путём трансляции входной информации на общий универсальный \textit{язык внутреннего смыслового представления знаний}.
\end{textitemize}

\begin{SCn}
\scnheader{следует отличать*}
\begin{scnhaselementset}
	\scnitem{индивидуальная интеллектуальная компьютерная система нового поколения}
	\scnitem{коллективная \textit{интеллектуальная компьютерная система нового поколения}}
	\begin{scnindent}
		\begin{scnsubdividing}
			\scnitem{коллектив индивидуальных интеллектуальных компьютерных систем нового поколения}
			\scnitem{иерархический коллектив интеллектуальных компьютерных систем нового поколения}
			\begin{scnindent}
				\scnidtf{коллектив интеллектуальных компьютерных систем нового поколения, членами которого могут быть как коллективные, так и \textit{индивидуальные интеллектуальные компьютерные системы нового поколения}}
			\end{scnindent}
		\end{scnsubdividing}
	\end{scnindent} 
\end{scnhaselementset}
\end{SCn}

\begin{SCn}
	\scnheader{индивидуальная интеллектуальная компьютерная система нового поколения}
	\begin{scnrelfromlist}{особенности}
		\scnitem{\textup{\textit{индивидуальную интеллектуальную компьютерную систему нового поколения} невозможно декомпозировать на подсистемы, которые можно разрабатывать абсолютно независимо друг от друга и согласовывать только входам-выходам, реализуя принцип "черного ящика"{}.}}
		\scnitem{\textup{В \textit{индивидуальной интеллектуальной компьютерной системе нового поколения} необходима \myuline{конвергенция}, совместимость и "осмысленное"{} взаимодействие самых различных видов знаний и моделей решения задач. То есть \textit{индивидуальная интеллектуальная компьютерная система нового поколения} должна быть \textit{гибридной системой}.}}
	\end{scnrelfromlist}
\end{SCn}

\begin{SCn}
	\scnheader{Технология OSTIS}
	\scnidtf{Предложенная нами Технология разработки и сопровождения \textit{интеллектуальных компьютерных систем нового поколения}}
	\scnidtf{Open Semantic Technology for Intelligent Systems}
	\begin{scnrelfromlist}{предъявляемые требования}
		\scnitem{\textup{\myuline{комплексность} --- \textit{Технология OSTIS} обеспечивает совместимость всех частных технологий \textit{Искусственного интеллекта}; совместимость, самообучаемость и интероперабельность разрабатываемых \textit{интеллектуальных компьютерных систем}, а также поддержку не только проектирования \textit{интеллектуальных компьютерных систем}, но и всего их жизненного цикла}}
		\scnitem{\textup{\myuline{универсальность} --- \textit{Технология OSTIS} ориентирована на разработку и сопровождение \textit{интеллектуальных компьютерных систем нового поколения} \myuline{любого} назначения}}
		\scnitem{\textup{\myuline{самообучаемость} --- \textit{Технология OSTIS} обеспечивает перманентную эволюцию самой \textit{Технологии OSTIS} (самой себя) благодаря тому, что она реализована в виде \textit{интеллектуальной компьютерной системы нового поколения}, которая "знает"{} \textit{Технологию OSTIS} и "умеет"{} ее использовать}}
	\end{scnrelfromlist}

	\scnheader{ostis-система}
	\scnidtf{\textit{интеллектуальная компьютерная система}, построенная по \textit{Технологии OSTIS}}

	\scnheader{Экосистема OSTIS}
	\scnidtf{Основной продукт \textit{Технологии OSTIS}, представляющий собой Глобальную сеть \textit{ostis-систем}}
	\scniselement{иерархический коллектив интеллектуальных компьютерных систем нового поколения}

\end{SCn}

Основными компонентами \textit{Технологии OSTIS} являются:
\vspace{-0.4\baselineskip}
\begin{scnlist}
	\scnitem{\textbf{Стандарт OSTIS}}
	\begin{scnindent}
		\vspace{-0.5\baselineskip}
		\scnidtf{Стандарт интеллектуальных компьютерных систем нового поколения, а также методик, методов и средств поддержки их жизненного цикла}
		\scnidtf{\textit{Стандарт Технологии OSTIS}}
		\vspace{-0.3\baselineskip}
	\end{scnindent}
	\scnitem{\textbf{Метасистема OSTIS}}	
	\begin{scnindent}
		\vspace{-0.5\baselineskip}
		\scnidtf{Ядро Системы автоматизации поддержки жизненного цикла \textit{ostis-систем}}
		\vspace{-0.2\baselineskip}
	\end{scnindent}
	\scnitem{\textbf{Библиотека OSTIS}}	
	\begin{scnindent}
		\vspace{-0.5\baselineskip}		
		\scnidtf{Распределенная библиотека типовых (многократно используемых) компонентов \textit{ostis-систем}}
		\vspace{-0.2\baselineskip}		
	\end{scnindent}
\end{scnlist}	

\section*{Задачи текущего этапа разработки теории и технологии интеллектуальных компьютерных систем нового поколения}
\label{concl_tasks_current_stage_theory_and_technology_development}

Создание \textit{интеллектуальных компьютерных систем нового поколения} требует получения ответов на следующие вопросы:
\begin{textitemize}
	\item Какие требования предъявляются к \textit{интеллектуальным компьютерным системам}, обеспечивающим указанную выше \myuline{комплексную} автоматизацию \textit{человеческой деятельности};
	\item Почему современные интеллектуальные компьютерные системы указанным требованиям не удовлетворяют и, соответственно, почему необходим переход к принципиально новому поколению \textit{интеллектуальных компьютерных систем};
	\item Какие фундаментальные принципы должны лежать в основе \textit{интеллектуальных компьютерных систем нового поколения};
	\item Какие принципы должны лежать в основе \myuline{максимально автоматизируемой технологии} проектирования и поддержки всего жизненного цикла \textit{интеллектуальной компьютерных систем нового поколения};
	\item Какие принципы должны лежать в основе структуры и организации различных видов и областей \textit{человеческой деятельности} для обеспечения её комплексной и максимально возможной автоматизации с помощью \textit{интеллектуальных компьютерных систем нового поколения} (как известно, прежде, чем автоматизировать какую-либо \textit{человеческую деятельность}, необходимо привести её в порядок --- автоматизация беспорядка приводит к ещё большему беспорядку).
\end{textitemize}

%в другое место
%Теория человеческой деятельности в рамках Общества 5.0
%\begin{textitemize}
%	\item чёткая иерархичность (от индивидуальной деятельности людей и индивидуальных интеллектуальных компьютерных систем к различным уровням коллективов)
%	\item фрактальность !!!
%	\item стирание грани между деятельностью людей и интеллектуальной компьютерной системой
%	\item максимально возможная унификация
%	\item баланс унификации / многообразия
%	\item децентрализация
%\end{textitemize}

К числу текущих фундаментальных задач по созданию теории и технологии \textit{интеллектуальных компьютерных систем нового поколения} относятся:
\begin{textitemize}
	\item Разработка \myuline{теории} \textit{иерархических многоагентных систем}, агентами в которых являются индивидуальные или коллективные \textit{интерооперабельные интеллектуальные компьютерные системы}.
	\medskip
	\item \myuline{Унификация} и \myuline{стандартизация} различных моделей представления и обработки знаний. Эффект от данной унификации будет виден не сразу. Но, если этого происходить не будет, мы никогда не придем к эффективной \myuline{комплексной} автоматизации \textit{человеческой деятельности}. Эклектичное многообразие методов и средств автоматизации приводит не только к необоснованному \myuline{дублированию} разрабатываемых систем, но также и к повышению сложности их использования и сопровождения.
	\medskip
	\item \myuline{Конвергенция} и \myuline{интеграция} различных направлений \textit{Искусственного интеллекта}.
	
	Сейчас различные направления \textit{Искусственного интеллекта} имеют достаточно высокий уровень развития (signal processing, natural language processing, логические модели, искусственные нейронные сети, онтологические модели, многоагентные модели и многое другое). Интеграция всех этих направлений является пусть и достаточно трудоёмкой задачей, но задачей вполне решаемой, в основе решения которой лежит согласование смежных понятий.
	\medskip
	\item \myuline{Конвергенция} таких видов деятельности в области \textit{Искусственного интеллекта}, как:
	\begin{textitemize}
		\item подготовка специалистов в области \textit{Искусственного интеллекта};
		\item инженерная деятельность по разработке прикладных \textit{интеллектуальных компьютерных систем нового поколения};
		\item развитие \textit{технологии} проектирования и поддержки жизненного цикла \textit{интеллектуальных компьютерных систем нового поколения};
		\item научно-исследовательская деятельность в области \textit{Искусственного интеллекта}.
	\end{textitemize}
	\item Для развития Технологии \textit{интеллектуальных компьютерных систем нового поколения} необходима также \myuline{конвергенция} этой Технологии \myuline{со всеми} видами и областями \textit{человеческой деятельности}, которые не входят в состав деятельности в области \textit{Искусственного интеллекта}. Развитие Технологии \textit{интеллектуальных компьютерных систем нового поколения} носит ярко выраженный \myuline{междисциплинарный характер}. Это означает что \myuline{все знания}, накапливаемые \textit{человеческим обществом} в самых различных областях должны быть представлены в составе Глобальной \textit{базы знаний} Экосистемы \textit{интеллектуальных компьютерных систем нового поколения} (с помощью порталов научно-технических, административных и прочих знаний), должны быть чётко \myuline{стратифицированы} в виде иерархической системы \myuline{семантически совместимых} многократно используемых \textit{онтологий} и превращены в иерархическую систему семантически совместимых \textit{формальных} компонентов баз знаний \textit{интеллектуальных компьютерных систем} различного прикладного назначения.
	\medskip
	\item Обеспечение \myuline{семантической совместимости} \textit{интеллектуальных компьютерных систем нового поколения} не только на этапе их проектирования, но и на всех последующих этапах их жизненного цикла.
	\medskip
	\item Разработка модели \myuline{коллективного} поведения \textit{интеллектуальных компьютерных систем нового поколения}, то есть модели децентрализованного коллективного решения задач на уровне:
	\begin{textitemize}
		\item \textit{многоагентной системы}, агенты которой являются \myuline{внутренними агентами} \textit{индивидуальной интеллектуальной компьютерной системы}, взаимодействующими через общую память (через общую \textit{базу знаний}, хранимую в \myuline{одной} памяти);
		\item \textit{многоагентной системы}, агенты которой являются \textit{интероперабельными интеллектуальными компьютерными системами}, взаимодействующими через общую \textit{базу знаний}, хранимую в памяти \textit{корпоративной интеллектуальной компьютерной системы} или в памяти координатора деятельности \myuline{временного} коллектива \textit{интеллектуальных компьютерных систем}. 
	\end{textitemize}
%	\vspace{-1\baselineskip}
	В рамках теории \textit{коллективного решения задач} можно выделить следующие задачные ситуации:
	\begin{textitemize}
		\item задача, которая может быть решена той \textit{индивидуальной интеллектуальной компьютерной системой}, в которой эта задача инициирована;
		\item задача, соответствующая компетенции того \textit{\myuline{коллектива} интеллектуальных компьютерных систем}, в рамках которого эта задача инициирована;
		\item задача, выходящая за пределы компетенции того \textit{коллектива интеллектуальных компьютерных систем}, в рамках которого эта задача инициирована. Такая задача требует формирования \myuline{временного} коллектива, \myuline{координатором} (но не менеджером) которого становится та \textit{интеллектуальная компьютерная система}, в рамках которой указанная задача инициировалась. Для этого необходимо найти те \textit{интеллектуальные компьютерные системы}, которые в совокупности обеспечат необходимую компетенцию.
	\end{textitemize}
%	\vspace{-1\baselineskip}
	Отметим при этом, что каждая \textit{интероперабельная интеллектуальная компьютерная система} (как индивидуальная, так и коллективная) должна \myuline{знать} свою компетенцию для того, чтобы определить, сможет или не сможет она решить ту или иную заданную (возникшую) задачу. Это, в частности, необходимо для формирования \myuline{временных} коллективов \textit{интеллектуальных компьютерных систем}.
	\medskip
	\item Разработка принципов, лежащих в основе мощной Библиотеки многократно используемых и совместимых компонентов \textit{интеллектуальных компьютерных систем нового поколения}, которая обеспечивает \myuline{полную} автоматизацию интеграции этих компонентов в процессе сборки проектируемых систем.
	\item Разработка методик и средств перманентного расширения Библиотеки многократно используемых компонентов \textit{интеллектуальных компьютерных систем нового поколения} в самых различных областях \textit{человеческой деятельности}:
	\begin{textitemize}
		\item Научно-техническая деятельность в любой области должна сводиться к развитию \textit{баз знаний} различных \textit{интеллектуальных порталов научно-технических знаний}. При этом \textit{база знаний} каждого такого портала должна декомпозироваться на фрагменты, включаемые в состав Библиотеки многократно используемых компонентов \textit{баз знаний} \textit{интеллектуальных компьютерных систем нового поколения}, которые могут иерархически входить друг в друга. Для этого указанные компоненты должны соответствующим образом специфицироваться.
		\item Разработчики любой \textit{интеллектуальной компьютерной системы} должны \myuline{декомпозировать} разработанную систему на множество компонентов, включаемых в состав Библиотеки компонентов\textit{ интеллектуальных компьютерных систем нового поколения} --- так, чтобы разработка любой аналогичной системы свелась к сборке компонентов из этой Библиотеки.
		\item \myuline{Все}(!) разработчики должны заботиться о \myuline{расширении} Библиотеки многократно используемых (типовых) компонентов \textit{интеллектуальных компьютерных систем нового поколения}, что приведёт к существенному снижению трудоёмкости разработки новых \textit{интеллектуальных компьютерных систем нового поколения} в рамках Экосистемы таких систем. При этом авторство компонентов  указанной Библиотеки должно \myuline{поощряться}, что является фундаментальной основной развития рынка знаний, экономики знаний. 
		
		Если грамотно развивать и использовать Технологию \textit{интеллектуальных компьютерных систем нового поколения}, то разработка любой новой \textit{интеллектуальной компьютерной системы} будет в основном сводиться к её автоматической сборке из указываемых разработчиком компонентов этой системы. Некоторые компоненты разрабатываемой \textit{интеллектуальной компьютерной системы} могут входить в текущее состояние Библиотеки компонентов \textit{интеллектуальных компьютерных систем нового поколения}, а некоторые из них будут требовать дополнительной разработки. Но при этом каждый такой новый компонент чаще всего является результатом модификации существующих компонентов из  указанной Библиотеки и \myuline{должен быть} специфицирован и включен в эту Библиотеку. Таким образом разработчик прикладной \textit{интеллектуальной компьютерной системы} должен разработать не только эту систему, но и внести вклад в развитие Библиотеки компонентов \textit{интеллектуальных компьютерных систем нового поколения}, в результате которого разрабатываемая им следующая \textit{интеллектуальная компьютерная система} может быть собрана без дополнительно разрабатываемых компонентов, а только из компонентов Библиотеки компонентов. Если все разработчики прикладных систем будут так действовать, то темпы повышения уровня автоматизации \textit{человеческой деятельности} будут существенно возрастать.
	\end{textitemize}
\end{textitemize}

\section*{Методологические проблемы текущего этапа работ в области Искусственного интеллекта}
\label{concl_methodological_problems_current_stage_work_field_AI}

\begin{SCn}
	\scntext{эпиграф}
	{Самые сложные проблемы --- это те, которые мы не осознаем и, особенно, те, причиной которых является наше несовершенство.
	}
	
	\begin{scnrelfromlist}{подраздел}
		\scnitem{Социальная ответственность специалистов в области Искусственного интеллекта}
		\scnitem{Глобальная цель деятельности в области Искусственного интеллекта}
		\scnitem{Общие требования, предъявляемые к специалистам в области Искусственного интеллекта}
		\scnitem{Требования, предъявляемые к фундаментальной подготовке специалистов в области Искусственного интеллекта}
		\scnitem{Проблемы текущего этапа разработки теории и технологии интеллектуальных компьютерных систем нового поколения}
	\end{scnrelfromlist}
\end{SCn}

\subsection*{Социальная ответственность специалистов в области Искусственного интеллекта}
Современный этап развития теории и практики \textit{Искусственного интеллекта} обнажает целый спектр проблем, препятствующих этому развитию. Дальнейшее развитие технологий \textit{Искусственного интеллекта}
\begin{textitemize}
	\item с одной стороны, может и достаточно быстро осуществить переход современного общества на принципиально новый уровень его эволюции, обеспечивающий \myuline{комплексную} автоматизацию всех подлежащих автоматизации видов и областей \textit{человеческой деятельности}, а также обеспечивающий максимально возможный комфорт и максимально возможное раскрытие творческого потенциала \myuline{каждого} человека;
	\item с другой стороны, может достаточно долго и весьма убедительно для неграмотного обывателя \myuline{имитировать} указанный прогресс автоматизации \textit{человеческой деятельности} --- любая даже весьма достойная цель может быть загублена имитацией её достижения;
	\item с третьей стороны, может достаточно быстро привести \textit{человеческое общество} к деградации и самоуничтожению.
\end{textitemize}

Таким образом, на современном этапе развития технологий \textit{Искусственного интеллекта}, \myuline{уровень социальной} \myuline{ответственности} специалистов в области \textit{Искусственного интеллекта} является определяющим фактором развития \textit{человеческого общества}. Опасность для \textit{человеческого общества} исходит не от \textit{интеллектуальных компьютерных систем}, а от мотивации специалистов, которые разрабатывают эти системы. Очевидно, что создание \textit{интеллектуальных компьютерных систем}, предназначенных для \myuline{осознанного} нанесения любого ущерба \textit{человеческому обществу}, и требующих создания соответствующих интеллектуальных средств обеспечения безопасности, является короткой дорогой к самоуничтожению.

Усилия специалистов в области \textit{Искусственного интеллекта} должны быть направлены на существенное повышение уровня интеллекта \textit{человеческого общества} в целом, основой чего является \myuline{комплексная} автоматизация \myuline{всех} тех видом и областей \textit{человеческой деятельности}, которые принципиально имеет смысл автоматизировать.

\subsection*{Глобальная цель деятельности в области Искусственного интеллекта}
Почему современный этап деятельности в области \textit{Искусственного интеллекта} требует формулировки \myuline{глобальной} \myuline{цели} этой деятельности и перманентного её уточнения.

Современное состояние \textit{Искусственного интеллекта} можно охарактеризовать как глубокий методологический кризис, обусловленный:
\begin{textitemize}
	\item тем, что научные результаты в этой области вышли из научных лабораторий и стали оказывать реальное практическое воздействие;
	\item отсутствием понимания того, что получение серьезных научных результатов в той или иной области и создание \myuline{технологий}, обеспечивающих \myuline{эффективное} практическое использование этих результатов --- это соизмеримые по значимости и сложности задачи. Особенно это касается \textit{Искусственного интеллекта}.
\end{textitemize}

Последнее обстоятельство приводит к неоправданной эйфории, иллюзии благоплучия и к бурно расцветающей эклектике, которая абсолютно игнорирует даже казалось бы очевидные законы общей теории систем.

К сожалению, локальное внедрение результатов научных исследований в области \textit{Искусственного интеллекта}, локальная автоматизация бизнес-процессов какой-либо организации без учёта системной организации всего комплекса методов и средств автоматизации различных видов и областей человеческой деятельности приводит к неоправданному дублированию результатов. 

Если в ближайшее время не произойдёт осознания глобальной (стратегической) цели работ в области \textit{Искусственного интеллекта}, то деятельность в этой области в целом будет осуществляться в стиле "лебедя, рака и щуки"{}. Трата усилий не приведёт к целостному практически значимому результату. "Вектора"{} конкретных направлений этой деятельности, "вектора"{} наших усилий не будут иметь одинаковую направленность, что существенно снизит общую  производительность всей этой деятельности и качество общего (суммарного) результата.

Какова же должна быть \textit{стратегическая задача} (сверхзадача), которую должны решить специалисты в области \textit{Искусственного интеллекта}. Очевидно, что такой сверхзадачей является переход всего комплекса \textit{человеческой деятельности} на принципиально новый уровень максимально возможной его автоматизации, в рамках которого принципиально неавтоматизируемой частью человеческой деятельности остаётся \textit{творческая} деятельность, в частности, научно-исследовательская деятельность, преподавательская и воспитательная деятельность, перманентное повышение уровня комплексной автоматизации \textit{человеческой деятельности}. Основная цель \myuline{комплексной} автоматизации \textit{человеческой деятельности} заключается не только в том, чтобы автоматизировать то, что \myuline{можно эффективно} автоматизировать с помощью методов \textit{Искусственного интеллекта}, а в том, чтобы автоматизировать \myuline{все}(!) "узкие места"{} \textit{человеческой деятельности}, которые определяют общую её производительность в различных областях.

Таким образом, в настоящее время технологии \textit{Искусственного интеллекта} находятся на пороге перехода к принципиально новому уровню развития --- на пороге перехода от решения частных (локальных) задач к решению глобальной задачи комплексной автоматизации всех видов и областей \textit{человеческой деятельности}, что требует автоматизации решения не только частных актуальных и важных задач, но и автоматизации решения задач всё более и более высокого уровня, для которых автоматизируемые сейчас задачи становятся подзадачами. Другими словами, при автоматизации решения комплексных задач (надзадач) автоматизация фокусируется на разработке методов и средств \myuline{взаимодействия между средствами решения локальных задач} (частных задач).

Перенос акцента на автоматизацию решения не просто \textit{интеллектуальных задач}, а на автоматизацию решения \myuline{комплексных} задач, подзадачами которых являются \myuline{разнообразные} \textit{интеллектуальные задачи}, не только переводит технологии \textit{Искусственного интеллекта} на принципиально новый уровень, но и окажет существенное влияние \myuline{на все стороны \textit{человеческой деятельности}}:
\begin{textitemize}
	\item  научно-исследовательские и научно-технические работы должны приобрести конвергентный взаимообогащающий характер;
	\item основой образования должна стать междисциплинарность;
	\item основой глобальной автоматизации \textit{человеческой деятельности} должна стать общая комплексная формальная и перманентно совершенствуемая теория \textit{человеческой деятельности}, в основе которой должна лежать междисциплинарная конвергентная методика, направленная на преодоление эклектичного подхода.
\end{textitemize}

Следовательно, основной целью комплексной автоматизации всевозможных видов и областей человеческий деятельности с помощью \textit{интероперабельных интеллектуальных компьютерных систем} является существенное повышение \textit{уровня интеллекта}  человеческого общества в целом.

Современное \textit{человеческое общество} --- это сложнейшая распределённая многоагентная \textit{кибернетическая система}, развитие которой осуществляется, к сожалению, с нарушением многих законов Кибернетики и, в частности, с нарушением критериев, определяющих уровень \textit{интеллекта} иерархических многоагентных систем. Уровень \textit{интеллекта} таких систем определяется целым рядом казалось бы очевидных факторов:
\begin{textitemize}
	\item тем, каков объём и \textit{качество знаний}, накопленных \textit{многоагентной системой} и доступных всем агентам (субъектам), входящим в эту систему
	\begin{textitemize}
		\item насколько этих \textit{знаний} достаточно для организации управления деятельностью этой системы;
		\item насколько эти знания корректны (не противоречивы) и адекватны;
		\item насколько велика конвергентность, компактность и чистота этих \textit{знаний} (здесь учитывается наличие информационного мусора, информационного дублирования);
		\item насколько хорошо структурированы (систематизированы) накапливаемые знания;
	\end{textitemize}
	\item тем, как осуществляется доступ каждого агента \textit{многоагентной системы} к знаниям, хранимым в общей памяти всей \textit{многоагентной системы};
	\medskip
	\item тем, как эти \textit{знания} накапливаются и эволюционируют, как многоагентная система самообучается
	\begin{textitemize}
		\item как \textit{многоагентная система} учится на собственных ошибках,
		\item как \textit{многоагентная система} повышает качество своих \textit{знаний};
	\end{textitemize}
	\item тем, как \textit{многоагентная система} в целом и каждый агент в частности используют накопленные в общей памяти \textit{знания} для решения различных \textit{задач}.
\end{textitemize}

Таким образом, если рассматривать современное \textit{человеческое общество} с позиций теории \textit{многоагентных систем}, являющихся сообществами \textit{интеллектуальных систем} (не только искусственных, но и естественных интеллектуальных систем), то очевидно, что следующий этап его эволюции требует:
\begin{textitemize}
	\item автоматизации накопления, анализа и перманентного повышения \textbf{\textit{качества знаний}}, накапливаемых человечеством;
	\item автоматизации эффективного использования накопленных человечеством знаний при решении задач самого различного уровня, требующих формирования различных кратковременных или долговременных \textbf{\textit{сообществ}} \textbf{\textit{из людей и интеллектуальных компьютерных систем}}. Каждое такое сообщество предназначается либо для решения какой-либо одной конкретной задачи, либо для решений некоторого множества задач в некоторой области;
	\item повышения уровня \textbf{\textit{конвергенции}} знаний, методов, действий, а также создаваемых новых технических систем;
	\item повышения уровня \textbf{\textit{интероперабельности}} как для интеллектуальных компьютерных систем, так и для людей.
\end{textitemize}

\subsection*{Общие требования, предъявляемые к специалистам в области Искусственного интеллекта}

\begin{SCn}
	\begin{scnrelfromlist}{эпиграф}
		\scnitem{\textup{Требования, предъявляемые к специалистам в области \textit{Искусственного интеллекта} на \myuline{новом} этапе развития этой области, являются отражением требований, предъявляемых к \textit{интеллектуальным компьютерным системам \myuline{нового} поколения} и соответствующим технологиям}}
		\scnitem{\textup{Уровень \textit{интеллекта} (в том числе коллективного интеллекта) разработчиков \textit{интеллектуальной компьютерной системы} не может быть ниже уровня \textit{интеллекта} создаваемых \textit{интеллектуальных компьютерных систем}}}
		\scnitem{\textup{Уровень \textit{интеллекта} коллектива агентов далеко не всегда выше уровня \textit{интеллекта} входящих в него агентов}}
		\scnitem{\textup{Разработка \textit{интероперабельных интеллектуальных компьютерных систем} может быть только коллективной}}
		\scnitem{\textup{Коллектив \myuline{не}интероперабельных разработчиков не может создавать \textit{интероперабельные интеллектуальные компьютерные системы}}}
	\end{scnrelfromlist}
\end{SCn}

Высокий уровень \textit{социальной ответственности}, требуемый от \textit{специалистов в области Искусственного интеллекта}, предъявляет к ним целый ряд очевидных, но, к сожалению, часто не учитываемых \myuline{общих требований}, необходимых для качественного участия в сложных коллективных социально значимых проектах. К таким общим требованиям относятся:

\begin{textitemize}
	\item  высокий уровень \textbf{\textit{мотивации}} к участию в \myuline{перманентной} эволюции целостного технологического комплекса, обеспечивающего разработку эффективных \textit{интероперабельных интеллектуальных компьютерных систем}. \myuline{Комплексная} и \myuline{высококачественная} технология разработки и сопровождения \textit{интероперабельных интеллектуальных компьютерных систем} должна рассматриваться как \myuline{ключевой} продукт коллективной деятельности в области \textit{Искусственного интеллекта}. Указанная мотивация предполагает соответствующую целеустремлённость, отсутствие эгоизма, высокомерия, индивидуализма, изоляционизма, паразитизма;
	\medskip
	\item высокий уровень \textbf{\textit{созидательной активности}}, пассионарности, смелости;
	\medskip
	\item высокий уровень \textbf{\textit{рефлексии}} --- способности анализировать собственные цели и действия и исправлять собственные ошибки, а также анализировать цели, действия и ошибки, совершаемые коллективом, членом которого специалист является. Одно дело искренне признавать логичность и целесообразность соблюдения тех или иных правил (принципов, требований) и совсем другое дело уметь видеть и исправлять \myuline{собственные} нарушения этих правил. Без такой рефлексии прогресс коллективного творчества невозможен. Знать то, как \myuline{надо} делать и реально следовать этому --- не одно и то же.
	\medskip
	\item высокий уровень \textbf{\textit{\myuline{собственной} интероперабельности}}:
	\begin{textitemize}
		\item способности к \textit{взаимопониманию} и обеспечению \textit{семантической совместимости}, требующей перманентного мониторинга текущего состояния и эволюции технологического комплекса;
		\item \textit{договороспособности} --- способности оперативно согласовывать свои цели и планы, детонационную семантику понятий и терминов, а также децентрализовано распределять подзадачи коллективно решаемой задачи;
		\item \textit{способности координировать} и синхронизировать свои действия с коллегами в условиях возможного возникновения непредсказуемых обстоятельств.
	\end{textitemize}
	
	Без высокого уровня \textit{интероперабельности} разработчиков, невозможно обеспечить:
	\begin{textitemize}
		\item \textbf{\textit{конвергенцию}}, унификацию, стандартизацию \textit{интероперабельных интеллектуальных компьютерных систем};
		\item формирование мощной \textbf{\textit{Библиотеки типовых компонентов интеллектуальных компьютерных систем нового поколения}};
		\item существенное снижение трудоёмкости и повышение уровня автоматизации разработки и сопровождения \textit{интеллектуальных компьютерных систем  нового поколения};
		\item построение общей теории \textbf{\textit{Экосистемы интеллектуальных компьютерных систем нового поколения}} и, соответственно, общей теории \textit{человеческой деятельности}.
	\end{textitemize}

		Таким образом, для создания \textit{интероперабельных интеллектуальных компьютерных систем} необходимо. чтобы сами их создатели имели высокий уровень \textit{интероперабельности}. Проблема обеспечить это является основным вызовом, который адресован специалистам в области \textit{Искусственного интеллекта} на текущем этапе развития этой области.\\
		Основной причиной, препятствующий формированию необходимого уровня \textit{интероперабельности} у специалистов в области \textit{Искусственного интеллекта}, является \myuline{конкурентный стиль взаимоотношений} между специалистами. Этот стиль взаимоотношений является широко распространённым способом стимулировать активность сотрудников. Но это не единственный способ стимулировать творческую активность в решении стратегически важных задач, каковой, в частности, является задача эффективной комплексной автоматизации всех видов и областей \textit{человеческой деятельности} с помощью \textit{интероперабельных интеллектуальных компьютерных систем}. Более того, конкуренция провоцирует эгоизм и игнорирование интересов иных субъектов (в том числе, и интересов того коллектива, членом которого субъект является). Таким образом конкуренция явно противоречит принципам \textit{интероперабельности} и, соответственно, принципам организации \textit{интеллектуальных сообществ}, интеллектуальных творческих коллективов и организаций.\\
		От конкурентного стиля взаимоотношений необходимо переходить к \myuline{взаимовыгодному} взаимодействию между субъектами всех уровней иерархии. В этом заключается основная суть \textit{интероперабельности} и перехода к \textit{интеллектуальным коллективам} и интеллектуальному обществу.
	\end{textitemize}

Заметим, что перечисленные общие требования, предъявляемые к специалистам в области Искусственного интеллекта на современном этапе развития \textit{технологий Искусственного интеллекта}, должны предъявляться не только к ним, но и ко всем людям, готовым способствовать технологическому прогрессу. Просто на данном этапе основная ответственность за это лежит именно на специалистах в области \textit{Искусственного интеллекта}.

\subsection*{Требования, предъявляемые к фундаментальной подготовке специалистов в области Искусственного интеллекта}
Необходимость существенного повышения уровня практической значимости и эффективности работ в области \textit{Искуссвтеного интеллекта}, требующего перехода к \textit{интеллектуальным компьютерным системам нового поколения} и к принципиально новому технологическому комплексу, предъявляет к специалистам в области \textit{Искусственного интеллекта} не только общие требования, необходимые для эффективного участия в сложных коллективных социально значимых проектах, но так же и высокие требования к их \myuline{фундаментальной} профессиональной подготовке:
\begin{textitemize}
	\item высокому уровню системной культуры, позволяющей "видеть"{} иерархию сложных систем, связи между различными уровнями и иерархии, разницу между тактическими и стратегическими задачами;
	\item высокому уровню математической культуры, культуры формализации;
	\item высокому уровню технологической культуры и технологической дисциплины;
	\item высокому уровню самообучаемости в условиях быстрого изменения технологической инфраструктуры
\end{textitemize}

\subsection*{Проблемы текущего этапа разработки теории и технологии интеллектуальных компьютерных систем нового поколения}
Перечислим основные методологические проблемы текущего этапа работ в области Искусственного интеллекта, которые препятствуют решению рассматриваемых выше фундаментальных задач:
\begin{textitemize}
	\item 
	Недостаточно высокий уровень осознания специалистами в области \textit{Искусственного интеллекта} своей социальной ответственности.
	\medskip
	\item 
	Отсутствие согласованного осознания глобальной цели работ в области \textit{Искусственного интеллекта}, которая заключается в поэтапном повышении \textbf{\textit{уровня интеллекта}} \textit{человеческого общества} путем \myuline{комплексной} автоматизации всех аспектов его деятельности с помощью сети взаимодействующих между собой \textit{интеллектуальных компьютерных систем}.
	\medskip
	\item 
	Недостаточно высокий уровень \textit{интероперабельности} специалистов в области \textit{Искусственного интеллекта} и преобладание конкурентного стиля взаимоотношений. Следствием этого является недостаточное количество мотивированных специалистов в области \textit{Искусственного интеллекта}, способных к эффективному \myuline{творческому} взаимодействию. Для того, чтобы они появились в достаточном количестве, хорошей системы их \myuline{профессиональной} подготовки недостаточно. Следует также отметить, что хорошие человеческие отношения, психологическая атмосфера и Team Building в команде разработчиков, к которому серьёзно относятся многие компании, является необходимым, но далеко не достаточным условием результативности коллективной разработки сложных компьютерных систем (особенно это касается \textit{интеллектуальных компьютерных систем нового поколения}).
	\medskip
	\item 
	Недостаточно высокий уровень \myuline{комплексной фундаментальной} подготовки специалистов в области \textit{Искусственного интеллекта}.
	\medskip
	\item 
	Ярко выраженный \myuline{междисциплинарный} характер \textit{Искусственного интеллекта} как области человеческой деятельности, требующий от специалистов умения работать на стыках наук.
	\medskip
	\item 
	Отсутствие осознания необходимости глубокой \textit{конвергенции} между различными направлениями \textit{Искусственного интеллекта} и формализации всего комплекса знаний в области \textit{Искусственного интеллекта} для их использования в \textit{базах знаний} интеллектуальных компьютерных систем (прежде всего, инструментальных \textit{интеллектуальных компьютерных систем}, входящих в состав технологического комплекса разработки и сопровождения \textit{интеллектуальных компьютерных систем} различного назначения).
	\medskip
	\item 
	Высокий уровень сложности комплексной формализации \myuline{всех} накапливаемых человечеством знаний (прежде всего, в области математики и общей теории систем) и их \myuline{конвергенции} с комплексом знаний, накапливаемых и формализуемых в области \textit{Искусственного интеллекта}. Это необходимо для \myuline{непосредственного} использования накапливаемых человечеством знаний в \textit{интеллектуальных компьютерных системах} различного назначения.
	\medskip
	\item 
	Отсутствие осознания необходимости глубокой \textit{конвергенции} и согласованности между
	\begin{textitemize}
		\item научно-исследовательской деятельностью в области \textit{Искусственного интеллекта};
		\item деятельностью, направленной на развитие частных технологий \textit{Искусственного интеллекта}, а также комплексной технологии проектирования и поддержки жизненного цикла \textit{интеллектуальных компьютерных систем};
		\item инженерной деятельностью, направленной на разработку конкретных \textit{интеллектуальных компьютерных систем} различного назначения;
		\item образовательной деятельностью, направленной на подготовку специалистов в области \textit{Искусственного интеллекта}.
	\end{textitemize}
	\item 
	Проблема обеспечения \textbf{\textit{семантической совместимости}} \textit{интеллектуальных компьютерных систем нового поколения} не только на этапе их проектирования, но и на протяжении всего их жизненного цикла в условиях перманентной эволюции самих \textit{интеллектуальных компьютерных систем} в ходе их эксплуатации, а также перманентной эволюции комплексной технологии их разработки.
\end{textitemize}

Основная часть указанных проблем заключается в необходимости перехода к принципиально новому \myuline{стилю и орга-} \myuline{низации} взаимодействия специалистов в области \textit{Искусственного интеллекта}, без чего невозможен переход от частных теорий \textit{Искусственного интеллекта} к \textbf{\textit{Общей теории интеллектуальных компьютерных систем}}, обеспечивающей \myuline{совместимость} всех частных теорий \textit{Искусственного интеллекта}, а также переход от частных технологий \textit{Искусственного интеллекта} к \textbf{\textit{Комплексной технологии Искусственного интеллекта}}, обеспечивающей совместимость всех частных технологий \textit{Искусственного интеллекта}. В основе перехода к новому стилю взаимодействия специалистов в области \textit{Искусственного интеллекта} лежит переход от конкуренции к \myuline{синергетическому} взаимовыгодному взаимодействию, направленному на \myuline{конвергенцию} и глубокую интеграцию частных (локальных) результатов, что приведёт к преобразованию современного сообщества специалистов в области \textit{Искусственного интеллекта} в \textbf{\textit{интеллектуальное сообщество}} (см.~\textit{\scncite{Tarasov2002}}).

\section*{Предпосылки перехода к интеллектуальным компьютерным системам нового поколения}
\label{concl_transition_background}

\begin{textitemize}
\item Активно расширяющееся многообразие информационных ресурсов и сервисов, эффективность использования которых имеет низкий уровень из-за отсутствия их систематизации и совместимости 
\item Появление формальных онтологий как средства обеспечения семантической совместимости накапливаемых человечеством информационных ресурсов, Semantic Web
\item Активное развитие теории многоагентных систем, их самоорганизации, эмерджентности, синергии, теории интеллектуальных сообществ и организаций
\item Развитие теории децентрализованного ситуационного управления ("оркестр играет без дирижера"{})
\item Умный дом, умная больница, умный город
\item Industry 4.0, University 4.0
\item Появление работ, направленных на уточнение кибернетических принципов, лежащих в основе Общества 5.0
\end{textitemize}

\section*{Предыстория и история разработки Технологии OSTIS}
\label{concl_ostis_history}

\begin{textitemize}
\item 1981 -- Японский и американский проекты ЭВМ пятого поколения
\item 1984 -- Защита В.В. Голенковым кандидатской диссертации ``Структурная организация и переработка информации в электронных математических машинах, управляемых потоком сложноструктурированных данных''
\item Совет Д.А. Поспелова: <<Прежде, чем проектировать компьютеры, ориентированные на реализацию интеллектуальных компьютерных систем, необходимо:

	\begin{textitemize}
		\item разработать базовое математическое  и программное обеспечение таких компьютеров;
		\item разработать основы технологии проектирования интеллектуальных компьютерных систем, реализуемых на базе указанных компьютеров;
		\item разработать на современных компьютерах программную модель (эмулятор) создаваемого компьютера нового поколения;
		\item реализовать несколько конкретных \textit{интеллектуальных компьютерных систем} на базе указанной выше технологии и указанной программной модели будущего компьютера.
	\end{textitemize}
\vspace{-0.8\baselineskip}
Если всего этого не сделать, то разработанный компьютер нового поколения будет талантливо сделанным "железом"{}, которое непонятно как использовать и которое, следовательно, быстро морально устареет. Именно поэтому все проекты ЭВМ пятого поколения были обречены.>>
\vspace{0.5\baselineskip}
\item 1992 -- Прототип семантического компьютера на транспьютерах
\item 1995 -- Открытие в БГУИР учебной специальности ``Искусственный интеллект'' и создание соответствующей выпускающей кафедры
\item 1996 -- Защита В.В. Голенковым докторской диссертации ``Графодинамические модели и методы параллельной асинхронной переработки информации в интеллектуальных системах''
\item 2010 -- Создание открытого \textit{Проекта OSTIS}, направленного на создание открытой комплексной технологии проектирования \textit{интеллектуальных компьютерных систем}, реализация которых ориентируется на использование \textit{компьютеров нового поколения}
\item 2011 -- Начало проведения \myuline{ежегодных} конференций OSTIS, направленных на развитие открытого \textit{Проекта OSTIS}
\item 2019 -- На базе учреждения образования ``Белорусский государственный университет информатики и радиоэлектроники'' создано Учебно-научное объединение по направлению ``Искусственный интеллект'';
\item 2021 -- Издание \myuline{прототипа} \textit{Стандарта Технологии OSTIS}, представленного в виде \myuline{формализованного} текста, являющегося исходным текстом базы знаний Метасистемы поддержки проектирования интеллектуальных компьютерных систем, разрабатываемых по \textit{Технологии OSTIS};
\item 2023 -- Издание коллективной монографии по \textit{Технологии OSTIS}, которая рассматривается как основа дальнейшего развития и официального признания формализованного \textit{Стандарта Технологии OSTIS} и существенного расширения соответствующего авторского коллектива
\end{textitemize}

Резюмируя наш опыт работ в области \textit{Искусственного интеллекта}, можно сказать следующее:

\begin{textitemize}
	\item
	Требования, предъявляемые к \textit{интеллектуальным компьютерным системам} следующего поколения (высокий уровень \textit{самообучаемости}, \textit{интероперабельности}, \textit{самостоятельности}, \textit{универсальности}), предполагает создание \textit{принципиально новой комплексной} технологии, которая интегрирует, обеспечивает совместимость всего многообразия существующих \textit{частных технологий Искусственного интеллекта} и которая поддерживает все этапы жизненного цикла разрабатываемых \textit{интеллектуальных компьютерных систем};
	\item
	Сложность реализации \textit{интеллектуальных компьютерных систем нового поколения} (из-за несоответствия базовых принципов обработки информации в таких системах и принципов машины фон-Неймана, лежащих в основе современных компьютеров) требует создания компьютеров, специально ориентированных на реализацию \textit{интеллектуальных компьютерных систем нового поколения}. Но создавать указанные компьютеры нового поколения необходимо на основе (а, точнее, в рамках) указанной выше комплексной технологии проектирования и поддержки последующих этапов жизненного цикла интеллектуальных \textit{компьютерных систем нового поколения}.
	\item
	Эпицентром создания и последующей эволюции указанной комплексной технологии для \textit{интеллектуальных компьютерных систем нового поколения} является:
	\begin{textitemize}
		\item подготовка нового поколения специалистов в области \textit{Искусственного интеллекта}, которые изначально ориентированы на конвергенцию, на обеспечение совместимости своих результатов с результатами своих коллег и на спецификацию своих результатов в рамках Библиотеки типовых (многократно используемых) компонентов;
		\item перманентное развитие \textit{Стандарта Технологии OSTIS}, представленного в виде \myuline{формализованного} текста \textit{базы знаний} Метасистемы поддержки проектирования \textit{интеллектуальных компьютерных систем}, разрабатываемых по \textit{Технологии OSTIS}.
	\end{textitemize}
\end{textitemize}

\section*{Особенности, достоинства и новизна Технологии OSTIS}
\label{concl_novelty}
Новизна \textit{Технологии OSTIS} прежде всего заключается:
\begin{textitemize}
	\item в требованиях, предъявляемых к системам, создаваемым и сопровождаемым с помощью этой Технологии (к \textit{интеллектуальным компьютерным системам нового поколения}) -- гибридность, интероперабельность, самообучаемость.
	\item в требованиях, предъявляемых к самой \textit{Технологии OSTIS} (к используемым ею методикам, автоматизируемым методам и средствам) -- комплексность технологии, её универсальность и самообучаемость
\end{textitemize}

Дополнительные факторы новизны \textit{Технологии OSTIS} заключаются: 
\begin{textitemize}
	\item в том, что интенсивная эволюция самой \textit{Технологии OSTIS} (переход на новые её версии) не приводит к моральному старению уже эксплуатируемых \textit{интеллектуальных компьютерных систем} (\textit{ostis-систем}), поскольку в процессе эксплуатации этих систем возможна автоматическая их модификация (модернизация) в направлении их приведения в соответствие с текущей версией \textit{Технологии OSTIS}; 
	\item в том, что обеспечивается перманентная поддержка семантической совместимости эксплуатируемых \textit{интеллектуальных компьютерных систем} (\textit{ostis-систем}) в ходе их собственной эволюции, а также в ходе эволюции самой \textit{Технологии OSTIS}; 
	\item в том, что основой деятельности (функционирования) иерархических коллективов \textit{ostis-систем} является \myuline{децентрализованное} планирование, инициирование и ситуационное управление коллективно выполняемыми действиями (процессами), осуществляемыми в рамках как долговременно существующих, так и временно существующих \textit{коллективов ostis-систем}; 
	\item в существенном повышении эффективности расширения и использования Библиотеки типовых (многократно используемых) компонентов \textit{ostis-систем} (\textit{Библиотеки OSTIS}) благодаря: 
	\begin{textitemize}
		\item исключению семантической эквивалентности компонентов;
		\item существенному сокращению многообразия логически и функционально эквивалентных компонентов; 
		\item наличию простой и достаточно легко автоматизируемой процедуры интеграции компонентов указанной библиотеки и, соответственно, процедуры сборки \textit{ostis-систем} из готовых компонентов \textit{Библиотеки OSTIS}; 
	\end{textitemize}
	\item  в ориентации на создание \myuline{комплексной} модели, обеспечивающий \myuline{согласование} всего многообразия видов и областей \textit{человеческой деятельности} и на разработку архитектуры глобального комплекса \textit{ostis-систем}, обеспечивающего автоматизацию указанного многообразия (\textit{Экосистемы OSTIS}).
\end{textitemize}

\section*{Актуальные проекты текущего этапа работ по развитию Технологии OSTIS}
\label{concl_actual_projects_current_stage_work_development_technology}

Перечислим некоторые актуальные на данном этапе проекты прикладных \textit{интеллектуальных компьютерных систем нового поколения} и средства их разработки:
\begin{textitemize}
	\item 
	Разработка формализованного \textbf{\textit{Стандарта интеллектуальных компьютерных систем нового поколения}}, представленного в составе \textit{базы знаний} интеллектуального портала научно-технических знаний по теории \textit{интеллектуальных компьютерных систем нового поколения} и обеспечивающего \textit{семантическую совместимость компьютерных систем} этого класса.
	\medskip
	\item 
	Разработка формализованного \textbf{\textit{Стандарта методов и средств поддержки жизненного цикла интеллектуальных компьютерных систем нового поколения}}, представленного в составе \textit{базы знаний} интеллектуальной \textit{Метасистемы автоматизации поддержки жизненного цикла компьютерных систем нового поколения} (\textit{Метасистемы OSTIS}).
	\medskip
	\item 
	Разработка комплексной \textbf{\textit{Библиотеки типовых компонентов интеллектуальных компьютерных систем нового поколения}} (\textit{Библиотеки OSTIS}), обеспечивающей совместимость типовых (многократно используемых) компонентов и полную автоматизацию их интеграции (соединения) в процессе сборочного (компонентного) проектирования семантически совместимых \textit{интеллектуальных компьютерных систем нового поколения}.
	\medskip
	\item 
	В рамках \textbf{\textit{Метасистемы OSTIS}} обеспечение \myuline{широкого} доступа к текущему состоянию \textit{Стандарта OSTIS} и разработка соответствующих средств семантической визуализации и навигации.
	\medskip
	\item 
	В рамках \textbf{\textit{Метасистемы OSTIS}} разработка средств автоматизации и управления процессом коллективного совершенствования (модернизации, реинжиниринга) \textit{Стандарта OSTIS}.
	\medskip
	\item
	Разработка \textbf{\textit{программной платформы}} для реализации \textit{интеллектуальных компьютерных систем нового поколения}.
	\medskip
	\item 
	Разработка \textbf{\textit{ассоциативного семантического компьютера}} для реализации \textit{интеллектуальных компьютерных систем нового поколения}. Это \myuline{универсальный} компьютер, в котором осуществляется аппаратная реализация ассоциативной реконфигурируемой (структурно перестраиваемой) памяти, в которой переработка информации сводится к реконфигурации связей между элементами памяти.
	\medskip
	\item Разработка архитектуры \textit{интеллектуальной компьютерной системы нового поколения}, которая является \textbf{\textit{\myuline{персональным} интеллектуальным ассистентом}} (секретарем, референтом) для \myuline{каждого} пользователя, обеспечивающим максимально возможную автоматизацию процесса взаимодействия пользователя со всей Глобальной экосистемой \textit{интеллектуальных компьютерных систем нового поколения} (\textit{Экосистемой OSTIS}). \textit{База знаний} каждого такого \textit{персонального интеллектуального ассистента} включает в себя:
	\begin{textitemize}
		\item персональную информацию соответствующего пользователя, доступ к которой другим \textit{интеллектуальным компьютерным системам} предоставляет персональный интеллектуальный ассистент этого пользователя, но обязательно с разрешения этого пользователя и с сообщением пользователю соответствующих факторов риска. Персональная информация пользователя --- это его медицинские данные, биографические данные, личные фотографии, неопубликованная интеллектуальная собственность, формируемые или отправленные сообщения, адресуемые другим пользователям или различным сообществам.
		\item информацию о различных сообществах \textit{Глобальной экосистемы интеллектуальных компьютерных систем нового поколения}, членом которых является соответствующий (ассистируемый) пользователь, с указанием роли (должности, обязанности), которую выполняет указанный пользователь в рамках каждого такого сообщества. Указанных сообществ может быть много --- профессиональные сообщества, друзья, родственники, сообщества потребителей-производителей, административно-гражданские сообщества, банки, сообщества медицинского обслуживания и др.
		\item информацию о собственных планах и намерениях (как о стратегических, так и о ближайших, включая встречи, переговоры, совещания)
	\end{textitemize}
	\textit{Решатель задач} персонального интеллектуального ассистента
	\begin{textitemize}
		\item обеспечивает максимально возможную автоматизацию различных видов профессиональной \myuline{индивидуаль-} \myuline{ной} деятельности соответствующего (обслуживаемого) пользователя; 
		\item обеспечивает интеллектуальное посредничество (представление интересов) обслуживаемого пользователя в рамках всех сообществ, в состав которых он входит.
	\end{textitemize}
	\textit{Пользовательский интерфейс} персонального интеллектуального ассистента
	\begin{textitemize}
		\item предоставляет пользователю средства управления его \myuline{индивидуальной} деятельностью, осуществляемой \myuline{совместно} с соответствующим ему персональным интеллектуальным ассистентом;
		\item обеспечивает \myuline{унифицированный} характер взаимодействия пользователей в рамках различных сообществ, в которые он входит. Простейшим видом сообществ является разовый диалог двух пользователей.
	\end{textitemize}
	\item
	Разработка унифицированного \textbf{\textit{комплекса средств автоматизации индивидуального проектирования фрагментов баз знаний}}, входящего в состав \textit{персонального интеллектуального ассистента} каждого пользователя и обеспечивающего поддержку индивидуального вклада в развитие как собственной (персональной) \textit{базы знаний}, так и \textit{базы знаний} других систем, входящих в состав \textit{Экосистемы интеллектуальных компьютерных систем}. В состав указанного комплекса средств автоматизации входят
	\begin{textitemize}
		\item редактор внутреннего представления знаний (редактор \textit{sc-текстов});
		\item редакторы различных внешних форм представления знаний (\textit{sc.g-текстов}, \textit{sc.n-текстов});
		\item трансляторы с внутреннего представления знаний на различные внешние формы представления;
		\item трансляторы с каждой формы внешнего представления знаний во внутреннее их представление;
		\item средства синтаксического и семантического анализа проектируемого фрагмента \textit{базы знаний};
		\item транслятор, обеспечивающий преобразование внутреннего представления знаний (в \textit{SC-коде}) в естествен\-но-языковое представление в формате языка разметки LaTeX, удовлетворяющее требованиям, предъявляемым к оформлению статей в сборниках научно-технических материалов. Данный транслятор позволит сконцентрировать усилия разработчиков различных интеллектуальных компьютерных систем на формализацию научно-технических знаний, используемых в \textit{интеллектуальных компьютерных системах}, и существенно снизить трудоемкость подготовки и оформления публикаций соответствующих научно-технических результатов.\\
		В перспективе различные научно-технические журналы должны быть преобразованы в интеллектуальные порталы коллективно разрабатываемых научно-технических знаний в различных областях.
	\end{textitemize}
	\item
	Разработка в рамках персонального интеллектуального ассистента набора \textbf{\textit{средств индивидуального комплексного перманентного медицинского контроля и мониторинга}} соответствующего (обслуживаемого) пользователя
	\medskip
	\item Разработка для каждого сообщества \textit{интеллектуальных компьютерных систем нового поколения} унифицированного комплекса \textbf{\textit{средств коллективной разработки общей базы знаний}} этого сообщества (\textit{базы знаний} корпоративной системы указанного сообщества), в состав которого входят:
	\begin{textitemize}
		\item средства сборки (интеграции) разрабатываемой \textit{базы знаний} из индивидуально разрабатываемых её фрагментов;
		\item средства согласования индивидуально разработанных фрагментов (персональных точек зрения, эпицентром чего является согласование используемых \textit{понятий});
		\item средства взаимного рецензирования;
		\item средства согласованной корректировки \textit{базы знаний};
		\item средства формирования и согласования плана совершенствования коллективно разрабатываемой \textit{базы знаний};
		\item средства контроля и управления процессом совершенствования коллективно разрабатываемой \textit{базы знаний}.
	\end{textitemize}
	\item Расширение набора \textbf{\textit{средств автоматизации проектирования}} различных видов компонентов \textit{интеллектуальных компьютерных систем нового поколения} (\textit{ostis-систем}) и различных классов таких систем.
	\medskip
	\item Разработка формальной структуры глобального комплекса автоматизируемой человеческой деятельности и соответствующей этому архитектуры \textbf{\textit{Экосистемы OSTIS}}. Существенное расширение направлений применения \textit{Технологии OSTIS} (медицина, промышленность, строительство, юриспруденция и так далее).
	\medskip
	\item Разработка в рамках \textit{Экосистемы OSTIS} \textbf{\textit{комплекса средств и методик подготовки специалистов в области Искусственного интеллекта}} (на уровне обучения студентов, магистрантов и аспирантов).
	\medskip
	\item Разработка в рамках \textit{Экосистемы OSTIS} \textbf{\textit{комплекса средств информатизации среднего образования}} с помощью семантически совместимых \textit{интеллектуальных компьютерных систем нового поколения}.
	\medskip
	\item Разработка в рамках \textit{Экосистемы OSTIS} \textbf{\textit{комплекса средств информатизации высшего \myuline{технического} образования}} с помощью семантически совместимых \textit{интеллектуальных компьютерных систем нового поколения}.
\end{textitemize}