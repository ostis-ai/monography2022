\chapter*{\LARGE Заключение. Основные направления, проблемы и перспективы развития интеллектуальных компьютерных систем нового поколения и соответствующей им технологии}
\label{chap_conclusion}
\addcontentsline{toc}{part}{Заключение. Основные направления, проблемы и перспективы развития интеллектуальных компьютерных систем нового поколения и соответствующей им технологии}
\chapauthortoc{Голенков В.В.\\Гулякина Н.А.}

\begin{SCn}
	\begin{scnrelfromlist}{автор}
		\scnitem{Голенков В.~В.}
		\scnitem{Гулякина Н.~А.}
	\end{scnrelfromlist}
\end{SCn}

\section*{Особенность текущего состояния работ в области Искусственного интеллекта}

Необходимость перехода от современных \textit{компьютерных систем} (в том числе, и от современных \textit{интеллектуальных компьютерных систем}) к \textit{интеллектуальным компьютерных системам нового поколения} обусловлена необходимостью перехода к автоматизации более сложных (комплексных) видов и областей деятельности требующих создания целых комплексов компьютерных систем, способных эффективно взаимодействовать между собой в процессе \uline{коллективного} решения сложных задач.

Компьютерные системы, обладающие указанными способностями называют \textit{интероперабельными компьютерными системами}. Поскольку указанные компьютерные системы не могут не иметь высокого \textit{уровня интеллекта}, мы также называем их \textit{интеллектуальными компьютерными системами нового поколения}. Высокий \textit{уровень интеллекта} интероперабельным компьютерным системам необходим:

\begin{textitemize}
	\item для адекватной оценки собственной компетенции и компетенции своих партнеров;
	\item для обеспечения взаимопонимания, договороспособности и координации (согласованности) своих действий с действиями партнеров в ходе \textit{коллективного решения сложных задач} в условиях возможного возникновения непредсказуемых (нештатных) обстоятельств.
\end{textitemize}

Очевидно, что для создания и эксплуатации таких \textit{интероперабельных интеллектуальных компьютерных систем} необходимо:

\begin{textitemize}
	\item разработать общую формальную теорию таких систем;
	\item разработать комплексную технологию проектирования и поддержки жизненного цикла этих систем;
	\item разработать общую формальную теорию всего многообразия видом и областей человеческой деятельности, которые целесообразно автоматизировать.
\end{textitemize}

В основе предлагаемого нами подхода к построению \textit{\textit{интероперабельных интеллектуальных компьютерных систем}} лежат следующие принципы:

\begin{textitemize}
	\item смысловое представление знаний, хранимых в памяти интерооперабельных интеллектуальных компьютерных систем;
	\item онтологическая структуризация и систематизация хранимых в памяти знаний;
	\item децентрализованная ситуационная агенто-ориентированная организация процесса решения задач;
	\item конвергенция и глубокая (диффузная) интеграция различных моделей решения задач и, как следствие гибридный характер решателей задач;
	\item смысловая интеграция информации поступающей из вне по разным сенсорным каналам и на разных языках в результате трансляции входной информации на общий язык внутреннего смыслового представления знаний.
\end{textitemize}

\begin{SCn}
	\scnheader{следует отличать*}
	\begin{scnhaselementset}
		\scnitem{индивидуальная интеллектуальная компьютерная система нового поколения}
		\scnitem{коллективная интеллектуальная компьютерная система нового поколения}
		\begin{scnindent}
			\scnsubset{коллектив индивидуальных интеллектуальных компьютерных систем нового поколения}
			\scnsubset{иерархический коллектив интеллектуальных компьютерных систем нового поколения}
			\begin{scnindent}
				\scnidtf{коллектив интеллектуальных компьютерных систем нового поколения, членами которого могут быть как коллективные, так и индивидуальные интеллектуальные компьютерные системы}
			\end{scnindent}
		\end{scnindent} 
	\end{scnhaselementset}\scnheader{}
\end{SCn}

Индивидуальные интеллектуальные компьютерные системы нового поколения имеют следующие особенности:

\begin{textitemize}
	\item Индивидуальные интеллектуальные компьютерные системы нового поколения невозможно декомпозировать на подсистемы, которые можно разрабатывать независимо друг от друга и согласовывать только входам-выходам, реализуя принцип ``черного ящика''
	\item В индивидуальной интеллектуальной компьютерной системы нового поколения необходима конвергенция, совместимость и ``осмысленное'' взаимодействие самых различных видов знаний и моделей решения задач. То есть индивидуальная интеллектуальная компьютерная система нового поколения должна быть гибридной системой
\end{textitemize}

\begin{SCn}
	\scnheader{Технология OSTIS}
	\scnidtf{Комплексная технология Искусственного интеллекта, обеспечивающая
		\begin{scnitemize}
			\item совместимость всех частных технологий Искусственного интеллекта,
			\item совместимость и интероперабельность разрабатываемых интеллектуальных компьютерных систем
			\item поддержку не только проектирования интеллектуальных компьютерных систем, но и всего их жизненного пути
		\end{scnitemize}
	}
	\begin{scnrelfromlist}{предъявляемые требования}
		\scnitem{универсальность --- Технология OSTIS должна быть ориентирована на разработку и сопровождение интероперабельных интеллектуальных компьютерных систем \uline{любого} назначения}
		\scnitem{Технология OSTIS должна обеспечить перманентную эволюцию самой Технологии OSTIS (самой себя)}
	\end{scnrelfromlist}
\end{SCn}

\section*{Задачи текущего этапа работ в области Искусственного интеллекта в направлении создания интероперабельных интеллектуальных компьютерных систем}

Достижение указанной цели требует получения авторов на следующие вопросы:
\begin{textitemize}
	\item Какие требования предъявляются к интеллектуальным компьютерным системам, обеспечивающим указанную \uline{комплексную} автоматизацию человеческой деятельности
	\item Почему современные интеллектуальные компьютерные системы указанным выше требованиям не удовлетворяют и, соответственно, почему необходим переход к принципиально новому поколению интеллектуальных компьютерных систем
	\item Какие функциональные принципы должны лежать в основе интеллектуальных компьютерных систем нового поколения;
	\item Какие принципы должны лежать в основе \uline{максимально автоматизируемой} (!) технология проектирования и поддержки всего жизненного цикла интеллектуальной компьютерных систем нового поколения
	\item Какие принципы должны лежать в основе структуры и организации различных видов и областей человеческой деятельности в условиях её комплексной и максимально возможной автоматизации с помощью интеллектуальных компьютерных систем нового поколения
\end{textitemize}

%в другое место
%Теория человеческой деятельности в рамках Общества 5.0
%\begin{textitemize}
%	\item чёткая иерархичность (от индивидуальной деятельности людей и индивидуальных интеллектуальных компьютерных систем к различным уровням коллективов)
%	\item фрактальность !!!
%	\item стирание грани между деятельностью людей и интеллектуальной компьютерной системой
%	\item максимально возможная унификация
%	\item баланс унификации / многообразия
%	\item децентрализация
%\end{textitemize}

К числу текущих задач по созданию теории и технологии интеллектуальных компьютерных систем нового поколения относятся:
\begin{textitemize}
	\item Разработка теории многоагентных систем, агентами в которых являются индивидуальные или коллективные интерооперабельные компьютерные системы.
	\item Унификация и стандартизация различных моделей представления и обработки знаний. Эффект от данной унификации будет виден не сразу. Но, если этого происходить не будет, мы никогда не придем к эффективной \uline{комплексной} автоматизации человеческой деятельности. Эффективное многообразие методов и средств автоматизации приводит только к необоснованному \uline{дублированию} и повышению сложности использования и сопровождения разрабатываемых систем.
	\item Конвергенция и интеграция различных направлений Искусственного интеллекта.
	\item Различные направления Искусственного интеллекта сейчас имеют достаточно высокий уровень развития (signal processing, natural language processing (NLP), логические модели, онтологические модели, модели интеллектуальных компьютерных систем, многоагентные модели). Интеграция всех этих направлений является пусть и достаточно трудоёмкой задачей, но задачей вполне решаемой, в основе решения которой лежит согласование смежных понятий --- ``точек соприкосновения''. Но почему-то этого не происходит!!
	\item Конвергенция таких видов деятельности в области Искусственного интеллекта, как:
	\begin{textitemize}
		\item подготовка специалистов в области искусственного интеллекта;
		\item инженерная деятельность по разработке прикладных интеллектуальных компьютерных систем нового поколения;
		\item развитие технологии проектирования и поддержки жизненного цикла интеллектуальных компьютерных систем нового поколения;
		\item научно-исследовательская деятельность в области Искусственного интеллекта.
	\end{textitemize}
	\item Конвергенция деятельности в области Искусственного интеллекта с другими областями человеческой деятельности. Для развития Технологии OSTIS необходима также конвергенция Технологии OSTIS \uline{со всеми} видами и областями человеческой деятельности, которые не входят в состав деятельности в области Искусственного интеллекта. То есть дальнейшее развитие Технологии OSTIS носит ярко выраженный \uline{междисциплинарный характер}. Это означает что \uline{все знания}, накапливаемые человеческим обществом в самых различных областях должны быть представлены в составе Глобальной базы знаний Экосистемы OSTIS (с помощью порталов научно-технических, административных и прочих знаний), должны быть чётко \uline{стратифицированы} в виде иерархической системы \uline{семантически} совместимых \uline{онтологий} и тем самым превращены в иерархическую систему семантически совместимых компонентов баз знаний ostis-систем различного прикладного назначения.
	\item Обеспечение семантической совместимости интеллектуальных компьютерных систем нового поколения не только на этапе их проектирования, но и на всех последующих этапах их жизненного цикла.
	\item Разработка Модели \uline{коллективного} поведения интероперабельных интеллектуальных компьютерных систем, то есть модели децентрализованного коллективного решения задач в рамках:
	\begin{textitemize}
		\item многоагентной системы, агенты которой являются внутренними агентами индивидуальной интеллектуальной компьютерной системы, взаимодействующими через общую память (через общую базу знаний, хранимую в \uline{одной} памяти)
		\item многоагентной системы, агенты которой являются интероперабельными интеллектуальными компьютерными системами, взаимодействующими через общую базу знаний, хранимую в памяти корпоративной интеллектуальной компьютерной системы или в памяти координатора деятельности \uline{временного} коллектива интеллектуальной компьютерной системы 
	\end{textitemize}
	\item В рамках теории коллективного решения задач следует выделить два типа задач:
	\begin{textitemize}
		\item задача, соответствующая компетенции того коллектива интеллектуальных компьютерных систем, в рамках которого эта задача инициирована
		\item задача, выходящая за пределы компетенции того коллектива интеллектуальной компьютерной системы, в рамках которого эта задача инициирована. Эта задача требует формирования \uline{временного} коллектива, \uline{координатором} (но не менеджером) которого становится та интеллектуальная компьютерная система, в рамках которой эта задача инициировалась. Для этого необходимо найти те интеллектуальные компьютерные системы, которые в совокупности обеспечат необходимую компетенцию.
	\end{textitemize}
	Отметим при этом, что каждая интероперабельная интеллектуальная компьютерная система (как индивидуальная, так и коллективная) должна \uline{знать} свою компетенцию для того, чтобы каждая интероперабельная интеллектуальная компьютерная система могла быстро определить, сможет или не сможет она решить ту или иную заданную (возникшую) задачу. Это, в частности, необходимо для формирования \uline{временных} коллективов интеллектуальной компьютерной системы
	\item Разработка мощной библиотеки многократно используемых и совместимых компонентов интеллектуальной компьютерной системы нового поколения, которая обеспечивает \uline{полную} автоматизацию интеграции этих компонентов в процессе сборки проектируемых систем
	\item Перманентное расширение библиотеки многократно используемых компонентов интероперабельных интеллектуальных компьютерных систем должно стать важнейшим направлением в самых различных областях человеческой деятельности:
	\begin{textitemize}
		\item Разработчики любой интеллектуальной компьютерной интеллектуальной системы должны \uline{декомпозировать} разработанную интеллектуальную компьютерную систему на множество компонентов, вводимых в состав Библиотеки OSTIS --- так, чтобы разработка любой аналогичной системы свелась к сборке компонентов из Библиотеки OSTIS.
		\item Научно-техническая деятельность должна сводиться к развитию баз знаний различных интеллектуальных порталов научно-технических знаний. При этом база знаний каждого такого портала должна декомпозироваться на фрагменты, включаемые в состав Библиотеки многократно используемых компонентов баз знаний ostis-системы (для этого они соответствующим образом специфицируются), которые могут иерархически входить друг в друга.
		\item Таким образом \uline{все}(!) должны заботиться о \uline{расширении} Библиотеки OSTIS, это приводит к существенному снижению трудоёмкости разработки новых ostis-систем в рамках Экосистемы OSTIS. При этом авторство компонентов Библиотеки OSTIS должно \uline{поощряться}, что является фундаментальной основной развития рынка знаний, экономики знаний.  
	\end{textitemize}
\end{textitemize}

Если грамотно использовать Технологию OSTIS, то разработка любой ostis-системы сводиться в её автоматической сборке из указываемых разработчиком компонентов этой системы. Некоторые компоненты разрабатываемой ostis-системы входят в текущее состояние Библиотеки OSTIS, а некоторые из них дополнительно разработаны. Но при этом каждый такой новый компонент является либо результатом модификации существующего компонента из Библиотеки OSTIS, либо компонентом, который может и \uline{должен быть} специфицирован и включен в Библиотеку OSTIS. Таким образом разработчик прикладной ostis-системы должен разработать не только эту систему, но и внести вклад в развитие Библиотеки OSTIS, в результате которого разрабатываемая им ostis-система может быть собрана без дополнительно разрабатываемых компонентов, а только из компонентов Библиотеки OSTIS. Если все разработчики прикладных систем будут так действовать, то темпы повышения уровня автоматизации человеческой деятельности будут существенно возрастать.

\section*{Методологические проблемы текущего этапа работ в области Искусственного интеллекта}

\begin{SCn}
	\scntext{эпиграф}
	{Самые сложные проблемы --- это те, которые мы не осознаем и, особенно, те, причиной которых является наше несовершенство.
	}
	
	\begin{scnrelfromlist}{подраздел}
		\scnitem{Социальная ответственность специалистов в области Искусственного интеллекта}
		\scnitem{Глобальная цель деятельности в области Искусственного интеллекта}
		\scnitem{Общие требования, предъявляемые к специалистам в области Искусственного интеллекта}
		\scnitem{Требования, предъявляемые к фундаментальной подготовке специалистов в области Искусственного интеллекта}
		\scnitem{Проблемы текущего этапа разработки теории и технологии интероперабельных интеллектуальных компьютерных систем}
	\end{scnrelfromlist}
\end{SCn}

\subsection*{Социальная ответственность специалистов в области Искусственного интеллекта}
Современный этап развития теории и практики Искусственного интеллекта обнажает целый спектр проблем, препятствующих этому развитию. Дальнейшее развитие технологий Искусственного интеллекта
\begin{textitemize}
	\item с одной стороны, может и достаточно быстро осуществить переход современного общества на принципиально новый уровень его эволюции, обеспечивающий \uline{комплексную} автоматизацию всех подлежащих автоматизации видов и областей человеческой деятельности, а также обеспечивающий максимально возможный комфорт и максимально возможное раскрытие творческого потенциала \uline{каждого} человека;
	\item с другой стороны, может достаточно долго и весьма убедительно для неграмотного обывателя \uline|{имитировать} указанный прогресс автоматизации человеческой деятельности --- любая даже весьма достойная цель может быть загублена имитацией её достижения;
	\item с третьей стороны, может достаточно быстро привести человеческое общество к деградации и самоуничтожению.
\end{textitemize}

Таким образом на современном этапе развития технологий Искусственного интеллекта, \uline{уровень социальной ответственности} специалистов в области Искусственного интеллекта является определяющим фактором развития человеческого общества.
Опасность для человеческого общества исходит не от интеллектуальных компьютерных систем, а от мотивации специалистов, которые разрабатывают эти системы. Очевидно, что создание интеллектуальных компьютерных систем, предназначенных \uline{осознанного} нанесения любого ущерба человеческому обществу, и требующих создания соответствующих интеллектуальных средств обеспечения безопасности, является короткой дорогой к самоуничтожению.

Усилия специалистов в области Искусственного интеллекта должны быть направлены на существенное повышение уровня интеллекта человеческого общества в целом, основой чего является \uline{комплексная} автоматизация \uline{всех} тех видом и областей человеческой деятельности, которые принципиально имеет смысл автоматизировать.

\subsection*{Глобальная цель деятельности в области Искусственного интеллекта}
Рассмотрим вопрос, Почему современный этап деятельности в области Искусственного интеллекта требует формулировки глобальной цели этой деятельности и перманентного её уточнения.

Современное состояние Искусственного интеллекта можно охарактеризовать как глубокий методологический кризис, обусловленных:
\begin{textitemize}
	\item тем, что научные результаты в этой области вышли из научных лабораторий и стали приносить реальную практическую пользу;
	\item отсутствием понимания того, что получение серьезных научных результатов в той или иной области и создание технологий, обеспечивающих \uline{эффективное} практическое использование этих результатов --- это соизмеримые по значимости и сложности задачи. Особенно это касается Искусственного интеллекта.
\end{textitemize}

Последнее обстоятельство приводит к неоправданной эйфории, иллюзии благоплучия и к бурно расцветающей эклектике, которая абсолютно игнорирует даже казалось бы очевидные законы общей теории систем.

К сожалению, локальное внедрение результатов научных исследований в области Искусственного интеллекта, локальная автоматизация бизнес-процессов какой-либо организации без учёта системной организации всего комплекса методов и средств автоматизации различных видов и областей человеческой деятельности приводит к неоправданному дублированию результатов. 

Современное состояние работ в области Искусственного интеллекта требует уточнения. Если в ближайшее время не произойдёт осознания \textit{глобальной (стратегической) цели} работ в области Искусственного интеллекта, то деятельность в этой области в целом будет осуществляться в стиле ''лебедя, рака и щуки''. Трата усилий не приведёт к целостному практически значимому результату. ''Вектора'' конкретных направлений этой деятельности, ''вектора'' наших усилий не будут иметь одинаковую направленность, что существенно снизит общую  производительность всей этой деятельности и качество общего (суммарного) результата.

Какова же должна быть \textit{стратегическая задача} (сверхзадача), которую должны решить специалисты в области Искусственного интеллекта. Очевидно, что такой сверхзадачей является переход всего комплекса человеческой деятельности на принципиально новый уровень максимально возможной его автоматизации, в рамках которого принципиально неавтоматизируемой частью человеческой деятельности остаётся, прежде всего, \textit{творческая} деятельность, в частности, научно-исследовательская деятельность, преподавательская и воспитательная деятельность, перманентное повышение уровня комплексной автоматизации человеческой деятельности. Основная цель \textit{комплексной} автоматизации человеческой деятельности заключается не только в том, чтобы автоматизировать то, что \uline{можно эффективно} автоматизировать с помощью методов Искусственного интеллекта, а в том, чтобы автоматизировать \uline{все}(!) ''узкие места'' человеческой деятельности, которые определяют общую её производительность в различных областях.

Таким образом, в настоящее время технологии Искусственного интеллекта находятся на пороге перехода к принципиально новому уровню развития --- на пороге перехода от решения частных (локальных) задач к решению глобальной задачи комплексной автоматизации всех видов и областей человеческой деятельности, что требует автоматизации решения не только частных актуальных и важных задач, но и автоматизации решения задач всё более и более высокого уровня, для которых автоматизируемые сейчас задачи становятся подзадачами. Другими словами, при автоматизации решения комплексных задач (надзадач) автоматизация фокусируется на разработке методов и средств \textit{взаимодействия между средствами решения локальных задач} (частных задач).

Перенос акцента на автоматизацию решения не просто интеллектуальных задач, а на автоматизацию решения \uline{комплексных} задач, подзадачами которых являются \uline{разнообразные} интеллектуальные задачи, не только переводит технологии Искусственного интеллекта на принципиально новый уровень, но и окажет существенное влияние \uline{на все стороны человеческой деятельности}:
\begin{textitemize}
	\item  научно-исследовательские и научно-технические работы должны приобрести конвергентный взаимообогащающий характер;
	\item основой образования должна стать междисциплинарность;
	\item основой глобальной автоматизации человеческой деятельности должна стать общая комплексная формальная и перманентно совершенствуемая теория человеческой деятельности, в основе которой должна лежать междисциплинарная конвергентная методика, направленная на преодоление эклектичного подхода.
\end{textitemize}

Следовательно, основной целью комплексной автоматизации всевозможных видов и областей человеческий деятельности с помощью \textit{интеоперабельных интеллектуальных компьютерных систем} является существенное повышение \textit{уровня интеллекта}  человеческого общества в целом.

Современное человеческое общество --- это сложнейшая распределённая многоагентная кибернетическая система, развитие которой осуществляется, к сожалению, с нарушением многих законов кибернетики и, в частности, с нарушением критериев, определяющих уровень интеллекта иерархических многоагентных систем. Уровень интеллекта таких систем определяется целым рядом казалось бы очевидных факторов:
\begin{textitemize}
	\item тем, каков объём и качество знаний, накопленных многоагентной системой и доступных всем агентам (субъектам), входящим в эту систему
	\begin{textitemize}
		\item насколько этих знаний достаточно для организации управления деятельностью этой системы;
		\item насколько эти знания корректны (противоречивы) и адекватны;
		\item насколько велика конвергентность, компактность и чистота этих знаний (здесь учитывается наличие информационного мусора, информационного дублирования);
		\item насколько хорошо структурированы (систематизированы) накапливаемые знаний;
	\end{textitemize}
	\item тем, как осуществляется доступ каждого агента многоагентной системы к знаниям, хранимым в общей памяти всей многоагентной системы;
	\item тем, как эти знания накапливаются и эволюционируют (как многоагентная система самообучается)
	\begin{textitemize}
		\item как многоагентная система учится на собственных ошибках,
		\item как многоагентная система повышает качество своих знаний;
	\end{textitemize}
	\item тем, как многоагентная система в целом и каждый агент в частности используют накопленные в общей памяти знания для решения различных задач.
\end{textitemize}

Таким образом, если рассматривать современное человеческое общество с позиций теории многоагентных систем, являющихся коллективами интеллектуальных систем (в данном случае, не искусственных, а естественных интеллектуальных систем), то очевидно, что слудующий этап его эволюций требует:
\begin{textitemize}
	\item автоматизации накопления, анализа и перманентного повышения качества накапливаемых человеком знаний;
	\item автоматизации эффективного использования накопленных человечеством знаний при решении задач самого различного уровня, требующих формирования различных коллективов, состоящих из людей и интеллектуальных компьютерных систем. Каждый такой коллектив предназначается либо для решения какой-либо одной конкретной задачи, либо для решений некоторого множества задач в некоторой области;
	\item повышения уровня конвергенции знаний, методов, действий, создаваемых технических систем;
	\item повышения уровня интероперабельности (как для интеллектуальных компьютерных систем, так и для людей) 
\end{textitemize}
