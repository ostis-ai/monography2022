\chapauthor{Загорский А.С.\\Голенков В.В.\\Шункевич Д.В.}
\chapter{Факторы, определяющие уровень интеллекта кибернетических систем}
\chapauthortoc{Загорский А.С.\\Голенков В.В.\\Шункевич Д.В.}
{\label{chapter_intro}}

\abstract{
	Рассмотрена иерархическая система свойств (в т.ч. способностей) кибернетических систем, определяющих их качество и позволяющих сформулировать требования, которым должна удовлетворять высокоинтеллектуальная система (кибернетическая система с сильным интеллектом). 
	Уровень качества кибернетических систем определяется достаточно большим набором свойств (параметров, характеристик) кибернетических систем, каждое из которых определяет уровень качества кибернетической системы в соответствующем аспекте (ракурсе), указывая (задавая) уровень развития конкретных  способностей и возможностей кибернетической системы. 
	При этом важно подчеркнуть следующее:
	\begin{itemize}
		\item существенное значение имеет не столько сам набор свойств, а иерархия этих свойств, позволяющая уточнять направления проявления каждого свойства;
		\item существенное значение также имеет \uline{баланс} уровней развития различных свойств --- вклад разных свойств, обеспечивающих значение одного и того же свойства более высокого уровня иерархии, а значение этого свойства более высокого уровня может быть разным.
		Из этого следует, что не всегда следует акцентировать внимание на развитие некоторых свойств (характеристик).
		Нужен целостный, коллективный подход;
		\item рассмотренная иерархия свойств кибернетических систем является общей как для естественных, так и для искусственных кибернетических систем;
		\item приведенная иерархическая детализация свойств кибернетических систем (с помощью отношения ``\textit{частное свойство*}'' и отношения ``\textit{свойство-предпосылка*}''), определяющих качество таких систем, (1) дает возможность четко определить направления совершенствования (развития) кибернетических систем и (2) дает ориентир (систему критериев) для обоснования конкретных предложений по совершенствованию компьютерных систем, а также для сравнения различных альтернативных предположений;
		\item особое значение для развития кибернетических систем имеют такие их свойства, как стратифицированность, рефлексивность и социализация;
		\item важное значение имеет не только совершенствование кибернетических систем в соответствии с иерархической системой их свойств, но и совершенствование (в том числе, детализация) самой этой иерархической системы свойств.
	\end{itemize}
	Свойства (способности), которым должны удовлетворять \textit{интеллектуальные системы}, рассматриваются в целом ряде публикаций. 
	Тем не менее, для \uline{практической} реализации \textit{компьютерных систем}, обладающих указанными свойствами (способностями), т.е. \textit{интеллектуальных компьютерных систем}, необходимо детализировать (уточнить) эти \textit{свойства}, пытаясь свести их к более конструктивным, прозрачным и понятным для реализации свойствам.
}

\section{Кибернетические системы и их интеллектуализация}
% Уточнение понятия кибернетической системы
% Типология кибернетических систем
\input{author/part1/chapter_intro/intro_hs_segment1.tex}


\section{Многоагентные системы и их интеллектуализация} 
{\label{section_mas}} 

\section{Уровень интеллекта компьютерных систем}

%\input{author/references}