\chapter{Факторы, определяющие уровень интеллекта кибернетических систем}
\chapauthortoc{Загорский А.С.\\Голенков В.В.\\Шункевич Д.В.}
{\label{chap_intro}}

\vspace{-7\baselineskip}

\begin{SCn}
\begin{scnrelfromlist}{автор}
	\scnitem{Загорский А.С.}
	\scnitem{Голенков В.В.}
	\scnitem{Шункевич Д.В.}
\end{scnrelfromlist}

\bigskip

\scntext{аннотация}{
	Рассмотрена иерархическая система свойств (в т.ч. способностей) кибернетических систем, определяющих их качество и позволяющих сформулировать требования, которым должна удовлетворять высокоинтеллектуальная система (кибернетическая система с сильным интеллектом). 
	Уровень качества кибернетических систем определяется достаточно большим набором свойств (параметров, характеристик) кибернетических систем, каждое из которых определяет уровень качества кибернетической системы в соответствующем аспекте (ракурсе), указывая (задавая) уровень развития конкретных  способностей и возможностей кибернетической системы. 
	При этом важно подчеркнуть следующее:
	\begin{textitemize}
		\item существенное значение имеет не столько сам набор свойств, а иерархия этих свойств, позволяющая уточнять направления проявления каждого свойства;
		\item существенное значение также имеет \uline{баланс} уровней развития различных свойств --- вклад разных свойств, обеспечивающих значение одного и того же свойства более высокого уровня иерархии, а значение этого свойства более высокого уровня может быть разным.
		Из этого следует, что не всегда следует акцентировать внимание на развитие некоторых свойств (характеристик).
		Нужен целостный, коллективный подход;
		\item рассмотренная иерархия свойств кибернетических систем является общей как для естественных, так и для искусственных кибернетических систем;
		\item приведенная иерархическая детализация свойств кибернетических систем (с помощью отношения ``\textit{частное свойство*}'' и отношения ``\textit{свойство-предпосылка*}''), определяющих качество таких систем, (1) дает возможность четко определить направления совершенствования (развития) кибернетических систем и (2) дает ориентир (систему критериев) для обоснования конкретных предложений по совершенствованию компьютерных систем, а также для сравнения различных альтернативных предположений;
		\item особое значение для развития кибернетических систем имеют такие их свойства, как стратифицированность, рефлексивность и социализация;
		\item важное значение имеет не только совершенствование кибернетических систем в соответствии с иерархической системой их свойств, но и совершенствование (в том числе, детализация) самой этой иерархической системы свойств.
	\end{textitemize}
	Свойства (способности), которым должны удовлетворять \textit{интеллектуальные системы}, рассматриваются в целом ряде публикаций. 
	Тем не менее, для \uline{практической} реализации \textit{компьютерных систем}, обладающих указанными свойствами (способностями), т.е. \textit{интеллектуальных компьютерных систем}, необходимо детализировать (уточнить) эти \textit{свойства}, пытаясь свести их к более конструктивным, прозрачным и понятным для реализации свойствам.
}

\bigskip

\begin{scnrelfromlist}{подраздел}
	\scnitem{\ref{sec_cyb_syst}~\nameref{sec_cyb_syst}}
	\scnitem{\ref{sec_mas}~\nameref{sec_mas}}
	\scnitem{\ref{sec_comp_syst}~\nameref{sec_comp_syst}}
\end{scnrelfromlist}

\end{SCn}

\begin{SCn}

\bigskip

\begin{scnrelfromlist}{ключевое понятие}
	\scnitem{кибернетическая система}
	\scnitem{качество кибернетических систем\scnsupergroupsign}
	\scnitem{уровень интеллекта кибернетических систем\scnsupergroupsign}
	\scnitem{интеллектуальная кибернетическая система}
\end{scnrelfromlist}

\bigskip

\begin{scnrelfromlist}{ключевое знание}
	\scnitem{Классификация кибернетических систем}
	\scnitem{Обобщенная архитектура кибернетических систем}
	\scnitem{Факторы, определяющие качество кибернетических систем}
	\scnitem{Факторы, определяющие уровень интеллекта кибернетических систем}
\end{scnrelfromlist}

\bigskip

\begin{scnrelfromlist}{библиографическая ссылка}
	\scnitem{\scncite{Glushkov1979}}
\end{scnrelfromlist}

\end{SCn}

% Уточнение понятия кибернетической системы
Кибернетическая система есть cистема, которая способна \uline{управлять} своими \uline{действиями}, адаптируясь к изменениям состояния внешней среды (среды своего "обитания") в целях самосохранения (сохранения своей целостности и "комфортности"{} существования путем удержания своих "жизненно"{} важных параметров в определенных рамках "комфортности") и/или в целях формирования определенных реакций (воздействий на внешнюю среду) в ответ на определенные стимулы (на определенные ситуации или события во внешней среде), а также которая способна (при соответствующем уровне развития) эволюционировать в направлении:

\begin{itemize}
    \item изучения своей внешней среды как минимум для предсказания последствий своих воздействий на внешнюю среду, а также для предсказания изменений внешней среды, которые не зависят от собственных воздействий;
    \item изучения самой себя и, в частности, своего взаимодействия с внешней средой;
    \item создания технологий (методов и средств), обеспечивающих изменение своей внешней среды (условий своего существования) в собственных интересах.
\end{itemize}

В работе \scncite{Glushkov1979} сформировано определение кибернетической системы.


\scnheader{кибернетическая система}
\scnidtf{адаптивная система}
\scnidtf{целенаправленная система}
\scnidtf{активный субъект самостоятельной деятельности}
\scnidtf{материальная сущность, способная целенаправленно (в своих интересах) воздействовать на среду своего обитания как минимум для сохранения своей целостности, жизнеспособности, безопасности}
\scntext{примечание}{Уровень (степень) адаптивности, целенаправленности, активности у систем, основанных на обработке информации может быть самым различным.}
\scnidtf{система, организация функционирования которой основано на обработке информации о той среде, в которой существует эта система}
\scnidtf{материальная сущность, способная к активной  целенаправленной деятельности, которая  на определенном уровне развития указанной сущности становится "осмысленной", планируемой, преднамеренной деятельностью}
\scnidtf{субъект, способный на самостоятельное выполнение некоторых "внутренних"{} и "внешних"{} действий либо порученных извне, либо инициированных самим субъектом}
\scnidtf{сущность, способная выполнять роль субъекта деятельности}
\scnidtf{естественная или искусственно созданная система, способная мониторить и анализировать свое состояние и состояние окружающей среды, а также способная достаточно активно воздействовать на собственное на собственное состояние и на состояние окружающей среды}
\scnidtf{система, способная в достаточной степени самостоятельно взаимодействовать со своей средой, решая различные задачи}
\scnidtf{система, основанная на обработке информации}

% Типология кибернетических систем

\begin{SCn}
\scnheader{кибернетическая система}
\scnrelfrom{разбиение}{Признак естественности или искусственности кибернетических систем}
\begin{scnindent}
	\begin{scneqtoset}
		\scnitem{естественная кибернетическая система}
		\begin{scnindent}
			\scnidtf{кибернетическая система естественного происхождения}
   			\scnsuperset{человек}
    	\end{scnindent}
		\scnitem{компьютерная система}
		\begin{scnindent}
			\scnidtf{искусственная кибернетическая система}
			\scnidtf{кибернетическая система искусственного происхождения}
			\scnidtf{технически реализованная кибернетическая система}
		\end{scnindent}
		\scnitem{симбиоз естественных и искусственных кибернетических систем}
		\begin{scnindent}
			\scnidtf{кибернетическая система, в состав которой входят компоненты как естественного, так и
			искусственного происхождения}
			\scnsuperset{сообщество компьютерных систем и людей}
		\end{scnindent}
	\end{scneqtoset}
\end{scnindent}
\end{SCn}

Особенностью компьютерных систем является то, что они могут выполнять "роль" не только продуктов соответствующих действий по реализации этих систем, но и сами являются \textit{субъектами*}, способными выполнять (автоматизировать) широкий спектр действий.
При этом интеллектуализация этих систем существенно расширяет этот спектр.

\begin{SCn}
\scnheader{кибернетическая система}
\scnrelfrom{разбиение}{Структурная классификация кибернетических систем}
\begin{scnindent}
	\begin{scneqtoset}
		\scnitem{простая кибернетическая система}
		\scnitem{индивидуальная кибернетическая система}
		\scnitem{многоагентая система}
	\end{scneqtoset}
\end{scnindent}
\scnrelfrom{разбиение}{Классификация кибернетических систем по признаку наличия надсистемы и роли в рамках этой надсистемы}
\begin{scnindent}
	\begin{scneqtoset}
		\scnitem{кибернетическая система, не являющаяся частью никакой другой кибернетической системы}
		\begin{scnindent}
			\scnidtf{кибернетическая система, не имеющая надсистем}	
		\end{scnindent}
		\scnitem{кибернетическая система, встроенная в индивидуальную кибернетическую систему}
		\scnitem{агент многоагентной системы}
		\begin{scnindent}
			\scnidtf{кибернетическая система, являющаяся агентом одной или нескольких многоагентных систем}
		\end{scnindent}
	\end{scneqtoset}
\end{scnindent}
\end{SCn}

Простая кибернетическая система есть \textit{кибернетическая система}, уровень развития которой находится ниже уровня \textit{индивидуальных кибернетических систем} и которая является специализированным специализированным решателем задач, реализующим (интерпретирующим) чаще всего один \textit{метод} решения задач и, соответственно, решающим только \textit{задачи} заданного \textit{класса задач}.
Простая кибернетическая система может быть \textit{компонентом*}, встроенным в \textit{индивидуальную кибернетическую систему}, а также может быть \textit{агентом*} \textit{многоагентной системы}, являющейся коллективом из простых кибернетических систем.

Индивидуальная кибернетическая система — условно выделенный уровень развития \textit{кибернетических систем}, в основе которого лежит переход от \textit{специализированного решателя задач} к \textit{индивидуальному решателю задач}, обеспечивающему интерпретацию произвольного (нефиксированного) набора \textit{методов} (программ) решения задач при условии, если эти \textit{методы} введены (загружены, записаны) в \textit{память} \textit{кибернетической системы}.
Признаками индивидуальных кибернетических систем являются:
\begin{itemize}
    \item наличие \textit{памяти}, предназначенной для хранения как минимум интерпретируемых \textit{методов}	(программ)  и обеспечивающей корректировку (редактирование) хранимых \textit{методов}, а также их удаление из	\textit{памяти} и ввод (запись) в \textit{память} новых \textit{методов};
    \item легкая возможность "перепрограммировать"{} \textit{кибернетическую систему} на решение других задач, что обеспечивается наличием \textit{универсальной модели решения задач} и, соответственно, \textit{универсальным интерпретатором \uline{любых} моделей}, представленных (записанных) на соответствующем \textit{языке};
    \item наличие пусть даже простых средств коммуникации (обмена информацией) с другими \textit{кибернетическими системами} (например, с людьми);
    \item способность входить в различные \textit{коллективы кибернетических систем}.
\end{itemize}

Класс \textit{индивидуальных кибернетических систем} — это определенный этап эволюции кибернетических систем, означающий переход к кибернетическим системам, которые способны самостоятельно "выживать". \textit{индивидуальная кибернетическая система} может быть агентом (членом) \textit{многоагентной системы} (членом коллектива индивидуальных кибернетических систем), но некоторые многоагентные системы могут состоять из агентов, не являющихся \textit{индивидуальными кибернетическими системами}, представляющих собой простые специализированные кибернетические системы, выполняющие достаточно простые действия (\scncite{Stefanuk}, \scncite{fonNeuman}).

Примерами кибернетической системы, встроенной в индивидуальную кибернетическую систему, являются sc-агент ostis-системы, \textit{решатель задач ostis-системы}.

Многоагентная система представляет собой коллектив взаимодействующих автономных кибернетических систем, имеющих общую среду обитания (жизнедеятельности). 

\begin{SCn}
\scnheader{многоагентая система}
\begin{scnrelfromset}{разбиение}
	\scnitem{коллектив из простых кибернетических систем}
	\scnitem{коллектив из индивидуальных кибернетических систем}
	\scnitem{коллектив из индивидуальных и простых кибернетических систем}
\end{scnrelfromset}
\begin{scnrelfromset}{разбиение}
	\scnitem{одноуровневый коллектив кибернетических систем}
	\begin{scnindent}
		\scnidtf{многоагентная система, агентами которой не могут быть многоагентные системы}
	\end{scnindent}
	\scnitem{иерархический коллектив кибернетических систем}
	\begin{scnindent}
		\scnidtf{многоагентная система, по крайней мере одним  агентом которой является многоагентная система}
	\end{scnindent}
\end{scnrelfromset}
\end{SCn}


% коллектив из простых кибернетических систем ?

Коллектив индивидуальных кибернетических систем --- многоагентная система, агентами (членами) которой являются \uline{индивидуальные} (!) кибернетические системы. Примерами коллектива индивидуальных кибернетических систем могут быть как коллективы людей, так и коллективы компьютерных систем и людей.

% коллектив из индивидуальных и простых кибернетических систем ?

Одноуровневый коллектив кибернетических систем определён как специализированное средство решения задач, реализующее либо \uline{одну} модель параллельного (распределенного) решения задач соответствующего класса, либо комбинацию \uline{фиксированного числа} разных и параллельно реализованных моделей решения задач.

Агентами иерархического коллектива кибернетических систем могут быть:
\begin{itemize}
    \item индивидуальные кибернетические системы;
    \item коллективы индивидуальных кибернетических систем;
    \item коллективы, состоящие из индивидуальных кибернетических систем и коллективов индивидуальных кибернетических систем и т.д.
\end{itemize}


Рассмотрим структуру кибернетической системы.
\begin{SCn}
	\scnheader{кибернетическая система}
	\begin{scnrelfromset}{обобщенная декомпозиция}
		\scnitem{информация, хранимая в памяти кибернетической системы}
		\scnitem{абстрактная память кибернетической системы}
		\scnitem{решатель задач кибернетической системы}
		\scnitem{физическая оболочка кибернетической системы}
	\end{scnrelfromset}
\end{SCn}

Информация, хранимая в памяти \textit{кибернетической системы}, представляет собой информационную модель среды, в которой действует (существует, функционирует) эта \textit{кибернетическая система}. Текущее состояние памяти кибернетической системы.

Абстрактная память кибернетической системы есть внутренняя абстрактная информационная среда кибернетической системы, представляющая собой динамическую информационную  конструкцию, каждое состояние которой есть не что иное, как информация, хранимая в памяти кибернетической системы в соответствующий момент времени. Является подмножеством динамической информационной конструкции.

Решателем задач кибернетической системы называют совокупность всех навыков (умений), приобретенных кибернетической системой к рассматриваемому моменту. Встроенный в кибернетическую систему субъект, способный выполнять целенаправленные ("осознанные") действия во внешней среде этой кибернетической системы, а также в её внутренней среде (в абстрактной памяти).


% Возможно фрагмент излишний, стоит переместить в другие разделы, например понятие действия в sc-памяти
Рассмотрим подробнее понятие действия кибернетической системы.

\begin{SCn}
\scnheader{действие кибернетической системы}
\scnsubset{действие}
\scnidtf{целенаправленное действие, выполняемое кибернетической системой, а точнее, её решателем задач}
\begin{scnrelfromset}{разбиение}
	\scnitem{внешнее действие кибернетической системы}
	\begin{scnindent}
		\scnidtf{действие, выполняемое кибернетической системой в её внешней среде}
		\scnidtf{поведенческое действие}
	\end{scnindent}

	\scnitem{действие кибернетической системы, выполняемое в собственной физической оболочке}
	\scnitem{действие кибернетической системы, выполняемое в собственной абстрактной памяти}
\end{scnrelfromset}
\end{SCn}

Говоря о действиях кибернетической системы, выполняемых в собственной абстрактной памяти, подразумеваются действия, направленные на преобразование информации, хранимой в памяти, но никак не на преобразование физической памяти (физической оболочки абстрактной памяти).

Каждое \uline{сложное} действие,выполняемое кибернетической системой вне собственный абстрактной памяти, включает в себя поддействия, выполняемые в указанной абстрактной памяти. 
Это означает, что все внешние действия кибернетической системы \uline{управляются} внутренними её действиями (действиями в абстрактной памяти).

% понятие задачи
% типология задач, решаемых кибернетической системой
% понятие навыка

Интерфейс кибернетической системы --- условно выделяемый компонент \textit{решателя задач кибернетической системы}, обеспечивающий решение \textit{интерфейсных задач}, направленных на \uline{непосредственную} реализацию взаимодействия \textit{кибернетической системы} с её \textit{внешней средой}. Решатель интерфейсных задач кибернетической системы. Стоит отличать понятие интерфейса кибернетической системы и понятие физического обеспечения интерфейса кибернетической системы.

Физическая оболочка кибернетической системы --- часть кибернетической системы, являющаяся "посредником"{} между её внутренней средой (памятью, в которой хранится и обрабатывается информация кибернетической системы) и её внешней средой.

\begin{SCn}
	\scnheader{кибернетическая система}
	\begin{scnrelfromset}{обобщенная декомпозиция}
		\scnitem{память кибернетической системы}
		\scnitem{процессор кибернетической системы}
		\scnitem{физическое обеспечение интерфейса кибернетической системы}
		\scnitem{корпус кибернетической системы}
	\end{scnrelfromset}
\end{SCn}

% физическое обеспечение интерфейса кибернетической системы

\begin{SCn}
	\scnheader{физическое обеспечение интерфейса кибернетической системы}
	\begin{scnrelfromset}{обобщенная декомпозиция}
		\scnitem{сенсорная подсистема физической оболочки кибернетической системы}
		\scnitem{эффекторная подсистема физической оболочки кибернетической системы}
	\end{scnrelfromset}
\end{SCn}

Память кибернетической системы --- физическая оболочка (реализация) абстрактной \textit{памяти кибернетической системы}, внутренней среды \textit{кибернетической системы}, в рамках которой \textit{кибернетическая система} формирует и использует (обрабатывает) информационную модель своей внешней среды.

Не каждая \textit{кибернетическая система} имеет \textit{память}. 
В \textit{кибернетических системах}, которые не имеют \textit{памяти}, обработка информации сводится к обмену сигналами между компонентами этих систем. Появление в \textit{кибернетических системах} памяти как среды для "централизованного"{} хранения и обработки \textit{информации} является важнейшим этапом их эволюции. Дальнейшая эволюция \textit{кибернетических систем} во многом определяется:
\begin{itemize}
	\item \textit{качеством памяти} как среды для хранения и обработки информации;
	\item качеством информации (информационной модели), хранимой в памяти кибернетической системы;
\end{itemize}

Компонент кибернетической системы, в рамках которого \textit{кибернетическая система} осуществляет отображение (формирование информационной модели) среды своего существования, а также использование этой информационной модели для управления собственным поведением в указанной среде

Сам факт появления в кибернетической системе памяти, которая (1) обеспечивает представление различного виды информации о среде, в рамках которой кибернетическая система решает различные задачи (выполняет различные действия), (2) обеспечивает хранение достаточно полной информационной модели указанной среды (достаточно полной для реализации своей деятельности), (3) обеспечивает высокую степень гибкости указанной хранимой в памяти информационной модели среды жизнедеятельности (т.е. лёгкость внесения изменений в эту информационную модель), существенно повышает уровень адаптивности кибернетической системы к различным изменениям своей среды.
Появление{} \uline{\textit{памяти}} в кибернетических системах является основным признаком перехода от "простых"{} автоматов к компьютерным системам, от роботов 1-го поколения к роботам следующих поколений.

Принципы организации памяти кибернетической системы могут быть разными (ассоциативная, адресная, структурно фиксированная/структурно перестраиваемая, нелинейная/линейная). 
От организации памяти во многом зависит её качество.

% Уровни структурной эволюции кибернетических систем

Процессором кибернетической системы называют физически реализованный интерпретатор хранимых в памяти кибернетической системы программ, соответствующих базовой модели решения задач для данной кибернетической системы. Такая модель решения задач для данной кибернетической системы является моделью решения задач самого нижнего уровня и не может быть интерпретирована с помощью другой модели решения задач, используемой этой же кибернетической системой. Она может быть проинтерпретирована либо путем аппаратной реализации такого интерпретатора, либо путём его программной реализации, например, на современных компьютерах. В последнем случае, кроме собственного интерпретатора, необходимо также построить модель памяти реализуемой кибернетической системы.

Развитие рынка интеллектуальных компьютерных систем существенно сдерживается неприспособленностью современного поколения компьютеров к реализации на их основе интеллектуальных компьютерных систем. Попытки создания компьютеров, приспособленных к реализации интеллектуальных компьютерных систем, не привели к успеху, т.к. эти проекты были направлены на выполнение частных требований, предъявляемых к аппаратному уровню интеллектуальных систем, что неминуемо приводило к приспособленности создаваемых компьютеров к реализации не всего многообразия интеллектуальных компьютерных систем, а только некоторых подмножеств таких систем. Указанные подмножества интеллектуальных компьютерных систем в основном определялись ориентацией на конкретные используемые модели решения интеллектуальных задач, тогда как важнейшим фактором, определяющим уровень интеллекта кибернетических систем, является их универсальность в плане многообразия используемых моделей решения задач. Следовательно, компьютер для интеллектуальных компьютерных систем должен быть эффективным аппаратным интерпретатором любых моделей решения задач, как интеллектуальных, так и достаточно простых.

Таким образом, выделено следующее семейство отношений, заданных на множестве кибернетических систем.

\begin{SCn}
\scnheader{отношение, заданное на множестве кибернетических систем}
\scnhaselement{память кибернетической системы*}
\scnhaselement{процессор кибернетической системы*}
\scnhaselement{член коллектива*}
\scnhaselement{внешняя среда кибернетической системы*}
\scnhaselement{сенсор кибернетической системы*}
\scnhaselement{эффектор кибернетической системы*}
\scnhaselement{физическая оболочка кибернетической системы*}
\scnhaselement{информация, хранимая в памяти кибернетической системы*}
\scnhaselement{абстрактная память кибернетической системы*}
\scnhaselement{часть*}
\begin{scnindent}
	\scnsuperset{встроенная кибернетическая система*}
\end{scnindent}
\end{SCn}

Понятие \textit{внешней среды кибернетической системы*} является понятием относительным, т.к. (1) разные кибернетические системы могут иметь разную внешнюю среду и (2) одна кибернетическая система может входить в состав внешней среды другой кибернетической системы. 
В общем случае среда жизнедеятельности \textit{кибернетической системы} включает в себя (1) \textit{внешнюю среду*} этой системы, (2) \textit{физическую оболочку*} этой системы и (3) её \textit{абстрактную память}, т.е. внутреннюю среду*, которая является хранилищем информационной модели всей среды


\begin{SCn}

\bigskip

\begin{scnrelfromlist}{ключевое понятие}
    \scnitem{индивидуальная кибернетическая система}
    \scnitem{многоагентная система}
        \begin{scnindent}
        \scnsuperset{кибернетическая система} 
        \end{scnindent}
    \scnitem{иерархическая кибернетическая система}
    \scnitem{агент}
    \scnitem{многоагентная система}
        \begin{scnindent}
        \scnidtf{самостоятельный компонент многоагентной системы}
        \scnidtf{кибернетическая система, являющаяся агентом по крайней мере одной многоагентной системы}
        \end{scnindent}
    \scnitem{агент*}
        \begin{scnindent}
        \scnidtf{быть агентом заданной многоагентной системы*}
        \scnrelfrom{второй домен}{агент}
        \end{scnindent}
    \scnitem{индивидуальный агент}
    \scnitem{коллективный агент}
    \scnitem{встроенный агент}
        \begin{scnindent}
        \scnidtf{внутренний агент индивидуальной кибернетической системы}
        \end{scnindent}
    \scnitem{специализированный агент}
        \begin{scnindent}
        \scnidtf{интеллектуальный агент}
        \end{scnindent}
    \scnitem{мультиагент}
        \begin{scnindent}
        \scnidtf{кибернетическая система, являющаяся агентом более чем одной многоагентной системы}
        \end{scnindent}
\end{scnrelfromlist}

\bigskip
    
\begin{scnrelfromlist}{ключевое знание}
    \scnitem{Классификация многоагентных систем} 
    \scnitem{Обобщенная архитектура кибернетических систем} 
    \scnitem{Факторы, определяющие качество многоагентных систем}
    \scnitem{Факторы, определяющие уровень интеллекта многоагентных систем}
\end{scnrelfromlist}

\bigskip

\begin{scnrelfromlist}{рассматриваемый вопрос}
    \scnitem{Почему переход от отдельных кибернетических систем к их коллективным, многоагентным системам, агентами которых являются заданные кибернетические системы, является одним из направлений повышения уровня интеллекта исходных кибернетических систем.}
    \scnitem{Всегда ли объединение кибернетических систем в коллектив этих систем приводит у повышению уровня интеллекта.}
    \scnitem{Почему не каждое соединение достаточно качественных кибернетических систем порождает качественную многоагентную систему.}
\end{scnrelfromlist}

\bigskip

\begin{scnrelfromlist}{библиографическая ссылка}
    \scnitem{\scncite{Tarasov2002}}
    \scnitem{\scncite{Varshavskiy1984}}
\end{scnrelfromlist}

\bigskip

\end{SCn}

\subsection{Комплекс свойств, определяющих качество многоагентной системы}
{\label{sec_mas_overall_quality}} 

Переход от кибернетических систем к коллективам взаимодействующих между собой кибернетических систем, т.е. к социальной организации кибернетических систем, является важнейшим фактором эволюции кибернетических систем. 
Кибернетическую систему, представляющую собой коллектив взаимодействующих кибернетических систем, обладающих определенной степенью самостоятельности (самодостаточности, свободы выбора), будем называть многоагентной системой.

\begin{SCn}
\scnheader{кибернетическая система}
\begin{scnrelfromset}{разбиение}
    \scnitem{индивидуальная кибернетическая система}
    \scnitem{кибернетическая система, являющаяся минимальным компонентом индивидуальной кибернетической системы}
    \scnitem{кибернетическая система, являющаяся комплексом компонентов соответствующей индивидуальной кибернетической системы}
    \scnitem{сообщество индивидуальных кибернетических систем}
    \begin{scnindent}
    \begin{scnrelfromset}{разбиение}
        \scnitem{простое сообщество индивидуальных кибернетических систем}
        \scnitem{иерархическое сообщество индивидуальных кибернетических систем}
    \end{scnrelfromset}
    \end{scnindent}
\end{scnrelfromset}
\end{SCn}

Агенты многоагентной системы могут (но вовсе не обязательно должны) быть интеллектуальными системами. 
Так, например, агенты интеллектуального решателя задач, имеющего агентно-ориентированную архитектуру, не являются интеллектуальными системами. 
Агентом иерархической многоагентной системы может быть другая многоагентная система.

В многоагентной системе с централизованным управлением специально выделяются агенты, которые принимают решения в определенной области деятельности многоагентной системы и обеспечивают выполнение этих решений путем управления деятельностью остальных агентов, входящих в состав этой системы.

В многоагентной системе с децентрализованным управлением решения принимаются коллегиально и "автоматически"{} (решения о признании новой кем-то предложенной информации – в том числе, об инициировании некоторой задачи, решения о коррекции (уточнении) уже ранее признанной, согласованной информации) на основе четко продуманной и постоянно совершенствуемой методики, а также на основе активного участия всех агентов в формировании новых предложений, подлежащих признанию или согласованию. 
В такой многоагентной системе все агенты участвуют в управлении этой системы. 
Примером такой системы является оркестр, способный играть без дирижера.

Переход к многоагентным системам является важнейшим фактором повышения качества (и, в частности, уровня интеллекта) кибернетических систем, т.к. уровень интеллекта многоагентной системы может быть значительно выше уровня интеллекта каждого входящего в неё агента. 
Это происходит далеко не всегда, поскольку важнейшим фактором качества многоагентных систем является не только качество входящих в неё агентов, но и организация взаимодействия агентов и, в частности, переход от централизованного к децентрализованному управлению. 
Количество не всегда переходит в новое качество.

Качество индивидуальных кибернетических систем определяется, кроме всего прочего тем, насколько большой вклад индивидуальная кибернетическая система вносит в повышение качества тех коллективов, в состав которых она входит.
Указанное свойство индивидуальных кибернетических систем будем называть уровнем их интероперабельности \cite{Ouksel1999interoperability}.

Синергетическая кибернетическая система есть многоагентная система, обладающая высоким уровнем коллективного интеллекта, атомарными агентами которой являются индивидуальные интеллектуальные системы, имеющие высокий уровень интероперабельности \cite{Lopes2022semantic} \cite{Hamilton2006Interoperability}.
Примером синергетической кибернетической системы является творческий коллектив, реализующий сложный наукоемкий проект.

Эффективность творческого коллектива (например в области научно-технической деятельности) определяется:
\begin{itemize}
    \item{согласованностью мотивации, целевой установки всего коллектива и каждого его члена (не должно быть противоречий между целью коллектива и творческой самореализацией каждого его члена);}
    \item{эффективной организацией децентрализованного управления деятельностью членов сообщества;}
    \item{четкой, оперативной и доступной всем фиксацией документации текущего состояния содеянного и направлений его дальнейшего развития;}
    \item{уровнем трудоемкости оперативности фиксации индивидуальных результатов в рамках коллективно создаваемого общего результата;}
    \item{уровнем структурированности и, прежде всего, стратифицированности обобщенной документации (базы знаний);}
    \item{эффективностью ассоциативного доступа к фрагментам документации;}
    \item{гибкостью коллективно создаваемой базы;}
    \item{автоматизацией анализа содеянного и управления проектом.}
\end{itemize}

Уровень интеллекта многоагентной системы может быть значительно ниже уровня интеллекта самого "глупого"{} члена этого коллектива, но может быть и значительно выше уровня интеллекта самого "умного"{} члена указанного коллектива.
Для того, чтобы количество интеллектуальных систем переходило в существенно более интеллектуальное качество коллектива таких систем, все объединяемые в коллектив интеллектуальные системы должны иметь высокий уровень интероперабельности, что накладывает дополнительные требования, предъявляемые к информации, хранимой в памяти, а также к решателям задач интеллектуальных систем, объединяемых в коллектив.

Интероперабельность кибернетической системы есть способность кибернетической системы взаимодействовать с другими кибернетическими системами в целях создания коллектива кибернетических систем (многоагентных систем), уровень качества и, в частности, уровень интеллекта которого выше уровня качества каждой кибернетической системы, входящей в состав этого коллектива.

Для того, чтобы количество членов коллектива кибернетической системы перешло в более высокое качество самого коллектива, члены коллектива должны обладать дополнительными способностями, которые будем называть свойствами интероперабельности.
Основными такими свойствами являются способность устанавливать и поддерживать достаточный уровень семантической совместимости (взаимопонимания) с другими кибернетическими системами и договороспособность (способность согласовывать свои действия с другими) \cite{NEIVA2016Interoperability}.

Целенаправленный обмен информацией между кибернетическими системами существенно ускоряет процесс их обучения (процесс накопления знаний и навыков).
Следовательно, способность эффективно использовать указанный канал накопления знаний и навыков существенно повышает уровень обучаемости кибернетических систем.
Повышение уровня интероперабельности кибернетической системы является, с одной стороны, дополнительным повышением уровня интеллекта самой этой кибернетической системы, а также фактором повышения уровня интеллекта тех коллективов, тех многоагентных систем, в состав которых эта кибернетическая система входит.

\begin{SCn}
\scnheader{интероперабельность кибернетической системы}
\begin{scnrelfromlist}{cвойство-предпосылка}
    \scnitem{договороспособность кибернетической системы}
    \scnitem{социальная ответственность кибернетической системы}
    \scnitem{социальная активность кибернетической системы}
\end{scnrelfromlist}
\end{SCn}

Свойства-предпосылки уровня договороспособности кибернетической системы представлены ниже:

\begin{SCn}
\scnheader{договороспособность кибернетической системы}
\begin{scnrelfromlist}{cвойство-предпосылка}
    \scnitem{способность кибернетической системы к пониманию принимаемых сообщений}
    \scnitem{способность кибернетической системы к формированию передаваемых сообщений, понятных адресатам}
    \scnitem{способность кибернетической системы к обеспечению семантической совместимости с партнёрами}
    \scnitem{коммуникабельность кибернетической системы}
    \scnitem{способность кибернетической системы к обсуждению и согласованию целей и планов коллективной деятельности}
    \scnitem{способность кибернетической системы брать на себя выполнение актуальных задач в рамках согласованных планов коллективной деятельности}
\end{scnrelfromlist}
\end{SCn}

Понимание информации, поступающей извне, включает в себя:
\begin{itemize}
    \item{перевод этой информации на внутренний язык кибернетической системы;}
    \item{локальную верификацию вводимой информации;}
    \item{погружение (конвергенцию, размещение) текста, являющегося результатом указанного перевода в состав хранимой информации (в частности, в состав базы знаний).}
\end{itemize}

Погружение вводимой информации в состав базы знаний кибернетической системы сводится к выявлению и устранению противоречий, возникающих между погружаемым текстом и текущего состояния базы знаний.
Сложность проблемы понимания вводимой вербальной информации заключается не только в сложности непротиворечивого погружения вводимой информации в текущее состояние базы знаний, но и в сложности трансляции этой информации с внешнего языка на внутренний язык кибернетической системы, т. е. в сложности генерации текста внутреннего языка, семантически эквивалентного вводимому тексту внешнего языка.
Для естественных языков указанная трансляция является сложной задачей, так как в настоящее время проблема формализации синтаксиса и семантики естественных языков не решена.

Семантическая совместимость двух заданных кибернетических систем определяется согласованностью систем понятий, используемых обеими взаимодействующими кибернетическими системами.
Проблема обеспечения перманентной поддержки семантической совместимости взаимодействующих кибернетических систем является необходимым условием обеспечения высокого уровня взаимопонимания кибернетических систем и, как следствие, эффективного их взаимодействия.

Коммуникабельность кибернетической системы есть способность кибернетической системы к установлению взаимовыгодных контактов с другими кибернетическими системами (в том числе, с коллективами интеллектуальных систем) путем честного выявления взаимовыгодных общих целей (интересов).

Свойства-предпосылки уровня социальной ответственности кибернетической системы представлены ниже:

\begin{SCn}
\scnheader{социальная ответственность кибернетической системы}
\begin{scnrelfromlist}{cвойство-предпосылка}
    \scnitem{способность кибернетической системы выполнять качественно и в срок взятые на себя обязательства в рамках соответствующих коллективов}
    \scnitem{способность кибернетической системы адекватно оценивать свои возможности при распределении коллективной деятельности}
    \scnitem{альтруизм/эгоизм кибернетической системы}
    \scnitem{отсутствие/наличие действий, которые по безграмотности кибернетической системы снижают качество коллективов, в состав которых она входит}
    \scnitem{отсутствие/наличие "осознанных"{}, мотивированных действий, снижающих качество коллективов, в состав которых кибернетическая система входит}
\end{scnrelfromlist}
\end{SCn}

Свойства-предпосылки уровня социальной активности кибернетической системы представлены ниже:

\begin{SCn}
\scnheader{социальная активность кибернетической системы}
\begin{scnrelfromlist}{cвойство-предпосылка}
    \scnitem{способность кибернетической системы к генерации предлагаемых целей и планов коллективной деятельности}
    \scnitem{активность кибернетической системы в экспертизе результатов других участников коллективной деятельности}
    \scnitem{способность кибернетической системы к анализу качества всех коллективов, в состав которых она входит, а также всех членов этих коллективов}
    \scnitem{способность кибернетической системы к участию в формировании новых коллективов}
    \scnitem{количество и качество тех коллективов, в состав которых кибернетическая система входит или входила}
\end{scnrelfromlist}
\end{SCn}

Формирование специализированного коллектива кибернетических систем сводится к тому, что в памяти каждой кибернетической системы, входящей в коллектив, генерируется спецификация этого коллектива, включающая в себя:
\begin{itemize}
    \item{перечень весь членов коллектива;}
    \item{способности каждого из членов коллектива;}
    \item{их обязанности в рамках коллектива;}
    \item{спецификацию всего множества задач (вида деятельности), для решения (выполнения) которых сформирован данный коллектив кибернетических систем.}
\end{itemize}

Каждая кибернетическая система может входить в состав большого количества коллективов, выполняя при этом в разных коллективах в общем случае разные "должностные обязанности"{}, разные "бизнес-процессы"{}.


\section{Эволюция традиционных и интеллектуальных компьютерных систем}
{\label{sec_comp_syst}} 

\begin{SCn}

\bigskip

\begin{scnrelfromlist}{ключевое понятие}
    \scnitem{компьютерная система} 
    \scnitem{память компьютерной системы}
    \scnitem{информационный процесс в памяти компьютерной системы}
    \scnitem{компьютер}
    \scnitem{интерфейс компьютерной системы}
    \scnitem{пользовательский интерфейс компьютерной системы}
    \scnitem{традиционная компьютерная система}
    \scnitem{интеллектуальная компьютерная система}
    \scnitem{база знаний}
    \scnitem{интеллектуальный решатель задач}
\end{scnrelfromlist}

\bigskip

\begin{scnrelfromlist}{ключевое знание}
    \scnitem{Классификация компьютерных систем}
    \scnitem{Обобщённая архитектура компьютерных систем}
    \scnitem{Классификация традиционных компьютерных систем}
    \scnitem{Эволюция традиционных компьютерных систем}
    \scnitem{Эволюция языков программирования}
    \scnitem{Эволюция систем программирования}
    \scnitem{Классификация интеллектуальных компьютерных систем}
    \scnitem{Обобщённая архитектура интеллектуальных компьютерных систем}
    \scnitem{Отличия интеллектуальных компьютерных систем от традиционных компьютерных систем}
    \scnitem{Эволюция интеллектуальных компьютерных систем}
\end{scnrelfromlist}

\bigskip

\begin{scnrelfromlist}{библиографическая ссылка}
	\scnitem{\scncite{Cho2019}}
	\scnitem{\scncite{Sherif1988}}
	\scnitem{\scncite{Laird2009}}
	\scnitem{\scncite{Gao2002}}
\end{scnrelfromlist}

\bigskip

\end{SCn}

Были предложены различные системные показатели для измерения \textit{качества компьютерных систем}. Поскольку компьютерные системы становятся все более сложными и включают множество подсистем или компонентов, измерение их качества в нескольких измерениях становится сложной задачей (см. \scncite{Cho2019}). Метрики качества включают в себя набор мер, которые могут описывать атрибуты системы в терминах, не зависящих от структуры, которая приводит к этим атрибутам; эти меры должны быть выражены количественно и должны иметь значительный уровень точности и надежности (см. \scncite{Sherif1988}).

Целью является построение \textit{компьютерных систем}, которые демонстрируют весь спектр когнитивных способностей, которые мы обнаруживаем у людей (см. \scncite{Laird2009}).

Исследователи \textit{Искусственного интеллекта} определяют \textit{интеллект} как неотъемлемое свойство машины (см. \scncite{Gao2002}).

% \input{author/references}