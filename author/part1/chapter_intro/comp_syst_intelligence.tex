\section{Эволюция традиционных и интеллектуальных компьютерных систем}
{\label{sec_comp_syst}} 

\begin{SCn}

\bigskip

\begin{scnrelfromlist}{ключевое понятие}
    \scnitem{компьютерная система} 
    \scnitem{память компьютерной системы}
    \scnitem{информационный процесс в памяти компьютерной системы}
    \scnitem{компьютер}
    \scnitem{интерфейс компьютерной системы}
    \scnitem{пользовательский интерфейс компьютерной системы}
    \scnitem{традиционная компьютерная система}
    \scnitem{интеллектуальная компьютерная система}
    \scnitem{база знаний}
    \scnitem{интеллектуальный решатель задач}
\end{scnrelfromlist}

\bigskip

\begin{scnrelfromlist}{ключевое знание}
    \scnitem{Классификация компьютерных систем}
    \scnitem{Обобщённая архитектура компьютерных систем}
    \scnitem{Классификация традиционных компьютерных систем}
    \scnitem{Эволюция традиционных компьютерных систем}
    \scnitem{Эволюция языков программирования}
    \scnitem{Эволюция систем программирования}
    \scnitem{Классификация интеллектуальных компьютерных систем}
    \scnitem{Обобщённая архитектура интеллектуальных компьютерных систем}
    \scnitem{Отличия интеллектуальных компьютерных систем от традиционных компьютерных систем}
    \scnitem{Эволюция интеллектуальных компьютерных систем}
\end{scnrelfromlist}

\bigskip

\begin{scnrelfromlist}{библиографическая ссылка}
	\scnitem{\scncite{Cho2019}}
	\scnitem{\scncite{Sherif1988}}
	\scnitem{\scncite{Laird2009}}
	\scnitem{\scncite{Gao2002}}
\end{scnrelfromlist}

\bigskip

\end{SCn}

Были предложены различные системные показатели для измерения \textit{качества компьютерных систем}. Поскольку компьютерные системы становятся все более сложными и включают множество подсистем или компонентов, измерение их качества в нескольких измерениях становится сложной задачей (см. \scncite{Cho2019}). Метрики качества включают в себя набор мер, которые могут описывать атрибуты системы в терминах, не зависящих от структуры, которая приводит к этим атрибутам. Эти меры должны быть выражены количественно и должны иметь значительный уровень точности и надежности (см. \scncite{Sherif1988}).

Целью является построение \textit{компьютерных систем}, которые демонстрируют весь спектр когнитивных способностей, которые мы обнаруживаем у людей (см. \scncite{Laird2009}).

Исследователи \textit{Искусственного интеллекта} определяют \textit{интеллект} как неотъемлемое свойство машины (см. \scncite{Gao2002}).