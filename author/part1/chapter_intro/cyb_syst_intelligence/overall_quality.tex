\subsection{Комплекс свойств, определяющий общий уровень качества кибернетической системы}
{\label{sec_cyb_syst_overall_quality}} 

Качество кибернетической системы есть интегральная, комплексная оценка уровня развития кибернетической системы. 
Это свойство, характеристика кибернетических систем, признак их классификации, который позволяет разместить эти системы по "ступенькам"{} некоторой условной "эволюционной лестницы"{}.
На каждую такую "ступеньку"{} попадают кибернетические системы, имеющие одинаковый уровень развития, каждому их которых соответствует свой набор значений дополнительных свойств кибернетических систем, которые уточняют соответствующий уровень развития кибернетических систем. 
Такой эволюционный подход к рассмотрению кибернетических систем даёт возможность, во-первых, детализировать направления эволюции кибернетических систем и, во-вторых, уточнить то место этой эволюции, где и благодаря чему осуществляется переход от неинтеллектуальных кибернетических систем к интеллектуальным.

В основе эволюционного подхода к рассмотрению многообразия кибернетических систем лежит положение о том, что идеальных кибернетических систем не существует, но существует постоянное стремление к идеалу, к большему совершенству. 
Важно уточнить, что конкретно в каждой кибернетической системе следует изменить, чтобы привести эту систему к более совершенному виду.
Так, например, развитие технологий разработки компьютерных систем должно быть направлено на переход к таким новым архитектурным и функциональным принципам, лежащим в основе компьютерных систем, которые обеспечивают существенное снижение трудоемкости их разработки и сокращение сроков разработки, а также обеспечивают существенное повышение уровня интеллекта и, в частности, уровня обучаемости разрабатываемых компьютерных систем, например, путём перехода от поддержки обучения с учителем к реализации эффективного самообучения (к автоматизации организации самостоятельного обучения). 

Проблема выделения критериев интеллектуальности систем рассмотрена в ряде работ \scncite{Finn2021}, \scncite{Nilsson2005}, \scncite{Kerr2006}, \scncite{Antsyferov2013}.
Тем не менее, для практической реализации интеллектуальных компьютерных систем необходимо детализировать и уточнить эти свойства, пытаясь свести их к более конструктивным, прозрачным и понятным для реализации свойствам.

Для детализации понятия качества кибернетической системы необходимо задать метрику качества кибернетических систем и построить иерархическую систему свойств, параметров, признаков, определяющих качество кибернетических систем.

\begin{SCn}
\scnheader{качество кибернетической системы}
\begin{scnrelfromlist}{cвойство-предпосылка}
    \scnitem{качество физической оболочки кибернетической системы}
    \scnitem{качество информации, хранимой в памяти кибернетической системы}
    \scnitem{качество решателя задач кибернетической системы}
    \scnitem{гибридность кибернетической системы}
    \begin{scnrelfromset}{частное свойство}
        \scnitem{многообразие видов знаний, хранимых в памяти кибернетической системы}
        \scnitem{многообразие моделей решения задач}
        \scnitem{многообразие видов сенсоров и эффекторов}
    \end{scnrelfromset}
    \scnitem{приспособленность кибернетической системы к её совершенствованию}
    \scnitem{производительность кибернетической системы}
    \scnitem{надежность кибернетической системы}
    \scnitem{интероперабельность кибернетической системы}
\end{scnrelfromlist}
\end{SCn}


\subsection{Комплекс свойств, определяющих качество физической оболочки кибернетической системы}
{\label{sec_cyb_syst_physical_shell_quality}} 

Качество физической оболочки кибернетической системы есть интегральное качество физической, аппаратной основы кибернетической системы.
Выделенное множество свойств, определяющих качество физической оболочки кибернетической системы, приведена ниже:

\begin{SCn}
\scnheader{качество физической оболочки кибернетической системы}
\begin{scnrelfromlist}{cвойство-предпосылка}
    \scnitem{качество памяти кибернетической системы}
    \scnitem{качество процессора кибернетической системы}
    \scnitem{качество сенсоров кибернетической системы}
    \scnitem{качество эффекторов кибернетической системы}
    \scnitem{приспособленность физической оболочки кибернетической системы к ее совершенствованию}
    \scnitem{удобство транспортировки кибернетической системы}
    \scnitem{надежность физической оболочки кибернетической системы}
\end{scnrelfromlist}
\end{SCn}

Важно, чтобы память обеспечивала высокий уровень гибкости указанной информационной модели.
Важно также, чтобы эта информационная модель была моделью не только внешней среды кибернетической системы, но также и моделью самой этой информационной модели – описанием её текущей ситуации, предыстории, закономерностей.

\begin{SCn}
\scnheader{качество памяти кибернетической системы}
\begin{scnrelfromlist}{cвойство-предпосылка}
    \scnitem{способность памяти кибернетической системы обеспечить хранение высококачественной информации}
    \scnitem{способность памяти кибернетической системы обеспечить функционирование высококачественного решателя задач}
    \scnitem{объём памяти}
\end{scnrelfromlist}
\end{SCn}

Факт возникновения памяти в кибернетической системе является важнейшим этапом её эволюции.
Дальнейшее развитие памяти кибернетической системы, обеспечивающее хранение все более качественной информации, хранимой в памяти и все более качественную организацию обработки этой информации, т.е. переход на поддержку все более качественных моделей обработки информации, является важнейшим фактором эволюции кибернетических систем.

Способность памяти кибернетической системы обеспечить функционирование высококачественного решателя задач основывается на качестве доступа к информации, хранимой в памяти кибернетической системы, логико-семантической гибкости памяти кибернетической системы, способности памяти кибернетической системы обеспечить интерпретацию широкого многообразия моделей решения задач.

Качество процессора кибернетической системы определяется его способностью обеспечить функционирования высококачественного решателя задач.

\begin{SCn}
\scnheader{способность процессора кибернетической системы обеспечить функционирования
высококачественного решателя задач}
\begin{scnrelfromlist}{cвойство-предпосылка}
    \scnitem{многообразие моделей решения задач, интерпретируемых процессором
кибернетической системы}
    \scnitem{простота и качество интерпретации процессором системы широкого многообразия моделей решения задач}
    \scnitem{обеспечение процессором кибернетической системы качественного управления
информационными процессами в памяти}
    \scnitem{быстродействие процессора кибернетической системы}
\end{scnrelfromlist}
\end{SCn}

Максимальным уровнем качества процессора кибернетической системы по параметру многообразия моделей решения задач, интерпретируемых процессором кибернетической системы, является его универсальность, т.е. его принципиальная возможность интерпретировать любую модель решения как интеллектуальных, так и неинтеллектуальных задач. 
Простота определяется степенью близости интерпретируемых моделей решения задач к “физическому” уровню организации процессора кибернетической системы. 
Качественное управление информационными процессами в памяти подразумевает грамотное сочетание таких аспектов управление процессами, как централизация и децентрализация \scncite{Melekhova2018}, синхронность и асинхронность, последовательность и параллельность.

Качество сенсоров и эффекторов кибернетической системы сводится к многообразию видов сенсоров и эффекторов кибернетической системы, т.е. к многообразию средств восприятия и воздействия на информацию о текущем состоянии внешней среды и собственной физической оболочки.
Приспособленность физической оболочки кибернетической системы к её совершенствованию определяется гибкостью и стратифицированностью физической оболочки кибернетической системы.



\subsection{Комплекс свойств, определяющих качество информации, хранимой в памяти кибернетической системы}
{\label{sec_cyb_syst_information_quality}} 

Качество информационной модели среды "обитания"{} кибернетической системы, в частности, определяется:
\begin{textitemize}
    \item корректностью этой модели, отсутствием в ней ошибок;
    \item адекватностью этой модели;
    \item полнотой, достаточностью находящейся в ней информации для эффективного функционирования кибернетической системы;
    \item структурированностью, систематизированностью.
\end{textitemize}

Важнейшим этапом эволюции информационной модели среды кибернетической системы является переход от недостаточно полной и несистематизированной информационные модели среды к базе знаний.

\begin{SCn}
\scnheader{качество информации, хранимой в памяти кибернетической системы}
\begin{scnrelfromlist}{cвойство-предпосылка}
    \scnitem{семантическая мощность языка представления информации в памяти кибернетической системы}
    \scnitem{объём информации, загруженной в память кибернетической системы}
    \scnitem{степень конвергенции и интеграции различного вида знаний, хранимых в памяти кибернетической
системы}
    \scnitem{стратифицированность информации, хранимой в памяти кибернетической системы}
    \scnitem{простота и локальность выполнения семантически целостных операций над информацией,
хранимой в памяти кибернетической системы}
    \scnitem{корректность/некорректность информации, хранимой в памяти кибернетической системы}
    \scnitem{однозначность/неоднозначность информации, хранимой в памяти кибернетической системы}
    \scnitem{целостность/нецелостность информации, хранимой в памяти кибернетической системы}
    \scnitem{чистота/загрязненность информации, хранимой в памяти кибернетической системы}
    \scnitem{достоверность/недостоверность информации, хранимой в памяти кибернетической системы}
    \scnitem{точность/неточность информации, хранимой в памяти кибернетической системы}
    \scnitem{четкость/нечеткость информации, хранимой в памяти кибернетической системы}
    \scnitem{определенность/недоопределенность информации, хранимой в памяти кибернетической системы}
\end{scnrelfromlist}
\end{SCn}

Корректность/некорректность информации есть уровень адекватности хранимой информации той среды, в которой существует кибернетическая система и информационной моделью которой эта хранимая информация является. 
Непротиворечивость/противоречивость информации означает уровень присутствия в хранимой информации различного вида противоречий и, в частности, ошибок. 
Ошибки в хранимой информации могут быть синтаксическими и семантическими, противоречащими некоторым правилам, которые явно в памяти могут быть не представлены и считаются априори истинными.

Полнота/неполнота информации --- уровень того, насколько информация, хранимая в памяти кибернетической системы, описывает среду существования этой системы и используемые ею методы решения задач достаточно полно для того, чтобы кибернетическая система могла действительно решать все множество соответствующих ей задач. 
Чем полнее информация, хранимая в памяти кибернетической системы, чем полнее информационное обеспечение деятельности этой системы это системы, тем эффективнее (качественнее) сама эта деятельность. 
Полнота определяется структурированностью информации и многообразием видов знаний, хранимых в памяти кибернетической системы.

Однозначность/неоднозначность информации определяется многообразием форм дублирования информации и частотой дублирования информации.

Целостность/нецелостность информации есть уровень содержательной информативности информации, уровень того, насколько семантически связной является информация, насколько полно специфицированы все описываемые в памяти сущности (путём описания необходимого набора связей этих сущностей с другими описываемыми сущностями), насколько редко или часто в рамках хранимой информации встречаются информационные дыры, соответствующие явной недостаточности некоторых спецификаций. 
Примерами информационных дыр являются:
\begin{textitemize}
    \item отсутствующий метод решения часто встречающихся задач;
    \item отсутствующее определение используемого определяемого понятия;
    \item недостаточно подробная спецификация часто рассматриваемых сущностей.
\end{textitemize}

Чистота/загрязненность информации означает многообразие форм и общее количество информационного мусора, входящего в состав информации, хранимой в памяти кибернетической системы.
Под информационным мусором понимается информационный фрагмент, входящий в состав информации, удаление которого существенно не усложнит деятельность кибернетической системы.
Примерами информационного мусора являются:
\begin{textitemize}
    \item информация, которая нечасто востребована, но при необходимости может быть легко логически выведена;
    \item информация, актуальность которой истекла.
\end{textitemize}

Семантическая мощность языка представления информации в памяти кибернетической системы определяется гибридностью информации, хранимой в памяти кибернетической системы.
Язык, информационные конструкции которого могут представить любую конфигурацию любых связей между любыми сущностями, является универсальным языком.
Универсальность внутреннего языка кибернетической системы является важнейшим фактором её интеллектуальности.

Гибридность информации, хранимой в памяти кибернетической системы определяется многообразием видов знаний и степенью конвергенции и интеграции различного вида знаний.

Стратифицированность информации есть способность кибернетической системы выделять такие разделы информации, хранимой в памяти этой системы, которые бы ограничивали области действия агентов решателя задач кибернетической системы, являющиеся достаточными для решения заданных задач.
Стратифицированность определяется структурированностью и рефлексивностью информации, хранимой в памяти кибернетической системы.
Рефлексивность информации, хранимой в памяти кибернетической системы, т.е. наличие метаязыковых средств, является фактором, обеспечивающим не только структуризацию хранимой информации, но возможность описания синтаксиса и семантики самых различных языков, используемых кибернетической системой.

База знаний является примером информации, хранимой в памяти кибернетической системы и имеющей высокий уровень качества по всем показателям и, в частности, высокий уровень:
\begin{textitemize}
    \item семантической мощности языка представления информации хранимой в памяти кибернетической
системы;
    \item гибридности информации, хранимой в памяти кибернетической системы;
    \item многообразия видов знаний, хранимых в памяти кибернетической системы;
    \item формализованности информации, хранимой в памяти кибернетической системы;
    \item структурированности информации, хранимой в памяти кибернетической системы;
\end{textitemize}

Переход информации, хранимой в памяти кибернетической системы на уровень качества, соответствующий базам знаний, является важнейшим этапом эволюции кибернетических систем.


\subsection{Комплекс свойств, определяющих качество решателя задач кибернетической системы}
{\label{sec_cyb_syst_problem_solver_quality}} 

Качество решателя задач кибернетической системы --- интегральная качественная оценка множества задач, которые кибернетическая система способна выполнять в заданный момент.
Основным свойством и назначением решателя задач кибернетической системы является способность решать задачи на основе накапливаемых, приобретаемых кибернетической системой различного вида навыков с использованием процессора кибернетической системы, являющегося универсальным интерпретатором всевозможных накопленных навыков. 
При этом качество указанной способности определяется целым рядом дополнительных факторов.

\begin{SCn}
\scnheader{общая характеристика решателя задач кибернетической системы}
\begin{scnrelfromset}{cвойство-предпосылка}
    \scnitem{общий объем задач, решаемых кибернетической системой}
    \scnitem{многообразие видов задач, решаемых кибернетической системой}
    \scnitem{способность кибернетической системы к анализу решаемых задач}
    \scnitem{способность кибернетической системы к решению задач, методы решения которых в текущий момент известны}
    \scnitem{способность кибернетической системы к решению задач, методы решения которых ей в текущий момент не известны}
    \scnitem{множество навыков, используемых кибернетической системой}
    \scnitem{степень конвергенции и интеграции различного вида моделей решения задач, используемых кибернетической системой}
    \scnitem{качество организации взаимодействия процессов решения задач в кибернетической системе}
    \scnitem{быстродействие решателя задач кибернетической системы}
    \scnitem{способность кибернетической системы решать задачи, предполагающие использование информации, обладающей различного рода не-факторами}
    \scnitem{многообразие и качество решения задач информационного поиска}
    \scnitem{способность кибернетической системы генерировать ответы на вопросы различного вида в случае, если они целиком или частично отсутствуют в текущем состоянии информации, хранимой в памяти}
    \scnitem{способность кибернетической системы к рассуждениям различного вида}
    \scnitem{качество целеполагания}
    \scnitem{качество реализации планов собственных действий}
    \scnitem{способность кибернетической системы к локализации такой области информации,хранимой в ее памяти, которой достаточно для обеспечения решения заданной задачи}
    \scnitem{способность кибернетической системы к выявлению существенного в информации, хранимой в ее памяти}
    \scnitem{активность кибернетической системы}
\end{scnrelfromset}
\end{SCn}

Общий объем задач, решаемых кибернетической системой, определяется мощностью языка представления задач, решаемых кибернетической системой.
Мощность языка представления задач прежде всего определяется многообразием видов представляемых задач (многообразием видов описываемых действий). 
Каждая задача есть спецификация соответствующего (описываемого) действия. 
Поэтому рассмотрение многообразия видов задач, решаемых кибернетической системой, полностью соответствует многообразию видов деятельности, осуществляемой этой системой. 
Важно заметить, что есть виды деятельности кибернетической системы, которые определяют качество и, в частности, уровень интеллекта кибернетической системы.

Способность кибернетической системы к анализу решаемых задач предполагает оценку задачи на предмет:
\begin{textitemize}
    \item сложности достижения;
    \item целесообразности достижения (нужности, важности, приоритетности);
    \item соответствия цели существующим нормам (правилам) соответствующей деятельности.
\end{textitemize}


Метод решения задач – это вид знаний, хранимых в памяти кибернетической системы и содержащих информацию, которой достаточно либо для сведения каждой задачи из соответствующего класса задач к полной системе подзадач, решение которых гарантирует решение исходной задачи, либо для окончательного решения этой задачи из указанного класса задач. 
Методами для решения задач могут быть не только алгоритмы, но также и функциональные программы, продукционные системы, логические исчисления, генетические алгоритмы, искусственные нейронные сети различного вида.
Задачи, для которых не находятся соответствующие им методы, решаются с помощью метаметодов (стратегий) решения задач, направленных:
\begin{textitemize}
    \item на генерацию нужных исходных данных (нужного контекста), необходимых для решения каждой задачи;
    \item на генерацию плана решения задачи, описывающего сведение исходной задачи к подзадачам (до тех подзадач, методы решения которых системы известны);
    \item на сужение области решения задачи (на сужения контекста задачи, достаточного для ее решения).
\end{textitemize}

Качество решения каждой задачи определяется:
\begin{textitemize}
    \item временем её решения (чем быстрее задача решается, тем выше качество её решения);
    \item полнотой и корректностью результата решения задачи;
    \item затраченными для решения задачи ресурсами памяти (объемом фрагмента хранимой информации, используемой для решения задачи);
    \item затраченным для решения задачи ресурсами решателя задач (количеством используемых внутренних агентов).
\end{textitemize}

Таким образом, повышение качества процесса решения каждой конкретной задачи, а также каждого класса задач (путем совершенствования соответствующего метода, в частности, алгоритма) является важным фактором повышения качества решателя задач в целом.

Перспективным вариантом построения решателя задач кибернетической системы является реализация агентно-ориентированной модели обработки информации, т.е. построение решателя задач в виде многоагентной системы, агенты которой осуществляют обработку информации, хранимой в памяти кибернетической системы, и управляются этой информацией (точнее, её текущим состоянием). 
Особое место среди этих агентов занимают сенсорные (рецепторные) и эффекторные агенты, которые, соответственно, воспринимают информацию о текущем состоянии внешней среды и воздействуют на внешнюю среду, в частности, путем изменения состояния физической оболочки кибернетической системы.

Указанная агентно-ориентированная модель организации взаимодействия процессов решения задач в кибернетической системе по сути есть не что иное, как модель ситуационного управления процессами решения задач, решаемых кибернетической системой как в своей внешней среде, так и в своей памяти.

Быстродействие решателя задач кибернетической системы сводится ко скорости решения задач, быстродействию решателя задач, скорости реакции кибернетической системы на различные задачные ситуации.
Во многом свойство определяется быстродействием процессора кибернетической системы.

Примерами задач, предполагающих использование информации, обладающей различного рода не-факторами, являются задачи проектирования, распознавания, целеполагания, прогнозирования, и т.д. Зачастую это:
\begin{textitemize}
    \item нечетко сформулированые задачи ("делай то, не знаю что");
    \item задачи, которые решаются в условиях неполноты, неточности, противоречивости исходных данных;
    \item задачи, принадлежащие классам задач, для которых практически невозможно построить соответствующие алгоритмы.
\end{textitemize}
Для таких задач характерны:
\begin{textitemize}
    \item неточность и недостоверность исходных данных;
    \item отсутствие критерия качества результата;
    \item невозможность или высокая трудоемкость разработки алгоритма;
    \item необходимость учета контекста задачи.
\end{textitemize}

Способность кибернетической системы генерировать (порождать, строить, синтезировать, выводить) ответы на самые различные вопросы и, в частности, на вопросы типа “что это такое”{}, на почему-вопросы, означает способность кибернетической системы объяснять (обосновывать корректность) своих действий.

Самостоятельность целеполагания есть способность кибернетической системы генерировать, инициировать и решать задачи, которые не являются подзадачами, инициированными внешними (другими) субъектами, а также способность на основе анализа своих возможностей отказаться от выполнения задачи, инициированной извне, переадресовав её другой кибернетической системе, либо на основе анализа самой этой задачи обосновать её нецелесообразность или некорректность. 
Повышение уровня самостоятельности существенно расширяет возможности кибернетической системы, т.е. объем тех задач, которые она может решать не только в "идеальных"{} условиях, но и в реальных, осложненных обстоятельствах.
Способность системы адекватно расставлять приоритеты своим целям и не "распыляться"{} на достижение неприоритетных (несущественных) целей есть способность анализа целесообразности деятельности.

Способность кибернетической системы к выявлению существенного в информации, хранимой в ее памяти есть способность к выявлению (обнаружению, выделению) таких фрагментов информации, хранимой в памяти кибернетической системы, которые существенны (важны) для достижения соответствующих целей.
Понятие существенного (важного) фрагмента информации, хранимой в памяти кибернетической системы, относительно и определяется соответствующей задачей.
Тем не менее, есть важные перманентно решаемые задачи, в частности задачи анализа качества информации, хранимой в памяти кибернетической системы.
Существенные фрагменты хранимой информации, выделяемые в процессе решения этих задач, являются относительными не столько по отношению к решаемой задаче, сколько по отношению к текущему состоянию хранимой информации.

Уровень активности кибернетической системы может быть разным для разных решаемых задач, для разных классов выполняемых действий, для разных видов деятельности. 
Чем выше активность кибернетической системы, тем (при прочих равных условиях) она больше успевает сделать, следовательно, тем выше ее качество (эффективность).
Обратным свойством является понятие пассивности кибернетической системы. 


\subsection{Комплекс свойств, определяющих уровень обучаемости кибернетической системы}
{\label{sec_cyb_syst_learnability_quality}} 

Обучаемость кибернетической системы есть способность кибернетической системы повышать своё качество, адаптируясь к решению новых задач, качество внутренней информации модели своей среды, качество своего решателя задач и даже качество своей физической оболочки. 
Способность кибернетической системы к совершенствованию (к эволюции, к повышению уровня своего качества), к самосовершенствованию с различной степенью самостоятельности.

Максимальный уровень обучаемости кибернетической системы – это её способность эволюционировать (повышать уровень своего качества) максимально быстро и в любом направлении, т.е. способность быстро и без каких-либо ограничений приобретать любые новые знания и навыки.

Реализация способности кибернетической системы обучаться, т.е. решать перманентно инициированную сверхзадачу самообучения, накладывает дополнительные требования, предъявляемые к информации, хранимой в памяти кибернетической системы, к решателю задач кибернетической системы, а в перспективе также и к физической оболочке кибернетической системы.

Важнейшей характеристикой кибернетической системы является не только то, какой уровень интеллекта кибернетическая система имеет в текущий момент, какое множество действий (задач) она способна выполнять, но и то, насколько быстро этот уровень может повышаться.

\begin{SCn}
\scnheader{обучаемость кибернетической системы}
\begin{scnrelfromlist}{cвойство-предпосылка}
    \scnitem{гибкость кибернетической системы}
    \scnitem{стратифицированность кибернетической системы}
    \scnitem{рефлексивность кибернетической системы}
    \scnitem{ограниченность обучения кибернетической системы}
    \scnitem{познавательная активность кибернетической системы}
    \scnitem{способность кибернетической системы к самосохранению}
\end{scnrelfromlist}
\end{SCn}

Поскольку обучение всегда сводится к внесению тех или иных изменений в обучаемую кибернетическую систему, без высокого уровня гибкости этой системы не может быть высокого уровня её обучаемости.
Гибкость возможных самоизменений кибернетической системы определяется простотой и многообразием возможных самоизменений кибернетической системы.

При наличии стратифицированности кибернетической системы появляется возможность четкого определения области действия различных изменений, вносимых в кибернетическую систему, т.е. возможность четкого ограничения тех частей кибернетической системы, за пределы которых нет необходимости выходить для учета последствий внесенных в систему первичных изменений (осуществлять дополнительные изменения, являющиеся последствиями первичных изменений).

Рефлексивность кибернетической системы есть способность кибернетической системы к самоанализу.
Конструктивным результатом рефлексии кибернетической системы является генерация в её памяти спецификации различных негативных или подозрительных особенностей, которые следует учитывать для повышения качества кибернетической системы.
Такими особенностями (недостатками) могут быть выявленные противоречия (ошибки), выявленные пары синонимичных знаков, омонимичные знаки, информационные дыры.

Ограниченность обучения кибернетической системы определяет границу между теми знаниями и навыками, которые соответствующая кибернетическая система принципиально может приобрести, и теми знаниями и навыками, которые указанная кибернетическая система не сможет приобрести никогда.
Данное свойство определяет максимальный уровень потенциальных возможностей соответствующей кибернетической системы.
Максимальная степень отсутствия ограничений в приобретении новых знаний и навыков – это полное отсутствие ограничений, т.е. полная универсальность возможностей соответствующих кибернетических систем.

Познавательная активность кибернетической системы --- любознательность, активность и самостоятельность в приобретении новых знаний и навыков.
Следует отличать способность приобретать новые знания и навыки, а также их совершенствовать, от желания это делать.
Желание (целевая установка) научиться решать те или иные задачи может быть сформулировано кибернетической системой либо самостоятельно, либо извне (некоторым учителем).

\begin{SCn}
\scnheader{познавательная активность кибернетической системы}
\begin{scnrelfromlist}{cвойство-предпосылка}
    \scnitem{способность кибернетической системы к синтезу познавательных целей и процедур}
    \scnitem{способность кибернетической системы к самоорганизации собственного обучения}
    \scnitem{способность кибернетической системы к экспериментальным действиям}
\end{scnrelfromlist}
\end{SCn}

Способность кибернетической системы к синтезу познавательных целей и процедур является способностью планировать своё обучение и управлять процессом обучения, умение задавать вопросы или целенаправленные последовательности вопросов, способность генерировать четкую спецификацию своей информационной потребности.
Способность кибернетической системы к самоорганизации собственного обучения есть способность осуществлять управление своим обучением, способность кибернетической системы самой выполнять роль своего учителя.
Способность кибернетической системы к экспериментальным действиям --- способность к отклонениям от составленных планов своих действий для повышения качества результата или сохранении целенаправленности этих действий, способность к импровизации.

Чем выше уровень безопасности кибернетической системы, тем выше её уровень обучаемости.
Способность кибернетической системы к самосохранению означает способность кибернетической системы к выявлению и устранению угроз, направленных на снижение её качества и даже на её уничтожение, что означает полную потерю необходимого качества.


\subsection{Комплекс свойств, определяющих уровень интеллекта кибернетической системы}
{\label{sec_cyb_syst_intelligence_quality}} 

Основным свойством, характеристикой кибернетической системы является уровень ее интеллекта, который является интегральной характеристикой, определяющей уровень эффективности взаимодействия кибернетической системы со средой своего существования.
Процесс эволюции кибернетических систем следует рассматривать как процесс повышения уровня их качества по целому ряду свойств и как процесс повышения уровня их интеллекта.

Кибернетическая система может быть как интеллектуальной, так и неинтеллектуальной. В свою очередь, интеллектуальные система может быть как слабоинтеллектуальной, так и высокоинтеллектуальной.

\begin{SCn}
\scnheader{уровень интеллекта кибернетической системы}
\begin{scnrelfromlist}{cвойство-предпосылка}
    \scnitem{образованность кибернетической системы}
    \scnitem{обучаемость кибернетической системы}
    \scnitem{интероперабельность кибернетической системы}
\end{scnrelfromlist}
\end{SCn}

Образованность кибернетической системы есть уровень навыков, а также иных знаний, приобретенных кибернетической системой к заданному моменту. 

\begin{SCn}
\scnheader{образованность кибернетической системы}
\begin{scnrelfromlist}{cвойство-предпосылка}
    \scnitem{качество навыков, приобретенных кибернетической системой}
    \scnitem{качество информации, хранимой в памяти кибернетической системы}
\end{scnrelfromlist}
\end{SCn}

Примерами образованной кибернетической системы являются:
\begin{textitemize}
    \item кибернетическая система, основанная на знаниях;
    \item кибернетическая система, управляемая знаниями;
    \item целенаправленная кибернетическая система;
    \item гибридная кибернетическая система;
    \item потенциально универсальная кибернетическая система.
\end{textitemize}

Обучаемая кибернетическая система есть кибернетическая система, способная познавать среду своего обитания, то есть строить и постоянно уточнять в своей памяти информационную модель этой среды, а также использовать эту модель для решения различных задач (для организации своей деятельности) в указанной среде. 

Примерами обучаемой кибернетической системы являются:
\begin{textitemize}
    \item кибернетическая система с высоким уровнем стратифицированности своих знаний и навыков;
    \item рефлексивная кибернетическая система;
    \item самообучаемая кибернетическая система;
    \item кибернетическая система с высоким уровнем познавательной активности.
\end{textitemize}

Интеллект кибернетической системы, как и лежащий в его основе познавательный процесс, выполняемый кибернетической системой, имеет социальный характер, поскольку наиболее эффективно формируется и развивается в форме взаимодействия кибернетической системы с другими кибернетическими системами.
Социально ориентированная кибернетическая система имеет достаточно высокий уровень интеллекта, чтобы быть полезным членом различных, в том числе, и человеко-машинных сообществ.
Определенный уровень социально значимых качеств является необходимым условием интеллектуальности кибернетической системы.
Примерами социально ориентированной кибернетической системы являются:
\begin{textitemize}
    \item кибернетическая система, способная устанавливать и поддерживать высокий уровень семантической совместимости и взаимопонимания с другими системами;
    \item договороспособная кибернетическая система.
\end{textitemize}

Все свойства, присущие кибернетическим системам, в различных кибернетических системах могут иметь самый различный уровень.
Более того, в некоторых кибернетических системах некоторые из этих свойств могут вообще отсутствовать.
При этом в кибернетических системах, которые условно будем называть интеллектуальными системами, все указанные выше свойства должны быть представлены в достаточно развитом виде.