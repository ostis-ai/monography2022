\subsection{Комплекс свойств, определяющий общий уровень качества кибернетической системы}
{\label{sec_cyb_syst_overall_quality}} 

Интегральная, комплексная оценка уровня развития \textit{кибернетической системы} определяется как \textbf{\textit{качество кибернетической системы}}. Это свойство, характеристика \textit{кибернетических систем}, признак их классификации, который позволяет разместить эти системы по "ступенькам"{} некоторой условной "эволюционной лестницы"{}.
На каждую такую "ступеньку"{} попадают \textit{кибернетические системы}, имеющие одинаковый уровень развития, каждому из которых соответствует свой набор значений дополнительных свойств \textit{кибернетических систем}, которые уточняют соответствующий уровень развития \textit{кибернетических систем}. Такой эволюционный подход к рассмотрению \textit{кибернетических систем} даёт возможность, во-первых, детализировать направления эволюции \textit{кибернетических систем} и, во-вторых, уточнить то место этой эволюции, где и благодаря чему осуществляется переход от неинтеллектуальных \textit{кибернетических систем} к интеллектуальным.

В основе эволюционного подхода к рассмотрению многообразия \textit{кибернетических систем} лежит положение о том, что идеальных \textit{кибернетических систем} не существует, но существует постоянное стремление к идеалу, к большему совершенству. 
Важно уточнить, что конкретно в каждой \textit{кибернетической системе} следует изменить, чтобы привести эту систему к более совершенному виду. Так, например, развитие технологий разработки компьютерных систем должно быть направлено на переход к таким новым архитектурным и функциональным принципам, лежащим в основе \textit{компьютерных систем}, которые обеспечивают существенное снижение трудоемкости их разработки и сокращение сроков разработки, а также обеспечивают существенное повышение уровня интеллекта и, в частности, уровня обучаемости разрабатываемых \textit{компьютерных систем}, например, путём перехода от поддержки обучения с учителем к реализации эффективного самообучения (к автоматизации организации самостоятельного обучения). 

Проблема выделения критериев интеллектуальности компьютерных систем рассмотрена в ряде работ: \scncite{Finn2021}, \scncite{Nilsson2005}, \scncite{Kerr2006}, \scncite{Antsyferov2013}. Были предложены различные системные показатели для измерения \textit{качества компьютерных систем}. Поскольку системы становятся все более сложными и включают множество подсистем или компонентов, измерение их качества в нескольких измерениях становится сложной задачей (см. \scncite{Cho2019}). Метрики качества включают в себя набор мер, которые могут описывать атрибуты системы в терминах, не зависящих от структуры, которая приводит к этим атрибутам. Эти меры должны быть выражены количественно и должны иметь значительный уровень точности и надежности (см. \scncite{Sherif1988}). 

Тем не менее, для практической реализации \textit{интеллектуальных компьютерных систем} необходимо детализировать и уточнить эти свойства, пытаясь свести их к более конструктивным, прозрачным и понятным для реализации свойствам. Для детализации понятия качества кибернетической системы необходимо задать метрику \textit{качества кибернетических систем} и построить иерархическую систему свойств, параметров, признаков, определяющих качество кибернетических систем.

\begin{SCn}
\scnheader{качество кибернетической системы}
\begin{scnrelfromlist}{cвойство-предпосылка}
    \scnitem{качество физической оболочки кибернетической системы}
    \scnitem{качество информации, хранимой в памяти кибернетической системы}
    \scnitem{качество решателя задач кибернетической системы}
    \scnitem{гибридность кибернетической системы}
    \begin{scnindent}
        \begin{scnrelfromset}{частное свойство}
            \scnitem{многообразие видов знаний, хранимых в памяти кибернетической системы}
            \scnitem{многообразие моделей решения задач}
            \scnitem{многообразие видов сенсоров и эффекторов}
        \end{scnrelfromset}
    \end{scnindent}
    \scnitem{приспособленность кибернетической системы к её совершенствованию}
    \scnitem{производительность кибернетической системы}
    \scnitem{надежность кибернетической системы}
    \scnitem{интероперабельность кибернетической системы}
\end{scnrelfromlist}
\end{SCn}


\subsection{Комплекс свойств, определяющих качество физической оболочки кибернетической системы}
{\label{sec_cyb_syst_physical_shell_quality}} 

\textbf{\textit{качество физической оболочки кибернетической системы}} определяется как интегральное качество физической, аппаратной основы \textit{кибернетической системы}. Выделенное множество свойств, определяющих \textit{качество физической оболочки кибернетической системы}, приведена ниже:
 
\begin{SCn}
\scnheader{качество физической оболочки кибернетической системы}
\begin{scnrelfromlist}{cвойство-предпосылка}
    \scnitem{качество памяти кибернетической системы}
    \scnitem{качество процессора кибернетической системы}
    \scnitem{качество сенсоров кибернетической системы}
    \scnitem{качество эффекторов кибернетической системы}
    \scnitem{приспособленность физической оболочки кибернетической системы к ее совершенствованию}
    \scnitem{удобство транспортировки кибернетической системы}
    \scnitem{надежность физической оболочки кибернетической системы}
\end{scnrelfromlist}
\end{SCn}

Важно, чтобы память обеспечивала высокий уровень гибкости указанной информационной модели. Важно также, чтобы эта информационная модель была моделью не только \textit{внешней среды кибернетической системы}, но также и моделью самой этой информационной модели --- описанием её текущей ситуации, предыстории, закономерностей.

\begin{SCn}
\scnheader{качество памяти кибернетической системы}
\begin{scnrelfromlist}{cвойство-предпосылка}
    \scnitem{способность памяти кибернетической системы обеспечить хранение высококачественной информации}
    \scnitem{способность памяти кибернетической системы обеспечить функционирование высококачественного решателя задач}
    \scnitem{объём памяти}
\end{scnrelfromlist}
\end{SCn}

Факт возникновения \textit{памяти в кибернетической системе} является важнейшим этапом её эволюции.
Дальнейшее развитие памяти кибернетической системы, обеспечивающее хранение все более качественной информации, хранимой в памяти и все более качественную организацию обработки этой информации, то есть переход на поддержку все более качественных моделей обработки информации, является важнейшим фактором эволюции \textit{кибернетических систем}.

\textbf{\textit{способность \textit{памяти кибернетической системы} обеспечить функционирование высококачественного \textit{решателя задач}}} основывается на качестве доступа к информации, хранимой в \textit{памяти кибернетической системы}, логико-семантической гибкости \textit{памяти кибернетической системы}, способности \textit{памяти кибернетической системы} обеспечить интерпретацию широкого многообразия \textit{моделей решения задач}.

\textbf{\textit{качество процессора кибернетической системы}} определяется его способностью обеспечить функционирования высококачественного \textit{решателя задач}.

\begin{SCn}
\scnheader{качество процессора кибернетической системы}
\begin{scnrelfromlist}{cвойство-предпосылка}
    \scnitem{многообразие моделей решения задач, интерпретируемых процессором
кибернетической системы}
    \scnitem{простота и качество интерпретации процессором системы широкого многообразия моделей решения задач}
    \scnitem{обеспечение процессором кибернетической системы качественного управления
информационными процессами в памяти}
    \scnitem{быстродействие процессора кибернетической системы}
\end{scnrelfromlist}
\end{SCn}

Максимальным уровнем \textit{качества процессора кибернетической системы} по параметру \textbf{\textit{многообразия \textit{моделей решения задач}, интерпретируемых \textit{процессором кибернетической системы}}}, является его универсальность, то есть его принципиальная возможность интерпретировать любую модель решения как интеллектуальных, так и неинтеллектуальных задач. 
Простота определяется степенью близости интерпретируемых моделей решения задач к "физическому"{} уровню организации процессора кибернетической системы. 
Качественное управление информационными процессами в памяти подразумевает грамотное сочетание таких аспектов управление процессами, как централизация и децентрализация (см. \scncite{Melekhova2018}), синхронность и асинхронность, последовательность и параллельность.

\textbf{\textit{качество сенсоров}} и \textbf{\textit{эффекторов кибернетической системы}} сводится к многообразию видов сенсоров и эффекторов кибернетической системы, то есть к многообразию средств восприятия и воздействия на информацию о текущем состоянии внешней среды и собственной физической оболочки.
\textbf{\textit{приспособленность физической оболочки кибернетической системы к её совершенствованию}} определяется гибкостью и стратифицированностью физической оболочки кибернетической системы.

\subsection{Комплекс свойств, определяющих качество информации, хранимой в памяти кибернетической системы}
{\label{sec_cyb_syst_information_quality}} 

Качество информационной модели среды "обитания"{} \textit{кибернетической системы}, в частности, определяется:
\begin{textitemize}
    \item корректностью этой модели, отсутствием в ней ошибок;
    \item адекватностью этой модели;
    \item полнотой, достаточностью находящейся в ней информации для эффективного функционирования кибернетической системы;
    \item структурированностью, систематизированностью.
\end{textitemize}

Важнейшим этапом эволюции информационной модели среды \textit{кибернетической системы} является переход от недостаточно полной и несистематизированной информационные модели среды к \textit{базе знаний}.

\begin{SCn}
\scnheader{качество информации, хранимой в памяти кибернетической системы}
\begin{scnrelfromlist}{cвойство-предпосылка}
    \scnitem{простота и локальность выполнения семантически целостных операций над информацией,
хранимой в памяти кибернетической системы}
    \scnitem{корректность/некорректность информации, хранимой в памяти кибернетической системы}
    \scnitem{однозначность/неоднозначность информации, хранимой в памяти кибернетической системы}
    \scnitem{целостность/нецелостность информации, хранимой в памяти кибернетической системы}
    \scnitem{чистота/загрязненность информации, хранимой в памяти кибернетической системы}
    \scnitem{достоверность/недостоверность информации, хранимой в памяти кибернетической системы}
    \scnitem{точность/неточность информации, хранимой в памяти кибернетической системы}
    \scnitem{четкость/нечеткость информации, хранимой в памяти кибернетической системы}
    \scnitem{определенность/недоопределенность информации, хранимой в памяти кибернетической системы}
    \scnitem{семантическая мощность языка представления информации в памяти кибернетической системы}
    \scnitem{объём информации, загруженной в память кибернетической системы}
    \scnitem{гибридность информации, хранимой в памяти кибернетической системы}
    \scnitem{стратифицированность информации, хранимой в памяти кибернетической системы}
\end{scnrelfromlist}
\end{SCn}

\textbf{\textit{корректность/некорректность информации, хранимой в памяти кибернетической системы}} определяется её уровнем адекватности в той среде, в которой существует \textit{кибернетическая система} и информационной моделью, которой эта хранимая информация является. 
\textbf{\textit{непротиворечивость/противоречивость информации, хранимой в памяти кибернетической системы}} означает уровень присутствия в хранимой информации различного вида противоречий и, в частности, ошибок. 
Ошибки в хранимой информации могут быть синтаксическими и семантическими, противоречащими некоторым правилам, которые явно в памяти могут быть не представлены и считаются априори истинными.

\textbf{\textit{полнота/неполнота информации, хранимой в памяти кибернетической системы}} --- уровень того, насколько \textit{информация, хранимая в памяти кибернетической системы}, описывает среду существования этой системы и используемые ею \textit{методы решения задач} достаточно полно для того, чтобы кибернетическая система могла действительно решать все множество соответствующих ей задач. 
Чем полнее \textit{информация, хранимая в памяти кибернетической системы}, и чем полнее информационное обеспечение деятельности этой системы, тем эффективнее (качественнее) сама эта деятельность. 
Полнота определяется структурированностью информации и многообразием видов знаний, хранимых в \textit{памяти кибернетической системы}.

\textbf{\textit{однозначность/неоднозначность информации, хранимой в памяти кибернетической системы}} определяется многообразием форм дублирования информации и частотой дублирования информации.

\textbf{\textit{целостность/нецелостность информации, хранимой в памяти кибернетической системы}} --- уровень содержательной информативности информации, то есть уровень того, насколько семантически связной является информация, насколько полно специфицированы все описываемые в памяти сущности (путём описания необходимого набора связей этих сущностей с другими описываемыми сущностями), насколько редко или часто в рамках хранимой информации встречаются \textbf{\textit{информационные дыры}}, соответствующие явной недостаточности некоторых спецификаций. 
Примерами \textit{информационных дыр} являются:
\begin{textitemize}
    \item отсутствующий \textit{метод решения} часто встречающихся \textit{задач};
    \item отсутствующее определение используемого определяемого \textit{понятия};
    \item недостаточно подробная спецификация часто рассматриваемых сущностей.
\end{textitemize}

\textbf{\textit{чистота/загрязненность информации, хранимой в памяти кибернетической системы}} означает многообразие форм и общее количество информационного мусора, входящего в состав \textit{информации, хранимой в памяти кибернетической системы}.
Под \textbf{\textit{информационным мусором}} понимается информационный фрагмент, входящий в состав информации, удаление которого существенно не усложнит деятельность \textit{кибернетической системы}.
Примерами \textit{информационного мусора} являются:
\begin{textitemize}
    \item информация, которая нечасто востребована, но при необходимости может быть легко логически выведена;
    \item информация, актуальность которой истекла.
\end{textitemize}

\textbf{\textit{семантическая мощность языка представления информации в памяти кибернетической системы}} определяется гибридностью информации, хранимой в памяти кибернетической системы. Язык, \textit{информационные конструкции} которого могут представить любую конфигурацию любых связей между любыми сущностями, является \textit{универсальным языком}.
Универсальность \textit{внутреннего языка кибернетической системы} является важнейшим фактором её интеллектуальности.

\textbf{\textit{гибридность информации, хранимой в памяти кибернетической системы}} определяется многообразием видов знаний и степенью конвергенции и интеграции различного вида знаний.

\textbf{\textit{стратифицированность информации, хранимой в памяти кибернетической системы}} есть способность \textit{кибернетической системы} выделять такие разделы \textit{информации, хранимой в памяти этой системы}, которые бы ограничивали области действия \textit{агентов} \textit{решателя задач} \textit{кибернетической системы}, являющиеся достаточными для решения заданных задач.
Стратифицированность определяется структурированностью и рефлексивностью \textit{информации, хранимой в памяти кибернетической системы}.
Рефлексивность \textit{информации, хранимой в памяти кибернетической системы}, то есть наличие метаязыковых средств, является фактором, обеспечивающим не только структуризацию хранимой информации, но и возможность описания \textit{синтаксиса} и \textit{семантики} самых различных \textit{языков}, используемых кибернетической системой.

\textit{база знаний} является примером \textit{информации, хранимой в памяти кибернетической системы} и имеющей высокий уровень качества по всем показателям и, в частности, высокий уровень:
\begin{textitemize}
    \item \textit{семантической мощности языка представления информации хранимой в памяти кибернетической
системы};
    \item \textit{гибридности информации, хранимой в памяти кибернетической системы};
    \item \textit{многообразия видов знаний, хранимых в памяти кибернетической системы};
    \item \textit{формализованности информации, хранимой в памяти кибернетической системы};
    \item \textit{структурированности информации, хранимой в памяти кибернетической системы};
\end{textitemize}

Переход \textit{информации, хранимой в памяти кибернетической системы на уровень качества}, соответствующий \textit{базам знаний}, является важнейшим этапом эволюции \textit{кибернетических систем} (см. \textit{Главу \ref{chapter_kb}~\nameref{chapter_kb}}).

\subsection{Комплекс свойств, определяющих качество решателя задач кибернетической системы}
{\label{sec_cyb_syst_problem_solver_quality}} 

\textbf{\textit{качество решателя задач кибернетической системы}} --- интегральная качественная оценка множества задач, которые \textit{кибернетическая система} способна выполнять в заданный момент.
Основным свойством и назначением \textit{решателя задач} \textit{кибернетической системы} является способность решать задачи на основе накапливаемых, приобретаемых \textit{кибернетической системой} различного вида навыков с использованием \textit{процессора кибернетической системы}, являющегося универсальным интерпретатором всевозможных накопленных навыков. 
При этом качество указанной способности определяется целым рядом дополнительных факторов.

\begin{SCn}
\scnheader{общая характеристика решателя задач кибернетической системы}
\begin{scnrelfromset}{cвойство-предпосылка}
    \scnitem{общий объем задач, решаемых кибернетической системой}
    \scnitem{многообразие видов задач, решаемых кибернетической системой}
    \scnitem{способность кибернетической системы к анализу решаемых задач}
    \scnitem{способность кибернетической системы к решению задач, методы решения которых в текущий момент известны}
    \scnitem{способность кибернетической системы к решению задач, методы решения которых ей в текущий момент не известны}
    \scnitem{множество навыков, используемых кибернетической системой}
    \scnitem{степень конвергенции и интеграции различного вида моделей решения задач, используемых кибернетической системой}
    \scnitem{качество организации взаимодействия процессов решения задач в кибернетической системе}
    \scnitem{быстродействие решателя задач кибернетической системы}
    \scnitem{способность кибернетической системы решать задачи, предполагающие использование информации, обладающей различного рода не-факторами}
    \scnitem{многообразие и качество решения задач информационного поиска}
    \scnitem{способность кибернетической системы генерировать ответы на вопросы}
    \scnitem{способность кибернетической системы к рассуждениям различного вида}
    \scnitem{самостоятельность целеполагания кибернетической системы}
    \scnitem{качество реализации планов собственных действий}
    \scnitem{способность кибернетической системы к локализации такой области информации, хранимой в ее памяти, которой достаточно для обеспечения решения заданной задачи}
    \scnitem{способность кибернетической системы к выявлению существенного в информации, хранимой в ее памяти}
    \scnitem{активность кибернетической системы}
\end{scnrelfromset}
\end{SCn}

\textbf{\textit{общий объем задач, решаемых кибернетической системой}}, определяется \textit{мощностью языка представления задач, решаемых кибернетической системой}.
Мощность языка представления задач прежде всего определяется \textbf{\textit{многообразием видов задач, решаемых кибернетической системой}} (многообразием видов описываемых действий). Каждая задача есть спецификация соответствующего (описываемого) действия. 
Поэтому рассмотрение \textit{многообразия видов задач, решаемых кибернетической системой}, полностью соответствует многообразию видов деятельности, осуществляемой этой системой. Важно заметить, что есть виды \textit{деятельности кибернетической системы}, которые определяют качество и, в частности, \textit{уровень интеллекта кибернетической системы}.

\textbf{\textit{способность кибернетической системы к анализу решаемых задач}} предполагает оценку задачи на предмет:
\begin{textitemize}
    \item сложности достижения;
    \item целесообразности достижения (нужности, важности, приоритетности);
    \item соответствия цели существующим нормам (правилам) соответствующей деятельности.
\end{textitemize}

\textbf{\textit{метод решения задач}} --- это вид \textit{знаний, хранимых в памяти кибернетической системы} и содержащих информацию, которой достаточно либо для сведения каждой задачи из соответствующего класса задач к полной системе подзадач, решение которых гарантирует решение исходной задачи, либо для окончательного решения этой задачи из указанного класса задач. 
\textit{методами для решения задач} могут быть не только алгоритмы, но также и \textit{функциональные программы}, \textit{продукционные системы}, \textit{логические исчисления}, \textit{генетические алгоритмы}, \textit{искусственные нейронные сети} различного вида и так далее.
Задачи, для которых не находятся соответствующие им методы, решаются с помощью метаметодов (стратегий) решения задач, направленных:
\begin{textitemize}
    \item на генерацию нужных исходных данных (нужного контекста), необходимых для решения каждой задачи;
    \item на генерацию плана решения задачи, описывающего сведение исходной задачи к подзадачам (до тех подзадач, методы решения которых системы известны);
    \item на сужение области решения задачи (на сужения контекста задачи, достаточного для ее решения).
\end{textitemize}

\textit{качество решения каждой задачи} определяется:
\begin{textitemize}
    \item \textit{временем её решения} (чем быстрее задача решается, тем выше качество её решения);
    \item \textit{полнотой и корректностью результата решения задачи};
    \item \textit{затраченными для решения задачи ресурсами памяти} (объемом фрагмента хранимой информации, используемой для решения задачи);
    \item \textit{затраченным для решения задачи ресурсами решателя задач} (количеством используемых внутренних агентов).
\end{textitemize}

Таким образом, повышение \textit{качества решения каждой конкретной задачи}, а также каждого \textit{класса задач} (путем совершенствования соответствующего метода, в частности, алгоритма) является важным фактором повышения \textit{качества решателя задач кибернетической системы} в целом.

Перспективным вариантом построения \textit{решателя задач} \textit{кибернетической системы} является реализация \textit{агентно-ориентированной модели обработки информации}, то есть построение \textit{решателя задач} в виде \textit{многоагентной системы}, агенты которой осуществляют обработку \textit{информации, хранимой в памяти кибернетической системы}, и управляются этой информацией (точнее, её текущим состоянием). 
Особое место среди этих агентов занимают \textit{сенсорные} (рецепторные) и \textit{эффекторные агенты}, которые, соответственно, воспринимают информацию о текущем состоянии внешней среды и воздействуют на внешнюю среду, в частности, путем изменения состояния \textit{физической оболочки кибернетической системы} (см. \textit{Главу \ref{chapter_situation_management}~\nameref{chapter_situation_management}}).

Указанная агентно-ориентированная модель организации взаимодействия процессов решения задач в \textit{кибернетической системе} по сути есть не что иное, как модель ситуационного управления процессами решения задач, решаемых \textit{кибернетической системой} как в своей внешней среде, так и в своей памяти.

\textbf{\textit{быстродействие решателя задач кибернетической системы}} сводится к скорости решения задач, быстродействию решателя задач, скорости реакции \textit{кибернетической системы} на различные задачные ситуации. Во многом свойство определяется \textit{быстродействием процессора кибернетической системы}.

\textbf{\textit{способность кибернетической системы решать задачи, предполагающие использование информации, обладающей различного рода не-факторами}} включает:
\begin{textitemize}
	\item нечетко сформулированные задачи ("делай то, не знаю что");
	\item задачи, которые решаются в условиях неполноты, неточности, противоречивости исходных данных;
	\item задачи, принадлежащие классам задач, для которых практически невозможно построить соответствующие алгоритмы.
\end{textitemize}
Примерами задач, предполагающих использование информации, обладающей различного рода \textit{не-факторами}, являются задачи проектирования, распознавания, целеполагания, прогнозирования, и так далее.
Для таких задач характерны:
\begin{textitemize}
    \item неточность и недостоверность исходных данных;
    \item отсутствие критерия качества результата;
    \item невозможность или высокая трудоемкость разработки алгоритма;
    \item необходимость учета контекста задачи.
\end{textitemize}

\textbf{\textit{способность кибернетической системы генерировать (порождать, строить, синтезировать, выводить) ответы на самые различные вопросы}} и, в частности, на \textit{вопросы} типа \scnqqi{что это такое?}, на почему-вопросы, означает способность \textit{кибернетической системы} объяснять (обосновывать корректность) своих действий (см. \textit{Главу \ref{chapter_requests}~\nameref{chapter_requests}}).

\textbf{\textit{самостоятельность целеполагания кибернетической системы}} --- способность \textit{кибернетической системы} генерировать, инициировать и решать задачи, которые не являются подзадачами, инициированными внешними (другими) субъектами, а также способность на основе анализа своих возможностей отказаться от выполнения задачи, инициированной извне, переадресовав её другой \textit{кибернетической системе}, либо на основе анализа самой этой задачи обосновать её нецелесообразность или некорректность.  Повышение уровня самостоятельности существенно расширяет возможности \textit{кибернетической системы}, то есть объем тех задач, которые она может решать не только в "идеальных"{} условиях, но и в реальных, осложненных обстоятельствах. Способность \textit{кибернетической системы} адекватно расставлять приоритеты своим целям и не "распыляться"{} на достижение неприоритетных (несущественных) целей есть способность анализа целесообразности деятельности.

\textbf{\textit{способность кибернетической системы к выявлению существенного в информации, хранимой в ее памяти}} --- способность к выявлению (обнаружению, выделению) таких фрагментов \textit{информации, хранимой в памяти кибернетической системы}, которые существенны (важны) для достижения соответствующих целей. Понятие \textit{существенного (важного) фрагмента информации, хранимой в памяти кибернетической системы}, относительно и определяется соответствующей задачей. Тем не менее, есть важные перманентно решаемые задачи, в частности \textit{задачи анализа качества информации, хранимой в памяти кибернетической системы}. Существенные фрагменты хранимой информации, выделяемые в процессе решения этих задач, являются относительными не столько по отношению к решаемой задаче, сколько по отношению к текущему состоянию хранимой информации.

Уровень \textbf{\textit{активности кибернетической системы}} может быть разным для разных решаемых задач, для разных классов выполняемых действий, для разных \textit{видов деятельности}. Чем выше \textit{активность кибернетической системы}, тем (при прочих равных условиях) она больше успевает сделать, следовательно, тем выше ее качество (эффективность). Обратным свойством является понятие \textbf{\textit{пассивности кибернетической системы}}. 

\subsection{Комплекс свойств, определяющих уровень обучаемости кибернетической системы}
{\label{sec_cyb_syst_learnability_quality}} 

\textbf{\textit{обучаемость кибернетической системы}} --- способность \textit{кибернетической системы} повышать своё качество, адаптируясь к решению новых задач, \textit{качество внутренней информации модели своей среды}, \textit{качество} своего \textit{решателя задач} и даже \textit{качество} своей \textit{физической оболочки}. 

Максимальный уровень \textbf{\textit{обучаемости кибернетической системы}} --- это её способность эволюционировать (повышать уровень своего качества) максимально быстро и в любом направлении, то есть способность быстро и без каких-либо ограничений приобретать любые новые \textit{знания} и \textit{навыки}.

Реализация способности \textit{кибернетической системы} обучаться, то есть решать перманентно инициированную сверхзадачу самообучения, накладывает дополнительные требования, предъявляемые к \textit{информации, хранимой в памяти кибернетической системы}, к \textit{решателю задач кибернетической системы}, а в перспективе также и к \textit{физической оболочке кибернетической системы}.

Важнейшей характеристикой \textit{кибернетической системы} является не только то, какой уровень интеллекта \textit{кибернетическая система} имеет в текущий момент, какое множество действий (задач) она способна выполнять, но и то, насколько быстро этот уровень может повышаться.

\begin{SCn}
\scnheader{обучаемость кибернетической системы}
\begin{scnrelfromlist}{cвойство-предпосылка}
    \scnitem{гибкость кибернетической системы}
    \scnitem{стратифицированность кибернетической системы}
    \scnitem{рефлексивность кибернетической системы}
    \scnitem{ограниченность обучения кибернетической системы}
    \scnitem{познавательная активность кибернетической системы}
    \scnitem{способность кибернетической системы к самосохранению}
\end{scnrelfromlist}
\end{SCn}

Поскольку обучение всегда сводится к внесению тех или иных изменений в обучаемую \textit{кибернетическую систему}, без высокого уровня гибкости этой системы не может быть высокого уровня её обучаемости.
\textbf{\textit{гибкость кибернетической системы}} определяется простотой и многообразием возможных самоизменений \textit{кибернетической системы}.

При наличии \textbf{\textit{стратифицированности кибернетической системы}} появляется возможность четкого определения области действия различных изменений, вносимых в кибернетическую систему, то есть возможность четкого ограничения тех частей кибернетической системы, за пределы которых нет необходимости выходить для учета последствий внесенных в систему первичных изменений (осуществлять дополнительные изменения, являющиеся последствиями первичных изменений).

\textbf{\textit{рефлексивность кибернетической системы}} есть способность \textit{кибернетической системы} к самоанализу.
Конструктивным результатом рефлексии \textit{кибернетической системы} является генерация в её памяти спецификации различных негативных или подозрительных особенностей, которые следует учитывать для повышения качества кибернетической системы. Такими особенностями (недостатками) могут быть выявленные противоречия (ошибки), выявленные пары \textit{синонимичных знаков}, \textit{омонимичные знаки}, \textit{информационные дыры}.

\textbf{\textit{ограниченность обучения кибернетической системы}} определяет границу между теми \textit{знаниями} и \textit{навыками}, которые соответствующая \textit{кибернетическая система} принципиально может приобрести, и теми \textit{знаниями} и \textit{навыками}, которые указанная \textit{кибернетическая система} не сможет приобрести никогда. Данное свойство определяет максимальный уровень потенциальных возможностей соответствующей кибернетической системы. Максимальная степень отсутствия ограничений в приобретении новых \textit{знаний} и \textit{навыков} --- это полное отсутствие ограничений, то есть полная универсальность возможностей соответствующих кибернетических систем.

\textbf{\textit{познавательная активность кибернетической системы}} --- любознательность, активность и самостоятельность в приобретении новых знаний и навыков. Следует отличать способность приобретать новые \textit{знания} и \textit{навыки}, а также их совершенствовать, от желания это делать. Желание (целевая установка) научиться решать те или иные задачи может быть сформулировано \textit{кибернетической системой} либо самостоятельно, либо извне (некоторым учителем).

\begin{SCn}
\scnheader{познавательная активность кибернетической системы}
\begin{scnrelfromlist}{cвойство-предпосылка}
    \scnitem{способность кибернетической системы к синтезу познавательных целей и процедур}
    \scnitem{способность кибернетической системы к самоорганизации собственного обучения}
    \scnitem{способность кибернетической системы к экспериментальным действиям}
\end{scnrelfromlist}
\end{SCn}

\textbf{\textit{способность кибернетической системы к синтезу познавательных целей и процедур}} является способностью планировать своё обучение и управлять процессом обучения, умение задавать \textit{вопросы} или целенаправленные последовательности \textit{вопросов}, способность генерировать четкую спецификацию своей информационной потребности. \textbf{\textit{cпособность кибернетической системы к самоорганизации собственного обучения}} есть способность осуществлять управление своим обучением, способность кибернетической системы самой выполнять роль своего учителя. \textbf{\textit{способность кибернетической системы к экспериментальным действиям}} --- способность к отклонениям от составленных планов своих действий для повышения качества результата или сохранении целенаправленности этих действий, способность к импровизации.

Чем выше уровень \textit{безопасности кибернетической системы}, тем выше уровень \textit{обучаемости кибернетической системы}.
\textbf{\textit{способность кибернетической системы к самосохранению}} означает способность \textit{кибернетической системы} к выявлению и устранению угроз, направленных на снижение её качества и даже на её уничтожение, что означает полную потерю необходимого качества (см. \textit{Главу \ref{chapter_security}~\nameref{chapter_security}}).

\subsection{Комплекс свойств, определяющих уровень интеллекта кибернетической системы}
{\label{sec_cyb_syst_intelligence_quality}} 

Исследователи \textit{Искусственного интеллекта} определяют \textit{интеллект} как неотъемлемое свойство машины (см. \scncite{Gao2002}). 
Их целью является построение систем, которые демонстрируют весь спектр когнитивных способностей, которые мы обнаруживаем у людей (см. \scncite{Laird2009}).

Основным свойством, характеристикой \textit{кибернетической системы} является уровень ее интеллекта, который является интегральной характеристикой, определяющей уровень эффективности взаимодействия \textit{кибернетической системы} со средой своего существования. 
Процесс эволюции \textit{кибернетических систем} следует рассматривать как процесс повышения уровня их качества по целому ряду свойств и как процесс повышения уровня их интеллекта.

\textit{кибернетическая система} может быть как \textit{интеллектуальной}, так и \textit{неинтеллектуальной}. В свою очередь, \textit{интеллектуальные система} может быть как \textit{слабоинтеллектуальной}, так и \textit{высокоинтеллектуальной}.

\begin{SCn}
\scnheader{уровень интеллекта кибернетической системы}
\begin{scnrelfromlist}{cвойство-предпосылка}
    \scnitem{образованность кибернетической системы}
    \scnitem{обучаемость кибернетической системы}
    \scnitem{интероперабельность кибернетической системы}
\end{scnrelfromlist}
\end{SCn}

\textbf{\textit{образованность кибернетической системы}} --- уровень \textit{навыков}, а также иных \textit{знаний}, приобретенных \textit{кибернетической системой} к заданному моменту. 

\begin{SCn}
\scnheader{образованность кибернетической системы}
\begin{scnrelfromlist}{cвойство-предпосылка}
    \scnitem{качество навыков, приобретенных кибернетической системой}
    \scnitem{качество информации, хранимой в памяти кибернетической системы}
\end{scnrelfromlist}
\end{SCn}

Примерами \textit{образованной кибернетической системы} являются:
\begin{textitemize}
    \item \textit{кибернетическая система, основанная на знаниях};
    \item \textit{кибернетическая система, управляемая знаниями};
    \item \textit{целенаправленная кибернетическая система};
    \item \textit{гибридная кибернетическая система};
    \item \textit{потенциально универсальная кибернетическая система}.
\end{textitemize}

\textbf{\textit{обучаемая кибернетическая система}} --- \textit{кибернетическая система}, способная познавать среду своего обитания, то есть строить и постоянно уточнять в своей памяти информационную модель этой среды, а также использовать эту модель для решения различных задач (для организации своей деятельности) в указанной среде. 

Примерами \textit{обучаемой кибернетической системы} являются:
\begin{textitemize}
    \item \textit{кибернетическая система} с высоким уровнем стратифицированности своих знаний и навыков;
    \item \textit{рефлексивная кибернетическая система};
    \item \textit{самообучаемая кибернетическая система};
    \item \textit{кибернетическая система} с высоким уровнем познавательной активности.
\end{textitemize}

\textit{интеллект кибернетической системы}, как и лежащий в его основе познавательный процесс, выполняемый кибернетической системой, имеет социальный характер, поскольку наиболее эффективно формируется и развивается в форме взаимодействия кибернетической системы с другими кибернетическими системами.
\textbf{социально ориентированная \textit{кибернетическая система}} имеет достаточно высокий уровень интеллекта, чтобы быть полезным членом различных, в том числе и человеко-машинных сообществ. Определенный уровень социально значимых качеств является необходимым условием \textit{интеллектуальности кибернетической системы}.
Примерами \textit{социально ориентированной кибернетической системы} являются:
\begin{textitemize}
    \item \textit{кибернетическая система}, способная устанавливать и поддерживать высокий уровень семантической совместимости и взаимопонимания с другими системами;
    \item \textit{договороспособная кибернетическая система}.
\end{textitemize}

Все свойства, присущие \textit{кибернетическим системам}, в различных \textit{кибернетических системах} могут иметь самый различный уровень. Более того, в некоторых \textit{кибернетических системах} некоторые из этих свойств могут вообще отсутствовать.
При этом в \textit{кибернетических системах}, которые условно будем называть \textit{интеллектуальными системами}, все указанные выше свойства должны быть представлены в достаточно развитом виде.
