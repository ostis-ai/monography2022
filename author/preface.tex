\begin{partbacktext}
\part*{Предисловие}
\markboth{ПРЕДИСЛОВИЕ}{ПРЕДИСЛОВИЕ}
\label{part_preface}
\addcontentsline{toc}{part}{Предисловие}

В настоящее время практически для любых государств является актуальным рассмотрение проблем обеспечения технологического суверенитета.

Всеобщая цифровизация потребовала массовой разработки софта и в программисты массово хлынула молодежь, начиная уже со школьной скамьи.

Действительно, зачем толковым школьникам идти на инженерные, естественнонаучные, конструкторские, сложносистемные специальности, если можно быстро вписаться в "технологический конвейер"{} разработки софта и гарантированно получать зарплату, которая намного выше зарплаты инженеров, конструкторов, проектировщиков сложных систем? И этот технологический конвейер быстро и массово превращает, среднестатистического программиста в ремесленника, даже, наверное, более точно в IT-пролетариат. Причем это не локальный IT-пролетариат, а глобальный IT-пролетариат (можно сказать даже глобально управляемый), формируемый по клише "граждан мира"{}. Хотя, с резким торможением глобализации перспективы гражданства мира стали не такими уж и очевидными и ясными.

Известный английский историк А.Дж.Тойнби в свое время ввел понятие "опасные классы"{}. Он использовал его при описании процессов конца 18-начала 19 веков, когда вырванное из сельской местности население превратилось в "опасные классы"{}. За десятилетия удалось их превратить в более-менее устойчивый пролетариат. Возможно сейчас процессы с опасными классами повторяются, в частности, с айтишниками, которые оказываются мало приспособленными к реалиям современной жизни. Но глобализация с цифровизацией превратила их в "граждан мира"{} (как им очень хотелось), но в виде недоформированного IT-пролетариата (что их жестко и неприятно приземлило).

Сейчас затруднительно адекватно использовать термины айтишник, программист. Всеобщая цифровизация практически быстро размывает эти понятия. На постсоветском пространстве понятие айтишник, а иногда даже и программист часто рассматривается как нечто единое, объединяющее и тестировщиков, и кодировщиков, и специалистов по искусственному интеллекиу, анализу больших данных, и системных архитекторов и т.д. В современной мировой практике этой цельности нет -- все это совершенно разные категории специалистов. Поэтому, когда иногда говорят о новом классе -- айтишников, то такого класса строго говоря и нет. Есть, наверное, достаточно массовый айтишный пролетариат (IT-пролетариат), которому очень хотелось бы идентифицировать себя с хайтеком. Но, если специалисты в области искусственного интеллекта, анализа больших данных, системные архитекторы, бизнес-аналитики, получившие серьезное образование, будут востребованы, то профессии тестировщика и кодировщика и близкие к ним, но с модными названиями могут быстро исчезнуть. В частности, искусственный интеллект способен этому сильно помочь. А своего рода миф о том, что нужно так много программистов отчасти поддерживается тем, что просто создается очень много некачественного кода (софта).

Мы всегда должны помнить, что в эпоху всеобщей цифровизации роль технологического суверенитета резко возрастает. Но без молодежи невозможно развивать технологический суверенитет. А молодежь мы упускаем. То есть молодежь вроде бы и стремится к новым технологиям, мы вроде бы и поощряем такое стремление, но этот процесс какой-то однобокий. Возьмем опять IT-сферу на постсоветском пространстве.

В университетах, где готовят IT-специалистов, много учебных центров и классов по продуктам Microsoft, Oracle, Cisco, SAP и т. п. Но это не научно-исследовательские центры, а банальные центры навыков, компетенций по простому кодированию, тестированию, внедрению и эксплуатации. Конечно, это неплохо, если бы в итоге не подменялись основные университетские курсы. Студенты видели, что в таких центрах разработки не нужны серьезная математика, физика. Терялась мотивация. Наиболее способные молодые специалисты плавно перетекали из центров разработки в центры за пределами страны, где этапы жизненных циклов разработки более интересны. Как правило, они не возвращались. Например, для нашей небольшой страны это существенные потери. Как выясняется, нередко в подобной эмиграции главное даже не заработки, а возможность творчества вместо ремесла и промышленного программирования. По сути, такая модель ускорила разрушение инженерного, конструкторского, общесистемного образования. Утрачивалась мотивация к изучению математических, физических, сложносистемных дисциплин. Действительно, студенту незачем тратить время на их глубокое освоение, если достаточно получить навыки по кодированию и тестированию софта на основе несложных технологических платформ и легко найти работу в одной из конвейерных компаний по разработке софта. По сути образование подстраивалось под такую модель и планомерно разрушалось при активном участии тех же западных конвейеров.

Вспомним, что традиционные западные обыватели мало интересуются глобальными вещами. Их интерес в основном локален. Например, решения муниципальных властей для них интереснее решений властей штата или земли, не говоря уже о решениях федеральных властей. Конечно могут быть исключения, но они лишь, как обычно, подтверждают общность. Точно таким же формируется у нас и молодое поколение, особенно айтишное. Действительно система образования и воспитания не воспроизводит сейчас системность во взглядах и оценках. Хорошая системность предполагает умение анализировать и синтезировать информацию, в том числе и по-глобальному. Советская система образования и воспитания, пусть и несколько односторонне, все-таки этому способствовала. Теперь же, в связи со старением и уходом поколений-носителей такой концепции, такая система образования и воспитания полностью исчезает. А новая не создана, вернее лихорадочно, бессистемно создается как ухудшенная калька с быстроразрушающейся западной системы, причем, зачастую, даже не с исконно западной, а как калька с восточно-европейской кальки. И у молодежи формируется основная парадигма, зачем думать о глобальном, системно, главное, чтобы в локальном мирке было все хорошо и как следствие нынешние молодежные установки: "жизнь в моменте"{}, "хорошо в моменте"{}, "хорошо здесь и сейчас"{} и тому подобное.

И именно айтишный пролетариат являются основным носителем таких установок. И понятно почему. Они работают практически полностью в западном (американском) конвейере, причем выполняя в основном самые примитивные операции, то есть работают как современный IT-пролетариат, даже не как инженера, технологи, не говоря уже о конструкторах или архитекторах. То есть совершенно не элита.

По такой модели мы получили лишь ускорение потери суверенитета технологического, и как следствие разрушение странового суверенитета.

Хотя последние несколько лет наконец-то многие в мире спохватились, что нужен суверенитет для стран. Глобалистский мир оказался не столь уж безопасен и привлекателен, как был распиарен за последние лет тридцать, граждане мира, как мировые кочевники, не столь уж и конструктивны и креативны -- собственно, все как всегда: хороша только виртуальная картинка, в существенно иных условиях идеи глобализма могли бы быть и продуктивны, но народ не тот, элиты не те и тому подобное. Модно стало говорить о цифровом суверенитете, \myuline{суверенитете в области высоких технологий}, суверенитете в области инфокоммуникационных технологий и так далее. Соответственно, откликаясь на моду, хайп (а сейчас многое пытаются делать на волнах хайпа) стали писать концепции и стратегии для разного вида суверенитета. Хотя после стольких лет построения глобалисткого мира (как видим теперь не очень-то удачного), строить суверенитеты многим странам крайне затруднительно.

Для небольших стран хотелось бы отметить следующее.

В условиях достаточно жесткого противостояния и противоборства нет смысла стремится к топовым с точки зрения глобальных экспертов, существующих, как обычно, на средства транснациональных компаний, решениям в области инфокоммуникационных технологий. Те же топовые чипы 2-3 нано зачем нужны на данном этапе для оборонки, государства, индустрии, социальной сферы и т.п. Можно и 90 нано обойтись. Не нужно путать требования потребительского рынка и требования для решения государственных, страновых задач. Тот же Apple, или Microsoft, или Samsung, создавая гаджеты и сервисы для конечного потребителя решают совершенно иные задачи. Им нужно удержать лидерство на рынке, чтобы сохранить миллиардные прибыли, Отсюда и 2-3 нано в чипах, и все новые и новые операционные системы, и все новые и новые приложения. Бег по кругу, новый системный и прикладной софт требует нового железа, новое железо требует нового софта -- и за все расплачивается конечный потребитель. А решение задачи сохранения суверенитета страны в текущей ситуации такой безудержной гонки совершенно и не требует. И чипы нужны попроще, но понадежнее, и операционные системы нужны не навороченные, и приложения нужны не такие уж многофункциональные. Классический пример с MS Word. Обычный пользователь использует не более 5\% возможностей редактора. А, например, госслужащим и им подобным и 5\% много. Точно также и с большинством софта. Нужны гораздо более простые, но стабильные и надежные системы. И тогда, чтобы создать такие системы нужны не сотни тысяч IT-пролетариата, а сотни, ну может быть тысячи высококвалифицированных разработчиков и относительно небольшое количество IT-пролетариев. Еще раз подчеркну, \myuline{страна для своего суверенитета} \myuline{и транснациональный IT-бизнес} \myuline{решают совершенно разные задачи и преследуют совершенно разные цели}. Можно еще напомнить, что массы IT-пролетариата бизнесу нужны и для того, чтобы раздувать значимость фирмы, создавать иллюзию гигантизма, особенно, если твоя фирма на IPO в Нью-Йорке или Лондоне, или хотя бы в Гонконге. Массовый инвестор больше поверит в фирму с десятками тысяч работников, чем в небольшую команду.

Но коль уж мы от цифровизации никуда не денемся, то \myuline{нужна цифровизация хорошо продуманная,} \myuline{просчитанная,} \myuline{с глубоким прогнозированием последствий}, а не безудержная, с агрессивной рекламой, что создает у людей совершенно ложные ориентиры. В подавляющем случае это делается опять же непрофессионально айтишным пролетариатом, втянутым в разработку на волне хайпа о неограниченной цифровизации всего и вся. А для того чтобы это понимать, молодежи нужно соответствующее образование. И выбора нет -- \myuline{либо мы такое} \myuline{образование очень} \myuline{оперативно сформируем,} \myuline{либо окончательно} \myuline{потеряем} \myuline{думающую} \myuline{молодежь,} \myuline{и, соответственно,} \myuline{окончательно} \myuline{потеряем} \myuline{и суверенитет} в условиях глобального перехода информационных технологий на принципиально новый уровень, требующий создания комплексов эффективно взаимодействующих компьютерных систем.
 
%\vspace{\baselineskip}
\begin{flushright}\noindent
\hfill {\it А.~Н. Курбацкий}\\
\hfill доктор технических наук, профессор,\\
\hfill заведующий кафедрой технологий программирования\\
\hfill факультета прикладной математики и информатики\\
\hfill Белорусского государственного университета
\end{flushright}

\end{partbacktext}