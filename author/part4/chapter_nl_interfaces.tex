\chapauthor{Никифоров С.А.\\Гойло А.А.\\Крощенко А.А.\\Захарьев В.А.\\Цянь Л.}
\chapter{Естественно-языковые интерфейсы ostis-систем}
\chapauthortoc{Никифоров С.А.\\Гойло А.А.\\Крощенко А.А.\\Захарьев В.А.\\Цянь Л.}
\label{chapter_nl_interfaces}

\abstract{
    В данной главе рассматривается подход к реализации естественно-языковых интерфейсов интеллектуальных компьютерных систем нового поколения, построенных по технологии OSTIS, а также предлагается модель контекста диалога.
    В данном подходе все этапы анализа, включая лексический, синтаксический и семантический анализ могут производиться непосредственно в базе знаний такой системы. Такой подход позволит эффективно решать такие задачи как управление глобальным и локальным контекстами диалога, а также разрешение языковых явлений таких как анафоры, омонимия и эллиптические фразы.

    \begin{itemize}
        \item расписать сопоставление токенов с лексемами;
        \item перерисоват ькартинки, чать из них предназначалась для размещения в колонке;
        \item расписать синтаксический анализ, как находятся составляющие,
        \item пройтись по подсказкам от плагина, поправит форматирование,
        \item формализовать пр. о..
    \end{itemize}
    Список вопросов:
    \begin{itemize}
        \item что-то придумать с аналогичными картинками, как те, что были использованы в главе про языки - дублировать в одной книге плохо, но ссылаться чрез 100 страниц тоже.
    \end{itemize}
}

%Введение
В настоящее время существует большое количество различных интерфейсов компьютерных систем, что усложняет интероперабельность между такими системами и людьми в силу необходимости ознакомления с интерфейсом каждой новой системы, который зачастую может быть не интуитивно понятен.

Одной из основных особенностей интеллектуальных компьютерных систем нового поколения должен являться пользовательский интерфейс, способный обеспечить эффективное взаимодействие пользователя с системой в условиях его общей профессиональной неподготовленности.

Одной из наиболее естественных и удобных форм передачи информации между людьми является речь, что обуславливает все большее распространение естественно-языковых интерфейсов\scncite{GlobalMarket}. В настоящий момент времени уже ни у кого не вызывает сомнения, что данная форма взаимодействия человека и машины играет и будет играть значительную роль во взаимодействии с различными компьютерными системами.

Однако необходимо отметить, что большое многообразие языков (как естественных, так и искусственных) ведет к необходимости упрощения процесса создания таких интерфейсов для каждого отдельно взятого языка.

%Анализ
В основе большинства подходов к обработке и пониманию естественного языка лежит машинное обучение\scncite{NLP_as_a_service},\scncite{NLP_in_pharmacology}. Несомненно, для большинства широко распространенных языков модели для обработки естественного языка работают очень хорошо и совершенствуются с каждым днем, но несмотря на успехи в данной области, данный подход имеет ряд недостатков:
\begin{itemize}
    \item проблемы при работе с различными областями, например, значения слов или предложений могут быть различными в зависимости от предметной области. Таким образом, модели для NLP могут хорошо работать для отдельной предметной области, но не подходить для широкого применения\scncite{NLPOverview};
    \item создание новой модели модели требует наличия большого объема данных, а качество таких данных напрямую влияет на качество получаемой модели, что ведет к большим затратам на ее обучение\scncite{strubell2019energy}\scncite{large_language_models};
    \item данные модели представляют собой ``черный ящик'', т. к. данные модели не обладают средствами для обоснования своего вывода;
    \item каждая такая модель решает только свой узкий класс задач, отсутствует общий подход к обработке естественного языка.\scncite{NLPOverview}
\end{itemize}

Данные недостатки используемых методов являются причиной части недостатков современных систем, реализующих естественно-языковой интерфейс, так, несмотря на то, что сейчас существует большое количество речевых ассистентов, создаваемых разными компаниями\scncite{site_url_alexa},\scncite{site_url_siri},\scncite{site_url_gassist},\scncite{site_url_cortana}, они обладают схожими недостатками, например, исключительно распределенной реализацией, в силу недостаточной для запуска ресурсоемких моделей производительности устройств конечных пользователей. Это в свою очередь ведет к проблемам с приватностью\scncite{PVA}.

Подмодуль понимания речи данных систем формирует конструкцию, отражающую смысл сообщения используя фреймовую модель. Упрощенный пример такой конструкции приведен на рисунке \textit{\nameref{fig:message_intents}}.

\begin{figure}[h]
    \centerline{\includegraphics[width=\linewidth]{images/part4/chapter_nl_interfaces/message_intents.png}}
    \caption{Пример формализованного смысла сообщения}
    \label{fig:message_intents}
\end{figure}

При этом для представления результатов промежуточных этапов обработки используются иные форматы, модули которые их реализуют не имеют какой-либо единой основы и взаимодействуют посредством специализированных программных интерфейсов между ними, что приводит к несовместимости способов представления результатов на различных этапах обработки и конечного результата обработки текстов. Данная несовместимость в свою очередь ведет к существенным накладным расходам при разработке такой системы и в особенности при ее модификации.

В качестве решения проблемы совместимости предлагается использование подхода к обработке естественного языка на основе его формальной модели в виде набора онтологий, сформированных с использованием универсальных средств представления знаний, что будет способствовать интероперабельности как компонента по обработке естественного языка в целом с другими компонентами системы, так и между составляющими самого данного компонента.

Целью главы является формирование модели интерфейса, в основе которой лежит подход к обработке естественного языка на основе онтологий, содержащих формальное описание естественного языка.

\section{Предметная область и онтология естественно-языковых интерфейсов ostis-систем}

\textit{Естественно-языковой интерфейс} -- SILK-интерфейс (Speech – речь, Image – образ, Language – язык, Knowledge – знание), обмен информацией между компьютерной системой и пользователем в котором происходит за счёт диалога. Диалог ведётся на одном из естественных языков.

\begin{SCn}

    \scnheader{естественно-языковой интерфейс}
    \scnsuperset{речевой интерфейс}

\end{SCn}

\textit{Речевой интерфейс} -- SILK-интерфейс, обмен информацией в котором происходит за счёт диалога, в процессе которого компьютерная система и пользователь общаются с помощью речи. Данный вид интерфейса наиболее приближен к естественному общению между людьми.

В предлагаемом подходе можно выделить следующие этапы обработки естественного языка:
\begin{itemize}
    \item лексический анализ;
    \item синтаксический анализ;
    \item понимание сообщения.
\end{itemize}

В свою очередь, лексический анализ в включает в себя декомпозицию текста на токены и их сопоставление с лексемами.

Понимание сообщения сводится к генерации вариантов значения сообщения и выбору из них корректного на основании контекста, а также погружение его в данный контекст.

Ниже приведена структура решателя задач естественно-языкового интерфейса.

\begin{SCn}

    \scnheader{Решатель задач естественно-языкового интерфейса}
    \begin{scnrelfromset}{декомпозиция абстрактного sc-агента}
        \scnitem{Абстрактный sc-агент лексического анализа}
        \begin{scnindent}
            \begin{scnrelfromset}{декомпозиция абстрактного sc-агента}
                \scnitem{Абстрактный sc-агент декомпозиции текста на токены}
                \scnitem{Абстрактный sc-агент сопоставления токенов с лексемами}
            \end{scnrelfromset}
        \end{scnindent}
        \scnitem{Абстрактный sc-агент синтаксического анализа}
        \scnitem{Абстрактный sc-агент понимания сообщения}
    \end{scnrelfromset}

\end{SCn}

В свою очередь, \textit{Абстрактный sc-агент понимания сообщения} декомпозируется на:

\begin{SCn}

    \scnheader{Агент понимания сообщения}
    \begin{scnrelfromset}{декомпозиция абстрактного sc-агента}
        \scnitem{Абстрактный sc-агент генерации вариантов значения сообщения}
        \scnitem{Абстрактный sc-агент выбора и обновления контекста}
        \begin{scnindent}
            \begin{scnrelfromset}{декомпозиция абстрактного sc-агента}
                \scnitem{Абстрактный sc-агент разрешения контекста}
                \scnitem{Абстрактный sc-агент выбора смысла сообщения на основе контекста}
                \scnitem{Абстрактный sc-агент погружения сообщения в контекст}
            \end{scnrelfromset}
        \end{scnindent}
    \end{scnrelfromset}

\end{SCn}

Для каждого агента в базе знаний должна находиться спецификация, пример фрагмента такой спецификации приведен на рисунке \textit{\nameref{fig:agent_spec}}.

\begin{figure}[h]
    \centering
    \includegraphics[width=0.4\textwidth]{images/part4/chapter_nl_interfaces/agent_spec.png}
    \caption{Пример спецификации агента.}
    \label{fig:agent_spec}
\end{figure}

\section{Предметная область и онтология лексического анализа естественно-языковых сообщений, входящих в ostis-систему}

\begin{SCn}

    \scnheader{действие. лексический анализ естественно-языкового сообщения}
    \begin{scnrelfromset}{обобщенная декомпозиция}
        \scnitem{действие. декомпозиция текста на токены}
        \scnitem{действие. сопоставление токенов с лексемами}
    \end{scnrelfromset}

\end{SCn}

С точки зрения ostis-системы, любой естественно-языковой текст является \textit{файлом} (т.е. SC-узлом с содержимым).

Этап лексического анализа представляет собой декомпозицию текста на последовательность токенов и сопоставление лексем с получившимися при данной декомпозиции токенами. Следует отметить, что данные токены при необходимости могут сопоставляться не с лексемами, а с их подмножествами, входящими в ее морфологическую парадигму, соответствующими определенным грамматическим категориям: падежу, числу, роду и т.д.

Результат лексического анализа представлен на рисунке \textit{\nameref{fig:lexical_result}}.

\begin{figure}[h]
    \centering
    \includegraphics[width=0.4\textwidth]{images/part4/chapter_nl_interfaces/lexical.png}
    \caption{Пример результата лексического анализа.}
    \label{fig:lexical_result}
\end{figure}

Для осуществления лексического анализа, в базе знаний системы также должен присутствовать словарь, содержащий лексемы и их различные формы.

Под лексемой понимается единица словарного состава языка, которая представляет собой множество всех форм некоторого слова.
Пример спецификации лексемы в базе знаний приведен на рисунке \textit{\nameref{fig:lexeme_example}}.

\begin{figure}[h]
    \centering
    \includegraphics[width=0.4\textwidth]{images/part4/chapter_nl_interfaces/lexeme_example.png}
    \caption{Пример спецификации лексемы в базе знаний.}
    \label{fig:lexeme_example}
\end{figure}

\section{Предметная область и онтология синтаксического анализа естественно-языковых сообщений, входящих в ostis-систему}

Агент синтаксического анализа выполняет переход от размеченного на лексемы текста к его синтаксической структуре.
При этом из-за невозможности разрешения структурной неоднозначности на этапе синтаксического анализа, его результатом в общем случае будет являться множество потенциальных синтаксических структур.

Пример одной синтаксической структуры представлен на рисунке \textit{\nameref{fig:syntactic_result}}.

\begin{figure*}[h]
    \centering
    \includegraphics[width=\textwidth]{images/part4/chapter_nl_interfaces/syntactic.png}
    \caption{Пример синтаксической структуры.}
    \label{fig:syntactic_result}
\end{figure*}


\section{Предметная область и онтология понимания естественно-языковых сообщений, входящих в ostis-систему}

\begin{SCn}

    \scnheader{действие. понимание естественно-языкового сообщения}
    \begin{scnrelfromset}{обобщенная декомпозиция}
        \scnitem{действие. генерация вариантов значения сообщения}
        \scnitem{действие. выбор и обновление контекста}
        \begin{scnindent}
            \begin{scnrelfromset}{обобщенная декомпозиция}
                \scnitem{действие. разрешение контекста}
                \scnitem{действие. выбор смысла сообщения на основе контекста}
                \scnitem{действие. погружение сообщения в контекст}
            \end{scnrelfromset}
        \end{scnindent}
    \end{scnrelfromset}

\end{SCn}

\textit{Действие. генерация вариантов значения сообщения} -- действие, в ходе которого осуществляется формирование строгой дизъюнкции потенциально эквивалентных структур.

\textit{Потенциально эквивалентная структура*} -- бинарное ориентированное отношение, связывающее структуру и множество структур, которые потенциально могут быть эквивалентны ей, однако для достоверного определения факта требуются дополнительные действия.

При этом, переход от результата синтаксического анализа к потенциально эквивалентным сообщению структурам осуществляется по правилам, содержащихся в предметной области денотационной семантики. Пример одного из правил представлен на рисунке \textit{\nameref{fig:transition_to_semanic_rule}}.

\begin{figure*}[h]
    \centering
    \includegraphics[width=0.7\textwidth]{images/part4/chapter_nl_interfaces/d_sem_3.png}
    \caption{Пример правила перехода от синтаксической структуры к семантике.}
    \label{fig:transition_to_semanic_rule}
\end{figure*}

В результате данного действия в базе знаний формируется структура, описывающая возможные варианты смысла сообщения, пример такой структуры в терминах грамматики составляющих\scncite{X_bar_syntax} приведен на \textit{\nameref{fig:messsage_meaning_variants}}. Наличие нескольких таких структур объясняется тем, что в общем случае на этапе синтаксического анализа выполняется генерация нескольких вариантов синтаксической структуры. Выбор корректного значения сообщения будет осуществлен в ходе выполнения последующих действий.

\begin{figure}[h]
    \centering
    \includegraphics[width=0.45\textwidth]{images/part4/chapter_nl_interfaces/messsage_meaning_variants.png}
    \caption{Пример конструкции, описывающей потенциальные смыслы сообщения.}
    \label{fig:messsage_meaning_variants}
\end{figure}

Следует отметить, что при необходимости смысл сообщения может быть сгенерирован не только на основании его синтаксической структуры в терминах грамматики составляющих, но и других знаний о данном сообщении, например выделенных из текста данного сообщения троек вида субъект-отношение-объект, результата его классификации и т. п.

Дальнейшие этапы процесса понимания сообщения выполняются на основе контекста.

\textit{Контекст} - sc-структура, содержащая знания, которыми оперирует система в ходе одного или нескольких диалогов.
В общем случае, данные знания включают в себя как предварительно занесенные в БЗ, так и полученные в ходе работы с сенсоров и/или диалога.

\begin{SCn}

    \scnheader{контекст диалога}
    \scnsubset{контекст}
    \scnrelfrom{subdividing}{\scnkeyword{Типология контекстов диалога по глобальности\scnsupergroupsign}}
    \begin{scnindent}
        \begin{scneqtoset}
            \scnitem{тематический контекст}
            \scnitem{пользовательский контекст}
            \scnitem{глобальный контекст}
        \end{scneqtoset}
    \end{scnindent}

\end{SCn}

\textit{Тематический контекст} -- контекст диалога, содержащий специфические для темы сведения (сведения, полученные во время ведения диалога, на определенную тематику, например, при диалоге об определенном наборе сущностей).

\textit{Множество тематических контекстов диалога*} -- бинарное ориентированное отношение, диалог с ориентированным множеством его тематических контекстов.

\textit{Пользовательский контекст} -- контекст диалога, содержащие специфические для пользователя сведения, которые могут быть использованы в диалоге с ним на любую тематику. В общем случае пользовательский контекст имеет пересечение с согласованной частью БЗ (предварительно занесенная в БЗ достоверная информация о пользователе, прошедшая необходимую модерацию), но не включается в нее целиком (часть, полученная в ходе диалога в которой мы не уверены).
Пример соотнесения различных типов контекстов с согласованной частью базы знаний приведен на рисунке \textit{\nameref{fig:context_in_KB}}.

\textit{Глобальный контекст} -- контекст диалога, содержащий сведения, которые могут быть необходимы при ведении диалога с любым пользователем. Глобальный контекст -- подмножество согласованной части БЗ, содержащее те сведения, что допустимо использовать в диалоге. Например, в диалоге с определенным пользователем не нужно использовать:
\begin{itemize}
    \item находящуюся в базе знаний служебную информацию, необходимую для работы системы, но не предназначенную для использования в диалоге;
    \item части пользовательских контекстов иных пользователей.
\end{itemize}

\begin{figure}[h]
    \centering
    \includegraphics[width=0.4\textwidth]{images/part4/chapter_nl_interfaces/context_in_KB.png}
    \caption{Соотношение контекстов с согласованной частью баз знаний.}
    \label{fig:context_in_KB}
\end{figure}

\begin{SCn}

    \scnheader{контекст диалога}
    \scnrelfrom{subdividing}{\scnkeyword{Типология контекство по сроку достоверности знаний\scnsupergroupsign}}
    \begin{scnindent}
        \begin{scneqtoset}
            \scnitem{неизменяемый в ходе работы системы контекст диалога}
            \scnitem{изменяемый в ходе работы системы контекст диалога}
        \end{scneqtoset}
    \end{scnindent}

\end{SCn}

\textit{Неизменяемый в ходе работы системы контекст диалога} содержит в себе знания, необходимые для обеспечения выполнения системой своих функций,  которые были заложены в нее априорно ее разработчиками и/или администраторами и не изменяются в ходе ее функционирования на постоянной основе.

\textit{Изменяемый в ходе работы системы контекст диалога} содержит в себе знания, необходимые для обеспечения выполнения системой своих функций,  которые были ей получены в ходе ее работы и/или достоверность которых скоротечна.

\begin{SCn}


    \scnheader{изменяемый в ходе работы системы контекст диалога}
    \scnrelfrom{subdividing}{\scnkeyword{Типология изменяемых в ходе работы системы контекстов по источнику знаний\scnsupergroupsign}}
    \begin{scnindent}
        \begin{scneqtoset}
            \scnitem{контекст диалога, содержащий знания из внешних источников}
            \scnitem{контекст диалога, содержащий знания, полученные в ходе диалога}
        \end{scneqtoset}
    \end{scnindent}
    \scnrelfrom{subdividing}{\scnkeyword{Типология изменяемых контекстов по степени их достоверности\scnsupergroupsign}}
    \begin{scnindent}
        \begin{scneqtoset}
            \scnitem{достоверный контекст диалога}
            \scnitem{недостоверный контекст диалога}
        \end{scneqtoset}
    \end{scnindent}

\end{SCn}

Подмножество контекста может включаться в согласованную часть БЗ, например, если речь идет о каких-то предварительно занесенных в БЗ биографических сведениях -- дате рождения и т. п.

В каждый момент времени с пользователем связан 1 пользовательский диалоговый контекст (содержащий, по крайней мере известные заранее факты о нем: имя, возраст и т. п.) и несколько тематических.
Пример спецификации контекстов представлен на рисунке \textit{\nameref{fig:user_context}}.

\begin{figure}[h]
    \centering
    \includegraphics[width=0.5\textwidth]{images/part4/chapter_nl_interfaces/user_context.png}
    \caption{Пример спецификации контекстов.}
    \label{fig:user_context}
\end{figure}

Так, \textbf{действие. разрешение контекста} сводится к сопоставлению каждому варианту его значения соответствующего контекста.
Выбор производится на основании значения функции $F_{CTD}(T, C)$, где T - вариант трансляции, C - тематический контекст.
Подходящим контекстом для варианта трансляции считается тот, для которого значение этой функции максимально.
В случае, если подходящий контекст не найден, генерируется новый.
Пример результата данного действия представлен на рисунке \textit{\nameref{fig:relevant_contexts}}.

\begin{figure}[h]
    \centering
    \includegraphics[width=0.45\textwidth]{images/part4/chapter_nl_interfaces/relevant_contexts.png}
    \caption{Пример сообщения, всем вариантам значения которого сопоставлен контекст.}
    \label{fig:relevant_contexts}
\end{figure}

\textbf{Действие. выбор смысла сообщения} представляет собой выбор из множества вариантов трансляции и соответствующих им контекстов одной пары и обозначение ее как эквивалентной сообщению конструкции. В простейшем случае, на данном этапе допустимо выполнить выбор в соответствии с рассчитанными на предыдущем этапе для пар потенциально эквивалентных структур и соответствующих им контекстов значениями функции $F_{CTD}(T, C)$ и выбрать пару, для которой оно максимально, однако при необходимости также возможно введение и отдельной функции.
Пример результата данного действия представлен на рисунке \textit{\nameref{fig:message_equivalent_structure}}.

\begin{figure}[h]
    \centering
    \includegraphics[width=0.35\textwidth]{images/part4/chapter_nl_interfaces/message_equivalent_structure.png}
    \caption{Пример конструкции, описывающей эквивалентную сообщению структуру.}
    \label{fig:message_equivalent_structure}
\end{figure}

\textbf{Действие. погружение сообщения в контекст} представляет собой погружение полученного смысла сообщения в контекст.
Кроме выбранного смысла сообщения, в контекст может добавляться и иная необходимая для обработки сообщения информация.
Кроме того, на данном этапе на основе хранящихся в контексте сведений также должно выполняться разрешение местоимений.
Примеры контекста до погружения в него сообщения и после погружения представлены на рисунках \textit{\nameref{fig:context_before_update}} и \textit{\nameref{fig:updated_context}}.

\begin{figure}[h]
    \centering
    \includegraphics[width=0.4\textwidth]{images/part4/chapter_nl_interfaces/context_1.png}
    \caption{Пример контекста до погружения сообщения.}
    \label{fig:context_before_update}
\end{figure}

\begin{figure*}[h]
    \centering
    \includegraphics[width=0.7\textwidth]{images/part4/chapter_nl_interfaces/context_2.png}
    \caption{Пример контекста после погружения сообщения.}
    \label{fig:updated_context}
\end{figure*}

Таким образом, актуальная информация собирается в тематический контекст, объединив который с контекстом пользователя и глобальным контекстом можно получить общий контекст, на основании которого должны осуществляться требуемые действия системы, включая генерацию ответа системы.

%Под вопросом
\section{Предметная область и онтология синтеза естественно-языковых сообщений ostis-системы}
\section{Модели, методы и средства адаптации пользовательских интерфейсов к носителям китайского языка}
\label{section_chinese_interfaces}

%\input{author/references}