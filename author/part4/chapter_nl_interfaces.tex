\chapauthor{Никифоров С.А.\\Гойло А.А.\\Крощенко А.А.\\Захарьев В.А.\\Цянь Л.}
\chapter{Естественно-языковые интерфейсы ostis-систем}
\chapauthortoc{Никифоров С.А.\\Гойло А.А.\\Крощенко А.А.\\Захарьев В.А.\\Цянь Л.}
\label{chapter_nl_interfaces}

\abstract{
    В данной главе рассматривается подход к реализации естественно-языковых интерфейсов интеллектуальных компьютерных систем нового поколения, построенных по технологии OSTIS, а также предлагается модель контекста диалога.
    В данном подходе все этапы анализа, включая лексический, синтаксический и семантический анализ могут производиться непосредственно в базе знаний такой системы. Такой подход позволит эффективно решать такие задачи как управление глобальным и локальным контекстами диалога, а также разрешение языковых явлений таких как анафоры, омонимия и эллиптические фразы.

    \begin{textitemize}
        \item расписать сопоставление токенов с лексемами;
        \item перерисоват ькартинки, чать из них предназначалась для размещения в колонке;
        \item расписать синтаксический анализ, как находятся составляющие,
        \item пройтись по подсказкам от плагина, поправит форматирование,
        \item формализовать пр. о..
    \end{textitemize}
    Список вопросов:
    \begin{textitemize}
        \item что придумать с аналогичными картинками, как те, что были использованы в главе про языки - дублировать в одной книге плохо, но ссылаться чрез 100 страниц тоже.
    \end{textitemize}
}

%Введение
В настоящее время существует большое количество различных интерфейсов компьютерных систем, что усложняет интероперабельность между такими системами и людьми в силу необходимости ознакомления с интерфейсом каждой новой системы, который зачастую может быть не интуитивно понятен.

Одной из основных особенностей интеллектуальных компьютерных систем нового поколения должен являться пользовательский интерфейс, способный обеспечить эффективное взаимодействие пользователя с системой в условиях его общей профессиональной неподготовленности.

Одной из наиболее естественных и удобных форм передачи информации между людьми является речь, что обуславливает все большее распространение естественно-языковых интерфейсов\scncite{GlobalMarket}. В настоящий момент времени уже ни у кого не вызывает сомнения, что данная форма взаимодействия человека и машины играет и будет играть значительную роль во взаимодействии с различными компьютерными системами.

Однако необходимо отметить, что большое многообразие языков (как естественных, так и искусственных) ведет к необходимости упрощения процесса создания таких интерфейсов для каждого отдельно взятого языка.

%Анализ
В основе большинства подходов к обработке и пониманию естественного языка лежит машинное обучение\scncite{NLP_as_a_service},\scncite{NLP_in_pharmacology}. Несомненно, для большинства широко распространенных языков модели для обработки естественного языка работают очень хорошо и совершенствуются с каждым днем, но несмотря на успехи в данной области, данный подход имеет ряд недостатков:
\begin{textitemize}
    \item проблемы при работе с различными областями, например, значения слов или предложений могут быть различными в зависимости от предметной области. Таким образом, модели для NLP могут хорошо работать для отдельной предметной области, но не подходить для широкого применения\scncite{NLPOverview};
    \item создание новой модели модели требует наличия большого объема данных, а качество таких данных напрямую влияет на качество получаемой модели, что ведет к большим затратам на ее обучение\scncite{strubell2019energy}\scncite{large_language_models};
    \item данные модели представляют собой "черный ящик"{}, т. к. данные модели не обладают средствами для обоснования своего вывода;
    \item каждая такая модель решает только свой узкий класс задач, отсутствует общий подход к обработке естественного языка.\scncite{NLPOverview}
\end{textitemize}

Данные недостатки используемых методов являются причиной части недостатков современных систем, реализующих естественно-языковой интерфейс, так, несмотря на то, что сейчас существует большое количество речевых ассистентов, создаваемых разными компаниями\scncite{site_url_alexa},\scncite{site_url_siri},\scncite{site_url_gassist},\scncite{site_url_cortana}, они обладают схожими недостатками, например, исключительно распределенной реализацией, в силу недостаточной для запуска ресурсоемких моделей производительности устройств конечных пользователей. Это в свою очередь ведет к проблемам с приватностью\scncite{PVA}.

Подмодуль понимания речи данных систем формирует конструкцию, отражающую смысл сообщения используя фреймовую модель. Упрощенный пример такой конструкции приведен на рисунке \textit{\nameref{fig:message_intents}}.

\begin{figure}[h]
    \centerline{\includegraphics[width=\linewidth]{images/part4/chapter_nl_interfaces/message_intents.png}}
    \caption{Пример формализованного смысла сообщения}
    \label{fig:message_intents}
\end{figure}

При этом для представления результатов промежуточных этапов обработки используются иные форматы, модули которые их реализуют не имеют какой-либо единой основы и взаимодействуют посредством специализированных программных интерфейсов между ними, что приводит к несовместимости способов представления результатов на различных этапах обработки и конечного результата обработки текстов. Данная несовместимость в свою очередь ведет к существенным накладным расходам при разработке такой системы и в особенности при ее модификации.

В качестве решения проблемы совместимости предлагается использование подхода к обработке естественного языка на основе его формальной модели в виде набора онтологий, сформированных с использованием универсальных средств представления знаний, что будет способствовать интероперабельности как компонента по обработке естественного языка в целом с другими компонентами системы, так и между составляющими самого данного компонента.

Целью главы является формирование модели интерфейса, в основе которой лежит подход к обработке естественного языка на основе онтологий, содержащих формальное описание естественного языка.

\section{Предметная область и онтология естественно-языковых интерфейсов ostis-систем}

\textit{Естественно-языковой интерфейс} -- SILK-интерфейс (Speech – речь, Image – образ, Language – язык, Knowledge – знание), обмен информацией между компьютерной системой и пользователем в котором происходит за счёт диалога. Диалог ведётся на одном из естественных языков.

\begin{SCn}

    \scnheader{естественно-языковой интерфейс}
    \scnsuperset{речевой интерфейс}

\end{SCn}

\textit{Речевой интерфейс} -- SILK-интерфейс, обмен информацией в котором происходит за счёт диалога, в процессе которого компьютерная система и пользователь общаются с помощью речи. Данный вид интерфейса наиболее приближен к естественному общению между людьми.

В предлагаемом подходе можно выделить следующие этапы обработки естественного языка:
\begin{textitemize}
    \item лексический анализ;
    \item синтаксический анализ;
    \item понимание сообщения.
\end{textitemize}

В свою очередь, лексический анализ в включает в себя декомпозицию текста на токены и их сопоставление с лексемами.

Понимание сообщения сводится к генерации вариантов значения сообщения и выбору из них корректного на основании контекста, а также погружение его в данный контекст.

Ниже приведена структура решателя задач естественно-языкового интерфейса.

\begin{SCn}

    \scnheader{Решатель задач естественно-языкового интерфейса}
    \begin{scnrelfromset}{декомпозиция абстрактного sc-агента}
        \scnitem{Абстрактный sc-агент лексического анализа}
        \begin{scnindent}
            \begin{scnrelfromset}{декомпозиция абстрактного sc-агента}
                \scnitem{Абстрактный sc-агент декомпозиции текста на токены}
                \scnitem{Абстрактный sc-агент сопоставления токенов с лексемами}
            \end{scnrelfromset}
        \end{scnindent}
        \scnitem{Абстрактный sc-агент синтаксического анализа}
        \scnitem{Абстрактный sc-агент понимания сообщения}
    \end{scnrelfromset}

\end{SCn}

В свою очередь, \textit{Абстрактный sc-агент понимания сообщения} декомпозируется на:

\begin{SCn}

    \scnheader{Агент понимания сообщения}
    \begin{scnrelfromset}{декомпозиция абстрактного sc-агента}
        \scnitem{Абстрактный sc-агент генерации вариантов значения сообщения}
        \scnitem{Абстрактный sc-агент выбора и обновления контекста}
        \begin{scnindent}
            \begin{scnrelfromset}{декомпозиция абстрактного sc-агента}
                \scnitem{Абстрактный sc-агент разрешения контекста}
                \scnitem{Абстрактный sc-агент выбора смысла сообщения на основе контекста}
                \scnitem{Абстрактный sc-агент погружения сообщения в контекст}
            \end{scnrelfromset}
        \end{scnindent}
    \end{scnrelfromset}

\end{SCn}

Для каждого агента в базе знаний должна находиться спецификация, пример фрагмента такой спецификации приведен на рисунке \textit{\nameref{fig:agent_spec}}.

\begin{figure}[h]
    \centering
    \includegraphics[width=0.4\textwidth]{images/part4/chapter_nl_interfaces/agent_spec.png}
    \caption{Пример спецификации агента.}
    \label{fig:agent_spec}
\end{figure}

\section{Предметная область и онтология лексического анализа естественно-языковых сообщений, входящих в ostis-систему}

\begin{SCn}

    \scnheader{действие. лексический анализ естественно-языкового сообщения}
    \begin{scnrelfromset}{обобщенная декомпозиция}
        \scnitem{действие. декомпозиция текста на токены}
        \scnitem{действие. сопоставление токенов с лексемами}
    \end{scnrelfromset}

\end{SCn}

С точки зрения ostis-системы, любой естественно-языковой текст является \textit{файлом} (т.е. SC-узлом с содержимым).

Этап лексического анализа представляет собой декомпозицию текста на последовательность токенов и сопоставление лексем с получившимися при данной декомпозиции токенами. Следует отметить, что данные токены при необходимости могут сопоставляться не с лексемами, а с их подмножествами, входящими в ее морфологическую парадигму, соответствующими определенным грамматическим категориям: падежу, числу, роду и т.д.

Результат лексического анализа представлен на рисунке \textit{\nameref{fig:lexical_result}}.

\begin{figure}[h]
    \centering
    \includegraphics[width=0.4\textwidth]{images/part4/chapter_nl_interfaces/lexical.png}
    \caption{Пример результата лексического анализа.}
    \label{fig:lexical_result}
\end{figure}

Для осуществления лексического анализа, в базе знаний системы также должен присутствовать словарь, содержащий лексемы и их различные формы.

Под лексемой понимается единица словарного состава языка, которая представляет собой множество всех форм некоторого слова.
Пример спецификации лексемы в базе знаний приведен на рисунке \textit{\nameref{fig:lexeme_example}}.

\begin{figure}[h]
    \centering
    \includegraphics[width=0.4\textwidth]{images/part4/chapter_nl_interfaces/lexeme_example.png}
    \caption{Пример спецификации лексемы в базе знаний.}
    \label{fig:lexeme_example}
\end{figure}

\section{Предметная область и онтология синтаксического анализа естественно-языковых сообщений, входящих в ostis-систему}

Агент синтаксического анализа выполняет переход от размеченного на лексемы текста к его синтаксической структуре.
При этом из-за невозможности разрешения структурной неоднозначности на этапе синтаксического анализа, его результатом в общем случае будет являться множество потенциальных синтаксических структур.

Пример одной синтаксической структуры представлен на рисунке \textit{\nameref{fig:syntactic_result}}.

\begin{figure*}[h]
    \centering
    \includegraphics[width=\textwidth]{images/part4/chapter_nl_interfaces/syntactic.png}
    \caption{Пример синтаксической структуры.}
    \label{fig:syntactic_result}
\end{figure*}


\section{Предметная область и онтология понимания естественно-языковых сообщений, входящих в ostis-систему}

\begin{SCn}

    \scnheader{действие. понимание естественно-языкового сообщения}
    \begin{scnrelfromset}{обобщенная декомпозиция}
        \scnitem{действие. генерация вариантов значения сообщения}
        \scnitem{действие. выбор и обновление контекста}
        \begin{scnindent}
            \begin{scnrelfromset}{обобщенная декомпозиция}
                \scnitem{действие. разрешение контекста}
                \scnitem{действие. выбор смысла сообщения на основе контекста}
                \scnitem{действие. погружение сообщения в контекст}
            \end{scnrelfromset}
        \end{scnindent}
    \end{scnrelfromset}

\end{SCn}

\textit{Действие. генерация вариантов значения сообщения} -- действие, в ходе которого осуществляется формирование строгой дизъюнкции потенциально эквивалентных структур.

\textit{Потенциально эквивалентная структура*} -- бинарное ориентированное отношение, связывающее структуру и множество структур, которые потенциально могут быть эквивалентны ей, однако для достоверного определения факта требуются дополнительные действия.

При этом, переход от результата синтаксического анализа к потенциально эквивалентным сообщению структурам осуществляется по правилам, содержащихся в предметной области денотационной семантики. Пример одного из правил представлен на рисунке \textit{\nameref{fig:transition_to_semanic_rule}}.

\begin{figure*}[h]
    \centering
    \includegraphics[width=0.7\textwidth]{images/part4/chapter_nl_interfaces/d_sem_3.png}
    \caption{Пример правила перехода от синтаксической структуры к семантике.}
    \label{fig:transition_to_semanic_rule}
\end{figure*}

В результате данного действия в базе знаний формируется структура, описывающая возможные варианты смысла сообщения, пример такой структуры в терминах грамматики составляющих\scncite{X_bar_syntax} приведен на \textit{\nameref{fig:messsage_meaning_variants}}. Наличие нескольких таких структур объясняется тем, что в общем случае на этапе синтаксического анализа выполняется генерация нескольких вариантов синтаксической структуры. Выбор корректного значения сообщения будет осуществлен в ходе выполнения последующих действий.

\begin{figure}[h]
    \centering
    \includegraphics[width=0.45\textwidth]{images/part4/chapter_nl_interfaces/messsage_meaning_variants.png}
    \caption{Пример конструкции, описывающей потенциальные смыслы сообщения.}
    \label{fig:messsage_meaning_variants}
\end{figure}

Следует отметить, что при необходимости смысл сообщения может быть сгенерирован не только на основании его синтаксической структуры в терминах грамматики составляющих, но и других знаний о данном сообщении, например выделенных из текста данного сообщения троек вида субъект-отношение-объект, результата его классификации и т. п.

Дальнейшие этапы процесса понимания сообщения выполняются на основе контекста.

\textit{Контекст} - sc-структура, содержащая знания, которыми оперирует система в ходе одного или нескольких диалогов.
В общем случае, данные знания включают в себя как предварительно занесенные в БЗ, так и полученные в ходе работы с сенсоров и/или диалога.

\begin{SCn}

    \scnheader{контекст диалога}
    \scnsubset{контекст}
    \scnrelfrom{subdividing}{\scnkeyword{Типология контекстов диалога по глобальности\scnsupergroupsign}}
    \begin{scnindent}
        \begin{scneqtoset}
            \scnitem{тематический контекст}
            \scnitem{пользовательский контекст}
            \scnitem{глобальный контекст}
        \end{scneqtoset}
    \end{scnindent}

\end{SCn}

\textit{Тематический контекст} -- контекст диалога, содержащий специфические для темы сведения (сведения, полученные во время ведения диалога, на определенную тематику, например, при диалоге об определенном наборе сущностей).

\textit{Множество тематических контекстов диалога*} -- бинарное ориентированное отношение, диалог с ориентированным множеством его тематических контекстов.

\textit{Пользовательский контекст} -- контекст диалога, содержащие специфические для пользователя сведения, которые могут быть использованы в диалоге с ним на любую тематику. В общем случае пользовательский контекст имеет пересечение с согласованной частью БЗ (предварительно занесенная в БЗ достоверная информация о пользователе, прошедшая необходимую модерацию), но не включается в нее целиком (часть, полученная в ходе диалога в которой мы не уверены).
Пример соотнесения различных типов контекстов с согласованной частью базы знаний приведен на рисунке \textit{\nameref{fig:context_in_KB}}.

\textit{Глобальный контекст} -- контекст диалога, содержащий сведения, которые могут быть необходимы при ведении диалога с любым пользователем. Глобальный контекст -- подмножество согласованной части БЗ, содержащее те сведения, что допустимо использовать в диалоге. Например, в диалоге с определенным пользователем не нужно использовать:
\begin{itemize}
    \item находящуюся в базе знаний служебную информацию, необходимую для работы системы, но не предназначенную для использования в диалоге;
    \item части пользовательских контекстов иных пользователей.
\end{itemize}

\begin{figure}[h]
    \centering
    \includegraphics[width=0.4\textwidth]{images/part4/chapter_nl_interfaces/context_in_KB.png}
    \caption{Соотношение контекстов с согласованной частью баз знаний.}
    \label{fig:context_in_KB}
\end{figure}

\begin{SCn}

    \scnheader{контекст диалога}
    \scnrelfrom{subdividing}{\scnkeyword{Типология контекство по сроку достоверности знаний\scnsupergroupsign}}
    \begin{scnindent}
        \begin{scneqtoset}
            \scnitem{неизменяемый в ходе работы системы контекст диалога}
            \scnitem{изменяемый в ходе работы системы контекст диалога}
        \end{scneqtoset}
    \end{scnindent}

\end{SCn}

\textit{Неизменяемый в ходе работы системы контекст диалога} содержит в себе знания, необходимые для обеспечения выполнения системой своих функций,  которые были заложены в нее априорно ее разработчиками и/или администраторами и не изменяются в ходе ее функционирования на постоянной основе.

\textit{Изменяемый в ходе работы системы контекст диалога} содержит в себе знания, необходимые для обеспечения выполнения системой своих функций,  которые были ей получены в ходе ее работы и/или достоверность которых скоротечна.

\begin{SCn}


    \scnheader{изменяемый в ходе работы системы контекст диалога}
    \scnrelfrom{subdividing}{\scnkeyword{Типология изменяемых в ходе работы системы контекстов по источнику знаний\scnsupergroupsign}}
    \begin{scnindent}
        \begin{scneqtoset}
            \scnitem{контекст диалога, содержащий знания из внешних источников}
            \scnitem{контекст диалога, содержащий знания, полученные в ходе диалога}
        \end{scneqtoset}
    \end{scnindent}
    \scnrelfrom{subdividing}{\scnkeyword{Типология изменяемых контекстов по степени их достоверности\scnsupergroupsign}}
    \begin{scnindent}
        \begin{scneqtoset}
            \scnitem{достоверный контекст диалога}
            \scnitem{недостоверный контекст диалога}
        \end{scneqtoset}
    \end{scnindent}

\end{SCn}

Подмножество контекста может включаться в согласованную часть БЗ, например, если речь идет о каких-то предварительно занесенных в БЗ биографических сведениях -- дате рождения и т. п.

В каждый момент времени с пользователем связан 1 пользовательский диалоговый контекст (содержащий, по крайней мере известные заранее факты о нем: имя, возраст и т. п.) и несколько тематических.
Пример спецификации контекстов представлен на рисунке \textit{\nameref{fig:user_context}}.

\begin{figure}[h]
    \centering
    \includegraphics[width=0.5\textwidth]{images/part4/chapter_nl_interfaces/user_context.png}
    \caption{Пример спецификации контекстов.}
    \label{fig:user_context}
\end{figure}

Так, \textbf{действие. разрешение контекста} сводится к сопоставлению каждому варианту его значения соответствующего контекста.
Выбор производится на основании значения функции $F_{CTD}(T, C)$, где T - вариант трансляции, C - тематический контекст.
Подходящим контекстом для варианта трансляции считается тот, для которого значение этой функции максимально.
В случае, если подходящий контекст не найден, генерируется новый.
Пример результата данного действия представлен на рисунке \textit{\nameref{fig:relevant_contexts}}.

\begin{figure}[h]
    \centering
    \includegraphics[width=0.45\textwidth]{images/part4/chapter_nl_interfaces/relevant_contexts.png}
    \caption{Пример сообщения, всем вариантам значения которого сопоставлен контекст.}
    \label{fig:relevant_contexts}
\end{figure}

\textbf{Действие. выбор смысла сообщения} представляет собой выбор из множества вариантов трансляции и соответствующих им контекстов одной пары и обозначение ее как эквивалентной сообщению конструкции. В простейшем случае, на данном этапе допустимо выполнить выбор в соответствии с рассчитанными на предыдущем этапе для пар потенциально эквивалентных структур и соответствующих им контекстов значениями функции $F_{CTD}(T, C)$ и выбрать пару, для которой оно максимально, однако при необходимости также возможно введение и отдельной функции.
Пример результата данного действия представлен на рисунке \textit{\nameref{fig:message_equivalent_structure}}.

\begin{figure}[h]
    \centering
    \includegraphics[width=0.35\textwidth]{images/part4/chapter_nl_interfaces/message_equivalent_structure.png}
    \caption{Пример конструкции, описывающей эквивалентную сообщению структуру.}
    \label{fig:message_equivalent_structure}
\end{figure}

\textbf{Действие. погружение сообщения в контекст} представляет собой погружение полученного смысла сообщения в контекст.
Кроме выбранного смысла сообщения, в контекст может добавляться и иная необходимая для обработки сообщения информация.
Кроме того, на данном этапе на основе хранящихся в контексте сведений также должно выполняться разрешение местоимений.
Примеры контекста до погружения в него сообщения и после погружения представлены на рисунках \textit{\nameref{fig:context_before_update}} и \textit{\nameref{fig:updated_context}}.

\begin{figure}[h]
    \centering
    \includegraphics[width=0.4\textwidth]{images/part4/chapter_nl_interfaces/context_1.png}
    \caption{Пример контекста до погружения сообщения.}
    \label{fig:context_before_update}
\end{figure}

\begin{figure*}[h]
    \centering
    \includegraphics[width=0.7\textwidth]{images/part4/chapter_nl_interfaces/context_2.png}
    \caption{Пример контекста после погружения сообщения.}
    \label{fig:updated_context}
\end{figure*}

Таким образом, актуальная информация собирается в тематический контекст, объединив который с контекстом пользователя и глобальным контекстом можно получить общий контекст, на основании которого должны осуществляться требуемые действия системы, включая генерацию ответа системы.

%Под вопросом
\section{Предметная область и онтология синтеза естественно-языковых сообщений ostis-системы}
\section{Модели, методы и средства адаптации пользовательских интерфейсов к носителям китайского языка}
\label{section_chinese_interfaces}
Данный раздел посвящена рассмотрению разработки естествено-языкового интерфейса интеллектуальнных систем, ориентированного на решение задач преобразования текстов естественного языка в фрагменты базы знаний, и задач генерации текстов естественного языка из фрагментов базы знаний. Предложена семантическая модель естествено-языкового интерфейса, которая включает в себя модель базы знаний лингвистики, а также модель решателей задач естествено-языкового интерфейса, что позволяет проводить объединение лингвистических знаний на различных уровнях в единую базу знаний, а также глубокую интеграцию различных моделей решения задач для обработки естественного языка. Описание принципов построения китайско-языкового интерфейса осуществляется на основе предложенной единой модели естествено-языкового интерфейса.

\section{Прикладные применения естествено-языкового интерфейса}
В последнее время интеллектуальные системы, основанные на знаниях, широко применяются в различных областях (например, в медицине, в образовании и так далее). Как один из ключевых компонентов данных интеллектуальных систем, естественно-языковые интерфейсы призваны обеспечить более интеллектуальное взаимодействие между пользователями и интеллектуальными системами.

При реализации естественно-языкового интерфейса, необходимо рассмотреть две основные формы взаимодействия между пользователями и интеллектуальными системами на естественном языке:
\begin{itemize}
	\item форма текстов, которая означает взаимодействие, основанное на распознавании текстов естественного языка;
	\item форма речевых сообщений, которая означает взаимодействие, основанное на распознавании речи.
\end{itemize}

В настоящее время интерфейсы на основе распознавания речи широко применяется в различных интеллектуальных системах, например, современные голосовые помощники [\scncite{McTear2016}] (например, Siri, Xiaoai classmate и другие) реализуют взаимодействие человека с компьютерами на основе распознавания речи. Данные помощники в основном ориентированы на анализ и обработку речевой информации человека. Интерфейсы на основе распознавания текстов естественного языка, являются наиболее распространенными компонентами любых поисковых систем (например, Google, Yandex, Baidu и другие) или вопросно-ответных систем на основе базы знаний [\scncite{Wu2019}]. В отличие от голосовых помощников, данные интерфейсы, как правило, ориентированы на анализ и обработку упорядоченных строк текстов естественного языка. В обычных современных поисковых системах вопросно-ответные системы начали разрабатываться как их подсистемы, ориентированные на обработку вопросительных предложений (или декларативных предложений).

Естественно-языковые интерфейсы интеллектуальных систем, основанных на знаниях, ориентированы на достижение информационного обмена между текстами естественного языка (особенно декларативными предложениями), предоставляемыми человеческими пользователями, и базами знаний интеллектуальных систем, в которых хранятся конкретные знания. Модульная схема (рисунок \textit{\nameref{fig:schema-natural-interface}}) описывает общий автоматизированный процесс информационного взаимодействия между человеческими пользователями и интеллектуальными системами на основе классической архитектуры современных вопросно-ответных систем, основанных на знаниях. 

\begin{figure}[H]
	\includegraphics[scale=0.8,width=1.0\textwidth]{author/part4/figures/schema.png}
	\caption{Модульная схема естественно-языкового интерфейса интеллектуальных систем, основанных на знаниях}
	\label{fig:schema-natural-interface}
\end{figure}

Разработка естественно-языковых интерфейсов интеллектуальных систем, основанных на знаниях, в основном включает в себя реализацию следующих двух компонентов:
\begin{itemize}
	\item преобразование входных текстов естественного языка в фрагменты базы знаний интеллектуальных систем (приобретение/извлечение фактографических знаний);
	\item генерация текстов естественного языка из фрагментов базы знаний интеллектуальных систем (генерация текстов естественного языка).
\end{itemize}

Как приобретение фактографических знаний, так и генерация текстов естественного языка считаются подзадачами обработки естественного языка. С точки зрения интеллектуальных систем, данные подзадачи представляют собой разновидности так называемых комплексных задач, решение которых до сих пор является актуальной темой исследований. В целом, решение данных подзадач требует сочетание различных типов лингвистических знаний и интеграция различных моделей решения задач по обработке естественного языка. Однако в современных интеллектуальных системах отсутствует единая основа для интеграции различных типов лингвистических знаний и моделей решения задач в процессе разработки естественно-языковых интерфейсов.

Разработка естественно-языковых интерфейсов для современных интеллектуальных систем, основанных на знаниях, в общем случае требует рассмотрения следующих двух основных аспектов:
\begin{itemize}
	\item типы обрабатываемого естественного языка, т. е. характеристики различных типов естественного языка;
	\item диапазон базы знаний интеллектуальных систем, т. е. широта знаний в базах знаний интеллектуальных систем.
\end{itemize}

По статистике лингвистов, в мире существуют тысячи естественных языков. Однако в Интернете существуют лишь десятки естественных языков, которые широко используются конечными пользователями для общения, например, русский язык, английский язык, китайский язык и т. д. Каждый естественный язык имеет свои уникальные особенности. Таким образом, при использовании моделей, методики и средств для обработки текстов различных естественных языков необходимо учитывать соответствующие характеристики различных естественных языков. Кроме того, диапазон базы знаний также влияет на сложность разработки естественно-языковых интерфейсов. Например, базы знаний всемирного масштаба [\scncite{LiuYC2020}] [\scncite{Abu2020}], в которых источники знаний ориентированы на Интернет, т. е. в базах знаний интеллектуальных систем хранятся знания здравого смысла (англ. commonsense knowledge). А базы знаний специализированные [\scncite{Xu2016}], в которых источники знаний ориентированы на различные энциклопедии (например, автомобильная энциклопедия, энциклопедия о фильмах, энциклопедия о разных дисциплинах (дискретная математика, история и другие) и т.д.), т. е. в базах знаний интеллектуальных систем хранятся конкретные отраслевые знания (т. е. знания об ограниченной предметной области). Общие базы знаний обращают внимание на широту хранимых знаний, подчеркивают интеграцию больше количества именованных сущностей по различным предметным областям и отношений между ними. Таким образом, общие базы знаний не обеспечивают высокую точность хранимых в них знаний [\scncite{LiuYC2020}]. А также огромное количество именованных сущностей и отношений между ними в базе знаний приводит к высокой сложности поиска и обработки фрагментов базы знаний. В отличие от общих баз знаний, в специализированных базах знаний уделяется больше внимания глубине и точности хранящихся знаний, удовлетворяющих решение различных задач в интеллектуальных системах с низкой сложностью. 

В соответствии с двумя перечисленными выше аспектами, естественно-языковой интерфейс интеллектуальных систем можно разделить на следующие классы:
\begin{itemize}
	\item \textbf{интерфейс, не зависящий от конкретного естественного языка и конкретной предметной области.} В данном случае это означает, что интерфейс может обрабатывать тексты любых видов естественных языков, например, русский язык, арабский язык, английский язык и китайский язык и так далее, а также обрабатывать любую сложную структуру знаний, т. е. знания в базах знаний всемирного масштаба.
	\item интерфейс, зависящий от конкретного естественного языка, но не зависящий от конкретной предметной области. В этом случае интерфейс может анализировать только тексты определенного естественного языка и обрабатывать любую сложную структуру знаний. 
	\item \textbf{интерфейс, не зависящий от конкретного естественного языка, но зависящий от конкретной предметной области.} В этом случае интерфейс может анализировать тексты любых видов естественных языков, обрабатывать тексты естественного языка, описывающие факты в конкретной предметной области. И специализированная база знаний является основой базы знаний данной интеллектуальной системы. Следует отметить, что описание фактов в разных предметных областях может использовать особенные виды предложений. Таким образом, обработка текстов естественного языка, описывающих факты в разных предметных областях (например, исторический текст, юридический текст, текст в дисциплинах), имеет разную степень сложности.
	\item \textbf{интерфейс, зависящий от конкретного естественного языка и зависящий от конкретной предметной области.} Данный интерфейс считается самым основным естественно-языковым интерфейсом в интеллектуальных системах. В этом случае интерфейсу нужно только анализировать тексты определенного естественного языка и обрабатывать лёгкую структуру знаний в специализированной базе знаний. 	
\end{itemize}

Всем известно, что построение базы знаний всемирного масштаба является огромном проектом. А также база знаний всемирного масштаба в основном используется в Интернет ориентированных поисках, рекомендациях и вопросно-ответных задачах. Сценарий применения относительно единствено. Однако плотность знаний в специализированной базе знаний более высока, \textbf{интерфейс, не зависящий от конкретного естественного языка, но зависящий от конкретной предметной области} могут выполняться больше сценарии применения. 

На данный момент существует большое количество моделей, методов для приобретения фактографических знаний и генерации текстов естественного языка, которые в основном можно разделить на два направления:
\begin{itemize}
	\item методы на основе правил и лингвистических знаний;
	\item методы машинного обучения, основанные на математической статистике и теории информации.
\end{itemize}

Первое направление предполагает разработку систем логических рассуждений на основе правил, соответствующих грамматике определенного естественного языка, сформулированных вычислительными лингвистами и другими лингвистами. В данных системах строятся модели логических рассуждений (правила извлечения, шаблоны для генерации текстов и т.д.) на основе лингвистических знаниях для извлечения фактографических знаний и генерации текстов. 

Под извлечением фактографических знаний понимается приобретение фактической информации, такой как понятия, именованные сущности, отношения между ними, а также определенный тип событий из текстов естественного языка, причем извлеченные результаты формализуются в виде внутреннего языка представления знаний интеллектуальных систем (например, RDF, \scnkeyword{SC-код} и другие). По сути, с точки зрения построения базы знаний интеллектуальных систем, отношения между именованными сущностями рассматриваются как особенные сущности, которые извлекаются из текстов естественного языка. 

Задача извлечения фактографических знаний из текстов естественного языка можно разделить на два направления по сфере извлеченной области:
\begin{itemize}
	\item извлечение фактографических знаний из закрытых областей;
	\item извлечение фактографических знаний из открытых областей.
\end{itemize}

Задача извлечения фактографических знаний из закрытых областей часто требует заранее определенных типов именованных сущностей и отношений между ними. Однако в области извлечения фактографических знаний может быть не существуют ограниченные и определенные типы именованных сущностей и отношений между ними. Таким образом, цель извлечения фактографических знаний из открытой области состоит в том, чтобы извлекать различные наборы именованных сущностей и отношений между ними из массивных и разнородных корпусов текстов естественного языка без требований заранее заданного словаря для определения типа данных именованных сущностей и отношений между ними.

Извлечение фактографических знаний из открытых областей напрямую определяет относительные слова или словосочетания в текстах, анализируя тексты естественного языка (в частности, предложения), чтобы реализовать моделирование классификаций именованных сущностей и отношений между ними без необходимости заранее определений категорий данных именованных сущностей и отношений между ними.  Например, система CORE [\scncite{tseng-2014-chinese}] использует ряд технологий обработки китайского языка, таких как сегментация слов, автоматическая морфологическая разметка (POS tagging), скорректированная и подходящая для текстов китайского языка, синтаксический анализ, семантический анализ и правила извлечения для извлечения именованных сущностей и отношений между ними из текстов китайского языка. В данных системах построение правил вручную неэффективно, а также построенные правила сложно переносятся в других интеллектуальных системах для обработки китайского языка.

Для генерации текстов естественного языка существует классическая конвейерная архитектура [\scncite{Gatt2017}], на основе которой большинство систем генерации текстов имеет три основных модуля:
\begin{itemize}
	\item Планирование содержания текстов: уточнение, какая информация из входных данных будет включена в сгенерированные тексты, и как она будет организована.
	\item Микро-планирование: решение, каким образом выбранная информация будет реализована языковыми средствами в виде предложений на естественном языке. В этом процессе преобразования информации в лингвистические выражения, обычно представлены предложения в виде структур семантических и/или синтаксических отношений.
	\item Реализация на естественном языке: производство грамматически правильных предложений естественного языка.
\end{itemize}

В самом начале системы генерации текстов широко используют шаблоны и грамматические правила, построенные исследователями для генерации текстов естественного языка. Система NaturalOWL [\scncite{Androutsopoulos2013}] представляет собой классическую систему генерации естественного языка, которая рационально применяет шаблоны и лингвистические правила для генерации текстов естественного языка (в основном английский язык), описывающих фрагменты OWL онтологий. Однако, в данной системе не предоставлена модель семантического представления знаний, что позволяет представлять различные виды знаний в единой базе знаний, включая лингвистические знания, логический вывод на онтологиях лингвистики и шаблоны, построенные на онтологиях. 

Второе направление, широко применяемое в настоящее время в современных интеллектуальных системах, предполагает использование различных моделей машинного обучения, основанных на математической статистике и теории информации, которые направлены на моделирование текстов естественного языка.

В настоящее время подходы, основанные на моделей машинного обучения (в частности модели нейронных сетей), в основном используются для извлечения фактографических знаний из закрытых областей, например, серия моделей предварительной подготовки BERT [\scncite{Devlin2018}], GPT [\scncite{Brown2020}] и т.д. Использование моделей нейронных сетей всегда требует очень больших корпусов данных, которые аннотируются вручную человеком. Более того, производительность данных моделей сильно зависит от качества аннотирования обучающих выборок. 

Для обучения моделей нейронных сетей, ориентированных на генерацию текстов естественного языка, обучающие данные обычно представляются в виде пары входов (фрагменты базы знаний) и выходов (тексты естественного языка) [\scncite{moryossef2019}]. В последние годы в проекте WebNLG [\scncite{Gardent2017}] основная задача направлена на разработку нейронных генераторов с помощью моделей нейронных сетей для решения задач генерации текстов. Для разработки нейронных генераторов команда проекта предоставляет обучающие данные, которые состоят из пар Данные/Тексты для английского языка и русского языка, где данные представляют собой фрагменты базы знаний в виде RDF, извлеченных из DBpedia [\scncite{Lehmann2015}]. Для других естественных языков, таких как китайский язык, наборы данных для обучения генераторов отсутствуют, хотя и существуют несколько известных баз знаний на китайском языке в области обработки китайского языка, например, CN-DBpedia [\scncite{Xu2017}], zhishi.me [\scncite{Niu2011}] и другие, которые могут предоставлять фрагменты базы знаний в виде RDF. Построение высококачественных наборов данных для китайского языка является трудоемким и трудозатратным процессом. 

Разработанные системы извлечения фактографических знаний из открытых областей и разработанные различные генераторы были успешно применены в различных областях. Тем не менее, к проблемам существующих решений к разработке естественно-языкового интерфейса и обработке естественного языка можно отнести:
\begin{itemize}
	\item Отсутствие единой основы в процессе разработки естественно-языковых интерфейсов и интеграция различных компонентов, разработанных разными разработчиками, приводит к невозможности распараллеливания разработки компонентов;
	\item Отсутствие возможности использования единой основы для представления различных видов лингвистических знаний в единой базе знаний для разработки естественно-языкового интерфейса;
	\item Отсутствие подходов к возможности интеграции различных моделей решения задач для извлечения фактографических знаний и генерации текстов естественного языка;
	\item Для приобретения фактографических знаний в существующих онтологических подходах построенные правила часто применимы только в извлечениях фактографических знаний из закрытых областей и ограниченных условиях.
	\item Качество извлеченных фактографических знаний и генерируемых текстов зависит от искусственно сконструированных шаблонов и правил, но построение конкретных шаблонов и правил для текстов различных естественных языков является чрезвычайно сложной и трудоёмкой задачей из-за отсутствия единой основы.
	\item Для приобретения фактографических знаний и генерации текстов, Большие языковые модели на основе моделей нейронных сетей
	используется для обработки ЕЯ. Используя модели нейронных сетей, необходимо дорогие аппаратные оборудования и высококачественные согласованные учебные корпусы, которые  крайне сложно получить и построить.
\end{itemize}

При решении комплексной задачи приобретения фактографических знаний и генерации текстов в естественно-языковом интерфейсе построение единой семантической модели осуществляется на основе \scnkeyword{Технологии OSTIS} [\scncite{Golenkov2014a}], что обеспечивает единую основу для эффективного объединения различных лингвистических знаний, включая правила извлечения, правила и шаблоны для генерации текстов и другие лингвистические знания, а также глубокой интеграции различных моделей решения задач (логические модели на основе правил, модели нейронных сетей и другие) для решения задач преобразования текстов естественного языка в фрагменты базы знаний и генерации текстов естественного языка из фрагментов базы знаний (рисунок \textit{\nameref{fig:structure-sc-model-natural-interface}}). 

\begin{figure}[H]
	\includegraphics[scale=0.8,width=1.0\textwidth]{author/part4/figures/structure_interface.png}
	\caption{Общая структура sc-модели естественно-языкового интерфейса}
	\label{fig:structure-sc-model-natural-interface}
\end{figure}

Семантическая модель естественно-языкового интерфейса интеллектуальных систем по конкретнной предметной области позволяет, в принципе, потенциально реализовать преобразование текстов различных естественных языков в фрагменты базы знаний и генерации текстов различных естественных языков из фрагментов базы знаний, просто сложность построения базы знаний по обработке конкретного естественного языка, объединяющей различные виды лингвистических знаний, будет разным в зависимости от особенностей конкретного естественного языка, а также потребуются определенные дополнительные решатели задач по особенностям конкретного естественного языка, интегрирующие различные модели решения задач для обработки текстов конкретного естественного языка.

\section{sc-модель базы знаний лингвистики}
База знаний лингвистики содержит формальное описание необходимых лингвистических знаний для анализа текстов естественного языка и генерации текстов естественного языка [\scncite{Qian2020}]. Соответствующие онтологии обеспечивают формальное описание понятий, используемых для представления таких лингвистических знаний на различных уровнях от базовых слов до синтаксических и семантических структур текстов естественного языка. Лингвистические теории, предложенные лингвистами, обеспечивают теоретическую основу для построения базы знаний лингвистики. 

В рамках \scnkeyword{Технологии OSTIS} sc-модель базы знаний рассматривается, как иерархическая система выделенных предметных областей и соответствующих им онтологий [\scncite{Davydenko2018}]. Основная иерархия базы знаний лингвистики, используемая для решения задач преобразований текстов естественного языка в фрагменты базы знаний и генерации текстов естественного языка из фрагментов базы знаний:
\begin{SCn}
	\scnheader{база знаний лингвистики}
	\scnidtf{база знаний по обработки естественного языка}
	\scnidtf{база знаний естественно-языкового интерфейса}
	\begin{scnrelfromset}{декомпозиция раздела}
		\scnitem {Раздел. Предметная область лексического анализа}
		\scnitem {Раздел. Предметная область синтаксического анализа}
		\scnitem {Раздел. Предметная область семантического анализа}
	\end{scnrelfromset}
\end{SCn}

\subsection{Предметная область лексического анализа}
Предметная область лексического анализа включает в себя ряд онтологий для лексического анализа, которые описывают характеристики слов и синтаксические функции слов, части речи и т.д. В обработке естественного языка \textit{слово} -- это наименьшая единица естественного языка, несущая семантику, которая служит для наименования объектов, их качеств, характеристик и взаимодействий, а также для служебных целей. \textit{Номинативная единица} -- устойчивая последовательность комбинаторных вариантов лексем.
\begin{SCn}
	\scnheader{текст естественного языка}
	\scnsubset{файл}
	\scnheader{слово}
	\scnsubset{файл}
	\scnheader{номинативная единица}
	\scnsubset{файл}
\end{SCn} 

В русском, английском или других европейских языках структура слова изучается в рамках морфологии. Лексема рассматривается для описания особенностей слов в европейских языках, обладающих признаком морфологической парадигмы. В китайском языке из-за письменной привычки (текст китайского языка состоит потока иероглифов без естественных пробелов), единица сегментации рассматривается как наименьшая единица для обработки текстов китайского языка, был предложен в государственном стандарте «Стандарт сегментации слов современного китайского языка, используемый для обработки информации». 
\begin{SCn}
	\scnheader{единица сегментации}
	\scnidtfdef{базовая единица для обработки китайского языка с определенными семантическими или грамматическими функциями}
	\scnsubset{файл}
\end{SCn}

Стоит отметить, что описание единица сегментации ориентировано на компьютерную обработку китайского языка и не полностью совпадает с описанием слов в китайской лингвистике.

\textit{Часть речи} -- это категория слов естественного языка, определяемая морфологическими, синтаксическими и семантическими особенностями. Хотя часть речи, предложеная для анализа текстов европейских языков, не полностью подходит к анализу текстов китайского языка, но часть речи может частично решить задачи в обработке китайского языка. В 2001 г. был предложен государственный стандарт «Принцип частеречной разметки
в обработке современной китайской информации», устанавливающий конкретный стандарт морфологической разметки. 
\begin{SCn}
	\scnheader{часть речи по обработке китайского языка}
	\scnrelto{семейство подмножеств}{единица сегментации}
	\scnhaselement{существительное}
	\scnhaselement{прилагателбное}
	\scnhaselement{глагол}
	\scnhaselement{наречие}
	\scnhaselement{идиома}
	\scnhaselement{союз}
	\scnhaselement{географическое название}
	\scnhaselement{модальный глагол}
\end{SCn}

Категории в части речи по обработке китайского языка построены с учёта особенности китайского языка на лигвистических теорий европейских языков. 

\subsection{Предметная область синтаксического анализа}
Предметная область синтаксического анализа описывает характеристики синтаксиса естественного языка, функциональные характеристики синтаксических компонентов (таких как, слово, словосочетание, предложение и т.д.). Среди них в области обработки естественного языка предложение всегда рассматривается как наименьшая единица исследования. Анализ предложений является важным промежуточным этапом, связывающим анализ всих текстов и анализ отдельных слов. 

\begin{SCn}
	\scnheader{предложение}
	\begin{scnrelfromset}{разбиение}
		\scnitem{простое предложение}
		\scnidtfdef{предложение содержит в себе одну предикативную единицу}
		\scnitem{сложное предложение}
		\scnidtfdef{предложение содержит в себе больше одну предикативную единицу}
	\end{scnrelfromset}
\end{SCn}

Текст китайского языка состоит потока иероглифов без естественных пробелов на основе письменной формы. В зависимости от особенностей китайского языка существуют соответствующие различные синтаксические структуры для предложений китайского языка.

\begin{SCn}
	\scnheader{предложение китайского языка}
	\begin{scnrelfromset}{разбиение}
		\scnitem{предложение с подлежащим и сказуемым}
		\scnitem{предложение без подлежащего и сказуемого}
		\scnitem{предложение с особенным знаком алфавита синтаксиса}
	\end{scnrelfromset}
\end{SCn}

\begin{SCn}
	\scnheader{член предложения'}
	\begin{scnrelfromset}{разбиение}
		\scnitem{главный член предложения'}
		\begin{scnindent}
			\begin{scnrelfromset}{разбиение}
				\scnitem{подлежащее'}
				\scnitem{сказуемое'}
				\scnitem{прямое дополнение'}
			\end{scnrelfromset}
		\end{scnindent}
		\scnitem{второстепенный член предложения'} 
		\begin{scnindent}
			\begin{scnrelfromset}{разбиение}
				\scnitem{косвенное дополнение'}
				\scnitem{определение'}
				\scnitem{обстоятельство'}
			\end{scnrelfromset}
		\end{scnindent}
	\end{scnrelfromset}
\end{SCn}

\textit{Член предложения'} -- рольное отношение, связывающее декомпозицию текста с файлом, содержимое которого (отдельное слово или словосочетание из текста) играет в декомпозируемом тексте определенную синтаксическую роль [\scncite{Hardzei2022}].

\begin{SCn}
	\scnheader{отношение зависимости*}
	\scnidtfdef{описание зависимости отдельных слов или словосочетаний в предложении друг от друга.}
	\scnhaselement{суъбект*}
	\scnhaselement{объект*}
	\scnhaselement{дополнение глагола*}
	\scnhaselement{определитель*}
	\scnhaselement{атрибут*}
	\scnhaselement{модификатор*}
\end{SCn}

На основе теории синтаксиса грамматики зависимостей, \textit{отношение зависимости*} -- отношение, связывающее декомпозированные два файла направленными связями, содержимое которых отдельно называется исходным словом (англ. head), которое часто играет роль сказуемое' в предложении, и целевым словом (англ. depedent). Каждое целевое слово несет синтаксическую функцию по зависимости к исходному слову. В общем случае, при синтаксическом анализе предложений глагол (также называется конечный глагол, исходное слово) часто считается структурным центром придаточной структуры. Все остальные синтаксические единицы (слова или словосочетания) прямо или косвенно связаны с глаголом посредством направленных связяей (т. е. направление с глаголом до других синтаксических единиц). В области обработки китайского языка была предложена Харбинским политехническим университетом схема отношений зависимости [\scncite{liu2006}].

В логической онтологии предметной области синтаксического анализа можно описать ряды логических определений и логических утверждений для предложений. В виде логических утверждений могут быть построены шаблоны или правила для приобретения фактографических знаний и генерации текстов, которые могут использоваться решателями задач для решения конкретных задач в обработках текстов естественного языка.
\begin{figure}[H]
	\centering
	\includegraphics[scale=0.8]{author/part4/figures/ruleGeneration.jpg}
	\caption{Логическое утверждение про генерации предложения на основе шаблона}
	\label{fig:template-generation}
\end{figure}

На рисунке \textit{\nameref{fig:template-generation}} указано на языке SCg, что эвристическое правило, используемое для генерации текстов естественного языка. Как показано на рисунке, если установлено, что акция в ``message triple'' принадлежит к доменно-независимое отношение, то шаблон используется для генерации текстов. Компоненты субъект и объект в ``message triple'' заполняются в параметр 1 и параметр 3 в шаблоне, соответственно, а затем соответствующий глагол для акции в ``message triple'' выбирается в качестве параметра 2 в шаблоне. 

\subsection{Предметная область семантического анализа}
Предметная область семантического анализа описывает семантические характеристики слов и семантическую структуру предложений естественного языка, функциональные характеристики семантики компонентов (слов, словосочетаний и предложений и т.д.), семантические роли, правила семантического анализа и так далее.

Предметная область семантического анализа на уровне лексемы описывает семантические классификации общих базовых понятий, выраженных лексемой, или привязку лексемы к отдельным сущностям. Предметная область была построена на основе различных типов баз семантических знаний о естественном языке, используемых для семантического анализа на уровне лексемы, например, WordNet, ConcetpNet, The Semantic Knowledge base of Modern Chinese [\scncite{Wang2006}], TAPAZ-2 и т. д.

\begin{SCn}
	\scnheader{семантические категории в естественном языке}
	\scnhaselement{семантические категории для глаголов}
	\scnhaselement{семантические категории для существительных}
	\scnhaselement{семантические категории для прилагателбных}
	\scnhaselement{семантические категории для наречий}
\end{SCn}


\begin{SCn}
	\scnheader{участник воздействия*}
	\scnidtf{участник акции*}
	\scnidtf{участник события*}
	\scniselement{неролевое отношение}
	\scnrelfrom{первый домен}{индивид}
	\scnrelfrom{вторый домен}{акция}
	\begin{scnrelfromset}{разбиение}
		\scnitem {субъект*}
		\begin{scnindent}
			\begin{scnrelfromset}{разбиение}
				\scnitem {инициатор*}
				\scnitem {вдохновитель*}
				\scnitem {распространитель*}
				\scnitem {вершитель*}
			\end{scnrelfromset}
		\end{scnindent}
		\scnitem {объект*}
		\begin{scnindent}
			\begin{scnrelfromset}{разбиение}
				\scnitem {покрытие*}
				\scnitem {корпус*}
				\scnitem {прослойка*}
				\scnitem {сердцевина*}
			\end{scnrelfromset}
		\end{scnindent}
	\end{scnrelfromset}
\end{SCn}

\textit{Индивид} является разновидностью стереотипа как отдельной сущности, которая представляет собой  экземпляр конкретных понятий. 

\textit{Участник действия*} -- это неролевое отношение, которое связывает действие с индивидом, участвующим в нем, в определенной степени его можно рассматривать как семантическую роль действия. Соответственно действие обычно выражается глаголом или глагольной фразой в предложении.

В свою очередь, аналогично предметная область семантического анализа на уровне предложения описывает семантическую структуру предложений, семантические отношения между компонентами (словами, словосочетаниями и т.д.) в предложении, а также семантические отношения между предложениями в текстах. Некоторые открытые базы знаний, например, TAPAZ-2, PropBank и другие, послужили основой для построения данной предметной области.

\subsection{sc-модель действий для естествено-языкового интерфейса}
В sc-модели естественно-языкового интерфейса база знаний лингвистики просто предоставляет статические лингвистические знания различные модели решения задач для решения задач в естественно-языковом интерфейсе. Кроме того, естественно-языковой интерфейс должен иметь динамическую возможность выполнять некоторые действия, направленные на решение конкретных связанных задач в естественно-языковом интерфейсе. 
\begin{SCn}
	\scnheader{действие для естественно-языкового интерфейса}
	\begin{scnrelfromset}{декомпозиция}
		\scnitem{действие для преобразования текстов естественного языка в фрагменты базы знаний}
		\scnitem{действие для генерации текстов естественного языка из фрагментов базы знаний}
	\end{scnrelfromset}
\end{SCn}

\begin{SCn}
	\scnheader{действие для преобразования текстов естественного языка в фрагменты базы знаний}
	\begin{scnrelfromset}{декомпозиция}
		\scnitem{действие для разбиений текстов на отдельные единицы}
		\scnitem{действие для разметки отдельных единиц}
		\scnitem{действие для синтаксического анализа}
		\scnitem{действие для семантического анализа}
		\scnitem{действие для определений структур знаний}
		\scnitem{ействие для связываний структур знаний в базе знаний}
		\scnitem{действие для определений противоречий}
		\scnitem{действие для устранений противоречий}
	\end{scnrelfromset}
\end{SCn}

\begin{SCn}
	\scnheader{действие для генерации текстов естественного языка из фрагментов базы знаний}
	\begin{scnrelfromset}{декомпозиция}
		\scnitem{действие для определений фрагментов базы знаний}
		\scnitem{действие для преобразований фрагментов в стандартные базовые sc-конструкции}
		\scnitem{действие для определений проективных базовых sc-конструкций}
		\scnitem{действие для преобразований проективных sc-конструкций в message triple (субъект-акция-объект)}
		\scnitem{действие для генерации результирующих текстов из message triple}
	\end{scnrelfromset}
\end{SCn}

С точки зрения генерации текста из фрагментов базы знаний ostis-системы, следует отметить, что ``message triple'' представляет собой вид упорядоченной тройки, представленной в виде <семантический субъект, акция, семантический объект>, где семантический субъект -- это всегда идентификатор множества sc-узлов, обозначающих понятия, не являющиеся отношениями, или элемент данного множества sc-узлов; акция -- идентификатор множества sc-узлов, обозначающих ролевые или неролевые отношения, которые указывает на связку, соединяющую семантический субъект и объект; семантический объект -- идентификатор множества sc-узлов, обозначающих понятия, не являющиеся отношениями, или элемент данного множества sc-узлов.

В процессе генерации, ``message triple'' рассматривается как промежуточное преобразование между sc-структурой и результирующим текстом, главным образом потому что данный вид тройки легче выразить в виде текстов естественного языка (в основном предложения), используя либо модели шаблонов или правил, либо современные модели нейронной сети. Кроме того, ``message triple'' легко преобразуется в текстовом виде, который, в свою очередь, может быть сгенерированы в виде текстов естественного языка с помощью современных зрелых моделей нейронной сети.

\section{sc-модель решателя задач естествено-языкового интерфейса}
Для реализации естественно-языкового интерфейса ostis-систем необходимо разработать модель решателя задач естественно-языкового интерфейса, в рамках которой решатель задач рассматривается как иерархическая система агентов, способного выполнять вышеупомянутые действия в sc-памяти для решения конкретных задач в естественно-языковом интерфейсе [\scncite{Shunkevich2017}].
\begin{SCn}
	\scnheader{решатель задач естественно-языкового интерфейса}
	\scnidtf{решатель задач по обработке естественного языка}
	\begin{scnrelfromset}{декомпозиция}
		\scnitem{решатель задач анализа текстов естественного языка}
		\scnitem{решатель задач генерации текстов естественного языка}
	\end{scnrelfromset}
\end{SCn}

В естественно-языковом интерфейсе решатель задач анализа текстов естественного языка предназначен для решения задач извлечения фактографических знаний (как правило, именованные сущности, понятия, отношения между ними и т.д.) из текстов естественного языка в открытых областях.

\begin{SCn}
	\scnheader{решатель задач анализа текстов естественного языка}
	\scnidtf{решатель задач приобретения фактографических знаний}
	\begin{scnrelfromset}{декомпозиция абстрактного sc-агента}
		\scnitem{абстрактный sc-агент лексического анализа}
		\begin{scnindent}
			\begin{scnrelfromset}{декомпозиция абстрактного sc-агента}
				\scnitem {абстрактный sc-агент разбиения текстов на отдельные единицы}
				\scnitem {абстрактный sc-агент разметки отдельных единиц}
			\end{scnrelfromset}
		\end{scnindent}
		\scnitem{абстрактный sc-агент синтаксического анализа}
		\scnitem{абстрактный sc-агент семантического анализа}
		\scnitem{абстрактный sc-агент извлечения структур знаний в базу знаний}
		\begin{scnindent}
			\begin{scnrelfromset}{декомпозиция абстрактного sc-агента}
				\scnitem {абстрактный sc-агент определения структур знаний}
				\scnitem {абстрактный sc-агент связывания структур знаний в базу знаний}
				\scnitem {абстрактный sc-агент определения противоречий}
				\scnitem {абстрактный sc-агент устранения противоречий}
			\end{scnrelfromset}
		\end{scnindent}
		\scnitem{абстрактный sc-агент логического вывода}
	\end{scnrelfromset}
\end{SCn}

\textit{\textbf{Абстрактный sc-агент разбиения текстов на отдельные единицы}} -- группы агентов, реализующие механизмы декомпозиции входных текстов естественного языка на лексические единицы. Компоненты (слова, словосочетания, предложения и другие) входных текстов естественного языка могут быть определены по определенному стандарту конкретного естественного языка.

\textit{\textbf{Абстрактный sc-агент разметки отдельных единиц}} -- агенты, реализующие механизмы разметки раздельных лексических единиц по определенному принципу конкретного естественного языка, устанавливающему конкретный стандарт разметки. 

\textit{\textbf{Абстрактный sc-агент синтаксического анализа}} -- агенты, реализующие механизмы построения синтаксической структуры входных текстов естественного языка.

\textit{\textbf{Абстрактный sc-агент семантического анализа}} -- агенты, реализующие механизмы построения семантической структуры входных текстов естественного языка.

\textit{\textbf{Абстрактный sc-агент извлечения структур знаний в базе знаний}} -- агенты, реализующие механизмы интеграции семантического эквивалента sc-конструкции в базу знаний конкретной ostis-системы.

\textit{\textbf{Абстрактный sc-агент определения структур знаний}} -- агенты, реализующие механизмы, которые определяют структуры знаний во входных текстах естественного языка (т.е. определение именованных сущностей и отношений между ними).

\textit{\textbf{Абстрактный sc-агент связывания структур знаний в базе знаний}} -- агенты, реализующие механизмы связывания структур знаний в базе знаний конкретной ostis-системы, включая наполнение экземпляров понятий с использованием именованных сущностей, сопоставление именованных сущностей и отношений между ними в базе знаний конкретной ostis-системы.

В процессе отображения структур знаний в базу знаний, при сравнении структур знаний, извлеченных в результате анализа входных текстов естественного языка, с знаниями, хранящимися в базе знаний конкретной ostis-системы, \textit{\textbf{абстрактный sc-агент определения противоречий}} -- агенты, реализующие механизмы определения противоречий, например, для определенной именованной сущности во входных текстах естественного языка может быть существует несколько соответствующих экземпляров понятий. 

\textit{\textbf{Абстрактный sc-агент устранения противоречий}} -- агенты, реализующие механизмы устранения противоречий, в случае обнаружения противоречий. 

\textit{\textbf{Абстрактный sc-агент логического вывода}} -- агенты, реализующие механизмы, которые используют логические правила, записанные средствами SC-кода, для синтаксического анализа, семантического анализа и также извлечения структур знаний. Таким образом, данный агент может взаимодействовать с агентами синтаксического, семантического анализа и агентом извлечения структур знаний.

При решении задач генерации текстов sc-модель решатель задач генерации текстов естественного языка разработана на основе классического конвейера. Хотя разработка модели данного решателя основана на классическом конвейере, разработка конкретных составов решателя может быть гибка за счет использования многоагентного подхода [\scncite{Qian2022}]. Для генерации текстов конкретного естественного языка, составы решателя задач можно соответствующим образом легко модифицировать.
\begin{SCn}
	\scnheader{решатель задач генерации текстов естественного языка}
	\begin{scnrelfromset}{декомпозиция абстрактного sc-агента}
		\scnitem{абстрактный sc-агент определения фрагмента базы знаний}
		\scnitem{абстрактный sc-агент планирования структуры текстов}
		\scnitem{абстрактный sc-агент микро-планирования}
		\scnitem{абстрактный sc-агент реализации текстов естественного языка}
	\end{scnrelfromset}
\end{SCn}

\begin{SCn}
	\scnheader{абстрактный sc-агент определения фрагмента базы знаний}
	\begin{scnrelfromset}{декомпозиция абстрактного sc-агента}
		\scnitem{абстрактный sc-агент определения sc-структуры}
		\scnitem{абстрактный sc-агент разделения определенной sc-структуры на базовые стандартные sc-конструкции}
		\scnitem{абстрактный sc-агент определения конкретных проектных sc-конструкций}
		\scnitem{абстрактный sc-агент перевода конкретных sc-конструкций в соответствующие им ``message triple''}
	\end{scnrelfromset}
\end{SCn}

\textit{\textbf{Абстрактный sc-агент определения sc-структуры}} -- группа агентов, предоставляющих поиск из конкретной предметной области базы знаний sc-структуры, составляющего из sc-конструкций стандартного вида, из которых будем преобразовывать тексты естественного языка. С точки зрения генерации текстов, на этом этапе, по сути, определяется, о чем мы хотим поговорить. 

\textit{\textbf{Абстрактный sc-агент разделения определенной sc-структуры на базовые стандартные sc-конструкции}} -- группа агентов, реализующих механизмы декомпозиции поисковой sc-структуры на базовые стандартные sc-конструкции, которые могут быть переведены в ``message triples''. 

В некоторых случаях разделенные базовые sc-конструкции из sc-структуры являются избыточными, т.е. не все разделенные базовые sc-конструкции нужно преобразовывать в ``message triple'', а затем в тексты естественного языка, а некоторые не являются той информацией, которую ожидают пользователи. Таким образом, естественно-языковой интерфейс требует окончательного выбора некоторых среди разделенных sc-конструкций в качестве проектных sc-конструкций, предоставляющих максимально полезную информацию пользователям, и даже подходящую для пользователей (для разных пользователей нужно предоставлять сгенерированные тексты различных стилей, например, для специалистов нужно использовать специальные термины в сгенерированных текстах). 

\textit{\textbf{Абстрактный sc-агент определения конкретных sc-конструкций}} -- группа агентов, реализующих механизмы определения подходящих конкретных sc-конструкций, предоставляющих информацию пользователям и возможно удовлетворяющих конкретных пользователей. 

\textit{\textbf{Абстрактный sc-агент преобразования конкретных sc-конструкций в соответствующие ними ``message triple''}} -- группа агентов, реализующих механизмы перевода выбранных конкретных sc-конструкций в соответствующие ``message triple''.

\textit{\textbf{Абстрактный sc-агент планирования структуры текстов}} -- группа агентов, реализующих функцию структурирования сгенерированных текстов, т.е. определения упорядоченности ``message triple'', в данной упорядоченности ``message triple'' будут представлены в сгенерированных текстах. 
\begin{SCn}
	\scnheader{абстрактный sc-агент планирования структуры текстов}
	\begin{scnrelfromset}{декомпозиция абстрактного sc-агента}
		\scnitem{абстрактный sc-агент упорядочения ``message triple''}
		\scnitem{абстрактный sc-агент упорядочения именованных сущностей в каждом ``message triple''}
	\end{scnrelfromset}
\end{SCn}

В общем случае, в базе знаний ostis-систем в качестве идентификаторов sc-элементов чаще всего используются имена (термины) соответствующих сущностей, представленные отдельными словами или словосочетаниями на различных естественных языках, но также могут использоваться иероглифы на китайском языке. Таким образом, в качестве идентификаторов (названий, терминов) именованных сущностей в ``message triple'' можно прямо использовать в сгенерированных текстах естественного языка. 

\textit{\textbf{Абстрактный sc-агент микро-планирования}} -- группа агентов, реализующих механизмы превращения ``message triple'' в абстрактные спецификации предложений естественного языка, которые варьируются в зависимости от систем генерации естественного языка, например, текстовые шаблоны с частично незаполненными слотами, которые будут заполнены для генерации результирующих предложений естественного языка.
\begin{SCn}
	\scnheader{абстрактный sc-агент микро-планирования}
	\begin{scnrelfromset}{декомпозиция абстрактного sc-агента}
		\scnitem{абстрактный sc-агент составления планирования предложений естественного языка}
		\scnitem{абстрактный sc-агент агрегирования предложений}
		\scnitem{абстрактный sc-агент генерации ссылающегося выражения}
	\end{scnrelfromset}
\end{SCn}

\textit{\textbf{Абстрактный sc-агент составления планирования предложений естественного языка}} -- группа агентов, реализующие механизмы построения абстрактной спецификации для каждой ``message triple'', которая используется для генерации предложения естественного языка.

Для доменно-независимых отношений можно построить конкретные шаблоны или правил как абстрактные спецификации предложений в соответствующих логических онтологиях для генерации текстов естественного языка. Однако для доменно-зависимых отношений из-за широкого диапазона баз знаний использование моделей нейронной сети позволяет сгенерировать болше беглые разнообразные тексты естественного языка. 

\textit{\textbf{Абстрактный sc-агент агрегирования предложений}} -- группа агентов, реализующих механизмы решения, какие ``message triple'' могут представлять в отдельных предложениях. В некоторых случаях несколько ``message triple'' могут быть объединены в отдельное предложение.

\textit{\textbf{Абстрактный sc-агент генерации ссылающегося выражения}} -- группа агентов, реализующих механизмы генерации ссылающихся выражений (слова или словосочетания, например, местоимение-существительное), которые соответствуют конкретным именованным сущностям ``message triple''.

\textit{\textbf{Абстрактный sc-агент реализации текстов естественного языка}} -- группа агентов, реализующих механизмы конкатенации sc-ссылок (т.е. sc-узел с содержанием), включающих фрагменты текстов естественного языка (слова, словосочетания и т.д.), для генерации результирующих текстов естественного языка. 
\begin{SCn}
	\scnheader{абстрактный sc-агент реализации текстов естественного языка}
	\begin{scnrelfromset}{декомпозиция абстрактного sc-агента}
		\scnitem{абстрактный sc-агент лексикализаций}
		\scnitem{абстрактный sc-агент объединения всех sc-ссылок}
	\end{scnrelfromset}
\end{SCn}

\textit{\textbf{Абстрактный sc-агент лексикализаций}} -- группа агентов, реализующих механизмы поиска правильных морфологических парадигм лексических единиц в виде sc-ссылок, соответствующих идентификаторам именованных сущностей в ``message triple'', для генерации грамматически правильных текстов естественного языка.

\textit{\textbf{Абстрактный sc-агент объединения всех sc-ссылок}} -- группа агентов, реализующих механизмы объединения всех sc-ссылок или заполнения всех sc-ссылок в подходящие слоты соответствующих построенных шаблонов или правил для генерации результирующих текстов. 

Следует отметить, что проектированные sc-агенты собственно естественно-языковой интерфейс является универсальными без учета характеристик конкретного естественного языка. Другими словами, в данной главе основное внимание уделяется построению унифицированной агентно-ориентированной модели решателя задач естественно-языкового интерфейса, на основе которой можно разработать специфическую иерархическую систему агентов для решения задач обработки текстов конкретного естественного языка по характеристикам разных уровней конкретного естественного языка в конкретном естественно-языковом интерфейсе.
\section{Методика и средство для разработки естественно-языкового интерфейса}
Методика разработки естественно-языкового интерфейса включает несколько этапов, в которых нужно использовать методику построения и модификации гибридных баз знаний и гибридных решателей задач [\scncite{Davydenko2018}]. На рисунке \textit{\nameref{fig:method-interface}} представлен перечень таких этапов с указанием последовательности их выполнения.
\begin{figure}[H]
	\centering
	\includegraphics[scale=0.8,width=1.0\textwidth]{author/part4/figures/method.png}
	\caption{Этапы процесса разработки естественно-языкового интерфейса}
	\label{fig:method-interface}
\end{figure}

Данная методика может быть применена при разработке конкретного естественно-языкового интерфейса по конкретной предметной области. Далее рассмотрим процесс разработки естественно-языкового интерфейса ostis-систем на этапах.

\textbf{Этап 1. Формирование требований и особенностей конкретного естественного языка.}

На данном этапе необходимо четко рассматривать особенности конкретного естественного языка. Затем можно разработать базу знаний по обработке конкретного естественного языка и соответствующие решатели задач для обработки конкретного естественного языка. После определения конкретного естественного языка существует вероятность того, что в составе библиотеки компонентов уже есть реализованный вариант требуемой базы знаний и соответствующих решателей. В противном случае, тем не менее, у разработчика появляется возможность включить разработанную базу знаний по обработке конкретного естественного языка и соответствующие решатели задач в библиотеку компонентов для последующего использования.

\textbf{Этап 2. Разработка базы знаний по обработке конкретного естественного языка.}

Общие принципы на основе методики согласованного построения и модификации гибридных баз знаний [\scncite{Davydenko2017}] используются для разработки базы знаний по обработке конкретного естественного языка.

\textbf{Этап 3. Разработка решателя задач естественно-языкового интерфейса.}

Общие принципы на основе методики согласованного построения и модификации гибридных решателей задач [\scncite{Shunkevich2018}] используются для разработки решателей задач естественно-языкового интерфейса для решения задач приобретения фактографических знаний и генерации текстов конкретного естественного языка.

\textbf{Этап 4. Верификация разработанных каждых компонентов.}

На данном этапе верификация разработанных компонентов (база знаний по обработке конкретного естественного языка и соответствующие решатели задач для решения задач приобретения фактографических знаний и генерации текстов конкретного естественного языка) конкретного естественно-языкового интерфейса может осуществляться вручную

\textbf{Этап 5. Отладка разработанных каждых компонентов. Исправление ошибок.}

Как правило, этапы 4, 5 могут выполняться циклически до тех пор, пока разработанные компоненты не будут соответствовать предъявляемым требованиям.

Библиотека многократно используемых компонентов является важнейшим понятием в рамках технологии OSTIS. С помощью Библиотеки многократно используемых компонентов естественно-языкового интерфейса могут выбрать компонент по требованию или набору компонентов в одной из библиотек и включить их в разрабатываемой конкретной естественно-языковой интерфейс других ostis-систем, т.е. разработанный компонент естественно-языкового интерфейса может быть повторно использован при разработке естественно-языкового интерфейса в других ostis-системах. Аналогично, для разработки конкретного естественно-языкового интерфейса в других ostis-системах Библиотеки многократно используемых компонентов конкретного естественно-языкового интерфейса тоже могут использованы повторно для сокращения сроки и трудоемкости разработки конкретного естественно-языкового интерфейса. Была разработана Библиотеки многократно используемых компонентов естественно-языкового интерфейса и Библиотеки многократно используемых компонентов китайско-языкового интерфейса в качестве средства для разработки естественно-языкового интерфейса.
\begin{SCn}
	\scnheader{Библиотека многократно используемых компонентов естественно-языкового интерфейса}
	\begin{scnrelfromset}{разбиение}
		\scnitem{Библиотека многократно используемых компонентов базы знаний естественно-языкового интерфейса}
		\begin{scnindent}
			\scnidtf{Библиотека многократно используемых компонентов базы знаний лигвистики}
			\scnidtf{Библиотека многократно используемых компонентов базы знаний по обработке естественного языка}
		\end{scnindent}
		\scnitem{Библиотека многократно используемых компонентов решателей задач естественно-языкового интерфейса}
		\begin{scnindent}
			\scnidtf{Библиотека многократно используемых компонентов решателей задач для обработки естественного языка}
		\end{scnindent}
	\end{scnrelfromset}
\end{SCn}

\begin{SCn}
	\scnheader{Библиотека многократно используемых компонентов китайско-языкового интерфейса}
	\begin{scnrelfromset}{разбиение}
		\scnitem{Библиотека многократно используемых компонентов базы знаний китайско-языкового интерфейса}
		\begin{scnindent}
			\scnidtf{Библиотека многократно используемых компонентов базы знаний по обработке китайского языка}
		\end{scnindent}
		\scnitem{Библиотека многократно используемых компонентов решателей задач китайско-языкового интерфейса}
		\begin{scnindent}
			\scnidtf{Библиотека многократно используемых компонентов решателей задач для обработки китайского языка}
		\end{scnindent}
	\end{scnrelfromset}
\end{SCn}

На данный момент на основе модели базы знаний лингвистики были разработаны онтологии предметных областей в качестве компонентов, описывающих различные виды лингвистических знаний, которые могут использоваться для реализации конкретного естественно-языкового интерфейса:
— Предметная область слов;

— Предметная область словосочетаний;

— Предметная область предложений;

— Предметная область параграфов;

— и другие.

Различные виды лингвистических знаний для обработки китайского языка были специфицированы с использованием разработанных онтологий предметных областей в базе знаний по обработке китайского языка, которые могут быть разработаны в качестве компонентов для разработки китайско-языкового интерфейса в большинстве других ostis-систем.
— Предметная область единиц сегментации;

— Предметная область словосочетаний китайского языка;

— Предметная область предложений китайского языка;

— Предметная область параграфов китайского языка;

— и другие.

На данный момент на основе модели решателей задач естественно-языкового интерфейса основное внимание уделено многократно используемым sc-агентам, входящим в состав решателей задач естественно-языкового интерфейса:
\begin{SCn}
	\scnheader{Библиотека многократно используемых атомарных абстрактных sc-агентов естественно-языкового интерфейса}
	\begin{scnrelfromset}{разбиение}
		\scnitem{Библиотека sc-агентов определения структур знаний}
		\scnitem{Библиотека sc-агентов определения конкретных sc-конструкций}
		\scnitem{Библиотека sc-агентов перевода sc-конструкций в соответствующие им message triple}
	\end{scnrelfromset}
\end{SCn}

Аналогично, при разработке китайско-языкового интерфейса новой ostis-системы, многократно используемый абстрактный sc-агент нужно быть скопирован в этой новой ostis-системе, после того необходимо сгенерировать sc-узел, обозначающий конкретный sc-агент, работающий в этому китайско-языковом интерфейсе данной системы. На данный момент многократно используемым sc-агентам, входящим в состав решателей задач китайско-языкового интерфейса:
\begin{SCn}
	\scnheader{Библиотека многократно используемых атомарных абстрактных sc-агентов китайско-языкового интерфейса}
	\scnidtf{Библиотека многократно используемых атомарных абстрактных sc-агентов для обработки китайского языка}
	\begin{scnrelfromset}{разбиение}
		\scnitem{Библиотека sc-агентов разбиения текстов на единицы сегментации}
		\scnitem{Библиотека sc-агентов разметки отдельных единиц сегментации}
		\scnitem{Библиотека sc-агентов перевода конкретных sc-конструкций в соответствующие им message triples на китайском языке}
		\scnitem{Библиотека sc-агентов генерации текстов китайского языка из message triples}
	\end{scnrelfromset}
\end{SCn}

\section {Реализация китайско-языкового интерфейса}
На основе sc-модели естественно-языкового интерфейса ostis-систем могут быть реализованы конкретные естественно-языковые интерфейсы интеллектуальных справочных систем по различным предметным областям. Однако из-за трудоемкости и сложности разработки интеллектуальных справочных систем по конкретной предметной области, а также автор как носитель китайского языка, прототип интерфейса на китайском языке был реализован в интеллектуальной справочной системе по дискретной математике.

\subsection{Извлечение фактографических знаний из текстов китайского языка}
Преобразование текстов естественного языка в фрагменты базы знаний рассматривается как задачи извлечения фактографических знаний. В данном случае преобразование текстов китайского языка в фрагменты базы знаний интеллектуальной справочной системы по дискретной математике заключается в извлечении именованных сущностей и отношений между ними из текстов китайского языка по дискретной математике, а наконец извлеченные результаты сохраняются в виде SC-кода в базе знаний интеллектуальной справочной системе по дискретной математике.

Несколько требований былы определены для задач преобразования текстов китайского языка в фрагменты базы знаний:
\begin{itemize}
	\item Вход представляет собой стандартное повествовательное предложение китайского языка, а не речевое.
	\item Входное повествовательное предложение китайского языка имеет завершенный смысл и фактическое знание.
	\item Не нужен заранее заданный словарь, включающий заранее определенные типы именованных сущностей и отношений между ними.
	\item Выход является sc-структура, формально представленная в базе знаний.
\end{itemize}

Любой текст естественного языка представлен в виде sc-ссылок (т.е. sc-узла с содержимым). В данном подразделе показан пример процесса анализа предложения китайского языка, которое представлено на таком узле на рисунке \textit{\nameref{fig:chinese-sentence-sc}}. Содержимое такого узла демонстрирует предложение китайского языка, которое относится к области дискретной математики и описывает: \begin{CJK}{UTF8}{gbsn}«从结构形式化的角度 (с точки зрения формализации структуры) ,结构(структуру) 可以 (можно) 划分 (разделить) 为 (на) 形式化的结构 (формальную структуру) 和 (и) 非形式化的结构 (неформальную структуру) 。». \end{CJK}
\begin{figure}[H]
	\centering
	\includegraphics[scale=0.8]{author/part4/figures/chinese_sentence.png}
	\caption{Представление предложения китайского языка в ostis-системе}
	\label{fig:chinese-sentence-sc}
\end{figure}

\textbf{\textit{Шаг 1:}} Предложение китайского языка декомпозируется на отдельные единицы сегментаций, а также помечаются данные единицы сегментаций стандартными категориями частей речи в китайском языке (рисунок \textit{\nameref{fig:segment-chinese}}).
\begin{figure}[H]
	\centering
	\includegraphics[scale=0.7]{author/part4/figures/segment_chinese_sentence.png}
	\caption{Результат разделения и разметки отдельных единиц сегментации}
	\label{fig:segment-chinese}
\end{figure}

Согласно особенности текстов китайского языка, текст китайского языка состоит из потока иероглифов без естественных пробелов на основе письменной формы. При этом в китайском языке отсутствуют четкие показатели категорий числа, падежа и рода, такие как в русском языке и других европейских языках, функция слова в китайском языке становится понятной не на основании морфологических изменений слова, а благодаря его связи с другими словами. В связи с этим в процессе анализа текстов китайского языка сначала необходимо выполнить лексический анализ, разбивающий поток иероглифов в тексте китайского языка на отдельные значимые единицы сегментации китайского языка. 

\textbf{\textit{Шаг 2:}} Синтаксический и семантический анализ выполняет построение синтаксической структуры или семантической структуры предложения китайского языка (рисунок \textit{\nameref{fig:syntac-structure}}).

При обработке китайского языка задачи синтаксического анализа являются анализами взаимосвязей (или отношений зависимости) между входными предложениями китайского языка и раздельными единицами сегментаций и между раздельными единицами сегментации в предложениях китайского языка для раскрытия их синтаксических структур. На основе построенной синтаксической структуры можно построить коллектив логических рассуждений для некоторых анализов приложений китайского языка в области обработки китайского языка. Таким образом, синтаксический анализ китайского языка позволяет создать основу (т.е. ряд логических правил) для извлечения фактографических знаний из предложений китайского языка.
\begin{figure}[H]
	\centering
	\includegraphics[scale=0.7]{author/part4/figures/syntac_structure.png}
	\caption{Результат синтаксического анализа предложения китайского языка}
	\label{fig:syntac-structure}
\end{figure}

При обработке китайского языка отношения между единицами сегментации и китайским предложением, или отношения между единицами сегментации китайского предложения, обычно рассматриваются как определенные типы лингвистических знаний, включенных в базе знаний по обработке китайского языка.

\textbf{\textit{Шаг 3:}} Структура фактографических знаний извлекается из соответствующей синтаксико-семантической структуры входного предложения китайского языка, и слияние структуры фактографических знаний в базу знаний интеллектуальной справочной системы (т.е. построение фрагмента базы знаний) осуществляется на основе логических правил извлечения.


\begin{figure}[H]
	\centering
	\includegraphics[scale=0.7]{author/part4/figures/generated_sc_structure.png}
	\caption{Результирующий построенный фрагмент в базе знаний}
	\label{fig:generated-structure}
\end{figure}

С точки зрения извлечения фактографических знаний из открытой области, основными задачами является извлечение именованных сущностей и отношений между ними из предложений китайского языка, не требуя предопределенных типов именованных сущностей и отношений. Фактически, в рамках \scnkeyword{Технологии OSTIS} отношение рассматривается как специальная сущность, которая определяет определенное отношение между парами независимых именованных сущностей, которое обычно представлена соответственно как sc-узел, обозначающий неролевое отношение, или sc-узел, обозначающий ролевое отношение в форме SCg-кода.

В целом, пары независимых именных сущностей должны появляться в анализируемой синтаксико-семантической структуре в виде именных фраз, относящихся к номинативным единицам. В дальнейшем связи между этими именными фразами и входным китайским предложением, а также путь, связывающий две именные фразы через другие сегментационные единицы, будут отражать соответствующие отношения между парами именных сущностей. Более точно, именные фразы, появляющиеся в анализируемой синтактико-семантической структуре входного китайского предложения, представляются как идентификаторы sc-узлов, обозначающих название некоторых именованных сущностей или понятий, хранящихся в базе знаний интеллектуальной справочной системы по дискретной математике.

На рисунке \textit{\nameref{fig:generated-structure}} показан построенный фрагмент базы знаний из входного предложения китайского языка в интеллектуальной справочной системе по дискретной математике без обнаружения противоречий.

Как видно из рисунка \textit{\nameref{fig:generated-structure}}, в данном случае фрагмент базы знаний может быть построен напрямую без связывания именованных сущностей, упомянутых во входном исходном китайском предложении, с соответствующими существующими определенными сущностями в базе знаний интеллектуальной справочной системы по дискретной математике. В некоторых случаях для именованной сущности в базе знаний существуют различные названия в текстах на естественном языке для описания этой сущности. В этом случае необходимо выполнить устранение противоречий, чтобы связать разные именованные сущности (именно идентификаторы именованных сущностей) в текстах на естественном языке с одинаковыми именованными сущностями в базе знаний ostis-системы. 
\subsection{Генерация текстов из фрагментов базы знаний}
Генерация текстов китайского языка осуществляется в соответствии с общей архитектурой генерации текстов. Для реализации генерации китайского языка из фрагментов базы знаний ostis-системы условно делится на два этапа: символьный генератор, преобразующий фрагменты (sc-структуру) базы знаний в ``message triple''; генератор на основе правил и шаблонов или статистический генератор (при наличии высококачественных выровненных наборов данных), переводящий ``message triple'' в результирующие тексты китайского языка. Пример генерации текстов китайского языка демонстрирует процесс генерации повествовательного предложения китайского языка на основе правил и шаблонов.

Из-за сложности и разнообразия задач генерации текстов, при генерации текстов из фрагментов базы знаний ostis-системы соблюдаются следующие ограничения:
\begin{itemize}
	\item входной фрагмент базы знаний обладает законченной sc-структурой и определенным смыслом.
	\item выходным результатом является повествовательное предложение китайского языка.
	\item sc-узлы во входном фрагменте базы знаний (представляющие сущности, отношения, или класс сущностей, понятия) необходимо обладать идентификаторами на китайском языке, которые будут использоваться в результирующем предложении китайского языка.
\end{itemize}

\textbf{\textit{Шаг 1:}} Определен заданный фрагмент (sc-структура) из базы знаний интеллектуальной справочной системы по дискретной математике, представленный в виде SC-кода. Для визуального представления данного фрагмента базы знаний, фрагмент формально представлен в виде SCg (рисунок \textit{\nameref{fig:knowledge-base-fragment}}).
\begin{figure}[H]
	\centering
	\includegraphics[scale=0.7]{author/part4/figures/fragment_knowledge_base.png}
	\caption{Фрагмент базы знаний ИСС по дискретной математике}
	\label{fig:knowledge-base-fragment}
\end{figure}

\textbf{\textit{Шаг 2:}} Данный фрагмент разделяется на стандартные базовые sc-конструкции, из которых выбираются проектные sc-конструкции для генерации результирующих текстов китайского языка. Проектная sc-конструкция (принадлежащая к стандартной базовой sc-конструкции) показана на SCg (рисунок \textit{\nameref{fig:candidate-sc-construction}}).
\begin{figure}[H]
	\centering
	\includegraphics[scale=0.8]{author/part4/figures/candidate_sc_structure.png}
	\caption{Определение проектных sc-конструкций}
	\label{fig:candidate-sc-construction}
\end{figure}

\textbf{\textit{Шаг 3:}} Проектная sc-конструкция переносится в ``message triple'', каждый sc-элемент в ``message triple'' -- это файл (sc-узел с содержимым), соответствующий определенной единице сегеметаций в китайском языке (например, представленный в виде sc-узла ``I\_ch\_graph'')(рисунок \textit{\nameref{fig:message-triple}}).
\begin{figure}[H]
	\centering
	\includegraphics[scale=0.7]{author/part4/figures/message_triple.png}
	\caption{Перевод проектной sc-конструкции в ``message triple''}
	\label{fig:message-triple}
\end{figure}

\textbf{\textit{Шаг 4:}} Sc-ссылки соответствующих sc-элементов ``message triple'' должны быть заполнены и конкатенированы для генерации результирующего предложения китайского языка в соответствии с допустимым упорядочиванием (рисунок \textit{\nameref{fig:sentence-generated}}).
\begin{figure}[H]
	\centering
	\includegraphics[scale=0.7]{author/part4/figures/chinese_sentence_generated.png}
	\caption{Генерация предложения китайского языка, соответствующего sc-структуре}
	\label{fig:sentence-generated}
\end{figure}

В отличие от европейских языков, в результирующих текстах лексические единицы в sc-ссылках требуют определенной формы склонения (например, единственного или множественного числа и других форм склонения) в соответствии с синтаксисом конкретного естественного языка. Однако в связи с особенностями китайского языка, обработка данного шага относительно проще. В данном примере мы просто рассматриваем генерацию повествовательное предложение китайского языка, ссылочное выражение каждой единицы сегментации означает конечную форму единицы сегментации в результирующем предложении китайского языка. Исходя из шаблона, в качестве подлежащего выступает ссылочное выражение к единице сегментации \begin{CJK}{UTF8}{gbsn} ``图''(граф) \end{CJK}. Единица сегментации \begin{CJK}{UTF8}{gbsn} ``临界图''(критический граф) \end{CJK} выступает в качестве объекта. 

Некоторые отношения уже предопределены в \scnkeyword{IMS} системе для разработки ostis-подсистемы, так \textit{включение*}, \textit{эквиваленция*} и так далее. Для данных отношений, предопределенных в \scnkeyword{IMS} системе, будем называть доменно-независимые отношения. Для бесконечных доменно-зависимых отношений в различных предметных областей модель нейронной сети (в частности, еncoder-decoder архитектура на основе модели Transformer) интегрирована в решатель задач к генерации текстов китайского языка. На рисунке \textit{\nameref{fig:pre-training-model}} показано процесс использовании формы предварительной подготовки и тонкой настройки для решения задач генерации текстов китайского языка из фрагментов базы знаний. 
\begin{figure}[H]
	\includegraphics[scale=0.8,width=1.0\textwidth]{author/part4/figures/promptCLUE4SC.png}
	\caption{Диаграмма для генерации текстов на основе модели предварительной подготовки PromptCLUE}
	\label{fig:pre-training-model}
\end{figure}

Мы используем модель предварительной подготовки PromptCLUE, который использует еncoder-decoder архитектуру на основе модели Transformer и предварительно обучается на нескольких наборов задач про обработки китайского языка с использованием огромного китайского корпуса (сотни ГБ китайского корпуса).  На основе модели PromptCLUE, мы провели тонкую настройку данной модели с использованием построенных корпусов для генерации текстов китайского языка из фрагментов базы знаний (sc-структур). После обучения тонкой настройки на модели PromptCLUE, наша обученная модель может быть использована для генерации текстов китайского языка после предварительной обработки sc-структур. 

\section{Оценка китайско-языкового интерфейса}
Разработанный китайско-языковой интерфейс может применяться в ostis-системах по некоторой конкретной предметной области. Для того, чтобы доказать эффективность sc-модели естественно-языкового интерфейса в рамках \scnkeyword{Технологии OSTIS}, на данный момент осуществляется оценка разработанного китайско-языкового интерфейса в основном по следующим трём аспектам:
\begin{itemize}
	\item оценка базы знаний по обработке китайского языка.
	\item оценка эффективности формирования sc-структур.
	\item оценка сгенерированных текстов китайского языка.
\end{itemize}

Для сравнительного анализа базы знаний по обработке китайского языка с другими аналогичными существующими базами знаний для обработки китайского языка выделены следующие критерии, которые предложены в рамках \scnkeyword{Технологии OSTIS}:
\begin{itemize}
	\item форма структуризации базы знаний.
	\item достаточная независимость друг от друга предметных областей и соответствующих им онтологий.
	\item вид представления знаний и форма хранения знаний в базе знаний.
	\item наличие логических утверждений, построенных на основе лингвистических знаний.
	\item наличие возможности решения задач с использованием логических утверждений;
	\item наличие средств визуализации базы знаний.
\end{itemize}

В таблице \textit{\nameref{table:knowledge-base}} представлены результаты сравнения разработанной базы знаний по обработке китайского языка согласно выделенным критериям.

\renewcommand\arraystretch{3}
\begin{table}[]
	\caption{Таблица. Оценка базы знаний по обработке китайского языка}
	\label{table:knowledge-base}
	%\resizebox{\textwidth}{50mm}
	\begin{tabular}{|m{7em}<{\centering}|m{7em}|m{10em}|m{7em}|m{7em}|m{6em}|m{10em}|}
		\hline
		\multirow{2}{*}{название}
		& \multicolumn{6}{c|}{критерий}
		\\ \cline{2-7}
		{}
		& Форма структуризации знаний   
		& Предметная область, описывающая лингвистические знания
		& форма хранения знаний в базе знаний
		& наличие логических утверждений
		& использование логических утверждений 
		& наличие средства визуализации базы знаний \\
		\hline
		Grammatical KB of Contemporary (GKB) 
		& таблица базы данных
		& предметная область слов
		& база данных
		& -
		& -
		& -  \\ 
		\hline
		Mandarin   VerbNet
		&на основе фрейма
		& предметная область слов
		& база данных
		& -
		& -
		& - \\ 
		\hline
		HowNet
		& Knowledge dictionary Mark-up language
		&предметная область слов
		&база данных
		& -     
		& -
		& - \\ 
		\hline
		Chinese   Treebank 8.0
		& текстовый файл
		& предметная область предложений
		& текст/xml файл
		& -
		& -
		& - \\ 
		\hline
		БЗ   по обработке китайского языка
		& семантическая сеть с   теоретико-множественной интерпретацией
		& лингвистические знания на различных   уровнях (слов, словосочетаний, предложений и другие)
		& scs   файл
		& +
		& +
		& + \\
		\hline
	\end{tabular}
\end{table}

В принципе, на основе sc-модели базы знаний лингвистика в процессе разработки базы знаний по обработке китайского языка может интегрировать различные типы лингвистических знаний в существующих базах знаний, которые были построены для обработки текстов китайского текста. Кроме того, в отношении построенной нами базы знаний по обработке китайского языка следует отметить, что при анализе текстов китайского языка или генерации текстов китайского языка в базе знаний могут быть построены различные правила для извлечения фактографических знаний или правила и шаблоны для генерации текстов китайского текста. Это преимущество полностью отсутствует у других баз знаний. Более того, разработанная база знаний по обработке китайского языка структурирована на соответствующие выбранные предметные области и соответствующие им онтологии. Достаточная независимость между предметными областями позволяет осуществлять командную разработку базы знаний, что значительно сокращает затраты на время и труды при разработке базы знаний по обработке китайского языка по сравнению с построением других баз знаний. 

Для оценки эффективности анализа текстов китайского языка (преобразование текстов китайского языка в фрагменты базы знаний) идеальный способ включается в том, что вычисление сходства между sc-структурой, существующей в базе знаний, и соответствующей sc-структурой, сформированной решателем задач анализа текстов китайского языка. В нашей ситуации sc-структура представляет собой графическую структуру, сопровождаемую идентификаторами на китайском языке. В [\scncite{Li2022}] был предложен подход вычисления сходства между семантическими графами (sc-структурами), ориентированный на проверку ответа на целевой вопрос. При этом вычисление сходства между ответом (в виде sc-структуры) пользователя на целевой вопрос и стандартным ответом в базе знаний может отражать проверку ответа на целевой вопрос. Таким образом, данный подход может использоваться для вычисления сходства между sc-структурой, существующей в базе знаний, и соответствующей sc-структурой, сформированной решателем задач анализа текстов китайского языка. 

Однако данный подход не учитывает влияние идентификаторов каждого элемента в sc-структурах на вычисление сходств между sc-структурами. Таким образом, предлагаемая нами ещё метрика оценки эффективности анализа текстов китайского языка проводилась на основе точного совпадения. Точное совпадение означает, что идентификатор каждого извлеченного элемента (именованный сущность и отношение между ними) должен точно соответствовать идентификатору элемента в базе знаний. 

Для вычисления сходств между стандартными sc-структурами в базе знаний и sc-структурами, сформированными решателем задач анализа текстов китайского языка, мы вручную выбрали несколько различных видов sc-структур, существующих в базе знаний ostis-систем по различным конкретным предметным областям, из стандартной sc-конструкции (например, одноэлементной sc-конструкции) до sc-конструкции нестандартного вида. По правилам построения sc-конструкции, выбранные sc-конструкции нестандартного вида состоят из некоторых стандартных sc-конструкций.

В таблице \textit{\nameref{table:sc-structure-generation}} представлены результаты. В зависимости от сложности sc-структур, мы выбрили разные числа для разных типов sc-структур, вычислили средний балл сходства для данных sc-структур, и наконец, вычислили общий балл сходства для оценки эффективности решателя задач.
\renewcommand\arraystretch{2}
\begin{table}[]
	\caption{Таблица. Оценка сходств между sc-структурами}
	\centering
	\begin{tabular}{|m{8em}|m{8em}|m{8em}|m{8em}|m{8em}|m{8em}|}
		\hline
		{}
		& одноэлементная sc-конструкция
		& трехэлементная sc-конструкция
		& пятиэлементная sc-конструкция
		& sc-конструкция нестандартного вида
		& \makecell[c]{общее} \\
		\hline 
		количество
		& 10
		& 15
		& 15
		& 10
		& 50 \\
		\hline
		средний балл сходства
		& 0.8571
		& 0.8125
		& 0.8387
		& 0.7273
		& \textbf{0.8089} \\
		\hline
	\end{tabular}
	\label{table:sc-structure-generation}
\end{table}

Как видно из таблицы \textit{\nameref{table:sc-structure-generation}}, по мере увеличения сложности sc-структур, балл сходства уменьшается. Но разработанный решатель задач по-прежнему достигает относительно хороший результат.

В соответствии с метрикой оценки точного совпадения идентификатор каждого элементов выбранных sc-структур был вручную добавлен на китайском языке обученными носителями китайского языка и проверен другими. В настоящее время существует система CORE, который в основе извлекает тройки на RDF для предложений китайского языка в области журналистики. Для точного совпадения система CORE может использоваться для оценки эффективности разработанного решателя задач анализа текстов китайского языка.
\renewcommand\arraystretch{2}
\begin{table}[]
	\caption{Таблица. Оценка точного совпадения идентификаторов}
	\centering
	\begin{tabular}{|m{8em}|m{8em}|m{8em}|m{8em}|}
		\hline
		{}
		& \makecell[c]{Precision} 
		& \makecell[c]{Recall}
		& \makecell[c]{F1} \\
		\hline 
		Решатель задач анализа текстов китайского языка 
		& 0.8289 
		& 0.7875
		& \textbf{0.8076} \\
		\hline
		CORE 
		& 0.8308 
		& 0.6750 
		& 0.7448 \\
		\hline
	\end{tabular}
	\label{table:evaluation-extraction}
\end{table}

В таблице \textit{\nameref{table:evaluation-extraction}} представлены экспериментальные результаты. В итоге, результаты показывают, что использование рядов анализа текстов китайского языка и построенных правил извлечения в базе знаний по обработке китайского языка эффективно для извлечения фрагментов базы знаний без каких-либо конкретных вмешательства человека. 

Для оценки сгенерированных текстов китайского языка, в настоящее время автоматические метрики широко используются для оценки эффективности в других системах, ориентированных для генерации текстов. Среди них, в других системах генерации текстов балл BLEU-4 и ROUGE-L обычно используются для оценки качества сгенерированных текстов.

Для оценки качества сгенерированных текстов китайского языка, в соответствии sc-структуры мы построили ручную справочные предложения, соответствующие различным sc-структурам. Чтобы сделать оценку более общей, мы выбрали около 150 различных видов sc-структур, существующих в базе знаний ostis-систем по различным конкретными предметным областям, из стандартной sc-конструкции (не включение одноэлементной sc-конструкции) до sc-конструкции нестандартного вида.

В настоящее время существует несколько систем для генерации текстов китайского текста, но система для генерации текстов китайского текста из структурированных данных (т. е. данные в форме графической структуры) ещё не существует. В настоящее время только существует самая лучшая система Melbourne в WebNLG для генерации текстов английского языка, который ориентирован на генерации предложений английского языка из тройки на RDF [\scncite{Trisedya2018}]. С появлением модели предварительной подготовки, в WebNLG предусмотрена базовая система, реализованная на основе модели предварительной подготовки T5, для генерации текстов английского языка [\scncite{kale2020}]. Без другой системы для генерации текстов китайского языка для сравнения производительности, мы используем одну и ту же метрику оценки для оценки производительности нашей решателя задач генерации текстов китайского языка в построенных наборов данных с другими системами для генерации английского языка.

Результаты сравнения решателя задач генерации текстов китайского языка и другие системы по метрикам BLEU-4, ROUGE-L сведены в таблице \textit{\nameref{table:text-generation}}.

\renewcommand\arraystretch{2}
\begin{table}[]
	\caption{Таблица. Оценка эффективности для системы генерации текстов китайского языка}
	\centering
	\begin{tabular}{|m{8em}|m{8em}|m{8em}|}
		\hline
		{}& \makecell[c]{BLEU-4} & \makecell[c]{ROUGE-L} \\
		\hline 
		Решатель задач генерации текстов китайского языка & \textbf{0.5885} & \textbf{0.6793} \\
		\hline
		T5-baseline & 0.5520 & 0.6543 \\
		\hline
		Melbourne & 0.5452 & 0.6350 \\
		\hline
	\end{tabular}
	\label{table:text-generation}
\end{table}

Как видно из таблицы \textit{\nameref{table:text-generation}}, разработанный решатель задач достигал относительно многообещающих баллов BLEU-4 и ROUGE-L по сравнению с другими системами, ориентированными на английский язык. Экспериментальные результаты означают, что разработанный решатель задач больше подходит для генерации текстов китайского языка. 

\section {Заключение}
В главе рассмотрена на основе \scnkeyword{Технологии OSTIS} разработка единой семантической модели естественно-языкового интерфейса интеллектуальной системы, основанной на знаниях, которая имеет возможность преобразовывать тексты естественного языка в фрагменты базы знаний и генерировать тексты естественного языка из фрагментов базы знаний. Разработка семантической модели естественно-языкового интерфейса в основном требует разработка sc-модели базы знаний лингвистики, которая представлена в виде описанных предметных областей и соответствующих им онтологий о лингвистических знаниях, а также sc-модели решателя задач естественно-языкового интерфейса, состоящий из sc-агентов, разработанных независимо друг от друга с использованием различных моделей решения задач.

С помощью предложенной семантической модели естественно-языкового интерфейса реализован китайско–языковой интерфейс прототипа ИСС по дискретной математике, способный решать задачи преобразования текстов китайского языка в фрагменты базы знаний и генерации текстов китайского языка из фрагментов базы знаний. Для реализации китайско–языкового интерфейса разработана база знаний по обработке китайского языка и соответствующие решатели задач анализа текстов и генерации текстов китайского языка. Чтобы доказать эффективность модели естественно–языкового интерфейса ostis–систем, проведена оценка разработанного китайско–языкового интерфейса в трех аспектах: оценка базы знаний по обработке китайского языка, оценка эффективности формирования sc–структур и оценка сгенерированных текстов китайского языка. В соответствии с оценками разработанный китайско–языковой интерфейс получил отношетельно лучше результат чем другие системы.
%\input{author/references}